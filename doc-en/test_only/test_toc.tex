%% For testing TOC styles

\part{Manual}
\chapter{Julia 1.8-DEV Documentation}
    \subsection{Introduction}
\chapter{Getting Started}
    \section{Resources}
\chapter{Variables}
    \section{Allowed Variable Names}
    \section{Stylistic Conventions}
\chapter{Integers and Floating-Point Numbers}
    \section{Integers}
    \subsection{Overflow behavior}
    \subsection{Division errors}
    \section{Floating-Point Numbers}
    \subsection{Floating-point zero}
    \subsection{Special floating-point values}
    \subsection{Machine epsilon}
    \subsection{Rounding modes}
    \subsection{Background and References}
    \section{Arbitrary Precision Arithmetic}
    \section{Numeric Literal Coefficients}
    \subsection{Syntax Conflicts}
    \section{Literal zero and one}
\chapter{Mathematical Operations and Elementary Functions}
    \section{Arithmetic Operators}
    \section{Boolean Operators}
    \section{Bitwise Operators}
    \section{Updating operators}
    \section{Vectorized {\textquotedbl}dot{\textquotedbl} operators}
    \section{Numeric Comparisons}
    \subsection{Chaining comparisons}
    \subsection{Elementary Functions}
    \section{Operator Precedence and Associativity}
    \section{Numerical Conversions}
    \subsection{Rounding functions}
    \subsection{Division functions}
    \subsection{Sign and absolute value functions}
    \subsection{Powers, logs and roots}
    \subsection{Trigonometric and hyperbolic functions}
    \subsection{Special functions}
\chapter{Complex and Rational Numbers}
    \section{Complex Numbers}
    \section{Rational Numbers}
\chapter{Strings}
    \section{Characters}
    \section{String Basics}
    \section{Unicode and UTF-8}
    \section{Concatenation}
    \section{Interpolation}
    \section{Triple-Quoted String Literals}
    \section{Common Operations}
    \section{Non-Standard String Literals}
    \section{Regular Expressions}
    \section{Byte Array Literals}
    \section{Version Number Literals}
    \section{Raw String Literals}
\chapter{Functions}
    \section{Argument Passing Behavior}
    \section{Argument-type declarations}
    \section{The \texttt{return} Keyword}
    \subsection{Return type}
    \subsection{Returning nothing}
    \section{Operators Are Functions}
    \section{Operators With Special Names}
    \section{Anonymous Functions}
    \section{Tuples}
    \section{Named Tuples}
    \section{Destructuring Assignment and Multiple Return Values}
    \section{Property destructuring}
    \section{Argument destructuring}
    \section{Varargs Functions}
    \section{Optional Arguments}
    \section{Keyword Arguments}
    \section{Evaluation Scope of Default Values}
    \section{Do-Block Syntax for Function Arguments}
    \section{Function composition and piping}
    \section{Dot Syntax for Vectorizing Functions}
    \section{Further Reading}
\chapter{Control Flow}
    \section{Compound Expressions}
    \section{Conditional Evaluation}
    \section{Short-Circuit Evaluation}
    \section{Repeated Evaluation: Loops}
    \section{Exception Handling}
    \subsection{Built-in \texttt{Exception}s}
    \subsection{The \texttt{throw} function}
    \subsection{Errors}
    \subsection{The \texttt{try/catch} statement}
    \subsection{\texttt{finally} Clauses}
    \section{Tasks (aka Coroutines)}
\chapter{Scope of Variables}
    \subsection{Scope constructs}
    \section{Global Scope}
    \section{Local Scope}
    \subsection{Let Blocks}
    \subsection{Loops and Comprehensions}
    \section{Constants}
\chapter{Types}
    \section{Type Declarations}
    \section{Abstract Types}
    \section{Primitive Types}
    \section{Composite Types}
    \section{Mutable Composite Types}
    \section{Declared Types}
    \section{Type Unions}
    \section{Parametric Types}
    \subsection{Parametric Composite Types}
    \subsection{Parametric Abstract Types}
\chapter{Command-line Options}
    \section{Using arguments inside scripts}
    \section{Command-line switches for Julia}




\part{Base}
\chapter{Essentials}
    \section{Introduction}
    \section{Getting Around}
    \section{Keywords}
    \section{Standard Modules}
    \section{Base Submodules}
    \section{All Objects}
    \section{Properties of Types}
    \subsection{Type relations}
    \subsection{Declared structure}
    \subsection{Memory layout}
    \subsection{Special values}
    \section{Special Types}
    \section{Generic Functions}
    \section{Syntax}
    \section{Missing Values}
    \section{System}
    \section{Versioning}
    \section{Errors}
    \section{Events}
    \section{Reflection}
    \section{Internals}
    \section{Meta}
\chapter{Collections and Data Structures}
    \section{Iteration}
    \section{Constructors and Types}
    \section{General Collections}
    \section{Iterable Collections}
    \section{Indexable Collections}
    \section{Dictionaries}
    \section{Set-Like Collections}
    \section{Dequeues}
    \section{Utility Collections}
\chapter{Mathematics}
    \section{Mathematical Operators}
    \section{Mathematical Functions}
\chapter{Examples}
    \section{Customizable binary operators}
\chapter{Numbers}
    \section{Standard Numeric Types}
    \subsection{Abstract number types}
    \subsection{Concrete number types}
    \section{Data Formats}
    \section{General Number Functions and Constants}
    \subsection{Integers}
    \section{BigFloats and BigInts}
\chapter{Strings}
\chapter{Arrays}
    \section{Constructors and Types}
    \section{Basic functions}
    \section{Broadcast and vectorization}
    \section{Indexing and assignment}
    \section{Views (SubArrays and other view types)}
    \section{Concatenation and permutation}
    \section{Array functions}
    \section{Combinatorics}
\chapter{Tasks}
    \section{Scheduling}
    \section{Synchronization}
    \section{Synchronization}
    \section{Channels}
    \section{Low-level synchronization using \texttt{schedule} and \texttt{wait}}
\chapter{Multi-Threading}
    \section{Atomic operations}
    \section{ccall using a threadpool (Experimental)}
    \section{Low-level synchronization primitives}
\chapter{Constants}
\chapter{Filesystem}
\chapter{I/O and Network}
    \section{General I/O}
    \section{Text I/O}
    \section{Multimedia I/O}
    \section{Network I/O}
\chapter{Punctuation}
\chapter{Sorting and Related Functions}
    \section{Sorting Functions}
    \section{Order-Related Functions}
    \section{Sorting Algorithms}
    \section{Alternate orderings}
\chapter{Iteration utilities}
\chapter{C Interface}
\chapter{LLVM Interface}
\chapter{C Standard Library}
\chapter{StackTraces}
\chapter{SIMD Support}




\part{Standard Library}
\chapter{ArgTools}
    \section{Argument Handling}
    \section{Function Testing}
\chapter{Artifacts}
\chapter{Base64}
\chapter{CRC32c}
\chapter{Dates}
    \section{Constructors}
    \section{Durations/Comparisons}
    \section{Accessor Functions}
    \section{Query Functions}
    \section{TimeType-Period Arithmetic}
    \section{Adjuster Functions}
    \section{Period Types}
    \section{Rounding}
    \subsection{Rounding Epoch}
\chapter{API reference}
    \section{Dates and Time Types}
    \section{Dates Functions}
    \subsection{Accessor Functions}
    \subsection{Query Functions}
    \subsection{Adjuster Functions}
    \subsection{Periods}
    \subsection{Rounding Functions}
    \subsection{Conversion Functions}
    \subsection{Constants}
\chapter{Delimited Files}
\chapter{Distributed Computing}
    \section{Cluster Manager Interface}
\chapter{Downloads}
\chapter{File Events}
\chapter{Future}
\chapter{Interactive Utilities}
\chapter{Lazy Artifacts}
\chapter{LibCURL}
\chapter{LibGit2}
    \subsection{Functionality}
\chapter{Dynamic Linker}
\chapter{Linear Algebra}
    \section{Special matrices}
    \subsection{Elementary operations}
    \subsection{Matrix factorizations}
    \subsection{The uniform scaling operator}
    \section{Matrix factorizations}
    \section{Standard functions}
    \section{Low-level matrix operations}
    \section{BLAS functions}
    \subsection{BLAS character arguments}
    \section{LAPACK functions}
\chapter{Logging}
    \section{Log event structure}
    \section{Processing log events}
    \subsection{Loggers}
    \subsection{Early filtering and message handling}
    \section{Testing log events}
    \section{Environment variables}
    \section{Examples}
    \subsection{Example: Writing log events to a file}
    \subsection{Example: Enable debug-level messages}
    \section{Reference}
    \subsection{Logging module}
    \subsection{Creating events}
    \subsection{Processing events with AbstractLogger}
    \subsection{Using Loggers}
\chapter{Markdown}
    \section{Inline elements}
    \subsection{Bold}
    \subsection{Italics}
    \subsection{Literals}
    \subsection{\LaTeX}
    \subsection{Links}
    \subsection{Footnote references}
    \section{Toplevel elements}
    \subsection{Paragraphs}
    \subsection{Headers}
    \subsection{Code blocks}
    \subsection{Block quotes}
    \subsection{Images}
    \subsection{Lists}
    \subsection{Display equations}
    \subsection{Footnotes}
    \subsection{Horizontal rules}
    \subsection{Tables}
    \subsection{Admonitions}
    \section{Markdown Syntax Extensions}
\chapter{Memory-mapped I/O}
\chapter{NetworkOptions}
\chapter{Pkg}
\chapter{Printf}
\chapter{Profiling}
\chapter{The Julia REPL}
    \section{The different prompt modes}
    \subsection{The Julian mode}
    \subsection{Help mode}
    \subsection{Shell mode}
    \subsection{Pkg mode}
    \subsection{Search modes}
    \section{Key bindings}
    \subsection{Customizing keybindings}
    \section{Tab completion}
    \section{Customizing Colors}
    \section{TerminalMenus}
    \subsection{Examples}
    \subsection{Customization / Configuration}
    \section{References}
    \subsection{REPL}
    \subsection{TerminalMenus}
\chapter{Random Numbers}
    \section{Random numbers module}
    \section{Random generation functions}
    \section{Subsequences, permutations and shuffling}
    \section{Generators (creation and seeding)}
    \section{Hooking into the \texttt{Random} API}
    \subsection{Generating random values of custom types}
    \subsection{Creating new generators}
\chapter{Reproducibility}
\chapter{SHA}
\chapter{Serialization}
\chapter{Shared Arrays}
\chapter{Sockets}
\chapter{Sparse Arrays}
    \section{Compressed Sparse Column (CSC) Sparse Matrix Storage}
    \section{Sparse Vector Storage}
    \section{Sparse Vector and Matrix Constructors}
    \section{Sparse matrix operations}
    \section{Correspondence of dense and sparse methods}
\chapter{Sparse Arrays}
\chapter{Statistics}
\chapter{Sparse Linear Algebra}
\chapter{TOML}
    \section{Parsing TOML data}
    \section{Exporting data to TOML file}
    \section{References}
\chapter{Tar}
\chapter{Unit Testing}
    \section{Testing Base Julia}
    \section{Basic Unit Tests}
    \section{Working with Test Sets}
    \section{Other Test Macros}
    \section{Broken Tests}
    \section{Creating Custom \texttt{AbstractTestSet} Types}
    \section{Test utilities}
\chapter{UUIDs}
\chapter{Unicode}




\part{Developer Documentation}
\chapter{Reflection and introspection}
    \section{Module bindings}
    \section{DataType fields}
    \section{Subtypes}
    \section{DataType layout}
    \section{Function methods}
    \section{Expansion and lowering}
    \section{Intermediate and compiled representations}
\chapter{Documentation of Julia's Internals}
    \section{Initialization of the Julia runtime}
    \section{Julia ASTs}
    \section{More about types}
    \section{Memory layout of Julia Objects}
    \section{Eval of Julia code}
    \section{Calling Conventions}
    \section{High-level Overview of the Native-Code Generation Process}
    \section{Julia Functions}
    \section{Base.Cartesian}
    \section{Talking to the compiler (the \texttt{:meta} mechanism)}
    \section{SubArrays}
    \section{isbits Union Optimizations}
    \section{System Image Building}
    \section{Working with LLVM}
    \section{printf() and stdio in the Julia runtime}
    \section{Bounds checking}
    \section{Proper maintenance and care of multi-threading locks}
    \section{Arrays with custom indices}
    \section{Module loading}
    \section{Inference}
    \section{Julia SSA-form IR}
    \section{Static analyzer annotations for GC correctness in C code}
\chapter{Developing/debugging Julia's C code}
    \section{Reporting and analyzing crashes (segfaults)}
    \section{gdb debugging tips}
    \section{Using Valgrind with Julia}
    \section{Sanitizer support}
    \section{Instrumenting Julia with DTrace, and bpftrace}
\chapter{Building Julia}
    \section{Building Julia (Detailed)}
    \section{Linux}
    \section{macOS}
    \section{Windows}
    \section{FreeBSD}
    \section{ARM (Linux)}
    \section{Binary distributions}
    \section{Point releasing 101}




\part{Julia v1.8 Release Notes}
\chapter{New language features}
\chapter{Language changes}
\chapter{Compiler/Runtime improvements}
\chapter{Command-line option changes}
\chapter{Multi-threading changes}
\chapter{Build system changes}
\chapter{New library functions}
\chapter{New library features}
\chapter{Standard library changes}
    \subsection{InteractiveUtils}
    \subsection{Package Manager}
    \subsection{LinearAlgebra}
    \subsection{Markdown}
    \subsection{Printf}
    \subsection{Profile}
    \subsection{Random}
    \subsection{REPL}
    \subsection{SparseArrays}
    \subsection{Dates}
    \subsection{Downloads}
    \subsection{Statistics}
    \subsection{Sockets}
    \subsection{Tar}
    \subsection{Distributed}
    \subsection{UUIDs}
    \subsection{Mmap}
    \subsection{DelimitedFiles}
    \subsection{Logging}
    \subsection{Unicode}
\chapter{Deprecated or removed}
\chapter{External dependencies}
\chapter{Tooling Improvements}
