\part{Developer Documentation}
  \chapter{Reflection and introspection}
    \section{Module bindings}
    \section{DataType fields}
    \section{Subtypes}
    \section{DataType layout}
    \section{Function methods}
    \section{Expansion and lowering}
    \section{Intermediate and compiled representations}
    \subsection{Printing of debug information}
  \chapter{Documentation of Julia's Internals}
    \section{Initialization of the Julia runtime}
    \subsection{\texttt{main()}}
    \subsection{\texttt{julia\_init()}}
    \subsection{\texttt{repl\_entrypoint()}}
    \subsection{\texttt{Base.\_start}}
    \subsection{\texttt{Core.eval}}
    \subsection{\texttt{jl\_atexit\_hook()}}
    \subsection{\texttt{julia\_save()}}
    \section{Julia ASTs}
    \subsection{Surface syntax AST}
    \subsection{Lowered form}
    \section{More about types}
    \subsection{Types and sets (and \texttt{Any} and \texttt{Union\{\}}/\texttt{Bottom})}
    \subsection{UnionAll types}
    \subsection{Free variables}
    \subsection{TypeNames}
    \subsection{Tuple types}
    \subsection{Diagonal types}
    \subsection{Subtyping diagonal variables}
    \subsection{Introduction to the internal machinery}
    \subsection{Subtyping and method sorting}
    \section{Memory layout of Julia Objects}
    \subsection{Object layout (\texttt{jl\_value\_t})}
    \subsection{Garbage collector mark bits}
    \subsection{Object allocation}
    \section{Eval of Julia code}
    \subsection{Julia Execution}
    \subsection{Parsing}
    \subsection{Macro Expansion}
    \subsection{Type Inference}
    \subsection{JIT Code Generation}
    \subsection{System Image}
    \section{Calling Conventions}
    \subsection{Julia Native Calling Convention}
    \subsection{JL Call Convention}
    \subsection{C ABI}
    \section{High-level Overview of the Native-Code Generation Process}
    \subsection{Representation of Pointers}
    \subsection{Representation of Intermediate Values}
    \subsection{Union representation}
    \subsection{Specialized Calling Convention Signature Representation}
    \section{Julia Functions}
    \subsection{Method Tables}
    \subsection{Function calls}
    \subsection{Adding methods}
    \subsection{Creating generic functions}
    \subsection{Closures}
    \subsection{Constructors}
    \subsection{Builtins}
    \subsection{Keyword arguments}
    \subsection{Compiler efficiency issues}
    \section{Base.Cartesian}
    \subsection{Principles of usage}
    \subsection{Basic syntax}
    \section{Talking to the compiler (the \texttt{:meta} mechanism)}
    \section{SubArrays}
    \subsection{Index replacement}
    \subsection{SubArray design}
    \section{isbits Union Optimizations}
    \subsection{isbits Union Structs}
    \subsection{isbits Union Arrays}
    \section{System Image Building}
    \subsection{Building the Julia system image}
    \subsection{System image optimized for multiple microarchitectures}
    \section{Working with LLVM}
    \subsection{Overview of Julia to LLVM Interface}
    \subsection{Building Julia with a different version of LLVM}
    \subsection{Passing options to LLVM}
    \subsection{Debugging LLVM transformations in isolation}
    \subsection{Improving LLVM optimizations for Julia}
    \subsection{The jlcall calling convention}
    \subsection{GC root placement}
    \section{printf() and stdio in the Julia runtime}
    \subsection{Libuv wrappers for stdio}
    \subsection{Interface between JL\_STD* and Julia code}
    \subsection{printf() during initialization}
    \subsection{Legacy \texttt{ios.c} library}
    \section{Bounds checking}
    \subsection{Eliding bounds checks}
    return r
    \subsection{Propagating inbounds}
    \subsection{The bounds checking call hierarchy}
    \subsection{Emit bounds checks}
    \section{Proper maintenance and care of multi-threading locks}
    \subsection{Locks}
    \subsection{Broken Locks}
    \subsection{Shared Global Data Structures}
    \section{Arrays with custom indices}
    \subsection{Generalizing existing code}
    \subsection{Writing custom array types with non-1 indexing}
    \section{Module loading}
    \subsection{Experimental features}
    \section{Inference}
    \subsection{How inference works}
    \subsection{Debugging compiler.jl}
    \subsection{The inlining algorithm (\texttt{inline\_worthy})}
    \section{Julia SSA-form IR}
    \subsection{Background}
    \subsection{New IR nodes}
    \subsection{Main SSA data structure}
    \section{Static analyzer annotations for GC correctness in C code}
    \subsection{Running the analysis}
    \subsection{General Overview}
    \subsection{GC Invariants}
    \subsection{Static Analysis Algorithm}
    \subsection{The analyzer annotations}
    \subsection{Completeness of analysis}
  \chapter{Developing/debugging Julia's C code}
    \section{Reporting and analyzing crashes (segfaults)}
    \subsection{Version/Environment info}
    \subsection{Segfaults during bootstrap (\texttt{sysimg.jl})}
    \subsection{Segfaults when running a script}
    \subsection{Errors during Julia startup}
    \subsection{Other generic segfaults or unreachables reached}
    \subsection{Glossary}
    \section{gdb debugging tips}
    \subsection{Displaying Julia variables}
    \subsection{Useful Julia variables for Inspecting}
    \subsection{Useful Julia functions for Inspecting those variables}
    \subsection{Inserting breakpoints for inspection from gdb}
    \subsection{Inserting breakpoints upon certain conditions}
    \subsection{Dealing with signals}
    \subsection{Debugging during Julia{\textquotesingle}s build process (bootstrap)}
    \subsection{Debugging precompilation errors}
    \subsection{Mozilla{\textquotesingle}s Record and Replay Framework (rr)}
    \section{Using Valgrind with Julia}
    \subsection{General considerations}
    \subsection{Suppressions}
    \subsection{Running the Julia test suite under Valgrind}
    \subsection{Caveats}
    \section{Sanitizer support}
    \subsection{General considerations}
    \subsection{Address Sanitizer (ASAN)}
    \subsection{Memory Sanitizer (MSAN)}
    \subsection{Thread Sanitizer (TSAN)}
    \section{Instrumenting Julia with DTrace, and bpftrace}
    \subsection{Enabling support}
    Description
    NT_GNU_BUILD_ID (unique build ID bitstring)
    Description
    NT_STAPSDT (SystemTap probe descriptors)
    NT_STAPSDT (SystemTap probe descriptors)
    NT_STAPSDT (SystemTap probe descriptors)
    NT_STAPSDT (SystemTap probe descriptors)
    \subsection{Adding probes in libjulia}
    \subsection{Available probes}
    \subsection{Notes on using \texttt{bpftrace}}
    @start[pid] = nsecs;
    \subsection{Useful references:}
  \chapter{Building Julia}
    \section{Building Julia (Detailed)}
    \subsection{Downloading the Julia source code}
    \subsection{Building Julia}
    \subsection{Updating an existing source tree}
    \subsection{General troubleshooting}
    \subsection{Platform-Specific Notes}
    \subsection{Required Build Tools and External Libraries}
    \subsection{Build dependencies}
    \subsection{Source distributions of releases}
    \subsection{Building Julia from source with a Git checkout of a stdlib}
    \section{Linux}
    \subsection{Architecture Customization}
    \subsection{Linux Build Troubleshooting}
    \section{macOS}
    \section{Windows}
    \subsection{General Information for Windows}
    \subsection{Binary distribution}
    \subsection{Source distribution}
    \subsection{Debugging a cross-compiled build under wine}
    \subsection{After compiling}
    \subsection{Windows Build Debugging}
    \section{FreeBSD}
    \section{ARM (Linux)}
    \subsection{32-bit (ARMv6, ARMv7)}
    \subsection{AArch64 (ARMv8)}
    \section{Binary distributions}
    \subsection{Versioning and Git}
    \subsection{Target Architectures}
    \subsection{Linux}
    \subsection{OS X}
    \subsection{Windows}
    \subsection{Notes on BLAS and LAPACK}
    \section{Point releasing 101}
    \subsection{Backporting commits}
    \subsection{Checking for performance regressions}
    \subsection{Building test binaries}
    \subsection{Checking for package breakages}
    \subsection{Merging backports into the release branch}
    \subsection{Tagging the release}
    \subsection{Signing binaries}
    \subsection{Uploading binaries}
