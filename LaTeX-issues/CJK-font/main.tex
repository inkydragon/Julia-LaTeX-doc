\documentclass{book}

%% font settings
\usepackage{fontspec}
% \setmainfont{DejaVu Sans}
\setsansfont{DejaVu Sans}[Scale=MatchLowercase, Ligatures=TeX]
\setmonofont{DejaVu Sans Mono}[Scale=MatchLowercase]
\renewcommand{\familydefault}{\sfdefault}

% CJK font
\usepackage{xeCJK}
\setCJKmainfont{Noto Sans CJK SC}
\setCJKfallbackfamilyfont{\CJKrmdefault}{ 
    {Noto Sans CJK KR},
    {Noto Sans CJK JP},
}
\setCJKmonofont{Noto Sans Mono CJK SC}
\setCJKfallbackfamilyfont{\CJKttdefault}{ 
    {Noto Sans Mono CJK KR},
    {Noto Sans Mono CJK JP},
}
%%

%% listings
\usepackage{listings, minted}

\lstset{
    basicstyle = \small\ttfamily,
    breaklines = true,
    columns = fullflexible,
    frame = leftline,
    keepspaces = true,
    showstringspaces = false,
    xleftmargin = 3pt,
}

\setminted{
    breaklines = true,
    fontsize = \small,
    frame = leftline,
}
%

%% 
\usepackage{polyglossia}
\setdefaultlanguage{english}
% \setotherlanguage{korean}


\begin{document}

Julia provides an extremely flexible system for naming variables. Variable names are case-sensitive, and have no semantic meaning (that is, the language will not treat variables differently based on their names).

\begin{minted}{jlcon}
julia> x = 1.0
1.0

julia> y = -3
-3

julia> Z = "My string"
"My string"

julia> customary_phrase = "Hello world!"
"Hello world!"

julia> UniversalDeclarationOfHumanRightsStart = "人人生而自由,在尊严和权利上一律平等。"
"人人生而自由,在尊严和权利上一律平等。"
\end{minted}

Unicode names (in UTF-8 encoding) are allowed:

\begin{minted}{jlcon}
julia> δ = 0.00001
1.0e-5

julia> 안녕하세요 = "Hello"
"Hello"
\end{minted}

\end{document}