%!TEX program = lualatex
\documentclass[oneside]{ctexbook} %% 1. {ctexbook} + \newCJKfontfamily
% \documentclass[oneside]{book} %% 2. {book} + \newfontface

\usepackage[
  hmargin=0.1cm, % left and right margin
  vmargin=0.5cm, % top and bottom margin
]{geometry}

% 表格
\usepackage{longtable,booktabs}

% emoji
\usepackage{./emoji}
\setemojifont{Twemoji Mozilla}
\newCJKfontfamily\EmojiFont{Twemoji Mozilla}[Renderer=HarfBuzz] %% 1. {ctexbook} + \newCJKfontfamily
% \newfontface\EmojiFont{Twemoji Mozilla}[Renderer=HarfBuzz] %% 2. {book} + \newfontface


\begin{document}

% \verb!\newCJKfontfamily\EmojiFont! \\
% U+03030 | {\EmojiFont 〰}
% U+0303D | {\EmojiFont 〽}
% U+03297 | {\EmojiFont ㊗}
% U+03299 | {\EmojiFont ㊙}
% U+02194 | {\EmojiFont ↔} 
% U+02195 | {\EmojiFont ↕} \\

% \verb!\newCJKfontfamily\EmojiFont + \emoji{}! \\
% U+03030 | \emoji{wavy-dash}
% U+0303D | \emoji{part-alternation-mark}
% U+03297 | \emoji{congratulations}
% U+03299 | \emoji{secret}
% U+02194 | \emoji{left-right-arrow} 
% U+02195 | \emoji{left-right-arrow} \\

\verb!\newCJKfontfamily\EmojiFont! \\ %% 1. {ctexbook} + \newCJKfontfamily
% \verb! \newfontface\EmojiFont ! \\ %% 2. {book} + \newfontface


% 不管右侧溢出,确保一行一个 emoji
\begin{longtable}{ccll}
    \caption{Unicode emoji} \\
    %% all head
    \toprule
    Code point & Char & Tab seq & Unicode name \\
    \hline \endhead
    %% all footer
    \multicolumn{4}{l}{To be continued...} \\ 
    \midrule \endfoot
    %% last footer
    \bottomrule \endlastfoot
    
    U+000A9 & {\EmojiFont ©} & {\textbackslash}:copyright:, {\textbackslash}copyright & Copyright Sign \\ \hline
U+000AE & {\EmojiFont ®} & {\textbackslash}:registered:, {\textbackslash}circledR & Registered Sign / Registered Trade Mark Sign \\ \hline
U+0203C & {\EmojiFont ‼} & {\textbackslash}:bangbang: & Double Exclamation Mark \\ \hline
U+02049 & {\EmojiFont ⁉} & {\textbackslash}:interrobang: & Exclamation Question Mark \\ \hline
U+02122 & {\EmojiFont ™} & {\textbackslash}:tm:, {\textbackslash}trademark & Trade Mark Sign / Trademark \\ \hline
U+02139 & {\EmojiFont ℹ} & {\textbackslash}:information\_source: & Information Source \\ \hline
U+02194 & {\EmojiFont ↔} & {\textbackslash}:left\_right\_arrow:, {\textbackslash}leftrightarrow & Left Right Arrow \\ \hline
U+02195 & {\EmojiFont ↕} & {\textbackslash}:arrow\_up\_down:, {\textbackslash}updownarrow & Up Down Arrow \\ \hline
U+02196 & {\EmojiFont ↖} & {\textbackslash}:arrow\_upper\_left:, {\textbackslash}nwarrow & North West Arrow / Upper Left Arrow \\ \hline
U+02197 & {\EmojiFont ↗} & {\textbackslash}:arrow\_upper\_right:, {\textbackslash}nearrow & North East Arrow / Upper Right Arrow \\ \hline
U+02198 & {\EmojiFont ↘} & {\textbackslash}:arrow\_lower\_right:, {\textbackslash}searrow & South East Arrow / Lower Right Arrow \\ \hline
U+02199 & {\EmojiFont ↙} & {\textbackslash}:arrow\_lower\_left:, {\textbackslash}swarrow & South West Arrow / Lower Left Arrow \\ \hline
U+021A9 & {\EmojiFont ↩} & {\textbackslash}:leftwards\_arrow\_with\_hook:, {\textbackslash}hookleftarrow & Leftwards Arrow With Hook / Left Arrow With Hook \\ \hline
U+021AA & {\EmojiFont ↪} & {\textbackslash}:arrow\_right\_hook:, {\textbackslash}hookrightarrow & Rightwards Arrow With Hook / Right Arrow With Hook \\ \hline
U+0231A & {\EmojiFont ⌚} & {\textbackslash}:watch: & Watch \\ \hline
U+0231B & {\EmojiFont ⌛} & {\textbackslash}:hourglass: & Hourglass \\ \hline
U+023E9 & {\EmojiFont ⏩} & {\textbackslash}:fast\_forward: & Black Right-Pointing Double Triangle \\ \hline
U+023EA & {\EmojiFont ⏪} & {\textbackslash}:rewind: & Black Left-Pointing Double Triangle \\ \hline
U+023EB & {\EmojiFont ⏫} & {\textbackslash}:arrow\_double\_up: & Black Up-Pointing Double Triangle \\ \hline
U+023EC & {\EmojiFont ⏬} & {\textbackslash}:arrow\_double\_down: & Black Down-Pointing Double Triangle \\ \hline
U+023F0 & {\EmojiFont ⏰} & {\textbackslash}:alarm\_clock: & Alarm Clock \\ \hline
U+023F3 & {\EmojiFont ⏳} & {\textbackslash}:hourglass\_flowing\_sand: & Hourglass With Flowing Sand \\ \hline
U+024C2 & {\EmojiFont Ⓜ} & {\textbackslash}:m: & Circled Latin Capital Letter M \\ \hline
U+025AA & {\EmojiFont ▪} & {\textbackslash}:black\_small\_square:, {\textbackslash}smblksquare & Black Small Square \\ \hline
U+025AB & {\EmojiFont ▫} & {\textbackslash}:white\_small\_square:, {\textbackslash}smwhtsquare & White Small Square \\ \hline
U+025B6 & {\EmojiFont ▶} & {\textbackslash}:arrow\_forward:, {\textbackslash}blacktriangleright & Black Right-Pointing Triangle / Black Right Pointing Triangle \\ \hline
U+025C0 & {\EmojiFont ◀} & {\textbackslash}:arrow\_backward:, {\textbackslash}blacktriangleleft & Black Left-Pointing Triangle / Black Left Pointing Triangle \\ \hline
U+025FB & {\EmojiFont ◻} & {\textbackslash}:white\_medium\_square:, {\textbackslash}mdwhtsquare & White Medium Square \\ \hline
U+025FC & {\EmojiFont ◼} & {\textbackslash}:black\_medium\_square:, {\textbackslash}mdblksquare & Black Medium Square \\ \hline
U+025FD & {\EmojiFont ◽} & {\textbackslash}:white\_medium\_small\_square:, {\textbackslash}mdsmwhtsquare & White Medium Small Square \\ \hline
U+025FE & {\EmojiFont ◾} & {\textbackslash}:black\_medium\_small\_square:, {\textbackslash}mdsmblksquare & Black Medium Small Square \\ \hline
U+02600 & {\EmojiFont ☀} & {\textbackslash}:sunny: & Black Sun With Rays \\ \hline
U+02601 & {\EmojiFont ☁} & {\textbackslash}:cloud: & Cloud \\ \hline
U+0260E & {\EmojiFont ☎} & {\textbackslash}:phone: & Black Telephone \\ \hline
U+02611 & {\EmojiFont ☑} & {\textbackslash}:ballot\_box\_with\_check: & Ballot Box With Check \\ \hline
U+02614 & {\EmojiFont ☔} & {\textbackslash}:umbrella: & Umbrella With Rain Drops \\ \hline
U+02615 & {\EmojiFont ☕} & {\textbackslash}:coffee: & Hot Beverage \\ \hline
U+0261D & {\EmojiFont ☝} & {\textbackslash}:point\_up: & White Up Pointing Index \\ \hline
U+0263A & {\EmojiFont ☺} & {\textbackslash}:relaxed: & White Smiling Face \\ \hline
U+02648 & {\EmojiFont ♈} & {\textbackslash}:aries:, {\textbackslash}aries & Aries \\ \hline
U+02649 & {\EmojiFont ♉} & {\textbackslash}:taurus:, {\textbackslash}taurus & Taurus \\ \hline
U+0264A & {\EmojiFont ♊} & {\textbackslash}:gemini:, {\textbackslash}gemini & Gemini \\ \hline
U+0264B & {\EmojiFont ♋} & {\textbackslash}:cancer:, {\textbackslash}cancer & Cancer \\ \hline
U+0264C & {\EmojiFont ♌} & {\textbackslash}:leo:, {\textbackslash}leo & Leo \\ \hline
U+0264D & {\EmojiFont ♍} & {\textbackslash}:virgo:, {\textbackslash}virgo & Virgo \\ \hline
U+0264E & {\EmojiFont ♎} & {\textbackslash}:libra:, {\textbackslash}libra & Libra \\ \hline
U+0264F & {\EmojiFont ♏} & {\textbackslash}:scorpius:, {\textbackslash}scorpio & Scorpius \\ \hline
U+02650 & {\EmojiFont ♐} & {\textbackslash}:sagittarius:, {\textbackslash}sagittarius & Sagittarius \\ \hline
U+02651 & {\EmojiFont ♑} & {\textbackslash}:capricorn:, {\textbackslash}capricornus & Capricorn \\ \hline
U+02652 & {\EmojiFont ♒} & {\textbackslash}:aquarius:, {\textbackslash}aquarius & Aquarius \\ \hline
U+02653 & {\EmojiFont ♓} & {\textbackslash}:pisces:, {\textbackslash}pisces & Pisces \\ \hline
U+02660 & {\EmojiFont ♠} & {\textbackslash}:spades:, {\textbackslash}spadesuit & Black Spade Suit \\ \hline
U+02663 & {\EmojiFont ♣} & {\textbackslash}:clubs:, {\textbackslash}clubsuit & Black Club Suit \\ \hline
U+02665 & {\EmojiFont ♥} & {\textbackslash}:hearts:, {\textbackslash}varheartsuit & Black Heart Suit \\ \hline
U+02666 & {\EmojiFont ♦} & {\textbackslash}:diamonds:, {\textbackslash}vardiamondsuit & Black Diamond Suit \\ \hline
U+02668 & {\EmojiFont ♨} & {\textbackslash}:hotsprings: & Hot Springs \\ \hline
U+0267B & {\EmojiFont ♻} & {\textbackslash}:recycle: & Black Universal Recycling Symbol \\ \hline
U+0267F & {\EmojiFont ♿} & {\textbackslash}:wheelchair: & Wheelchair Symbol \\ \hline
U+02693 & {\EmojiFont ⚓} & {\textbackslash}:anchor: & Anchor \\ \hline
U+026A0 & {\EmojiFont ⚠} & {\textbackslash}:warning: & Warning Sign \\ \hline
U+026A1 & {\EmojiFont ⚡} & {\textbackslash}:zap: & High Voltage Sign \\ \hline
U+026AA & {\EmojiFont ⚪} & {\textbackslash}:white\_circle:, {\textbackslash}mdwhtcircle & Medium White Circle \\ \hline
U+026AB & {\EmojiFont ⚫} & {\textbackslash}:black\_circle:, {\textbackslash}mdblkcircle & Medium Black Circle \\ \hline
U+026BD & {\EmojiFont ⚽} & {\textbackslash}:soccer: & Soccer Ball \\ \hline
U+026BE & {\EmojiFont ⚾} & {\textbackslash}:baseball: & Baseball \\ \hline
U+026C4 & {\EmojiFont ⛄} & {\textbackslash}:snowman: & Snowman Without Snow \\ \hline
U+026C5 & {\EmojiFont ⛅} & {\textbackslash}:partly\_sunny: & Sun Behind Cloud \\ \hline
U+026CE & {\EmojiFont ⛎} & {\textbackslash}:ophiuchus: & Ophiuchus \\ \hline
U+026D4 & {\EmojiFont ⛔} & {\textbackslash}:no\_entry: & No Entry \\ \hline
U+026EA & {\EmojiFont ⛪} & {\textbackslash}:church: & Church \\ \hline
U+026F2 & {\EmojiFont ⛲} & {\textbackslash}:fountain: & Fountain \\ \hline
U+026F3 & {\EmojiFont ⛳} & {\textbackslash}:golf: & Flag In Hole \\ \hline
U+026F5 & {\EmojiFont ⛵} & {\textbackslash}:boat: & Sailboat \\ \hline
U+026FA & {\EmojiFont ⛺} & {\textbackslash}:tent: & Tent \\ \hline
U+026FD & {\EmojiFont ⛽} & {\textbackslash}:fuelpump: & Fuel Pump \\ \hline
U+02702 & {\EmojiFont ✂} & {\textbackslash}:scissors: & Black Scissors \\ \hline
U+02705 & {\EmojiFont ✅} & {\textbackslash}:white\_check\_mark: & White Heavy Check Mark \\ \hline
U+02708 & {\EmojiFont ✈} & {\textbackslash}:airplane: & Airplane \\ \hline
U+02709 & {\EmojiFont ✉} & {\textbackslash}:email: & Envelope \\ \hline
U+0270A & {\EmojiFont ✊} & {\textbackslash}:fist: & Raised Fist \\ \hline
U+0270B & {\EmojiFont ✋} & {\textbackslash}:hand: & Raised Hand \\ \hline
U+0270C & {\EmojiFont ✌} & {\textbackslash}:v: & Victory Hand \\ \hline
U+0270F & {\EmojiFont ✏} & {\textbackslash}:pencil2: & Pencil \\ \hline
U+02712 & {\EmojiFont ✒} & {\textbackslash}:black\_nib: & Black Nib \\ \hline
U+02714 & {\EmojiFont ✔} & {\textbackslash}:heavy\_check\_mark: & Heavy Check Mark \\ \hline
U+02716 & {\EmojiFont ✖} & {\textbackslash}:heavy\_multiplication\_x: & Heavy Multiplication X \\ \hline
U+02728 & {\EmojiFont ✨} & {\textbackslash}:sparkles: & Sparkles \\ \hline
U+02733 & {\EmojiFont ✳} & {\textbackslash}:eight\_spoked\_asterisk: & Eight Spoked Asterisk \\ \hline
U+02734 & {\EmojiFont ✴} & {\textbackslash}:eight\_pointed\_black\_star: & Eight Pointed Black Star \\ \hline
U+02744 & {\EmojiFont ❄} & {\textbackslash}:snowflake: & Snowflake \\ \hline
U+02747 & {\EmojiFont ❇} & {\textbackslash}:sparkle: & Sparkle \\ \hline
U+0274C & {\EmojiFont ❌} & {\textbackslash}:x: & Cross Mark \\ \hline
U+0274E & {\EmojiFont ❎} & {\textbackslash}:negative\_squared\_cross\_mark: & Negative Squared Cross Mark \\ \hline
U+02753 & {\EmojiFont ❓} & {\textbackslash}:question: & Black Question Mark Ornament \\ \hline
U+02754 & {\EmojiFont ❔} & {\textbackslash}:grey\_question: & White Question Mark Ornament \\ \hline
U+02755 & {\EmojiFont ❕} & {\textbackslash}:grey\_exclamation: & White Exclamation Mark Ornament \\ \hline
U+02757 & {\EmojiFont ❗} & {\textbackslash}:exclamation: & Heavy Exclamation Mark Symbol \\ \hline
U+02764 & {\EmojiFont ❤} & {\textbackslash}:heart: & Heavy Black Heart \\ \hline
U+02795 & {\EmojiFont ➕} & {\textbackslash}:heavy\_plus\_sign: & Heavy Plus Sign \\ \hline
U+02796 & {\EmojiFont ➖} & {\textbackslash}:heavy\_minus\_sign: & Heavy Minus Sign \\ \hline
U+02797 & {\EmojiFont ➗} & {\textbackslash}:heavy\_division\_sign: & Heavy Division Sign \\ \hline
U+027A1 & {\EmojiFont ➡} & {\textbackslash}:arrow\_right: & Black Rightwards Arrow / Black Right Arrow \\ \hline
U+027B0 & {\EmojiFont ➰} & {\textbackslash}:curly\_loop: & Curly Loop \\ \hline
U+027BF & {\EmojiFont ➿} & {\textbackslash}:loop: & Double Curly Loop \\ \hline
U+02934 & {\EmojiFont ⤴} & {\textbackslash}:arrow\_heading\_up: & Arrow Pointing Rightwards Then Curving Upwards \\ \hline
U+02935 & {\EmojiFont ⤵} & {\textbackslash}:arrow\_heading\_down: & Arrow Pointing Rightwards Then Curving Downwards \\ \hline
U+02B05 & {\EmojiFont ⬅} & {\textbackslash}:arrow\_left: & Leftwards Black Arrow \\ \hline
U+02B06 & {\EmojiFont ⬆} & {\textbackslash}:arrow\_up: & Upwards Black Arrow \\ \hline
U+02B07 & {\EmojiFont ⬇} & {\textbackslash}:arrow\_down: & Downwards Black Arrow \\ \hline
U+02B1B & {\EmojiFont ⬛} & {\textbackslash}:black\_large\_square:, {\textbackslash}lgblksquare & Black Large Square \\ \hline
U+02B1C & {\EmojiFont ⬜} & {\textbackslash}:white\_large\_square:, {\textbackslash}lgwhtsquare & White Large Square \\ \hline
U+02B50 & {\EmojiFont ⭐} & {\textbackslash}:star:, {\textbackslash}medwhitestar & White Medium Star \\ \hline
U+02B55 & {\EmojiFont ⭕} & {\textbackslash}:o: & Heavy Large Circle \\ \hline
U+03030 & {\EmojiFont 〰} & {\textbackslash}:wavy\_dash: & Wavy Dash \\ \hline
U+0303D & {\EmojiFont 〽} & {\textbackslash}:part\_alternation\_mark: & Part Alternation Mark \\ \hline
U+03297 & {\EmojiFont ㊗} & {\textbackslash}:congratulations: & Circled Ideograph Congratulation \\ \hline
U+03299 & {\EmojiFont ㊙} & {\textbackslash}:secret: & Circled Ideograph Secret \\ \hline
U+1F004 & {\EmojiFont 🀄} & {\textbackslash}:mahjong: & Mahjong Tile Red Dragon \\ \hline
U+1F0CF & {\EmojiFont 🃏} & {\textbackslash}:black\_joker: & Playing Card Black Joker \\ \hline
U+1F170 & {\EmojiFont 🅰} & {\textbackslash}:a: & Negative Squared Latin Capital Letter A \\ \hline
U+1F171 & {\EmojiFont 🅱} & {\textbackslash}:b: & Negative Squared Latin Capital Letter B \\ \hline
U+1F17E & {\EmojiFont 🅾} & {\textbackslash}:o2: & Negative Squared Latin Capital Letter O \\ \hline
U+1F17F & {\EmojiFont 🅿} & {\textbackslash}:parking: & Negative Squared Latin Capital Letter P \\ \hline
U+1F18E & {\EmojiFont 🆎} & {\textbackslash}:ab: & Negative Squared Ab \\ \hline
U+1F191 & {\EmojiFont 🆑} & {\textbackslash}:cl: & Squared Cl \\ \hline
U+1F192 & {\EmojiFont 🆒} & {\textbackslash}:cool: & Squared Cool \\ \hline
U+1F193 & {\EmojiFont 🆓} & {\textbackslash}:free: & Squared Free \\ \hline
U+1F194 & {\EmojiFont 🆔} & {\textbackslash}:id: & Squared Id \\ \hline
U+1F195 & {\EmojiFont 🆕} & {\textbackslash}:new: & Squared New \\ \hline
U+1F196 & {\EmojiFont 🆖} & {\textbackslash}:ng: & Squared Ng \\ \hline
U+1F197 & {\EmojiFont 🆗} & {\textbackslash}:ok: & Squared Ok \\ \hline
U+1F198 & {\EmojiFont 🆘} & {\textbackslash}:sos: & Squared Sos \\ \hline
U+1F199 & {\EmojiFont 🆙} & {\textbackslash}:up: & Squared Up With Exclamation Mark \\ \hline
U+1F19A & {\EmojiFont 🆚} & {\textbackslash}:vs: & Squared Vs \\ \hline
U+1F201 & {\EmojiFont 🈁} & {\textbackslash}:koko: & Squared Katakana Koko \\ \hline
U+1F202 & {\EmojiFont 🈂} & {\textbackslash}:sa: & Squared Katakana Sa \\ \hline
U+1F21A & {\EmojiFont 🈚} & {\textbackslash}:u7121: & Squared Cjk Unified Ideograph-7121 \\ \hline
U+1F22F & {\EmojiFont 🈯} & {\textbackslash}:u6307: & Squared Cjk Unified Ideograph-6307 \\ \hline
U+1F232 & {\EmojiFont 🈲} & {\textbackslash}:u7981: & Squared Cjk Unified Ideograph-7981 \\ \hline
U+1F233 & {\EmojiFont 🈳} & {\textbackslash}:u7a7a: & Squared Cjk Unified Ideograph-7A7A \\ \hline
U+1F234 & {\EmojiFont 🈴} & {\textbackslash}:u5408: & Squared Cjk Unified Ideograph-5408 \\ \hline
U+1F235 & {\EmojiFont 🈵} & {\textbackslash}:u6e80: & Squared Cjk Unified Ideograph-6E80 \\ \hline
U+1F236 & {\EmojiFont 🈶} & {\textbackslash}:u6709: & Squared Cjk Unified Ideograph-6709 \\ \hline
U+1F237 & {\EmojiFont 🈷} & {\textbackslash}:u6708: & Squared Cjk Unified Ideograph-6708 \\ \hline
U+1F238 & {\EmojiFont 🈸} & {\textbackslash}:u7533: & Squared Cjk Unified Ideograph-7533 \\ \hline
U+1F239 & {\EmojiFont 🈹} & {\textbackslash}:u5272: & Squared Cjk Unified Ideograph-5272 \\ \hline
U+1F23A & {\EmojiFont 🈺} & {\textbackslash}:u55b6: & Squared Cjk Unified Ideograph-55B6 \\ \hline
U+1F250 & {\EmojiFont 🉐} & {\textbackslash}:ideograph\_advantage: & Circled Ideograph Advantage \\ \hline
U+1F251 & {\EmojiFont 🉑} & {\textbackslash}:accept: & Circled Ideograph Accept \\ \hline
U+1F300 & {\EmojiFont 🌀} & {\textbackslash}:cyclone: & Cyclone \\ \hline
U+1F301 & {\EmojiFont 🌁} & {\textbackslash}:foggy: & Foggy \\ \hline
U+1F302 & {\EmojiFont 🌂} & {\textbackslash}:closed\_umbrella: & Closed Umbrella \\ \hline
U+1F303 & {\EmojiFont 🌃} & {\textbackslash}:night\_with\_stars: & Night With Stars \\ \hline
U+1F304 & {\EmojiFont 🌄} & {\textbackslash}:sunrise\_over\_mountains: & Sunrise Over Mountains \\ \hline
U+1F305 & {\EmojiFont 🌅} & {\textbackslash}:sunrise: & Sunrise \\ \hline
U+1F306 & {\EmojiFont 🌆} & {\textbackslash}:city\_sunset: & Cityscape At Dusk \\ \hline
U+1F307 & {\EmojiFont 🌇} & {\textbackslash}:city\_sunrise: & Sunset Over Buildings \\ \hline
U+1F308 & {\EmojiFont 🌈} & {\textbackslash}:rainbow: & Rainbow \\ \hline
U+1F309 & {\EmojiFont 🌉} & {\textbackslash}:bridge\_at\_night: & Bridge At Night \\ \hline
U+1F30A & {\EmojiFont 🌊} & {\textbackslash}:ocean: & Water Wave \\ \hline
U+1F30B & {\EmojiFont 🌋} & {\textbackslash}:volcano: & Volcano \\ \hline
U+1F30C & {\EmojiFont 🌌} & {\textbackslash}:milky\_way: & Milky Way \\ \hline
U+1F30D & {\EmojiFont 🌍} & {\textbackslash}:earth\_africa: & Earth Globe Europe-Africa \\ \hline
U+1F30E & {\EmojiFont 🌎} & {\textbackslash}:earth\_americas: & Earth Globe Americas \\ \hline
U+1F30F & {\EmojiFont 🌏} & {\textbackslash}:earth\_asia: & Earth Globe Asia-Australia \\ \hline
U+1F310 & {\EmojiFont 🌐} & {\textbackslash}:globe\_with\_meridians: & Globe With Meridians \\ \hline
U+1F311 & {\EmojiFont 🌑} & {\textbackslash}:new\_moon: & New Moon Symbol \\ \hline
U+1F312 & {\EmojiFont 🌒} & {\textbackslash}:waxing\_crescent\_moon: & Waxing Crescent Moon Symbol \\ \hline
U+1F313 & {\EmojiFont 🌓} & {\textbackslash}:first\_quarter\_moon: & First Quarter Moon Symbol \\ \hline
U+1F314 & {\EmojiFont 🌔} & {\textbackslash}:moon: & Waxing Gibbous Moon Symbol \\ \hline
U+1F315 & {\EmojiFont 🌕} & {\textbackslash}:full\_moon: & Full Moon Symbol \\ \hline
U+1F316 & {\EmojiFont 🌖} & {\textbackslash}:waning\_gibbous\_moon: & Waning Gibbous Moon Symbol \\ \hline
U+1F317 & {\EmojiFont 🌗} & {\textbackslash}:last\_quarter\_moon: & Last Quarter Moon Symbol \\ \hline
U+1F318 & {\EmojiFont 🌘} & {\textbackslash}:waning\_crescent\_moon: & Waning Crescent Moon Symbol \\ \hline
U+1F319 & {\EmojiFont 🌙} & {\textbackslash}:crescent\_moon: & Crescent Moon \\ \hline
U+1F31A & {\EmojiFont 🌚} & {\textbackslash}:new\_moon\_with\_face: & New Moon With Face \\ \hline
U+1F31B & {\EmojiFont 🌛} & {\textbackslash}:first\_quarter\_moon\_with\_face: & First Quarter Moon With Face \\ \hline
U+1F31C & {\EmojiFont 🌜} & {\textbackslash}:last\_quarter\_moon\_with\_face: & Last Quarter Moon With Face \\ \hline
U+1F31D & {\EmojiFont 🌝} & {\textbackslash}:full\_moon\_with\_face: & Full Moon With Face \\ \hline
U+1F31E & {\EmojiFont 🌞} & {\textbackslash}:sun\_with\_face: & Sun With Face \\ \hline
U+1F31F & {\EmojiFont 🌟} & {\textbackslash}:star2: & Glowing Star \\ \hline
U+1F320 & {\EmojiFont 🌠} & {\textbackslash}:stars: & Shooting Star \\ \hline
U+1F330 & {\EmojiFont 🌰} & {\textbackslash}:chestnut: & Chestnut \\ \hline
U+1F331 & {\EmojiFont 🌱} & {\textbackslash}:seedling: & Seedling \\ \hline
U+1F332 & {\EmojiFont 🌲} & {\textbackslash}:evergreen\_tree: & Evergreen Tree \\ \hline
U+1F333 & {\EmojiFont 🌳} & {\textbackslash}:deciduous\_tree: & Deciduous Tree \\ \hline
U+1F334 & {\EmojiFont 🌴} & {\textbackslash}:palm\_tree: & Palm Tree \\ \hline
U+1F335 & {\EmojiFont 🌵} & {\textbackslash}:cactus: & Cactus \\ \hline
U+1F337 & {\EmojiFont 🌷} & {\textbackslash}:tulip: & Tulip \\ \hline
U+1F338 & {\EmojiFont 🌸} & {\textbackslash}:cherry\_blossom: & Cherry Blossom \\ \hline
U+1F339 & {\EmojiFont 🌹} & {\textbackslash}:rose: & Rose \\ \hline
U+1F33A & {\EmojiFont 🌺} & {\textbackslash}:hibiscus: & Hibiscus \\ \hline
U+1F33B & {\EmojiFont 🌻} & {\textbackslash}:sunflower: & Sunflower \\ \hline
U+1F33C & {\EmojiFont 🌼} & {\textbackslash}:blossom: & Blossom \\ \hline
U+1F33D & {\EmojiFont 🌽} & {\textbackslash}:corn: & Ear Of Maize \\ \hline
U+1F33E & {\EmojiFont 🌾} & {\textbackslash}:ear\_of\_rice: & Ear Of Rice \\ \hline
U+1F33F & {\EmojiFont 🌿} & {\textbackslash}:herb: & Herb \\ \hline
U+1F340 & {\EmojiFont 🍀} & {\textbackslash}:four\_leaf\_clover: & Four Leaf Clover \\ \hline
U+1F341 & {\EmojiFont 🍁} & {\textbackslash}:maple\_leaf: & Maple Leaf \\ \hline
U+1F342 & {\EmojiFont 🍂} & {\textbackslash}:fallen\_leaf: & Fallen Leaf \\ \hline
U+1F343 & {\EmojiFont 🍃} & {\textbackslash}:leaves: & Leaf Fluttering In Wind \\ \hline
U+1F344 & {\EmojiFont 🍄} & {\textbackslash}:mushroom: & Mushroom \\ \hline
U+1F345 & {\EmojiFont 🍅} & {\textbackslash}:tomato: & Tomato \\ \hline
U+1F346 & {\EmojiFont 🍆} & {\textbackslash}:eggplant: & Aubergine \\ \hline
U+1F347 & {\EmojiFont 🍇} & {\textbackslash}:grapes: & Grapes \\ \hline
U+1F348 & {\EmojiFont 🍈} & {\textbackslash}:melon: & Melon \\ \hline
U+1F349 & {\EmojiFont 🍉} & {\textbackslash}:watermelon: & Watermelon \\ \hline
U+1F34A & {\EmojiFont 🍊} & {\textbackslash}:tangerine: & Tangerine \\ \hline
U+1F34B & {\EmojiFont 🍋} & {\textbackslash}:lemon: & Lemon \\ \hline
U+1F34C & {\EmojiFont 🍌} & {\textbackslash}:banana: & Banana \\ \hline
U+1F34D & {\EmojiFont 🍍} & {\textbackslash}:pineapple: & Pineapple \\ \hline
U+1F34E & {\EmojiFont 🍎} & {\textbackslash}:apple: & Red Apple \\ \hline
U+1F34F & {\EmojiFont 🍏} & {\textbackslash}:green\_apple: & Green Apple \\ \hline
U+1F350 & {\EmojiFont 🍐} & {\textbackslash}:pear: & Pear \\ \hline
U+1F351 & {\EmojiFont 🍑} & {\textbackslash}:peach: & Peach \\ \hline
U+1F352 & {\EmojiFont 🍒} & {\textbackslash}:cherries: & Cherries \\ \hline
U+1F353 & {\EmojiFont 🍓} & {\textbackslash}:strawberry: & Strawberry \\ \hline
U+1F354 & {\EmojiFont 🍔} & {\textbackslash}:hamburger: & Hamburger \\ \hline
U+1F355 & {\EmojiFont 🍕} & {\textbackslash}:pizza: & Slice Of Pizza \\ \hline
U+1F356 & {\EmojiFont 🍖} & {\textbackslash}:meat\_on\_bone: & Meat On Bone \\ \hline
U+1F357 & {\EmojiFont 🍗} & {\textbackslash}:poultry\_leg: & Poultry Leg \\ \hline
U+1F358 & {\EmojiFont 🍘} & {\textbackslash}:rice\_cracker: & Rice Cracker \\ \hline
U+1F359 & {\EmojiFont 🍙} & {\textbackslash}:rice\_ball: & Rice Ball \\ \hline
U+1F35A & {\EmojiFont 🍚} & {\textbackslash}:rice: & Cooked Rice \\ \hline
U+1F35B & {\EmojiFont 🍛} & {\textbackslash}:curry: & Curry And Rice \\ \hline
U+1F35C & {\EmojiFont 🍜} & {\textbackslash}:ramen: & Steaming Bowl \\ \hline
U+1F35D & {\EmojiFont 🍝} & {\textbackslash}:spaghetti: & Spaghetti \\ \hline
U+1F35E & {\EmojiFont 🍞} & {\textbackslash}:bread: & Bread \\ \hline
U+1F35F & {\EmojiFont 🍟} & {\textbackslash}:fries: & French Fries \\ \hline
U+1F360 & {\EmojiFont 🍠} & {\textbackslash}:sweet\_potato: & Roasted Sweet Potato \\ \hline
U+1F361 & {\EmojiFont 🍡} & {\textbackslash}:dango: & Dango \\ \hline
U+1F362 & {\EmojiFont 🍢} & {\textbackslash}:oden: & Oden \\ \hline
U+1F363 & {\EmojiFont 🍣} & {\textbackslash}:sushi: & Sushi \\ \hline
U+1F364 & {\EmojiFont 🍤} & {\textbackslash}:fried\_shrimp: & Fried Shrimp \\ \hline
U+1F365 & {\EmojiFont 🍥} & {\textbackslash}:fish\_cake: & Fish Cake With Swirl Design \\ \hline
U+1F366 & {\EmojiFont 🍦} & {\textbackslash}:icecream: & Soft Ice Cream \\ \hline
U+1F367 & {\EmojiFont 🍧} & {\textbackslash}:shaved\_ice: & Shaved Ice \\ \hline
U+1F368 & {\EmojiFont 🍨} & {\textbackslash}:ice\_cream: & Ice Cream \\ \hline
U+1F369 & {\EmojiFont 🍩} & {\textbackslash}:doughnut: & Doughnut \\ \hline
U+1F36A & {\EmojiFont 🍪} & {\textbackslash}:cookie: & Cookie \\ \hline
U+1F36B & {\EmojiFont 🍫} & {\textbackslash}:chocolate\_bar: & Chocolate Bar \\ \hline
U+1F36C & {\EmojiFont 🍬} & {\textbackslash}:candy: & Candy \\ \hline
U+1F36D & {\EmojiFont 🍭} & {\textbackslash}:lollipop: & Lollipop \\ \hline
U+1F36E & {\EmojiFont 🍮} & {\textbackslash}:custard: & Custard \\ \hline
U+1F36F & {\EmojiFont 🍯} & {\textbackslash}:honey\_pot: & Honey Pot \\ \hline
U+1F370 & {\EmojiFont 🍰} & {\textbackslash}:cake: & Shortcake \\ \hline
U+1F371 & {\EmojiFont 🍱} & {\textbackslash}:bento: & Bento Box \\ \hline
U+1F372 & {\EmojiFont 🍲} & {\textbackslash}:stew: & Pot Of Food \\ \hline
U+1F373 & {\EmojiFont 🍳} & {\textbackslash}:egg: & Cooking \\ \hline
U+1F374 & {\EmojiFont 🍴} & {\textbackslash}:fork\_and\_knife: & Fork And Knife \\ \hline
U+1F375 & {\EmojiFont 🍵} & {\textbackslash}:tea: & Teacup Without Handle \\ \hline
U+1F376 & {\EmojiFont 🍶} & {\textbackslash}:sake: & Sake Bottle And Cup \\ \hline
U+1F377 & {\EmojiFont 🍷} & {\textbackslash}:wine\_glass: & Wine Glass \\ \hline
U+1F378 & {\EmojiFont 🍸} & {\textbackslash}:cocktail: & Cocktail Glass \\ \hline
U+1F379 & {\EmojiFont 🍹} & {\textbackslash}:tropical\_drink: & Tropical Drink \\ \hline
U+1F37A & {\EmojiFont 🍺} & {\textbackslash}:beer: & Beer Mug \\ \hline
U+1F37B & {\EmojiFont 🍻} & {\textbackslash}:beers: & Clinking Beer Mugs \\ \hline
U+1F37C & {\EmojiFont 🍼} & {\textbackslash}:baby\_bottle: & Baby Bottle \\ \hline
U+1F380 & {\EmojiFont 🎀} & {\textbackslash}:ribbon: & Ribbon \\ \hline
U+1F381 & {\EmojiFont 🎁} & {\textbackslash}:gift: & Wrapped Present \\ \hline
U+1F382 & {\EmojiFont 🎂} & {\textbackslash}:birthday: & Birthday Cake \\ \hline
U+1F383 & {\EmojiFont 🎃} & {\textbackslash}:jack\_o\_lantern: & Jack-O-Lantern \\ \hline
U+1F384 & {\EmojiFont 🎄} & {\textbackslash}:christmas\_tree: & Christmas Tree \\ \hline
U+1F385 & {\EmojiFont 🎅} & {\textbackslash}:santa: & Father Christmas \\ \hline
U+1F386 & {\EmojiFont 🎆} & {\textbackslash}:fireworks: & Fireworks \\ \hline
U+1F387 & {\EmojiFont 🎇} & {\textbackslash}:sparkler: & Firework Sparkler \\ \hline
U+1F388 & {\EmojiFont 🎈} & {\textbackslash}:balloon: & Balloon \\ \hline
U+1F389 & {\EmojiFont 🎉} & {\textbackslash}:tada: & Party Popper \\ \hline
U+1F38A & {\EmojiFont 🎊} & {\textbackslash}:confetti\_ball: & Confetti Ball \\ \hline
U+1F38B & {\EmojiFont 🎋} & {\textbackslash}:tanabata\_tree: & Tanabata Tree \\ \hline
U+1F38C & {\EmojiFont 🎌} & {\textbackslash}:crossed\_flags: & Crossed Flags \\ \hline
U+1F38D & {\EmojiFont 🎍} & {\textbackslash}:bamboo: & Pine Decoration \\ \hline
U+1F38E & {\EmojiFont 🎎} & {\textbackslash}:dolls: & Japanese Dolls \\ \hline
U+1F38F & {\EmojiFont 🎏} & {\textbackslash}:flags: & Carp Streamer \\ \hline
U+1F390 & {\EmojiFont 🎐} & {\textbackslash}:wind\_chime: & Wind Chime \\ \hline
U+1F391 & {\EmojiFont 🎑} & {\textbackslash}:rice\_scene: & Moon Viewing Ceremony \\ \hline
U+1F392 & {\EmojiFont 🎒} & {\textbackslash}:school\_satchel: & School Satchel \\ \hline
U+1F393 & {\EmojiFont 🎓} & {\textbackslash}:mortar\_board: & Graduation Cap \\ \hline
U+1F3A0 & {\EmojiFont 🎠} & {\textbackslash}:carousel\_horse: & Carousel Horse \\ \hline
U+1F3A1 & {\EmojiFont 🎡} & {\textbackslash}:ferris\_wheel: & Ferris Wheel \\ \hline
U+1F3A2 & {\EmojiFont 🎢} & {\textbackslash}:roller\_coaster: & Roller Coaster \\ \hline
U+1F3A3 & {\EmojiFont 🎣} & {\textbackslash}:fishing\_pole\_and\_fish: & Fishing Pole And Fish \\ \hline
U+1F3A4 & {\EmojiFont 🎤} & {\textbackslash}:microphone: & Microphone \\ \hline
U+1F3A5 & {\EmojiFont 🎥} & {\textbackslash}:movie\_camera: & Movie Camera \\ \hline
U+1F3A6 & {\EmojiFont 🎦} & {\textbackslash}:cinema: & Cinema \\ \hline
U+1F3A7 & {\EmojiFont 🎧} & {\textbackslash}:headphones: & Headphone \\ \hline
U+1F3A8 & {\EmojiFont 🎨} & {\textbackslash}:art: & Artist Palette \\ \hline
U+1F3A9 & {\EmojiFont 🎩} & {\textbackslash}:tophat: & Top Hat \\ \hline
U+1F3AA & {\EmojiFont 🎪} & {\textbackslash}:circus\_tent: & Circus Tent \\ \hline
U+1F3AB & {\EmojiFont 🎫} & {\textbackslash}:ticket: & Ticket \\ \hline
U+1F3AC & {\EmojiFont 🎬} & {\textbackslash}:clapper: & Clapper Board \\ \hline
U+1F3AD & {\EmojiFont 🎭} & {\textbackslash}:performing\_arts: & Performing Arts \\ \hline
U+1F3AE & {\EmojiFont 🎮} & {\textbackslash}:video\_game: & Video Game \\ \hline
U+1F3AF & {\EmojiFont 🎯} & {\textbackslash}:dart: & Direct Hit \\ \hline
U+1F3B0 & {\EmojiFont 🎰} & {\textbackslash}:slot\_machine: & Slot Machine \\ \hline
U+1F3B1 & {\EmojiFont 🎱} & {\textbackslash}:8ball: & Billiards \\ \hline
U+1F3B2 & {\EmojiFont 🎲} & {\textbackslash}:game\_die: & Game Die \\ \hline
U+1F3B3 & {\EmojiFont 🎳} & {\textbackslash}:bowling: & Bowling \\ \hline
U+1F3B4 & {\EmojiFont 🎴} & {\textbackslash}:flower\_playing\_cards: & Flower Playing Cards \\ \hline
U+1F3B5 & {\EmojiFont 🎵} & {\textbackslash}:musical\_note: & Musical Note \\ \hline
U+1F3B6 & {\EmojiFont 🎶} & {\textbackslash}:notes: & Multiple Musical Notes \\ \hline
U+1F3B7 & {\EmojiFont 🎷} & {\textbackslash}:saxophone: & Saxophone \\ \hline
U+1F3B8 & {\EmojiFont 🎸} & {\textbackslash}:guitar: & Guitar \\ \hline
U+1F3B9 & {\EmojiFont 🎹} & {\textbackslash}:musical\_keyboard: & Musical Keyboard \\ \hline
U+1F3BA & {\EmojiFont 🎺} & {\textbackslash}:trumpet: & Trumpet \\ \hline
U+1F3BB & {\EmojiFont 🎻} & {\textbackslash}:violin: & Violin \\ \hline
U+1F3BC & {\EmojiFont 🎼} & {\textbackslash}:musical\_score: & Musical Score \\ \hline
U+1F3BD & {\EmojiFont 🎽} & {\textbackslash}:running\_shirt\_with\_sash: & Running Shirt With Sash \\ \hline
U+1F3BE & {\EmojiFont 🎾} & {\textbackslash}:tennis: & Tennis Racquet And Ball \\ \hline
U+1F3BF & {\EmojiFont 🎿} & {\textbackslash}:ski: & Ski And Ski Boot \\ \hline
U+1F3C0 & {\EmojiFont 🏀} & {\textbackslash}:basketball: & Basketball And Hoop \\ \hline
U+1F3C1 & {\EmojiFont 🏁} & {\textbackslash}:checkered\_flag: & Chequered Flag \\ \hline
U+1F3C2 & {\EmojiFont 🏂} & {\textbackslash}:snowboarder: & Snowboarder \\ \hline
U+1F3C3 & {\EmojiFont 🏃} & {\textbackslash}:runner: & Runner \\ \hline
U+1F3C4 & {\EmojiFont 🏄} & {\textbackslash}:surfer: & Surfer \\ \hline
U+1F3C6 & {\EmojiFont 🏆} & {\textbackslash}:trophy: & Trophy \\ \hline
U+1F3C7 & {\EmojiFont 🏇} & {\textbackslash}:horse\_racing: & Horse Racing \\ \hline
U+1F3C8 & {\EmojiFont 🏈} & {\textbackslash}:football: & American Football \\ \hline
U+1F3C9 & {\EmojiFont 🏉} & {\textbackslash}:rugby\_football: & Rugby Football \\ \hline
U+1F3CA & {\EmojiFont 🏊} & {\textbackslash}:swimmer: & Swimmer \\ \hline
U+1F3E0 & {\EmojiFont 🏠} & {\textbackslash}:house: & House Building \\ \hline
U+1F3E1 & {\EmojiFont 🏡} & {\textbackslash}:house\_with\_garden: & House With Garden \\ \hline
U+1F3E2 & {\EmojiFont 🏢} & {\textbackslash}:office: & Office Building \\ \hline
U+1F3E3 & {\EmojiFont 🏣} & {\textbackslash}:post\_office: & Japanese Post Office \\ \hline
U+1F3E4 & {\EmojiFont 🏤} & {\textbackslash}:european\_post\_office: & European Post Office \\ \hline
U+1F3E5 & {\EmojiFont 🏥} & {\textbackslash}:hospital: & Hospital \\ \hline
U+1F3E6 & {\EmojiFont 🏦} & {\textbackslash}:bank: & Bank \\ \hline
U+1F3E7 & {\EmojiFont 🏧} & {\textbackslash}:atm: & Automated Teller Machine \\ \hline
U+1F3E8 & {\EmojiFont 🏨} & {\textbackslash}:hotel: & Hotel \\ \hline
U+1F3E9 & {\EmojiFont 🏩} & {\textbackslash}:love\_hotel: & Love Hotel \\ \hline
U+1F3EA & {\EmojiFont 🏪} & {\textbackslash}:convenience\_store: & Convenience Store \\ \hline
U+1F3EB & {\EmojiFont 🏫} & {\textbackslash}:school: & School \\ \hline
U+1F3EC & {\EmojiFont 🏬} & {\textbackslash}:department\_store: & Department Store \\ \hline
U+1F3ED & {\EmojiFont 🏭} & {\textbackslash}:factory: & Factory \\ \hline
U+1F3EE & {\EmojiFont 🏮} & {\textbackslash}:izakaya\_lantern: & Izakaya Lantern \\ \hline
U+1F3EF & {\EmojiFont 🏯} & {\textbackslash}:japanese\_castle: & Japanese Castle \\ \hline
U+1F3F0 & {\EmojiFont 🏰} & {\textbackslash}:european\_castle: & European Castle \\ \hline
U+1F3FB & {\EmojiFont 🏻} & {\textbackslash}:skin-tone-2: & Emoji Modifier Fitzpatrick Type-1-2 \\ \hline
U+1F3FC & {\EmojiFont 🏼} & {\textbackslash}:skin-tone-3: & Emoji Modifier Fitzpatrick Type-3 \\ \hline
U+1F3FD & {\EmojiFont 🏽} & {\textbackslash}:skin-tone-4: & Emoji Modifier Fitzpatrick Type-4 \\ \hline
U+1F3FE & {\EmojiFont 🏾} & {\textbackslash}:skin-tone-5: & Emoji Modifier Fitzpatrick Type-5 \\ \hline
U+1F3FF & {\EmojiFont 🏿} & {\textbackslash}:skin-tone-6: & Emoji Modifier Fitzpatrick Type-6 \\ \hline
U+1F400 & {\EmojiFont 🐀} & {\textbackslash}:rat: & Rat \\ \hline
U+1F401 & {\EmojiFont 🐁} & {\textbackslash}:mouse2: & Mouse \\ \hline
U+1F402 & {\EmojiFont 🐂} & {\textbackslash}:ox: & Ox \\ \hline
U+1F403 & {\EmojiFont 🐃} & {\textbackslash}:water\_buffalo: & Water Buffalo \\ \hline
U+1F404 & {\EmojiFont 🐄} & {\textbackslash}:cow2: & Cow \\ \hline
U+1F405 & {\EmojiFont 🐅} & {\textbackslash}:tiger2: & Tiger \\ \hline
U+1F406 & {\EmojiFont 🐆} & {\textbackslash}:leopard: & Leopard \\ \hline
U+1F407 & {\EmojiFont 🐇} & {\textbackslash}:rabbit2: & Rabbit \\ \hline
U+1F408 & {\EmojiFont 🐈} & {\textbackslash}:cat2: & Cat \\ \hline
U+1F409 & {\EmojiFont 🐉} & {\textbackslash}:dragon: & Dragon \\ \hline
U+1F40A & {\EmojiFont 🐊} & {\textbackslash}:crocodile: & Crocodile \\ \hline
U+1F40B & {\EmojiFont 🐋} & {\textbackslash}:whale2: & Whale \\ \hline
U+1F40C & {\EmojiFont 🐌} & {\textbackslash}:snail: & Snail \\ \hline
U+1F40D & {\EmojiFont 🐍} & {\textbackslash}:snake: & Snake \\ \hline
U+1F40E & {\EmojiFont 🐎} & {\textbackslash}:racehorse: & Horse \\ \hline
U+1F40F & {\EmojiFont 🐏} & {\textbackslash}:ram: & Ram \\ \hline
U+1F410 & {\EmojiFont 🐐} & {\textbackslash}:goat: & Goat \\ \hline
U+1F411 & {\EmojiFont 🐑} & {\textbackslash}:sheep: & Sheep \\ \hline
U+1F412 & {\EmojiFont 🐒} & {\textbackslash}:monkey: & Monkey \\ \hline
U+1F413 & {\EmojiFont 🐓} & {\textbackslash}:rooster: & Rooster \\ \hline
U+1F414 & {\EmojiFont 🐔} & {\textbackslash}:chicken: & Chicken \\ \hline
U+1F415 & {\EmojiFont 🐕} & {\textbackslash}:dog2: & Dog \\ \hline
U+1F416 & {\EmojiFont 🐖} & {\textbackslash}:pig2: & Pig \\ \hline
U+1F417 & {\EmojiFont 🐗} & {\textbackslash}:boar: & Boar \\ \hline
U+1F418 & {\EmojiFont 🐘} & {\textbackslash}:elephant: & Elephant \\ \hline
U+1F419 & {\EmojiFont 🐙} & {\textbackslash}:octopus: & Octopus \\ \hline
U+1F41A & {\EmojiFont 🐚} & {\textbackslash}:shell: & Spiral Shell \\ \hline
U+1F41B & {\EmojiFont 🐛} & {\textbackslash}:bug: & Bug \\ \hline
U+1F41C & {\EmojiFont 🐜} & {\textbackslash}:ant: & Ant \\ \hline
U+1F41D & {\EmojiFont 🐝} & {\textbackslash}:bee: & Honeybee \\ \hline
U+1F41E & {\EmojiFont 🐞} & {\textbackslash}:beetle: & Lady Beetle \\ \hline
U+1F41F & {\EmojiFont 🐟} & {\textbackslash}:fish: & Fish \\ \hline
U+1F420 & {\EmojiFont 🐠} & {\textbackslash}:tropical\_fish: & Tropical Fish \\ \hline
U+1F421 & {\EmojiFont 🐡} & {\textbackslash}:blowfish: & Blowfish \\ \hline
U+1F422 & {\EmojiFont 🐢} & {\textbackslash}:turtle: & Turtle \\ \hline
U+1F423 & {\EmojiFont 🐣} & {\textbackslash}:hatching\_chick: & Hatching Chick \\ \hline
U+1F424 & {\EmojiFont 🐤} & {\textbackslash}:baby\_chick: & Baby Chick \\ \hline
U+1F425 & {\EmojiFont 🐥} & {\textbackslash}:hatched\_chick: & Front-Facing Baby Chick \\ \hline
U+1F426 & {\EmojiFont 🐦} & {\textbackslash}:bird: & Bird \\ \hline
U+1F427 & {\EmojiFont 🐧} & {\textbackslash}:penguin: & Penguin \\ \hline
U+1F428 & {\EmojiFont 🐨} & {\textbackslash}:koala: & Koala \\ \hline
U+1F429 & {\EmojiFont 🐩} & {\textbackslash}:poodle: & Poodle \\ \hline
U+1F42A & {\EmojiFont 🐪} & {\textbackslash}:dromedary\_camel: & Dromedary Camel \\ \hline
U+1F42B & {\EmojiFont 🐫} & {\textbackslash}:camel: & Bactrian Camel \\ \hline
U+1F42C & {\EmojiFont 🐬} & {\textbackslash}:dolphin: & Dolphin \\ \hline
U+1F42D & {\EmojiFont 🐭} & {\textbackslash}:mouse: & Mouse Face \\ \hline
U+1F42E & {\EmojiFont 🐮} & {\textbackslash}:cow: & Cow Face \\ \hline
U+1F42F & {\EmojiFont 🐯} & {\textbackslash}:tiger: & Tiger Face \\ \hline
U+1F430 & {\EmojiFont 🐰} & {\textbackslash}:rabbit: & Rabbit Face \\ \hline
U+1F431 & {\EmojiFont 🐱} & {\textbackslash}:cat: & Cat Face \\ \hline
U+1F432 & {\EmojiFont 🐲} & {\textbackslash}:dragon\_face: & Dragon Face \\ \hline
U+1F433 & {\EmojiFont 🐳} & {\textbackslash}:whale: & Spouting Whale \\ \hline
U+1F434 & {\EmojiFont 🐴} & {\textbackslash}:horse: & Horse Face \\ \hline
U+1F435 & {\EmojiFont 🐵} & {\textbackslash}:monkey\_face: & Monkey Face \\ \hline
U+1F436 & {\EmojiFont 🐶} & {\textbackslash}:dog: & Dog Face \\ \hline
U+1F437 & {\EmojiFont 🐷} & {\textbackslash}:pig: & Pig Face \\ \hline
U+1F438 & {\EmojiFont 🐸} & {\textbackslash}:frog: & Frog Face \\ \hline
U+1F439 & {\EmojiFont 🐹} & {\textbackslash}:hamster: & Hamster Face \\ \hline
U+1F43A & {\EmojiFont 🐺} & {\textbackslash}:wolf: & Wolf Face \\ \hline
U+1F43B & {\EmojiFont 🐻} & {\textbackslash}:bear: & Bear Face \\ \hline
U+1F43C & {\EmojiFont 🐼} & {\textbackslash}:panda\_face: & Panda Face \\ \hline
U+1F43D & {\EmojiFont 🐽} & {\textbackslash}:pig\_nose: & Pig Nose \\ \hline
U+1F43E & {\EmojiFont 🐾} & {\textbackslash}:feet: & Paw Prints \\ \hline
U+1F440 & {\EmojiFont 👀} & {\textbackslash}:eyes: & Eyes \\ \hline
U+1F442 & {\EmojiFont 👂} & {\textbackslash}:ear: & Ear \\ \hline
U+1F443 & {\EmojiFont 👃} & {\textbackslash}:nose: & Nose \\ \hline
U+1F444 & {\EmojiFont 👄} & {\textbackslash}:lips: & Mouth \\ \hline
U+1F445 & {\EmojiFont 👅} & {\textbackslash}:tongue: & Tongue \\ \hline
U+1F446 & {\EmojiFont 👆} & {\textbackslash}:point\_up\_2: & White Up Pointing Backhand Index \\ \hline
U+1F447 & {\EmojiFont 👇} & {\textbackslash}:point\_down: & White Down Pointing Backhand Index \\ \hline
U+1F448 & {\EmojiFont 👈} & {\textbackslash}:point\_left: & White Left Pointing Backhand Index \\ \hline
U+1F449 & {\EmojiFont 👉} & {\textbackslash}:point\_right: & White Right Pointing Backhand Index \\ \hline
U+1F44A & {\EmojiFont 👊} & {\textbackslash}:facepunch: & Fisted Hand Sign \\ \hline
U+1F44B & {\EmojiFont 👋} & {\textbackslash}:wave: & Waving Hand Sign \\ \hline
U+1F44C & {\EmojiFont 👌} & {\textbackslash}:ok\_hand: & Ok Hand Sign \\ \hline
U+1F44D & {\EmojiFont 👍} & {\textbackslash}:+1: & Thumbs Up Sign \\ \hline
U+1F44E & {\EmojiFont 👎} & {\textbackslash}:-1: & Thumbs Down Sign \\ \hline
U+1F44F & {\EmojiFont 👏} & {\textbackslash}:clap: & Clapping Hands Sign \\ \hline
U+1F450 & {\EmojiFont 👐} & {\textbackslash}:open\_hands: & Open Hands Sign \\ \hline
U+1F451 & {\EmojiFont 👑} & {\textbackslash}:crown: & Crown \\ \hline
U+1F452 & {\EmojiFont 👒} & {\textbackslash}:womans\_hat: & Womans Hat \\ \hline
U+1F453 & {\EmojiFont 👓} & {\textbackslash}:eyeglasses: & Eyeglasses \\ \hline
U+1F454 & {\EmojiFont 👔} & {\textbackslash}:necktie: & Necktie \\ \hline
U+1F455 & {\EmojiFont 👕} & {\textbackslash}:shirt: & T-Shirt \\ \hline
U+1F456 & {\EmojiFont 👖} & {\textbackslash}:jeans: & Jeans \\ \hline
U+1F457 & {\EmojiFont 👗} & {\textbackslash}:dress: & Dress \\ \hline
U+1F458 & {\EmojiFont 👘} & {\textbackslash}:kimono: & Kimono \\ \hline
U+1F459 & {\EmojiFont 👙} & {\textbackslash}:bikini: & Bikini \\ \hline
U+1F45A & {\EmojiFont 👚} & {\textbackslash}:womans\_clothes: & Womans Clothes \\ \hline
U+1F45B & {\EmojiFont 👛} & {\textbackslash}:purse: & Purse \\ \hline
U+1F45C & {\EmojiFont 👜} & {\textbackslash}:handbag: & Handbag \\ \hline
U+1F45D & {\EmojiFont 👝} & {\textbackslash}:pouch: & Pouch \\ \hline
U+1F45E & {\EmojiFont 👞} & {\textbackslash}:mans\_shoe: & Mans Shoe \\ \hline
U+1F45F & {\EmojiFont 👟} & {\textbackslash}:athletic\_shoe: & Athletic Shoe \\ \hline
U+1F460 & {\EmojiFont 👠} & {\textbackslash}:high\_heel: & High-Heeled Shoe \\ \hline
U+1F461 & {\EmojiFont 👡} & {\textbackslash}:sandal: & Womans Sandal \\ \hline
U+1F462 & {\EmojiFont 👢} & {\textbackslash}:boot: & Womans Boots \\ \hline
U+1F463 & {\EmojiFont 👣} & {\textbackslash}:footprints: & Footprints \\ \hline
U+1F464 & {\EmojiFont 👤} & {\textbackslash}:bust\_in\_silhouette: & Bust In Silhouette \\ \hline
U+1F465 & {\EmojiFont 👥} & {\textbackslash}:busts\_in\_silhouette: & Busts In Silhouette \\ \hline
U+1F466 & {\EmojiFont 👦} & {\textbackslash}:boy: & Boy \\ \hline
U+1F467 & {\EmojiFont 👧} & {\textbackslash}:girl: & Girl \\ \hline
U+1F468 & {\EmojiFont 👨} & {\textbackslash}:man: & Man \\ \hline
U+1F469 & {\EmojiFont 👩} & {\textbackslash}:woman: & Woman \\ \hline
U+1F46A & {\EmojiFont 👪} & {\textbackslash}:family: & Family \\ \hline
U+1F46B & {\EmojiFont 👫} & {\textbackslash}:couple: & Man And Woman Holding Hands \\ \hline
U+1F46C & {\EmojiFont 👬} & {\textbackslash}:two\_men\_holding\_hands: & Two Men Holding Hands \\ \hline
U+1F46D & {\EmojiFont 👭} & {\textbackslash}:two\_women\_holding\_hands: & Two Women Holding Hands \\ \hline
U+1F46E & {\EmojiFont 👮} & {\textbackslash}:cop: & Police Officer \\ \hline
U+1F46F & {\EmojiFont 👯} & {\textbackslash}:dancers: & Woman With Bunny Ears \\ \hline
U+1F470 & {\EmojiFont 👰} & {\textbackslash}:bride\_with\_veil: & Bride With Veil \\ \hline
U+1F471 & {\EmojiFont 👱} & {\textbackslash}:person\_with\_blond\_hair: & Person With Blond Hair \\ \hline
U+1F472 & {\EmojiFont 👲} & {\textbackslash}:man\_with\_gua\_pi\_mao: & Man With Gua Pi Mao \\ \hline
U+1F473 & {\EmojiFont 👳} & {\textbackslash}:man\_with\_turban: & Man With Turban \\ \hline
U+1F474 & {\EmojiFont 👴} & {\textbackslash}:older\_man: & Older Man \\ \hline
U+1F475 & {\EmojiFont 👵} & {\textbackslash}:older\_woman: & Older Woman \\ \hline
U+1F476 & {\EmojiFont 👶} & {\textbackslash}:baby: & Baby \\ \hline
U+1F477 & {\EmojiFont 👷} & {\textbackslash}:construction\_worker: & Construction Worker \\ \hline
U+1F478 & {\EmojiFont 👸} & {\textbackslash}:princess: & Princess \\ \hline
U+1F479 & {\EmojiFont 👹} & {\textbackslash}:japanese\_ogre: & Japanese Ogre \\ \hline
U+1F47A & {\EmojiFont 👺} & {\textbackslash}:japanese\_goblin: & Japanese Goblin \\ \hline
U+1F47B & {\EmojiFont 👻} & {\textbackslash}:ghost: & Ghost \\ \hline
U+1F47C & {\EmojiFont 👼} & {\textbackslash}:angel: & Baby Angel \\ \hline
U+1F47D & {\EmojiFont 👽} & {\textbackslash}:alien: & Extraterrestrial Alien \\ \hline
U+1F47E & {\EmojiFont 👾} & {\textbackslash}:space\_invader: & Alien Monster \\ \hline
U+1F47F & {\EmojiFont 👿} & {\textbackslash}:imp: & Imp \\ \hline
U+1F480 & {\EmojiFont 💀} & {\textbackslash}:skull: & Skull \\ \hline
U+1F481 & {\EmojiFont 💁} & {\textbackslash}:information\_desk\_person: & Information Desk Person \\ \hline
U+1F482 & {\EmojiFont 💂} & {\textbackslash}:guardsman: & Guardsman \\ \hline
U+1F483 & {\EmojiFont 💃} & {\textbackslash}:dancer: & Dancer \\ \hline
U+1F484 & {\EmojiFont 💄} & {\textbackslash}:lipstick: & Lipstick \\ \hline
U+1F485 & {\EmojiFont 💅} & {\textbackslash}:nail\_care: & Nail Polish \\ \hline
U+1F486 & {\EmojiFont 💆} & {\textbackslash}:massage: & Face Massage \\ \hline
U+1F487 & {\EmojiFont 💇} & {\textbackslash}:haircut: & Haircut \\ \hline
U+1F488 & {\EmojiFont 💈} & {\textbackslash}:barber: & Barber Pole \\ \hline
U+1F489 & {\EmojiFont 💉} & {\textbackslash}:syringe: & Syringe \\ \hline
U+1F48A & {\EmojiFont 💊} & {\textbackslash}:pill: & Pill \\ \hline
U+1F48B & {\EmojiFont 💋} & {\textbackslash}:kiss: & Kiss Mark \\ \hline
U+1F48C & {\EmojiFont 💌} & {\textbackslash}:love\_letter: & Love Letter \\ \hline
U+1F48D & {\EmojiFont 💍} & {\textbackslash}:ring: & Ring \\ \hline
U+1F48E & {\EmojiFont 💎} & {\textbackslash}:gem: & Gem Stone \\ \hline
U+1F48F & {\EmojiFont 💏} & {\textbackslash}:couplekiss: & Kiss \\ \hline
U+1F490 & {\EmojiFont 💐} & {\textbackslash}:bouquet: & Bouquet \\ \hline
U+1F491 & {\EmojiFont 💑} & {\textbackslash}:couple\_with\_heart: & Couple With Heart \\ \hline
U+1F492 & {\EmojiFont 💒} & {\textbackslash}:wedding: & Wedding \\ \hline
U+1F493 & {\EmojiFont 💓} & {\textbackslash}:heartbeat: & Beating Heart \\ \hline
U+1F494 & {\EmojiFont 💔} & {\textbackslash}:broken\_heart: & Broken Heart \\ \hline
U+1F495 & {\EmojiFont 💕} & {\textbackslash}:two\_hearts: & Two Hearts \\ \hline
U+1F496 & {\EmojiFont 💖} & {\textbackslash}:sparkling\_heart: & Sparkling Heart \\ \hline
U+1F497 & {\EmojiFont 💗} & {\textbackslash}:heartpulse: & Growing Heart \\ \hline
U+1F498 & {\EmojiFont 💘} & {\textbackslash}:cupid: & Heart With Arrow \\ \hline
U+1F499 & {\EmojiFont 💙} & {\textbackslash}:blue\_heart: & Blue Heart \\ \hline
U+1F49A & {\EmojiFont 💚} & {\textbackslash}:green\_heart: & Green Heart \\ \hline
U+1F49B & {\EmojiFont 💛} & {\textbackslash}:yellow\_heart: & Yellow Heart \\ \hline
U+1F49C & {\EmojiFont 💜} & {\textbackslash}:purple\_heart: & Purple Heart \\ \hline
U+1F49D & {\EmojiFont 💝} & {\textbackslash}:gift\_heart: & Heart With Ribbon \\ \hline
U+1F49E & {\EmojiFont 💞} & {\textbackslash}:revolving\_hearts: & Revolving Hearts \\ \hline
U+1F49F & {\EmojiFont 💟} & {\textbackslash}:heart\_decoration: & Heart Decoration \\ \hline
U+1F4A0 & {\EmojiFont 💠} & {\textbackslash}:diamond\_shape\_with\_a\_dot\_inside: & Diamond Shape With A Dot Inside \\ \hline
U+1F4A1 & {\EmojiFont 💡} & {\textbackslash}:bulb: & Electric Light Bulb \\ \hline
U+1F4A2 & {\EmojiFont 💢} & {\textbackslash}:anger: & Anger Symbol \\ \hline
U+1F4A3 & {\EmojiFont 💣} & {\textbackslash}:bomb: & Bomb \\ \hline
U+1F4A4 & {\EmojiFont 💤} & {\textbackslash}:zzz: & Sleeping Symbol \\ \hline
U+1F4A5 & {\EmojiFont 💥} & {\textbackslash}:boom: & Collision Symbol \\ \hline
U+1F4A6 & {\EmojiFont 💦} & {\textbackslash}:sweat\_drops: & Splashing Sweat Symbol \\ \hline
U+1F4A7 & {\EmojiFont 💧} & {\textbackslash}:droplet: & Droplet \\ \hline
U+1F4A8 & {\EmojiFont 💨} & {\textbackslash}:dash: & Dash Symbol \\ \hline
U+1F4A9 & {\EmojiFont 💩} & {\textbackslash}:hankey: & Pile Of Poo \\ \hline
U+1F4AA & {\EmojiFont 💪} & {\textbackslash}:muscle: & Flexed Biceps \\ \hline
U+1F4AB & {\EmojiFont 💫} & {\textbackslash}:dizzy: & Dizzy Symbol \\ \hline
U+1F4AC & {\EmojiFont 💬} & {\textbackslash}:speech\_balloon: & Speech Balloon \\ \hline
U+1F4AD & {\EmojiFont 💭} & {\textbackslash}:thought\_balloon: & Thought Balloon \\ \hline
U+1F4AE & {\EmojiFont 💮} & {\textbackslash}:white\_flower: & White Flower \\ \hline
U+1F4AF & {\EmojiFont 💯} & {\textbackslash}:100: & Hundred Points Symbol \\ \hline
U+1F4B0 & {\EmojiFont 💰} & {\textbackslash}:moneybag: & Money Bag \\ \hline
U+1F4B1 & {\EmojiFont 💱} & {\textbackslash}:currency\_exchange: & Currency Exchange \\ \hline
U+1F4B2 & {\EmojiFont 💲} & {\textbackslash}:heavy\_dollar\_sign: & Heavy Dollar Sign \\ \hline
U+1F4B3 & {\EmojiFont 💳} & {\textbackslash}:credit\_card: & Credit Card \\ \hline
U+1F4B4 & {\EmojiFont 💴} & {\textbackslash}:yen: & Banknote With Yen Sign \\ \hline
U+1F4B5 & {\EmojiFont 💵} & {\textbackslash}:dollar: & Banknote With Dollar Sign \\ \hline
U+1F4B6 & {\EmojiFont 💶} & {\textbackslash}:euro: & Banknote With Euro Sign \\ \hline
U+1F4B7 & {\EmojiFont 💷} & {\textbackslash}:pound: & Banknote With Pound Sign \\ \hline
U+1F4B8 & {\EmojiFont 💸} & {\textbackslash}:money\_with\_wings: & Money With Wings \\ \hline
U+1F4B9 & {\EmojiFont 💹} & {\textbackslash}:chart: & Chart With Upwards Trend And Yen Sign \\ \hline
U+1F4BA & {\EmojiFont 💺} & {\textbackslash}:seat: & Seat \\ \hline
U+1F4BB & {\EmojiFont 💻} & {\textbackslash}:computer: & Personal Computer \\ \hline
U+1F4BC & {\EmojiFont 💼} & {\textbackslash}:briefcase: & Briefcase \\ \hline
U+1F4BD & {\EmojiFont 💽} & {\textbackslash}:minidisc: & Minidisc \\ \hline
U+1F4BE & {\EmojiFont 💾} & {\textbackslash}:floppy\_disk: & Floppy Disk \\ \hline
U+1F4BF & {\EmojiFont 💿} & {\textbackslash}:cd: & Optical Disc \\ \hline
U+1F4C0 & {\EmojiFont 📀} & {\textbackslash}:dvd: & Dvd \\ \hline
U+1F4C1 & {\EmojiFont 📁} & {\textbackslash}:file\_folder: & File Folder \\ \hline
U+1F4C2 & {\EmojiFont 📂} & {\textbackslash}:open\_file\_folder: & Open File Folder \\ \hline
U+1F4C3 & {\EmojiFont 📃} & {\textbackslash}:page\_with\_curl: & Page With Curl \\ \hline
U+1F4C4 & {\EmojiFont 📄} & {\textbackslash}:page\_facing\_up: & Page Facing Up \\ \hline
U+1F4C5 & {\EmojiFont 📅} & {\textbackslash}:date: & Calendar \\ \hline
U+1F4C6 & {\EmojiFont 📆} & {\textbackslash}:calendar: & Tear-Off Calendar \\ \hline
U+1F4C7 & {\EmojiFont 📇} & {\textbackslash}:card\_index: & Card Index \\ \hline
U+1F4C8 & {\EmojiFont 📈} & {\textbackslash}:chart\_with\_upwards\_trend: & Chart With Upwards Trend \\ \hline
U+1F4C9 & {\EmojiFont 📉} & {\textbackslash}:chart\_with\_downwards\_trend: & Chart With Downwards Trend \\ \hline
U+1F4CA & {\EmojiFont 📊} & {\textbackslash}:bar\_chart: & Bar Chart \\ \hline
U+1F4CB & {\EmojiFont 📋} & {\textbackslash}:clipboard: & Clipboard \\ \hline
U+1F4CC & {\EmojiFont 📌} & {\textbackslash}:pushpin: & Pushpin \\ \hline
U+1F4CD & {\EmojiFont 📍} & {\textbackslash}:round\_pushpin: & Round Pushpin \\ \hline
U+1F4CE & {\EmojiFont 📎} & {\textbackslash}:paperclip: & Paperclip \\ \hline
U+1F4CF & {\EmojiFont 📏} & {\textbackslash}:straight\_ruler: & Straight Ruler \\ \hline
U+1F4D0 & {\EmojiFont 📐} & {\textbackslash}:triangular\_ruler: & Triangular Ruler \\ \hline
U+1F4D1 & {\EmojiFont 📑} & {\textbackslash}:bookmark\_tabs: & Bookmark Tabs \\ \hline
U+1F4D2 & {\EmojiFont 📒} & {\textbackslash}:ledger: & Ledger \\ \hline
U+1F4D3 & {\EmojiFont 📓} & {\textbackslash}:notebook: & Notebook \\ \hline
U+1F4D4 & {\EmojiFont 📔} & {\textbackslash}:notebook\_with\_decorative\_cover: & Notebook With Decorative Cover \\ \hline
U+1F4D5 & {\EmojiFont 📕} & {\textbackslash}:closed\_book: & Closed Book \\ \hline
U+1F4D6 & {\EmojiFont 📖} & {\textbackslash}:book: & Open Book \\ \hline
U+1F4D7 & {\EmojiFont 📗} & {\textbackslash}:green\_book: & Green Book \\ \hline
U+1F4D8 & {\EmojiFont 📘} & {\textbackslash}:blue\_book: & Blue Book \\ \hline
U+1F4D9 & {\EmojiFont 📙} & {\textbackslash}:orange\_book: & Orange Book \\ \hline
U+1F4DA & {\EmojiFont 📚} & {\textbackslash}:books: & Books \\ \hline
U+1F4DB & {\EmojiFont 📛} & {\textbackslash}:name\_badge: & Name Badge \\ \hline
U+1F4DC & {\EmojiFont 📜} & {\textbackslash}:scroll: & Scroll \\ \hline
U+1F4DD & {\EmojiFont 📝} & {\textbackslash}:memo: & Memo \\ \hline
U+1F4DE & {\EmojiFont 📞} & {\textbackslash}:telephone\_receiver: & Telephone Receiver \\ \hline
U+1F4DF & {\EmojiFont 📟} & {\textbackslash}:pager: & Pager \\ \hline
U+1F4E0 & {\EmojiFont 📠} & {\textbackslash}:fax: & Fax Machine \\ \hline
U+1F4E1 & {\EmojiFont 📡} & {\textbackslash}:satellite: & Satellite Antenna \\ \hline
U+1F4E2 & {\EmojiFont 📢} & {\textbackslash}:loudspeaker: & Public Address Loudspeaker \\ \hline
U+1F4E3 & {\EmojiFont 📣} & {\textbackslash}:mega: & Cheering Megaphone \\ \hline
U+1F4E4 & {\EmojiFont 📤} & {\textbackslash}:outbox\_tray: & Outbox Tray \\ \hline
U+1F4E5 & {\EmojiFont 📥} & {\textbackslash}:inbox\_tray: & Inbox Tray \\ \hline
U+1F4E6 & {\EmojiFont 📦} & {\textbackslash}:package: & Package \\ \hline
U+1F4E7 & {\EmojiFont 📧} & {\textbackslash}:e-mail: & E-Mail Symbol \\ \hline
U+1F4E8 & {\EmojiFont 📨} & {\textbackslash}:incoming\_envelope: & Incoming Envelope \\ \hline
U+1F4E9 & {\EmojiFont 📩} & {\textbackslash}:envelope\_with\_arrow: & Envelope With Downwards Arrow Above \\ \hline
U+1F4EA & {\EmojiFont 📪} & {\textbackslash}:mailbox\_closed: & Closed Mailbox With Lowered Flag \\ \hline
U+1F4EB & {\EmojiFont 📫} & {\textbackslash}:mailbox: & Closed Mailbox With Raised Flag \\ \hline
U+1F4EC & {\EmojiFont 📬} & {\textbackslash}:mailbox\_with\_mail: & Open Mailbox With Raised Flag \\ \hline
U+1F4ED & {\EmojiFont 📭} & {\textbackslash}:mailbox\_with\_no\_mail: & Open Mailbox With Lowered Flag \\ \hline
U+1F4EE & {\EmojiFont 📮} & {\textbackslash}:postbox: & Postbox \\ \hline
U+1F4EF & {\EmojiFont 📯} & {\textbackslash}:postal\_horn: & Postal Horn \\ \hline
U+1F4F0 & {\EmojiFont 📰} & {\textbackslash}:newspaper: & Newspaper \\ \hline
U+1F4F1 & {\EmojiFont 📱} & {\textbackslash}:iphone: & Mobile Phone \\ \hline
U+1F4F2 & {\EmojiFont 📲} & {\textbackslash}:calling: & Mobile Phone With Rightwards Arrow At Left \\ \hline
U+1F4F3 & {\EmojiFont 📳} & {\textbackslash}:vibration\_mode: & Vibration Mode \\ \hline
U+1F4F4 & {\EmojiFont 📴} & {\textbackslash}:mobile\_phone\_off: & Mobile Phone Off \\ \hline
U+1F4F5 & {\EmojiFont 📵} & {\textbackslash}:no\_mobile\_phones: & No Mobile Phones \\ \hline
U+1F4F6 & {\EmojiFont 📶} & {\textbackslash}:signal\_strength: & Antenna With Bars \\ \hline
U+1F4F7 & {\EmojiFont 📷} & {\textbackslash}:camera: & Camera \\ \hline
U+1F4F9 & {\EmojiFont 📹} & {\textbackslash}:video\_camera: & Video Camera \\ \hline
U+1F4FA & {\EmojiFont 📺} & {\textbackslash}:tv: & Television \\ \hline
U+1F4FB & {\EmojiFont 📻} & {\textbackslash}:radio: & Radio \\ \hline
U+1F4FC & {\EmojiFont 📼} & {\textbackslash}:vhs: & Videocassette \\ \hline
U+1F500 & {\EmojiFont 🔀} & {\textbackslash}:twisted\_rightwards\_arrows: & Twisted Rightwards Arrows \\ \hline
U+1F501 & {\EmojiFont 🔁} & {\textbackslash}:repeat: & Clockwise Rightwards And Leftwards Open Circle Arrows \\ \hline
U+1F502 & {\EmojiFont 🔂} & {\textbackslash}:repeat\_one: & Clockwise Rightwards And Leftwards Open Circle Arrows With Circled One Overlay \\ \hline
U+1F503 & {\EmojiFont 🔃} & {\textbackslash}:arrows\_clockwise: & Clockwise Downwards And Upwards Open Circle Arrows \\ \hline
U+1F504 & {\EmojiFont 🔄} & {\textbackslash}:arrows\_counterclockwise: & Anticlockwise Downwards And Upwards Open Circle Arrows \\ \hline
U+1F505 & {\EmojiFont 🔅} & {\textbackslash}:low\_brightness: & Low Brightness Symbol \\ \hline
U+1F506 & {\EmojiFont 🔆} & {\textbackslash}:high\_brightness: & High Brightness Symbol \\ \hline
U+1F507 & {\EmojiFont 🔇} & {\textbackslash}:mute: & Speaker With Cancellation Stroke \\ \hline
U+1F508 & {\EmojiFont 🔈} & {\textbackslash}:speaker: & Speaker \\ \hline
U+1F509 & {\EmojiFont 🔉} & {\textbackslash}:sound: & Speaker With One Sound Wave \\ \hline
U+1F50A & {\EmojiFont 🔊} & {\textbackslash}:loud\_sound: & Speaker With Three Sound Waves \\ \hline
U+1F50B & {\EmojiFont 🔋} & {\textbackslash}:battery: & Battery \\ \hline
U+1F50C & {\EmojiFont 🔌} & {\textbackslash}:electric\_plug: & Electric Plug \\ \hline
U+1F50D & {\EmojiFont 🔍} & {\textbackslash}:mag: & Left-Pointing Magnifying Glass \\ \hline
U+1F50E & {\EmojiFont 🔎} & {\textbackslash}:mag\_right: & Right-Pointing Magnifying Glass \\ \hline
U+1F50F & {\EmojiFont 🔏} & {\textbackslash}:lock\_with\_ink\_pen: & Lock With Ink Pen \\ \hline
U+1F510 & {\EmojiFont 🔐} & {\textbackslash}:closed\_lock\_with\_key: & Closed Lock With Key \\ \hline
U+1F511 & {\EmojiFont 🔑} & {\textbackslash}:key: & Key \\ \hline
U+1F512 & {\EmojiFont 🔒} & {\textbackslash}:lock: & Lock \\ \hline
U+1F513 & {\EmojiFont 🔓} & {\textbackslash}:unlock: & Open Lock \\ \hline
U+1F514 & {\EmojiFont 🔔} & {\textbackslash}:bell: & Bell \\ \hline
U+1F515 & {\EmojiFont 🔕} & {\textbackslash}:no\_bell: & Bell With Cancellation Stroke \\ \hline
U+1F516 & {\EmojiFont 🔖} & {\textbackslash}:bookmark: & Bookmark \\ \hline
U+1F517 & {\EmojiFont 🔗} & {\textbackslash}:link: & Link Symbol \\ \hline
U+1F518 & {\EmojiFont 🔘} & {\textbackslash}:radio\_button: & Radio Button \\ \hline
U+1F519 & {\EmojiFont 🔙} & {\textbackslash}:back: & Back With Leftwards Arrow Above \\ \hline
U+1F51A & {\EmojiFont 🔚} & {\textbackslash}:end: & End With Leftwards Arrow Above \\ \hline
U+1F51B & {\EmojiFont 🔛} & {\textbackslash}:on: & On With Exclamation Mark With Left Right Arrow Above \\ \hline
U+1F51C & {\EmojiFont 🔜} & {\textbackslash}:soon: & Soon With Rightwards Arrow Above \\ \hline
U+1F51D & {\EmojiFont 🔝} & {\textbackslash}:top: & Top With Upwards Arrow Above \\ \hline
U+1F51E & {\EmojiFont 🔞} & {\textbackslash}:underage: & No One Under Eighteen Symbol \\ \hline
U+1F51F & {\EmojiFont 🔟} & {\textbackslash}:keycap\_ten: & Keycap Ten \\ \hline
U+1F520 & {\EmojiFont 🔠} & {\textbackslash}:capital\_abcd: & Input Symbol For Latin Capital Letters \\ \hline
U+1F521 & {\EmojiFont 🔡} & {\textbackslash}:abcd: & Input Symbol For Latin Small Letters \\ \hline
U+1F522 & {\EmojiFont 🔢} & {\textbackslash}:1234: & Input Symbol For Numbers \\ \hline
U+1F523 & {\EmojiFont 🔣} & {\textbackslash}:symbols: & Input Symbol For Symbols \\ \hline
U+1F524 & {\EmojiFont 🔤} & {\textbackslash}:abc: & Input Symbol For Latin Letters \\ \hline
U+1F525 & {\EmojiFont 🔥} & {\textbackslash}:fire: & Fire \\ \hline
U+1F526 & {\EmojiFont 🔦} & {\textbackslash}:flashlight: & Electric Torch \\ \hline
U+1F527 & {\EmojiFont 🔧} & {\textbackslash}:wrench: & Wrench \\ \hline
U+1F528 & {\EmojiFont 🔨} & {\textbackslash}:hammer: & Hammer \\ \hline
U+1F529 & {\EmojiFont 🔩} & {\textbackslash}:nut\_and\_bolt: & Nut And Bolt \\ \hline
U+1F52A & {\EmojiFont 🔪} & {\textbackslash}:hocho: & Hocho \\ \hline
U+1F52B & {\EmojiFont 🔫} & {\textbackslash}:gun: & Pistol \\ \hline
U+1F52C & {\EmojiFont 🔬} & {\textbackslash}:microscope: & Microscope \\ \hline
U+1F52D & {\EmojiFont 🔭} & {\textbackslash}:telescope: & Telescope \\ \hline
U+1F52E & {\EmojiFont 🔮} & {\textbackslash}:crystal\_ball: & Crystal Ball \\ \hline
U+1F52F & {\EmojiFont 🔯} & {\textbackslash}:six\_pointed\_star: & Six Pointed Star With Middle Dot \\ \hline
U+1F530 & {\EmojiFont 🔰} & {\textbackslash}:beginner: & Japanese Symbol For Beginner \\ \hline
U+1F531 & {\EmojiFont 🔱} & {\textbackslash}:trident: & Trident Emblem \\ \hline
U+1F532 & {\EmojiFont 🔲} & {\textbackslash}:black\_square\_button: & Black Square Button \\ \hline
U+1F533 & {\EmojiFont 🔳} & {\textbackslash}:white\_square\_button: & White Square Button \\ \hline
U+1F534 & {\EmojiFont 🔴} & {\textbackslash}:red\_circle: & Large Red Circle \\ \hline
U+1F535 & {\EmojiFont 🔵} & {\textbackslash}:large\_blue\_circle: & Large Blue Circle \\ \hline
U+1F536 & {\EmojiFont 🔶} & {\textbackslash}:large\_orange\_diamond: & Large Orange Diamond \\ \hline
U+1F537 & {\EmojiFont 🔷} & {\textbackslash}:large\_blue\_diamond: & Large Blue Diamond \\ \hline
U+1F538 & {\EmojiFont 🔸} & {\textbackslash}:small\_orange\_diamond: & Small Orange Diamond \\ \hline
U+1F539 & {\EmojiFont 🔹} & {\textbackslash}:small\_blue\_diamond: & Small Blue Diamond \\ \hline
U+1F53A & {\EmojiFont 🔺} & {\textbackslash}:small\_red\_triangle: & Up-Pointing Red Triangle \\ \hline
U+1F53B & {\EmojiFont 🔻} & {\textbackslash}:small\_red\_triangle\_down: & Down-Pointing Red Triangle \\ \hline
U+1F53C & {\EmojiFont 🔼} & {\textbackslash}:arrow\_up\_small: & Up-Pointing Small Red Triangle \\ \hline
U+1F53D & {\EmojiFont 🔽} & {\textbackslash}:arrow\_down\_small: & Down-Pointing Small Red Triangle \\ \hline
U+1F550 & {\EmojiFont 🕐} & {\textbackslash}:clock1: & Clock Face One Oclock \\ \hline
U+1F551 & {\EmojiFont 🕑} & {\textbackslash}:clock2: & Clock Face Two Oclock \\ \hline
U+1F552 & {\EmojiFont 🕒} & {\textbackslash}:clock3: & Clock Face Three Oclock \\ \hline
U+1F553 & {\EmojiFont 🕓} & {\textbackslash}:clock4: & Clock Face Four Oclock \\ \hline
U+1F554 & {\EmojiFont 🕔} & {\textbackslash}:clock5: & Clock Face Five Oclock \\ \hline
U+1F555 & {\EmojiFont 🕕} & {\textbackslash}:clock6: & Clock Face Six Oclock \\ \hline
U+1F556 & {\EmojiFont 🕖} & {\textbackslash}:clock7: & Clock Face Seven Oclock \\ \hline
U+1F557 & {\EmojiFont 🕗} & {\textbackslash}:clock8: & Clock Face Eight Oclock \\ \hline
U+1F558 & {\EmojiFont 🕘} & {\textbackslash}:clock9: & Clock Face Nine Oclock \\ \hline
U+1F559 & {\EmojiFont 🕙} & {\textbackslash}:clock10: & Clock Face Ten Oclock \\ \hline
U+1F55A & {\EmojiFont 🕚} & {\textbackslash}:clock11: & Clock Face Eleven Oclock \\ \hline
U+1F55B & {\EmojiFont 🕛} & {\textbackslash}:clock12: & Clock Face Twelve Oclock \\ \hline
U+1F55C & {\EmojiFont 🕜} & {\textbackslash}:clock130: & Clock Face One-Thirty \\ \hline
U+1F55D & {\EmojiFont 🕝} & {\textbackslash}:clock230: & Clock Face Two-Thirty \\ \hline
U+1F55E & {\EmojiFont 🕞} & {\textbackslash}:clock330: & Clock Face Three-Thirty \\ \hline
U+1F55F & {\EmojiFont 🕟} & {\textbackslash}:clock430: & Clock Face Four-Thirty \\ \hline
U+1F560 & {\EmojiFont 🕠} & {\textbackslash}:clock530: & Clock Face Five-Thirty \\ \hline
U+1F561 & {\EmojiFont 🕡} & {\textbackslash}:clock630: & Clock Face Six-Thirty \\ \hline
U+1F562 & {\EmojiFont 🕢} & {\textbackslash}:clock730: & Clock Face Seven-Thirty \\ \hline
U+1F563 & {\EmojiFont 🕣} & {\textbackslash}:clock830: & Clock Face Eight-Thirty \\ \hline
U+1F564 & {\EmojiFont 🕤} & {\textbackslash}:clock930: & Clock Face Nine-Thirty \\ \hline
U+1F565 & {\EmojiFont 🕥} & {\textbackslash}:clock1030: & Clock Face Ten-Thirty \\ \hline
U+1F566 & {\EmojiFont 🕦} & {\textbackslash}:clock1130: & Clock Face Eleven-Thirty \\ \hline
U+1F567 & {\EmojiFont 🕧} & {\textbackslash}:clock1230: & Clock Face Twelve-Thirty \\ \hline
U+1F5FB & {\EmojiFont 🗻} & {\textbackslash}:mount\_fuji: & Mount Fuji \\ \hline
U+1F5FC & {\EmojiFont 🗼} & {\textbackslash}:tokyo\_tower: & Tokyo Tower \\ \hline
U+1F5FD & {\EmojiFont 🗽} & {\textbackslash}:statue\_of\_liberty: & Statue Of Liberty \\ \hline
U+1F5FE & {\EmojiFont 🗾} & {\textbackslash}:japan: & Silhouette Of Japan \\ \hline
U+1F5FF & {\EmojiFont 🗿} & {\textbackslash}:moyai: & Moyai \\ \hline
U+1F600 & {\EmojiFont 😀} & {\textbackslash}:grinning: & Grinning Face \\ \hline
U+1F601 & {\EmojiFont 😁} & {\textbackslash}:grin: & Grinning Face With Smiling Eyes \\ \hline
U+1F602 & {\EmojiFont 😂} & {\textbackslash}:joy: & Face With Tears Of Joy \\ \hline
U+1F603 & {\EmojiFont 😃} & {\textbackslash}:smiley: & Smiling Face With Open Mouth \\ \hline
U+1F604 & {\EmojiFont 😄} & {\textbackslash}:smile: & Smiling Face With Open Mouth And Smiling Eyes \\ \hline
U+1F605 & {\EmojiFont 😅} & {\textbackslash}:sweat\_smile: & Smiling Face With Open Mouth And Cold Sweat \\ \hline
U+1F606 & {\EmojiFont 😆} & {\textbackslash}:laughing: & Smiling Face With Open Mouth And Tightly-Closed Eyes \\ \hline
U+1F607 & {\EmojiFont 😇} & {\textbackslash}:innocent: & Smiling Face With Halo \\ \hline
U+1F608 & {\EmojiFont 😈} & {\textbackslash}:smiling\_imp: & Smiling Face With Horns \\ \hline
U+1F609 & {\EmojiFont 😉} & {\textbackslash}:wink: & Winking Face \\ \hline
U+1F60A & {\EmojiFont 😊} & {\textbackslash}:blush: & Smiling Face With Smiling Eyes \\ \hline
U+1F60B & {\EmojiFont 😋} & {\textbackslash}:yum: & Face Savouring Delicious Food \\ \hline
U+1F60C & {\EmojiFont 😌} & {\textbackslash}:relieved: & Relieved Face \\ \hline
U+1F60D & {\EmojiFont 😍} & {\textbackslash}:heart\_eyes: & Smiling Face With Heart-Shaped Eyes \\ \hline
U+1F60E & {\EmojiFont 😎} & {\textbackslash}:sunglasses: & Smiling Face With Sunglasses \\ \hline
U+1F60F & {\EmojiFont 😏} & {\textbackslash}:smirk: & Smirking Face \\ \hline
U+1F610 & {\EmojiFont 😐} & {\textbackslash}:neutral\_face: & Neutral Face \\ \hline
U+1F611 & {\EmojiFont 😑} & {\textbackslash}:expressionless: & Expressionless Face \\ \hline
U+1F612 & {\EmojiFont 😒} & {\textbackslash}:unamused: & Unamused Face \\ \hline
U+1F613 & {\EmojiFont 😓} & {\textbackslash}:sweat: & Face With Cold Sweat \\ \hline
U+1F614 & {\EmojiFont 😔} & {\textbackslash}:pensive: & Pensive Face \\ \hline
U+1F615 & {\EmojiFont 😕} & {\textbackslash}:confused: & Confused Face \\ \hline
U+1F616 & {\EmojiFont 😖} & {\textbackslash}:confounded: & Confounded Face \\ \hline
U+1F617 & {\EmojiFont 😗} & {\textbackslash}:kissing: & Kissing Face \\ \hline
U+1F618 & {\EmojiFont 😘} & {\textbackslash}:kissing\_heart: & Face Throwing A Kiss \\ \hline
U+1F619 & {\EmojiFont 😙} & {\textbackslash}:kissing\_smiling\_eyes: & Kissing Face With Smiling Eyes \\ \hline
U+1F61A & {\EmojiFont 😚} & {\textbackslash}:kissing\_closed\_eyes: & Kissing Face With Closed Eyes \\ \hline
U+1F61B & {\EmojiFont 😛} & {\textbackslash}:stuck\_out\_tongue: & Face With Stuck-Out Tongue \\ \hline
U+1F61C & {\EmojiFont 😜} & {\textbackslash}:stuck\_out\_tongue\_winking\_eye: & Face With Stuck-Out Tongue And Winking Eye \\ \hline
U+1F61D & {\EmojiFont 😝} & {\textbackslash}:stuck\_out\_tongue\_closed\_eyes: & Face With Stuck-Out Tongue And Tightly-Closed Eyes \\ \hline
U+1F61E & {\EmojiFont 😞} & {\textbackslash}:disappointed: & Disappointed Face \\ \hline
U+1F61F & {\EmojiFont 😟} & {\textbackslash}:worried: & Worried Face \\ \hline
U+1F620 & {\EmojiFont 😠} & {\textbackslash}:angry: & Angry Face \\ \hline
U+1F621 & {\EmojiFont 😡} & {\textbackslash}:rage: & Pouting Face \\ \hline
U+1F622 & {\EmojiFont 😢} & {\textbackslash}:cry: & Crying Face \\ \hline
U+1F623 & {\EmojiFont 😣} & {\textbackslash}:persevere: & Persevering Face \\ \hline
U+1F624 & {\EmojiFont 😤} & {\textbackslash}:triumph: & Face With Look Of Triumph \\ \hline
U+1F625 & {\EmojiFont 😥} & {\textbackslash}:disappointed\_relieved: & Disappointed But Relieved Face \\ \hline
U+1F626 & {\EmojiFont 😦} & {\textbackslash}:frowning: & Frowning Face With Open Mouth \\ \hline
U+1F627 & {\EmojiFont 😧} & {\textbackslash}:anguished: & Anguished Face \\ \hline
U+1F628 & {\EmojiFont 😨} & {\textbackslash}:fearful: & Fearful Face \\ \hline
U+1F629 & {\EmojiFont 😩} & {\textbackslash}:weary: & Weary Face \\ \hline
U+1F62A & {\EmojiFont 😪} & {\textbackslash}:sleepy: & Sleepy Face \\ \hline
U+1F62B & {\EmojiFont 😫} & {\textbackslash}:tired\_face: & Tired Face \\ \hline
U+1F62C & {\EmojiFont 😬} & {\textbackslash}:grimacing: & Grimacing Face \\ \hline
U+1F62D & {\EmojiFont 😭} & {\textbackslash}:sob: & Loudly Crying Face \\ \hline
U+1F62E & {\EmojiFont 😮} & {\textbackslash}:open\_mouth: & Face With Open Mouth \\ \hline
U+1F62F & {\EmojiFont 😯} & {\textbackslash}:hushed: & Hushed Face \\ \hline
U+1F630 & {\EmojiFont 😰} & {\textbackslash}:cold\_sweat: & Face With Open Mouth And Cold Sweat \\ \hline
U+1F631 & {\EmojiFont 😱} & {\textbackslash}:scream: & Face Screaming In Fear \\ \hline
U+1F632 & {\EmojiFont 😲} & {\textbackslash}:astonished: & Astonished Face \\ \hline
U+1F633 & {\EmojiFont 😳} & {\textbackslash}:flushed: & Flushed Face \\ \hline
U+1F634 & {\EmojiFont 😴} & {\textbackslash}:sleeping: & Sleeping Face \\ \hline
U+1F635 & {\EmojiFont 😵} & {\textbackslash}:dizzy\_face: & Dizzy Face \\ \hline
U+1F636 & {\EmojiFont 😶} & {\textbackslash}:no\_mouth: & Face Without Mouth \\ \hline
U+1F637 & {\EmojiFont 😷} & {\textbackslash}:mask: & Face With Medical Mask \\ \hline
U+1F638 & {\EmojiFont 😸} & {\textbackslash}:smile\_cat: & Grinning Cat Face With Smiling Eyes \\ \hline
U+1F639 & {\EmojiFont 😹} & {\textbackslash}:joy\_cat: & Cat Face With Tears Of Joy \\ \hline
U+1F63A & {\EmojiFont 😺} & {\textbackslash}:smiley\_cat: & Smiling Cat Face With Open Mouth \\ \hline
U+1F63B & {\EmojiFont 😻} & {\textbackslash}:heart\_eyes\_cat: & Smiling Cat Face With Heart-Shaped Eyes \\ \hline
U+1F63C & {\EmojiFont 😼} & {\textbackslash}:smirk\_cat: & Cat Face With Wry Smile \\ \hline
U+1F63D & {\EmojiFont 😽} & {\textbackslash}:kissing\_cat: & Kissing Cat Face With Closed Eyes \\ \hline
U+1F63E & {\EmojiFont 😾} & {\textbackslash}:pouting\_cat: & Pouting Cat Face \\ \hline
U+1F63F & {\EmojiFont 😿} & {\textbackslash}:crying\_cat\_face: & Crying Cat Face \\ \hline
U+1F640 & {\EmojiFont 🙀} & {\textbackslash}:scream\_cat: & Weary Cat Face \\ \hline
U+1F645 & {\EmojiFont 🙅} & {\textbackslash}:no\_good: & Face With No Good Gesture \\ \hline
U+1F646 & {\EmojiFont 🙆} & {\textbackslash}:ok\_woman: & Face With Ok Gesture \\ \hline
U+1F647 & {\EmojiFont 🙇} & {\textbackslash}:bow: & Person Bowing Deeply \\ \hline
U+1F648 & {\EmojiFont 🙈} & {\textbackslash}:see\_no\_evil: & See-No-Evil Monkey \\ \hline
U+1F649 & {\EmojiFont 🙉} & {\textbackslash}:hear\_no\_evil: & Hear-No-Evil Monkey \\ \hline
U+1F64A & {\EmojiFont 🙊} & {\textbackslash}:speak\_no\_evil: & Speak-No-Evil Monkey \\ \hline
U+1F64B & {\EmojiFont 🙋} & {\textbackslash}:raising\_hand: & Happy Person Raising One Hand \\ \hline
U+1F64C & {\EmojiFont 🙌} & {\textbackslash}:raised\_hands: & Person Raising Both Hands In Celebration \\ \hline
U+1F64D & {\EmojiFont 🙍} & {\textbackslash}:person\_frowning: & Person Frowning \\ \hline
U+1F64E & {\EmojiFont 🙎} & {\textbackslash}:person\_with\_pouting\_face: & Person With Pouting Face \\ \hline
U+1F64F & {\EmojiFont 🙏} & {\textbackslash}:pray: & Person With Folded Hands \\ \hline
U+1F680 & {\EmojiFont 🚀} & {\textbackslash}:rocket: & Rocket \\ \hline
U+1F681 & {\EmojiFont 🚁} & {\textbackslash}:helicopter: & Helicopter \\ \hline
U+1F682 & {\EmojiFont 🚂} & {\textbackslash}:steam\_locomotive: & Steam Locomotive \\ \hline
U+1F683 & {\EmojiFont 🚃} & {\textbackslash}:railway\_car: & Railway Car \\ \hline
U+1F684 & {\EmojiFont 🚄} & {\textbackslash}:bullettrain\_side: & High-Speed Train \\ \hline
U+1F685 & {\EmojiFont 🚅} & {\textbackslash}:bullettrain\_front: & High-Speed Train With Bullet Nose \\ \hline
U+1F686 & {\EmojiFont 🚆} & {\textbackslash}:train2: & Train \\ \hline
U+1F687 & {\EmojiFont 🚇} & {\textbackslash}:metro: & Metro \\ \hline
U+1F688 & {\EmojiFont 🚈} & {\textbackslash}:light\_rail: & Light Rail \\ \hline
U+1F689 & {\EmojiFont 🚉} & {\textbackslash}:station: & Station \\ \hline
U+1F68A & {\EmojiFont 🚊} & {\textbackslash}:tram: & Tram \\ \hline
U+1F68B & {\EmojiFont 🚋} & {\textbackslash}:train: & Tram Car \\ \hline
U+1F68C & {\EmojiFont 🚌} & {\textbackslash}:bus: & Bus \\ \hline
U+1F68D & {\EmojiFont 🚍} & {\textbackslash}:oncoming\_bus: & Oncoming Bus \\ \hline
U+1F68E & {\EmojiFont 🚎} & {\textbackslash}:trolleybus: & Trolleybus \\ \hline
U+1F68F & {\EmojiFont 🚏} & {\textbackslash}:busstop: & Bus Stop \\ \hline
U+1F690 & {\EmojiFont 🚐} & {\textbackslash}:minibus: & Minibus \\ \hline
U+1F691 & {\EmojiFont 🚑} & {\textbackslash}:ambulance: & Ambulance \\ \hline
U+1F692 & {\EmojiFont 🚒} & {\textbackslash}:fire\_engine: & Fire Engine \\ \hline
U+1F693 & {\EmojiFont 🚓} & {\textbackslash}:police\_car: & Police Car \\ \hline
U+1F694 & {\EmojiFont 🚔} & {\textbackslash}:oncoming\_police\_car: & Oncoming Police Car \\ \hline
U+1F695 & {\EmojiFont 🚕} & {\textbackslash}:taxi: & Taxi \\ \hline
U+1F696 & {\EmojiFont 🚖} & {\textbackslash}:oncoming\_taxi: & Oncoming Taxi \\ \hline
U+1F697 & {\EmojiFont 🚗} & {\textbackslash}:car: & Automobile \\ \hline
U+1F698 & {\EmojiFont 🚘} & {\textbackslash}:oncoming\_automobile: & Oncoming Automobile \\ \hline
U+1F699 & {\EmojiFont 🚙} & {\textbackslash}:blue\_car: & Recreational Vehicle \\ \hline
U+1F69A & {\EmojiFont 🚚} & {\textbackslash}:truck: & Delivery Truck \\ \hline
U+1F69B & {\EmojiFont 🚛} & {\textbackslash}:articulated\_lorry: & Articulated Lorry \\ \hline
U+1F69C & {\EmojiFont 🚜} & {\textbackslash}:tractor: & Tractor \\ \hline
U+1F69D & {\EmojiFont 🚝} & {\textbackslash}:monorail: & Monorail \\ \hline
U+1F69E & {\EmojiFont 🚞} & {\textbackslash}:mountain\_railway: & Mountain Railway \\ \hline
U+1F69F & {\EmojiFont 🚟} & {\textbackslash}:suspension\_railway: & Suspension Railway \\ \hline
U+1F6A0 & {\EmojiFont 🚠} & {\textbackslash}:mountain\_cableway: & Mountain Cableway \\ \hline
U+1F6A1 & {\EmojiFont 🚡} & {\textbackslash}:aerial\_tramway: & Aerial Tramway \\ \hline
U+1F6A2 & {\EmojiFont 🚢} & {\textbackslash}:ship: & Ship \\ \hline
U+1F6A3 & {\EmojiFont 🚣} & {\textbackslash}:rowboat: & Rowboat \\ \hline
U+1F6A4 & {\EmojiFont 🚤} & {\textbackslash}:speedboat: & Speedboat \\ \hline
U+1F6A5 & {\EmojiFont 🚥} & {\textbackslash}:traffic\_light: & Horizontal Traffic Light \\ \hline
U+1F6A6 & {\EmojiFont 🚦} & {\textbackslash}:vertical\_traffic\_light: & Vertical Traffic Light \\ \hline
U+1F6A7 & {\EmojiFont 🚧} & {\textbackslash}:construction: & Construction Sign \\ \hline
U+1F6A8 & {\EmojiFont 🚨} & {\textbackslash}:rotating\_light: & Police Cars Revolving Light \\ \hline
U+1F6A9 & {\EmojiFont 🚩} & {\textbackslash}:triangular\_flag\_on\_post: & Triangular Flag On Post \\ \hline
U+1F6AA & {\EmojiFont 🚪} & {\textbackslash}:door: & Door \\ \hline
U+1F6AB & {\EmojiFont 🚫} & {\textbackslash}:no\_entry\_sign: & No Entry Sign \\ \hline
U+1F6AC & {\EmojiFont 🚬} & {\textbackslash}:smoking: & Smoking Symbol \\ \hline
U+1F6AD & {\EmojiFont 🚭} & {\textbackslash}:no\_smoking: & No Smoking Symbol \\ \hline
U+1F6AE & {\EmojiFont 🚮} & {\textbackslash}:put\_litter\_in\_its\_place: & Put Litter In Its Place Symbol \\ \hline
U+1F6AF & {\EmojiFont 🚯} & {\textbackslash}:do\_not\_litter: & Do Not Litter Symbol \\ \hline
U+1F6B0 & {\EmojiFont 🚰} & {\textbackslash}:potable\_water: & Potable Water Symbol \\ \hline
U+1F6B1 & {\EmojiFont 🚱} & {\textbackslash}:non-potable\_water: & Non-Potable Water Symbol \\ \hline
U+1F6B2 & {\EmojiFont 🚲} & {\textbackslash}:bike: & Bicycle \\ \hline
U+1F6B3 & {\EmojiFont 🚳} & {\textbackslash}:no\_bicycles: & No Bicycles \\ \hline
U+1F6B4 & {\EmojiFont 🚴} & {\textbackslash}:bicyclist: & Bicyclist \\ \hline
U+1F6B5 & {\EmojiFont 🚵} & {\textbackslash}:mountain\_bicyclist: & Mountain Bicyclist \\ \hline
U+1F6B6 & {\EmojiFont 🚶} & {\textbackslash}:walking: & Pedestrian \\ \hline
U+1F6B7 & {\EmojiFont 🚷} & {\textbackslash}:no\_pedestrians: & No Pedestrians \\ \hline
U+1F6B8 & {\EmojiFont 🚸} & {\textbackslash}:children\_crossing: & Children Crossing \\ \hline
U+1F6B9 & {\EmojiFont 🚹} & {\textbackslash}:mens: & Mens Symbol \\ \hline
U+1F6BA & {\EmojiFont 🚺} & {\textbackslash}:womens: & Womens Symbol \\ \hline
U+1F6BB & {\EmojiFont 🚻} & {\textbackslash}:restroom: & Restroom \\ \hline
U+1F6BC & {\EmojiFont 🚼} & {\textbackslash}:baby\_symbol: & Baby Symbol \\ \hline
U+1F6BD & {\EmojiFont 🚽} & {\textbackslash}:toilet: & Toilet \\ \hline
U+1F6BE & {\EmojiFont 🚾} & {\textbackslash}:wc: & Water Closet \\ \hline
U+1F6BF & {\EmojiFont 🚿} & {\textbackslash}:shower: & Shower \\ \hline
U+1F6C0 & {\EmojiFont 🛀} & {\textbackslash}:bath: & Bath \\ \hline
U+1F6C1 & {\EmojiFont 🛁} & {\textbackslash}:bathtub: & Bathtub \\ \hline
U+1F6C2 & {\EmojiFont 🛂} & {\textbackslash}:passport\_control: & Passport Control \\ \hline
U+1F6C3 & {\EmojiFont 🛃} & {\textbackslash}:customs: & Customs \\ \hline
U+1F6C4 & {\EmojiFont 🛄} & {\textbackslash}:baggage\_claim: & Baggage Claim \\ \hline
U+1F6C5 & {\EmojiFont 🛅} & {\textbackslash}:left\_luggage: & Left Luggage \\ \hline

\end{longtable}

\end{document}