%!TEX program = lualatex
\documentclass[oneside]{ctexbook} %% 1. {ctexbook} + \newCJKfontfamily
% \documentclass[oneside]{book} %% 2. {book} + \newfontface

\usepackage[
  hmargin=0.1cm, % left and right margin
  vmargin=0.5cm, % top and bottom margin
]{geometry}

% 表格
\usepackage{longtable,booktabs}

% emoji
\usepackage{./emoji}
\setemojifont{Twemoji Mozilla}
\newCJKfontfamily\EmojiFont{Twemoji Mozilla}[Renderer=HarfBuzz] %% 1. {ctexbook} + \newCJKfontfamily
% \newfontface\EmojiFont{Twemoji Mozilla}[Renderer=HarfBuzz] %% 2. {book} + \newfontface


\begin{document}

% \verb!\newCJKfontfamily\EmojiFont! \\
% U+03030 | {\EmojiFont 〰}
% U+0303D | {\EmojiFont 〽}
% U+03297 | {\EmojiFont ㊗}
% U+03299 | {\EmojiFont ㊙}
% U+02194 | {\EmojiFont ↔} 
% U+02195 | {\EmojiFont ↕} \\

% \verb!\newCJKfontfamily\EmojiFont + \emoji{}! \\
% U+03030 | \emoji{wavy-dash}
% U+0303D | \emoji{part-alternation-mark}
% U+03297 | \emoji{congratulations}
% U+03299 | \emoji{secret}
% U+02194 | \emoji{left-right-arrow} 
% U+02195 | \emoji{left-right-arrow} \\

\verb!\newCJKfontfamily\EmojiFont! \\ %% 1. {ctexbook} + \newCJKfontfamily
% \verb! \newfontface\EmojiFont ! \\ %% 2. {book} + \newfontface


% 不管右侧溢出,确保一行一个 emoji
\begin{longtable}{ccll}
    \caption{Unicode emoji} \\
    %% all head
    \toprule
    Code point & Char & Tab seq & Unicode name \\
    \hline \endhead
    %% all footer
    \multicolumn{4}{l}{To be continued...} \\ 
    \midrule \endfoot
    %% last footer
    \bottomrule \endlastfoot
    
    \input{emoji-test}
\end{longtable}

\end{document}