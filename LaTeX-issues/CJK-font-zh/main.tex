\documentclass[
  UTF8, % UTF-8 编码读入
  a4paper,  % A4 纸大小
  oneside,  % 单页模式,左右页边距相同
]{ctexbook} % book 类:需要 `\part{}' 命令

%% font settings
\usepackage{fontspec}
% \setmainfont{DejaVu Sans}
\setsansfont{DejaVu Sans}
\setmonofont{DejaVu Sans Mono}

% CJK font
\usepackage{xeCJK}
% \setCJKmainfont{Noto Sans CJK SC}
% \setCJKfallbackfamilyfont{\CJKrmdefault}{ % \textrm 和 \rmfamily
%   {Noto Sans Mono CJK SC},
%   {Noto Sans Mono CJK KR},
% }
% \setCJKfallbackfamilyfont{\CJKsfdefault}{ % \textsf 和 \sffamily
%   {Noto Sans Mono CJK SC},
%   {Noto Sans Mono CJK KR},
% }
\setCJKmonofont{Noto Sans Mono CJK SC}
\setCJKfallbackfamilyfont{\CJKttdefault}{ % \texttt 和 \ttfamily
  {Noto Sans Mono CJK KR},
}
%%

%% listings
\usepackage{listings, minted}
%% 保存中文缩进长度
\newlength{\zhParIndent}
\setlength{\zhParIndent}{\parindent + 2pt} % 保持视觉上的对齐
\lstset{
    basicstyle = \small\ttfamily,
    breaklines = true,
    columns = fullflexible,
    frame = leftline,
    keepspaces = true,
    showstringspaces = false,
    % 对齐段首缩进;listings 宏包的所及计算比较复杂
    xleftmargin=\dimexpr\zhParIndent+\lst@frametextsep+\lst@framerulewidth\relax,
}

\setminted{
    breaklines = true,
    fontsize = \small,
    frame = leftline,
    xleftmargin = \zhParIndent, % 对齐段首缩进
}
%%

%% 
% \usepackage{polyglossia}
% \setdefaultlanguage{english}
% \setotherlanguage{korean}


\begin{document}


Julia 提供了非常灵活的变量命名策略。变量名是大小写敏感的,且不包含语义,意思是说,Julia 不会根据变量的名字来区别对待它们。 (译者注:Julia \textbf{不会}自动将全大写的变量识别为常量,也\textbf{不会}将有特定前后缀的变量自动识别为某种特定类型的变量,即不会根据变量名字,自动判断变量的任何属性。)

\begin{minted}{jlcon}
julia> x = 1.0
1.0

julia> y = -3
-3

julia> Z = "My string"
"My string"

julia> customary_phrase = "Hello world!"
"Hello world!"

julia> UniversalDeclarationOfHumanRightsStart = "人人生而自由,在尊严和权利上一律平等。"
"人人生而自由,在尊严和权利上一律平等。"
\end{minted}

你还可以使用 UTF-8 编码的 Unicode 字符作为变量名:

\begin{minted}{jlcon}
julia> δ = 0.00001
1.0e-5

julia> 안녕하세요 = "Hello"
"Hello"
\end{minted}

\end{document}