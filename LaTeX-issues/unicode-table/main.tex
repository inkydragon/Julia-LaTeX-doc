\documentclass[
  UTF8, % UTF-8 编码读入
  a4paper,  % A4 纸大小
  oneside,  % 单页模式,左右页边距相同
]{ctexbook} % book 类:需要 `\part{}' 命令

\usepackage[% 上下、左右对称
  hmargin = 3.0cm, % left and right margin
  vmargin = 2.5cm, % top and bottom margin
]{geometry}

%% font settings
\usepackage{fontspec}
\setsansfont{DejaVu Sans}
\setmonofont{DejaVu Sans Mono}

% CJK font
% \usepackage{xeCJK}
% \setCJKmonofont{Noto Sans Mono CJK SC}
% \setCJKfallbackfamilyfont{\CJKttdefault}{ % \texttt 和 \ttfamily
%   {Noto Sans Mono CJK KR},
% }

% Unicode char fallback
% + Define fallback font for specific Unicode characters in LuaLaTeX - TeX - LaTeX Stack Exchange
%   https://tex.stackexchange.com/questions/224584/define-fallback-font-for-specific-unicode-characters-in-lualatex/224585
\usepackage{newunicodechar}
\newfontfamily{\fallbackfont}{DejaVu Sans}[Scale=MatchLowercase]
\DeclareTextFontCommand{\textfallback}{\fallbackfont}
\newcommand{\fallbackchar}[2][\textfallback]{%
    \newunicodechar{#2}{#1{#2}}%
}
\fallbackchar{⊻}

\usepackage{amsmath,unicode-math}
\setmathfont{STEP Math}
\newfontface\MathSymFontOne{STEP Math}
\newfontface\MathSymFontTwo{FreeSerif}


% emoji
% + 在 LaTeX 中使用 Emoji ✌️ 
%   https://zhuanlan.zhihu.com/p/109158588
\usepackage{emoji}
\setemojifont{Twemoji Mozilla} % texlive2020 自带;可换 Noto Color Emoji
\newfontface\EmojiFont{Twemoji Mozilla}[Renderer=HarfBuzz]
%% font settings END

% tables
% \usepackage{tabulary}
\usepackage{longtable,booktabs}
%


\begin{document}

Julia 提供了非常灵活的变量命名策略。
变量名是大小写敏感的,且不包含语义,意思是说,Julia 不会根据变量的名字来区别对待它们。 
(译者注:Julia \textbf{不会}自动将全大写的变量识别为常量,
也\textbf{不会}将有特定前后缀的变量自动识别为某种特定类型的变量,即不会根据变量名字,自动判断变量的任何属性。)

% \input{utable.tex}

\begin{longtable}{p{0.1\linewidth}p{0.05\linewidth}p{0.22\linewidth}p{0.53\linewidth}}
  \caption{Unicode 输入表} \label{unicode-input-table} \\
  %% 所有表格 表头
  \toprule
  \bf{码点} & \bf{字符} & \tt{TAB}\bf{补全序列} & \bf{Unicode 名称} \\
  \hline \endhead
  %% 所有表格 页脚
  \multicolumn{4}{r}{下页待续} \\ 
  \midrule \endfoot
  %% 表格最后一页 页脚
  \bottomrule \endlastfoot

  %% 表格正文
  \input{unicode-input-table.tex}
  
  \bottomrule
\end{longtable}

Julia 提供了非常灵活的变量命名策略。
变量名是大小写敏感的,且不包含语义,意思是说,Julia 不会根据变量的名字来区别对待它们。 
(译者注:Julia \textbf{不会}自动将全大写的变量识别为常量,
也\textbf{不会}将有特定前后缀的变量自动识别为某种特定类型的变量,即不会根据变量名字,自动判断变量的任何属性。)

\end{document} % 文档结尾