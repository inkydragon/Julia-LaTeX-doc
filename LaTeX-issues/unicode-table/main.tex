\documentclass[
  UTF8, % UTF-8 编码读入
  a4paper,  % A4 纸大小
  oneside,  % 单页模式,左右页边距相同
]{ctexbook} % book 类:需要 `\part{}' 命令

\usepackage[% 上下、左右对称
  hmargin = 3.0cm, % left and right margin
  vmargin = 2.5cm, % top and bottom margin
]{geometry}

%% font settings
\usepackage{fontspec}
% \setsansfont{DejaVu Sans}
% \setmonofont{DejaVu Sans Mono}

% CJK font
% \usepackage{xeCJK}
% \setCJKmonofont{Noto Sans Mono CJK SC}
% \setCJKfallbackfamilyfont{\CJKttdefault}{ % \texttt 和 \ttfamily
%   {Noto Sans Mono CJK KR},
% }

% Unicode Math
\usepackage{amsmath,unicode-math}
\setmathfont{STEP Math}
\newfontface\MathSymFontOne{STEP Math}
\newfontface\MathSymFontTwo{FreeSerif}

% emoji
% + 在 LaTeX 中使用 Emoji ✌️ 
%   https://zhuanlan.zhihu.com/p/109158588
\usepackage{emoji}
\setemojifont{Twemoji Mozilla} % texlive2020 自带;可换 Noto Color Emoji
\newfontface\EmojiFont{Twemoji Mozilla}[Renderer=HarfBuzz]

% Unicode char fallback
% + Define fallback font for specific Unicode characters in LuaLaTeX - TeX - LaTeX Stack Exchange
%   https://tex.stackexchange.com/questions/224584/define-fallback-font-for-specific-unicode-characters-in-lualatex/224585
% \usepackage{newunicodechar}
% \newfontfamily{\fallbackfont}{DejaVu Sans}[Scale=MatchLowercase]
% \DeclareTextFontCommand{\textfallback}{\fallbackfont}
% \newcommand{\fallbackchar}[2][\textfallback]{%
%     \newunicodechar{#2}{#1{#2}}%
% }
% \fallbackchar{⊻}
%% font settings END

% tables
% \usepackage{tabulary}
\usepackage{longtable,booktabs}
%


\begin{document}

Julia 提供了非常灵活的变量命名策略。
变量名是大小写敏感的,且不包含语义,意思是说,Julia 不会根据变量的名字来区别对待它们。 
(译者注:Julia \textbf{不会}自动将全大写的变量识别为常量,
也\textbf{不会}将有特定前后缀的变量自动识别为某种特定类型的变量,即不会根据变量名字,自动判断变量的任何属性。)

% \input{utable.tex}

\begin{longtable}{p{0.1\linewidth}p{0.05\linewidth}p{0.22\linewidth}p{0.53\linewidth}}
  \caption{Unicode 输入表} \label{unicode-input-table} \\
  %% 所有表格 表头
  \toprule
  \bf{码点} & \bf{字符} & \tt{TAB}\bf{补全序列} & \bf{Unicode 名称} \\
  \hline \endhead
  %% 所有表格 页脚
  \multicolumn{4}{r}{下页待续} \\ 
  \midrule \endfoot
  %% 表格最后一页 页脚
  \bottomrule \endlastfoot

  %% 表格正文
  % U+000A1 & $ ¡ $ & {\textbackslash}exclamdown & Inverted Exclamation Mark \\ \hline
U+000A3 & $ £ $ & {\textbackslash}sterling & Pound Sign \\ \hline
U+000A5 & $ ¥ $ & {\textbackslash}yen & Yen Sign \\ \hline
U+000A6 & $ ¦ $ & {\textbackslash}brokenbar & Broken Bar / Broken Vertical Bar \\ \hline
U+000A7 & $ § $ & {\textbackslash}S & Section Sign \\ \hline
U+000A9 & {\EmojiFont ©} & {\textbackslash}:copyright:, {\textbackslash}copyright & Copyright Sign \\ \hline
U+000AA & {\MathSymFontOne ª} & {\textbackslash}ordfeminine & Feminine Ordinal Indicator \\ \hline
U+000AC & $ ¬ $ & {\textbackslash}neg & Not Sign \\ \hline
U+000AE & {\EmojiFont ®} & {\textbackslash}:registered:, {\textbackslash}circledR & Registered Sign / Registered Trade Mark Sign \\ \hline
U+000AF & $ ¯ $ & {\textbackslash}highminus & Macron / Spacing Macron \\ \hline
U+000B0 & $ ° $ & {\textbackslash}degree & Degree Sign \\ \hline
U+000B1 & $ ± $ & {\textbackslash}pm & Plus-Minus Sign / Plus-Or-Minus Sign \\ \hline
U+000B2 & $ ² $ & {\textbackslash}{\textasciicircum}2 & Superscript Two / Superscript Digit Two \\ \hline
U+000B3 & $ ³ $ & {\textbackslash}{\textasciicircum}3 & Superscript Three / Superscript Digit Three \\ \hline
U+000B6 & $ ¶ $ & {\textbackslash}P & Pilcrow Sign / Paragraph Sign \\ \hline
U+000B7 & $ · $ & {\textbackslash}cdotp & Middle Dot \\ \hline
U+000B9 & $ ¹ $ & {\textbackslash}{\textasciicircum}1 & Superscript One / Superscript Digit One \\ \hline
U+000BA & {\MathSymFontOne º} & {\textbackslash}ordmasculine & Masculine Ordinal Indicator \\ \hline
U+000BC & $ ¼ $ & {\textbackslash}1/4 & Vulgar Fraction One Quarter / Fraction One Quarter \\ \hline
U+000BD & $ ½ $ & {\textbackslash}1/2 & Vulgar Fraction One Half / Fraction One Half \\ \hline
U+000BE & $ ¾ $ & {\textbackslash}3/4 & Vulgar Fraction Three Quarters / Fraction Three Quarters \\ \hline
U+000BF & $ ¿ $ & {\textbackslash}questiondown & Inverted Question Mark \\ \hline
U+000C5 & {\MathSymFontOne Å} & {\textbackslash}AA & Latin Capital Letter A With Ring Above / Latin Capital Letter A Ring \\ \hline
U+000C6 & {\MathSymFontOne Æ} & {\textbackslash}AE & Latin Capital Letter Ae / Latin Capital Letter A E \\ \hline
U+000D0 & {\MathSymFontOne Ð} & {\textbackslash}DH & Latin Capital Letter Eth \\ \hline
U+000D7 & $ × $ & {\textbackslash}times & Multiplication Sign \\ \hline
U+000D8 & {\MathSymFontOne Ø} & {\textbackslash}O & Latin Capital Letter O With Stroke / Latin Capital Letter O Slash \\ \hline
U+000DE & {\MathSymFontOne Þ} & {\textbackslash}TH & Latin Capital Letter Thorn \\ \hline
U+000DF & {\MathSymFontOne ß} & {\textbackslash}ss & Latin Small Letter Sharp S \\ \hline
U+000E5 & {\MathSymFontOne å} & {\textbackslash}aa & Latin Small Letter A With Ring Above / Latin Small Letter A Ring \\ \hline
U+000E6 & {\MathSymFontOne æ} & {\textbackslash}ae & Latin Small Letter Ae / Latin Small Letter A E \\ \hline
U+000F0 & $ ð $ & {\textbackslash}eth, {\textbackslash}dh & Latin Small Letter Eth \\ \hline
U+000F7 & $ ÷ $ & {\textbackslash}div & Division Sign \\ \hline
U+000F8 & {\MathSymFontOne ø} & {\textbackslash}o & Latin Small Letter O With Stroke / Latin Small Letter O Slash \\ \hline
U+000FE & {\MathSymFontOne þ} & {\textbackslash}th & Latin Small Letter Thorn \\ \hline
U+00110 & {\MathSymFontOne Đ} & {\textbackslash}DJ & Latin Capital Letter D With Stroke / Latin Capital Letter D Bar \\ \hline
U+00111 & {\MathSymFontOne đ} & {\textbackslash}dj & Latin Small Letter D With Stroke / Latin Small Letter D Bar \\ \hline
U+00127 & {\MathSymFontOne ħ} & {\textbackslash}hbar & Latin Small Letter H With Stroke / Latin Small Letter H Bar \\ \hline
U+00131 & $ ı $ & {\textbackslash}imath & Latin Small Letter Dotless I \\ \hline
U+00141 & {\MathSymFontOne Ł} & {\textbackslash}L & Latin Capital Letter L With Stroke / Latin Capital Letter L Slash \\ \hline
U+00142 & {\MathSymFontOne ł} & {\textbackslash}l & Latin Small Letter L With Stroke / Latin Small Letter L Slash \\ \hline
U+0014A & {\MathSymFontOne Ŋ} & {\textbackslash}NG & Latin Capital Letter Eng \\ \hline
U+0014B & {\MathSymFontOne ŋ} & {\textbackslash}ng & Latin Small Letter Eng \\ \hline
U+00152 & {\MathSymFontOne Œ} & {\textbackslash}OE & Latin Capital Ligature Oe / Latin Capital Letter O E \\ \hline
U+00153 & {\MathSymFontOne œ} & {\textbackslash}oe & Latin Small Ligature Oe / Latin Small Letter O E \\ \hline
U+00195 & {\MathSymFontOne ƕ} & {\textbackslash}hvlig & Latin Small Letter Hv / Latin Small Letter H V \\ \hline
U+0019E & {\MathSymFontOne ƞ} & {\textbackslash}nrleg & Latin Small Letter N With Long Right Leg \\ \hline
U+001B5 & $ Ƶ $ & {\textbackslash}Zbar & Latin Capital Letter Z With Stroke / Latin Capital Letter Z Bar \\ \hline
U+001C2 & {\MathSymFontOne ǂ} & {\textbackslash}doublepipe & Latin Letter Alveolar Click / Latin Letter Pipe Double Bar \\ \hline
U+00237 & $ ȷ $ & {\textbackslash}jmath & Latin Small Letter Dotless J \\ \hline
U+00250 & {\MathSymFontOne ɐ} & {\textbackslash}trna & Latin Small Letter Turned A \\ \hline
U+00252 & {\MathSymFontOne ɒ} & {\textbackslash}trnsa & Latin Small Letter Turned Alpha / Latin Small Letter Turned Script A \\ \hline
U+00254 & {\MathSymFontOne ɔ} & {\textbackslash}openo & Latin Small Letter Open O \\ \hline
U+00256 & {\MathSymFontOne ɖ} & {\textbackslash}rtld & Latin Small Letter D With Tail / Latin Small Letter D Retroflex Hook \\ \hline
U+00259 & {\MathSymFontOne ə} & {\textbackslash}schwa & Latin Small Letter Schwa \\ \hline
U+00263 & {\MathSymFontOne ɣ} & {\textbackslash}pgamma & Latin Small Letter Gamma \\ \hline
U+00264 & {\MathSymFontOne ɤ} & {\textbackslash}pbgam & Latin Small Letter Rams Horn / Latin Small Letter Baby Gamma \\ \hline
U+00265 & {\MathSymFontOne ɥ} & {\textbackslash}trnh & Latin Small Letter Turned H \\ \hline
U+0026C & {\MathSymFontOne ɬ} & {\textbackslash}btdl & Latin Small Letter L With Belt / Latin Small Letter L Belt \\ \hline
U+0026D & {\MathSymFontOne ɭ} & {\textbackslash}rtll & Latin Small Letter L With Retroflex Hook / Latin Small Letter L Retroflex Hook \\ \hline
U+0026F & {\MathSymFontOne ɯ} & {\textbackslash}trnm & Latin Small Letter Turned M \\ \hline
U+00270 & {\MathSymFontOne ɰ} & {\textbackslash}trnmlr & Latin Small Letter Turned M With Long Leg \\ \hline
U+00271 & {\MathSymFontOne ɱ} & {\textbackslash}ltlmr & Latin Small Letter M With Hook / Latin Small Letter M Hook \\ \hline
U+00272 & {\MathSymFontOne ɲ} & {\textbackslash}ltln & Latin Small Letter N With Left Hook / Latin Small Letter N Hook \\ \hline
U+00273 & {\MathSymFontOne ɳ} & {\textbackslash}rtln & Latin Small Letter N With Retroflex Hook / Latin Small Letter N Retroflex Hook \\ \hline
U+00277 & {\MathSymFontOne ɷ} & {\textbackslash}clomeg & Latin Small Letter Closed Omega \\ \hline
U+00278 & {\MathSymFontOne ɸ} & {\textbackslash}ltphi & Latin Small Letter Phi \\ \hline
U+00279 & {\MathSymFontOne ɹ} & {\textbackslash}trnr & Latin Small Letter Turned R \\ \hline
U+0027A & {\MathSymFontOne ɺ} & {\textbackslash}trnrl & Latin Small Letter Turned R With Long Leg \\ \hline
U+0027B & {\MathSymFontOne ɻ} & {\textbackslash}rttrnr & Latin Small Letter Turned R With Hook / Latin Small Letter Turned R Hook \\ \hline
U+0027C & {\MathSymFontOne ɼ} & {\textbackslash}rl & Latin Small Letter R With Long Leg \\ \hline
U+0027D & {\MathSymFontOne ɽ} & {\textbackslash}rtlr & Latin Small Letter R With Tail / Latin Small Letter R Hook \\ \hline
U+0027E & {\MathSymFontOne ɾ} & {\textbackslash}fhr & Latin Small Letter R With Fishhook / Latin Small Letter Fishhook R \\ \hline
U+00282 & {\MathSymFontOne ʂ} & {\textbackslash}rtls & Latin Small Letter S With Hook / Latin Small Letter S Hook \\ \hline
U+00283 & {\MathSymFontOne ʃ} & {\textbackslash}esh & Latin Small Letter Esh \\ \hline
U+00287 & {\MathSymFontOne ʇ} & {\textbackslash}trnt & Latin Small Letter Turned T \\ \hline
U+00288 & {\MathSymFontOne ʈ} & {\textbackslash}rtlt & Latin Small Letter T With Retroflex Hook / Latin Small Letter T Retroflex Hook \\ \hline
U+0028A & {\MathSymFontOne ʊ} & {\textbackslash}pupsil & Latin Small Letter Upsilon \\ \hline
U+0028B & {\MathSymFontOne ʋ} & {\textbackslash}pscrv & Latin Small Letter V With Hook / Latin Small Letter Script V \\ \hline
U+0028C & {\MathSymFontOne ʌ} & {\textbackslash}invv & Latin Small Letter Turned V \\ \hline
U+0028D & {\MathSymFontOne ʍ} & {\textbackslash}invw & Latin Small Letter Turned W \\ \hline
U+0028E & {\MathSymFontOne ʎ} & {\textbackslash}trny & Latin Small Letter Turned Y \\ \hline
U+00290 & {\MathSymFontOne ʐ} & {\textbackslash}rtlz & Latin Small Letter Z With Retroflex Hook / Latin Small Letter Z Retroflex Hook \\ \hline
U+00292 & {\MathSymFontOne ʒ} & {\textbackslash}yogh & Latin Small Letter Ezh / Latin Small Letter Yogh \\ \hline
U+00294 & {\MathSymFontOne ʔ} & {\textbackslash}glst & Latin Letter Glottal Stop \\ \hline
U+00295 & {\MathSymFontOne ʕ} & {\textbackslash}reglst & Latin Letter Pharyngeal Voiced Fricative / Latin Letter Reversed Glottal Stop \\ \hline
U+00296 & {\MathSymFontOne ʖ} & {\textbackslash}inglst & Latin Letter Inverted Glottal Stop \\ \hline
U+0029E & {\MathSymFontOne ʞ} & {\textbackslash}turnk & Latin Small Letter Turned K \\ \hline
U+002A4 & {\MathSymFontOne ʤ} & {\textbackslash}dyogh & Latin Small Letter Dezh Digraph / Latin Small Letter D Yogh \\ \hline
U+002A7 & {\MathSymFontOne ʧ} & {\textbackslash}tesh & Latin Small Letter Tesh Digraph / Latin Small Letter T Esh \\ \hline
U+002B0 & $ ʰ $ & {\textbackslash}{\textasciicircum}h & Modifier Letter Small H \\ \hline
U+002B2 & $ ʲ $ & {\textbackslash}{\textasciicircum}j & Modifier Letter Small J \\ \hline
U+002B3 & $ ʳ $ & {\textbackslash}{\textasciicircum}r & Modifier Letter Small R \\ \hline
U+002B7 & $ ʷ $ & {\textbackslash}{\textasciicircum}w & Modifier Letter Small W \\ \hline
U+002B8 & $ ʸ $ & {\textbackslash}{\textasciicircum}y & Modifier Letter Small Y \\ \hline
U+002BC & {\MathSymFontOne ʼ} & {\textbackslash}rasp & Modifier Letter Apostrophe \\ \hline
U+002C8 & {\MathSymFontOne ˈ} & {\textbackslash}verts & Modifier Letter Vertical Line \\ \hline
U+002CC & {\MathSymFontOne ˌ} & {\textbackslash}verti & Modifier Letter Low Vertical Line \\ \hline
U+002D0 & {\MathSymFontOne ː} & {\textbackslash}lmrk & Modifier Letter Triangular Colon \\ \hline
U+002D1 & {\MathSymFontOne ˑ} & {\textbackslash}hlmrk & Modifier Letter Half Triangular Colon \\ \hline
U+002D2 & $ ˒ $ & {\textbackslash}sbrhr & Modifier Letter Centred Right Half Ring / Modifier Letter Centered Right Half Ring \\ \hline
U+002D3 & $ ˓ $ & {\textbackslash}sblhr & Modifier Letter Centred Left Half Ring / Modifier Letter Centered Left Half Ring \\ \hline
U+002D4 & $ ˔ $ & {\textbackslash}rais & Modifier Letter Up Tack \\ \hline
U+002D5 & $ ˕ $ & {\textbackslash}low & Modifier Letter Down Tack \\ \hline
U+002D8 & $ ˘ $ & {\textbackslash}u & Breve / Spacing Breve \\ \hline
U+002DC & $ ˜ $ & {\textbackslash}tildelow & Small Tilde / Spacing Tilde \\ \hline
U+002E1 & $ ˡ $ & {\textbackslash}{\textasciicircum}l & Modifier Letter Small L \\ \hline
U+002E2 & $ ˢ $ & {\textbackslash}{\textasciicircum}s & Modifier Letter Small S \\ \hline
U+002E3 & $ ˣ $ & {\textbackslash}{\textasciicircum}x & Modifier Letter Small X \\ \hline
U+00300 & {\MathSymFontOne  ̀ } & {\textbackslash}grave & Combining Grave Accent / Non-Spacing Grave \\ \hline
U+00301 & {\MathSymFontOne  ́ } & {\textbackslash}acute & Combining Acute Accent / Non-Spacing Acute \\ \hline
U+00302 & {\MathSymFontOne  ̂ } & {\textbackslash}hat & Combining Circumflex Accent / Non-Spacing Circumflex \\ \hline
U+00303 & {\MathSymFontOne  ̃ } & {\textbackslash}tilde & Combining Tilde / Non-Spacing Tilde \\ \hline
U+00304 & {\MathSymFontOne  ̄ } & {\textbackslash}bar & Combining Macron / Non-Spacing Macron \\ \hline
U+00305 & {\MathSymFontOne  ̅ } & {\textbackslash}overbar & Combining Overline / Non-Spacing Overscore \\ \hline
U+00306 & {\MathSymFontOne  ̆ } & {\textbackslash}breve & Combining Breve / Non-Spacing Breve \\ \hline
U+00307 & {\MathSymFontOne  ̇ } & {\textbackslash}dot & Combining Dot Above / Non-Spacing Dot Above \\ \hline
U+00308 & {\MathSymFontOne  ̈ } & {\textbackslash}ddot & Combining Diaeresis / Non-Spacing Diaeresis \\ \hline
U+00309 & {\MathSymFontOne  ̉ } & {\textbackslash}ovhook & Combining Hook Above / Non-Spacing Hook Above \\ \hline
U+0030A & {\MathSymFontOne  ̊ } & {\textbackslash}ocirc & Combining Ring Above / Non-Spacing Ring Above \\ \hline
U+0030B & {\MathSymFontOne  ̋ } & {\textbackslash}H & Combining Double Acute Accent / Non-Spacing Double Acute \\ \hline
U+0030C & {\MathSymFontOne  ̌ } & {\textbackslash}check & Combining Caron / Non-Spacing Hacek \\ \hline
U+00310 & {\MathSymFontOne  ̐ } & {\textbackslash}candra & Combining Candrabindu / Non-Spacing Candrabindu \\ \hline
U+00312 & {\MathSymFontOne  ̒ } & {\textbackslash}oturnedcomma & Combining Turned Comma Above / Non-Spacing Turned Comma Above \\ \hline
U+00315 & {\MathSymFontOne  ̕ } & {\textbackslash}ocommatopright & Combining Comma Above Right / Non-Spacing Comma Above Right \\ \hline
U+0031A & {\MathSymFontOne  ̚ } & {\textbackslash}droang & Combining Left Angle Above / Non-Spacing Left Angle Above \\ \hline
U+00321 & {\MathSymFontOne  ̡ } & {\textbackslash}palh & Combining Palatalized Hook Below / Non-Spacing Palatalized Hook Below \\ \hline
U+00322 & {\MathSymFontOne  ̢ } & {\textbackslash}rh & Combining Retroflex Hook Below / Non-Spacing Retroflex Hook Below \\ \hline
U+00327 & {\MathSymFontOne  ̧ } & {\textbackslash}c & Combining Cedilla / Non-Spacing Cedilla \\ \hline
U+00328 & {\MathSymFontOne  ̨ } & {\textbackslash}k & Combining Ogonek / Non-Spacing Ogonek \\ \hline
U+0032A & {\MathSymFontOne  ̪ } & {\textbackslash}sbbrg & Combining Bridge Below / Non-Spacing Bridge Below \\ \hline
U+00330 & {\MathSymFontOne  ̰ } & {\textbackslash}wideutilde & Combining Tilde Below / Non-Spacing Tilde Below \\ \hline
U+00332 & {\MathSymFontOne  ̲ } & {\textbackslash}underbar & Combining Low Line / Non-Spacing Underscore \\ \hline
U+00336 & {\MathSymFontOne  ̶ } & {\textbackslash}strike, {\textbackslash}sout & Combining Long Stroke Overlay / Non-Spacing Long Bar Overlay \\ \hline
U+00338 & {\MathSymFontOne  ̸ } & {\textbackslash}not & Combining Long Solidus Overlay / Non-Spacing Long Slash Overlay \\ \hline
U+0034D & {\MathSymFontOne  ͍ } & {\textbackslash}underleftrightarrow & Combining Left Right Arrow Below \\ \hline
U+00391 & $ Α $ & {\textbackslash}Alpha & Greek Capital Letter Alpha \\ \hline
U+00392 & $ Β $ & {\textbackslash}Beta & Greek Capital Letter Beta \\ \hline
U+00393 & $ Γ $ & {\textbackslash}Gamma & Greek Capital Letter Gamma \\ \hline
U+00394 & $ Δ $ & {\textbackslash}Delta & Greek Capital Letter Delta \\ \hline
U+00395 & $ Ε $ & {\textbackslash}Epsilon & Greek Capital Letter Epsilon \\ \hline
U+00396 & $ Ζ $ & {\textbackslash}Zeta & Greek Capital Letter Zeta \\ \hline
U+00397 & $ Η $ & {\textbackslash}Eta & Greek Capital Letter Eta \\ \hline
U+00398 & $ Θ $ & {\textbackslash}Theta & Greek Capital Letter Theta \\ \hline
U+00399 & $ Ι $ & {\textbackslash}Iota & Greek Capital Letter Iota \\ \hline
U+0039A & $ Κ $ & {\textbackslash}Kappa & Greek Capital Letter Kappa \\ \hline
U+0039B & $ Λ $ & {\textbackslash}Lambda & Greek Capital Letter Lamda / Greek Capital Letter Lambda \\ \hline
U+0039C & $ Μ $ & {\textbackslash}upMu & Greek Capital Letter Mu \\ \hline
U+0039D & $ Ν $ & {\textbackslash}upNu & Greek Capital Letter Nu \\ \hline
U+0039E & $ Ξ $ & {\textbackslash}Xi & Greek Capital Letter Xi \\ \hline
U+0039F & $ Ο $ & {\textbackslash}upOmicron & Greek Capital Letter Omicron \\ \hline
U+003A0 & $ Π $ & {\textbackslash}Pi & Greek Capital Letter Pi \\ \hline
U+003A1 & $ Ρ $ & {\textbackslash}Rho & Greek Capital Letter Rho \\ \hline
U+003A3 & $ Σ $ & {\textbackslash}Sigma & Greek Capital Letter Sigma \\ \hline
U+003A4 & $ Τ $ & {\textbackslash}Tau & Greek Capital Letter Tau \\ \hline
U+003A5 & $ Υ $ & {\textbackslash}Upsilon & Greek Capital Letter Upsilon \\ \hline
U+003A6 & $ Φ $ & {\textbackslash}Phi & Greek Capital Letter Phi \\ \hline
U+003A7 & $ Χ $ & {\textbackslash}Chi & Greek Capital Letter Chi \\ \hline
U+003A8 & $ Ψ $ & {\textbackslash}Psi & Greek Capital Letter Psi \\ \hline
U+003A9 & $ Ω $ & {\textbackslash}Omega & Greek Capital Letter Omega \\ \hline
U+003B1 & $ α $ & {\textbackslash}alpha & Greek Small Letter Alpha \\ \hline
U+003B2 & $ β $ & {\textbackslash}beta & Greek Small Letter Beta \\ \hline
U+003B3 & $ γ $ & {\textbackslash}gamma & Greek Small Letter Gamma \\ \hline
U+003B4 & $ δ $ & {\textbackslash}delta & Greek Small Letter Delta \\ \hline
U+003B5 & $ ε $ & {\textbackslash}upepsilon, {\textbackslash}varepsilon & Greek Small Letter Epsilon \\ \hline
U+003B6 & $ ζ $ & {\textbackslash}zeta & Greek Small Letter Zeta \\ \hline
U+003B7 & $ η $ & {\textbackslash}eta & Greek Small Letter Eta \\ \hline
U+003B8 & $ θ $ & {\textbackslash}theta & Greek Small Letter Theta \\ \hline
U+003B9 & $ ι $ & {\textbackslash}iota & Greek Small Letter Iota \\ \hline
U+003BA & $ κ $ & {\textbackslash}kappa & Greek Small Letter Kappa \\ \hline
U+003BB & $ λ $ & {\textbackslash}lambda & Greek Small Letter Lamda / Greek Small Letter Lambda \\ \hline
U+003BC & $ μ $ & {\textbackslash}mu & Greek Small Letter Mu \\ \hline
U+003BD & $ ν $ & {\textbackslash}nu & Greek Small Letter Nu \\ \hline
U+003BE & $ ξ $ & {\textbackslash}xi & Greek Small Letter Xi \\ \hline
U+003BF & $ ο $ & {\textbackslash}upomicron & Greek Small Letter Omicron \\ \hline
U+003C0 & $ π $ & {\textbackslash}pi & Greek Small Letter Pi \\ \hline
U+003C1 & $ ρ $ & {\textbackslash}rho & Greek Small Letter Rho \\ \hline
U+003C2 & $ ς $ & {\textbackslash}varsigma & Greek Small Letter Final Sigma \\ \hline
U+003C3 & $ σ $ & {\textbackslash}sigma & Greek Small Letter Sigma \\ \hline
U+003C4 & $ τ $ & {\textbackslash}tau & Greek Small Letter Tau \\ \hline
U+003C5 & $ υ $ & {\textbackslash}upsilon & Greek Small Letter Upsilon \\ \hline
U+003C6 & $ φ $ & {\textbackslash}varphi & Greek Small Letter Phi \\ \hline
U+003C7 & $ χ $ & {\textbackslash}chi & Greek Small Letter Chi \\ \hline
U+003C8 & $ ψ $ & {\textbackslash}psi & Greek Small Letter Psi \\ \hline
U+003C9 & $ ω $ & {\textbackslash}omega & Greek Small Letter Omega \\ \hline
U+003D0 & {\MathSymFontOne ϐ} & {\textbackslash}upvarbeta & Greek Beta Symbol / Greek Small Letter Curled Beta \\ \hline
U+003D1 & $ ϑ $ & {\textbackslash}vartheta & Greek Theta Symbol / Greek Small Letter Script Theta \\ \hline
U+003D5 & $ ϕ $ & {\textbackslash}phi & Greek Phi Symbol / Greek Small Letter Script Phi \\ \hline
U+003D6 & $ ϖ $ & {\textbackslash}varpi & Greek Pi Symbol / Greek Small Letter Omega Pi \\ \hline
U+003D8 & {\MathSymFontOne Ϙ} & {\textbackslash}upoldKoppa & Greek Letter Archaic Koppa \\ \hline
U+003D9 & {\MathSymFontOne ϙ} & {\textbackslash}upoldkoppa & Greek Small Letter Archaic Koppa \\ \hline
U+003DA & {\MathSymFontOne Ϛ} & {\textbackslash}Stigma & Greek Letter Stigma / Greek Capital Letter Stigma \\ \hline
U+003DB & {\MathSymFontOne ϛ} & {\textbackslash}upstigma & Greek Small Letter Stigma \\ \hline
U+003DC & {\MathSymFontOne Ϝ} & {\textbackslash}Digamma & Greek Letter Digamma / Greek Capital Letter Digamma \\ \hline
U+003DD & {\MathSymFontOne ϝ} & {\textbackslash}digamma & Greek Small Letter Digamma \\ \hline
U+003DE & {\MathSymFontOne Ϟ} & {\textbackslash}Koppa & Greek Letter Koppa / Greek Capital Letter Koppa \\ \hline
U+003DF & {\MathSymFontOne ϟ} & {\textbackslash}upkoppa & Greek Small Letter Koppa \\ \hline
U+003E0 & {\MathSymFontOne Ϡ} & {\textbackslash}Sampi & Greek Letter Sampi / Greek Capital Letter Sampi \\ \hline
U+003E1 & {\MathSymFontOne ϡ} & {\textbackslash}upsampi & Greek Small Letter Sampi \\ \hline
U+003F0 & $ ϰ $ & {\textbackslash}varkappa & Greek Kappa Symbol / Greek Small Letter Script Kappa \\ \hline
U+003F1 & $ ϱ $ & {\textbackslash}varrho & Greek Rho Symbol / Greek Small Letter Tailed Rho \\ \hline
U+003F4 & $ ϴ $ & {\textbackslash}varTheta & Greek Capital Theta Symbol \\ \hline
U+003F5 & $ ϵ $ & {\textbackslash}epsilon & Greek Lunate Epsilon Symbol \\ \hline
U+003F6 & $ ϶ $ & {\textbackslash}backepsilon & Greek Reversed Lunate Epsilon Symbol \\ \hline
U+01D2C & $ ᴬ $ & {\textbackslash}{\textasciicircum}A & Modifier Letter Capital A \\ \hline
U+01D2E & $ ᴮ $ & {\textbackslash}{\textasciicircum}B & Modifier Letter Capital B \\ \hline
U+01D30 & $ ᴰ $ & {\textbackslash}{\textasciicircum}D & Modifier Letter Capital D \\ \hline
U+01D31 & $ ᴱ $ & {\textbackslash}{\textasciicircum}E & Modifier Letter Capital E \\ \hline
U+01D33 & $ ᴳ $ & {\textbackslash}{\textasciicircum}G & Modifier Letter Capital G \\ \hline
U+01D34 & $ ᴴ $ & {\textbackslash}{\textasciicircum}H & Modifier Letter Capital H \\ \hline
U+01D35 & $ ᴵ $ & {\textbackslash}{\textasciicircum}I & Modifier Letter Capital I \\ \hline
U+01D36 & $ ᴶ $ & {\textbackslash}{\textasciicircum}J & Modifier Letter Capital J \\ \hline
U+01D37 & $ ᴷ $ & {\textbackslash}{\textasciicircum}K & Modifier Letter Capital K \\ \hline
U+01D38 & $ ᴸ $ & {\textbackslash}{\textasciicircum}L & Modifier Letter Capital L \\ \hline
U+01D39 & $ ᴹ $ & {\textbackslash}{\textasciicircum}M & Modifier Letter Capital M \\ \hline
U+01D3A & $ ᴺ $ & {\textbackslash}{\textasciicircum}N & Modifier Letter Capital N \\ \hline
U+01D3C & $ ᴼ $ & {\textbackslash}{\textasciicircum}O & Modifier Letter Capital O \\ \hline
U+01D3E & $ ᴾ $ & {\textbackslash}{\textasciicircum}P & Modifier Letter Capital P \\ \hline
U+01D3F & $ ᴿ $ & {\textbackslash}{\textasciicircum}R & Modifier Letter Capital R \\ \hline
U+01D40 & $ ᵀ $ & {\textbackslash}{\textasciicircum}T & Modifier Letter Capital T \\ \hline
U+01D41 & $ ᵁ $ & {\textbackslash}{\textasciicircum}U & Modifier Letter Capital U \\ \hline
U+01D42 & $ ᵂ $ & {\textbackslash}{\textasciicircum}W & Modifier Letter Capital W \\ \hline
U+01D43 & $ ᵃ $ & {\textbackslash}{\textasciicircum}a & Modifier Letter Small A \\ \hline
U+01D45 & {\MathSymFontTwo ᵅ} & {\textbackslash}{\textasciicircum}alpha & Modifier Letter Small Alpha \\ \hline
U+01D47 & $ ᵇ $ & {\textbackslash}{\textasciicircum}b & Modifier Letter Small B \\ \hline
U+01D48 & $ ᵈ $ & {\textbackslash}{\textasciicircum}d & Modifier Letter Small D \\ \hline
U+01D49 & {\MathSymFontTwo ᵉ} & {\textbackslash}{\textasciicircum}e & Modifier Letter Small E \\ \hline
U+01D4B & {\MathSymFontTwo ᵋ} & {\textbackslash}{\textasciicircum}epsilon & Modifier Letter Small Open E \\ \hline
U+01D4D & $ ᵍ $ & {\textbackslash}{\textasciicircum}g & Modifier Letter Small G \\ \hline
U+01D4F & $ ᵏ $ & {\textbackslash}{\textasciicircum}k & Modifier Letter Small K \\ \hline
U+01D50 & $ ᵐ $ & {\textbackslash}{\textasciicircum}m & Modifier Letter Small M \\ \hline
U+01D52 & $ ᵒ $ & {\textbackslash}{\textasciicircum}o & Modifier Letter Small O \\ \hline
U+01D56 & $ ᵖ $ & {\textbackslash}{\textasciicircum}p & Modifier Letter Small P \\ \hline
U+01D57 & $ ᵗ $ & {\textbackslash}{\textasciicircum}t & Modifier Letter Small T \\ \hline
U+01D58 & $ ᵘ $ & {\textbackslash}{\textasciicircum}u & Modifier Letter Small U \\ \hline
U+01D5B & $ ᵛ $ & {\textbackslash}{\textasciicircum}v & Modifier Letter Small V \\ \hline
U+01D5D & $ ᵝ $ & {\textbackslash}{\textasciicircum}beta & Modifier Letter Small Beta \\ \hline
U+01D5E & $ ᵞ $ & {\textbackslash}{\textasciicircum}gamma & Modifier Letter Small Greek Gamma \\ \hline
U+01D5F & $ ᵟ $ & {\textbackslash}{\textasciicircum}delta & Modifier Letter Small Delta \\ \hline
U+01D60 & $ ᵠ $ & {\textbackslash}{\textasciicircum}phi & Modifier Letter Small Greek Phi \\ \hline
U+01D61 & $ ᵡ $ & {\textbackslash}{\textasciicircum}chi & Modifier Letter Small Chi \\ \hline
U+01D62 & $ ᵢ $ & {\textbackslash}\_i & Latin Subscript Small Letter I \\ \hline
U+01D63 & $ ᵣ $ & {\textbackslash}\_r & Latin Subscript Small Letter R \\ \hline
U+01D64 & $ ᵤ $ & {\textbackslash}\_u & Latin Subscript Small Letter U \\ \hline
U+01D65 & $ ᵥ $ & {\textbackslash}\_v & Latin Subscript Small Letter V \\ \hline
U+01D66 & $ ᵦ $ & {\textbackslash}\_beta & Greek Subscript Small Letter Beta \\ \hline
U+01D67 & $ ᵧ $ & {\textbackslash}\_gamma & Greek Subscript Small Letter Gamma \\ \hline
U+01D68 & $ ᵨ $ & {\textbackslash}\_rho & Greek Subscript Small Letter Rho \\ \hline
U+01D69 & $ ᵩ $ & {\textbackslash}\_phi & Greek Subscript Small Letter Phi \\ \hline
U+01D6A & $ ᵪ $ & {\textbackslash}\_chi & Greek Subscript Small Letter Chi \\ \hline
U+01D9C & $ ᶜ $ & {\textbackslash}{\textasciicircum}c & Modifier Letter Small C \\ \hline
U+01DA0 & $ ᶠ $ & {\textbackslash}{\textasciicircum}f & Modifier Letter Small F \\ \hline
U+01DA5 & {\MathSymFontTwo ᶥ} & {\textbackslash}{\textasciicircum}iota & Modifier Letter Small Iota \\ \hline
U+01DB2 & {\MathSymFontTwo ᶲ} & {\textbackslash}{\textasciicircum}Phi & Modifier Letter Small Phi \\ \hline
U+01DBB & $ ᶻ $ & {\textbackslash}{\textasciicircum}z & Modifier Letter Small Z \\ \hline
U+01DBF & $ ᶿ $ & {\textbackslash}{\textasciicircum}theta & Modifier Letter Small Theta \\ \hline
U+02002 & $   $ & {\textbackslash}enspace & En Space \\ \hline
U+02003 & $   $ & {\textbackslash}quad & Em Space \\ \hline
U+02005 & $   $ & {\textbackslash}thickspace & Four-Per-Em Space \\ \hline
U+02009 & $   $ & {\textbackslash}thinspace & Thin Space \\ \hline
U+0200A & $   $ & {\textbackslash}hspace & Hair Space \\ \hline
U+02013 & $ – $ & {\textbackslash}endash & En Dash \\ \hline
U+02014 & $ — $ & {\textbackslash}emdash & Em Dash \\ \hline
U+02016 & $ ‖ $ & {\textbackslash}Vert & Double Vertical Line / Double Vertical Bar \\ \hline
U+02018 & $ ‘ $ & {\textbackslash}lq & Left Single Quotation Mark / Single Turned Comma Quotation Mark \\ \hline
U+02019 & $ ’ $ & {\textbackslash}rq & Right Single Quotation Mark / Single Comma Quotation Mark \\ \hline
U+0201B & $ ‛ $ & {\textbackslash}reapos & Single High-Reversed-9 Quotation Mark / Single Reversed Comma Quotation Mark \\ \hline
U+0201C & $ “ $ & {\textbackslash}quotedblleft & Left Double Quotation Mark / Double Turned Comma Quotation Mark \\ \hline
U+0201D & $ ” $ & {\textbackslash}quotedblright & Right Double Quotation Mark / Double Comma Quotation Mark \\ \hline
U+02020 & $ † $ & {\textbackslash}dagger & Dagger \\ \hline
U+02021 & $ ‡ $ & {\textbackslash}ddagger & Double Dagger \\ \hline
U+02022 & $ • $ & {\textbackslash}bullet & Bullet \\ \hline
U+02026 & $ … $ & {\textbackslash}dots, {\textbackslash}ldots & Horizontal Ellipsis \\ \hline
U+02030 & $ ‰ $ & {\textbackslash}perthousand & Per Mille Sign \\ \hline
U+02031 & $ ‱ $ & {\textbackslash}pertenthousand & Per Ten Thousand Sign \\ \hline
U+02032 & $ ′ $ & {\textbackslash}prime & Prime \\ \hline
U+02033 & $ ″ $ & {\textbackslash}pprime & Double Prime \\ \hline
U+02034 & $ ‴ $ & {\textbackslash}ppprime & Triple Prime \\ \hline
U+02035 & $ ‵ $ & {\textbackslash}backprime & Reversed Prime \\ \hline
U+02036 & $ ‶ $ & {\textbackslash}backpprime & Reversed Double Prime \\ \hline
U+02037 & $ ‷ $ & {\textbackslash}backppprime & Reversed Triple Prime \\ \hline
U+02039 & $ ‹ $ & {\textbackslash}guilsinglleft & Single Left-Pointing Angle Quotation Mark / Left Pointing Single Guillemet \\ \hline
U+0203A & $ › $ & {\textbackslash}guilsinglright & Single Right-Pointing Angle Quotation Mark / Right Pointing Single Guillemet \\ \hline
U+0203C & {\EmojiFont ‼} & {\textbackslash}:bangbang: & Double Exclamation Mark \\ \hline
U+02040 & $ ⁀ $ & {\textbackslash}tieconcat & Character Tie \\ \hline
U+02049 & {\EmojiFont ⁉} & {\textbackslash}:interrobang: & Exclamation Question Mark \\ \hline
U+02057 & $ ⁗ $ & {\textbackslash}pppprime & Quadruple Prime \\ \hline
U+0205D & {\MathSymFontTwo ⁝} & {\textbackslash}tricolon & Tricolon \\ \hline
U+02060 & $ ⁠ $ & {\textbackslash}nolinebreak & Word Joiner \\ \hline
U+02070 & $ ⁰ $ & {\textbackslash}{\textasciicircum}0 & Superscript Zero / Superscript Digit Zero \\ \hline
U+02071 & $ ⁱ $ & {\textbackslash}{\textasciicircum}i & Superscript Latin Small Letter I \\ \hline
U+02074 & $ ⁴ $ & {\textbackslash}{\textasciicircum}4 & Superscript Four / Superscript Digit Four \\ \hline
U+02075 & $ ⁵ $ & {\textbackslash}{\textasciicircum}5 & Superscript Five / Superscript Digit Five \\ \hline
U+02076 & $ ⁶ $ & {\textbackslash}{\textasciicircum}6 & Superscript Six / Superscript Digit Six \\ \hline
U+02077 & $ ⁷ $ & {\textbackslash}{\textasciicircum}7 & Superscript Seven / Superscript Digit Seven \\ \hline
U+02078 & $ ⁸ $ & {\textbackslash}{\textasciicircum}8 & Superscript Eight / Superscript Digit Eight \\ \hline
U+02079 & $ ⁹ $ & {\textbackslash}{\textasciicircum}9 & Superscript Nine / Superscript Digit Nine \\ \hline
U+0207A & $ ⁺ $ & {\textbackslash}{\textasciicircum}+ & Superscript Plus Sign \\ \hline
U+0207B & $ ⁻ $ & {\textbackslash}{\textasciicircum}- & Superscript Minus / Superscript Hyphen-Minus \\ \hline
U+0207C & $ ⁼ $ & {\textbackslash}{\textasciicircum}= & Superscript Equals Sign \\ \hline
U+0207D & $ ⁽ $ & {\textbackslash}{\textasciicircum}( & Superscript Left Parenthesis / Superscript Opening Parenthesis \\ \hline
U+0207E & $ ⁾ $ & {\textbackslash}{\textasciicircum}) & Superscript Right Parenthesis / Superscript Closing Parenthesis \\ \hline
U+0207F & $ ⁿ $ & {\textbackslash}{\textasciicircum}n & Superscript Latin Small Letter N \\ \hline
U+02080 & $ ₀ $ & {\textbackslash}\_0 & Subscript Zero / Subscript Digit Zero \\ \hline
U+02081 & $ ₁ $ & {\textbackslash}\_1 & Subscript One / Subscript Digit One \\ \hline
U+02082 & $ ₂ $ & {\textbackslash}\_2 & Subscript Two / Subscript Digit Two \\ \hline
U+02083 & $ ₃ $ & {\textbackslash}\_3 & Subscript Three / Subscript Digit Three \\ \hline
U+02084 & $ ₄ $ & {\textbackslash}\_4 & Subscript Four / Subscript Digit Four \\ \hline
U+02085 & $ ₅ $ & {\textbackslash}\_5 & Subscript Five / Subscript Digit Five \\ \hline
U+02086 & $ ₆ $ & {\textbackslash}\_6 & Subscript Six / Subscript Digit Six \\ \hline
U+02087 & $ ₇ $ & {\textbackslash}\_7 & Subscript Seven / Subscript Digit Seven \\ \hline
U+02088 & $ ₈ $ & {\textbackslash}\_8 & Subscript Eight / Subscript Digit Eight \\ \hline
U+02089 & $ ₉ $ & {\textbackslash}\_9 & Subscript Nine / Subscript Digit Nine \\ \hline
U+0208A & $ ₊ $ & {\textbackslash}\_+ & Subscript Plus Sign \\ \hline
U+0208B & $ ₋ $ & {\textbackslash}\_- & Subscript Minus / Subscript Hyphen-Minus \\ \hline
U+0208C & $ ₌ $ & {\textbackslash}\_= & Subscript Equals Sign \\ \hline
U+0208D & $ ₍ $ & {\textbackslash}\_( & Subscript Left Parenthesis / Subscript Opening Parenthesis \\ \hline
U+0208E & $ ₎ $ & {\textbackslash}\_) & Subscript Right Parenthesis / Subscript Closing Parenthesis \\ \hline
U+02090 & $ ₐ $ & {\textbackslash}\_a & Latin Subscript Small Letter A \\ \hline
U+02091 & $ ₑ $ & {\textbackslash}\_e & Latin Subscript Small Letter E \\ \hline
U+02092 & $ ₒ $ & {\textbackslash}\_o & Latin Subscript Small Letter O \\ \hline
U+02093 & $ ₓ $ & {\textbackslash}\_x & Latin Subscript Small Letter X \\ \hline
U+02094 & {\MathSymFontTwo ₔ} & {\textbackslash}\_schwa & Latin Subscript Small Letter Schwa \\ \hline
U+02095 & $ ₕ $ & {\textbackslash}\_h & Latin Subscript Small Letter H \\ \hline
U+02096 & $ ₖ $ & {\textbackslash}\_k & Latin Subscript Small Letter K \\ \hline
U+02097 & $ ₗ $ & {\textbackslash}\_l & Latin Subscript Small Letter L \\ \hline
U+02098 & $ ₘ $ & {\textbackslash}\_m & Latin Subscript Small Letter M \\ \hline
U+02099 & $ ₙ $ & {\textbackslash}\_n & Latin Subscript Small Letter N \\ \hline
U+0209A & $ ₚ $ & {\textbackslash}\_p & Latin Subscript Small Letter P \\ \hline
U+0209B & $ ₛ $ & {\textbackslash}\_s & Latin Subscript Small Letter S \\ \hline
U+0209C & $ ₜ $ & {\textbackslash}\_t & Latin Subscript Small Letter T \\ \hline
U+020A7 & $ ₧ $ & {\textbackslash}pes & Peseta Sign \\ \hline
U+020AC & $ € $ & {\textbackslash}euro & Euro Sign \\ \hline
U+020D0 & {\MathSymFontOne  ⃐ } & {\textbackslash}leftharpoonaccent & Combining Left Harpoon Above / Non-Spacing Left Harpoon Above \\ \hline
U+020D1 & {\MathSymFontOne  ⃑ } & {\textbackslash}rightharpoonaccent & Combining Right Harpoon Above / Non-Spacing Right Harpoon Above \\ \hline
U+020D2 & {\MathSymFontOne  ⃒ } & {\textbackslash}vertoverlay & Combining Long Vertical Line Overlay / Non-Spacing Long Vertical Bar Overlay \\ \hline
U+020D6 & {\MathSymFontOne  ⃖ } & {\textbackslash}overleftarrow & Combining Left Arrow Above / Non-Spacing Left Arrow Above \\ \hline
U+020D7 & {\MathSymFontOne  ⃗ } & {\textbackslash}vec & Combining Right Arrow Above / Non-Spacing Right Arrow Above \\ \hline
U+020DB & {\MathSymFontOne  ⃛ } & {\textbackslash}dddot & Combining Three Dots Above / Non-Spacing Three Dots Above \\ \hline
U+020DC & {\MathSymFontOne  ⃜ } & {\textbackslash}ddddot & Combining Four Dots Above / Non-Spacing Four Dots Above \\ \hline
U+020DD & {\MathSymFontOne  ⃝ } & {\textbackslash}enclosecircle & Combining Enclosing Circle / Enclosing Circle \\ \hline
U+020DE & {\MathSymFontOne  ⃞ } & {\textbackslash}enclosesquare & Combining Enclosing Square / Enclosing Square \\ \hline
U+020DF & {\MathSymFontOne  ⃟ } & {\textbackslash}enclosediamond & Combining Enclosing Diamond / Enclosing Diamond \\ \hline
U+020E1 & {\MathSymFontOne  ⃡ } & {\textbackslash}overleftrightarrow & Combining Left Right Arrow Above / Non-Spacing Left Right Arrow Above \\ \hline
U+020E4 & {\MathSymFontOne  ⃤ } & {\textbackslash}enclosetriangle & Combining Enclosing Upward Pointing Triangle \\ \hline
U+020E7 & {\MathSymFontOne  ⃧ } & {\textbackslash}annuity & Combining Annuity Symbol \\ \hline
U+020E8 & {\MathSymFontOne  ⃨ } & {\textbackslash}threeunderdot & Combining Triple Underdot \\ \hline
U+020E9 & {\MathSymFontOne  ⃩ } & {\textbackslash}widebridgeabove & Combining Wide Bridge Above \\ \hline
U+020EC & {\MathSymFontOne  ⃬ } & {\textbackslash}underrightharpoondown & Combining Rightwards Harpoon With Barb Downwards \\ \hline
U+020ED & {\MathSymFontOne  ⃭ } & {\textbackslash}underleftharpoondown & Combining Leftwards Harpoon With Barb Downwards \\ \hline
U+020EE & {\MathSymFontOne  ⃮ } & {\textbackslash}underleftarrow & Combining Left Arrow Below \\ \hline
U+020EF & {\MathSymFontOne  ⃯ } & {\textbackslash}underrightarrow & Combining Right Arrow Below \\ \hline
U+020F0 & {\MathSymFontOne  ⃰ } & {\textbackslash}asteraccent & Combining Asterisk Above \\ \hline
U+02102 & $ ℂ $ & {\textbackslash}bbC & Double-Struck Capital C / Double-Struck C \\ \hline
U+02107 & $ ℇ $ & {\textbackslash}eulermascheroni & Euler Constant / Eulers \\ \hline
U+0210A & $ ℊ $ & {\textbackslash}scrg & Script Small G \\ \hline
U+0210B & $ ℋ $ & {\textbackslash}scrH & Script Capital H / Script H \\ \hline
U+0210C & $ ℌ $ & {\textbackslash}frakH & Black-Letter Capital H / Black-Letter H \\ \hline
U+0210D & $ ℍ $ & {\textbackslash}bbH & Double-Struck Capital H / Double-Struck H \\ \hline
U+0210E & $ ℎ $ & {\textbackslash}planck & Planck Constant \\ \hline
U+0210F & $ ℏ $ & {\textbackslash}hslash & Planck Constant Over Two Pi / Planck Constant Over 2 Pi \\ \hline
U+02110 & $ ℐ $ & {\textbackslash}scrI & Script Capital I / Script I \\ \hline
U+02111 & $ ℑ $ & {\textbackslash}Im & Black-Letter Capital I / Black-Letter I \\ \hline
U+02112 & $ ℒ $ & {\textbackslash}scrL & Script Capital L / Script L \\ \hline
U+02113 & $ ℓ $ & {\textbackslash}ell & Script Small L \\ \hline
U+02115 & $ ℕ $ & {\textbackslash}bbN & Double-Struck Capital N / Double-Struck N \\ \hline
U+02116 & $ № $ & {\textbackslash}numero & Numero Sign / Numero \\ \hline
U+02118 & $ ℘ $ & {\textbackslash}wp & Script Capital P / Script P \\ \hline
U+02119 & $ ℙ $ & {\textbackslash}bbP & Double-Struck Capital P / Double-Struck P \\ \hline
U+0211A & $ ℚ $ & {\textbackslash}bbQ & Double-Struck Capital Q / Double-Struck Q \\ \hline
U+0211B & $ ℛ $ & {\textbackslash}scrR & Script Capital R / Script R \\ \hline
U+0211C & $ ℜ $ & {\textbackslash}Re & Black-Letter Capital R / Black-Letter R \\ \hline
U+0211D & $ ℝ $ & {\textbackslash}bbR & Double-Struck Capital R / Double-Struck R \\ \hline
U+0211E & $ ℞ $ & {\textbackslash}xrat & Prescription Take \\ \hline
U+02122 & {\EmojiFont ™} & {\textbackslash}:tm:, {\textbackslash}trademark & Trade Mark Sign / Trademark \\ \hline
U+02124 & $ ℤ $ & {\textbackslash}bbZ & Double-Struck Capital Z / Double-Struck Z \\ \hline
U+02126 & {\MathSymFontOne Ω} & {\textbackslash}ohm & Ohm Sign / Ohm \\ \hline
U+02127 & $ ℧ $ & {\textbackslash}mho & Inverted Ohm Sign / Mho \\ \hline
U+02128 & $ ℨ $ & {\textbackslash}frakZ & Black-Letter Capital Z / Black-Letter Z \\ \hline
U+02129 & $ ℩ $ & {\textbackslash}turnediota & Turned Greek Small Letter Iota \\ \hline
U+0212B & $ Å $ & {\textbackslash}Angstrom & Angstrom Sign / Angstrom Unit \\ \hline
U+0212C & $ ℬ $ & {\textbackslash}scrB & Script Capital B / Script B \\ \hline
U+0212D & $ ℭ $ & {\textbackslash}frakC & Black-Letter Capital C / Black-Letter C \\ \hline
U+0212F & $ ℯ $ & {\textbackslash}scre, {\textbackslash}euler & Script Small E \\ \hline
U+02130 & $ ℰ $ & {\textbackslash}scrE & Script Capital E / Script E \\ \hline
U+02131 & $ ℱ $ & {\textbackslash}scrF & Script Capital F / Script F \\ \hline
U+02132 & $ Ⅎ $ & {\textbackslash}Finv & Turned Capital F / Turned F \\ \hline
U+02133 & $ ℳ $ & {\textbackslash}scrM & Script Capital M / Script M \\ \hline
U+02134 & $ ℴ $ & {\textbackslash}scro & Script Small O \\ \hline
U+02135 & $ ℵ $ & {\textbackslash}aleph & Alef Symbol / First Transfinite Cardinal \\ \hline
U+02136 & $ ℶ $ & {\textbackslash}beth & Bet Symbol / Second Transfinite Cardinal \\ \hline
U+02137 & $ ℷ $ & {\textbackslash}gimel & Gimel Symbol / Third Transfinite Cardinal \\ \hline
U+02138 & $ ℸ $ & {\textbackslash}daleth & Dalet Symbol / Fourth Transfinite Cardinal \\ \hline
U+02139 & {\EmojiFont ℹ} & {\textbackslash}:information\_source: & Information Source \\ \hline
U+0213C & $ ℼ $ & {\textbackslash}bbpi & Double-Struck Small Pi \\ \hline
U+0213D & $ ℽ $ & {\textbackslash}bbgamma & Double-Struck Small Gamma \\ \hline
U+0213E & $ ℾ $ & {\textbackslash}bbGamma & Double-Struck Capital Gamma \\ \hline
U+0213F & $ ℿ $ & {\textbackslash}bbPi & Double-Struck Capital Pi \\ \hline
U+02140 & $ ⅀ $ & {\textbackslash}bbsum & Double-Struck N-Ary Summation \\ \hline
U+02141 & $ ⅁ $ & {\textbackslash}Game & Turned Sans-Serif Capital G \\ \hline
U+02142 & $ ⅂ $ & {\textbackslash}sansLturned & Turned Sans-Serif Capital L \\ \hline
U+02143 & $ ⅃ $ & {\textbackslash}sansLmirrored & Reversed Sans-Serif Capital L \\ \hline
U+02144 & $ ⅄ $ & {\textbackslash}Yup & Turned Sans-Serif Capital Y \\ \hline
U+02145 & $ ⅅ $ & {\textbackslash}bbiD & Double-Struck Italic Capital D \\ \hline
U+02146 & $ ⅆ $ & {\textbackslash}bbid & Double-Struck Italic Small D \\ \hline
U+02147 & $ ⅇ $ & {\textbackslash}bbie & Double-Struck Italic Small E \\ \hline
U+02148 & $ ⅈ $ & {\textbackslash}bbii & Double-Struck Italic Small I \\ \hline
U+02149 & $ ⅉ $ & {\textbackslash}bbij & Double-Struck Italic Small J \\ \hline
U+0214A & $ ⅊ $ & {\textbackslash}PropertyLine & Property Line \\ \hline
U+0214B & $ ⅋ $ & {\textbackslash}upand & Turned Ampersand \\ \hline
U+02150 & $ ⅐ $ & {\textbackslash}1/7 & Vulgar Fraction One Seventh \\ \hline
U+02151 & $ ⅑ $ & {\textbackslash}1/9 & Vulgar Fraction One Ninth \\ \hline
U+02152 & $ ⅒ $ & {\textbackslash}1/10 & Vulgar Fraction One Tenth \\ \hline
U+02153 & $ ⅓ $ & {\textbackslash}1/3 & Vulgar Fraction One Third / Fraction One Third \\ \hline
U+02154 & $ ⅔ $ & {\textbackslash}2/3 & Vulgar Fraction Two Thirds / Fraction Two Thirds \\ \hline
U+02155 & $ ⅕ $ & {\textbackslash}1/5 & Vulgar Fraction One Fifth / Fraction One Fifth \\ \hline
U+02156 & $ ⅖ $ & {\textbackslash}2/5 & Vulgar Fraction Two Fifths / Fraction Two Fifths \\ \hline
U+02157 & $ ⅗ $ & {\textbackslash}3/5 & Vulgar Fraction Three Fifths / Fraction Three Fifths \\ \hline
U+02158 & $ ⅘ $ & {\textbackslash}4/5 & Vulgar Fraction Four Fifths / Fraction Four Fifths \\ \hline
U+02159 & $ ⅙ $ & {\textbackslash}1/6 & Vulgar Fraction One Sixth / Fraction One Sixth \\ \hline
U+0215A & $ ⅚ $ & {\textbackslash}5/6 & Vulgar Fraction Five Sixths / Fraction Five Sixths \\ \hline
U+0215B & $ ⅛ $ & {\textbackslash}1/8 & Vulgar Fraction One Eighth / Fraction One Eighth \\ \hline
U+0215C & $ ⅜ $ & {\textbackslash}3/8 & Vulgar Fraction Three Eighths / Fraction Three Eighths \\ \hline
U+0215D & $ ⅝ $ & {\textbackslash}5/8 & Vulgar Fraction Five Eighths / Fraction Five Eighths \\ \hline
U+0215E & $ ⅞ $ & {\textbackslash}7/8 & Vulgar Fraction Seven Eighths / Fraction Seven Eighths \\ \hline
U+0215F & {\MathSymFontTwo ⅟} & {\textbackslash}1/ & Fraction Numerator One \\ \hline
U+02189 & $ ↉ $ & {\textbackslash}0/3 & Vulgar Fraction Zero Thirds \\ \hline
U+02190 & $ ← $ & {\textbackslash}leftarrow & Leftwards Arrow / Left Arrow \\ \hline
U+02191 & $ ↑ $ & {\textbackslash}uparrow & Upwards Arrow / Up Arrow \\ \hline
U+02192 & $ → $ & {\textbackslash}to, {\textbackslash}rightarrow & Rightwards Arrow / Right Arrow \\ \hline
U+02193 & $ ↓ $ & {\textbackslash}downarrow & Downwards Arrow / Down Arrow \\ \hline
U+02194 & {\EmojiFont ↔} & {\textbackslash}:left\_right\_arrow:, {\textbackslash}leftrightarrow & Left Right Arrow \\ \hline
U+02195 & {\EmojiFont ↕} & {\textbackslash}:arrow\_up\_down:, {\textbackslash}updownarrow & Up Down Arrow \\ \hline
U+02196 & {\EmojiFont ↖} & {\textbackslash}:arrow\_upper\_left:, {\textbackslash}nwarrow & North West Arrow / Upper Left Arrow \\ \hline
U+02197 & {\EmojiFont ↗} & {\textbackslash}:arrow\_upper\_right:, {\textbackslash}nearrow & North East Arrow / Upper Right Arrow \\ \hline
U+02198 & {\EmojiFont ↘} & {\textbackslash}:arrow\_lower\_right:, {\textbackslash}searrow & South East Arrow / Lower Right Arrow \\ \hline
U+02199 & {\EmojiFont ↙} & {\textbackslash}:arrow\_lower\_left:, {\textbackslash}swarrow & South West Arrow / Lower Left Arrow \\ \hline
U+0219A & $ ↚ $ & {\textbackslash}nleftarrow & Leftwards Arrow With Stroke / Left Arrow With Stroke \\ \hline
U+0219B & $ ↛ $ & {\textbackslash}nrightarrow & Rightwards Arrow With Stroke / Right Arrow With Stroke \\ \hline
U+0219C & $ ↜ $ & {\textbackslash}leftwavearrow & Leftwards Wave Arrow / Left Wave Arrow \\ \hline
U+0219D & $ ↝ $ & {\textbackslash}rightwavearrow & Rightwards Wave Arrow / Right Wave Arrow \\ \hline
U+0219E & $ ↞ $ & {\textbackslash}twoheadleftarrow & Leftwards Two Headed Arrow / Left Two Headed Arrow \\ \hline
U+0219F & $ ↟ $ & {\textbackslash}twoheaduparrow & Upwards Two Headed Arrow / Up Two Headed Arrow \\ \hline
U+021A0 & $ ↠ $ & {\textbackslash}twoheadrightarrow & Rightwards Two Headed Arrow / Right Two Headed Arrow \\ \hline
U+021A1 & $ ↡ $ & {\textbackslash}twoheaddownarrow & Downwards Two Headed Arrow / Down Two Headed Arrow \\ \hline
U+021A2 & $ ↢ $ & {\textbackslash}leftarrowtail & Leftwards Arrow With Tail / Left Arrow With Tail \\ \hline
U+021A3 & $ ↣ $ & {\textbackslash}rightarrowtail & Rightwards Arrow With Tail / Right Arrow With Tail \\ \hline
U+021A4 & $ ↤ $ & {\textbackslash}mapsfrom & Leftwards Arrow From Bar / Left Arrow From Bar \\ \hline
U+021A5 & $ ↥ $ & {\textbackslash}mapsup & Upwards Arrow From Bar / Up Arrow From Bar \\ \hline
U+021A6 & $ ↦ $ & {\textbackslash}mapsto & Rightwards Arrow From Bar / Right Arrow From Bar \\ \hline
U+021A7 & $ ↧ $ & {\textbackslash}mapsdown & Downwards Arrow From Bar / Down Arrow From Bar \\ \hline
U+021A8 & $ ↨ $ & {\textbackslash}updownarrowbar & Up Down Arrow With Base \\ \hline
U+021A9 & {\EmojiFont ↩} & {\textbackslash}:leftwards\_arrow\_with\_hook:, {\textbackslash}hookleftarrow & Leftwards Arrow With Hook / Left Arrow With Hook \\ \hline
U+021AA & {\EmojiFont ↪} & {\textbackslash}:arrow\_right\_hook:, {\textbackslash}hookrightarrow & Rightwards Arrow With Hook / Right Arrow With Hook \\ \hline
U+021AB & $ ↫ $ & {\textbackslash}looparrowleft & Leftwards Arrow With Loop / Left Arrow With Loop \\ \hline
U+021AC & $ ↬ $ & {\textbackslash}looparrowright & Rightwards Arrow With Loop / Right Arrow With Loop \\ \hline
U+021AD & $ ↭ $ & {\textbackslash}leftrightsquigarrow & Left Right Wave Arrow \\ \hline
U+021AE & $ ↮ $ & {\textbackslash}nleftrightarrow & Left Right Arrow With Stroke \\ \hline
U+021AF & $ ↯ $ & {\textbackslash}downzigzagarrow & Downwards Zigzag Arrow / Down Zigzag Arrow \\ \hline
U+021B0 & $ ↰ $ & {\textbackslash}Lsh & Upwards Arrow With Tip Leftwards / Up Arrow With Tip Left \\ \hline
U+021B1 & $ ↱ $ & {\textbackslash}Rsh & Upwards Arrow With Tip Rightwards / Up Arrow With Tip Right \\ \hline
U+021B2 & $ ↲ $ & {\textbackslash}Ldsh & Downwards Arrow With Tip Leftwards / Down Arrow With Tip Left \\ \hline
U+021B3 & $ ↳ $ & {\textbackslash}Rdsh & Downwards Arrow With Tip Rightwards / Down Arrow With Tip Right \\ \hline
U+021B4 & $ ↴ $ & {\textbackslash}linefeed & Rightwards Arrow With Corner Downwards / Right Arrow With Corner Down \\ \hline
U+021B5 & $ ↵ $ & {\textbackslash}carriagereturn & Downwards Arrow With Corner Leftwards / Down Arrow With Corner Left \\ \hline
U+021B6 & $ ↶ $ & {\textbackslash}curvearrowleft & Anticlockwise Top Semicircle Arrow \\ \hline
U+021B7 & $ ↷ $ & {\textbackslash}curvearrowright & Clockwise Top Semicircle Arrow \\ \hline
U+021B8 & $ ↸ $ & {\textbackslash}barovernorthwestarrow & North West Arrow To Long Bar / Upper Left Arrow To Long Bar \\ \hline
U+021B9 & $ ↹ $ & {\textbackslash}barleftarrowrightarrowbar & Leftwards Arrow To Bar Over Rightwards Arrow To Bar / Left Arrow To Bar Over Right Arrow To Bar \\ \hline
U+021BA & $ ↺ $ & {\textbackslash}circlearrowleft & Anticlockwise Open Circle Arrow \\ \hline
U+021BB & $ ↻ $ & {\textbackslash}circlearrowright & Clockwise Open Circle Arrow \\ \hline
U+021BC & $ ↼ $ & {\textbackslash}leftharpoonup & Leftwards Harpoon With Barb Upwards / Left Harpoon With Barb Up \\ \hline
U+021BD & $ ↽ $ & {\textbackslash}leftharpoondown & Leftwards Harpoon With Barb Downwards / Left Harpoon With Barb Down \\ \hline
U+021BE & $ ↾ $ & {\textbackslash}upharpoonright & Upwards Harpoon With Barb Rightwards / Up Harpoon With Barb Right \\ \hline
U+021BF & $ ↿ $ & {\textbackslash}upharpoonleft & Upwards Harpoon With Barb Leftwards / Up Harpoon With Barb Left \\ \hline
U+021C0 & $ ⇀ $ & {\textbackslash}rightharpoonup & Rightwards Harpoon With Barb Upwards / Right Harpoon With Barb Up \\ \hline
U+021C1 & $ ⇁ $ & {\textbackslash}rightharpoondown & Rightwards Harpoon With Barb Downwards / Right Harpoon With Barb Down \\ \hline
U+021C2 & $ ⇂ $ & {\textbackslash}downharpoonright & Downwards Harpoon With Barb Rightwards / Down Harpoon With Barb Right \\ \hline
U+021C3 & $ ⇃ $ & {\textbackslash}downharpoonleft & Downwards Harpoon With Barb Leftwards / Down Harpoon With Barb Left \\ \hline
U+021C4 & $ ⇄ $ & {\textbackslash}rightleftarrows & Rightwards Arrow Over Leftwards Arrow / Right Arrow Over Left Arrow \\ \hline
U+021C5 & $ ⇅ $ & {\textbackslash}dblarrowupdown & Upwards Arrow Leftwards Of Downwards Arrow / Up Arrow Left Of Down Arrow \\ \hline
U+021C6 & $ ⇆ $ & {\textbackslash}leftrightarrows & Leftwards Arrow Over Rightwards Arrow / Left Arrow Over Right Arrow \\ \hline
U+021C7 & $ ⇇ $ & {\textbackslash}leftleftarrows & Leftwards Paired Arrows / Left Paired Arrows \\ \hline
U+021C8 & $ ⇈ $ & {\textbackslash}upuparrows & Upwards Paired Arrows / Up Paired Arrows \\ \hline
U+021C9 & $ ⇉ $ & {\textbackslash}rightrightarrows & Rightwards Paired Arrows / Right Paired Arrows \\ \hline
U+021CA & $ ⇊ $ & {\textbackslash}downdownarrows & Downwards Paired Arrows / Down Paired Arrows \\ \hline
U+021CB & $ ⇋ $ & {\textbackslash}leftrightharpoons & Leftwards Harpoon Over Rightwards Harpoon / Left Harpoon Over Right Harpoon \\ \hline
U+021CC & $ ⇌ $ & {\textbackslash}rightleftharpoons & Rightwards Harpoon Over Leftwards Harpoon / Right Harpoon Over Left Harpoon \\ \hline
U+021CD & $ ⇍ $ & {\textbackslash}nLeftarrow & Leftwards Double Arrow With Stroke / Left Double Arrow With Stroke \\ \hline
U+021CE & $ ⇎ $ & {\textbackslash}nLeftrightarrow & Left Right Double Arrow With Stroke \\ \hline
U+021CF & $ ⇏ $ & {\textbackslash}nRightarrow & Rightwards Double Arrow With Stroke / Right Double Arrow With Stroke \\ \hline
U+021D0 & $ ⇐ $ & {\textbackslash}Leftarrow & Leftwards Double Arrow / Left Double Arrow \\ \hline
U+021D1 & $ ⇑ $ & {\textbackslash}Uparrow & Upwards Double Arrow / Up Double Arrow \\ \hline
U+021D2 & $ ⇒ $ & {\textbackslash}Rightarrow & Rightwards Double Arrow / Right Double Arrow \\ \hline
U+021D3 & $ ⇓ $ & {\textbackslash}Downarrow & Downwards Double Arrow / Down Double Arrow \\ \hline
U+021D4 & $ ⇔ $ & {\textbackslash}Leftrightarrow & Left Right Double Arrow \\ \hline
U+021D5 & $ ⇕ $ & {\textbackslash}Updownarrow & Up Down Double Arrow \\ \hline
U+021D6 & $ ⇖ $ & {\textbackslash}Nwarrow & North West Double Arrow / Upper Left Double Arrow \\ \hline
U+021D7 & $ ⇗ $ & {\textbackslash}Nearrow & North East Double Arrow / Upper Right Double Arrow \\ \hline
U+021D8 & $ ⇘ $ & {\textbackslash}Searrow & South East Double Arrow / Lower Right Double Arrow \\ \hline
U+021D9 & $ ⇙ $ & {\textbackslash}Swarrow & South West Double Arrow / Lower Left Double Arrow \\ \hline
U+021DA & $ ⇚ $ & {\textbackslash}Lleftarrow & Leftwards Triple Arrow / Left Triple Arrow \\ \hline
U+021DB & $ ⇛ $ & {\textbackslash}Rrightarrow & Rightwards Triple Arrow / Right Triple Arrow \\ \hline
U+021DC & $ ⇜ $ & {\textbackslash}leftsquigarrow & Leftwards Squiggle Arrow / Left Squiggle Arrow \\ \hline
U+021DD & $ ⇝ $ & {\textbackslash}rightsquigarrow & Rightwards Squiggle Arrow / Right Squiggle Arrow \\ \hline
U+021DE & $ ⇞ $ & {\textbackslash}nHuparrow & Upwards Arrow With Double Stroke / Up Arrow With Double Stroke \\ \hline
U+021DF & $ ⇟ $ & {\textbackslash}nHdownarrow & Downwards Arrow With Double Stroke / Down Arrow With Double Stroke \\ \hline
U+021E0 & $ ⇠ $ & {\textbackslash}leftdasharrow & Leftwards Dashed Arrow / Left Dashed Arrow \\ \hline
U+021E1 & $ ⇡ $ & {\textbackslash}updasharrow & Upwards Dashed Arrow / Up Dashed Arrow \\ \hline
U+021E2 & $ ⇢ $ & {\textbackslash}rightdasharrow & Rightwards Dashed Arrow / Right Dashed Arrow \\ \hline
U+021E3 & $ ⇣ $ & {\textbackslash}downdasharrow & Downwards Dashed Arrow / Down Dashed Arrow \\ \hline
U+021E4 & $ ⇤ $ & {\textbackslash}barleftarrow & Leftwards Arrow To Bar / Left Arrow To Bar \\ \hline
U+021E5 & $ ⇥ $ & {\textbackslash}rightarrowbar & Rightwards Arrow To Bar / Right Arrow To Bar \\ \hline
U+021E6 & $ ⇦ $ & {\textbackslash}leftwhitearrow & Leftwards White Arrow / White Left Arrow \\ \hline
U+021E7 & $ ⇧ $ & {\textbackslash}upwhitearrow & Upwards White Arrow / White Up Arrow \\ \hline
U+021E8 & $ ⇨ $ & {\textbackslash}rightwhitearrow & Rightwards White Arrow / White Right Arrow \\ \hline
U+021E9 & $ ⇩ $ & {\textbackslash}downwhitearrow & Downwards White Arrow / White Down Arrow \\ \hline
U+021EA & $ ⇪ $ & {\textbackslash}whitearrowupfrombar & Upwards White Arrow From Bar / White Up Arrow From Bar \\ \hline
U+021F4 & $ ⇴ $ & {\textbackslash}circleonrightarrow & Right Arrow With Small Circle \\ \hline
U+021F5 & $ ⇵ $ & {\textbackslash}DownArrowUpArrow & Downwards Arrow Leftwards Of Upwards Arrow \\ \hline
U+021F6 & $ ⇶ $ & {\textbackslash}rightthreearrows & Three Rightwards Arrows \\ \hline
U+021F7 & $ ⇷ $ & {\textbackslash}nvleftarrow & Leftwards Arrow With Vertical Stroke \\ \hline
U+021F8 & $ ⇸ $ & {\textbackslash}nvrightarrow & Rightwards Arrow With Vertical Stroke \\ \hline
U+021F9 & $ ⇹ $ & {\textbackslash}nvleftrightarrow & Left Right Arrow With Vertical Stroke \\ \hline
U+021FA & $ ⇺ $ & {\textbackslash}nVleftarrow & Leftwards Arrow With Double Vertical Stroke \\ \hline
U+021FB & $ ⇻ $ & {\textbackslash}nVrightarrow & Rightwards Arrow With Double Vertical Stroke \\ \hline
U+021FC & $ ⇼ $ & {\textbackslash}nVleftrightarrow & Left Right Arrow With Double Vertical Stroke \\ \hline
U+021FD & $ ⇽ $ & {\textbackslash}leftarrowtriangle & Leftwards Open-Headed Arrow \\ \hline
U+021FE & $ ⇾ $ & {\textbackslash}rightarrowtriangle & Rightwards Open-Headed Arrow \\ \hline
U+021FF & $ ⇿ $ & {\textbackslash}leftrightarrowtriangle & Left Right Open-Headed Arrow \\ \hline
U+02200 & $ ∀ $ & {\textbackslash}forall & For All \\ \hline
U+02201 & $ ∁ $ & {\textbackslash}complement & Complement \\ \hline
U+02202 & $ ∂ $ & {\textbackslash}partial & Partial Differential \\ \hline
U+02203 & $ ∃ $ & {\textbackslash}exists & There Exists \\ \hline
U+02204 & $ ∄ $ & {\textbackslash}nexists & There Does Not Exist \\ \hline
U+02205 & $ ∅ $ & {\textbackslash}varnothing, {\textbackslash}emptyset & Empty Set \\ \hline
U+02206 & $ ∆ $ & {\textbackslash}increment & Increment \\ \hline
U+02207 & $ ∇ $ & {\textbackslash}del, {\textbackslash}nabla & Nabla \\ \hline
U+02208 & $ ∈ $ & {\textbackslash}in & Element Of \\ \hline
U+02209 & $ ∉ $ & {\textbackslash}notin & Not An Element Of \\ \hline
U+0220A & $ ∊ $ & {\textbackslash}smallin & Small Element Of \\ \hline
U+0220B & $ ∋ $ & {\textbackslash}ni & Contains As Member \\ \hline
U+0220C & $ ∌ $ & {\textbackslash}nni & Does Not Contain As Member \\ \hline
U+0220D & $ ∍ $ & {\textbackslash}smallni & Small Contains As Member \\ \hline
U+0220E & $ ∎ $ & {\textbackslash}QED & End Of Proof \\ \hline
U+0220F & $ ∏ $ & {\textbackslash}prod & N-Ary Product \\ \hline
U+02210 & $ ∐ $ & {\textbackslash}coprod & N-Ary Coproduct \\ \hline
U+02211 & $ ∑ $ & {\textbackslash}sum & N-Ary Summation \\ \hline
U+02212 & $ − $ & {\textbackslash}minus & Minus Sign \\ \hline
U+02213 & $ ∓ $ & {\textbackslash}mp & Minus-Or-Plus Sign \\ \hline
U+02214 & $ ∔ $ & {\textbackslash}dotplus & Dot Plus \\ \hline
U+02216 & $ ∖ $ & {\textbackslash}setminus & Set Minus \\ \hline
U+02217 & $ ∗ $ & {\textbackslash}ast & Asterisk Operator \\ \hline
U+02218 & $ ∘ $ & {\textbackslash}circ & Ring Operator \\ \hline
U+02219 & $ ∙ $ & {\textbackslash}vysmblkcircle & Bullet Operator \\ \hline
U+0221A & $ √ $ & {\textbackslash}surd, {\textbackslash}sqrt & Square Root \\ \hline
U+0221B & $ ∛ $ & {\textbackslash}cbrt & Cube Root \\ \hline
U+0221C & $ ∜ $ & {\textbackslash}fourthroot & Fourth Root \\ \hline
U+0221D & $ ∝ $ & {\textbackslash}propto & Proportional To \\ \hline
U+0221E & $ ∞ $ & {\textbackslash}infty & Infinity \\ \hline
U+0221F & $ ∟ $ & {\textbackslash}rightangle & Right Angle \\ \hline
U+02220 & $ ∠ $ & {\textbackslash}angle & Angle \\ \hline
U+02221 & $ ∡ $ & {\textbackslash}measuredangle & Measured Angle \\ \hline
U+02222 & $ ∢ $ & {\textbackslash}sphericalangle & Spherical Angle \\ \hline
U+02223 & $ ∣ $ & {\textbackslash}mid & Divides \\ \hline
U+02224 & $ ∤ $ & {\textbackslash}nmid & Does Not Divide \\ \hline
U+02225 & $ ∥ $ & {\textbackslash}parallel & Parallel To \\ \hline
U+02226 & $ ∦ $ & {\textbackslash}nparallel & Not Parallel To \\ \hline
U+02227 & $ ∧ $ & {\textbackslash}wedge & Logical And \\ \hline
U+02228 & $ ∨ $ & {\textbackslash}vee & Logical Or \\ \hline
U+02229 & $ ∩ $ & {\textbackslash}cap & Intersection \\ \hline
U+0222A & $ ∪ $ & {\textbackslash}cup & Union \\ \hline
U+0222B & $ ∫ $ & {\textbackslash}int & Integral \\ \hline
U+0222C & $ ∬ $ & {\textbackslash}iint & Double Integral \\ \hline
U+0222D & $ ∭ $ & {\textbackslash}iiint & Triple Integral \\ \hline
U+0222E & $ ∮ $ & {\textbackslash}oint & Contour Integral \\ \hline
U+0222F & $ ∯ $ & {\textbackslash}oiint & Surface Integral \\ \hline
U+02230 & $ ∰ $ & {\textbackslash}oiiint & Volume Integral \\ \hline
U+02231 & $ ∱ $ & {\textbackslash}clwintegral & Clockwise Integral \\ \hline
U+02232 & $ ∲ $ & {\textbackslash}varointclockwise & Clockwise Contour Integral \\ \hline
U+02233 & $ ∳ $ & {\textbackslash}ointctrclockwise & Anticlockwise Contour Integral \\ \hline
U+02234 & $ ∴ $ & {\textbackslash}therefore & Therefore \\ \hline
U+02235 & $ ∵ $ & {\textbackslash}because & Because \\ \hline
U+02237 & $ ∷ $ & {\textbackslash}Colon & Proportion \\ \hline
U+02238 & $ ∸ $ & {\textbackslash}dotminus & Dot Minus \\ \hline
U+0223A & $ ∺ $ & {\textbackslash}dotsminusdots & Geometric Proportion \\ \hline
U+0223B & $ ∻ $ & {\textbackslash}kernelcontraction & Homothetic \\ \hline
U+0223C & $ ∼ $ & {\textbackslash}sim & Tilde Operator \\ \hline
U+0223D & $ ∽ $ & {\textbackslash}backsim & Reversed Tilde \\ \hline
U+0223E & $ ∾ $ & {\textbackslash}lazysinv & Inverted Lazy S \\ \hline
U+0223F & $ ∿ $ & {\textbackslash}sinewave & Sine Wave \\ \hline
U+02240 & $ ≀ $ & {\textbackslash}wr & Wreath Product \\ \hline
U+02241 & $ ≁ $ & {\textbackslash}nsim & Not Tilde \\ \hline
U+02242 & $ ≂ $ & {\textbackslash}eqsim & Minus Tilde \\ \hline
U+02242 + U+00338 & $ ≂̸ $ & {\textbackslash}neqsim & Minus Tilde + Combining Long Solidus Overlay / Non-Spacing Long Slash Overlay \\ \hline
U+02243 & $ ≃ $ & {\textbackslash}simeq & Asymptotically Equal To \\ \hline
U+02244 & $ ≄ $ & {\textbackslash}nsime & Not Asymptotically Equal To \\ \hline
U+02245 & $ ≅ $ & {\textbackslash}cong & Approximately Equal To \\ \hline
U+02246 & $ ≆ $ & {\textbackslash}approxnotequal & Approximately But Not Actually Equal To \\ \hline
U+02247 & $ ≇ $ & {\textbackslash}ncong & Neither Approximately Nor Actually Equal To \\ \hline
U+02248 & $ ≈ $ & {\textbackslash}approx & Almost Equal To \\ \hline
U+02249 & $ ≉ $ & {\textbackslash}napprox & Not Almost Equal To \\ \hline
U+0224A & $ ≊ $ & {\textbackslash}approxeq & Almost Equal Or Equal To \\ \hline
U+0224B & $ ≋ $ & {\textbackslash}tildetrpl & Triple Tilde \\ \hline
U+0224C & $ ≌ $ & {\textbackslash}allequal & All Equal To \\ \hline
U+0224D & $ ≍ $ & {\textbackslash}asymp & Equivalent To \\ \hline
U+0224E & $ ≎ $ & {\textbackslash}Bumpeq & Geometrically Equivalent To \\ \hline
U+0224E + U+00338 & $ ≎̸ $ & {\textbackslash}nBumpeq & Geometrically Equivalent To + Combining Long Solidus Overlay / Non-Spacing Long Slash Overlay \\ \hline
U+0224F & $ ≏ $ & {\textbackslash}bumpeq & Difference Between \\ \hline
U+0224F + U+00338 & $ ≏̸ $ & {\textbackslash}nbumpeq & Difference Between + Combining Long Solidus Overlay / Non-Spacing Long Slash Overlay \\ \hline
U+02250 & $ ≐ $ & {\textbackslash}doteq & Approaches The Limit \\ \hline
U+02251 & $ ≑ $ & {\textbackslash}Doteq & Geometrically Equal To \\ \hline
U+02252 & $ ≒ $ & {\textbackslash}fallingdotseq & Approximately Equal To Or The Image Of \\ \hline
U+02253 & $ ≓ $ & {\textbackslash}risingdotseq & Image Of Or Approximately Equal To \\ \hline
U+02254 & $ ≔ $ & {\textbackslash}coloneq & Colon Equals / Colon Equal \\ \hline
U+02255 & $ ≕ $ & {\textbackslash}eqcolon & Equals Colon / Equal Colon \\ \hline
U+02256 & $ ≖ $ & {\textbackslash}eqcirc & Ring In Equal To \\ \hline
U+02257 & $ ≗ $ & {\textbackslash}circeq & Ring Equal To \\ \hline
U+02258 & $ ≘ $ & {\textbackslash}arceq & Corresponds To \\ \hline
U+02259 & $ ≙ $ & {\textbackslash}wedgeq & Estimates \\ \hline
U+0225A & $ ≚ $ & {\textbackslash}veeeq & Equiangular To \\ \hline
U+0225B & $ ≛ $ & {\textbackslash}starequal & Star Equals \\ \hline
U+0225C & $ ≜ $ & {\textbackslash}triangleq & Delta Equal To \\ \hline
U+0225D & $ ≝ $ & {\textbackslash}eqdef & Equal To By Definition \\ \hline
U+0225E & $ ≞ $ & {\textbackslash}measeq & Measured By \\ \hline
U+0225F & $ ≟ $ & {\textbackslash}questeq & Questioned Equal To \\ \hline
U+02260 & $ ≠ $ & {\textbackslash}ne & Not Equal To \\ \hline
U+02261 & $ ≡ $ & {\textbackslash}equiv & Identical To \\ \hline
U+02262 & $ ≢ $ & {\textbackslash}nequiv & Not Identical To \\ \hline
U+02263 & $ ≣ $ & {\textbackslash}Equiv & Strictly Equivalent To \\ \hline
U+02264 & $ ≤ $ & {\textbackslash}le, {\textbackslash}leq & Less-Than Or Equal To / Less Than Or Equal To \\ \hline
U+02265 & $ ≥ $ & {\textbackslash}ge, {\textbackslash}geq & Greater-Than Or Equal To / Greater Than Or Equal To \\ \hline
U+02266 & $ ≦ $ & {\textbackslash}leqq & Less-Than Over Equal To / Less Than Over Equal To \\ \hline
U+02267 & $ ≧ $ & {\textbackslash}geqq & Greater-Than Over Equal To / Greater Than Over Equal To \\ \hline
U+02268 & $ ≨ $ & {\textbackslash}lneqq & Less-Than But Not Equal To / Less Than But Not Equal To \\ \hline
U+02268 + U+0FE00 & $ ≨︀ $ & {\textbackslash}lvertneqq & Less-Than But Not Equal To / Less Than But Not Equal To + Variation Selector-1 \\ \hline
U+02269 & $ ≩ $ & {\textbackslash}gneqq & Greater-Than But Not Equal To / Greater Than But Not Equal To \\ \hline
U+02269 + U+0FE00 & $ ≩︀ $ & {\textbackslash}gvertneqq & Greater-Than But Not Equal To / Greater Than But Not Equal To + Variation Selector-1 \\ \hline
U+0226A & $ ≪ $ & {\textbackslash}ll & Much Less-Than / Much Less Than \\ \hline
U+0226A + U+00338 & $ ≪̸ $ & {\textbackslash}NotLessLess & Much Less-Than / Much Less Than + Combining Long Solidus Overlay / Non-Spacing Long Slash Overlay \\ \hline
U+0226B & $ ≫ $ & {\textbackslash}gg & Much Greater-Than / Much Greater Than \\ \hline
U+0226B + U+00338 & $ ≫̸ $ & {\textbackslash}NotGreaterGreater & Much Greater-Than / Much Greater Than + Combining Long Solidus Overlay / Non-Spacing Long Slash Overlay \\ \hline
U+0226C & $ ≬ $ & {\textbackslash}between & Between \\ \hline
U+0226D & $ ≭ $ & {\textbackslash}nasymp & Not Equivalent To \\ \hline
U+0226E & $ ≮ $ & {\textbackslash}nless & Not Less-Than / Not Less Than \\ \hline
U+0226F & $ ≯ $ & {\textbackslash}ngtr & Not Greater-Than / Not Greater Than \\ \hline
U+02270 & $ ≰ $ & {\textbackslash}nleq & Neither Less-Than Nor Equal To / Neither Less Than Nor Equal To \\ \hline
U+02271 & $ ≱ $ & {\textbackslash}ngeq & Neither Greater-Than Nor Equal To / Neither Greater Than Nor Equal To \\ \hline
U+02272 & $ ≲ $ & {\textbackslash}lesssim & Less-Than Or Equivalent To / Less Than Or Equivalent To \\ \hline
U+02273 & $ ≳ $ & {\textbackslash}gtrsim & Greater-Than Or Equivalent To / Greater Than Or Equivalent To \\ \hline
U+02274 & $ ≴ $ & {\textbackslash}nlesssim & Neither Less-Than Nor Equivalent To / Neither Less Than Nor Equivalent To \\ \hline
U+02275 & $ ≵ $ & {\textbackslash}ngtrsim & Neither Greater-Than Nor Equivalent To / Neither Greater Than Nor Equivalent To \\ \hline
U+02276 & $ ≶ $ & {\textbackslash}lessgtr & Less-Than Or Greater-Than / Less Than Or Greater Than \\ \hline
U+02277 & $ ≷ $ & {\textbackslash}gtrless & Greater-Than Or Less-Than / Greater Than Or Less Than \\ \hline
U+02278 & $ ≸ $ & {\textbackslash}notlessgreater & Neither Less-Than Nor Greater-Than / Neither Less Than Nor Greater Than \\ \hline
U+02279 & $ ≹ $ & {\textbackslash}notgreaterless & Neither Greater-Than Nor Less-Than / Neither Greater Than Nor Less Than \\ \hline
U+0227A & $ ≺ $ & {\textbackslash}prec & Precedes \\ \hline
U+0227B & $ ≻ $ & {\textbackslash}succ & Succeeds \\ \hline
U+0227C & $ ≼ $ & {\textbackslash}preccurlyeq & Precedes Or Equal To \\ \hline
U+0227D & $ ≽ $ & {\textbackslash}succcurlyeq & Succeeds Or Equal To \\ \hline
U+0227E & $ ≾ $ & {\textbackslash}precsim & Precedes Or Equivalent To \\ \hline
U+0227E + U+00338 & $ ≾̸ $ & {\textbackslash}nprecsim & Precedes Or Equivalent To + Combining Long Solidus Overlay / Non-Spacing Long Slash Overlay \\ \hline
U+0227F & $ ≿ $ & {\textbackslash}succsim & Succeeds Or Equivalent To \\ \hline
U+0227F + U+00338 & $ ≿̸ $ & {\textbackslash}nsuccsim & Succeeds Or Equivalent To + Combining Long Solidus Overlay / Non-Spacing Long Slash Overlay \\ \hline
U+02280 & $ ⊀ $ & {\textbackslash}nprec & Does Not Precede \\ \hline
U+02281 & $ ⊁ $ & {\textbackslash}nsucc & Does Not Succeed \\ \hline
U+02282 & $ ⊂ $ & {\textbackslash}subset & Subset Of \\ \hline
U+02283 & $ ⊃ $ & {\textbackslash}supset & Superset Of \\ \hline
U+02284 & $ ⊄ $ & {\textbackslash}nsubset & Not A Subset Of \\ \hline
U+02285 & $ ⊅ $ & {\textbackslash}nsupset & Not A Superset Of \\ \hline
U+02286 & $ ⊆ $ & {\textbackslash}subseteq & Subset Of Or Equal To \\ \hline
U+02287 & $ ⊇ $ & {\textbackslash}supseteq & Superset Of Or Equal To \\ \hline
U+02288 & $ ⊈ $ & {\textbackslash}nsubseteq & Neither A Subset Of Nor Equal To \\ \hline
U+02289 & $ ⊉ $ & {\textbackslash}nsupseteq & Neither A Superset Of Nor Equal To \\ \hline
U+0228A & $ ⊊ $ & {\textbackslash}subsetneq & Subset Of With Not Equal To / Subset Of Or Not Equal To \\ \hline
U+0228A + U+0FE00 & $ ⊊︀ $ & {\textbackslash}varsubsetneqq & Subset Of With Not Equal To / Subset Of Or Not Equal To + Variation Selector-1 \\ \hline
U+0228B & $ ⊋ $ & {\textbackslash}supsetneq & Superset Of With Not Equal To / Superset Of Or Not Equal To \\ \hline
U+0228B + U+0FE00 & $ ⊋︀ $ & {\textbackslash}varsupsetneq & Superset Of With Not Equal To / Superset Of Or Not Equal To + Variation Selector-1 \\ \hline
U+0228D & $ ⊍ $ & {\textbackslash}cupdot & Multiset Multiplication \\ \hline
U+0228E & $ ⊎ $ & {\textbackslash}uplus & Multiset Union \\ \hline
U+0228F & $ ⊏ $ & {\textbackslash}sqsubset & Square Image Of \\ \hline
U+0228F + U+00338 & $ ⊏̸ $ & {\textbackslash}NotSquareSubset & Square Image Of + Combining Long Solidus Overlay / Non-Spacing Long Slash Overlay \\ \hline
U+02290 & $ ⊐ $ & {\textbackslash}sqsupset & Square Original Of \\ \hline
U+02290 + U+00338 & $ ⊐̸ $ & {\textbackslash}NotSquareSuperset & Square Original Of + Combining Long Solidus Overlay / Non-Spacing Long Slash Overlay \\ \hline
U+02291 & $ ⊑ $ & {\textbackslash}sqsubseteq & Square Image Of Or Equal To \\ \hline
U+02292 & $ ⊒ $ & {\textbackslash}sqsupseteq & Square Original Of Or Equal To \\ \hline
U+02293 & $ ⊓ $ & {\textbackslash}sqcap & Square Cap \\ \hline
U+02294 & $ ⊔ $ & {\textbackslash}sqcup & Square Cup \\ \hline
U+02295 & $ ⊕ $ & {\textbackslash}oplus & Circled Plus \\ \hline
U+02296 & $ ⊖ $ & {\textbackslash}ominus & Circled Minus \\ \hline
U+02297 & $ ⊗ $ & {\textbackslash}otimes & Circled Times \\ \hline
U+02298 & $ ⊘ $ & {\textbackslash}oslash & Circled Division Slash \\ \hline
U+02299 & $ ⊙ $ & {\textbackslash}odot & Circled Dot Operator \\ \hline
U+0229A & $ ⊚ $ & {\textbackslash}circledcirc & Circled Ring Operator \\ \hline
U+0229B & $ ⊛ $ & {\textbackslash}circledast & Circled Asterisk Operator \\ \hline
U+0229C & $ ⊜ $ & {\textbackslash}circledequal & Circled Equals \\ \hline
U+0229D & $ ⊝ $ & {\textbackslash}circleddash & Circled Dash \\ \hline
U+0229E & $ ⊞ $ & {\textbackslash}boxplus & Squared Plus \\ \hline
U+0229F & $ ⊟ $ & {\textbackslash}boxminus & Squared Minus \\ \hline
U+022A0 & $ ⊠ $ & {\textbackslash}boxtimes & Squared Times \\ \hline
U+022A1 & $ ⊡ $ & {\textbackslash}boxdot & Squared Dot Operator \\ \hline
U+022A2 & $ ⊢ $ & {\textbackslash}vdash & Right Tack \\ \hline
U+022A3 & $ ⊣ $ & {\textbackslash}dashv & Left Tack \\ \hline
U+022A4 & $ ⊤ $ & {\textbackslash}top & Down Tack \\ \hline
U+022A5 & $ ⊥ $ & {\textbackslash}bot & Up Tack \\ \hline
U+022A7 & $ ⊧ $ & {\textbackslash}models & Models \\ \hline
U+022A8 & $ ⊨ $ & {\textbackslash}vDash & True \\ \hline
U+022A9 & $ ⊩ $ & {\textbackslash}Vdash & Forces \\ \hline
U+022AA & $ ⊪ $ & {\textbackslash}Vvdash & Triple Vertical Bar Right Turnstile \\ \hline
U+022AB & $ ⊫ $ & {\textbackslash}VDash & Double Vertical Bar Double Right Turnstile \\ \hline
U+022AC & $ ⊬ $ & {\textbackslash}nvdash & Does Not Prove \\ \hline
U+022AD & $ ⊭ $ & {\textbackslash}nvDash & Not True \\ \hline
U+022AE & $ ⊮ $ & {\textbackslash}nVdash & Does Not Force \\ \hline
U+022AF & $ ⊯ $ & {\textbackslash}nVDash & Negated Double Vertical Bar Double Right Turnstile \\ \hline
U+022B0 & $ ⊰ $ & {\textbackslash}prurel & Precedes Under Relation \\ \hline
U+022B1 & $ ⊱ $ & {\textbackslash}scurel & Succeeds Under Relation \\ \hline
U+022B2 & $ ⊲ $ & {\textbackslash}vartriangleleft & Normal Subgroup Of \\ \hline
U+022B3 & $ ⊳ $ & {\textbackslash}vartriangleright & Contains As Normal Subgroup \\ \hline
U+022B4 & $ ⊴ $ & {\textbackslash}trianglelefteq & Normal Subgroup Of Or Equal To \\ \hline
U+022B5 & $ ⊵ $ & {\textbackslash}trianglerighteq & Contains As Normal Subgroup Or Equal To \\ \hline
U+022B6 & $ ⊶ $ & {\textbackslash}original & Original Of \\ \hline
U+022B7 & $ ⊷ $ & {\textbackslash}image & Image Of \\ \hline
U+022B8 & $ ⊸ $ & {\textbackslash}multimap & Multimap \\ \hline
U+022B9 & $ ⊹ $ & {\textbackslash}hermitconjmatrix & Hermitian Conjugate Matrix \\ \hline
U+022BA & $ ⊺ $ & {\textbackslash}intercal & Intercalate \\ \hline
U+022BB & $ ⊻ $ & {\textbackslash}veebar, {\textbackslash}xor & Xor \\ \hline
U+022BC & $ ⊼ $ & {\textbackslash}barwedge & Nand \\ \hline
U+022BD & $ ⊽ $ & {\textbackslash}barvee & Nor \\ \hline
U+022BE & $ ⊾ $ & {\textbackslash}rightanglearc & Right Angle With Arc \\ \hline
U+022BF & $ ⊿ $ & {\textbackslash}varlrtriangle & Right Triangle \\ \hline
U+022C0 & $ ⋀ $ & {\textbackslash}bigwedge & N-Ary Logical And \\ \hline
U+022C1 & $ ⋁ $ & {\textbackslash}bigvee & N-Ary Logical Or \\ \hline
U+022C2 & $ ⋂ $ & {\textbackslash}bigcap & N-Ary Intersection \\ \hline
U+022C3 & $ ⋃ $ & {\textbackslash}bigcup & N-Ary Union \\ \hline
U+022C4 & $ ⋄ $ & {\textbackslash}diamond & Diamond Operator \\ \hline
U+022C5 & $ ⋅ $ & {\textbackslash}cdot & Dot Operator \\ \hline
U+022C6 & $ ⋆ $ & {\textbackslash}star & Star Operator \\ \hline
U+022C7 & $ ⋇ $ & {\textbackslash}divideontimes & Division Times \\ \hline
U+022C8 & $ ⋈ $ & {\textbackslash}bowtie & Bowtie \\ \hline
U+022C9 & $ ⋉ $ & {\textbackslash}ltimes & Left Normal Factor Semidirect Product \\ \hline
U+022CA & $ ⋊ $ & {\textbackslash}rtimes & Right Normal Factor Semidirect Product \\ \hline
U+022CB & $ ⋋ $ & {\textbackslash}leftthreetimes & Left Semidirect Product \\ \hline
U+022CC & $ ⋌ $ & {\textbackslash}rightthreetimes & Right Semidirect Product \\ \hline
U+022CD & $ ⋍ $ & {\textbackslash}backsimeq & Reversed Tilde Equals \\ \hline
U+022CE & $ ⋎ $ & {\textbackslash}curlyvee & Curly Logical Or \\ \hline
U+022CF & $ ⋏ $ & {\textbackslash}curlywedge & Curly Logical And \\ \hline
U+022D0 & $ ⋐ $ & {\textbackslash}Subset & Double Subset \\ \hline
U+022D1 & $ ⋑ $ & {\textbackslash}Supset & Double Superset \\ \hline
U+022D2 & $ ⋒ $ & {\textbackslash}Cap & Double Intersection \\ \hline
U+022D3 & $ ⋓ $ & {\textbackslash}Cup & Double Union \\ \hline
U+022D4 & $ ⋔ $ & {\textbackslash}pitchfork & Pitchfork \\ \hline
U+022D5 & $ ⋕ $ & {\textbackslash}equalparallel & Equal And Parallel To \\ \hline
U+022D6 & $ ⋖ $ & {\textbackslash}lessdot & Less-Than With Dot / Less Than With Dot \\ \hline
U+022D7 & $ ⋗ $ & {\textbackslash}gtrdot & Greater-Than With Dot / Greater Than With Dot \\ \hline
U+022D8 & $ ⋘ $ & {\textbackslash}verymuchless & Very Much Less-Than / Very Much Less Than \\ \hline
U+022D9 & $ ⋙ $ & {\textbackslash}ggg & Very Much Greater-Than / Very Much Greater Than \\ \hline
U+022DA & $ ⋚ $ & {\textbackslash}lesseqgtr & Less-Than Equal To Or Greater-Than / Less Than Equal To Or Greater Than \\ \hline
U+022DB & $ ⋛ $ & {\textbackslash}gtreqless & Greater-Than Equal To Or Less-Than / Greater Than Equal To Or Less Than \\ \hline
U+022DC & $ ⋜ $ & {\textbackslash}eqless & Equal To Or Less-Than / Equal To Or Less Than \\ \hline
U+022DD & $ ⋝ $ & {\textbackslash}eqgtr & Equal To Or Greater-Than / Equal To Or Greater Than \\ \hline
U+022DE & $ ⋞ $ & {\textbackslash}curlyeqprec & Equal To Or Precedes \\ \hline
U+022DF & $ ⋟ $ & {\textbackslash}curlyeqsucc & Equal To Or Succeeds \\ \hline
U+022E0 & $ ⋠ $ & {\textbackslash}npreccurlyeq & Does Not Precede Or Equal \\ \hline
U+022E1 & $ ⋡ $ & {\textbackslash}nsucccurlyeq & Does Not Succeed Or Equal \\ \hline
U+022E2 & $ ⋢ $ & {\textbackslash}nsqsubseteq & Not Square Image Of Or Equal To \\ \hline
U+022E3 & $ ⋣ $ & {\textbackslash}nsqsupseteq & Not Square Original Of Or Equal To \\ \hline
U+022E4 & $ ⋤ $ & {\textbackslash}sqsubsetneq & Square Image Of Or Not Equal To \\ \hline
U+022E5 & $ ⋥ $ & {\textbackslash}sqspne & Square Original Of Or Not Equal To \\ \hline
U+022E6 & $ ⋦ $ & {\textbackslash}lnsim & Less-Than But Not Equivalent To / Less Than But Not Equivalent To \\ \hline
U+022E7 & $ ⋧ $ & {\textbackslash}gnsim & Greater-Than But Not Equivalent To / Greater Than But Not Equivalent To \\ \hline
U+022E8 & $ ⋨ $ & {\textbackslash}precnsim & Precedes But Not Equivalent To \\ \hline
U+022E9 & $ ⋩ $ & {\textbackslash}succnsim & Succeeds But Not Equivalent To \\ \hline
U+022EA & $ ⋪ $ & {\textbackslash}ntriangleleft & Not Normal Subgroup Of \\ \hline
U+022EB & $ ⋫ $ & {\textbackslash}ntriangleright & Does Not Contain As Normal Subgroup \\ \hline
U+022EC & $ ⋬ $ & {\textbackslash}ntrianglelefteq & Not Normal Subgroup Of Or Equal To \\ \hline
U+022ED & $ ⋭ $ & {\textbackslash}ntrianglerighteq & Does Not Contain As Normal Subgroup Or Equal \\ \hline
U+022EE & $ ⋮ $ & {\textbackslash}vdots & Vertical Ellipsis \\ \hline
U+022EF & $ ⋯ $ & {\textbackslash}cdots & Midline Horizontal Ellipsis \\ \hline
U+022F0 & $ ⋰ $ & {\textbackslash}adots & Up Right Diagonal Ellipsis \\ \hline
U+022F1 & $ ⋱ $ & {\textbackslash}ddots & Down Right Diagonal Ellipsis \\ \hline
U+022F2 & $ ⋲ $ & {\textbackslash}disin & Element Of With Long Horizontal Stroke \\ \hline
U+022F3 & $ ⋳ $ & {\textbackslash}varisins & Element Of With Vertical Bar At End Of Horizontal Stroke \\ \hline
U+022F4 & $ ⋴ $ & {\textbackslash}isins & Small Element Of With Vertical Bar At End Of Horizontal Stroke \\ \hline
U+022F5 & $ ⋵ $ & {\textbackslash}isindot & Element Of With Dot Above \\ \hline
U+022F6 & $ ⋶ $ & {\textbackslash}varisinobar & Element Of With Overbar \\ \hline
U+022F7 & $ ⋷ $ & {\textbackslash}isinobar & Small Element Of With Overbar \\ \hline
U+022F8 & $ ⋸ $ & {\textbackslash}isinvb & Element Of With Underbar \\ \hline
U+022F9 & $ ⋹ $ & {\textbackslash}isinE & Element Of With Two Horizontal Strokes \\ \hline
U+022FA & $ ⋺ $ & {\textbackslash}nisd & Contains With Long Horizontal Stroke \\ \hline
U+022FB & $ ⋻ $ & {\textbackslash}varnis & Contains With Vertical Bar At End Of Horizontal Stroke \\ \hline
U+022FC & $ ⋼ $ & {\textbackslash}nis & Small Contains With Vertical Bar At End Of Horizontal Stroke \\ \hline
U+022FD & $ ⋽ $ & {\textbackslash}varniobar & Contains With Overbar \\ \hline
U+022FE & $ ⋾ $ & {\textbackslash}niobar & Small Contains With Overbar \\ \hline
U+022FF & $ ⋿ $ & {\textbackslash}bagmember & Z Notation Bag Membership \\ \hline
U+02300 & $ ⌀ $ & {\textbackslash}diameter & Diameter Sign \\ \hline
U+02302 & $ ⌂ $ & {\textbackslash}house & House \\ \hline
U+02305 & $ ⌅ $ & {\textbackslash}varbarwedge & Projective \\ \hline
U+02306 & $ ⌆ $ & {\textbackslash}vardoublebarwedge & Perspective \\ \hline
U+02308 & $ ⌈ $ & {\textbackslash}lceil & Left Ceiling \\ \hline
U+02309 & $ ⌉ $ & {\textbackslash}rceil & Right Ceiling \\ \hline
U+0230A & $ ⌊ $ & {\textbackslash}lfloor & Left Floor \\ \hline
U+0230B & $ ⌋ $ & {\textbackslash}rfloor & Right Floor \\ \hline
U+02310 & $ ⌐ $ & {\textbackslash}invnot & Reversed Not Sign \\ \hline
U+02311 & $ ⌑ $ & {\textbackslash}sqlozenge & Square Lozenge \\ \hline
U+02312 & $ ⌒ $ & {\textbackslash}profline & Arc \\ \hline
U+02313 & $ ⌓ $ & {\textbackslash}profsurf & Segment \\ \hline
U+02315 & $ ⌕ $ & {\textbackslash}recorder & Telephone Recorder \\ \hline
U+02317 & $ ⌗ $ & {\textbackslash}viewdata & Viewdata Square \\ \hline
U+02319 & $ ⌙ $ & {\textbackslash}turnednot & Turned Not Sign \\ \hline
U+0231A & {\EmojiFont ⌚} & {\textbackslash}:watch: & Watch \\ \hline
U+0231B & {\EmojiFont ⌛} & {\textbackslash}:hourglass: & Hourglass \\ \hline
U+0231C & $ ⌜ $ & {\textbackslash}ulcorner & Top Left Corner \\ \hline
U+0231D & $ ⌝ $ & {\textbackslash}urcorner & Top Right Corner \\ \hline
U+0231E & $ ⌞ $ & {\textbackslash}llcorner & Bottom Left Corner \\ \hline
U+0231F & $ ⌟ $ & {\textbackslash}lrcorner & Bottom Right Corner \\ \hline
U+02322 & $ ⌢ $ & {\textbackslash}frown & Frown \\ \hline
U+02323 & $ ⌣ $ & {\textbackslash}smile & Smile \\ \hline
U+0232C & $ ⌬ $ & {\textbackslash}varhexagonlrbonds & Benzene Ring \\ \hline
U+02332 & $ ⌲ $ & {\textbackslash}conictaper & Conical Taper \\ \hline
U+02336 & $ ⌶ $ & {\textbackslash}topbot & Apl Functional Symbol I-Beam \\ \hline
U+0233D & $ ⌽ $ & {\textbackslash}obar & Apl Functional Symbol Circle Stile \\ \hline
U+0233F & $ ⌿ $ & {\textbackslash}notslash & Apl Functional Symbol Slash Bar \\ \hline
U+02340 & $ ⍀ $ & {\textbackslash}notbackslash & Apl Functional Symbol Backslash Bar \\ \hline
U+02353 & $ ⍓ $ & {\textbackslash}boxupcaret & Apl Functional Symbol Quad Up Caret \\ \hline
U+02370 & $ ⍰ $ & {\textbackslash}boxquestion & Apl Functional Symbol Quad Question \\ \hline
U+02394 & $ ⎔ $ & {\textbackslash}hexagon & Software-Function Symbol \\ \hline
U+023A3 & $ ⎣ $ & {\textbackslash}dlcorn & Left Square Bracket Lower Corner \\ \hline
U+023B0 & $ ⎰ $ & {\textbackslash}lmoustache & Upper Left Or Lower Right Curly Bracket Section \\ \hline
U+023B1 & $ ⎱ $ & {\textbackslash}rmoustache & Upper Right Or Lower Left Curly Bracket Section \\ \hline
U+023B4 & $ ⎴ $ & {\textbackslash}overbracket & Top Square Bracket \\ \hline
U+023B5 & $ ⎵ $ & {\textbackslash}underbracket & Bottom Square Bracket \\ \hline
U+023B6 & $ ⎶ $ & {\textbackslash}bbrktbrk & Bottom Square Bracket Over Top Square Bracket \\ \hline
U+023B7 & $ ⎷ $ & {\textbackslash}sqrtbottom & Radical Symbol Bottom \\ \hline
U+023B8 & $ ⎸ $ & {\textbackslash}lvboxline & Left Vertical Box Line \\ \hline
U+023B9 & $ ⎹ $ & {\textbackslash}rvboxline & Right Vertical Box Line \\ \hline
U+023CE & $ ⏎ $ & {\textbackslash}varcarriagereturn & Return Symbol \\ \hline
U+023DE & $ ⏞ $ & {\textbackslash}overbrace & Top Curly Bracket \\ \hline
U+023DF & $ ⏟ $ & {\textbackslash}underbrace & Bottom Curly Bracket \\ \hline
U+023E2 & $ ⏢ $ & {\textbackslash}trapezium & White Trapezium \\ \hline
U+023E3 & $ ⏣ $ & {\textbackslash}benzenr & Benzene Ring With Circle \\ \hline
U+023E4 & $ ⏤ $ & {\textbackslash}strns & Straightness \\ \hline
U+023E5 & $ ⏥ $ & {\textbackslash}fltns & Flatness \\ \hline
U+023E6 & $ ⏦ $ & {\textbackslash}accurrent & Ac Current \\ \hline
U+023E7 & $ ⏧ $ & {\textbackslash}elinters & Electrical Intersection \\ \hline
U+023E9 & {\EmojiFont ⏩} & {\textbackslash}:fast\_forward: & Black Right-Pointing Double Triangle \\ \hline
U+023EA & {\EmojiFont ⏪} & {\textbackslash}:rewind: & Black Left-Pointing Double Triangle \\ \hline
U+023EB & {\EmojiFont ⏫} & {\textbackslash}:arrow\_double\_up: & Black Up-Pointing Double Triangle \\ \hline
U+023EC & {\EmojiFont ⏬} & {\textbackslash}:arrow\_double\_down: & Black Down-Pointing Double Triangle \\ \hline
U+023F0 & {\EmojiFont ⏰} & {\textbackslash}:alarm\_clock: & Alarm Clock \\ \hline
U+023F3 & {\EmojiFont ⏳} & {\textbackslash}:hourglass\_flowing\_sand: & Hourglass With Flowing Sand \\ \hline
U+02422 & {\MathSymFontTwo ␢} & {\textbackslash}blanksymbol & Blank Symbol / Blank \\ \hline
U+02423 & $ ␣ $ & {\textbackslash}visiblespace & Open Box \\ \hline
U+024C2 & {\EmojiFont Ⓜ} & {\textbackslash}:m: & Circled Latin Capital Letter M \\ \hline
U+024C8 & $ Ⓢ $ & {\textbackslash}circledS & Circled Latin Capital Letter S \\ \hline
U+02506 & $ ┆ $ & {\textbackslash}dshfnc & Box Drawings Light Triple Dash Vertical / Forms Light Triple Dash Vertical \\ \hline
U+02519 & {\MathSymFontTwo ┙} & {\textbackslash}sqfnw & Box Drawings Up Light And Left Heavy / Forms Up Light And Left Heavy \\ \hline
U+02571 & $ ╱ $ & {\textbackslash}diagup & Box Drawings Light Diagonal Upper Right To Lower Left / Forms Light Diagonal Upper Right To Lower Left \\ \hline
U+02572 & $ ╲ $ & {\textbackslash}diagdown & Box Drawings Light Diagonal Upper Left To Lower Right / Forms Light Diagonal Upper Left To Lower Right \\ \hline
U+02580 & $ ▀ $ & {\textbackslash}blockuphalf & Upper Half Block \\ \hline
U+02584 & $ ▄ $ & {\textbackslash}blocklowhalf & Lower Half Block \\ \hline
U+02588 & $ █ $ & {\textbackslash}blockfull & Full Block \\ \hline
U+0258C & $ ▌ $ & {\textbackslash}blocklefthalf & Left Half Block \\ \hline
U+02590 & $ ▐ $ & {\textbackslash}blockrighthalf & Right Half Block \\ \hline
U+02591 & $ ░ $ & {\textbackslash}blockqtrshaded & Light Shade \\ \hline
U+02592 & $ ▒ $ & {\textbackslash}blockhalfshaded & Medium Shade \\ \hline
U+02593 & $ ▓ $ & {\textbackslash}blockthreeqtrshaded & Dark Shade \\ \hline
U+025A0 & $ ■ $ & {\textbackslash}blacksquare & Black Square \\ \hline
U+025A1 & $ □ $ & {\textbackslash}square & White Square \\ \hline
U+025A2 & $ ▢ $ & {\textbackslash}squoval & White Square With Rounded Corners \\ \hline
U+025A3 & $ ▣ $ & {\textbackslash}blackinwhitesquare & White Square Containing Black Small Square \\ \hline
U+025A4 & $ ▤ $ & {\textbackslash}squarehfill & Square With Horizontal Fill \\ \hline
U+025A5 & $ ▥ $ & {\textbackslash}squarevfill & Square With Vertical Fill \\ \hline
U+025A6 & $ ▦ $ & {\textbackslash}squarehvfill & Square With Orthogonal Crosshatch Fill \\ \hline
U+025A7 & $ ▧ $ & {\textbackslash}squarenwsefill & Square With Upper Left To Lower Right Fill \\ \hline
U+025A8 & $ ▨ $ & {\textbackslash}squareneswfill & Square With Upper Right To Lower Left Fill \\ \hline
U+025A9 & $ ▩ $ & {\textbackslash}squarecrossfill & Square With Diagonal Crosshatch Fill \\ \hline
U+025AA & {\EmojiFont ▪} & {\textbackslash}:black\_small\_square:, {\textbackslash}smblksquare & Black Small Square \\ \hline
U+025AB & {\EmojiFont ▫} & {\textbackslash}:white\_small\_square:, {\textbackslash}smwhtsquare & White Small Square \\ \hline
U+025AC & $ ▬ $ & {\textbackslash}hrectangleblack & Black Rectangle \\ \hline
U+025AD & $ ▭ $ & {\textbackslash}hrectangle & White Rectangle \\ \hline
U+025AE & $ ▮ $ & {\textbackslash}vrectangleblack & Black Vertical Rectangle \\ \hline
U+025AF & $ ▯ $ & {\textbackslash}vrecto & White Vertical Rectangle \\ \hline
U+025B0 & $ ▰ $ & {\textbackslash}parallelogramblack & Black Parallelogram \\ \hline
U+025B1 & $ ▱ $ & {\textbackslash}parallelogram & White Parallelogram \\ \hline
U+025B2 & $ ▲ $ & {\textbackslash}bigblacktriangleup & Black Up-Pointing Triangle / Black Up Pointing Triangle \\ \hline
U+025B3 & $ △ $ & {\textbackslash}bigtriangleup & White Up-Pointing Triangle / White Up Pointing Triangle \\ \hline
U+025B4 & $ ▴ $ & {\textbackslash}blacktriangle & Black Up-Pointing Small Triangle / Black Up Pointing Small Triangle \\ \hline
U+025B5 & $ ▵ $ & {\textbackslash}vartriangle & White Up-Pointing Small Triangle / White Up Pointing Small Triangle \\ \hline
U+025B6 & {\EmojiFont ▶} & {\textbackslash}:arrow\_forward:, {\textbackslash}blacktriangleright & Black Right-Pointing Triangle / Black Right Pointing Triangle \\ \hline
U+025B7 & $ ▷ $ & {\textbackslash}triangleright & White Right-Pointing Triangle / White Right Pointing Triangle \\ \hline
U+025B8 & $ ▸ $ & {\textbackslash}smallblacktriangleright & Black Right-Pointing Small Triangle / Black Right Pointing Small Triangle \\ \hline
U+025B9 & $ ▹ $ & {\textbackslash}smalltriangleright & White Right-Pointing Small Triangle / White Right Pointing Small Triangle \\ \hline
U+025BA & $ ► $ & {\textbackslash}blackpointerright & Black Right-Pointing Pointer / Black Right Pointing Pointer \\ \hline
U+025BB & $ ▻ $ & {\textbackslash}whitepointerright & White Right-Pointing Pointer / White Right Pointing Pointer \\ \hline
U+025BC & $ ▼ $ & {\textbackslash}bigblacktriangledown & Black Down-Pointing Triangle / Black Down Pointing Triangle \\ \hline
U+025BD & $ ▽ $ & {\textbackslash}bigtriangledown & White Down-Pointing Triangle / White Down Pointing Triangle \\ \hline
U+025BE & $ ▾ $ & {\textbackslash}blacktriangledown & Black Down-Pointing Small Triangle / Black Down Pointing Small Triangle \\ \hline
U+025BF & $ ▿ $ & {\textbackslash}triangledown & White Down-Pointing Small Triangle / White Down Pointing Small Triangle \\ \hline
U+025C0 & {\EmojiFont ◀} & {\textbackslash}:arrow\_backward:, {\textbackslash}blacktriangleleft & Black Left-Pointing Triangle / Black Left Pointing Triangle \\ \hline
U+025C1 & $ ◁ $ & {\textbackslash}triangleleft & White Left-Pointing Triangle / White Left Pointing Triangle \\ \hline
U+025C2 & $ ◂ $ & {\textbackslash}smallblacktriangleleft & Black Left-Pointing Small Triangle / Black Left Pointing Small Triangle \\ \hline
U+025C3 & $ ◃ $ & {\textbackslash}smalltriangleleft & White Left-Pointing Small Triangle / White Left Pointing Small Triangle \\ \hline
U+025C4 & $ ◄ $ & {\textbackslash}blackpointerleft & Black Left-Pointing Pointer / Black Left Pointing Pointer \\ \hline
U+025C5 & $ ◅ $ & {\textbackslash}whitepointerleft & White Left-Pointing Pointer / White Left Pointing Pointer \\ \hline
U+025C6 & $ ◆ $ & {\textbackslash}mdlgblkdiamond & Black Diamond \\ \hline
U+025C7 & $ ◇ $ & {\textbackslash}mdlgwhtdiamond & White Diamond \\ \hline
U+025C8 & $ ◈ $ & {\textbackslash}blackinwhitediamond & White Diamond Containing Black Small Diamond \\ \hline
U+025C9 & $ ◉ $ & {\textbackslash}fisheye & Fisheye \\ \hline
U+025CA & $ ◊ $ & {\textbackslash}lozenge & Lozenge \\ \hline
U+025CB & $ ○ $ & {\textbackslash}bigcirc & White Circle \\ \hline
U+025CC & $ ◌ $ & {\textbackslash}dottedcircle & Dotted Circle \\ \hline
U+025CD & $ ◍ $ & {\textbackslash}circlevertfill & Circle With Vertical Fill \\ \hline
U+025CE & $ ◎ $ & {\textbackslash}bullseye & Bullseye \\ \hline
U+025CF & $ ● $ & {\textbackslash}mdlgblkcircle & Black Circle \\ \hline
U+025D0 & $ ◐ $ & {\textbackslash}cirfl & Circle With Left Half Black \\ \hline
U+025D1 & $ ◑ $ & {\textbackslash}cirfr & Circle With Right Half Black \\ \hline
U+025D2 & $ ◒ $ & {\textbackslash}cirfb & Circle With Lower Half Black \\ \hline
U+025D3 & $ ◓ $ & {\textbackslash}circletophalfblack & Circle With Upper Half Black \\ \hline
U+025D4 & $ ◔ $ & {\textbackslash}circleurquadblack & Circle With Upper Right Quadrant Black \\ \hline
U+025D5 & $ ◕ $ & {\textbackslash}blackcircleulquadwhite & Circle With All But Upper Left Quadrant Black \\ \hline
U+025D6 & $ ◖ $ & {\textbackslash}blacklefthalfcircle & Left Half Black Circle \\ \hline
U+025D7 & $ ◗ $ & {\textbackslash}blackrighthalfcircle & Right Half Black Circle \\ \hline
U+025D8 & $ ◘ $ & {\textbackslash}rvbull & Inverse Bullet \\ \hline
U+025D9 & $ ◙ $ & {\textbackslash}inversewhitecircle & Inverse White Circle \\ \hline
U+025DA & $ ◚ $ & {\textbackslash}invwhiteupperhalfcircle & Upper Half Inverse White Circle \\ \hline
U+025DB & $ ◛ $ & {\textbackslash}invwhitelowerhalfcircle & Lower Half Inverse White Circle \\ \hline
U+025DC & $ ◜ $ & {\textbackslash}ularc & Upper Left Quadrant Circular Arc \\ \hline
U+025DD & $ ◝ $ & {\textbackslash}urarc & Upper Right Quadrant Circular Arc \\ \hline
U+025DE & $ ◞ $ & {\textbackslash}lrarc & Lower Right Quadrant Circular Arc \\ \hline
U+025DF & $ ◟ $ & {\textbackslash}llarc & Lower Left Quadrant Circular Arc \\ \hline
U+025E0 & $ ◠ $ & {\textbackslash}topsemicircle & Upper Half Circle \\ \hline
U+025E1 & $ ◡ $ & {\textbackslash}botsemicircle & Lower Half Circle \\ \hline
U+025E2 & $ ◢ $ & {\textbackslash}lrblacktriangle & Black Lower Right Triangle \\ \hline
U+025E3 & $ ◣ $ & {\textbackslash}llblacktriangle & Black Lower Left Triangle \\ \hline
U+025E4 & $ ◤ $ & {\textbackslash}ulblacktriangle & Black Upper Left Triangle \\ \hline
U+025E5 & $ ◥ $ & {\textbackslash}urblacktriangle & Black Upper Right Triangle \\ \hline
U+025E6 & $ ◦ $ & {\textbackslash}smwhtcircle & White Bullet \\ \hline
U+025E7 & $ ◧ $ & {\textbackslash}sqfl & Square With Left Half Black \\ \hline
U+025E8 & $ ◨ $ & {\textbackslash}sqfr & Square With Right Half Black \\ \hline
U+025E9 & $ ◩ $ & {\textbackslash}squareulblack & Square With Upper Left Diagonal Half Black \\ \hline
U+025EA & $ ◪ $ & {\textbackslash}sqfse & Square With Lower Right Diagonal Half Black \\ \hline
U+025EB & $ ◫ $ & {\textbackslash}boxbar & White Square With Vertical Bisecting Line \\ \hline
U+025EC & $ ◬ $ & {\textbackslash}trianglecdot & White Up-Pointing Triangle With Dot / White Up Pointing Triangle With Dot \\ \hline
U+025ED & $ ◭ $ & {\textbackslash}triangleleftblack & Up-Pointing Triangle With Left Half Black / Up Pointing Triangle With Left Half Black \\ \hline
U+025EE & $ ◮ $ & {\textbackslash}trianglerightblack & Up-Pointing Triangle With Right Half Black / Up Pointing Triangle With Right Half Black \\ \hline
U+025EF & $ ◯ $ & {\textbackslash}lgwhtcircle & Large Circle \\ \hline
U+025F0 & $ ◰ $ & {\textbackslash}squareulquad & White Square With Upper Left Quadrant \\ \hline
U+025F1 & $ ◱ $ & {\textbackslash}squarellquad & White Square With Lower Left Quadrant \\ \hline
U+025F2 & $ ◲ $ & {\textbackslash}squarelrquad & White Square With Lower Right Quadrant \\ \hline
U+025F3 & $ ◳ $ & {\textbackslash}squareurquad & White Square With Upper Right Quadrant \\ \hline
U+025F4 & $ ◴ $ & {\textbackslash}circleulquad & White Circle With Upper Left Quadrant \\ \hline
U+025F5 & $ ◵ $ & {\textbackslash}circlellquad & White Circle With Lower Left Quadrant \\ \hline
U+025F6 & $ ◶ $ & {\textbackslash}circlelrquad & White Circle With Lower Right Quadrant \\ \hline
U+025F7 & $ ◷ $ & {\textbackslash}circleurquad & White Circle With Upper Right Quadrant \\ \hline
U+025F8 & $ ◸ $ & {\textbackslash}ultriangle & Upper Left Triangle \\ \hline
U+025F9 & $ ◹ $ & {\textbackslash}urtriangle & Upper Right Triangle \\ \hline
U+025FA & $ ◺ $ & {\textbackslash}lltriangle & Lower Left Triangle \\ \hline
U+025FB & {\EmojiFont ◻} & {\textbackslash}:white\_medium\_square:, {\textbackslash}mdwhtsquare & White Medium Square \\ \hline
U+025FC & {\EmojiFont ◼} & {\textbackslash}:black\_medium\_square:, {\textbackslash}mdblksquare & Black Medium Square \\ \hline
U+025FD & {\EmojiFont ◽} & {\textbackslash}:white\_medium\_small\_square:, {\textbackslash}mdsmwhtsquare & White Medium Small Square \\ \hline
U+025FE & {\EmojiFont ◾} & {\textbackslash}:black\_medium\_small\_square:, {\textbackslash}mdsmblksquare & Black Medium Small Square \\ \hline
U+025FF & $ ◿ $ & {\textbackslash}lrtriangle & Lower Right Triangle \\ \hline
U+02600 & {\EmojiFont ☀} & {\textbackslash}:sunny: & Black Sun With Rays \\ \hline
U+02601 & {\EmojiFont ☁} & {\textbackslash}:cloud: & Cloud \\ \hline
U+02605 & $ ★ $ & {\textbackslash}bigstar & Black Star \\ \hline
U+02606 & $ ☆ $ & {\textbackslash}bigwhitestar & White Star \\ \hline
U+02609 & $ ☉ $ & {\textbackslash}astrosun & Sun \\ \hline
U+0260E & {\EmojiFont ☎} & {\textbackslash}:phone: & Black Telephone \\ \hline
U+02611 & {\EmojiFont ☑} & {\textbackslash}:ballot\_box\_with\_check: & Ballot Box With Check \\ \hline
U+02614 & {\EmojiFont ☔} & {\textbackslash}:umbrella: & Umbrella With Rain Drops \\ \hline
U+02615 & {\EmojiFont ☕} & {\textbackslash}:coffee: & Hot Beverage \\ \hline
U+0261D & {\EmojiFont ☝} & {\textbackslash}:point\_up: & White Up Pointing Index \\ \hline
U+02621 & $ ☡ $ & {\textbackslash}danger & Caution Sign \\ \hline
U+0263A & {\EmojiFont ☺} & {\textbackslash}:relaxed: & White Smiling Face \\ \hline
U+0263B & $ ☻ $ & {\textbackslash}blacksmiley & Black Smiling Face \\ \hline
U+0263C & $ ☼ $ & {\textbackslash}sun & White Sun With Rays \\ \hline
U+0263D & $ ☽ $ & {\textbackslash}rightmoon & First Quarter Moon \\ \hline
U+0263E & $ ☾ $ & {\textbackslash}leftmoon & Last Quarter Moon \\ \hline
U+0263F & $ ☿ $ & {\textbackslash}mercury & Mercury \\ \hline
U+02640 & $ ♀ $ & {\textbackslash}venus, {\textbackslash}female & Female Sign \\ \hline
U+02642 & $ ♂ $ & {\textbackslash}male, {\textbackslash}mars & Male Sign \\ \hline
U+02643 & {\MathSymFontTwo ♃} & {\textbackslash}jupiter & Jupiter \\ \hline
U+02644 & {\MathSymFontTwo ♄} & {\textbackslash}saturn & Saturn \\ \hline
U+02645 & {\MathSymFontTwo ♅} & {\textbackslash}uranus & Uranus \\ \hline
U+02646 & {\MathSymFontTwo ♆} & {\textbackslash}neptune & Neptune \\ \hline
U+02647 & {\MathSymFontTwo ♇} & {\textbackslash}pluto & Pluto \\ \hline
U+02648 & {\EmojiFont ♈} & {\textbackslash}:aries:, {\textbackslash}aries & Aries \\ \hline
U+02649 & {\EmojiFont ♉} & {\textbackslash}:taurus:, {\textbackslash}taurus & Taurus \\ \hline
U+0264A & {\EmojiFont ♊} & {\textbackslash}:gemini:, {\textbackslash}gemini & Gemini \\ \hline
U+0264B & {\EmojiFont ♋} & {\textbackslash}:cancer:, {\textbackslash}cancer & Cancer \\ \hline
U+0264C & {\EmojiFont ♌} & {\textbackslash}:leo:, {\textbackslash}leo & Leo \\ \hline
U+0264D & {\EmojiFont ♍} & {\textbackslash}:virgo:, {\textbackslash}virgo & Virgo \\ \hline
U+0264E & {\EmojiFont ♎} & {\textbackslash}:libra:, {\textbackslash}libra & Libra \\ \hline
U+0264F & {\EmojiFont ♏} & {\textbackslash}:scorpius:, {\textbackslash}scorpio & Scorpius \\ \hline
U+02650 & {\EmojiFont ♐} & {\textbackslash}:sagittarius:, {\textbackslash}sagittarius & Sagittarius \\ \hline
U+02651 & {\EmojiFont ♑} & {\textbackslash}:capricorn:, {\textbackslash}capricornus & Capricorn \\ \hline
U+02652 & {\EmojiFont ♒} & {\textbackslash}:aquarius:, {\textbackslash}aquarius & Aquarius \\ \hline
U+02653 & {\EmojiFont ♓} & {\textbackslash}:pisces:, {\textbackslash}pisces & Pisces \\ \hline
U+02660 & {\EmojiFont ♠} & {\textbackslash}:spades:, {\textbackslash}spadesuit & Black Spade Suit \\ \hline
U+02661 & $ ♡ $ & {\textbackslash}heartsuit & White Heart Suit \\ \hline
U+02662 & $ ♢ $ & {\textbackslash}diamondsuit & White Diamond Suit \\ \hline
U+02663 & {\EmojiFont ♣} & {\textbackslash}:clubs:, {\textbackslash}clubsuit & Black Club Suit \\ \hline
U+02664 & $ ♤ $ & {\textbackslash}varspadesuit & White Spade Suit \\ \hline
U+02665 & {\EmojiFont ♥} & {\textbackslash}:hearts:, {\textbackslash}varheartsuit & Black Heart Suit \\ \hline
U+02666 & {\EmojiFont ♦} & {\textbackslash}:diamonds:, {\textbackslash}vardiamondsuit & Black Diamond Suit \\ \hline
U+02667 & $ ♧ $ & {\textbackslash}varclubsuit & White Club Suit \\ \hline
U+02668 & {\EmojiFont ♨} & {\textbackslash}:hotsprings: & Hot Springs \\ \hline
U+02669 & $ ♩ $ & {\textbackslash}quarternote & Quarter Note \\ \hline
U+0266A & $ ♪ $ & {\textbackslash}eighthnote & Eighth Note \\ \hline
U+0266B & $ ♫ $ & {\textbackslash}twonotes & Beamed Eighth Notes / Barred Eighth Notes \\ \hline
U+0266D & $ ♭ $ & {\textbackslash}flat & Music Flat Sign / Flat \\ \hline
U+0266E & $ ♮ $ & {\textbackslash}natural & Music Natural Sign / Natural \\ \hline
U+0266F & $ ♯ $ & {\textbackslash}sharp & Music Sharp Sign / Sharp \\ \hline
U+0267B & {\EmojiFont ♻} & {\textbackslash}:recycle: & Black Universal Recycling Symbol \\ \hline
U+0267E & $ ♾ $ & {\textbackslash}acidfree & Permanent Paper Sign \\ \hline
U+0267F & {\EmojiFont ♿} & {\textbackslash}:wheelchair: & Wheelchair Symbol \\ \hline
U+02680 & $ ⚀ $ & {\textbackslash}dicei & Die Face-1 \\ \hline
U+02681 & $ ⚁ $ & {\textbackslash}diceii & Die Face-2 \\ \hline
U+02682 & $ ⚂ $ & {\textbackslash}diceiii & Die Face-3 \\ \hline
U+02683 & $ ⚃ $ & {\textbackslash}diceiv & Die Face-4 \\ \hline
U+02684 & $ ⚄ $ & {\textbackslash}dicev & Die Face-5 \\ \hline
U+02685 & $ ⚅ $ & {\textbackslash}dicevi & Die Face-6 \\ \hline
U+02686 & $ ⚆ $ & {\textbackslash}circledrightdot & White Circle With Dot Right \\ \hline
U+02687 & $ ⚇ $ & {\textbackslash}circledtwodots & White Circle With Two Dots \\ \hline
U+02688 & $ ⚈ $ & {\textbackslash}blackcircledrightdot & Black Circle With White Dot Right \\ \hline
U+02689 & $ ⚉ $ & {\textbackslash}blackcircledtwodots & Black Circle With Two White Dots \\ \hline
U+02693 & {\EmojiFont ⚓} & {\textbackslash}:anchor: & Anchor \\ \hline
U+026A0 & {\EmojiFont ⚠} & {\textbackslash}:warning: & Warning Sign \\ \hline
U+026A1 & {\EmojiFont ⚡} & {\textbackslash}:zap: & High Voltage Sign \\ \hline
U+026A5 & $ ⚥ $ & {\textbackslash}hermaphrodite & Male And Female Sign \\ \hline
U+026AA & {\EmojiFont ⚪} & {\textbackslash}:white\_circle:, {\textbackslash}mdwhtcircle & Medium White Circle \\ \hline
U+026AB & {\EmojiFont ⚫} & {\textbackslash}:black\_circle:, {\textbackslash}mdblkcircle & Medium Black Circle \\ \hline
U+026AC & $ ⚬ $ & {\textbackslash}mdsmwhtcircle & Medium Small White Circle \\ \hline
U+026B2 & $ ⚲ $ & {\textbackslash}neuter & Neuter \\ \hline
U+026BD & {\EmojiFont ⚽} & {\textbackslash}:soccer: & Soccer Ball \\ \hline
U+026BE & {\EmojiFont ⚾} & {\textbackslash}:baseball: & Baseball \\ \hline
U+026C4 & {\EmojiFont ⛄} & {\textbackslash}:snowman: & Snowman Without Snow \\ \hline
U+026C5 & {\EmojiFont ⛅} & {\textbackslash}:partly\_sunny: & Sun Behind Cloud \\ \hline
U+026CE & {\EmojiFont ⛎} & {\textbackslash}:ophiuchus: & Ophiuchus \\ \hline
U+026D4 & {\EmojiFont ⛔} & {\textbackslash}:no\_entry: & No Entry \\ \hline
U+026EA & {\EmojiFont ⛪} & {\textbackslash}:church: & Church \\ \hline
U+026F2 & {\EmojiFont ⛲} & {\textbackslash}:fountain: & Fountain \\ \hline
U+026F3 & {\EmojiFont ⛳} & {\textbackslash}:golf: & Flag In Hole \\ \hline
U+026F5 & {\EmojiFont ⛵} & {\textbackslash}:boat: & Sailboat \\ \hline
U+026FA & {\EmojiFont ⛺} & {\textbackslash}:tent: & Tent \\ \hline
U+026FD & {\EmojiFont ⛽} & {\textbackslash}:fuelpump: & Fuel Pump \\ \hline
U+02702 & {\EmojiFont ✂} & {\textbackslash}:scissors: & Black Scissors \\ \hline
U+02705 & {\EmojiFont ✅} & {\textbackslash}:white\_check\_mark: & White Heavy Check Mark \\ \hline
U+02708 & {\EmojiFont ✈} & {\textbackslash}:airplane: & Airplane \\ \hline
U+02709 & {\EmojiFont ✉} & {\textbackslash}:email: & Envelope \\ \hline
U+0270A & {\EmojiFont ✊} & {\textbackslash}:fist: & Raised Fist \\ \hline
U+0270B & {\EmojiFont ✋} & {\textbackslash}:hand: & Raised Hand \\ \hline
U+0270C & {\EmojiFont ✌} & {\textbackslash}:v: & Victory Hand \\ \hline
U+0270F & {\EmojiFont ✏} & {\textbackslash}:pencil2: & Pencil \\ \hline
U+02712 & {\EmojiFont ✒} & {\textbackslash}:black\_nib: & Black Nib \\ \hline
U+02713 & $ ✓ $ & {\textbackslash}checkmark & Check Mark \\ \hline
U+02714 & {\EmojiFont ✔} & {\textbackslash}:heavy\_check\_mark: & Heavy Check Mark \\ \hline
U+02716 & {\EmojiFont ✖} & {\textbackslash}:heavy\_multiplication\_x: & Heavy Multiplication X \\ \hline
U+02720 & $ ✠ $ & {\textbackslash}maltese & Maltese Cross \\ \hline
U+02728 & {\EmojiFont ✨} & {\textbackslash}:sparkles: & Sparkles \\ \hline
U+0272A & $ ✪ $ & {\textbackslash}circledstar & Circled White Star \\ \hline
U+02733 & {\EmojiFont ✳} & {\textbackslash}:eight\_spoked\_asterisk: & Eight Spoked Asterisk \\ \hline
U+02734 & {\EmojiFont ✴} & {\textbackslash}:eight\_pointed\_black\_star: & Eight Pointed Black Star \\ \hline
U+02736 & $ ✶ $ & {\textbackslash}varstar & Six Pointed Black Star \\ \hline
U+0273D & $ ✽ $ & {\textbackslash}dingasterisk & Heavy Teardrop-Spoked Asterisk \\ \hline
U+02744 & {\EmojiFont ❄} & {\textbackslash}:snowflake: & Snowflake \\ \hline
U+02747 & {\EmojiFont ❇} & {\textbackslash}:sparkle: & Sparkle \\ \hline
U+0274C & {\EmojiFont ❌} & {\textbackslash}:x: & Cross Mark \\ \hline
U+0274E & {\EmojiFont ❎} & {\textbackslash}:negative\_squared\_cross\_mark: & Negative Squared Cross Mark \\ \hline
U+02753 & {\EmojiFont ❓} & {\textbackslash}:question: & Black Question Mark Ornament \\ \hline
U+02754 & {\EmojiFont ❔} & {\textbackslash}:grey\_question: & White Question Mark Ornament \\ \hline
U+02755 & {\EmojiFont ❕} & {\textbackslash}:grey\_exclamation: & White Exclamation Mark Ornament \\ \hline
U+02757 & {\EmojiFont ❗} & {\textbackslash}:exclamation: & Heavy Exclamation Mark Symbol \\ \hline
U+02764 & {\EmojiFont ❤} & {\textbackslash}:heart: & Heavy Black Heart \\ \hline
U+02795 & {\EmojiFont ➕} & {\textbackslash}:heavy\_plus\_sign: & Heavy Plus Sign \\ \hline
U+02796 & {\EmojiFont ➖} & {\textbackslash}:heavy\_minus\_sign: & Heavy Minus Sign \\ \hline
U+02797 & {\EmojiFont ➗} & {\textbackslash}:heavy\_division\_sign: & Heavy Division Sign \\ \hline
U+0279B & $ ➛ $ & {\textbackslash}draftingarrow & Drafting Point Rightwards Arrow / Drafting Point Right Arrow \\ \hline
U+027A1 & {\EmojiFont ➡} & {\textbackslash}:arrow\_right: & Black Rightwards Arrow / Black Right Arrow \\ \hline
U+027B0 & {\EmojiFont ➰} & {\textbackslash}:curly\_loop: & Curly Loop \\ \hline
U+027BF & {\EmojiFont ➿} & {\textbackslash}:loop: & Double Curly Loop \\ \hline
U+027C0 & $ ⟀ $ & {\textbackslash}threedangle & Three Dimensional Angle \\ \hline
U+027C1 & $ ⟁ $ & {\textbackslash}whiteinwhitetriangle & White Triangle Containing Small White Triangle \\ \hline
U+027C2 & $ ⟂ $ & {\textbackslash}perp & Perpendicular \\ \hline
U+027C8 & $ ⟈ $ & {\textbackslash}bsolhsub & Reverse Solidus Preceding Subset \\ \hline
U+027C9 & $ ⟉ $ & {\textbackslash}suphsol & Superset Preceding Solidus \\ \hline
U+027D1 & $ ⟑ $ & {\textbackslash}wedgedot & And With Dot \\ \hline
U+027D2 & $ ⟒ $ & {\textbackslash}upin & Element Of Opening Upwards \\ \hline
U+027D5 & $ ⟕ $ & {\textbackslash}leftouterjoin & Left Outer Join \\ \hline
U+027D6 & $ ⟖ $ & {\textbackslash}rightouterjoin & Right Outer Join \\ \hline
U+027D7 & $ ⟗ $ & {\textbackslash}fullouterjoin & Full Outer Join \\ \hline
U+027D8 & $ ⟘ $ & {\textbackslash}bigbot & Large Up Tack \\ \hline
U+027D9 & $ ⟙ $ & {\textbackslash}bigtop & Large Down Tack \\ \hline
U+027E6 & $ ⟦ $ & {\textbackslash}llbracket, {\textbackslash}openbracketleft & Mathematical Left White Square Bracket \\ \hline
U+027E7 & $ ⟧ $ & {\textbackslash}openbracketright, {\textbackslash}rrbracket & Mathematical Right White Square Bracket \\ \hline
U+027E8 & $ ⟨ $ & {\textbackslash}langle & Mathematical Left Angle Bracket \\ \hline
U+027E9 & $ ⟩ $ & {\textbackslash}rangle & Mathematical Right Angle Bracket \\ \hline
U+027F0 & $ ⟰ $ & {\textbackslash}UUparrow & Upwards Quadruple Arrow \\ \hline
U+027F1 & $ ⟱ $ & {\textbackslash}DDownarrow & Downwards Quadruple Arrow \\ \hline
U+027F5 & $ ⟵ $ & {\textbackslash}longleftarrow & Long Leftwards Arrow \\ \hline
U+027F6 & $ ⟶ $ & {\textbackslash}longrightarrow & Long Rightwards Arrow \\ \hline
U+027F7 & $ ⟷ $ & {\textbackslash}longleftrightarrow & Long Left Right Arrow \\ \hline
U+027F8 & $ ⟸ $ & {\textbackslash}impliedby, {\textbackslash}Longleftarrow & Long Leftwards Double Arrow \\ \hline
U+027F9 & $ ⟹ $ & {\textbackslash}implies, {\textbackslash}Longrightarrow & Long Rightwards Double Arrow \\ \hline
U+027FA & $ ⟺ $ & {\textbackslash}Longleftrightarrow, {\textbackslash}iff & Long Left Right Double Arrow \\ \hline
U+027FB & $ ⟻ $ & {\textbackslash}longmapsfrom & Long Leftwards Arrow From Bar \\ \hline
U+027FC & $ ⟼ $ & {\textbackslash}longmapsto & Long Rightwards Arrow From Bar \\ \hline
U+027FD & $ ⟽ $ & {\textbackslash}Longmapsfrom & Long Leftwards Double Arrow From Bar \\ \hline
U+027FE & $ ⟾ $ & {\textbackslash}Longmapsto & Long Rightwards Double Arrow From Bar \\ \hline
U+027FF & $ ⟿ $ & {\textbackslash}longrightsquigarrow & Long Rightwards Squiggle Arrow \\ \hline
U+02900 & $ ⤀ $ & {\textbackslash}nvtwoheadrightarrow & Rightwards Two-Headed Arrow With Vertical Stroke \\ \hline
U+02901 & $ ⤁ $ & {\textbackslash}nVtwoheadrightarrow & Rightwards Two-Headed Arrow With Double Vertical Stroke \\ \hline
U+02902 & $ ⤂ $ & {\textbackslash}nvLeftarrow & Leftwards Double Arrow With Vertical Stroke \\ \hline
U+02903 & $ ⤃ $ & {\textbackslash}nvRightarrow & Rightwards Double Arrow With Vertical Stroke \\ \hline
U+02904 & $ ⤄ $ & {\textbackslash}nvLeftrightarrow & Left Right Double Arrow With Vertical Stroke \\ \hline
U+02905 & $ ⤅ $ & {\textbackslash}twoheadmapsto & Rightwards Two-Headed Arrow From Bar \\ \hline
U+02906 & $ ⤆ $ & {\textbackslash}Mapsfrom & Leftwards Double Arrow From Bar \\ \hline
U+02907 & $ ⤇ $ & {\textbackslash}Mapsto & Rightwards Double Arrow From Bar \\ \hline
U+02908 & $ ⤈ $ & {\textbackslash}downarrowbarred & Downwards Arrow With Horizontal Stroke \\ \hline
U+02909 & $ ⤉ $ & {\textbackslash}uparrowbarred & Upwards Arrow With Horizontal Stroke \\ \hline
U+0290A & $ ⤊ $ & {\textbackslash}Uuparrow & Upwards Triple Arrow \\ \hline
U+0290B & $ ⤋ $ & {\textbackslash}Ddownarrow & Downwards Triple Arrow \\ \hline
U+0290C & $ ⤌ $ & {\textbackslash}leftbkarrow & Leftwards Double Dash Arrow \\ \hline
U+0290D & $ ⤍ $ & {\textbackslash}bkarow & Rightwards Double Dash Arrow \\ \hline
U+0290E & $ ⤎ $ & {\textbackslash}leftdbkarrow & Leftwards Triple Dash Arrow \\ \hline
U+0290F & $ ⤏ $ & {\textbackslash}dbkarow & Rightwards Triple Dash Arrow \\ \hline
U+02910 & $ ⤐ $ & {\textbackslash}drbkarrow & Rightwards Two-Headed Triple Dash Arrow \\ \hline
U+02911 & $ ⤑ $ & {\textbackslash}rightdotarrow & Rightwards Arrow With Dotted Stem \\ \hline
U+02912 & $ ⤒ $ & {\textbackslash}UpArrowBar & Upwards Arrow To Bar \\ \hline
U+02913 & $ ⤓ $ & {\textbackslash}DownArrowBar & Downwards Arrow To Bar \\ \hline
U+02914 & $ ⤔ $ & {\textbackslash}nvrightarrowtail & Rightwards Arrow With Tail With Vertical Stroke \\ \hline
U+02915 & $ ⤕ $ & {\textbackslash}nVrightarrowtail & Rightwards Arrow With Tail With Double Vertical Stroke \\ \hline
U+02916 & $ ⤖ $ & {\textbackslash}twoheadrightarrowtail & Rightwards Two-Headed Arrow With Tail \\ \hline
U+02917 & $ ⤗ $ & {\textbackslash}nvtwoheadrightarrowtail & Rightwards Two-Headed Arrow With Tail With Vertical Stroke \\ \hline
U+02918 & $ ⤘ $ & {\textbackslash}nVtwoheadrightarrowtail & Rightwards Two-Headed Arrow With Tail With Double Vertical Stroke \\ \hline
U+0291D & $ ⤝ $ & {\textbackslash}diamondleftarrow & Leftwards Arrow To Black Diamond \\ \hline
U+0291E & $ ⤞ $ & {\textbackslash}rightarrowdiamond & Rightwards Arrow To Black Diamond \\ \hline
U+0291F & $ ⤟ $ & {\textbackslash}diamondleftarrowbar & Leftwards Arrow From Bar To Black Diamond \\ \hline
U+02920 & $ ⤠ $ & {\textbackslash}barrightarrowdiamond & Rightwards Arrow From Bar To Black Diamond \\ \hline
U+02925 & $ ⤥ $ & {\textbackslash}hksearow & South East Arrow With Hook \\ \hline
U+02926 & $ ⤦ $ & {\textbackslash}hkswarow & South West Arrow With Hook \\ \hline
U+02927 & $ ⤧ $ & {\textbackslash}tona & North West Arrow And North East Arrow \\ \hline
U+02928 & $ ⤨ $ & {\textbackslash}toea & North East Arrow And South East Arrow \\ \hline
U+02929 & $ ⤩ $ & {\textbackslash}tosa & South East Arrow And South West Arrow \\ \hline
U+0292A & $ ⤪ $ & {\textbackslash}towa & South West Arrow And North West Arrow \\ \hline
U+0292B & $ ⤫ $ & {\textbackslash}rdiagovfdiag & Rising Diagonal Crossing Falling Diagonal \\ \hline
U+0292C & $ ⤬ $ & {\textbackslash}fdiagovrdiag & Falling Diagonal Crossing Rising Diagonal \\ \hline
U+0292D & $ ⤭ $ & {\textbackslash}seovnearrow & South East Arrow Crossing North East Arrow \\ \hline
U+0292E & $ ⤮ $ & {\textbackslash}neovsearrow & North East Arrow Crossing South East Arrow \\ \hline
U+0292F & $ ⤯ $ & {\textbackslash}fdiagovnearrow & Falling Diagonal Crossing North East Arrow \\ \hline
U+02930 & $ ⤰ $ & {\textbackslash}rdiagovsearrow & Rising Diagonal Crossing South East Arrow \\ \hline
U+02931 & $ ⤱ $ & {\textbackslash}neovnwarrow & North East Arrow Crossing North West Arrow \\ \hline
U+02932 & $ ⤲ $ & {\textbackslash}nwovnearrow & North West Arrow Crossing North East Arrow \\ \hline
U+02934 & {\EmojiFont ⤴} & {\textbackslash}:arrow\_heading\_up: & Arrow Pointing Rightwards Then Curving Upwards \\ \hline
U+02935 & {\EmojiFont ⤵} & {\textbackslash}:arrow\_heading\_down: & Arrow Pointing Rightwards Then Curving Downwards \\ \hline
U+02942 & $ ⥂ $ & {\textbackslash}Rlarr & Rightwards Arrow Above Short Leftwards Arrow \\ \hline
U+02944 & $ ⥄ $ & {\textbackslash}rLarr & Short Rightwards Arrow Above Leftwards Arrow \\ \hline
U+02945 & $ ⥅ $ & {\textbackslash}rightarrowplus & Rightwards Arrow With Plus Below \\ \hline
U+02946 & $ ⥆ $ & {\textbackslash}leftarrowplus & Leftwards Arrow With Plus Below \\ \hline
U+02947 & $ ⥇ $ & {\textbackslash}rarrx & Rightwards Arrow Through X \\ \hline
U+02948 & $ ⥈ $ & {\textbackslash}leftrightarrowcircle & Left Right Arrow Through Small Circle \\ \hline
U+02949 & $ ⥉ $ & {\textbackslash}twoheaduparrowcircle & Upwards Two-Headed Arrow From Small Circle \\ \hline
U+0294A & $ ⥊ $ & {\textbackslash}leftrightharpoonupdown & Left Barb Up Right Barb Down Harpoon \\ \hline
U+0294B & $ ⥋ $ & {\textbackslash}leftrightharpoondownup & Left Barb Down Right Barb Up Harpoon \\ \hline
U+0294C & $ ⥌ $ & {\textbackslash}updownharpoonrightleft & Up Barb Right Down Barb Left Harpoon \\ \hline
U+0294D & $ ⥍ $ & {\textbackslash}updownharpoonleftright & Up Barb Left Down Barb Right Harpoon \\ \hline
U+0294E & $ ⥎ $ & {\textbackslash}LeftRightVector & Left Barb Up Right Barb Up Harpoon \\ \hline
U+0294F & $ ⥏ $ & {\textbackslash}RightUpDownVector & Up Barb Right Down Barb Right Harpoon \\ \hline
U+02950 & $ ⥐ $ & {\textbackslash}DownLeftRightVector & Left Barb Down Right Barb Down Harpoon \\ \hline
U+02951 & $ ⥑ $ & {\textbackslash}LeftUpDownVector & Up Barb Left Down Barb Left Harpoon \\ \hline
U+02952 & $ ⥒ $ & {\textbackslash}LeftVectorBar & Leftwards Harpoon With Barb Up To Bar \\ \hline
U+02953 & $ ⥓ $ & {\textbackslash}RightVectorBar & Rightwards Harpoon With Barb Up To Bar \\ \hline
U+02954 & $ ⥔ $ & {\textbackslash}RightUpVectorBar & Upwards Harpoon With Barb Right To Bar \\ \hline
U+02955 & $ ⥕ $ & {\textbackslash}RightDownVectorBar & Downwards Harpoon With Barb Right To Bar \\ \hline
U+02956 & $ ⥖ $ & {\textbackslash}DownLeftVectorBar & Leftwards Harpoon With Barb Down To Bar \\ \hline
U+02957 & $ ⥗ $ & {\textbackslash}DownRightVectorBar & Rightwards Harpoon With Barb Down To Bar \\ \hline
U+02958 & $ ⥘ $ & {\textbackslash}LeftUpVectorBar & Upwards Harpoon With Barb Left To Bar \\ \hline
U+02959 & $ ⥙ $ & {\textbackslash}LeftDownVectorBar & Downwards Harpoon With Barb Left To Bar \\ \hline
U+0295A & $ ⥚ $ & {\textbackslash}LeftTeeVector & Leftwards Harpoon With Barb Up From Bar \\ \hline
U+0295B & $ ⥛ $ & {\textbackslash}RightTeeVector & Rightwards Harpoon With Barb Up From Bar \\ \hline
U+0295C & $ ⥜ $ & {\textbackslash}RightUpTeeVector & Upwards Harpoon With Barb Right From Bar \\ \hline
U+0295D & $ ⥝ $ & {\textbackslash}RightDownTeeVector & Downwards Harpoon With Barb Right From Bar \\ \hline
U+0295E & $ ⥞ $ & {\textbackslash}DownLeftTeeVector & Leftwards Harpoon With Barb Down From Bar \\ \hline
U+0295F & $ ⥟ $ & {\textbackslash}DownRightTeeVector & Rightwards Harpoon With Barb Down From Bar \\ \hline
U+02960 & $ ⥠ $ & {\textbackslash}LeftUpTeeVector & Upwards Harpoon With Barb Left From Bar \\ \hline
U+02961 & $ ⥡ $ & {\textbackslash}LeftDownTeeVector & Downwards Harpoon With Barb Left From Bar \\ \hline
U+02962 & $ ⥢ $ & {\textbackslash}leftharpoonsupdown & Leftwards Harpoon With Barb Up Above Leftwards Harpoon With Barb Down \\ \hline
U+02963 & $ ⥣ $ & {\textbackslash}upharpoonsleftright & Upwards Harpoon With Barb Left Beside Upwards Harpoon With Barb Right \\ \hline
U+02964 & $ ⥤ $ & {\textbackslash}rightharpoonsupdown & Rightwards Harpoon With Barb Up Above Rightwards Harpoon With Barb Down \\ \hline
U+02965 & $ ⥥ $ & {\textbackslash}downharpoonsleftright & Downwards Harpoon With Barb Left Beside Downwards Harpoon With Barb Right \\ \hline
U+02966 & $ ⥦ $ & {\textbackslash}leftrightharpoonsup & Leftwards Harpoon With Barb Up Above Rightwards Harpoon With Barb Up \\ \hline
U+02967 & $ ⥧ $ & {\textbackslash}leftrightharpoonsdown & Leftwards Harpoon With Barb Down Above Rightwards Harpoon With Barb Down \\ \hline
U+02968 & $ ⥨ $ & {\textbackslash}rightleftharpoonsup & Rightwards Harpoon With Barb Up Above Leftwards Harpoon With Barb Up \\ \hline
U+02969 & $ ⥩ $ & {\textbackslash}rightleftharpoonsdown & Rightwards Harpoon With Barb Down Above Leftwards Harpoon With Barb Down \\ \hline
U+0296A & $ ⥪ $ & {\textbackslash}leftharpoonupdash & Leftwards Harpoon With Barb Up Above Long Dash \\ \hline
U+0296B & $ ⥫ $ & {\textbackslash}dashleftharpoondown & Leftwards Harpoon With Barb Down Below Long Dash \\ \hline
U+0296C & $ ⥬ $ & {\textbackslash}rightharpoonupdash & Rightwards Harpoon With Barb Up Above Long Dash \\ \hline
U+0296D & $ ⥭ $ & {\textbackslash}dashrightharpoondown & Rightwards Harpoon With Barb Down Below Long Dash \\ \hline
U+0296E & $ ⥮ $ & {\textbackslash}UpEquilibrium & Upwards Harpoon With Barb Left Beside Downwards Harpoon With Barb Right \\ \hline
U+0296F & $ ⥯ $ & {\textbackslash}ReverseUpEquilibrium & Downwards Harpoon With Barb Left Beside Upwards Harpoon With Barb Right \\ \hline
U+02970 & $ ⥰ $ & {\textbackslash}RoundImplies & Right Double Arrow With Rounded Head \\ \hline
U+02980 & $ ⦀ $ & {\textbackslash}Vvert & Triple Vertical Bar Delimiter \\ \hline
U+02986 & $ ⦆ $ & {\textbackslash}Elroang & Right White Parenthesis \\ \hline
U+02999 & $ ⦙ $ & {\textbackslash}ddfnc & Dotted Fence \\ \hline
U+0299B & $ ⦛ $ & {\textbackslash}measuredangleleft & Measured Angle Opening Left \\ \hline
U+0299C & $ ⦜ $ & {\textbackslash}Angle & Right Angle Variant With Square \\ \hline
U+0299D & $ ⦝ $ & {\textbackslash}rightanglemdot & Measured Right Angle With Dot \\ \hline
U+0299E & $ ⦞ $ & {\textbackslash}angles & Angle With S Inside \\ \hline
U+0299F & $ ⦟ $ & {\textbackslash}angdnr & Acute Angle \\ \hline
U+029A0 & $ ⦠ $ & {\textbackslash}lpargt & Spherical Angle Opening Left \\ \hline
U+029A1 & $ ⦡ $ & {\textbackslash}sphericalangleup & Spherical Angle Opening Up \\ \hline
U+029A2 & $ ⦢ $ & {\textbackslash}turnangle & Turned Angle \\ \hline
U+029A3 & $ ⦣ $ & {\textbackslash}revangle & Reversed Angle \\ \hline
U+029A4 & $ ⦤ $ & {\textbackslash}angleubar & Angle With Underbar \\ \hline
U+029A5 & $ ⦥ $ & {\textbackslash}revangleubar & Reversed Angle With Underbar \\ \hline
U+029A6 & $ ⦦ $ & {\textbackslash}wideangledown & Oblique Angle Opening Up \\ \hline
U+029A7 & $ ⦧ $ & {\textbackslash}wideangleup & Oblique Angle Opening Down \\ \hline
U+029A8 & $ ⦨ $ & {\textbackslash}measanglerutone & Measured Angle With Open Arm Ending In Arrow Pointing Up And Right \\ \hline
U+029A9 & $ ⦩ $ & {\textbackslash}measanglelutonw & Measured Angle With Open Arm Ending In Arrow Pointing Up And Left \\ \hline
U+029AA & $ ⦪ $ & {\textbackslash}measanglerdtose & Measured Angle With Open Arm Ending In Arrow Pointing Down And Right \\ \hline
U+029AB & $ ⦫ $ & {\textbackslash}measangleldtosw & Measured Angle With Open Arm Ending In Arrow Pointing Down And Left \\ \hline
U+029AC & $ ⦬ $ & {\textbackslash}measangleurtone & Measured Angle With Open Arm Ending In Arrow Pointing Right And Up \\ \hline
U+029AD & $ ⦭ $ & {\textbackslash}measangleultonw & Measured Angle With Open Arm Ending In Arrow Pointing Left And Up \\ \hline
U+029AE & $ ⦮ $ & {\textbackslash}measangledrtose & Measured Angle With Open Arm Ending In Arrow Pointing Right And Down \\ \hline
U+029AF & $ ⦯ $ & {\textbackslash}measangledltosw & Measured Angle With Open Arm Ending In Arrow Pointing Left And Down \\ \hline
U+029B0 & $ ⦰ $ & {\textbackslash}revemptyset & Reversed Empty Set \\ \hline
U+029B1 & $ ⦱ $ & {\textbackslash}emptysetobar & Empty Set With Overbar \\ \hline
U+029B2 & $ ⦲ $ & {\textbackslash}emptysetocirc & Empty Set With Small Circle Above \\ \hline
U+029B3 & $ ⦳ $ & {\textbackslash}emptysetoarr & Empty Set With Right Arrow Above \\ \hline
U+029B4 & $ ⦴ $ & {\textbackslash}emptysetoarrl & Empty Set With Left Arrow Above \\ \hline
U+029B7 & $ ⦷ $ & {\textbackslash}circledparallel & Circled Parallel \\ \hline
U+029B8 & $ ⦸ $ & {\textbackslash}obslash & Circled Reverse Solidus \\ \hline
U+029BC & $ ⦼ $ & {\textbackslash}odotslashdot & Circled Anticlockwise-Rotated Division Sign \\ \hline
U+029BE & $ ⦾ $ & {\textbackslash}circledwhitebullet & Circled White Bullet \\ \hline
U+029BF & $ ⦿ $ & {\textbackslash}circledbullet & Circled Bullet \\ \hline
U+029C0 & $ ⧀ $ & {\textbackslash}olessthan & Circled Less-Than \\ \hline
U+029C1 & $ ⧁ $ & {\textbackslash}ogreaterthan & Circled Greater-Than \\ \hline
U+029C4 & $ ⧄ $ & {\textbackslash}boxdiag & Squared Rising Diagonal Slash \\ \hline
U+029C5 & $ ⧅ $ & {\textbackslash}boxbslash & Squared Falling Diagonal Slash \\ \hline
U+029C6 & $ ⧆ $ & {\textbackslash}boxast & Squared Asterisk \\ \hline
U+029C7 & $ ⧇ $ & {\textbackslash}boxcircle & Squared Small Circle \\ \hline
U+029CA & $ ⧊ $ & {\textbackslash}Lap & Triangle With Dot Above \\ \hline
U+029CB & $ ⧋ $ & {\textbackslash}defas & Triangle With Underbar \\ \hline
U+029CF & $ ⧏ $ & {\textbackslash}LeftTriangleBar & Left Triangle Beside Vertical Bar \\ \hline
U+029CF + U+00338 & $ ⧏̸ $ & {\textbackslash}NotLeftTriangleBar & Left Triangle Beside Vertical Bar + Combining Long Solidus Overlay / Non-Spacing Long Slash Overlay \\ \hline
U+029D0 & $ ⧐ $ & {\textbackslash}RightTriangleBar & Vertical Bar Beside Right Triangle \\ \hline
U+029D0 + U+00338 & $ ⧐̸ $ & {\textbackslash}NotRightTriangleBar & Vertical Bar Beside Right Triangle + Combining Long Solidus Overlay / Non-Spacing Long Slash Overlay \\ \hline
U+029DF & $ ⧟ $ & {\textbackslash}dualmap & Double-Ended Multimap \\ \hline
U+029E1 & $ ⧡ $ & {\textbackslash}lrtriangleeq & Increases As \\ \hline
U+029E2 & $ ⧢ $ & {\textbackslash}shuffle & Shuffle Product \\ \hline
U+029E3 & $ ⧣ $ & {\textbackslash}eparsl & Equals Sign And Slanted Parallel \\ \hline
U+029E4 & $ ⧤ $ & {\textbackslash}smeparsl & Equals Sign And Slanted Parallel With Tilde Above \\ \hline
U+029E5 & $ ⧥ $ & {\textbackslash}eqvparsl & Identical To And Slanted Parallel \\ \hline
U+029EB & $ ⧫ $ & {\textbackslash}blacklozenge & Black Lozenge \\ \hline
U+029F4 & $ ⧴ $ & {\textbackslash}RuleDelayed & Rule-Delayed \\ \hline
U+029F6 & $ ⧶ $ & {\textbackslash}dsol & Solidus With Overbar \\ \hline
U+029F7 & $ ⧷ $ & {\textbackslash}rsolbar & Reverse Solidus With Horizontal Stroke \\ \hline
U+029FA & $ ⧺ $ & {\textbackslash}doubleplus & Double Plus \\ \hline
U+029FB & $ ⧻ $ & {\textbackslash}tripleplus & Triple Plus \\ \hline
U+02A00 & $ ⨀ $ & {\textbackslash}bigodot & N-Ary Circled Dot Operator \\ \hline
U+02A01 & $ ⨁ $ & {\textbackslash}bigoplus & N-Ary Circled Plus Operator \\ \hline
U+02A02 & $ ⨂ $ & {\textbackslash}bigotimes & N-Ary Circled Times Operator \\ \hline
U+02A03 & $ ⨃ $ & {\textbackslash}bigcupdot & N-Ary Union Operator With Dot \\ \hline
U+02A04 & $ ⨄ $ & {\textbackslash}biguplus & N-Ary Union Operator With Plus \\ \hline
U+02A05 & $ ⨅ $ & {\textbackslash}bigsqcap & N-Ary Square Intersection Operator \\ \hline
U+02A06 & $ ⨆ $ & {\textbackslash}bigsqcup & N-Ary Square Union Operator \\ \hline
U+02A07 & $ ⨇ $ & {\textbackslash}conjquant & Two Logical And Operator \\ \hline
U+02A08 & $ ⨈ $ & {\textbackslash}disjquant & Two Logical Or Operator \\ \hline
U+02A09 & $ ⨉ $ & {\textbackslash}bigtimes & N-Ary Times Operator \\ \hline
U+02A0A & $ ⨊ $ & {\textbackslash}modtwosum & Modulo Two Sum \\ \hline
U+02A0B & $ ⨋ $ & {\textbackslash}sumint & Summation With Integral \\ \hline
U+02A0C & $ ⨌ $ & {\textbackslash}iiiint & Quadruple Integral Operator \\ \hline
U+02A0D & $ ⨍ $ & {\textbackslash}intbar & Finite Part Integral \\ \hline
U+02A0E & $ ⨎ $ & {\textbackslash}intBar & Integral With Double Stroke \\ \hline
U+02A0F & $ ⨏ $ & {\textbackslash}clockoint & Integral Average With Slash \\ \hline
U+02A10 & $ ⨐ $ & {\textbackslash}cirfnint & Circulation Function \\ \hline
U+02A11 & $ ⨑ $ & {\textbackslash}awint & Anticlockwise Integration \\ \hline
U+02A12 & $ ⨒ $ & {\textbackslash}rppolint & Line Integration With Rectangular Path Around Pole \\ \hline
U+02A13 & $ ⨓ $ & {\textbackslash}scpolint & Line Integration With Semicircular Path Around Pole \\ \hline
U+02A14 & $ ⨔ $ & {\textbackslash}npolint & Line Integration Not Including The Pole \\ \hline
U+02A15 & $ ⨕ $ & {\textbackslash}pointint & Integral Around A Point Operator \\ \hline
U+02A16 & $ ⨖ $ & {\textbackslash}sqrint & Quaternion Integral Operator \\ \hline
U+02A18 & $ ⨘ $ & {\textbackslash}intx & Integral With Times Sign \\ \hline
U+02A19 & $ ⨙ $ & {\textbackslash}intcap & Integral With Intersection \\ \hline
U+02A1A & $ ⨚ $ & {\textbackslash}intcup & Integral With Union \\ \hline
U+02A1B & $ ⨛ $ & {\textbackslash}upint & Integral With Overbar \\ \hline
U+02A1C & $ ⨜ $ & {\textbackslash}lowint & Integral With Underbar \\ \hline
U+02A1D & $ ⨝ $ & {\textbackslash}Join, {\textbackslash}join & Join \\ \hline
U+02A22 & $ ⨢ $ & {\textbackslash}ringplus & Plus Sign With Small Circle Above \\ \hline
U+02A23 & $ ⨣ $ & {\textbackslash}plushat & Plus Sign With Circumflex Accent Above \\ \hline
U+02A24 & $ ⨤ $ & {\textbackslash}simplus & Plus Sign With Tilde Above \\ \hline
U+02A25 & $ ⨥ $ & {\textbackslash}plusdot & Plus Sign With Dot Below \\ \hline
U+02A26 & $ ⨦ $ & {\textbackslash}plussim & Plus Sign With Tilde Below \\ \hline
U+02A27 & $ ⨧ $ & {\textbackslash}plussubtwo & Plus Sign With Subscript Two \\ \hline
U+02A28 & $ ⨨ $ & {\textbackslash}plustrif & Plus Sign With Black Triangle \\ \hline
U+02A29 & $ ⨩ $ & {\textbackslash}commaminus & Minus Sign With Comma Above \\ \hline
U+02A2A & $ ⨪ $ & {\textbackslash}minusdot & Minus Sign With Dot Below \\ \hline
U+02A2B & $ ⨫ $ & {\textbackslash}minusfdots & Minus Sign With Falling Dots \\ \hline
U+02A2C & $ ⨬ $ & {\textbackslash}minusrdots & Minus Sign With Rising Dots \\ \hline
U+02A2D & $ ⨭ $ & {\textbackslash}opluslhrim & Plus Sign In Left Half Circle \\ \hline
U+02A2E & $ ⨮ $ & {\textbackslash}oplusrhrim & Plus Sign In Right Half Circle \\ \hline
U+02A2F & $ ⨯ $ & {\textbackslash}Times & Vector Or Cross Product \\ \hline
U+02A30 & $ ⨰ $ & {\textbackslash}dottimes & Multiplication Sign With Dot Above \\ \hline
U+02A31 & $ ⨱ $ & {\textbackslash}timesbar & Multiplication Sign With Underbar \\ \hline
U+02A32 & $ ⨲ $ & {\textbackslash}btimes & Semidirect Product With Bottom Closed \\ \hline
U+02A33 & $ ⨳ $ & {\textbackslash}smashtimes & Smash Product \\ \hline
U+02A34 & $ ⨴ $ & {\textbackslash}otimeslhrim & Multiplication Sign In Left Half Circle \\ \hline
U+02A35 & $ ⨵ $ & {\textbackslash}otimesrhrim & Multiplication Sign In Right Half Circle \\ \hline
U+02A36 & $ ⨶ $ & {\textbackslash}otimeshat & Circled Multiplication Sign With Circumflex Accent \\ \hline
U+02A37 & $ ⨷ $ & {\textbackslash}Otimes & Multiplication Sign In Double Circle \\ \hline
U+02A38 & $ ⨸ $ & {\textbackslash}odiv & Circled Division Sign \\ \hline
U+02A39 & $ ⨹ $ & {\textbackslash}triangleplus & Plus Sign In Triangle \\ \hline
U+02A3A & $ ⨺ $ & {\textbackslash}triangleminus & Minus Sign In Triangle \\ \hline
U+02A3B & $ ⨻ $ & {\textbackslash}triangletimes & Multiplication Sign In Triangle \\ \hline
U+02A3C & $ ⨼ $ & {\textbackslash}intprod & Interior Product \\ \hline
U+02A3D & $ ⨽ $ & {\textbackslash}intprodr & Righthand Interior Product \\ \hline
U+02A3F & $ ⨿ $ & {\textbackslash}amalg & Amalgamation Or Coproduct \\ \hline
U+02A40 & $ ⩀ $ & {\textbackslash}capdot & Intersection With Dot \\ \hline
U+02A41 & $ ⩁ $ & {\textbackslash}uminus & Union With Minus Sign \\ \hline
U+02A42 & $ ⩂ $ & {\textbackslash}barcup & Union With Overbar \\ \hline
U+02A43 & $ ⩃ $ & {\textbackslash}barcap & Intersection With Overbar \\ \hline
U+02A44 & $ ⩄ $ & {\textbackslash}capwedge & Intersection With Logical And \\ \hline
U+02A45 & $ ⩅ $ & {\textbackslash}cupvee & Union With Logical Or \\ \hline
U+02A4A & $ ⩊ $ & {\textbackslash}twocups & Union Beside And Joined With Union \\ \hline
U+02A4B & $ ⩋ $ & {\textbackslash}twocaps & Intersection Beside And Joined With Intersection \\ \hline
U+02A4C & $ ⩌ $ & {\textbackslash}closedvarcup & Closed Union With Serifs \\ \hline
U+02A4D & $ ⩍ $ & {\textbackslash}closedvarcap & Closed Intersection With Serifs \\ \hline
U+02A4E & $ ⩎ $ & {\textbackslash}Sqcap & Double Square Intersection \\ \hline
U+02A4F & $ ⩏ $ & {\textbackslash}Sqcup & Double Square Union \\ \hline
U+02A50 & $ ⩐ $ & {\textbackslash}closedvarcupsmashprod & Closed Union With Serifs And Smash Product \\ \hline
U+02A51 & $ ⩑ $ & {\textbackslash}wedgeodot & Logical And With Dot Above \\ \hline
U+02A52 & $ ⩒ $ & {\textbackslash}veeodot & Logical Or With Dot Above \\ \hline
U+02A53 & $ ⩓ $ & {\textbackslash}And & Double Logical And \\ \hline
U+02A54 & $ ⩔ $ & {\textbackslash}Or & Double Logical Or \\ \hline
U+02A55 & $ ⩕ $ & {\textbackslash}wedgeonwedge & Two Intersecting Logical And \\ \hline
U+02A56 & $ ⩖ $ & {\textbackslash}ElOr & Two Intersecting Logical Or \\ \hline
U+02A57 & $ ⩗ $ & {\textbackslash}bigslopedvee & Sloping Large Or \\ \hline
U+02A58 & $ ⩘ $ & {\textbackslash}bigslopedwedge & Sloping Large And \\ \hline
U+02A5A & $ ⩚ $ & {\textbackslash}wedgemidvert & Logical And With Middle Stem \\ \hline
U+02A5B & $ ⩛ $ & {\textbackslash}veemidvert & Logical Or With Middle Stem \\ \hline
U+02A5C & $ ⩜ $ & {\textbackslash}midbarwedge & Logical And With Horizontal Dash \\ \hline
U+02A5D & $ ⩝ $ & {\textbackslash}midbarvee & Logical Or With Horizontal Dash \\ \hline
U+02A5E & $ ⩞ $ & {\textbackslash}perspcorrespond & Logical And With Double Overbar \\ \hline
U+02A5F & $ ⩟ $ & {\textbackslash}minhat & Logical And With Underbar \\ \hline
U+02A60 & $ ⩠ $ & {\textbackslash}wedgedoublebar & Logical And With Double Underbar \\ \hline
U+02A61 & $ ⩡ $ & {\textbackslash}varveebar & Small Vee With Underbar \\ \hline
U+02A62 & $ ⩢ $ & {\textbackslash}doublebarvee & Logical Or With Double Overbar \\ \hline
U+02A63 & $ ⩣ $ & {\textbackslash}veedoublebar & Logical Or With Double Underbar \\ \hline
U+02A66 & $ ⩦ $ & {\textbackslash}eqdot & Equals Sign With Dot Below \\ \hline
U+02A67 & $ ⩧ $ & {\textbackslash}dotequiv & Identical With Dot Above \\ \hline
U+02A6A & $ ⩪ $ & {\textbackslash}dotsim & Tilde Operator With Dot Above \\ \hline
U+02A6B & $ ⩫ $ & {\textbackslash}simrdots & Tilde Operator With Rising Dots \\ \hline
U+02A6C & $ ⩬ $ & {\textbackslash}simminussim & Similar Minus Similar \\ \hline
U+02A6D & $ ⩭ $ & {\textbackslash}congdot & Congruent With Dot Above \\ \hline
U+02A6E & $ ⩮ $ & {\textbackslash}asteq & Equals With Asterisk \\ \hline
U+02A6F & $ ⩯ $ & {\textbackslash}hatapprox & Almost Equal To With Circumflex Accent \\ \hline
U+02A70 & $ ⩰ $ & {\textbackslash}approxeqq & Approximately Equal Or Equal To \\ \hline
U+02A71 & $ ⩱ $ & {\textbackslash}eqqplus & Equals Sign Above Plus Sign \\ \hline
U+02A72 & $ ⩲ $ & {\textbackslash}pluseqq & Plus Sign Above Equals Sign \\ \hline
U+02A73 & $ ⩳ $ & {\textbackslash}eqqsim & Equals Sign Above Tilde Operator \\ \hline
U+02A74 & $ ⩴ $ & {\textbackslash}Coloneq & Double Colon Equal \\ \hline
U+02A75 & $ ⩵ $ & {\textbackslash}Equal & Two Consecutive Equals Signs \\ \hline
U+02A76 & $ ⩶ $ & {\textbackslash}eqeqeq & Three Consecutive Equals Signs \\ \hline
U+02A77 & $ ⩷ $ & {\textbackslash}ddotseq & Equals Sign With Two Dots Above And Two Dots Below \\ \hline
U+02A78 & $ ⩸ $ & {\textbackslash}equivDD & Equivalent With Four Dots Above \\ \hline
U+02A79 & $ ⩹ $ & {\textbackslash}ltcir & Less-Than With Circle Inside \\ \hline
U+02A7A & $ ⩺ $ & {\textbackslash}gtcir & Greater-Than With Circle Inside \\ \hline
U+02A7B & $ ⩻ $ & {\textbackslash}ltquest & Less-Than With Question Mark Above \\ \hline
U+02A7C & $ ⩼ $ & {\textbackslash}gtquest & Greater-Than With Question Mark Above \\ \hline
U+02A7D & $ ⩽ $ & {\textbackslash}leqslant & Less-Than Or Slanted Equal To \\ \hline
U+02A7D + U+00338 & $ ⩽̸ $ & {\textbackslash}nleqslant & Less-Than Or Slanted Equal To + Combining Long Solidus Overlay / Non-Spacing Long Slash Overlay \\ \hline
U+02A7E & $ ⩾ $ & {\textbackslash}geqslant & Greater-Than Or Slanted Equal To \\ \hline
U+02A7E + U+00338 & $ ⩾̸ $ & {\textbackslash}ngeqslant & Greater-Than Or Slanted Equal To + Combining Long Solidus Overlay / Non-Spacing Long Slash Overlay \\ \hline
U+02A7F & $ ⩿ $ & {\textbackslash}lesdot & Less-Than Or Slanted Equal To With Dot Inside \\ \hline
U+02A80 & $ ⪀ $ & {\textbackslash}gesdot & Greater-Than Or Slanted Equal To With Dot Inside \\ \hline
U+02A81 & $ ⪁ $ & {\textbackslash}lesdoto & Less-Than Or Slanted Equal To With Dot Above \\ \hline
U+02A82 & $ ⪂ $ & {\textbackslash}gesdoto & Greater-Than Or Slanted Equal To With Dot Above \\ \hline
U+02A83 & $ ⪃ $ & {\textbackslash}lesdotor & Less-Than Or Slanted Equal To With Dot Above Right \\ \hline
U+02A84 & $ ⪄ $ & {\textbackslash}gesdotol & Greater-Than Or Slanted Equal To With Dot Above Left \\ \hline
U+02A85 & $ ⪅ $ & {\textbackslash}lessapprox & Less-Than Or Approximate \\ \hline
U+02A86 & $ ⪆ $ & {\textbackslash}gtrapprox & Greater-Than Or Approximate \\ \hline
U+02A87 & $ ⪇ $ & {\textbackslash}lneq & Less-Than And Single-Line Not Equal To \\ \hline
U+02A88 & $ ⪈ $ & {\textbackslash}gneq & Greater-Than And Single-Line Not Equal To \\ \hline
U+02A89 & $ ⪉ $ & {\textbackslash}lnapprox & Less-Than And Not Approximate \\ \hline
U+02A8A & $ ⪊ $ & {\textbackslash}gnapprox & Greater-Than And Not Approximate \\ \hline
U+02A8B & $ ⪋ $ & {\textbackslash}lesseqqgtr & Less-Than Above Double-Line Equal Above Greater-Than \\ \hline
U+02A8C & $ ⪌ $ & {\textbackslash}gtreqqless & Greater-Than Above Double-Line Equal Above Less-Than \\ \hline
U+02A8D & $ ⪍ $ & {\textbackslash}lsime & Less-Than Above Similar Or Equal \\ \hline
U+02A8E & $ ⪎ $ & {\textbackslash}gsime & Greater-Than Above Similar Or Equal \\ \hline
U+02A8F & $ ⪏ $ & {\textbackslash}lsimg & Less-Than Above Similar Above Greater-Than \\ \hline
U+02A90 & $ ⪐ $ & {\textbackslash}gsiml & Greater-Than Above Similar Above Less-Than \\ \hline
U+02A91 & $ ⪑ $ & {\textbackslash}lgE & Less-Than Above Greater-Than Above Double-Line Equal \\ \hline
U+02A92 & $ ⪒ $ & {\textbackslash}glE & Greater-Than Above Less-Than Above Double-Line Equal \\ \hline
U+02A93 & $ ⪓ $ & {\textbackslash}lesges & Less-Than Above Slanted Equal Above Greater-Than Above Slanted Equal \\ \hline
U+02A94 & $ ⪔ $ & {\textbackslash}gesles & Greater-Than Above Slanted Equal Above Less-Than Above Slanted Equal \\ \hline
U+02A95 & $ ⪕ $ & {\textbackslash}eqslantless & Slanted Equal To Or Less-Than \\ \hline
U+02A96 & $ ⪖ $ & {\textbackslash}eqslantgtr & Slanted Equal To Or Greater-Than \\ \hline
U+02A97 & $ ⪗ $ & {\textbackslash}elsdot & Slanted Equal To Or Less-Than With Dot Inside \\ \hline
U+02A98 & $ ⪘ $ & {\textbackslash}egsdot & Slanted Equal To Or Greater-Than With Dot Inside \\ \hline
U+02A99 & $ ⪙ $ & {\textbackslash}eqqless & Double-Line Equal To Or Less-Than \\ \hline
U+02A9A & $ ⪚ $ & {\textbackslash}eqqgtr & Double-Line Equal To Or Greater-Than \\ \hline
U+02A9B & $ ⪛ $ & {\textbackslash}eqqslantless & Double-Line Slanted Equal To Or Less-Than \\ \hline
U+02A9C & $ ⪜ $ & {\textbackslash}eqqslantgtr & Double-Line Slanted Equal To Or Greater-Than \\ \hline
U+02A9D & $ ⪝ $ & {\textbackslash}simless & Similar Or Less-Than \\ \hline
U+02A9E & $ ⪞ $ & {\textbackslash}simgtr & Similar Or Greater-Than \\ \hline
U+02A9F & $ ⪟ $ & {\textbackslash}simlE & Similar Above Less-Than Above Equals Sign \\ \hline
U+02AA0 & $ ⪠ $ & {\textbackslash}simgE & Similar Above Greater-Than Above Equals Sign \\ \hline
U+02AA1 & $ ⪡ $ & {\textbackslash}NestedLessLess & Double Nested Less-Than \\ \hline
U+02AA1 + U+00338 & $ ⪡̸ $ & {\textbackslash}NotNestedLessLess & Double Nested Less-Than + Combining Long Solidus Overlay / Non-Spacing Long Slash Overlay \\ \hline
U+02AA2 & $ ⪢ $ & {\textbackslash}NestedGreaterGreater & Double Nested Greater-Than \\ \hline
U+02AA2 + U+00338 & $ ⪢̸ $ & {\textbackslash}NotNestedGreaterGreater & Double Nested Greater-Than + Combining Long Solidus Overlay / Non-Spacing Long Slash Overlay \\ \hline
U+02AA3 & $ ⪣ $ & {\textbackslash}partialmeetcontraction & Double Nested Less-Than With Underbar \\ \hline
U+02AA4 & $ ⪤ $ & {\textbackslash}glj & Greater-Than Overlapping Less-Than \\ \hline
U+02AA5 & $ ⪥ $ & {\textbackslash}gla & Greater-Than Beside Less-Than \\ \hline
U+02AA6 & $ ⪦ $ & {\textbackslash}ltcc & Less-Than Closed By Curve \\ \hline
U+02AA7 & $ ⪧ $ & {\textbackslash}gtcc & Greater-Than Closed By Curve \\ \hline
U+02AA8 & $ ⪨ $ & {\textbackslash}lescc & Less-Than Closed By Curve Above Slanted Equal \\ \hline
U+02AA9 & $ ⪩ $ & {\textbackslash}gescc & Greater-Than Closed By Curve Above Slanted Equal \\ \hline
U+02AAA & $ ⪪ $ & {\textbackslash}smt & Smaller Than \\ \hline
U+02AAB & $ ⪫ $ & {\textbackslash}lat & Larger Than \\ \hline
U+02AAC & $ ⪬ $ & {\textbackslash}smte & Smaller Than Or Equal To \\ \hline
U+02AAD & $ ⪭ $ & {\textbackslash}late & Larger Than Or Equal To \\ \hline
U+02AAE & $ ⪮ $ & {\textbackslash}bumpeqq & Equals Sign With Bumpy Above \\ \hline
U+02AAF & $ ⪯ $ & {\textbackslash}preceq & Precedes Above Single-Line Equals Sign \\ \hline
U+02AAF + U+00338 & $ ⪯̸ $ & {\textbackslash}npreceq & Precedes Above Single-Line Equals Sign + Combining Long Solidus Overlay / Non-Spacing Long Slash Overlay \\ \hline
U+02AB0 & $ ⪰ $ & {\textbackslash}succeq & Succeeds Above Single-Line Equals Sign \\ \hline
U+02AB0 + U+00338 & $ ⪰̸ $ & {\textbackslash}nsucceq & Succeeds Above Single-Line Equals Sign + Combining Long Solidus Overlay / Non-Spacing Long Slash Overlay \\ \hline
U+02AB1 & $ ⪱ $ & {\textbackslash}precneq & Precedes Above Single-Line Not Equal To \\ \hline
U+02AB2 & $ ⪲ $ & {\textbackslash}succneq & Succeeds Above Single-Line Not Equal To \\ \hline
U+02AB3 & $ ⪳ $ & {\textbackslash}preceqq & Precedes Above Equals Sign \\ \hline
U+02AB4 & $ ⪴ $ & {\textbackslash}succeqq & Succeeds Above Equals Sign \\ \hline
U+02AB5 & $ ⪵ $ & {\textbackslash}precneqq & Precedes Above Not Equal To \\ \hline
U+02AB6 & $ ⪶ $ & {\textbackslash}succneqq & Succeeds Above Not Equal To \\ \hline
U+02AB7 & $ ⪷ $ & {\textbackslash}precapprox & Precedes Above Almost Equal To \\ \hline
U+02AB8 & $ ⪸ $ & {\textbackslash}succapprox & Succeeds Above Almost Equal To \\ \hline
U+02AB9 & $ ⪹ $ & {\textbackslash}precnapprox & Precedes Above Not Almost Equal To \\ \hline
U+02ABA & $ ⪺ $ & {\textbackslash}succnapprox & Succeeds Above Not Almost Equal To \\ \hline
U+02ABB & $ ⪻ $ & {\textbackslash}Prec & Double Precedes \\ \hline
U+02ABC & $ ⪼ $ & {\textbackslash}Succ & Double Succeeds \\ \hline
U+02ABD & $ ⪽ $ & {\textbackslash}subsetdot & Subset With Dot \\ \hline
U+02ABE & $ ⪾ $ & {\textbackslash}supsetdot & Superset With Dot \\ \hline
U+02ABF & $ ⪿ $ & {\textbackslash}subsetplus & Subset With Plus Sign Below \\ \hline
U+02AC0 & $ ⫀ $ & {\textbackslash}supsetplus & Superset With Plus Sign Below \\ \hline
U+02AC1 & $ ⫁ $ & {\textbackslash}submult & Subset With Multiplication Sign Below \\ \hline
U+02AC2 & $ ⫂ $ & {\textbackslash}supmult & Superset With Multiplication Sign Below \\ \hline
U+02AC3 & $ ⫃ $ & {\textbackslash}subedot & Subset Of Or Equal To With Dot Above \\ \hline
U+02AC4 & $ ⫄ $ & {\textbackslash}supedot & Superset Of Or Equal To With Dot Above \\ \hline
U+02AC5 & $ ⫅ $ & {\textbackslash}subseteqq & Subset Of Above Equals Sign \\ \hline
U+02AC5 + U+00338 & $ ⫅̸ $ & {\textbackslash}nsubseteqq & Subset Of Above Equals Sign + Combining Long Solidus Overlay / Non-Spacing Long Slash Overlay \\ \hline
U+02AC6 & $ ⫆ $ & {\textbackslash}supseteqq & Superset Of Above Equals Sign \\ \hline
U+02AC6 + U+00338 & $ ⫆̸ $ & {\textbackslash}nsupseteqq & Superset Of Above Equals Sign + Combining Long Solidus Overlay / Non-Spacing Long Slash Overlay \\ \hline
U+02AC7 & $ ⫇ $ & {\textbackslash}subsim & Subset Of Above Tilde Operator \\ \hline
U+02AC8 & $ ⫈ $ & {\textbackslash}supsim & Superset Of Above Tilde Operator \\ \hline
U+02AC9 & $ ⫉ $ & {\textbackslash}subsetapprox & Subset Of Above Almost Equal To \\ \hline
U+02ACA & $ ⫊ $ & {\textbackslash}supsetapprox & Superset Of Above Almost Equal To \\ \hline
U+02ACB & $ ⫋ $ & {\textbackslash}subsetneqq & Subset Of Above Not Equal To \\ \hline
U+02ACC & $ ⫌ $ & {\textbackslash}supsetneqq & Superset Of Above Not Equal To \\ \hline
U+02ACD & $ ⫍ $ & {\textbackslash}lsqhook & Square Left Open Box Operator \\ \hline
U+02ACE & $ ⫎ $ & {\textbackslash}rsqhook & Square Right Open Box Operator \\ \hline
U+02ACF & $ ⫏ $ & {\textbackslash}csub & Closed Subset \\ \hline
U+02AD0 & $ ⫐ $ & {\textbackslash}csup & Closed Superset \\ \hline
U+02AD1 & $ ⫑ $ & {\textbackslash}csube & Closed Subset Or Equal To \\ \hline
U+02AD2 & $ ⫒ $ & {\textbackslash}csupe & Closed Superset Or Equal To \\ \hline
U+02AD3 & $ ⫓ $ & {\textbackslash}subsup & Subset Above Superset \\ \hline
U+02AD4 & $ ⫔ $ & {\textbackslash}supsub & Superset Above Subset \\ \hline
U+02AD5 & $ ⫕ $ & {\textbackslash}subsub & Subset Above Subset \\ \hline
U+02AD6 & $ ⫖ $ & {\textbackslash}supsup & Superset Above Superset \\ \hline
U+02AD7 & $ ⫗ $ & {\textbackslash}suphsub & Superset Beside Subset \\ \hline
U+02AD8 & $ ⫘ $ & {\textbackslash}supdsub & Superset Beside And Joined By Dash With Subset \\ \hline
U+02AD9 & $ ⫙ $ & {\textbackslash}forkv & Element Of Opening Downwards \\ \hline
U+02ADB & $ ⫛ $ & {\textbackslash}mlcp & Transversal Intersection \\ \hline
U+02ADC & $ ⫝̸ $ & {\textbackslash}forks & Forking \\ \hline
U+02ADD & $ ⫝ $ & {\textbackslash}forksnot & Nonforking \\ \hline
U+02AE3 & $ ⫣ $ & {\textbackslash}dashV & Double Vertical Bar Left Turnstile \\ \hline
U+02AE4 & $ ⫤ $ & {\textbackslash}Dashv & Vertical Bar Double Left Turnstile \\ \hline
U+02AF4 & $ ⫴ $ & {\textbackslash}interleave & Triple Vertical Bar Binary Relation \\ \hline
U+02AF6 & $ ⫶ $ & {\textbackslash}tdcol & Triple Colon Operator \\ \hline
U+02AF7 & $ ⫷ $ & {\textbackslash}lllnest & Triple Nested Less-Than \\ \hline
U+02AF8 & $ ⫸ $ & {\textbackslash}gggnest & Triple Nested Greater-Than \\ \hline
U+02AF9 & $ ⫹ $ & {\textbackslash}leqqslant & Double-Line Slanted Less-Than Or Equal To \\ \hline
U+02AFA & $ ⫺ $ & {\textbackslash}geqqslant & Double-Line Slanted Greater-Than Or Equal To \\ \hline
U+02B05 & {\EmojiFont ⬅} & {\textbackslash}:arrow\_left: & Leftwards Black Arrow \\ \hline
U+02B06 & {\EmojiFont ⬆} & {\textbackslash}:arrow\_up: & Upwards Black Arrow \\ \hline
U+02B07 & {\EmojiFont ⬇} & {\textbackslash}:arrow\_down: & Downwards Black Arrow \\ \hline
U+02B12 & $ ⬒ $ & {\textbackslash}squaretopblack & Square With Top Half Black \\ \hline
U+02B13 & $ ⬓ $ & {\textbackslash}squarebotblack & Square With Bottom Half Black \\ \hline
U+02B14 & $ ⬔ $ & {\textbackslash}squareurblack & Square With Upper Right Diagonal Half Black \\ \hline
U+02B15 & $ ⬕ $ & {\textbackslash}squarellblack & Square With Lower Left Diagonal Half Black \\ \hline
U+02B16 & $ ⬖ $ & {\textbackslash}diamondleftblack & Diamond With Left Half Black \\ \hline
U+02B17 & $ ⬗ $ & {\textbackslash}diamondrightblack & Diamond With Right Half Black \\ \hline
U+02B18 & $ ⬘ $ & {\textbackslash}diamondtopblack & Diamond With Top Half Black \\ \hline
U+02B19 & $ ⬙ $ & {\textbackslash}diamondbotblack & Diamond With Bottom Half Black \\ \hline
U+02B1A & $ ⬚ $ & {\textbackslash}dottedsquare & Dotted Square \\ \hline
U+02B1B & {\EmojiFont ⬛} & {\textbackslash}:black\_large\_square:, {\textbackslash}lgblksquare & Black Large Square \\ \hline
U+02B1C & {\EmojiFont ⬜} & {\textbackslash}:white\_large\_square:, {\textbackslash}lgwhtsquare & White Large Square \\ \hline
U+02B1D & $ ⬝ $ & {\textbackslash}vysmblksquare & Black Very Small Square \\ \hline
U+02B1E & $ ⬞ $ & {\textbackslash}vysmwhtsquare & White Very Small Square \\ \hline
U+02B1F & $ ⬟ $ & {\textbackslash}pentagonblack & Black Pentagon \\ \hline
U+02B20 & $ ⬠ $ & {\textbackslash}pentagon & White Pentagon \\ \hline
U+02B21 & $ ⬡ $ & {\textbackslash}varhexagon & White Hexagon \\ \hline
U+02B22 & $ ⬢ $ & {\textbackslash}varhexagonblack & Black Hexagon \\ \hline
U+02B23 & $ ⬣ $ & {\textbackslash}hexagonblack & Horizontal Black Hexagon \\ \hline
U+02B24 & $ ⬤ $ & {\textbackslash}lgblkcircle & Black Large Circle \\ \hline
U+02B25 & $ ⬥ $ & {\textbackslash}mdblkdiamond & Black Medium Diamond \\ \hline
U+02B26 & $ ⬦ $ & {\textbackslash}mdwhtdiamond & White Medium Diamond \\ \hline
U+02B27 & $ ⬧ $ & {\textbackslash}mdblklozenge & Black Medium Lozenge \\ \hline
U+02B28 & $ ⬨ $ & {\textbackslash}mdwhtlozenge & White Medium Lozenge \\ \hline
U+02B29 & $ ⬩ $ & {\textbackslash}smblkdiamond & Black Small Diamond \\ \hline
U+02B2A & $ ⬪ $ & {\textbackslash}smblklozenge & Black Small Lozenge \\ \hline
U+02B2B & $ ⬫ $ & {\textbackslash}smwhtlozenge & White Small Lozenge \\ \hline
U+02B2C & $ ⬬ $ & {\textbackslash}blkhorzoval & Black Horizontal Ellipse \\ \hline
U+02B2D & $ ⬭ $ & {\textbackslash}whthorzoval & White Horizontal Ellipse \\ \hline
U+02B2E & $ ⬮ $ & {\textbackslash}blkvertoval & Black Vertical Ellipse \\ \hline
U+02B2F & $ ⬯ $ & {\textbackslash}whtvertoval & White Vertical Ellipse \\ \hline
U+02B30 & $ ⬰ $ & {\textbackslash}circleonleftarrow & Left Arrow With Small Circle \\ \hline
U+02B31 & $ ⬱ $ & {\textbackslash}leftthreearrows & Three Leftwards Arrows \\ \hline
U+02B32 & $ ⬲ $ & {\textbackslash}leftarrowonoplus & Left Arrow With Circled Plus \\ \hline
U+02B33 & $ ⬳ $ & {\textbackslash}longleftsquigarrow & Long Leftwards Squiggle Arrow \\ \hline
U+02B34 & $ ⬴ $ & {\textbackslash}nvtwoheadleftarrow & Leftwards Two-Headed Arrow With Vertical Stroke \\ \hline
U+02B35 & $ ⬵ $ & {\textbackslash}nVtwoheadleftarrow & Leftwards Two-Headed Arrow With Double Vertical Stroke \\ \hline
U+02B36 & $ ⬶ $ & {\textbackslash}twoheadmapsfrom & Leftwards Two-Headed Arrow From Bar \\ \hline
U+02B37 & $ ⬷ $ & {\textbackslash}twoheadleftdbkarrow & Leftwards Two-Headed Triple Dash Arrow \\ \hline
U+02B38 & $ ⬸ $ & {\textbackslash}leftdotarrow & Leftwards Arrow With Dotted Stem \\ \hline
U+02B39 & $ ⬹ $ & {\textbackslash}nvleftarrowtail & Leftwards Arrow With Tail With Vertical Stroke \\ \hline
U+02B3A & $ ⬺ $ & {\textbackslash}nVleftarrowtail & Leftwards Arrow With Tail With Double Vertical Stroke \\ \hline
U+02B3B & $ ⬻ $ & {\textbackslash}twoheadleftarrowtail & Leftwards Two-Headed Arrow With Tail \\ \hline
U+02B3C & $ ⬼ $ & {\textbackslash}nvtwoheadleftarrowtail & Leftwards Two-Headed Arrow With Tail With Vertical Stroke \\ \hline
U+02B3D & $ ⬽ $ & {\textbackslash}nVtwoheadleftarrowtail & Leftwards Two-Headed Arrow With Tail With Double Vertical Stroke \\ \hline
U+02B3E & $ ⬾ $ & {\textbackslash}leftarrowx & Leftwards Arrow Through X \\ \hline
U+02B3F & $ ⬿ $ & {\textbackslash}leftcurvedarrow & Wave Arrow Pointing Directly Left \\ \hline
U+02B40 & $ ⭀ $ & {\textbackslash}equalleftarrow & Equals Sign Above Leftwards Arrow \\ \hline
U+02B41 & $ ⭁ $ & {\textbackslash}bsimilarleftarrow & Reverse Tilde Operator Above Leftwards Arrow \\ \hline
U+02B42 & $ ⭂ $ & {\textbackslash}leftarrowbackapprox & Leftwards Arrow Above Reverse Almost Equal To \\ \hline
U+02B43 & $ ⭃ $ & {\textbackslash}rightarrowgtr & Rightwards Arrow Through Greater-Than \\ \hline
U+02B44 & $ ⭄ $ & {\textbackslash}rightarrowsupset & Rightwards Arrow Through Superset \\ \hline
U+02B45 & $ ⭅ $ & {\textbackslash}LLeftarrow & Leftwards Quadruple Arrow \\ \hline
U+02B46 & $ ⭆ $ & {\textbackslash}RRightarrow & Rightwards Quadruple Arrow \\ \hline
U+02B47 & $ ⭇ $ & {\textbackslash}bsimilarrightarrow & Reverse Tilde Operator Above Rightwards Arrow \\ \hline
U+02B48 & $ ⭈ $ & {\textbackslash}rightarrowbackapprox & Rightwards Arrow Above Reverse Almost Equal To \\ \hline
U+02B49 & $ ⭉ $ & {\textbackslash}similarleftarrow & Tilde Operator Above Leftwards Arrow \\ \hline
U+02B4A & $ ⭊ $ & {\textbackslash}leftarrowapprox & Leftwards Arrow Above Almost Equal To \\ \hline
U+02B4B & $ ⭋ $ & {\textbackslash}leftarrowbsimilar & Leftwards Arrow Above Reverse Tilde Operator \\ \hline
U+02B4C & $ ⭌ $ & {\textbackslash}rightarrowbsimilar & Rightwards Arrow Above Reverse Tilde Operator \\ \hline
U+02B50 & {\EmojiFont ⭐} & {\textbackslash}:star:, {\textbackslash}medwhitestar & White Medium Star \\ \hline
U+02B51 & $ ⭑ $ & {\textbackslash}medblackstar & Black Small Star \\ \hline
U+02B52 & $ ⭒ $ & {\textbackslash}smwhitestar & White Small Star \\ \hline
U+02B53 & $ ⭓ $ & {\textbackslash}rightpentagonblack & Black Right-Pointing Pentagon \\ \hline
U+02B54 & $ ⭔ $ & {\textbackslash}rightpentagon & White Right-Pointing Pentagon \\ \hline
U+02B55 & {\EmojiFont ⭕} & {\textbackslash}:o: & Heavy Large Circle \\ \hline
U+02C7C & $ ⱼ $ & {\textbackslash}\_j & Latin Subscript Small Letter J \\ \hline
U+02C7D & $ ⱽ $ & {\textbackslash}{\textasciicircum}V & Modifier Letter Capital V \\ \hline
U+03012 & $ 〒 $ & {\textbackslash}postalmark & Postal Mark \\ \hline
U+03030 & {\EmojiFont 〰} & {\textbackslash}:wavy\_dash: & Wavy Dash \\ \hline
U+0303D & {\EmojiFont 〽} & {\textbackslash}:part\_alternation\_mark: & Part Alternation Mark \\ \hline
U+03297 & {\EmojiFont ㊗} & {\textbackslash}:congratulations: & Circled Ideograph Congratulation \\ \hline
U+03299 & {\EmojiFont ㊙} & {\textbackslash}:secret: & Circled Ideograph Secret \\ \hline
U+1D400 & $ 𝐀 $ & {\textbackslash}bfA & Mathematical Bold Capital A \\ \hline
U+1D401 & $ 𝐁 $ & {\textbackslash}bfB & Mathematical Bold Capital B \\ \hline
U+1D402 & $ 𝐂 $ & {\textbackslash}bfC & Mathematical Bold Capital C \\ \hline
U+1D403 & $ 𝐃 $ & {\textbackslash}bfD & Mathematical Bold Capital D \\ \hline
U+1D404 & $ 𝐄 $ & {\textbackslash}bfE & Mathematical Bold Capital E \\ \hline
U+1D405 & $ 𝐅 $ & {\textbackslash}bfF & Mathematical Bold Capital F \\ \hline
U+1D406 & $ 𝐆 $ & {\textbackslash}bfG & Mathematical Bold Capital G \\ \hline
U+1D407 & $ 𝐇 $ & {\textbackslash}bfH & Mathematical Bold Capital H \\ \hline
U+1D408 & $ 𝐈 $ & {\textbackslash}bfI & Mathematical Bold Capital I \\ \hline
U+1D409 & $ 𝐉 $ & {\textbackslash}bfJ & Mathematical Bold Capital J \\ \hline
U+1D40A & $ 𝐊 $ & {\textbackslash}bfK & Mathematical Bold Capital K \\ \hline
U+1D40B & $ 𝐋 $ & {\textbackslash}bfL & Mathematical Bold Capital L \\ \hline
U+1D40C & $ 𝐌 $ & {\textbackslash}bfM & Mathematical Bold Capital M \\ \hline
U+1D40D & $ 𝐍 $ & {\textbackslash}bfN & Mathematical Bold Capital N \\ \hline
U+1D40E & $ 𝐎 $ & {\textbackslash}bfO & Mathematical Bold Capital O \\ \hline
U+1D40F & $ 𝐏 $ & {\textbackslash}bfP & Mathematical Bold Capital P \\ \hline
U+1D410 & $ 𝐐 $ & {\textbackslash}bfQ & Mathematical Bold Capital Q \\ \hline
U+1D411 & $ 𝐑 $ & {\textbackslash}bfR & Mathematical Bold Capital R \\ \hline
U+1D412 & $ 𝐒 $ & {\textbackslash}bfS & Mathematical Bold Capital S \\ \hline
U+1D413 & $ 𝐓 $ & {\textbackslash}bfT & Mathematical Bold Capital T \\ \hline
U+1D414 & $ 𝐔 $ & {\textbackslash}bfU & Mathematical Bold Capital U \\ \hline
U+1D415 & $ 𝐕 $ & {\textbackslash}bfV & Mathematical Bold Capital V \\ \hline
U+1D416 & $ 𝐖 $ & {\textbackslash}bfW & Mathematical Bold Capital W \\ \hline
U+1D417 & $ 𝐗 $ & {\textbackslash}bfX & Mathematical Bold Capital X \\ \hline
U+1D418 & $ 𝐘 $ & {\textbackslash}bfY & Mathematical Bold Capital Y \\ \hline
U+1D419 & $ 𝐙 $ & {\textbackslash}bfZ & Mathematical Bold Capital Z \\ \hline
U+1D41A & $ 𝐚 $ & {\textbackslash}bfa & Mathematical Bold Small A \\ \hline
U+1D41B & $ 𝐛 $ & {\textbackslash}bfb & Mathematical Bold Small B \\ \hline
U+1D41C & $ 𝐜 $ & {\textbackslash}bfc & Mathematical Bold Small C \\ \hline
U+1D41D & $ 𝐝 $ & {\textbackslash}bfd & Mathematical Bold Small D \\ \hline
U+1D41E & $ 𝐞 $ & {\textbackslash}bfe & Mathematical Bold Small E \\ \hline
U+1D41F & $ 𝐟 $ & {\textbackslash}bff & Mathematical Bold Small F \\ \hline
U+1D420 & $ 𝐠 $ & {\textbackslash}bfg & Mathematical Bold Small G \\ \hline
U+1D421 & $ 𝐡 $ & {\textbackslash}bfh & Mathematical Bold Small H \\ \hline
U+1D422 & $ 𝐢 $ & {\textbackslash}bfi & Mathematical Bold Small I \\ \hline
U+1D423 & $ 𝐣 $ & {\textbackslash}bfj & Mathematical Bold Small J \\ \hline
U+1D424 & $ 𝐤 $ & {\textbackslash}bfk & Mathematical Bold Small K \\ \hline
U+1D425 & $ 𝐥 $ & {\textbackslash}bfl & Mathematical Bold Small L \\ \hline
U+1D426 & $ 𝐦 $ & {\textbackslash}bfm & Mathematical Bold Small M \\ \hline
U+1D427 & $ 𝐧 $ & {\textbackslash}bfn & Mathematical Bold Small N \\ \hline
U+1D428 & $ 𝐨 $ & {\textbackslash}bfo & Mathematical Bold Small O \\ \hline
U+1D429 & $ 𝐩 $ & {\textbackslash}bfp & Mathematical Bold Small P \\ \hline
U+1D42A & $ 𝐪 $ & {\textbackslash}bfq & Mathematical Bold Small Q \\ \hline
U+1D42B & $ 𝐫 $ & {\textbackslash}bfr & Mathematical Bold Small R \\ \hline
U+1D42C & $ 𝐬 $ & {\textbackslash}bfs & Mathematical Bold Small S \\ \hline
U+1D42D & $ 𝐭 $ & {\textbackslash}bft & Mathematical Bold Small T \\ \hline
U+1D42E & $ 𝐮 $ & {\textbackslash}bfu & Mathematical Bold Small U \\ \hline
U+1D42F & $ 𝐯 $ & {\textbackslash}bfv & Mathematical Bold Small V \\ \hline
U+1D430 & $ 𝐰 $ & {\textbackslash}bfw & Mathematical Bold Small W \\ \hline
U+1D431 & $ 𝐱 $ & {\textbackslash}bfx & Mathematical Bold Small X \\ \hline
U+1D432 & $ 𝐲 $ & {\textbackslash}bfy & Mathematical Bold Small Y \\ \hline
U+1D433 & $ 𝐳 $ & {\textbackslash}bfz & Mathematical Bold Small Z \\ \hline
U+1D434 & $ 𝐴 $ & {\textbackslash}itA & Mathematical Italic Capital A \\ \hline
U+1D435 & $ 𝐵 $ & {\textbackslash}itB & Mathematical Italic Capital B \\ \hline
U+1D436 & $ 𝐶 $ & {\textbackslash}itC & Mathematical Italic Capital C \\ \hline
U+1D437 & $ 𝐷 $ & {\textbackslash}itD & Mathematical Italic Capital D \\ \hline
U+1D438 & $ 𝐸 $ & {\textbackslash}itE & Mathematical Italic Capital E \\ \hline
U+1D439 & $ 𝐹 $ & {\textbackslash}itF & Mathematical Italic Capital F \\ \hline
U+1D43A & $ 𝐺 $ & {\textbackslash}itG & Mathematical Italic Capital G \\ \hline
U+1D43B & $ 𝐻 $ & {\textbackslash}itH & Mathematical Italic Capital H \\ \hline
U+1D43C & $ 𝐼 $ & {\textbackslash}itI & Mathematical Italic Capital I \\ \hline
U+1D43D & $ 𝐽 $ & {\textbackslash}itJ & Mathematical Italic Capital J \\ \hline
U+1D43E & $ 𝐾 $ & {\textbackslash}itK & Mathematical Italic Capital K \\ \hline
U+1D43F & $ 𝐿 $ & {\textbackslash}itL & Mathematical Italic Capital L \\ \hline
U+1D440 & $ 𝑀 $ & {\textbackslash}itM & Mathematical Italic Capital M \\ \hline
U+1D441 & $ 𝑁 $ & {\textbackslash}itN & Mathematical Italic Capital N \\ \hline
U+1D442 & $ 𝑂 $ & {\textbackslash}itO & Mathematical Italic Capital O \\ \hline
U+1D443 & $ 𝑃 $ & {\textbackslash}itP & Mathematical Italic Capital P \\ \hline
U+1D444 & $ 𝑄 $ & {\textbackslash}itQ & Mathematical Italic Capital Q \\ \hline
U+1D445 & $ 𝑅 $ & {\textbackslash}itR & Mathematical Italic Capital R \\ \hline
U+1D446 & $ 𝑆 $ & {\textbackslash}itS & Mathematical Italic Capital S \\ \hline
U+1D447 & $ 𝑇 $ & {\textbackslash}itT & Mathematical Italic Capital T \\ \hline
U+1D448 & $ 𝑈 $ & {\textbackslash}itU & Mathematical Italic Capital U \\ \hline
U+1D449 & $ 𝑉 $ & {\textbackslash}itV & Mathematical Italic Capital V \\ \hline
U+1D44A & $ 𝑊 $ & {\textbackslash}itW & Mathematical Italic Capital W \\ \hline
U+1D44B & $ 𝑋 $ & {\textbackslash}itX & Mathematical Italic Capital X \\ \hline
U+1D44C & $ 𝑌 $ & {\textbackslash}itY & Mathematical Italic Capital Y \\ \hline
U+1D44D & $ 𝑍 $ & {\textbackslash}itZ & Mathematical Italic Capital Z \\ \hline
U+1D44E & $ 𝑎 $ & {\textbackslash}ita & Mathematical Italic Small A \\ \hline
U+1D44F & $ 𝑏 $ & {\textbackslash}itb & Mathematical Italic Small B \\ \hline
U+1D450 & $ 𝑐 $ & {\textbackslash}itc & Mathematical Italic Small C \\ \hline
U+1D451 & $ 𝑑 $ & {\textbackslash}itd & Mathematical Italic Small D \\ \hline
U+1D452 & $ 𝑒 $ & {\textbackslash}ite & Mathematical Italic Small E \\ \hline
U+1D453 & $ 𝑓 $ & {\textbackslash}itf & Mathematical Italic Small F \\ \hline
U+1D454 & $ 𝑔 $ & {\textbackslash}itg & Mathematical Italic Small G \\ \hline
U+1D456 & $ 𝑖 $ & {\textbackslash}iti & Mathematical Italic Small I \\ \hline
U+1D457 & $ 𝑗 $ & {\textbackslash}itj & Mathematical Italic Small J \\ \hline
U+1D458 & $ 𝑘 $ & {\textbackslash}itk & Mathematical Italic Small K \\ \hline
U+1D459 & $ 𝑙 $ & {\textbackslash}itl & Mathematical Italic Small L \\ \hline
U+1D45A & $ 𝑚 $ & {\textbackslash}itm & Mathematical Italic Small M \\ \hline
U+1D45B & $ 𝑛 $ & {\textbackslash}itn & Mathematical Italic Small N \\ \hline
U+1D45C & $ 𝑜 $ & {\textbackslash}ito & Mathematical Italic Small O \\ \hline
U+1D45D & $ 𝑝 $ & {\textbackslash}itp & Mathematical Italic Small P \\ \hline
U+1D45E & $ 𝑞 $ & {\textbackslash}itq & Mathematical Italic Small Q \\ \hline
U+1D45F & $ 𝑟 $ & {\textbackslash}itr & Mathematical Italic Small R \\ \hline
U+1D460 & $ 𝑠 $ & {\textbackslash}its & Mathematical Italic Small S \\ \hline
U+1D461 & $ 𝑡 $ & {\textbackslash}itt & Mathematical Italic Small T \\ \hline
U+1D462 & $ 𝑢 $ & {\textbackslash}itu & Mathematical Italic Small U \\ \hline
U+1D463 & $ 𝑣 $ & {\textbackslash}itv & Mathematical Italic Small V \\ \hline
U+1D464 & $ 𝑤 $ & {\textbackslash}itw & Mathematical Italic Small W \\ \hline
U+1D465 & $ 𝑥 $ & {\textbackslash}itx & Mathematical Italic Small X \\ \hline
U+1D466 & $ 𝑦 $ & {\textbackslash}ity & Mathematical Italic Small Y \\ \hline
U+1D467 & $ 𝑧 $ & {\textbackslash}itz & Mathematical Italic Small Z \\ \hline
U+1D468 & $ 𝑨 $ & {\textbackslash}biA & Mathematical Bold Italic Capital A \\ \hline
U+1D469 & $ 𝑩 $ & {\textbackslash}biB & Mathematical Bold Italic Capital B \\ \hline
U+1D46A & $ 𝑪 $ & {\textbackslash}biC & Mathematical Bold Italic Capital C \\ \hline
U+1D46B & $ 𝑫 $ & {\textbackslash}biD & Mathematical Bold Italic Capital D \\ \hline
U+1D46C & $ 𝑬 $ & {\textbackslash}biE & Mathematical Bold Italic Capital E \\ \hline
U+1D46D & $ 𝑭 $ & {\textbackslash}biF & Mathematical Bold Italic Capital F \\ \hline
U+1D46E & $ 𝑮 $ & {\textbackslash}biG & Mathematical Bold Italic Capital G \\ \hline
U+1D46F & $ 𝑯 $ & {\textbackslash}biH & Mathematical Bold Italic Capital H \\ \hline
U+1D470 & $ 𝑰 $ & {\textbackslash}biI & Mathematical Bold Italic Capital I \\ \hline
U+1D471 & $ 𝑱 $ & {\textbackslash}biJ & Mathematical Bold Italic Capital J \\ \hline
U+1D472 & $ 𝑲 $ & {\textbackslash}biK & Mathematical Bold Italic Capital K \\ \hline
U+1D473 & $ 𝑳 $ & {\textbackslash}biL & Mathematical Bold Italic Capital L \\ \hline
U+1D474 & $ 𝑴 $ & {\textbackslash}biM & Mathematical Bold Italic Capital M \\ \hline
U+1D475 & $ 𝑵 $ & {\textbackslash}biN & Mathematical Bold Italic Capital N \\ \hline
U+1D476 & $ 𝑶 $ & {\textbackslash}biO & Mathematical Bold Italic Capital O \\ \hline
U+1D477 & $ 𝑷 $ & {\textbackslash}biP & Mathematical Bold Italic Capital P \\ \hline
U+1D478 & $ 𝑸 $ & {\textbackslash}biQ & Mathematical Bold Italic Capital Q \\ \hline
U+1D479 & $ 𝑹 $ & {\textbackslash}biR & Mathematical Bold Italic Capital R \\ \hline
U+1D47A & $ 𝑺 $ & {\textbackslash}biS & Mathematical Bold Italic Capital S \\ \hline
U+1D47B & $ 𝑻 $ & {\textbackslash}biT & Mathematical Bold Italic Capital T \\ \hline
U+1D47C & $ 𝑼 $ & {\textbackslash}biU & Mathematical Bold Italic Capital U \\ \hline
U+1D47D & $ 𝑽 $ & {\textbackslash}biV & Mathematical Bold Italic Capital V \\ \hline
U+1D47E & $ 𝑾 $ & {\textbackslash}biW & Mathematical Bold Italic Capital W \\ \hline
U+1D47F & $ 𝑿 $ & {\textbackslash}biX & Mathematical Bold Italic Capital X \\ \hline
U+1D480 & $ 𝒀 $ & {\textbackslash}biY & Mathematical Bold Italic Capital Y \\ \hline
U+1D481 & $ 𝒁 $ & {\textbackslash}biZ & Mathematical Bold Italic Capital Z \\ \hline
U+1D482 & $ 𝒂 $ & {\textbackslash}bia & Mathematical Bold Italic Small A \\ \hline
U+1D483 & $ 𝒃 $ & {\textbackslash}bib & Mathematical Bold Italic Small B \\ \hline
U+1D484 & $ 𝒄 $ & {\textbackslash}bic & Mathematical Bold Italic Small C \\ \hline
U+1D485 & $ 𝒅 $ & {\textbackslash}bid & Mathematical Bold Italic Small D \\ \hline
U+1D486 & $ 𝒆 $ & {\textbackslash}bie & Mathematical Bold Italic Small E \\ \hline
U+1D487 & $ 𝒇 $ & {\textbackslash}bif & Mathematical Bold Italic Small F \\ \hline
U+1D488 & $ 𝒈 $ & {\textbackslash}big & Mathematical Bold Italic Small G \\ \hline
U+1D489 & $ 𝒉 $ & {\textbackslash}bih & Mathematical Bold Italic Small H \\ \hline
U+1D48A & $ 𝒊 $ & {\textbackslash}bii & Mathematical Bold Italic Small I \\ \hline
U+1D48B & $ 𝒋 $ & {\textbackslash}bij & Mathematical Bold Italic Small J \\ \hline
U+1D48C & $ 𝒌 $ & {\textbackslash}bik & Mathematical Bold Italic Small K \\ \hline
U+1D48D & $ 𝒍 $ & {\textbackslash}bil & Mathematical Bold Italic Small L \\ \hline
U+1D48E & $ 𝒎 $ & {\textbackslash}bim & Mathematical Bold Italic Small M \\ \hline
U+1D48F & $ 𝒏 $ & {\textbackslash}bin & Mathematical Bold Italic Small N \\ \hline
U+1D490 & $ 𝒐 $ & {\textbackslash}bio & Mathematical Bold Italic Small O \\ \hline
U+1D491 & $ 𝒑 $ & {\textbackslash}bip & Mathematical Bold Italic Small P \\ \hline
U+1D492 & $ 𝒒 $ & {\textbackslash}biq & Mathematical Bold Italic Small Q \\ \hline
U+1D493 & $ 𝒓 $ & {\textbackslash}bir & Mathematical Bold Italic Small R \\ \hline
U+1D494 & $ 𝒔 $ & {\textbackslash}bis & Mathematical Bold Italic Small S \\ \hline
U+1D495 & $ 𝒕 $ & {\textbackslash}bit & Mathematical Bold Italic Small T \\ \hline
U+1D496 & $ 𝒖 $ & {\textbackslash}biu & Mathematical Bold Italic Small U \\ \hline
U+1D497 & $ 𝒗 $ & {\textbackslash}biv & Mathematical Bold Italic Small V \\ \hline
U+1D498 & $ 𝒘 $ & {\textbackslash}biw & Mathematical Bold Italic Small W \\ \hline
U+1D499 & $ 𝒙 $ & {\textbackslash}bix & Mathematical Bold Italic Small X \\ \hline
U+1D49A & $ 𝒚 $ & {\textbackslash}biy & Mathematical Bold Italic Small Y \\ \hline
U+1D49B & $ 𝒛 $ & {\textbackslash}biz & Mathematical Bold Italic Small Z \\ \hline
U+1D49C & $ 𝒜 $ & {\textbackslash}scrA & Mathematical Script Capital A \\ \hline
U+1D49E & $ 𝒞 $ & {\textbackslash}scrC & Mathematical Script Capital C \\ \hline
U+1D49F & $ 𝒟 $ & {\textbackslash}scrD & Mathematical Script Capital D \\ \hline
U+1D4A2 & $ 𝒢 $ & {\textbackslash}scrG & Mathematical Script Capital G \\ \hline
U+1D4A5 & $ 𝒥 $ & {\textbackslash}scrJ & Mathematical Script Capital J \\ \hline
U+1D4A6 & $ 𝒦 $ & {\textbackslash}scrK & Mathematical Script Capital K \\ \hline
U+1D4A9 & $ 𝒩 $ & {\textbackslash}scrN & Mathematical Script Capital N \\ \hline
U+1D4AA & $ 𝒪 $ & {\textbackslash}scrO & Mathematical Script Capital O \\ \hline
U+1D4AB & $ 𝒫 $ & {\textbackslash}scrP & Mathematical Script Capital P \\ \hline
U+1D4AC & $ 𝒬 $ & {\textbackslash}scrQ & Mathematical Script Capital Q \\ \hline
U+1D4AE & $ 𝒮 $ & {\textbackslash}scrS & Mathematical Script Capital S \\ \hline
U+1D4AF & $ 𝒯 $ & {\textbackslash}scrT & Mathematical Script Capital T \\ \hline
U+1D4B0 & $ 𝒰 $ & {\textbackslash}scrU & Mathematical Script Capital U \\ \hline
U+1D4B1 & $ 𝒱 $ & {\textbackslash}scrV & Mathematical Script Capital V \\ \hline
U+1D4B2 & $ 𝒲 $ & {\textbackslash}scrW & Mathematical Script Capital W \\ \hline
U+1D4B3 & $ 𝒳 $ & {\textbackslash}scrX & Mathematical Script Capital X \\ \hline
U+1D4B4 & $ 𝒴 $ & {\textbackslash}scrY & Mathematical Script Capital Y \\ \hline
U+1D4B5 & $ 𝒵 $ & {\textbackslash}scrZ & Mathematical Script Capital Z \\ \hline
U+1D4B6 & $ 𝒶 $ & {\textbackslash}scra & Mathematical Script Small A \\ \hline
U+1D4B7 & $ 𝒷 $ & {\textbackslash}scrb & Mathematical Script Small B \\ \hline
U+1D4B8 & $ 𝒸 $ & {\textbackslash}scrc & Mathematical Script Small C \\ \hline
U+1D4B9 & $ 𝒹 $ & {\textbackslash}scrd & Mathematical Script Small D \\ \hline
U+1D4BB & $ 𝒻 $ & {\textbackslash}scrf & Mathematical Script Small F \\ \hline
U+1D4BD & $ 𝒽 $ & {\textbackslash}scrh & Mathematical Script Small H \\ \hline
U+1D4BE & $ 𝒾 $ & {\textbackslash}scri & Mathematical Script Small I \\ \hline
U+1D4BF & $ 𝒿 $ & {\textbackslash}scrj & Mathematical Script Small J \\ \hline
U+1D4C0 & $ 𝓀 $ & {\textbackslash}scrk & Mathematical Script Small K \\ \hline
U+1D4C1 & $ 𝓁 $ & {\textbackslash}scrl & Mathematical Script Small L \\ \hline
U+1D4C2 & $ 𝓂 $ & {\textbackslash}scrm & Mathematical Script Small M \\ \hline
U+1D4C3 & $ 𝓃 $ & {\textbackslash}scrn & Mathematical Script Small N \\ \hline
U+1D4C5 & $ 𝓅 $ & {\textbackslash}scrp & Mathematical Script Small P \\ \hline
U+1D4C6 & $ 𝓆 $ & {\textbackslash}scrq & Mathematical Script Small Q \\ \hline
U+1D4C7 & $ 𝓇 $ & {\textbackslash}scrr & Mathematical Script Small R \\ \hline
U+1D4C8 & $ 𝓈 $ & {\textbackslash}scrs & Mathematical Script Small S \\ \hline
U+1D4C9 & $ 𝓉 $ & {\textbackslash}scrt & Mathematical Script Small T \\ \hline
U+1D4CA & $ 𝓊 $ & {\textbackslash}scru & Mathematical Script Small U \\ \hline
U+1D4CB & $ 𝓋 $ & {\textbackslash}scrv & Mathematical Script Small V \\ \hline
U+1D4CC & $ 𝓌 $ & {\textbackslash}scrw & Mathematical Script Small W \\ \hline
U+1D4CD & $ 𝓍 $ & {\textbackslash}scrx & Mathematical Script Small X \\ \hline
U+1D4CE & $ 𝓎 $ & {\textbackslash}scry & Mathematical Script Small Y \\ \hline
U+1D4CF & $ 𝓏 $ & {\textbackslash}scrz & Mathematical Script Small Z \\ \hline
U+1D4D0 & $ 𝓐 $ & {\textbackslash}bscrA & Mathematical Bold Script Capital A \\ \hline
U+1D4D1 & $ 𝓑 $ & {\textbackslash}bscrB & Mathematical Bold Script Capital B \\ \hline
U+1D4D2 & $ 𝓒 $ & {\textbackslash}bscrC & Mathematical Bold Script Capital C \\ \hline
U+1D4D3 & $ 𝓓 $ & {\textbackslash}bscrD & Mathematical Bold Script Capital D \\ \hline
U+1D4D4 & $ 𝓔 $ & {\textbackslash}bscrE & Mathematical Bold Script Capital E \\ \hline
U+1D4D5 & $ 𝓕 $ & {\textbackslash}bscrF & Mathematical Bold Script Capital F \\ \hline
U+1D4D6 & $ 𝓖 $ & {\textbackslash}bscrG & Mathematical Bold Script Capital G \\ \hline
U+1D4D7 & $ 𝓗 $ & {\textbackslash}bscrH & Mathematical Bold Script Capital H \\ \hline
U+1D4D8 & $ 𝓘 $ & {\textbackslash}bscrI & Mathematical Bold Script Capital I \\ \hline
U+1D4D9 & $ 𝓙 $ & {\textbackslash}bscrJ & Mathematical Bold Script Capital J \\ \hline
U+1D4DA & $ 𝓚 $ & {\textbackslash}bscrK & Mathematical Bold Script Capital K \\ \hline
U+1D4DB & $ 𝓛 $ & {\textbackslash}bscrL & Mathematical Bold Script Capital L \\ \hline
U+1D4DC & $ 𝓜 $ & {\textbackslash}bscrM & Mathematical Bold Script Capital M \\ \hline
U+1D4DD & $ 𝓝 $ & {\textbackslash}bscrN & Mathematical Bold Script Capital N \\ \hline
U+1D4DE & $ 𝓞 $ & {\textbackslash}bscrO & Mathematical Bold Script Capital O \\ \hline
U+1D4DF & $ 𝓟 $ & {\textbackslash}bscrP & Mathematical Bold Script Capital P \\ \hline
U+1D4E0 & $ 𝓠 $ & {\textbackslash}bscrQ & Mathematical Bold Script Capital Q \\ \hline
U+1D4E1 & $ 𝓡 $ & {\textbackslash}bscrR & Mathematical Bold Script Capital R \\ \hline
U+1D4E2 & $ 𝓢 $ & {\textbackslash}bscrS & Mathematical Bold Script Capital S \\ \hline
U+1D4E3 & $ 𝓣 $ & {\textbackslash}bscrT & Mathematical Bold Script Capital T \\ \hline
U+1D4E4 & $ 𝓤 $ & {\textbackslash}bscrU & Mathematical Bold Script Capital U \\ \hline
U+1D4E5 & $ 𝓥 $ & {\textbackslash}bscrV & Mathematical Bold Script Capital V \\ \hline
U+1D4E6 & $ 𝓦 $ & {\textbackslash}bscrW & Mathematical Bold Script Capital W \\ \hline
U+1D4E7 & $ 𝓧 $ & {\textbackslash}bscrX & Mathematical Bold Script Capital X \\ \hline
U+1D4E8 & $ 𝓨 $ & {\textbackslash}bscrY & Mathematical Bold Script Capital Y \\ \hline
U+1D4E9 & $ 𝓩 $ & {\textbackslash}bscrZ & Mathematical Bold Script Capital Z \\ \hline
U+1D4EA & $ 𝓪 $ & {\textbackslash}bscra & Mathematical Bold Script Small A \\ \hline
U+1D4EB & $ 𝓫 $ & {\textbackslash}bscrb & Mathematical Bold Script Small B \\ \hline
U+1D4EC & $ 𝓬 $ & {\textbackslash}bscrc & Mathematical Bold Script Small C \\ \hline
U+1D4ED & $ 𝓭 $ & {\textbackslash}bscrd & Mathematical Bold Script Small D \\ \hline
U+1D4EE & $ 𝓮 $ & {\textbackslash}bscre & Mathematical Bold Script Small E \\ \hline
U+1D4EF & $ 𝓯 $ & {\textbackslash}bscrf & Mathematical Bold Script Small F \\ \hline
U+1D4F0 & $ 𝓰 $ & {\textbackslash}bscrg & Mathematical Bold Script Small G \\ \hline
U+1D4F1 & $ 𝓱 $ & {\textbackslash}bscrh & Mathematical Bold Script Small H \\ \hline
U+1D4F2 & $ 𝓲 $ & {\textbackslash}bscri & Mathematical Bold Script Small I \\ \hline
U+1D4F3 & $ 𝓳 $ & {\textbackslash}bscrj & Mathematical Bold Script Small J \\ \hline
U+1D4F4 & $ 𝓴 $ & {\textbackslash}bscrk & Mathematical Bold Script Small K \\ \hline
U+1D4F5 & $ 𝓵 $ & {\textbackslash}bscrl & Mathematical Bold Script Small L \\ \hline
U+1D4F6 & $ 𝓶 $ & {\textbackslash}bscrm & Mathematical Bold Script Small M \\ \hline
U+1D4F7 & $ 𝓷 $ & {\textbackslash}bscrn & Mathematical Bold Script Small N \\ \hline
U+1D4F8 & $ 𝓸 $ & {\textbackslash}bscro & Mathematical Bold Script Small O \\ \hline
U+1D4F9 & $ 𝓹 $ & {\textbackslash}bscrp & Mathematical Bold Script Small P \\ \hline
U+1D4FA & $ 𝓺 $ & {\textbackslash}bscrq & Mathematical Bold Script Small Q \\ \hline
U+1D4FB & $ 𝓻 $ & {\textbackslash}bscrr & Mathematical Bold Script Small R \\ \hline
U+1D4FC & $ 𝓼 $ & {\textbackslash}bscrs & Mathematical Bold Script Small S \\ \hline
U+1D4FD & $ 𝓽 $ & {\textbackslash}bscrt & Mathematical Bold Script Small T \\ \hline
U+1D4FE & $ 𝓾 $ & {\textbackslash}bscru & Mathematical Bold Script Small U \\ \hline
U+1D4FF & $ 𝓿 $ & {\textbackslash}bscrv & Mathematical Bold Script Small V \\ \hline
U+1D500 & $ 𝔀 $ & {\textbackslash}bscrw & Mathematical Bold Script Small W \\ \hline
U+1D501 & $ 𝔁 $ & {\textbackslash}bscrx & Mathematical Bold Script Small X \\ \hline
U+1D502 & $ 𝔂 $ & {\textbackslash}bscry & Mathematical Bold Script Small Y \\ \hline
U+1D503 & $ 𝔃 $ & {\textbackslash}bscrz & Mathematical Bold Script Small Z \\ \hline
U+1D504 & $ 𝔄 $ & {\textbackslash}frakA & Mathematical Fraktur Capital A \\ \hline
U+1D505 & $ 𝔅 $ & {\textbackslash}frakB & Mathematical Fraktur Capital B \\ \hline
U+1D507 & $ 𝔇 $ & {\textbackslash}frakD & Mathematical Fraktur Capital D \\ \hline
U+1D508 & $ 𝔈 $ & {\textbackslash}frakE & Mathematical Fraktur Capital E \\ \hline
U+1D509 & $ 𝔉 $ & {\textbackslash}frakF & Mathematical Fraktur Capital F \\ \hline
U+1D50A & $ 𝔊 $ & {\textbackslash}frakG & Mathematical Fraktur Capital G \\ \hline
U+1D50D & $ 𝔍 $ & {\textbackslash}frakJ & Mathematical Fraktur Capital J \\ \hline
U+1D50E & $ 𝔎 $ & {\textbackslash}frakK & Mathematical Fraktur Capital K \\ \hline
U+1D50F & $ 𝔏 $ & {\textbackslash}frakL & Mathematical Fraktur Capital L \\ \hline
U+1D510 & $ 𝔐 $ & {\textbackslash}frakM & Mathematical Fraktur Capital M \\ \hline
U+1D511 & $ 𝔑 $ & {\textbackslash}frakN & Mathematical Fraktur Capital N \\ \hline
U+1D512 & $ 𝔒 $ & {\textbackslash}frakO & Mathematical Fraktur Capital O \\ \hline
U+1D513 & $ 𝔓 $ & {\textbackslash}frakP & Mathematical Fraktur Capital P \\ \hline
U+1D514 & $ 𝔔 $ & {\textbackslash}frakQ & Mathematical Fraktur Capital Q \\ \hline
U+1D516 & $ 𝔖 $ & {\textbackslash}frakS & Mathematical Fraktur Capital S \\ \hline
U+1D517 & $ 𝔗 $ & {\textbackslash}frakT & Mathematical Fraktur Capital T \\ \hline
U+1D518 & $ 𝔘 $ & {\textbackslash}frakU & Mathematical Fraktur Capital U \\ \hline
U+1D519 & $ 𝔙 $ & {\textbackslash}frakV & Mathematical Fraktur Capital V \\ \hline
U+1D51A & $ 𝔚 $ & {\textbackslash}frakW & Mathematical Fraktur Capital W \\ \hline
U+1D51B & $ 𝔛 $ & {\textbackslash}frakX & Mathematical Fraktur Capital X \\ \hline
U+1D51C & $ 𝔜 $ & {\textbackslash}frakY & Mathematical Fraktur Capital Y \\ \hline
U+1D51E & $ 𝔞 $ & {\textbackslash}fraka & Mathematical Fraktur Small A \\ \hline
U+1D51F & $ 𝔟 $ & {\textbackslash}frakb & Mathematical Fraktur Small B \\ \hline
U+1D520 & $ 𝔠 $ & {\textbackslash}frakc & Mathematical Fraktur Small C \\ \hline
U+1D521 & $ 𝔡 $ & {\textbackslash}frakd & Mathematical Fraktur Small D \\ \hline
U+1D522 & $ 𝔢 $ & {\textbackslash}frake & Mathematical Fraktur Small E \\ \hline
U+1D523 & $ 𝔣 $ & {\textbackslash}frakf & Mathematical Fraktur Small F \\ \hline
U+1D524 & $ 𝔤 $ & {\textbackslash}frakg & Mathematical Fraktur Small G \\ \hline
U+1D525 & $ 𝔥 $ & {\textbackslash}frakh & Mathematical Fraktur Small H \\ \hline
U+1D526 & $ 𝔦 $ & {\textbackslash}fraki & Mathematical Fraktur Small I \\ \hline
U+1D527 & $ 𝔧 $ & {\textbackslash}frakj & Mathematical Fraktur Small J \\ \hline
U+1D528 & $ 𝔨 $ & {\textbackslash}frakk & Mathematical Fraktur Small K \\ \hline
U+1D529 & $ 𝔩 $ & {\textbackslash}frakl & Mathematical Fraktur Small L \\ \hline
U+1D52A & $ 𝔪 $ & {\textbackslash}frakm & Mathematical Fraktur Small M \\ \hline
U+1D52B & $ 𝔫 $ & {\textbackslash}frakn & Mathematical Fraktur Small N \\ \hline
U+1D52C & $ 𝔬 $ & {\textbackslash}frako & Mathematical Fraktur Small O \\ \hline
U+1D52D & $ 𝔭 $ & {\textbackslash}frakp & Mathematical Fraktur Small P \\ \hline
U+1D52E & $ 𝔮 $ & {\textbackslash}frakq & Mathematical Fraktur Small Q \\ \hline
U+1D52F & $ 𝔯 $ & {\textbackslash}frakr & Mathematical Fraktur Small R \\ \hline
U+1D530 & $ 𝔰 $ & {\textbackslash}fraks & Mathematical Fraktur Small S \\ \hline
U+1D531 & $ 𝔱 $ & {\textbackslash}frakt & Mathematical Fraktur Small T \\ \hline
U+1D532 & $ 𝔲 $ & {\textbackslash}fraku & Mathematical Fraktur Small U \\ \hline
U+1D533 & $ 𝔳 $ & {\textbackslash}frakv & Mathematical Fraktur Small V \\ \hline
U+1D534 & $ 𝔴 $ & {\textbackslash}frakw & Mathematical Fraktur Small W \\ \hline
U+1D535 & $ 𝔵 $ & {\textbackslash}frakx & Mathematical Fraktur Small X \\ \hline
U+1D536 & $ 𝔶 $ & {\textbackslash}fraky & Mathematical Fraktur Small Y \\ \hline
U+1D537 & $ 𝔷 $ & {\textbackslash}frakz & Mathematical Fraktur Small Z \\ \hline
U+1D538 & $ 𝔸 $ & {\textbackslash}bbA & Mathematical Double-Struck Capital A \\ \hline
U+1D539 & $ 𝔹 $ & {\textbackslash}bbB & Mathematical Double-Struck Capital B \\ \hline
U+1D53B & $ 𝔻 $ & {\textbackslash}bbD & Mathematical Double-Struck Capital D \\ \hline
U+1D53C & $ 𝔼 $ & {\textbackslash}bbE & Mathematical Double-Struck Capital E \\ \hline
U+1D53D & $ 𝔽 $ & {\textbackslash}bbF & Mathematical Double-Struck Capital F \\ \hline
U+1D53E & $ 𝔾 $ & {\textbackslash}bbG & Mathematical Double-Struck Capital G \\ \hline
U+1D540 & $ 𝕀 $ & {\textbackslash}bbI & Mathematical Double-Struck Capital I \\ \hline
U+1D541 & $ 𝕁 $ & {\textbackslash}bbJ & Mathematical Double-Struck Capital J \\ \hline
U+1D542 & $ 𝕂 $ & {\textbackslash}bbK & Mathematical Double-Struck Capital K \\ \hline
U+1D543 & $ 𝕃 $ & {\textbackslash}bbL & Mathematical Double-Struck Capital L \\ \hline
U+1D544 & $ 𝕄 $ & {\textbackslash}bbM & Mathematical Double-Struck Capital M \\ \hline
U+1D546 & $ 𝕆 $ & {\textbackslash}bbO & Mathematical Double-Struck Capital O \\ \hline
U+1D54A & $ 𝕊 $ & {\textbackslash}bbS & Mathematical Double-Struck Capital S \\ \hline
U+1D54B & $ 𝕋 $ & {\textbackslash}bbT & Mathematical Double-Struck Capital T \\ \hline
U+1D54C & $ 𝕌 $ & {\textbackslash}bbU & Mathematical Double-Struck Capital U \\ \hline
U+1D54D & $ 𝕍 $ & {\textbackslash}bbV & Mathematical Double-Struck Capital V \\ \hline
U+1D54E & $ 𝕎 $ & {\textbackslash}bbW & Mathematical Double-Struck Capital W \\ \hline
U+1D54F & $ 𝕏 $ & {\textbackslash}bbX & Mathematical Double-Struck Capital X \\ \hline
U+1D550 & $ 𝕐 $ & {\textbackslash}bbY & Mathematical Double-Struck Capital Y \\ \hline
U+1D552 & $ 𝕒 $ & {\textbackslash}bba & Mathematical Double-Struck Small A \\ \hline
U+1D553 & $ 𝕓 $ & {\textbackslash}bbb & Mathematical Double-Struck Small B \\ \hline
U+1D554 & $ 𝕔 $ & {\textbackslash}bbc & Mathematical Double-Struck Small C \\ \hline
U+1D555 & $ 𝕕 $ & {\textbackslash}bbd & Mathematical Double-Struck Small D \\ \hline
U+1D556 & $ 𝕖 $ & {\textbackslash}bbe & Mathematical Double-Struck Small E \\ \hline
U+1D557 & $ 𝕗 $ & {\textbackslash}bbf & Mathematical Double-Struck Small F \\ \hline
U+1D558 & $ 𝕘 $ & {\textbackslash}bbg & Mathematical Double-Struck Small G \\ \hline
U+1D559 & $ 𝕙 $ & {\textbackslash}bbh & Mathematical Double-Struck Small H \\ \hline
U+1D55A & $ 𝕚 $ & {\textbackslash}bbi & Mathematical Double-Struck Small I \\ \hline
U+1D55B & $ 𝕛 $ & {\textbackslash}bbj & Mathematical Double-Struck Small J \\ \hline
U+1D55C & $ 𝕜 $ & {\textbackslash}bbk & Mathematical Double-Struck Small K \\ \hline
U+1D55D & $ 𝕝 $ & {\textbackslash}bbl & Mathematical Double-Struck Small L \\ \hline
U+1D55E & $ 𝕞 $ & {\textbackslash}bbm & Mathematical Double-Struck Small M \\ \hline
U+1D55F & $ 𝕟 $ & {\textbackslash}bbn & Mathematical Double-Struck Small N \\ \hline
U+1D560 & $ 𝕠 $ & {\textbackslash}bbo & Mathematical Double-Struck Small O \\ \hline
U+1D561 & $ 𝕡 $ & {\textbackslash}bbp & Mathematical Double-Struck Small P \\ \hline
U+1D562 & $ 𝕢 $ & {\textbackslash}bbq & Mathematical Double-Struck Small Q \\ \hline
U+1D563 & $ 𝕣 $ & {\textbackslash}bbr & Mathematical Double-Struck Small R \\ \hline
U+1D564 & $ 𝕤 $ & {\textbackslash}bbs & Mathematical Double-Struck Small S \\ \hline
U+1D565 & $ 𝕥 $ & {\textbackslash}bbt & Mathematical Double-Struck Small T \\ \hline
U+1D566 & $ 𝕦 $ & {\textbackslash}bbu & Mathematical Double-Struck Small U \\ \hline
U+1D567 & $ 𝕧 $ & {\textbackslash}bbv & Mathematical Double-Struck Small V \\ \hline
U+1D568 & $ 𝕨 $ & {\textbackslash}bbw & Mathematical Double-Struck Small W \\ \hline
U+1D569 & $ 𝕩 $ & {\textbackslash}bbx & Mathematical Double-Struck Small X \\ \hline
U+1D56A & $ 𝕪 $ & {\textbackslash}bby & Mathematical Double-Struck Small Y \\ \hline
U+1D56B & $ 𝕫 $ & {\textbackslash}bbz & Mathematical Double-Struck Small Z \\ \hline
U+1D56C & $ 𝕬 $ & {\textbackslash}bfrakA & Mathematical Bold Fraktur Capital A \\ \hline
U+1D56D & $ 𝕭 $ & {\textbackslash}bfrakB & Mathematical Bold Fraktur Capital B \\ \hline
U+1D56E & $ 𝕮 $ & {\textbackslash}bfrakC & Mathematical Bold Fraktur Capital C \\ \hline
U+1D56F & $ 𝕯 $ & {\textbackslash}bfrakD & Mathematical Bold Fraktur Capital D \\ \hline
U+1D570 & $ 𝕰 $ & {\textbackslash}bfrakE & Mathematical Bold Fraktur Capital E \\ \hline
U+1D571 & $ 𝕱 $ & {\textbackslash}bfrakF & Mathematical Bold Fraktur Capital F \\ \hline
U+1D572 & $ 𝕲 $ & {\textbackslash}bfrakG & Mathematical Bold Fraktur Capital G \\ \hline
U+1D573 & $ 𝕳 $ & {\textbackslash}bfrakH & Mathematical Bold Fraktur Capital H \\ \hline
U+1D574 & $ 𝕴 $ & {\textbackslash}bfrakI & Mathematical Bold Fraktur Capital I \\ \hline
U+1D575 & $ 𝕵 $ & {\textbackslash}bfrakJ & Mathematical Bold Fraktur Capital J \\ \hline
U+1D576 & $ 𝕶 $ & {\textbackslash}bfrakK & Mathematical Bold Fraktur Capital K \\ \hline
U+1D577 & $ 𝕷 $ & {\textbackslash}bfrakL & Mathematical Bold Fraktur Capital L \\ \hline
U+1D578 & $ 𝕸 $ & {\textbackslash}bfrakM & Mathematical Bold Fraktur Capital M \\ \hline
U+1D579 & $ 𝕹 $ & {\textbackslash}bfrakN & Mathematical Bold Fraktur Capital N \\ \hline
U+1D57A & $ 𝕺 $ & {\textbackslash}bfrakO & Mathematical Bold Fraktur Capital O \\ \hline
U+1D57B & $ 𝕻 $ & {\textbackslash}bfrakP & Mathematical Bold Fraktur Capital P \\ \hline
U+1D57C & $ 𝕼 $ & {\textbackslash}bfrakQ & Mathematical Bold Fraktur Capital Q \\ \hline
U+1D57D & $ 𝕽 $ & {\textbackslash}bfrakR & Mathematical Bold Fraktur Capital R \\ \hline
U+1D57E & $ 𝕾 $ & {\textbackslash}bfrakS & Mathematical Bold Fraktur Capital S \\ \hline
U+1D57F & $ 𝕿 $ & {\textbackslash}bfrakT & Mathematical Bold Fraktur Capital T \\ \hline
U+1D580 & $ 𝖀 $ & {\textbackslash}bfrakU & Mathematical Bold Fraktur Capital U \\ \hline
U+1D581 & $ 𝖁 $ & {\textbackslash}bfrakV & Mathematical Bold Fraktur Capital V \\ \hline
U+1D582 & $ 𝖂 $ & {\textbackslash}bfrakW & Mathematical Bold Fraktur Capital W \\ \hline
U+1D583 & $ 𝖃 $ & {\textbackslash}bfrakX & Mathematical Bold Fraktur Capital X \\ \hline
U+1D584 & $ 𝖄 $ & {\textbackslash}bfrakY & Mathematical Bold Fraktur Capital Y \\ \hline
U+1D585 & $ 𝖅 $ & {\textbackslash}bfrakZ & Mathematical Bold Fraktur Capital Z \\ \hline
U+1D586 & $ 𝖆 $ & {\textbackslash}bfraka & Mathematical Bold Fraktur Small A \\ \hline
U+1D587 & $ 𝖇 $ & {\textbackslash}bfrakb & Mathematical Bold Fraktur Small B \\ \hline
U+1D588 & $ 𝖈 $ & {\textbackslash}bfrakc & Mathematical Bold Fraktur Small C \\ \hline
U+1D589 & $ 𝖉 $ & {\textbackslash}bfrakd & Mathematical Bold Fraktur Small D \\ \hline
U+1D58A & $ 𝖊 $ & {\textbackslash}bfrake & Mathematical Bold Fraktur Small E \\ \hline
U+1D58B & $ 𝖋 $ & {\textbackslash}bfrakf & Mathematical Bold Fraktur Small F \\ \hline
U+1D58C & $ 𝖌 $ & {\textbackslash}bfrakg & Mathematical Bold Fraktur Small G \\ \hline
U+1D58D & $ 𝖍 $ & {\textbackslash}bfrakh & Mathematical Bold Fraktur Small H \\ \hline
U+1D58E & $ 𝖎 $ & {\textbackslash}bfraki & Mathematical Bold Fraktur Small I \\ \hline
U+1D58F & $ 𝖏 $ & {\textbackslash}bfrakj & Mathematical Bold Fraktur Small J \\ \hline
U+1D590 & $ 𝖐 $ & {\textbackslash}bfrakk & Mathematical Bold Fraktur Small K \\ \hline
U+1D591 & $ 𝖑 $ & {\textbackslash}bfrakl & Mathematical Bold Fraktur Small L \\ \hline
U+1D592 & $ 𝖒 $ & {\textbackslash}bfrakm & Mathematical Bold Fraktur Small M \\ \hline
U+1D593 & $ 𝖓 $ & {\textbackslash}bfrakn & Mathematical Bold Fraktur Small N \\ \hline
U+1D594 & $ 𝖔 $ & {\textbackslash}bfrako & Mathematical Bold Fraktur Small O \\ \hline
U+1D595 & $ 𝖕 $ & {\textbackslash}bfrakp & Mathematical Bold Fraktur Small P \\ \hline
U+1D596 & $ 𝖖 $ & {\textbackslash}bfrakq & Mathematical Bold Fraktur Small Q \\ \hline
U+1D597 & $ 𝖗 $ & {\textbackslash}bfrakr & Mathematical Bold Fraktur Small R \\ \hline
U+1D598 & $ 𝖘 $ & {\textbackslash}bfraks & Mathematical Bold Fraktur Small S \\ \hline
U+1D599 & $ 𝖙 $ & {\textbackslash}bfrakt & Mathematical Bold Fraktur Small T \\ \hline
U+1D59A & $ 𝖚 $ & {\textbackslash}bfraku & Mathematical Bold Fraktur Small U \\ \hline
U+1D59B & $ 𝖛 $ & {\textbackslash}bfrakv & Mathematical Bold Fraktur Small V \\ \hline
U+1D59C & $ 𝖜 $ & {\textbackslash}bfrakw & Mathematical Bold Fraktur Small W \\ \hline
U+1D59D & $ 𝖝 $ & {\textbackslash}bfrakx & Mathematical Bold Fraktur Small X \\ \hline
U+1D59E & $ 𝖞 $ & {\textbackslash}bfraky & Mathematical Bold Fraktur Small Y \\ \hline
U+1D59F & $ 𝖟 $ & {\textbackslash}bfrakz & Mathematical Bold Fraktur Small Z \\ \hline
U+1D5A0 & $ 𝖠 $ & {\textbackslash}sansA & Mathematical Sans-Serif Capital A \\ \hline
U+1D5A1 & $ 𝖡 $ & {\textbackslash}sansB & Mathematical Sans-Serif Capital B \\ \hline
U+1D5A2 & $ 𝖢 $ & {\textbackslash}sansC & Mathematical Sans-Serif Capital C \\ \hline
U+1D5A3 & $ 𝖣 $ & {\textbackslash}sansD & Mathematical Sans-Serif Capital D \\ \hline
U+1D5A4 & $ 𝖤 $ & {\textbackslash}sansE & Mathematical Sans-Serif Capital E \\ \hline
U+1D5A5 & $ 𝖥 $ & {\textbackslash}sansF & Mathematical Sans-Serif Capital F \\ \hline
U+1D5A6 & $ 𝖦 $ & {\textbackslash}sansG & Mathematical Sans-Serif Capital G \\ \hline
U+1D5A7 & $ 𝖧 $ & {\textbackslash}sansH & Mathematical Sans-Serif Capital H \\ \hline
U+1D5A8 & $ 𝖨 $ & {\textbackslash}sansI & Mathematical Sans-Serif Capital I \\ \hline
U+1D5A9 & $ 𝖩 $ & {\textbackslash}sansJ & Mathematical Sans-Serif Capital J \\ \hline
U+1D5AA & $ 𝖪 $ & {\textbackslash}sansK & Mathematical Sans-Serif Capital K \\ \hline
U+1D5AB & $ 𝖫 $ & {\textbackslash}sansL & Mathematical Sans-Serif Capital L \\ \hline
U+1D5AC & $ 𝖬 $ & {\textbackslash}sansM & Mathematical Sans-Serif Capital M \\ \hline
U+1D5AD & $ 𝖭 $ & {\textbackslash}sansN & Mathematical Sans-Serif Capital N \\ \hline
U+1D5AE & $ 𝖮 $ & {\textbackslash}sansO & Mathematical Sans-Serif Capital O \\ \hline
U+1D5AF & $ 𝖯 $ & {\textbackslash}sansP & Mathematical Sans-Serif Capital P \\ \hline
U+1D5B0 & $ 𝖰 $ & {\textbackslash}sansQ & Mathematical Sans-Serif Capital Q \\ \hline
U+1D5B1 & $ 𝖱 $ & {\textbackslash}sansR & Mathematical Sans-Serif Capital R \\ \hline
U+1D5B2 & $ 𝖲 $ & {\textbackslash}sansS & Mathematical Sans-Serif Capital S \\ \hline
U+1D5B3 & $ 𝖳 $ & {\textbackslash}sansT & Mathematical Sans-Serif Capital T \\ \hline
U+1D5B4 & $ 𝖴 $ & {\textbackslash}sansU & Mathematical Sans-Serif Capital U \\ \hline
U+1D5B5 & $ 𝖵 $ & {\textbackslash}sansV & Mathematical Sans-Serif Capital V \\ \hline
U+1D5B6 & $ 𝖶 $ & {\textbackslash}sansW & Mathematical Sans-Serif Capital W \\ \hline
U+1D5B7 & $ 𝖷 $ & {\textbackslash}sansX & Mathematical Sans-Serif Capital X \\ \hline
U+1D5B8 & $ 𝖸 $ & {\textbackslash}sansY & Mathematical Sans-Serif Capital Y \\ \hline
U+1D5B9 & $ 𝖹 $ & {\textbackslash}sansZ & Mathematical Sans-Serif Capital Z \\ \hline
U+1D5BA & $ 𝖺 $ & {\textbackslash}sansa & Mathematical Sans-Serif Small A \\ \hline
U+1D5BB & $ 𝖻 $ & {\textbackslash}sansb & Mathematical Sans-Serif Small B \\ \hline
U+1D5BC & $ 𝖼 $ & {\textbackslash}sansc & Mathematical Sans-Serif Small C \\ \hline
U+1D5BD & $ 𝖽 $ & {\textbackslash}sansd & Mathematical Sans-Serif Small D \\ \hline
U+1D5BE & $ 𝖾 $ & {\textbackslash}sanse & Mathematical Sans-Serif Small E \\ \hline
U+1D5BF & $ 𝖿 $ & {\textbackslash}sansf & Mathematical Sans-Serif Small F \\ \hline
U+1D5C0 & $ 𝗀 $ & {\textbackslash}sansg & Mathematical Sans-Serif Small G \\ \hline
U+1D5C1 & $ 𝗁 $ & {\textbackslash}sansh & Mathematical Sans-Serif Small H \\ \hline
U+1D5C2 & $ 𝗂 $ & {\textbackslash}sansi & Mathematical Sans-Serif Small I \\ \hline
U+1D5C3 & $ 𝗃 $ & {\textbackslash}sansj & Mathematical Sans-Serif Small J \\ \hline
U+1D5C4 & $ 𝗄 $ & {\textbackslash}sansk & Mathematical Sans-Serif Small K \\ \hline
U+1D5C5 & $ 𝗅 $ & {\textbackslash}sansl & Mathematical Sans-Serif Small L \\ \hline
U+1D5C6 & $ 𝗆 $ & {\textbackslash}sansm & Mathematical Sans-Serif Small M \\ \hline
U+1D5C7 & $ 𝗇 $ & {\textbackslash}sansn & Mathematical Sans-Serif Small N \\ \hline
U+1D5C8 & $ 𝗈 $ & {\textbackslash}sanso & Mathematical Sans-Serif Small O \\ \hline
U+1D5C9 & $ 𝗉 $ & {\textbackslash}sansp & Mathematical Sans-Serif Small P \\ \hline
U+1D5CA & $ 𝗊 $ & {\textbackslash}sansq & Mathematical Sans-Serif Small Q \\ \hline
U+1D5CB & $ 𝗋 $ & {\textbackslash}sansr & Mathematical Sans-Serif Small R \\ \hline
U+1D5CC & $ 𝗌 $ & {\textbackslash}sanss & Mathematical Sans-Serif Small S \\ \hline
U+1D5CD & $ 𝗍 $ & {\textbackslash}sanst & Mathematical Sans-Serif Small T \\ \hline
U+1D5CE & $ 𝗎 $ & {\textbackslash}sansu & Mathematical Sans-Serif Small U \\ \hline
U+1D5CF & $ 𝗏 $ & {\textbackslash}sansv & Mathematical Sans-Serif Small V \\ \hline
U+1D5D0 & $ 𝗐 $ & {\textbackslash}sansw & Mathematical Sans-Serif Small W \\ \hline
U+1D5D1 & $ 𝗑 $ & {\textbackslash}sansx & Mathematical Sans-Serif Small X \\ \hline
U+1D5D2 & $ 𝗒 $ & {\textbackslash}sansy & Mathematical Sans-Serif Small Y \\ \hline
U+1D5D3 & $ 𝗓 $ & {\textbackslash}sansz & Mathematical Sans-Serif Small Z \\ \hline
U+1D5D4 & $ 𝗔 $ & {\textbackslash}bsansA & Mathematical Sans-Serif Bold Capital A \\ \hline
U+1D5D5 & $ 𝗕 $ & {\textbackslash}bsansB & Mathematical Sans-Serif Bold Capital B \\ \hline
U+1D5D6 & $ 𝗖 $ & {\textbackslash}bsansC & Mathematical Sans-Serif Bold Capital C \\ \hline
U+1D5D7 & $ 𝗗 $ & {\textbackslash}bsansD & Mathematical Sans-Serif Bold Capital D \\ \hline
U+1D5D8 & $ 𝗘 $ & {\textbackslash}bsansE & Mathematical Sans-Serif Bold Capital E \\ \hline
U+1D5D9 & $ 𝗙 $ & {\textbackslash}bsansF & Mathematical Sans-Serif Bold Capital F \\ \hline
U+1D5DA & $ 𝗚 $ & {\textbackslash}bsansG & Mathematical Sans-Serif Bold Capital G \\ \hline
U+1D5DB & $ 𝗛 $ & {\textbackslash}bsansH & Mathematical Sans-Serif Bold Capital H \\ \hline
U+1D5DC & $ 𝗜 $ & {\textbackslash}bsansI & Mathematical Sans-Serif Bold Capital I \\ \hline
U+1D5DD & $ 𝗝 $ & {\textbackslash}bsansJ & Mathematical Sans-Serif Bold Capital J \\ \hline
U+1D5DE & $ 𝗞 $ & {\textbackslash}bsansK & Mathematical Sans-Serif Bold Capital K \\ \hline
U+1D5DF & $ 𝗟 $ & {\textbackslash}bsansL & Mathematical Sans-Serif Bold Capital L \\ \hline
U+1D5E0 & $ 𝗠 $ & {\textbackslash}bsansM & Mathematical Sans-Serif Bold Capital M \\ \hline
U+1D5E1 & $ 𝗡 $ & {\textbackslash}bsansN & Mathematical Sans-Serif Bold Capital N \\ \hline
U+1D5E2 & $ 𝗢 $ & {\textbackslash}bsansO & Mathematical Sans-Serif Bold Capital O \\ \hline
U+1D5E3 & $ 𝗣 $ & {\textbackslash}bsansP & Mathematical Sans-Serif Bold Capital P \\ \hline
U+1D5E4 & $ 𝗤 $ & {\textbackslash}bsansQ & Mathematical Sans-Serif Bold Capital Q \\ \hline
U+1D5E5 & $ 𝗥 $ & {\textbackslash}bsansR & Mathematical Sans-Serif Bold Capital R \\ \hline
U+1D5E6 & $ 𝗦 $ & {\textbackslash}bsansS & Mathematical Sans-Serif Bold Capital S \\ \hline
U+1D5E7 & $ 𝗧 $ & {\textbackslash}bsansT & Mathematical Sans-Serif Bold Capital T \\ \hline
U+1D5E8 & $ 𝗨 $ & {\textbackslash}bsansU & Mathematical Sans-Serif Bold Capital U \\ \hline
U+1D5E9 & $ 𝗩 $ & {\textbackslash}bsansV & Mathematical Sans-Serif Bold Capital V \\ \hline
U+1D5EA & $ 𝗪 $ & {\textbackslash}bsansW & Mathematical Sans-Serif Bold Capital W \\ \hline
U+1D5EB & $ 𝗫 $ & {\textbackslash}bsansX & Mathematical Sans-Serif Bold Capital X \\ \hline
U+1D5EC & $ 𝗬 $ & {\textbackslash}bsansY & Mathematical Sans-Serif Bold Capital Y \\ \hline
U+1D5ED & $ 𝗭 $ & {\textbackslash}bsansZ & Mathematical Sans-Serif Bold Capital Z \\ \hline
U+1D5EE & $ 𝗮 $ & {\textbackslash}bsansa & Mathematical Sans-Serif Bold Small A \\ \hline
U+1D5EF & $ 𝗯 $ & {\textbackslash}bsansb & Mathematical Sans-Serif Bold Small B \\ \hline
U+1D5F0 & $ 𝗰 $ & {\textbackslash}bsansc & Mathematical Sans-Serif Bold Small C \\ \hline
U+1D5F1 & $ 𝗱 $ & {\textbackslash}bsansd & Mathematical Sans-Serif Bold Small D \\ \hline
U+1D5F2 & $ 𝗲 $ & {\textbackslash}bsanse & Mathematical Sans-Serif Bold Small E \\ \hline
U+1D5F3 & $ 𝗳 $ & {\textbackslash}bsansf & Mathematical Sans-Serif Bold Small F \\ \hline
U+1D5F4 & $ 𝗴 $ & {\textbackslash}bsansg & Mathematical Sans-Serif Bold Small G \\ \hline
U+1D5F5 & $ 𝗵 $ & {\textbackslash}bsansh & Mathematical Sans-Serif Bold Small H \\ \hline
U+1D5F6 & $ 𝗶 $ & {\textbackslash}bsansi & Mathematical Sans-Serif Bold Small I \\ \hline
U+1D5F7 & $ 𝗷 $ & {\textbackslash}bsansj & Mathematical Sans-Serif Bold Small J \\ \hline
U+1D5F8 & $ 𝗸 $ & {\textbackslash}bsansk & Mathematical Sans-Serif Bold Small K \\ \hline
U+1D5F9 & $ 𝗹 $ & {\textbackslash}bsansl & Mathematical Sans-Serif Bold Small L \\ \hline
U+1D5FA & $ 𝗺 $ & {\textbackslash}bsansm & Mathematical Sans-Serif Bold Small M \\ \hline
U+1D5FB & $ 𝗻 $ & {\textbackslash}bsansn & Mathematical Sans-Serif Bold Small N \\ \hline
U+1D5FC & $ 𝗼 $ & {\textbackslash}bsanso & Mathematical Sans-Serif Bold Small O \\ \hline
U+1D5FD & $ 𝗽 $ & {\textbackslash}bsansp & Mathematical Sans-Serif Bold Small P \\ \hline
U+1D5FE & $ 𝗾 $ & {\textbackslash}bsansq & Mathematical Sans-Serif Bold Small Q \\ \hline
U+1D5FF & $ 𝗿 $ & {\textbackslash}bsansr & Mathematical Sans-Serif Bold Small R \\ \hline
U+1D600 & $ 𝘀 $ & {\textbackslash}bsanss & Mathematical Sans-Serif Bold Small S \\ \hline
U+1D601 & $ 𝘁 $ & {\textbackslash}bsanst & Mathematical Sans-Serif Bold Small T \\ \hline
U+1D602 & $ 𝘂 $ & {\textbackslash}bsansu & Mathematical Sans-Serif Bold Small U \\ \hline
U+1D603 & $ 𝘃 $ & {\textbackslash}bsansv & Mathematical Sans-Serif Bold Small V \\ \hline
U+1D604 & $ 𝘄 $ & {\textbackslash}bsansw & Mathematical Sans-Serif Bold Small W \\ \hline
U+1D605 & $ 𝘅 $ & {\textbackslash}bsansx & Mathematical Sans-Serif Bold Small X \\ \hline
U+1D606 & $ 𝘆 $ & {\textbackslash}bsansy & Mathematical Sans-Serif Bold Small Y \\ \hline
U+1D607 & $ 𝘇 $ & {\textbackslash}bsansz & Mathematical Sans-Serif Bold Small Z \\ \hline
U+1D608 & $ 𝘈 $ & {\textbackslash}isansA & Mathematical Sans-Serif Italic Capital A \\ \hline
U+1D609 & $ 𝘉 $ & {\textbackslash}isansB & Mathematical Sans-Serif Italic Capital B \\ \hline
U+1D60A & $ 𝘊 $ & {\textbackslash}isansC & Mathematical Sans-Serif Italic Capital C \\ \hline
U+1D60B & $ 𝘋 $ & {\textbackslash}isansD & Mathematical Sans-Serif Italic Capital D \\ \hline
U+1D60C & $ 𝘌 $ & {\textbackslash}isansE & Mathematical Sans-Serif Italic Capital E \\ \hline
U+1D60D & $ 𝘍 $ & {\textbackslash}isansF & Mathematical Sans-Serif Italic Capital F \\ \hline
U+1D60E & $ 𝘎 $ & {\textbackslash}isansG & Mathematical Sans-Serif Italic Capital G \\ \hline
U+1D60F & $ 𝘏 $ & {\textbackslash}isansH & Mathematical Sans-Serif Italic Capital H \\ \hline
U+1D610 & $ 𝘐 $ & {\textbackslash}isansI & Mathematical Sans-Serif Italic Capital I \\ \hline
U+1D611 & $ 𝘑 $ & {\textbackslash}isansJ & Mathematical Sans-Serif Italic Capital J \\ \hline
U+1D612 & $ 𝘒 $ & {\textbackslash}isansK & Mathematical Sans-Serif Italic Capital K \\ \hline
U+1D613 & $ 𝘓 $ & {\textbackslash}isansL & Mathematical Sans-Serif Italic Capital L \\ \hline
U+1D614 & $ 𝘔 $ & {\textbackslash}isansM & Mathematical Sans-Serif Italic Capital M \\ \hline
U+1D615 & $ 𝘕 $ & {\textbackslash}isansN & Mathematical Sans-Serif Italic Capital N \\ \hline
U+1D616 & $ 𝘖 $ & {\textbackslash}isansO & Mathematical Sans-Serif Italic Capital O \\ \hline
U+1D617 & $ 𝘗 $ & {\textbackslash}isansP & Mathematical Sans-Serif Italic Capital P \\ \hline
U+1D618 & $ 𝘘 $ & {\textbackslash}isansQ & Mathematical Sans-Serif Italic Capital Q \\ \hline
U+1D619 & $ 𝘙 $ & {\textbackslash}isansR & Mathematical Sans-Serif Italic Capital R \\ \hline
U+1D61A & $ 𝘚 $ & {\textbackslash}isansS & Mathematical Sans-Serif Italic Capital S \\ \hline
U+1D61B & $ 𝘛 $ & {\textbackslash}isansT & Mathematical Sans-Serif Italic Capital T \\ \hline
U+1D61C & $ 𝘜 $ & {\textbackslash}isansU & Mathematical Sans-Serif Italic Capital U \\ \hline
U+1D61D & $ 𝘝 $ & {\textbackslash}isansV & Mathematical Sans-Serif Italic Capital V \\ \hline
U+1D61E & $ 𝘞 $ & {\textbackslash}isansW & Mathematical Sans-Serif Italic Capital W \\ \hline
U+1D61F & $ 𝘟 $ & {\textbackslash}isansX & Mathematical Sans-Serif Italic Capital X \\ \hline
U+1D620 & $ 𝘠 $ & {\textbackslash}isansY & Mathematical Sans-Serif Italic Capital Y \\ \hline
U+1D621 & $ 𝘡 $ & {\textbackslash}isansZ & Mathematical Sans-Serif Italic Capital Z \\ \hline
U+1D622 & $ 𝘢 $ & {\textbackslash}isansa & Mathematical Sans-Serif Italic Small A \\ \hline
U+1D623 & $ 𝘣 $ & {\textbackslash}isansb & Mathematical Sans-Serif Italic Small B \\ \hline
U+1D624 & $ 𝘤 $ & {\textbackslash}isansc & Mathematical Sans-Serif Italic Small C \\ \hline
U+1D625 & $ 𝘥 $ & {\textbackslash}isansd & Mathematical Sans-Serif Italic Small D \\ \hline
U+1D626 & $ 𝘦 $ & {\textbackslash}isanse & Mathematical Sans-Serif Italic Small E \\ \hline
U+1D627 & $ 𝘧 $ & {\textbackslash}isansf & Mathematical Sans-Serif Italic Small F \\ \hline
U+1D628 & $ 𝘨 $ & {\textbackslash}isansg & Mathematical Sans-Serif Italic Small G \\ \hline
U+1D629 & $ 𝘩 $ & {\textbackslash}isansh & Mathematical Sans-Serif Italic Small H \\ \hline
U+1D62A & $ 𝘪 $ & {\textbackslash}isansi & Mathematical Sans-Serif Italic Small I \\ \hline
U+1D62B & $ 𝘫 $ & {\textbackslash}isansj & Mathematical Sans-Serif Italic Small J \\ \hline
U+1D62C & $ 𝘬 $ & {\textbackslash}isansk & Mathematical Sans-Serif Italic Small K \\ \hline
U+1D62D & $ 𝘭 $ & {\textbackslash}isansl & Mathematical Sans-Serif Italic Small L \\ \hline
U+1D62E & $ 𝘮 $ & {\textbackslash}isansm & Mathematical Sans-Serif Italic Small M \\ \hline
U+1D62F & $ 𝘯 $ & {\textbackslash}isansn & Mathematical Sans-Serif Italic Small N \\ \hline
U+1D630 & $ 𝘰 $ & {\textbackslash}isanso & Mathematical Sans-Serif Italic Small O \\ \hline
U+1D631 & $ 𝘱 $ & {\textbackslash}isansp & Mathematical Sans-Serif Italic Small P \\ \hline
U+1D632 & $ 𝘲 $ & {\textbackslash}isansq & Mathematical Sans-Serif Italic Small Q \\ \hline
U+1D633 & $ 𝘳 $ & {\textbackslash}isansr & Mathematical Sans-Serif Italic Small R \\ \hline
U+1D634 & $ 𝘴 $ & {\textbackslash}isanss & Mathematical Sans-Serif Italic Small S \\ \hline
U+1D635 & $ 𝘵 $ & {\textbackslash}isanst & Mathematical Sans-Serif Italic Small T \\ \hline
U+1D636 & $ 𝘶 $ & {\textbackslash}isansu & Mathematical Sans-Serif Italic Small U \\ \hline
U+1D637 & $ 𝘷 $ & {\textbackslash}isansv & Mathematical Sans-Serif Italic Small V \\ \hline
U+1D638 & $ 𝘸 $ & {\textbackslash}isansw & Mathematical Sans-Serif Italic Small W \\ \hline
U+1D639 & $ 𝘹 $ & {\textbackslash}isansx & Mathematical Sans-Serif Italic Small X \\ \hline
U+1D63A & $ 𝘺 $ & {\textbackslash}isansy & Mathematical Sans-Serif Italic Small Y \\ \hline
U+1D63B & $ 𝘻 $ & {\textbackslash}isansz & Mathematical Sans-Serif Italic Small Z \\ \hline
U+1D63C & $ 𝘼 $ & {\textbackslash}bisansA & Mathematical Sans-Serif Bold Italic Capital A \\ \hline
U+1D63D & $ 𝘽 $ & {\textbackslash}bisansB & Mathematical Sans-Serif Bold Italic Capital B \\ \hline
U+1D63E & $ 𝘾 $ & {\textbackslash}bisansC & Mathematical Sans-Serif Bold Italic Capital C \\ \hline
U+1D63F & $ 𝘿 $ & {\textbackslash}bisansD & Mathematical Sans-Serif Bold Italic Capital D \\ \hline
U+1D640 & $ 𝙀 $ & {\textbackslash}bisansE & Mathematical Sans-Serif Bold Italic Capital E \\ \hline
U+1D641 & $ 𝙁 $ & {\textbackslash}bisansF & Mathematical Sans-Serif Bold Italic Capital F \\ \hline
U+1D642 & $ 𝙂 $ & {\textbackslash}bisansG & Mathematical Sans-Serif Bold Italic Capital G \\ \hline
U+1D643 & $ 𝙃 $ & {\textbackslash}bisansH & Mathematical Sans-Serif Bold Italic Capital H \\ \hline
U+1D644 & $ 𝙄 $ & {\textbackslash}bisansI & Mathematical Sans-Serif Bold Italic Capital I \\ \hline
U+1D645 & $ 𝙅 $ & {\textbackslash}bisansJ & Mathematical Sans-Serif Bold Italic Capital J \\ \hline
U+1D646 & $ 𝙆 $ & {\textbackslash}bisansK & Mathematical Sans-Serif Bold Italic Capital K \\ \hline
U+1D647 & $ 𝙇 $ & {\textbackslash}bisansL & Mathematical Sans-Serif Bold Italic Capital L \\ \hline
U+1D648 & $ 𝙈 $ & {\textbackslash}bisansM & Mathematical Sans-Serif Bold Italic Capital M \\ \hline
U+1D649 & $ 𝙉 $ & {\textbackslash}bisansN & Mathematical Sans-Serif Bold Italic Capital N \\ \hline
U+1D64A & $ 𝙊 $ & {\textbackslash}bisansO & Mathematical Sans-Serif Bold Italic Capital O \\ \hline
U+1D64B & $ 𝙋 $ & {\textbackslash}bisansP & Mathematical Sans-Serif Bold Italic Capital P \\ \hline
U+1D64C & $ 𝙌 $ & {\textbackslash}bisansQ & Mathematical Sans-Serif Bold Italic Capital Q \\ \hline
U+1D64D & $ 𝙍 $ & {\textbackslash}bisansR & Mathematical Sans-Serif Bold Italic Capital R \\ \hline
U+1D64E & $ 𝙎 $ & {\textbackslash}bisansS & Mathematical Sans-Serif Bold Italic Capital S \\ \hline
U+1D64F & $ 𝙏 $ & {\textbackslash}bisansT & Mathematical Sans-Serif Bold Italic Capital T \\ \hline
U+1D650 & $ 𝙐 $ & {\textbackslash}bisansU & Mathematical Sans-Serif Bold Italic Capital U \\ \hline
U+1D651 & $ 𝙑 $ & {\textbackslash}bisansV & Mathematical Sans-Serif Bold Italic Capital V \\ \hline
U+1D652 & $ 𝙒 $ & {\textbackslash}bisansW & Mathematical Sans-Serif Bold Italic Capital W \\ \hline
U+1D653 & $ 𝙓 $ & {\textbackslash}bisansX & Mathematical Sans-Serif Bold Italic Capital X \\ \hline
U+1D654 & $ 𝙔 $ & {\textbackslash}bisansY & Mathematical Sans-Serif Bold Italic Capital Y \\ \hline
U+1D655 & $ 𝙕 $ & {\textbackslash}bisansZ & Mathematical Sans-Serif Bold Italic Capital Z \\ \hline
U+1D656 & $ 𝙖 $ & {\textbackslash}bisansa & Mathematical Sans-Serif Bold Italic Small A \\ \hline
U+1D657 & $ 𝙗 $ & {\textbackslash}bisansb & Mathematical Sans-Serif Bold Italic Small B \\ \hline
U+1D658 & $ 𝙘 $ & {\textbackslash}bisansc & Mathematical Sans-Serif Bold Italic Small C \\ \hline
U+1D659 & $ 𝙙 $ & {\textbackslash}bisansd & Mathematical Sans-Serif Bold Italic Small D \\ \hline
U+1D65A & $ 𝙚 $ & {\textbackslash}bisanse & Mathematical Sans-Serif Bold Italic Small E \\ \hline
U+1D65B & $ 𝙛 $ & {\textbackslash}bisansf & Mathematical Sans-Serif Bold Italic Small F \\ \hline
U+1D65C & $ 𝙜 $ & {\textbackslash}bisansg & Mathematical Sans-Serif Bold Italic Small G \\ \hline
U+1D65D & $ 𝙝 $ & {\textbackslash}bisansh & Mathematical Sans-Serif Bold Italic Small H \\ \hline
U+1D65E & $ 𝙞 $ & {\textbackslash}bisansi & Mathematical Sans-Serif Bold Italic Small I \\ \hline
U+1D65F & $ 𝙟 $ & {\textbackslash}bisansj & Mathematical Sans-Serif Bold Italic Small J \\ \hline
U+1D660 & $ 𝙠 $ & {\textbackslash}bisansk & Mathematical Sans-Serif Bold Italic Small K \\ \hline
U+1D661 & $ 𝙡 $ & {\textbackslash}bisansl & Mathematical Sans-Serif Bold Italic Small L \\ \hline
U+1D662 & $ 𝙢 $ & {\textbackslash}bisansm & Mathematical Sans-Serif Bold Italic Small M \\ \hline
U+1D663 & $ 𝙣 $ & {\textbackslash}bisansn & Mathematical Sans-Serif Bold Italic Small N \\ \hline
U+1D664 & $ 𝙤 $ & {\textbackslash}bisanso & Mathematical Sans-Serif Bold Italic Small O \\ \hline
U+1D665 & $ 𝙥 $ & {\textbackslash}bisansp & Mathematical Sans-Serif Bold Italic Small P \\ \hline
U+1D666 & $ 𝙦 $ & {\textbackslash}bisansq & Mathematical Sans-Serif Bold Italic Small Q \\ \hline
U+1D667 & $ 𝙧 $ & {\textbackslash}bisansr & Mathematical Sans-Serif Bold Italic Small R \\ \hline
U+1D668 & $ 𝙨 $ & {\textbackslash}bisanss & Mathematical Sans-Serif Bold Italic Small S \\ \hline
U+1D669 & $ 𝙩 $ & {\textbackslash}bisanst & Mathematical Sans-Serif Bold Italic Small T \\ \hline
U+1D66A & $ 𝙪 $ & {\textbackslash}bisansu & Mathematical Sans-Serif Bold Italic Small U \\ \hline
U+1D66B & $ 𝙫 $ & {\textbackslash}bisansv & Mathematical Sans-Serif Bold Italic Small V \\ \hline
U+1D66C & $ 𝙬 $ & {\textbackslash}bisansw & Mathematical Sans-Serif Bold Italic Small W \\ \hline
U+1D66D & $ 𝙭 $ & {\textbackslash}bisansx & Mathematical Sans-Serif Bold Italic Small X \\ \hline
U+1D66E & $ 𝙮 $ & {\textbackslash}bisansy & Mathematical Sans-Serif Bold Italic Small Y \\ \hline
U+1D66F & $ 𝙯 $ & {\textbackslash}bisansz & Mathematical Sans-Serif Bold Italic Small Z \\ \hline
U+1D670 & $ 𝙰 $ & {\textbackslash}ttA & Mathematical Monospace Capital A \\ \hline
U+1D671 & $ 𝙱 $ & {\textbackslash}ttB & Mathematical Monospace Capital B \\ \hline
U+1D672 & $ 𝙲 $ & {\textbackslash}ttC & Mathematical Monospace Capital C \\ \hline
U+1D673 & $ 𝙳 $ & {\textbackslash}ttD & Mathematical Monospace Capital D \\ \hline
U+1D674 & $ 𝙴 $ & {\textbackslash}ttE & Mathematical Monospace Capital E \\ \hline
U+1D675 & $ 𝙵 $ & {\textbackslash}ttF & Mathematical Monospace Capital F \\ \hline
U+1D676 & $ 𝙶 $ & {\textbackslash}ttG & Mathematical Monospace Capital G \\ \hline
U+1D677 & $ 𝙷 $ & {\textbackslash}ttH & Mathematical Monospace Capital H \\ \hline
U+1D678 & $ 𝙸 $ & {\textbackslash}ttI & Mathematical Monospace Capital I \\ \hline
U+1D679 & $ 𝙹 $ & {\textbackslash}ttJ & Mathematical Monospace Capital J \\ \hline
U+1D67A & $ 𝙺 $ & {\textbackslash}ttK & Mathematical Monospace Capital K \\ \hline
U+1D67B & $ 𝙻 $ & {\textbackslash}ttL & Mathematical Monospace Capital L \\ \hline
U+1D67C & $ 𝙼 $ & {\textbackslash}ttM & Mathematical Monospace Capital M \\ \hline
U+1D67D & $ 𝙽 $ & {\textbackslash}ttN & Mathematical Monospace Capital N \\ \hline
U+1D67E & $ 𝙾 $ & {\textbackslash}ttO & Mathematical Monospace Capital O \\ \hline
U+1D67F & $ 𝙿 $ & {\textbackslash}ttP & Mathematical Monospace Capital P \\ \hline
U+1D680 & $ 𝚀 $ & {\textbackslash}ttQ & Mathematical Monospace Capital Q \\ \hline
U+1D681 & $ 𝚁 $ & {\textbackslash}ttR & Mathematical Monospace Capital R \\ \hline
U+1D682 & $ 𝚂 $ & {\textbackslash}ttS & Mathematical Monospace Capital S \\ \hline
U+1D683 & $ 𝚃 $ & {\textbackslash}ttT & Mathematical Monospace Capital T \\ \hline
U+1D684 & $ 𝚄 $ & {\textbackslash}ttU & Mathematical Monospace Capital U \\ \hline
U+1D685 & $ 𝚅 $ & {\textbackslash}ttV & Mathematical Monospace Capital V \\ \hline
U+1D686 & $ 𝚆 $ & {\textbackslash}ttW & Mathematical Monospace Capital W \\ \hline
U+1D687 & $ 𝚇 $ & {\textbackslash}ttX & Mathematical Monospace Capital X \\ \hline
U+1D688 & $ 𝚈 $ & {\textbackslash}ttY & Mathematical Monospace Capital Y \\ \hline
U+1D689 & $ 𝚉 $ & {\textbackslash}ttZ & Mathematical Monospace Capital Z \\ \hline
U+1D68A & $ 𝚊 $ & {\textbackslash}tta & Mathematical Monospace Small A \\ \hline
U+1D68B & $ 𝚋 $ & {\textbackslash}ttb & Mathematical Monospace Small B \\ \hline
U+1D68C & $ 𝚌 $ & {\textbackslash}ttc & Mathematical Monospace Small C \\ \hline
U+1D68D & $ 𝚍 $ & {\textbackslash}ttd & Mathematical Monospace Small D \\ \hline
U+1D68E & $ 𝚎 $ & {\textbackslash}tte & Mathematical Monospace Small E \\ \hline
U+1D68F & $ 𝚏 $ & {\textbackslash}ttf & Mathematical Monospace Small F \\ \hline
U+1D690 & $ 𝚐 $ & {\textbackslash}ttg & Mathematical Monospace Small G \\ \hline
U+1D691 & $ 𝚑 $ & {\textbackslash}tth & Mathematical Monospace Small H \\ \hline
U+1D692 & $ 𝚒 $ & {\textbackslash}tti & Mathematical Monospace Small I \\ \hline
U+1D693 & $ 𝚓 $ & {\textbackslash}ttj & Mathematical Monospace Small J \\ \hline
U+1D694 & $ 𝚔 $ & {\textbackslash}ttk & Mathematical Monospace Small K \\ \hline
U+1D695 & $ 𝚕 $ & {\textbackslash}ttl & Mathematical Monospace Small L \\ \hline
U+1D696 & $ 𝚖 $ & {\textbackslash}ttm & Mathematical Monospace Small M \\ \hline
U+1D697 & $ 𝚗 $ & {\textbackslash}ttn & Mathematical Monospace Small N \\ \hline
U+1D698 & $ 𝚘 $ & {\textbackslash}tto & Mathematical Monospace Small O \\ \hline
U+1D699 & $ 𝚙 $ & {\textbackslash}ttp & Mathematical Monospace Small P \\ \hline
U+1D69A & $ 𝚚 $ & {\textbackslash}ttq & Mathematical Monospace Small Q \\ \hline
U+1D69B & $ 𝚛 $ & {\textbackslash}ttr & Mathematical Monospace Small R \\ \hline
U+1D69C & $ 𝚜 $ & {\textbackslash}tts & Mathematical Monospace Small S \\ \hline
U+1D69D & $ 𝚝 $ & {\textbackslash}ttt & Mathematical Monospace Small T \\ \hline
U+1D69E & $ 𝚞 $ & {\textbackslash}ttu & Mathematical Monospace Small U \\ \hline
U+1D69F & $ 𝚟 $ & {\textbackslash}ttv & Mathematical Monospace Small V \\ \hline
U+1D6A0 & $ 𝚠 $ & {\textbackslash}ttw & Mathematical Monospace Small W \\ \hline
U+1D6A1 & $ 𝚡 $ & {\textbackslash}ttx & Mathematical Monospace Small X \\ \hline
U+1D6A2 & $ 𝚢 $ & {\textbackslash}tty & Mathematical Monospace Small Y \\ \hline
U+1D6A3 & $ 𝚣 $ & {\textbackslash}ttz & Mathematical Monospace Small Z \\ \hline
U+1D6A4 & $ 𝚤 $ & {\textbackslash}itimath & Mathematical Italic Small Dotless I \\ \hline
U+1D6A5 & $ 𝚥 $ & {\textbackslash}itjmath & Mathematical Italic Small Dotless J \\ \hline
U+1D6A8 & $ 𝚨 $ & {\textbackslash}bfAlpha & Mathematical Bold Capital Alpha \\ \hline
U+1D6A9 & $ 𝚩 $ & {\textbackslash}bfBeta & Mathematical Bold Capital Beta \\ \hline
U+1D6AA & $ 𝚪 $ & {\textbackslash}bfGamma & Mathematical Bold Capital Gamma \\ \hline
U+1D6AB & $ 𝚫 $ & {\textbackslash}bfDelta & Mathematical Bold Capital Delta \\ \hline
U+1D6AC & $ 𝚬 $ & {\textbackslash}bfEpsilon & Mathematical Bold Capital Epsilon \\ \hline
U+1D6AD & $ 𝚭 $ & {\textbackslash}bfZeta & Mathematical Bold Capital Zeta \\ \hline
U+1D6AE & $ 𝚮 $ & {\textbackslash}bfEta & Mathematical Bold Capital Eta \\ \hline
U+1D6AF & $ 𝚯 $ & {\textbackslash}bfTheta & Mathematical Bold Capital Theta \\ \hline
U+1D6B0 & $ 𝚰 $ & {\textbackslash}bfIota & Mathematical Bold Capital Iota \\ \hline
U+1D6B1 & $ 𝚱 $ & {\textbackslash}bfKappa & Mathematical Bold Capital Kappa \\ \hline
U+1D6B2 & $ 𝚲 $ & {\textbackslash}bfLambda & Mathematical Bold Capital Lamda \\ \hline
U+1D6B3 & $ 𝚳 $ & {\textbackslash}bfMu & Mathematical Bold Capital Mu \\ \hline
U+1D6B4 & $ 𝚴 $ & {\textbackslash}bfNu & Mathematical Bold Capital Nu \\ \hline
U+1D6B5 & $ 𝚵 $ & {\textbackslash}bfXi & Mathematical Bold Capital Xi \\ \hline
U+1D6B6 & $ 𝚶 $ & {\textbackslash}bfOmicron & Mathematical Bold Capital Omicron \\ \hline
U+1D6B7 & $ 𝚷 $ & {\textbackslash}bfPi & Mathematical Bold Capital Pi \\ \hline
U+1D6B8 & $ 𝚸 $ & {\textbackslash}bfRho & Mathematical Bold Capital Rho \\ \hline
U+1D6B9 & $ 𝚹 $ & {\textbackslash}bfvarTheta & Mathematical Bold Capital Theta Symbol \\ \hline
U+1D6BA & $ 𝚺 $ & {\textbackslash}bfSigma & Mathematical Bold Capital Sigma \\ \hline
U+1D6BB & $ 𝚻 $ & {\textbackslash}bfTau & Mathematical Bold Capital Tau \\ \hline
U+1D6BC & $ 𝚼 $ & {\textbackslash}bfUpsilon & Mathematical Bold Capital Upsilon \\ \hline
U+1D6BD & $ 𝚽 $ & {\textbackslash}bfPhi & Mathematical Bold Capital Phi \\ \hline
U+1D6BE & $ 𝚾 $ & {\textbackslash}bfChi & Mathematical Bold Capital Chi \\ \hline
U+1D6BF & $ 𝚿 $ & {\textbackslash}bfPsi & Mathematical Bold Capital Psi \\ \hline
U+1D6C0 & $ 𝛀 $ & {\textbackslash}bfOmega & Mathematical Bold Capital Omega \\ \hline
U+1D6C1 & $ 𝛁 $ & {\textbackslash}bfnabla & Mathematical Bold Nabla \\ \hline
U+1D6C2 & $ 𝛂 $ & {\textbackslash}bfalpha & Mathematical Bold Small Alpha \\ \hline
U+1D6C3 & $ 𝛃 $ & {\textbackslash}bfbeta & Mathematical Bold Small Beta \\ \hline
U+1D6C4 & $ 𝛄 $ & {\textbackslash}bfgamma & Mathematical Bold Small Gamma \\ \hline
U+1D6C5 & $ 𝛅 $ & {\textbackslash}bfdelta & Mathematical Bold Small Delta \\ \hline
U+1D6C6 & $ 𝛆 $ & {\textbackslash}bfepsilon & Mathematical Bold Small Epsilon \\ \hline
U+1D6C7 & $ 𝛇 $ & {\textbackslash}bfzeta & Mathematical Bold Small Zeta \\ \hline
U+1D6C8 & $ 𝛈 $ & {\textbackslash}bfeta & Mathematical Bold Small Eta \\ \hline
U+1D6C9 & $ 𝛉 $ & {\textbackslash}bftheta & Mathematical Bold Small Theta \\ \hline
U+1D6CA & $ 𝛊 $ & {\textbackslash}bfiota & Mathematical Bold Small Iota \\ \hline
U+1D6CB & $ 𝛋 $ & {\textbackslash}bfkappa & Mathematical Bold Small Kappa \\ \hline
U+1D6CC & $ 𝛌 $ & {\textbackslash}bflambda & Mathematical Bold Small Lamda \\ \hline
U+1D6CD & $ 𝛍 $ & {\textbackslash}bfmu & Mathematical Bold Small Mu \\ \hline
U+1D6CE & $ 𝛎 $ & {\textbackslash}bfnu & Mathematical Bold Small Nu \\ \hline
U+1D6CF & $ 𝛏 $ & {\textbackslash}bfxi & Mathematical Bold Small Xi \\ \hline
U+1D6D0 & $ 𝛐 $ & {\textbackslash}bfomicron & Mathematical Bold Small Omicron \\ \hline
U+1D6D1 & $ 𝛑 $ & {\textbackslash}bfpi & Mathematical Bold Small Pi \\ \hline
U+1D6D2 & $ 𝛒 $ & {\textbackslash}bfrho & Mathematical Bold Small Rho \\ \hline
U+1D6D3 & $ 𝛓 $ & {\textbackslash}bfvarsigma & Mathematical Bold Small Final Sigma \\ \hline
U+1D6D4 & $ 𝛔 $ & {\textbackslash}bfsigma & Mathematical Bold Small Sigma \\ \hline
U+1D6D5 & $ 𝛕 $ & {\textbackslash}bftau & Mathematical Bold Small Tau \\ \hline
U+1D6D6 & $ 𝛖 $ & {\textbackslash}bfupsilon & Mathematical Bold Small Upsilon \\ \hline
U+1D6D7 & $ 𝛗 $ & {\textbackslash}bfvarphi & Mathematical Bold Small Phi \\ \hline
U+1D6D8 & $ 𝛘 $ & {\textbackslash}bfchi & Mathematical Bold Small Chi \\ \hline
U+1D6D9 & $ 𝛙 $ & {\textbackslash}bfpsi & Mathematical Bold Small Psi \\ \hline
U+1D6DA & $ 𝛚 $ & {\textbackslash}bfomega & Mathematical Bold Small Omega \\ \hline
U+1D6DB & $ 𝛛 $ & {\textbackslash}bfpartial & Mathematical Bold Partial Differential \\ \hline
U+1D6DC & $ 𝛜 $ & {\textbackslash}bfvarepsilon & Mathematical Bold Epsilon Symbol \\ \hline
U+1D6DD & $ 𝛝 $ & {\textbackslash}bfvartheta & Mathematical Bold Theta Symbol \\ \hline
U+1D6DE & $ 𝛞 $ & {\textbackslash}bfvarkappa & Mathematical Bold Kappa Symbol \\ \hline
U+1D6DF & $ 𝛟 $ & {\textbackslash}bfphi & Mathematical Bold Phi Symbol \\ \hline
U+1D6E0 & $ 𝛠 $ & {\textbackslash}bfvarrho & Mathematical Bold Rho Symbol \\ \hline
U+1D6E1 & $ 𝛡 $ & {\textbackslash}bfvarpi & Mathematical Bold Pi Symbol \\ \hline
U+1D6E2 & $ 𝛢 $ & {\textbackslash}itAlpha & Mathematical Italic Capital Alpha \\ \hline
U+1D6E3 & $ 𝛣 $ & {\textbackslash}itBeta & Mathematical Italic Capital Beta \\ \hline
U+1D6E4 & $ 𝛤 $ & {\textbackslash}itGamma & Mathematical Italic Capital Gamma \\ \hline
U+1D6E5 & $ 𝛥 $ & {\textbackslash}itDelta & Mathematical Italic Capital Delta \\ \hline
U+1D6E6 & $ 𝛦 $ & {\textbackslash}itEpsilon & Mathematical Italic Capital Epsilon \\ \hline
U+1D6E7 & $ 𝛧 $ & {\textbackslash}itZeta & Mathematical Italic Capital Zeta \\ \hline
U+1D6E8 & $ 𝛨 $ & {\textbackslash}itEta & Mathematical Italic Capital Eta \\ \hline
U+1D6E9 & $ 𝛩 $ & {\textbackslash}itTheta & Mathematical Italic Capital Theta \\ \hline
U+1D6EA & $ 𝛪 $ & {\textbackslash}itIota & Mathematical Italic Capital Iota \\ \hline
U+1D6EB & $ 𝛫 $ & {\textbackslash}itKappa & Mathematical Italic Capital Kappa \\ \hline
U+1D6EC & $ 𝛬 $ & {\textbackslash}itLambda & Mathematical Italic Capital Lamda \\ \hline
U+1D6ED & $ 𝛭 $ & {\textbackslash}itMu & Mathematical Italic Capital Mu \\ \hline
U+1D6EE & $ 𝛮 $ & {\textbackslash}itNu & Mathematical Italic Capital Nu \\ \hline
U+1D6EF & $ 𝛯 $ & {\textbackslash}itXi & Mathematical Italic Capital Xi \\ \hline
U+1D6F0 & $ 𝛰 $ & {\textbackslash}itOmicron & Mathematical Italic Capital Omicron \\ \hline
U+1D6F1 & $ 𝛱 $ & {\textbackslash}itPi & Mathematical Italic Capital Pi \\ \hline
U+1D6F2 & $ 𝛲 $ & {\textbackslash}itRho & Mathematical Italic Capital Rho \\ \hline
U+1D6F3 & $ 𝛳 $ & {\textbackslash}itvarTheta & Mathematical Italic Capital Theta Symbol \\ \hline
U+1D6F4 & $ 𝛴 $ & {\textbackslash}itSigma & Mathematical Italic Capital Sigma \\ \hline
U+1D6F5 & $ 𝛵 $ & {\textbackslash}itTau & Mathematical Italic Capital Tau \\ \hline
U+1D6F6 & $ 𝛶 $ & {\textbackslash}itUpsilon & Mathematical Italic Capital Upsilon \\ \hline
U+1D6F7 & $ 𝛷 $ & {\textbackslash}itPhi & Mathematical Italic Capital Phi \\ \hline
U+1D6F8 & $ 𝛸 $ & {\textbackslash}itChi & Mathematical Italic Capital Chi \\ \hline
U+1D6F9 & $ 𝛹 $ & {\textbackslash}itPsi & Mathematical Italic Capital Psi \\ \hline
U+1D6FA & $ 𝛺 $ & {\textbackslash}itOmega & Mathematical Italic Capital Omega \\ \hline
U+1D6FB & $ 𝛻 $ & {\textbackslash}itnabla & Mathematical Italic Nabla \\ \hline
U+1D6FC & $ 𝛼 $ & {\textbackslash}italpha & Mathematical Italic Small Alpha \\ \hline
U+1D6FD & $ 𝛽 $ & {\textbackslash}itbeta & Mathematical Italic Small Beta \\ \hline
U+1D6FE & $ 𝛾 $ & {\textbackslash}itgamma & Mathematical Italic Small Gamma \\ \hline
U+1D6FF & $ 𝛿 $ & {\textbackslash}itdelta & Mathematical Italic Small Delta \\ \hline
U+1D700 & $ 𝜀 $ & {\textbackslash}itepsilon & Mathematical Italic Small Epsilon \\ \hline
U+1D701 & $ 𝜁 $ & {\textbackslash}itzeta & Mathematical Italic Small Zeta \\ \hline
U+1D702 & $ 𝜂 $ & {\textbackslash}iteta & Mathematical Italic Small Eta \\ \hline
U+1D703 & $ 𝜃 $ & {\textbackslash}ittheta & Mathematical Italic Small Theta \\ \hline
U+1D704 & $ 𝜄 $ & {\textbackslash}itiota & Mathematical Italic Small Iota \\ \hline
U+1D705 & $ 𝜅 $ & {\textbackslash}itkappa & Mathematical Italic Small Kappa \\ \hline
U+1D706 & $ 𝜆 $ & {\textbackslash}itlambda & Mathematical Italic Small Lamda \\ \hline
U+1D707 & $ 𝜇 $ & {\textbackslash}itmu & Mathematical Italic Small Mu \\ \hline
U+1D708 & $ 𝜈 $ & {\textbackslash}itnu & Mathematical Italic Small Nu \\ \hline
U+1D709 & $ 𝜉 $ & {\textbackslash}itxi & Mathematical Italic Small Xi \\ \hline
U+1D70A & $ 𝜊 $ & {\textbackslash}itomicron & Mathematical Italic Small Omicron \\ \hline
U+1D70B & $ 𝜋 $ & {\textbackslash}itpi & Mathematical Italic Small Pi \\ \hline
U+1D70C & $ 𝜌 $ & {\textbackslash}itrho & Mathematical Italic Small Rho \\ \hline
U+1D70D & $ 𝜍 $ & {\textbackslash}itvarsigma & Mathematical Italic Small Final Sigma \\ \hline
U+1D70E & $ 𝜎 $ & {\textbackslash}itsigma & Mathematical Italic Small Sigma \\ \hline
U+1D70F & $ 𝜏 $ & {\textbackslash}ittau & Mathematical Italic Small Tau \\ \hline
U+1D710 & $ 𝜐 $ & {\textbackslash}itupsilon & Mathematical Italic Small Upsilon \\ \hline
U+1D711 & $ 𝜑 $ & {\textbackslash}itphi & Mathematical Italic Small Phi \\ \hline
U+1D712 & $ 𝜒 $ & {\textbackslash}itchi & Mathematical Italic Small Chi \\ \hline
U+1D713 & $ 𝜓 $ & {\textbackslash}itpsi & Mathematical Italic Small Psi \\ \hline
U+1D714 & $ 𝜔 $ & {\textbackslash}itomega & Mathematical Italic Small Omega \\ \hline
U+1D715 & $ 𝜕 $ & {\textbackslash}itpartial & Mathematical Italic Partial Differential \\ \hline
U+1D716 & $ 𝜖 $ & {\textbackslash}itvarepsilon & Mathematical Italic Epsilon Symbol \\ \hline
U+1D717 & $ 𝜗 $ & {\textbackslash}itvartheta & Mathematical Italic Theta Symbol \\ \hline
U+1D718 & $ 𝜘 $ & {\textbackslash}itvarkappa & Mathematical Italic Kappa Symbol \\ \hline
U+1D719 & $ 𝜙 $ & {\textbackslash}itvarphi & Mathematical Italic Phi Symbol \\ \hline
U+1D71A & $ 𝜚 $ & {\textbackslash}itvarrho & Mathematical Italic Rho Symbol \\ \hline
U+1D71B & $ 𝜛 $ & {\textbackslash}itvarpi & Mathematical Italic Pi Symbol \\ \hline
U+1D71C & $ 𝜜 $ & {\textbackslash}biAlpha & Mathematical Bold Italic Capital Alpha \\ \hline
U+1D71D & $ 𝜝 $ & {\textbackslash}biBeta & Mathematical Bold Italic Capital Beta \\ \hline
U+1D71E & $ 𝜞 $ & {\textbackslash}biGamma & Mathematical Bold Italic Capital Gamma \\ \hline
U+1D71F & $ 𝜟 $ & {\textbackslash}biDelta & Mathematical Bold Italic Capital Delta \\ \hline
U+1D720 & $ 𝜠 $ & {\textbackslash}biEpsilon & Mathematical Bold Italic Capital Epsilon \\ \hline
U+1D721 & $ 𝜡 $ & {\textbackslash}biZeta & Mathematical Bold Italic Capital Zeta \\ \hline
U+1D722 & $ 𝜢 $ & {\textbackslash}biEta & Mathematical Bold Italic Capital Eta \\ \hline
U+1D723 & $ 𝜣 $ & {\textbackslash}biTheta & Mathematical Bold Italic Capital Theta \\ \hline
U+1D724 & $ 𝜤 $ & {\textbackslash}biIota & Mathematical Bold Italic Capital Iota \\ \hline
U+1D725 & $ 𝜥 $ & {\textbackslash}biKappa & Mathematical Bold Italic Capital Kappa \\ \hline
U+1D726 & $ 𝜦 $ & {\textbackslash}biLambda & Mathematical Bold Italic Capital Lamda \\ \hline
U+1D727 & $ 𝜧 $ & {\textbackslash}biMu & Mathematical Bold Italic Capital Mu \\ \hline
U+1D728 & $ 𝜨 $ & {\textbackslash}biNu & Mathematical Bold Italic Capital Nu \\ \hline
U+1D729 & $ 𝜩 $ & {\textbackslash}biXi & Mathematical Bold Italic Capital Xi \\ \hline
U+1D72A & $ 𝜪 $ & {\textbackslash}biOmicron & Mathematical Bold Italic Capital Omicron \\ \hline
U+1D72B & $ 𝜫 $ & {\textbackslash}biPi & Mathematical Bold Italic Capital Pi \\ \hline
U+1D72C & $ 𝜬 $ & {\textbackslash}biRho & Mathematical Bold Italic Capital Rho \\ \hline
U+1D72D & $ 𝜭 $ & {\textbackslash}bivarTheta & Mathematical Bold Italic Capital Theta Symbol \\ \hline
U+1D72E & $ 𝜮 $ & {\textbackslash}biSigma & Mathematical Bold Italic Capital Sigma \\ \hline
U+1D72F & $ 𝜯 $ & {\textbackslash}biTau & Mathematical Bold Italic Capital Tau \\ \hline
U+1D730 & $ 𝜰 $ & {\textbackslash}biUpsilon & Mathematical Bold Italic Capital Upsilon \\ \hline
U+1D731 & $ 𝜱 $ & {\textbackslash}biPhi & Mathematical Bold Italic Capital Phi \\ \hline
U+1D732 & $ 𝜲 $ & {\textbackslash}biChi & Mathematical Bold Italic Capital Chi \\ \hline
U+1D733 & $ 𝜳 $ & {\textbackslash}biPsi & Mathematical Bold Italic Capital Psi \\ \hline
U+1D734 & $ 𝜴 $ & {\textbackslash}biOmega & Mathematical Bold Italic Capital Omega \\ \hline
U+1D735 & $ 𝜵 $ & {\textbackslash}binabla & Mathematical Bold Italic Nabla \\ \hline
U+1D736 & $ 𝜶 $ & {\textbackslash}bialpha & Mathematical Bold Italic Small Alpha \\ \hline
U+1D737 & $ 𝜷 $ & {\textbackslash}bibeta & Mathematical Bold Italic Small Beta \\ \hline
U+1D738 & $ 𝜸 $ & {\textbackslash}bigamma & Mathematical Bold Italic Small Gamma \\ \hline
U+1D739 & $ 𝜹 $ & {\textbackslash}bidelta & Mathematical Bold Italic Small Delta \\ \hline
U+1D73A & $ 𝜺 $ & {\textbackslash}biepsilon & Mathematical Bold Italic Small Epsilon \\ \hline
U+1D73B & $ 𝜻 $ & {\textbackslash}bizeta & Mathematical Bold Italic Small Zeta \\ \hline
U+1D73C & $ 𝜼 $ & {\textbackslash}bieta & Mathematical Bold Italic Small Eta \\ \hline
U+1D73D & $ 𝜽 $ & {\textbackslash}bitheta & Mathematical Bold Italic Small Theta \\ \hline
U+1D73E & $ 𝜾 $ & {\textbackslash}biiota & Mathematical Bold Italic Small Iota \\ \hline
U+1D73F & $ 𝜿 $ & {\textbackslash}bikappa & Mathematical Bold Italic Small Kappa \\ \hline
U+1D740 & $ 𝝀 $ & {\textbackslash}bilambda & Mathematical Bold Italic Small Lamda \\ \hline
U+1D741 & $ 𝝁 $ & {\textbackslash}bimu & Mathematical Bold Italic Small Mu \\ \hline
U+1D742 & $ 𝝂 $ & {\textbackslash}binu & Mathematical Bold Italic Small Nu \\ \hline
U+1D743 & $ 𝝃 $ & {\textbackslash}bixi & Mathematical Bold Italic Small Xi \\ \hline
U+1D744 & $ 𝝄 $ & {\textbackslash}biomicron & Mathematical Bold Italic Small Omicron \\ \hline
U+1D745 & $ 𝝅 $ & {\textbackslash}bipi & Mathematical Bold Italic Small Pi \\ \hline
U+1D746 & $ 𝝆 $ & {\textbackslash}birho & Mathematical Bold Italic Small Rho \\ \hline
U+1D747 & $ 𝝇 $ & {\textbackslash}bivarsigma & Mathematical Bold Italic Small Final Sigma \\ \hline
U+1D748 & $ 𝝈 $ & {\textbackslash}bisigma & Mathematical Bold Italic Small Sigma \\ \hline
U+1D749 & $ 𝝉 $ & {\textbackslash}bitau & Mathematical Bold Italic Small Tau \\ \hline
U+1D74A & $ 𝝊 $ & {\textbackslash}biupsilon & Mathematical Bold Italic Small Upsilon \\ \hline
U+1D74B & $ 𝝋 $ & {\textbackslash}biphi & Mathematical Bold Italic Small Phi \\ \hline
U+1D74C & $ 𝝌 $ & {\textbackslash}bichi & Mathematical Bold Italic Small Chi \\ \hline
U+1D74D & $ 𝝍 $ & {\textbackslash}bipsi & Mathematical Bold Italic Small Psi \\ \hline
U+1D74E & $ 𝝎 $ & {\textbackslash}biomega & Mathematical Bold Italic Small Omega \\ \hline
U+1D74F & $ 𝝏 $ & {\textbackslash}bipartial & Mathematical Bold Italic Partial Differential \\ \hline
U+1D750 & $ 𝝐 $ & {\textbackslash}bivarepsilon & Mathematical Bold Italic Epsilon Symbol \\ \hline
U+1D751 & $ 𝝑 $ & {\textbackslash}bivartheta & Mathematical Bold Italic Theta Symbol \\ \hline
U+1D752 & $ 𝝒 $ & {\textbackslash}bivarkappa & Mathematical Bold Italic Kappa Symbol \\ \hline
U+1D753 & $ 𝝓 $ & {\textbackslash}bivarphi & Mathematical Bold Italic Phi Symbol \\ \hline
U+1D754 & $ 𝝔 $ & {\textbackslash}bivarrho & Mathematical Bold Italic Rho Symbol \\ \hline
U+1D755 & $ 𝝕 $ & {\textbackslash}bivarpi & Mathematical Bold Italic Pi Symbol \\ \hline
U+1D756 & $ 𝝖 $ & {\textbackslash}bsansAlpha & Mathematical Sans-Serif Bold Capital Alpha \\ \hline
U+1D757 & $ 𝝗 $ & {\textbackslash}bsansBeta & Mathematical Sans-Serif Bold Capital Beta \\ \hline
U+1D758 & $ 𝝘 $ & {\textbackslash}bsansGamma & Mathematical Sans-Serif Bold Capital Gamma \\ \hline
U+1D759 & $ 𝝙 $ & {\textbackslash}bsansDelta & Mathematical Sans-Serif Bold Capital Delta \\ \hline
U+1D75A & $ 𝝚 $ & {\textbackslash}bsansEpsilon & Mathematical Sans-Serif Bold Capital Epsilon \\ \hline
U+1D75B & $ 𝝛 $ & {\textbackslash}bsansZeta & Mathematical Sans-Serif Bold Capital Zeta \\ \hline
U+1D75C & $ 𝝜 $ & {\textbackslash}bsansEta & Mathematical Sans-Serif Bold Capital Eta \\ \hline
U+1D75D & $ 𝝝 $ & {\textbackslash}bsansTheta & Mathematical Sans-Serif Bold Capital Theta \\ \hline
U+1D75E & $ 𝝞 $ & {\textbackslash}bsansIota & Mathematical Sans-Serif Bold Capital Iota \\ \hline
U+1D75F & $ 𝝟 $ & {\textbackslash}bsansKappa & Mathematical Sans-Serif Bold Capital Kappa \\ \hline
U+1D760 & $ 𝝠 $ & {\textbackslash}bsansLambda & Mathematical Sans-Serif Bold Capital Lamda \\ \hline
U+1D761 & $ 𝝡 $ & {\textbackslash}bsansMu & Mathematical Sans-Serif Bold Capital Mu \\ \hline
U+1D762 & $ 𝝢 $ & {\textbackslash}bsansNu & Mathematical Sans-Serif Bold Capital Nu \\ \hline
U+1D763 & $ 𝝣 $ & {\textbackslash}bsansXi & Mathematical Sans-Serif Bold Capital Xi \\ \hline
U+1D764 & $ 𝝤 $ & {\textbackslash}bsansOmicron & Mathematical Sans-Serif Bold Capital Omicron \\ \hline
U+1D765 & $ 𝝥 $ & {\textbackslash}bsansPi & Mathematical Sans-Serif Bold Capital Pi \\ \hline
U+1D766 & $ 𝝦 $ & {\textbackslash}bsansRho & Mathematical Sans-Serif Bold Capital Rho \\ \hline
U+1D767 & $ 𝝧 $ & {\textbackslash}bsansvarTheta & Mathematical Sans-Serif Bold Capital Theta Symbol \\ \hline
U+1D768 & $ 𝝨 $ & {\textbackslash}bsansSigma & Mathematical Sans-Serif Bold Capital Sigma \\ \hline
U+1D769 & $ 𝝩 $ & {\textbackslash}bsansTau & Mathematical Sans-Serif Bold Capital Tau \\ \hline
U+1D76A & $ 𝝪 $ & {\textbackslash}bsansUpsilon & Mathematical Sans-Serif Bold Capital Upsilon \\ \hline
U+1D76B & $ 𝝫 $ & {\textbackslash}bsansPhi & Mathematical Sans-Serif Bold Capital Phi \\ \hline
U+1D76C & $ 𝝬 $ & {\textbackslash}bsansChi & Mathematical Sans-Serif Bold Capital Chi \\ \hline
U+1D76D & $ 𝝭 $ & {\textbackslash}bsansPsi & Mathematical Sans-Serif Bold Capital Psi \\ \hline
U+1D76E & $ 𝝮 $ & {\textbackslash}bsansOmega & Mathematical Sans-Serif Bold Capital Omega \\ \hline
U+1D76F & $ 𝝯 $ & {\textbackslash}bsansnabla & Mathematical Sans-Serif Bold Nabla \\ \hline
U+1D770 & $ 𝝰 $ & {\textbackslash}bsansalpha & Mathematical Sans-Serif Bold Small Alpha \\ \hline
U+1D771 & $ 𝝱 $ & {\textbackslash}bsansbeta & Mathematical Sans-Serif Bold Small Beta \\ \hline
U+1D772 & $ 𝝲 $ & {\textbackslash}bsansgamma & Mathematical Sans-Serif Bold Small Gamma \\ \hline
U+1D773 & $ 𝝳 $ & {\textbackslash}bsansdelta & Mathematical Sans-Serif Bold Small Delta \\ \hline
U+1D774 & $ 𝝴 $ & {\textbackslash}bsansepsilon & Mathematical Sans-Serif Bold Small Epsilon \\ \hline
U+1D775 & $ 𝝵 $ & {\textbackslash}bsanszeta & Mathematical Sans-Serif Bold Small Zeta \\ \hline
U+1D776 & $ 𝝶 $ & {\textbackslash}bsanseta & Mathematical Sans-Serif Bold Small Eta \\ \hline
U+1D777 & $ 𝝷 $ & {\textbackslash}bsanstheta & Mathematical Sans-Serif Bold Small Theta \\ \hline
U+1D778 & $ 𝝸 $ & {\textbackslash}bsansiota & Mathematical Sans-Serif Bold Small Iota \\ \hline
U+1D779 & $ 𝝹 $ & {\textbackslash}bsanskappa & Mathematical Sans-Serif Bold Small Kappa \\ \hline
U+1D77A & $ 𝝺 $ & {\textbackslash}bsanslambda & Mathematical Sans-Serif Bold Small Lamda \\ \hline
U+1D77B & $ 𝝻 $ & {\textbackslash}bsansmu & Mathematical Sans-Serif Bold Small Mu \\ \hline
U+1D77C & $ 𝝼 $ & {\textbackslash}bsansnu & Mathematical Sans-Serif Bold Small Nu \\ \hline
U+1D77D & $ 𝝽 $ & {\textbackslash}bsansxi & Mathematical Sans-Serif Bold Small Xi \\ \hline
U+1D77E & $ 𝝾 $ & {\textbackslash}bsansomicron & Mathematical Sans-Serif Bold Small Omicron \\ \hline
U+1D77F & $ 𝝿 $ & {\textbackslash}bsanspi & Mathematical Sans-Serif Bold Small Pi \\ \hline
U+1D780 & $ 𝞀 $ & {\textbackslash}bsansrho & Mathematical Sans-Serif Bold Small Rho \\ \hline
U+1D781 & $ 𝞁 $ & {\textbackslash}bsansvarsigma & Mathematical Sans-Serif Bold Small Final Sigma \\ \hline
U+1D782 & $ 𝞂 $ & {\textbackslash}bsanssigma & Mathematical Sans-Serif Bold Small Sigma \\ \hline
U+1D783 & $ 𝞃 $ & {\textbackslash}bsanstau & Mathematical Sans-Serif Bold Small Tau \\ \hline
U+1D784 & $ 𝞄 $ & {\textbackslash}bsansupsilon & Mathematical Sans-Serif Bold Small Upsilon \\ \hline
U+1D785 & $ 𝞅 $ & {\textbackslash}bsansphi & Mathematical Sans-Serif Bold Small Phi \\ \hline
U+1D786 & $ 𝞆 $ & {\textbackslash}bsanschi & Mathematical Sans-Serif Bold Small Chi \\ \hline
U+1D787 & $ 𝞇 $ & {\textbackslash}bsanspsi & Mathematical Sans-Serif Bold Small Psi \\ \hline
U+1D788 & $ 𝞈 $ & {\textbackslash}bsansomega & Mathematical Sans-Serif Bold Small Omega \\ \hline
U+1D789 & $ 𝞉 $ & {\textbackslash}bsanspartial & Mathematical Sans-Serif Bold Partial Differential \\ \hline
U+1D78A & $ 𝞊 $ & {\textbackslash}bsansvarepsilon & Mathematical Sans-Serif Bold Epsilon Symbol \\ \hline
U+1D78B & $ 𝞋 $ & {\textbackslash}bsansvartheta & Mathematical Sans-Serif Bold Theta Symbol \\ \hline
U+1D78C & $ 𝞌 $ & {\textbackslash}bsansvarkappa & Mathematical Sans-Serif Bold Kappa Symbol \\ \hline
U+1D78D & $ 𝞍 $ & {\textbackslash}bsansvarphi & Mathematical Sans-Serif Bold Phi Symbol \\ \hline
U+1D78E & $ 𝞎 $ & {\textbackslash}bsansvarrho & Mathematical Sans-Serif Bold Rho Symbol \\ \hline
U+1D78F & $ 𝞏 $ & {\textbackslash}bsansvarpi & Mathematical Sans-Serif Bold Pi Symbol \\ \hline
U+1D790 & $ 𝞐 $ & {\textbackslash}bisansAlpha & Mathematical Sans-Serif Bold Italic Capital Alpha \\ \hline
U+1D791 & $ 𝞑 $ & {\textbackslash}bisansBeta & Mathematical Sans-Serif Bold Italic Capital Beta \\ \hline
U+1D792 & $ 𝞒 $ & {\textbackslash}bisansGamma & Mathematical Sans-Serif Bold Italic Capital Gamma \\ \hline
U+1D793 & $ 𝞓 $ & {\textbackslash}bisansDelta & Mathematical Sans-Serif Bold Italic Capital Delta \\ \hline
U+1D794 & $ 𝞔 $ & {\textbackslash}bisansEpsilon & Mathematical Sans-Serif Bold Italic Capital Epsilon \\ \hline
U+1D795 & $ 𝞕 $ & {\textbackslash}bisansZeta & Mathematical Sans-Serif Bold Italic Capital Zeta \\ \hline
U+1D796 & $ 𝞖 $ & {\textbackslash}bisansEta & Mathematical Sans-Serif Bold Italic Capital Eta \\ \hline
U+1D797 & $ 𝞗 $ & {\textbackslash}bisansTheta & Mathematical Sans-Serif Bold Italic Capital Theta \\ \hline
U+1D798 & $ 𝞘 $ & {\textbackslash}bisansIota & Mathematical Sans-Serif Bold Italic Capital Iota \\ \hline
U+1D799 & $ 𝞙 $ & {\textbackslash}bisansKappa & Mathematical Sans-Serif Bold Italic Capital Kappa \\ \hline
U+1D79A & $ 𝞚 $ & {\textbackslash}bisansLambda & Mathematical Sans-Serif Bold Italic Capital Lamda \\ \hline
U+1D79B & $ 𝞛 $ & {\textbackslash}bisansMu & Mathematical Sans-Serif Bold Italic Capital Mu \\ \hline
U+1D79C & $ 𝞜 $ & {\textbackslash}bisansNu & Mathematical Sans-Serif Bold Italic Capital Nu \\ \hline
U+1D79D & $ 𝞝 $ & {\textbackslash}bisansXi & Mathematical Sans-Serif Bold Italic Capital Xi \\ \hline
U+1D79E & $ 𝞞 $ & {\textbackslash}bisansOmicron & Mathematical Sans-Serif Bold Italic Capital Omicron \\ \hline
U+1D79F & $ 𝞟 $ & {\textbackslash}bisansPi & Mathematical Sans-Serif Bold Italic Capital Pi \\ \hline
U+1D7A0 & $ 𝞠 $ & {\textbackslash}bisansRho & Mathematical Sans-Serif Bold Italic Capital Rho \\ \hline
U+1D7A1 & $ 𝞡 $ & {\textbackslash}bisansvarTheta & Mathematical Sans-Serif Bold Italic Capital Theta Symbol \\ \hline
U+1D7A2 & $ 𝞢 $ & {\textbackslash}bisansSigma & Mathematical Sans-Serif Bold Italic Capital Sigma \\ \hline
U+1D7A3 & $ 𝞣 $ & {\textbackslash}bisansTau & Mathematical Sans-Serif Bold Italic Capital Tau \\ \hline
U+1D7A4 & $ 𝞤 $ & {\textbackslash}bisansUpsilon & Mathematical Sans-Serif Bold Italic Capital Upsilon \\ \hline
U+1D7A5 & $ 𝞥 $ & {\textbackslash}bisansPhi & Mathematical Sans-Serif Bold Italic Capital Phi \\ \hline
U+1D7A6 & $ 𝞦 $ & {\textbackslash}bisansChi & Mathematical Sans-Serif Bold Italic Capital Chi \\ \hline
U+1D7A7 & $ 𝞧 $ & {\textbackslash}bisansPsi & Mathematical Sans-Serif Bold Italic Capital Psi \\ \hline
U+1D7A8 & $ 𝞨 $ & {\textbackslash}bisansOmega & Mathematical Sans-Serif Bold Italic Capital Omega \\ \hline
U+1D7A9 & $ 𝞩 $ & {\textbackslash}bisansnabla & Mathematical Sans-Serif Bold Italic Nabla \\ \hline
U+1D7AA & $ 𝞪 $ & {\textbackslash}bisansalpha & Mathematical Sans-Serif Bold Italic Small Alpha \\ \hline
U+1D7AB & $ 𝞫 $ & {\textbackslash}bisansbeta & Mathematical Sans-Serif Bold Italic Small Beta \\ \hline
U+1D7AC & $ 𝞬 $ & {\textbackslash}bisansgamma & Mathematical Sans-Serif Bold Italic Small Gamma \\ \hline
U+1D7AD & $ 𝞭 $ & {\textbackslash}bisansdelta & Mathematical Sans-Serif Bold Italic Small Delta \\ \hline
U+1D7AE & $ 𝞮 $ & {\textbackslash}bisansepsilon & Mathematical Sans-Serif Bold Italic Small Epsilon \\ \hline
U+1D7AF & $ 𝞯 $ & {\textbackslash}bisanszeta & Mathematical Sans-Serif Bold Italic Small Zeta \\ \hline
U+1D7B0 & $ 𝞰 $ & {\textbackslash}bisanseta & Mathematical Sans-Serif Bold Italic Small Eta \\ \hline
U+1D7B1 & $ 𝞱 $ & {\textbackslash}bisanstheta & Mathematical Sans-Serif Bold Italic Small Theta \\ \hline
U+1D7B2 & $ 𝞲 $ & {\textbackslash}bisansiota & Mathematical Sans-Serif Bold Italic Small Iota \\ \hline
U+1D7B3 & $ 𝞳 $ & {\textbackslash}bisanskappa & Mathematical Sans-Serif Bold Italic Small Kappa \\ \hline
U+1D7B4 & $ 𝞴 $ & {\textbackslash}bisanslambda & Mathematical Sans-Serif Bold Italic Small Lamda \\ \hline
U+1D7B5 & $ 𝞵 $ & {\textbackslash}bisansmu & Mathematical Sans-Serif Bold Italic Small Mu \\ \hline
U+1D7B6 & $ 𝞶 $ & {\textbackslash}bisansnu & Mathematical Sans-Serif Bold Italic Small Nu \\ \hline
U+1D7B7 & $ 𝞷 $ & {\textbackslash}bisansxi & Mathematical Sans-Serif Bold Italic Small Xi \\ \hline
U+1D7B8 & $ 𝞸 $ & {\textbackslash}bisansomicron & Mathematical Sans-Serif Bold Italic Small Omicron \\ \hline
U+1D7B9 & $ 𝞹 $ & {\textbackslash}bisanspi & Mathematical Sans-Serif Bold Italic Small Pi \\ \hline
U+1D7BA & $ 𝞺 $ & {\textbackslash}bisansrho & Mathematical Sans-Serif Bold Italic Small Rho \\ \hline
U+1D7BB & $ 𝞻 $ & {\textbackslash}bisansvarsigma & Mathematical Sans-Serif Bold Italic Small Final Sigma \\ \hline
U+1D7BC & $ 𝞼 $ & {\textbackslash}bisanssigma & Mathematical Sans-Serif Bold Italic Small Sigma \\ \hline
U+1D7BD & $ 𝞽 $ & {\textbackslash}bisanstau & Mathematical Sans-Serif Bold Italic Small Tau \\ \hline
U+1D7BE & $ 𝞾 $ & {\textbackslash}bisansupsilon & Mathematical Sans-Serif Bold Italic Small Upsilon \\ \hline
U+1D7BF & $ 𝞿 $ & {\textbackslash}bisansphi & Mathematical Sans-Serif Bold Italic Small Phi \\ \hline
U+1D7C0 & $ 𝟀 $ & {\textbackslash}bisanschi & Mathematical Sans-Serif Bold Italic Small Chi \\ \hline
U+1D7C1 & $ 𝟁 $ & {\textbackslash}bisanspsi & Mathematical Sans-Serif Bold Italic Small Psi \\ \hline
U+1D7C2 & $ 𝟂 $ & {\textbackslash}bisansomega & Mathematical Sans-Serif Bold Italic Small Omega \\ \hline
U+1D7C3 & $ 𝟃 $ & {\textbackslash}bisanspartial & Mathematical Sans-Serif Bold Italic Partial Differential \\ \hline
U+1D7C4 & $ 𝟄 $ & {\textbackslash}bisansvarepsilon & Mathematical Sans-Serif Bold Italic Epsilon Symbol \\ \hline
U+1D7C5 & $ 𝟅 $ & {\textbackslash}bisansvartheta & Mathematical Sans-Serif Bold Italic Theta Symbol \\ \hline
U+1D7C6 & $ 𝟆 $ & {\textbackslash}bisansvarkappa & Mathematical Sans-Serif Bold Italic Kappa Symbol \\ \hline
U+1D7C7 & $ 𝟇 $ & {\textbackslash}bisansvarphi & Mathematical Sans-Serif Bold Italic Phi Symbol \\ \hline
U+1D7C8 & $ 𝟈 $ & {\textbackslash}bisansvarrho & Mathematical Sans-Serif Bold Italic Rho Symbol \\ \hline
U+1D7C9 & $ 𝟉 $ & {\textbackslash}bisansvarpi & Mathematical Sans-Serif Bold Italic Pi Symbol \\ \hline
U+1D7CA & $ 𝟊 $ & {\textbackslash}bfDigamma & Mathematical Bold Capital Digamma \\ \hline
U+1D7CB & $ 𝟋 $ & {\textbackslash}bfdigamma & Mathematical Bold Small Digamma \\ \hline
U+1D7CE & $ 𝟎 $ & {\textbackslash}bfzero & Mathematical Bold Digit Zero \\ \hline
U+1D7CF & $ 𝟏 $ & {\textbackslash}bfone & Mathematical Bold Digit One \\ \hline
U+1D7D0 & $ 𝟐 $ & {\textbackslash}bftwo & Mathematical Bold Digit Two \\ \hline
U+1D7D1 & $ 𝟑 $ & {\textbackslash}bfthree & Mathematical Bold Digit Three \\ \hline
U+1D7D2 & $ 𝟒 $ & {\textbackslash}bffour & Mathematical Bold Digit Four \\ \hline
U+1D7D3 & $ 𝟓 $ & {\textbackslash}bffive & Mathematical Bold Digit Five \\ \hline
U+1D7D4 & $ 𝟔 $ & {\textbackslash}bfsix & Mathematical Bold Digit Six \\ \hline
U+1D7D5 & $ 𝟕 $ & {\textbackslash}bfseven & Mathematical Bold Digit Seven \\ \hline
U+1D7D6 & $ 𝟖 $ & {\textbackslash}bfeight & Mathematical Bold Digit Eight \\ \hline
U+1D7D7 & $ 𝟗 $ & {\textbackslash}bfnine & Mathematical Bold Digit Nine \\ \hline
U+1D7D8 & $ 𝟘 $ & {\textbackslash}bbzero & Mathematical Double-Struck Digit Zero \\ \hline
U+1D7D9 & $ 𝟙 $ & {\textbackslash}bbone & Mathematical Double-Struck Digit One \\ \hline
U+1D7DA & $ 𝟚 $ & {\textbackslash}bbtwo & Mathematical Double-Struck Digit Two \\ \hline
U+1D7DB & $ 𝟛 $ & {\textbackslash}bbthree & Mathematical Double-Struck Digit Three \\ \hline
U+1D7DC & $ 𝟜 $ & {\textbackslash}bbfour & Mathematical Double-Struck Digit Four \\ \hline
U+1D7DD & $ 𝟝 $ & {\textbackslash}bbfive & Mathematical Double-Struck Digit Five \\ \hline
U+1D7DE & $ 𝟞 $ & {\textbackslash}bbsix & Mathematical Double-Struck Digit Six \\ \hline
U+1D7DF & $ 𝟟 $ & {\textbackslash}bbseven & Mathematical Double-Struck Digit Seven \\ \hline
U+1D7E0 & $ 𝟠 $ & {\textbackslash}bbeight & Mathematical Double-Struck Digit Eight \\ \hline
U+1D7E1 & $ 𝟡 $ & {\textbackslash}bbnine & Mathematical Double-Struck Digit Nine \\ \hline
U+1D7E2 & $ 𝟢 $ & {\textbackslash}sanszero & Mathematical Sans-Serif Digit Zero \\ \hline
U+1D7E3 & $ 𝟣 $ & {\textbackslash}sansone & Mathematical Sans-Serif Digit One \\ \hline
U+1D7E4 & $ 𝟤 $ & {\textbackslash}sanstwo & Mathematical Sans-Serif Digit Two \\ \hline
U+1D7E5 & $ 𝟥 $ & {\textbackslash}sansthree & Mathematical Sans-Serif Digit Three \\ \hline
U+1D7E6 & $ 𝟦 $ & {\textbackslash}sansfour & Mathematical Sans-Serif Digit Four \\ \hline
U+1D7E7 & $ 𝟧 $ & {\textbackslash}sansfive & Mathematical Sans-Serif Digit Five \\ \hline
U+1D7E8 & $ 𝟨 $ & {\textbackslash}sanssix & Mathematical Sans-Serif Digit Six \\ \hline
U+1D7E9 & $ 𝟩 $ & {\textbackslash}sansseven & Mathematical Sans-Serif Digit Seven \\ \hline
U+1D7EA & $ 𝟪 $ & {\textbackslash}sanseight & Mathematical Sans-Serif Digit Eight \\ \hline
U+1D7EB & $ 𝟫 $ & {\textbackslash}sansnine & Mathematical Sans-Serif Digit Nine \\ \hline
U+1D7EC & $ 𝟬 $ & {\textbackslash}bsanszero & Mathematical Sans-Serif Bold Digit Zero \\ \hline
U+1D7ED & $ 𝟭 $ & {\textbackslash}bsansone & Mathematical Sans-Serif Bold Digit One \\ \hline
U+1D7EE & $ 𝟮 $ & {\textbackslash}bsanstwo & Mathematical Sans-Serif Bold Digit Two \\ \hline
U+1D7EF & $ 𝟯 $ & {\textbackslash}bsansthree & Mathematical Sans-Serif Bold Digit Three \\ \hline
U+1D7F0 & $ 𝟰 $ & {\textbackslash}bsansfour & Mathematical Sans-Serif Bold Digit Four \\ \hline
U+1D7F1 & $ 𝟱 $ & {\textbackslash}bsansfive & Mathematical Sans-Serif Bold Digit Five \\ \hline
U+1D7F2 & $ 𝟲 $ & {\textbackslash}bsanssix & Mathematical Sans-Serif Bold Digit Six \\ \hline
U+1D7F3 & $ 𝟳 $ & {\textbackslash}bsansseven & Mathematical Sans-Serif Bold Digit Seven \\ \hline
U+1D7F4 & $ 𝟴 $ & {\textbackslash}bsanseight & Mathematical Sans-Serif Bold Digit Eight \\ \hline
U+1D7F5 & $ 𝟵 $ & {\textbackslash}bsansnine & Mathematical Sans-Serif Bold Digit Nine \\ \hline
U+1D7F6 & $ 𝟶 $ & {\textbackslash}ttzero & Mathematical Monospace Digit Zero \\ \hline
U+1D7F7 & $ 𝟷 $ & {\textbackslash}ttone & Mathematical Monospace Digit One \\ \hline
U+1D7F8 & $ 𝟸 $ & {\textbackslash}tttwo & Mathematical Monospace Digit Two \\ \hline
U+1D7F9 & $ 𝟹 $ & {\textbackslash}ttthree & Mathematical Monospace Digit Three \\ \hline
U+1D7FA & $ 𝟺 $ & {\textbackslash}ttfour & Mathematical Monospace Digit Four \\ \hline
U+1D7FB & $ 𝟻 $ & {\textbackslash}ttfive & Mathematical Monospace Digit Five \\ \hline
U+1D7FC & $ 𝟼 $ & {\textbackslash}ttsix & Mathematical Monospace Digit Six \\ \hline
U+1D7FD & $ 𝟽 $ & {\textbackslash}ttseven & Mathematical Monospace Digit Seven \\ \hline
U+1D7FE & $ 𝟾 $ & {\textbackslash}tteight & Mathematical Monospace Digit Eight \\ \hline
U+1D7FF & $ 𝟿 $ & {\textbackslash}ttnine & Mathematical Monospace Digit Nine \\ \hline
U+1F004 & {\EmojiFont 🀄} & {\textbackslash}:mahjong: & Mahjong Tile Red Dragon \\ \hline
U+1F0CF & {\EmojiFont 🃏} & {\textbackslash}:black\_joker: & Playing Card Black Joker \\ \hline
U+1F170 & {\EmojiFont 🅰} & {\textbackslash}:a: & Negative Squared Latin Capital Letter A \\ \hline
U+1F171 & {\EmojiFont 🅱} & {\textbackslash}:b: & Negative Squared Latin Capital Letter B \\ \hline
U+1F17E & {\EmojiFont 🅾} & {\textbackslash}:o2: & Negative Squared Latin Capital Letter O \\ \hline
U+1F17F & {\EmojiFont 🅿} & {\textbackslash}:parking: & Negative Squared Latin Capital Letter P \\ \hline
U+1F18E & {\EmojiFont 🆎} & {\textbackslash}:ab: & Negative Squared Ab \\ \hline
U+1F191 & {\EmojiFont 🆑} & {\textbackslash}:cl: & Squared Cl \\ \hline
U+1F192 & {\EmojiFont 🆒} & {\textbackslash}:cool: & Squared Cool \\ \hline
U+1F193 & {\EmojiFont 🆓} & {\textbackslash}:free: & Squared Free \\ \hline
U+1F194 & {\EmojiFont 🆔} & {\textbackslash}:id: & Squared Id \\ \hline
U+1F195 & {\EmojiFont 🆕} & {\textbackslash}:new: & Squared New \\ \hline
U+1F196 & {\EmojiFont 🆖} & {\textbackslash}:ng: & Squared Ng \\ \hline
U+1F197 & {\EmojiFont 🆗} & {\textbackslash}:ok: & Squared Ok \\ \hline
U+1F198 & {\EmojiFont 🆘} & {\textbackslash}:sos: & Squared Sos \\ \hline
U+1F199 & {\EmojiFont 🆙} & {\textbackslash}:up: & Squared Up With Exclamation Mark \\ \hline
U+1F19A & {\EmojiFont 🆚} & {\textbackslash}:vs: & Squared Vs \\ \hline
U+1F201 & {\EmojiFont 🈁} & {\textbackslash}:koko: & Squared Katakana Koko \\ \hline
U+1F202 & {\EmojiFont 🈂} & {\textbackslash}:sa: & Squared Katakana Sa \\ \hline
U+1F21A & {\EmojiFont 🈚} & {\textbackslash}:u7121: & Squared Cjk Unified Ideograph-7121 \\ \hline
U+1F22F & {\EmojiFont 🈯} & {\textbackslash}:u6307: & Squared Cjk Unified Ideograph-6307 \\ \hline
U+1F232 & {\EmojiFont 🈲} & {\textbackslash}:u7981: & Squared Cjk Unified Ideograph-7981 \\ \hline
U+1F233 & {\EmojiFont 🈳} & {\textbackslash}:u7a7a: & Squared Cjk Unified Ideograph-7A7A \\ \hline
U+1F234 & {\EmojiFont 🈴} & {\textbackslash}:u5408: & Squared Cjk Unified Ideograph-5408 \\ \hline
U+1F235 & {\EmojiFont 🈵} & {\textbackslash}:u6e80: & Squared Cjk Unified Ideograph-6E80 \\ \hline
U+1F236 & {\EmojiFont 🈶} & {\textbackslash}:u6709: & Squared Cjk Unified Ideograph-6709 \\ \hline
U+1F237 & {\EmojiFont 🈷} & {\textbackslash}:u6708: & Squared Cjk Unified Ideograph-6708 \\ \hline
U+1F238 & {\EmojiFont 🈸} & {\textbackslash}:u7533: & Squared Cjk Unified Ideograph-7533 \\ \hline
U+1F239 & {\EmojiFont 🈹} & {\textbackslash}:u5272: & Squared Cjk Unified Ideograph-5272 \\ \hline
U+1F23A & {\EmojiFont 🈺} & {\textbackslash}:u55b6: & Squared Cjk Unified Ideograph-55B6 \\ \hline
U+1F250 & {\EmojiFont 🉐} & {\textbackslash}:ideograph\_advantage: & Circled Ideograph Advantage \\ \hline
U+1F251 & {\EmojiFont 🉑} & {\textbackslash}:accept: & Circled Ideograph Accept \\ \hline
U+1F300 & {\EmojiFont 🌀} & {\textbackslash}:cyclone: & Cyclone \\ \hline
U+1F301 & {\EmojiFont 🌁} & {\textbackslash}:foggy: & Foggy \\ \hline
U+1F302 & {\EmojiFont 🌂} & {\textbackslash}:closed\_umbrella: & Closed Umbrella \\ \hline
U+1F303 & {\EmojiFont 🌃} & {\textbackslash}:night\_with\_stars: & Night With Stars \\ \hline
U+1F304 & {\EmojiFont 🌄} & {\textbackslash}:sunrise\_over\_mountains: & Sunrise Over Mountains \\ \hline
U+1F305 & {\EmojiFont 🌅} & {\textbackslash}:sunrise: & Sunrise \\ \hline
U+1F306 & {\EmojiFont 🌆} & {\textbackslash}:city\_sunset: & Cityscape At Dusk \\ \hline
U+1F307 & {\EmojiFont 🌇} & {\textbackslash}:city\_sunrise: & Sunset Over Buildings \\ \hline
U+1F308 & {\EmojiFont 🌈} & {\textbackslash}:rainbow: & Rainbow \\ \hline
U+1F309 & {\EmojiFont 🌉} & {\textbackslash}:bridge\_at\_night: & Bridge At Night \\ \hline
U+1F30A & {\EmojiFont 🌊} & {\textbackslash}:ocean: & Water Wave \\ \hline
U+1F30B & {\EmojiFont 🌋} & {\textbackslash}:volcano: & Volcano \\ \hline
U+1F30C & {\EmojiFont 🌌} & {\textbackslash}:milky\_way: & Milky Way \\ \hline
U+1F30D & {\EmojiFont 🌍} & {\textbackslash}:earth\_africa: & Earth Globe Europe-Africa \\ \hline
U+1F30E & {\EmojiFont 🌎} & {\textbackslash}:earth\_americas: & Earth Globe Americas \\ \hline
U+1F30F & {\EmojiFont 🌏} & {\textbackslash}:earth\_asia: & Earth Globe Asia-Australia \\ \hline
U+1F310 & {\EmojiFont 🌐} & {\textbackslash}:globe\_with\_meridians: & Globe With Meridians \\ \hline
U+1F311 & {\EmojiFont 🌑} & {\textbackslash}:new\_moon: & New Moon Symbol \\ \hline
U+1F312 & {\EmojiFont 🌒} & {\textbackslash}:waxing\_crescent\_moon: & Waxing Crescent Moon Symbol \\ \hline
U+1F313 & {\EmojiFont 🌓} & {\textbackslash}:first\_quarter\_moon: & First Quarter Moon Symbol \\ \hline
U+1F314 & {\EmojiFont 🌔} & {\textbackslash}:moon: & Waxing Gibbous Moon Symbol \\ \hline
U+1F315 & {\EmojiFont 🌕} & {\textbackslash}:full\_moon: & Full Moon Symbol \\ \hline
U+1F316 & {\EmojiFont 🌖} & {\textbackslash}:waning\_gibbous\_moon: & Waning Gibbous Moon Symbol \\ \hline
U+1F317 & {\EmojiFont 🌗} & {\textbackslash}:last\_quarter\_moon: & Last Quarter Moon Symbol \\ \hline
U+1F318 & {\EmojiFont 🌘} & {\textbackslash}:waning\_crescent\_moon: & Waning Crescent Moon Symbol \\ \hline
U+1F319 & {\EmojiFont 🌙} & {\textbackslash}:crescent\_moon: & Crescent Moon \\ \hline
U+1F31A & {\EmojiFont 🌚} & {\textbackslash}:new\_moon\_with\_face: & New Moon With Face \\ \hline
U+1F31B & {\EmojiFont 🌛} & {\textbackslash}:first\_quarter\_moon\_with\_face: & First Quarter Moon With Face \\ \hline
U+1F31C & {\EmojiFont 🌜} & {\textbackslash}:last\_quarter\_moon\_with\_face: & Last Quarter Moon With Face \\ \hline
U+1F31D & {\EmojiFont 🌝} & {\textbackslash}:full\_moon\_with\_face: & Full Moon With Face \\ \hline
U+1F31E & {\EmojiFont 🌞} & {\textbackslash}:sun\_with\_face: & Sun With Face \\ \hline
U+1F31F & {\EmojiFont 🌟} & {\textbackslash}:star2: & Glowing Star \\ \hline
U+1F320 & {\EmojiFont 🌠} & {\textbackslash}:stars: & Shooting Star \\ \hline
U+1F330 & {\EmojiFont 🌰} & {\textbackslash}:chestnut: & Chestnut \\ \hline
U+1F331 & {\EmojiFont 🌱} & {\textbackslash}:seedling: & Seedling \\ \hline
U+1F332 & {\EmojiFont 🌲} & {\textbackslash}:evergreen\_tree: & Evergreen Tree \\ \hline
U+1F333 & {\EmojiFont 🌳} & {\textbackslash}:deciduous\_tree: & Deciduous Tree \\ \hline
U+1F334 & {\EmojiFont 🌴} & {\textbackslash}:palm\_tree: & Palm Tree \\ \hline
U+1F335 & {\EmojiFont 🌵} & {\textbackslash}:cactus: & Cactus \\ \hline
U+1F337 & {\EmojiFont 🌷} & {\textbackslash}:tulip: & Tulip \\ \hline
U+1F338 & {\EmojiFont 🌸} & {\textbackslash}:cherry\_blossom: & Cherry Blossom \\ \hline
U+1F339 & {\EmojiFont 🌹} & {\textbackslash}:rose: & Rose \\ \hline
U+1F33A & {\EmojiFont 🌺} & {\textbackslash}:hibiscus: & Hibiscus \\ \hline
U+1F33B & {\EmojiFont 🌻} & {\textbackslash}:sunflower: & Sunflower \\ \hline
U+1F33C & {\EmojiFont 🌼} & {\textbackslash}:blossom: & Blossom \\ \hline
U+1F33D & {\EmojiFont 🌽} & {\textbackslash}:corn: & Ear Of Maize \\ \hline
U+1F33E & {\EmojiFont 🌾} & {\textbackslash}:ear\_of\_rice: & Ear Of Rice \\ \hline
U+1F33F & {\EmojiFont 🌿} & {\textbackslash}:herb: & Herb \\ \hline
U+1F340 & {\EmojiFont 🍀} & {\textbackslash}:four\_leaf\_clover: & Four Leaf Clover \\ \hline
U+1F341 & {\EmojiFont 🍁} & {\textbackslash}:maple\_leaf: & Maple Leaf \\ \hline
U+1F342 & {\EmojiFont 🍂} & {\textbackslash}:fallen\_leaf: & Fallen Leaf \\ \hline
U+1F343 & {\EmojiFont 🍃} & {\textbackslash}:leaves: & Leaf Fluttering In Wind \\ \hline
U+1F344 & {\EmojiFont 🍄} & {\textbackslash}:mushroom: & Mushroom \\ \hline
U+1F345 & {\EmojiFont 🍅} & {\textbackslash}:tomato: & Tomato \\ \hline
U+1F346 & {\EmojiFont 🍆} & {\textbackslash}:eggplant: & Aubergine \\ \hline
U+1F347 & {\EmojiFont 🍇} & {\textbackslash}:grapes: & Grapes \\ \hline
U+1F348 & {\EmojiFont 🍈} & {\textbackslash}:melon: & Melon \\ \hline
U+1F349 & {\EmojiFont 🍉} & {\textbackslash}:watermelon: & Watermelon \\ \hline
U+1F34A & {\EmojiFont 🍊} & {\textbackslash}:tangerine: & Tangerine \\ \hline
U+1F34B & {\EmojiFont 🍋} & {\textbackslash}:lemon: & Lemon \\ \hline
U+1F34C & {\EmojiFont 🍌} & {\textbackslash}:banana: & Banana \\ \hline
U+1F34D & {\EmojiFont 🍍} & {\textbackslash}:pineapple: & Pineapple \\ \hline
U+1F34E & {\EmojiFont 🍎} & {\textbackslash}:apple: & Red Apple \\ \hline
U+1F34F & {\EmojiFont 🍏} & {\textbackslash}:green\_apple: & Green Apple \\ \hline
U+1F350 & {\EmojiFont 🍐} & {\textbackslash}:pear: & Pear \\ \hline
U+1F351 & {\EmojiFont 🍑} & {\textbackslash}:peach: & Peach \\ \hline
U+1F352 & {\EmojiFont 🍒} & {\textbackslash}:cherries: & Cherries \\ \hline
U+1F353 & {\EmojiFont 🍓} & {\textbackslash}:strawberry: & Strawberry \\ \hline
U+1F354 & {\EmojiFont 🍔} & {\textbackslash}:hamburger: & Hamburger \\ \hline
U+1F355 & {\EmojiFont 🍕} & {\textbackslash}:pizza: & Slice Of Pizza \\ \hline
U+1F356 & {\EmojiFont 🍖} & {\textbackslash}:meat\_on\_bone: & Meat On Bone \\ \hline
U+1F357 & {\EmojiFont 🍗} & {\textbackslash}:poultry\_leg: & Poultry Leg \\ \hline
U+1F358 & {\EmojiFont 🍘} & {\textbackslash}:rice\_cracker: & Rice Cracker \\ \hline
U+1F359 & {\EmojiFont 🍙} & {\textbackslash}:rice\_ball: & Rice Ball \\ \hline
U+1F35A & {\EmojiFont 🍚} & {\textbackslash}:rice: & Cooked Rice \\ \hline
U+1F35B & {\EmojiFont 🍛} & {\textbackslash}:curry: & Curry And Rice \\ \hline
U+1F35C & {\EmojiFont 🍜} & {\textbackslash}:ramen: & Steaming Bowl \\ \hline
U+1F35D & {\EmojiFont 🍝} & {\textbackslash}:spaghetti: & Spaghetti \\ \hline
U+1F35E & {\EmojiFont 🍞} & {\textbackslash}:bread: & Bread \\ \hline
U+1F35F & {\EmojiFont 🍟} & {\textbackslash}:fries: & French Fries \\ \hline
U+1F360 & {\EmojiFont 🍠} & {\textbackslash}:sweet\_potato: & Roasted Sweet Potato \\ \hline
U+1F361 & {\EmojiFont 🍡} & {\textbackslash}:dango: & Dango \\ \hline
U+1F362 & {\EmojiFont 🍢} & {\textbackslash}:oden: & Oden \\ \hline
U+1F363 & {\EmojiFont 🍣} & {\textbackslash}:sushi: & Sushi \\ \hline
U+1F364 & {\EmojiFont 🍤} & {\textbackslash}:fried\_shrimp: & Fried Shrimp \\ \hline
U+1F365 & {\EmojiFont 🍥} & {\textbackslash}:fish\_cake: & Fish Cake With Swirl Design \\ \hline
U+1F366 & {\EmojiFont 🍦} & {\textbackslash}:icecream: & Soft Ice Cream \\ \hline
U+1F367 & {\EmojiFont 🍧} & {\textbackslash}:shaved\_ice: & Shaved Ice \\ \hline
U+1F368 & {\EmojiFont 🍨} & {\textbackslash}:ice\_cream: & Ice Cream \\ \hline
U+1F369 & {\EmojiFont 🍩} & {\textbackslash}:doughnut: & Doughnut \\ \hline
U+1F36A & {\EmojiFont 🍪} & {\textbackslash}:cookie: & Cookie \\ \hline
U+1F36B & {\EmojiFont 🍫} & {\textbackslash}:chocolate\_bar: & Chocolate Bar \\ \hline
U+1F36C & {\EmojiFont 🍬} & {\textbackslash}:candy: & Candy \\ \hline
U+1F36D & {\EmojiFont 🍭} & {\textbackslash}:lollipop: & Lollipop \\ \hline
U+1F36E & {\EmojiFont 🍮} & {\textbackslash}:custard: & Custard \\ \hline
U+1F36F & {\EmojiFont 🍯} & {\textbackslash}:honey\_pot: & Honey Pot \\ \hline
U+1F370 & {\EmojiFont 🍰} & {\textbackslash}:cake: & Shortcake \\ \hline
U+1F371 & {\EmojiFont 🍱} & {\textbackslash}:bento: & Bento Box \\ \hline
U+1F372 & {\EmojiFont 🍲} & {\textbackslash}:stew: & Pot Of Food \\ \hline
U+1F373 & {\EmojiFont 🍳} & {\textbackslash}:egg: & Cooking \\ \hline
U+1F374 & {\EmojiFont 🍴} & {\textbackslash}:fork\_and\_knife: & Fork And Knife \\ \hline
U+1F375 & {\EmojiFont 🍵} & {\textbackslash}:tea: & Teacup Without Handle \\ \hline
U+1F376 & {\EmojiFont 🍶} & {\textbackslash}:sake: & Sake Bottle And Cup \\ \hline
U+1F377 & {\EmojiFont 🍷} & {\textbackslash}:wine\_glass: & Wine Glass \\ \hline
U+1F378 & {\EmojiFont 🍸} & {\textbackslash}:cocktail: & Cocktail Glass \\ \hline
U+1F379 & {\EmojiFont 🍹} & {\textbackslash}:tropical\_drink: & Tropical Drink \\ \hline
U+1F37A & {\EmojiFont 🍺} & {\textbackslash}:beer: & Beer Mug \\ \hline
U+1F37B & {\EmojiFont 🍻} & {\textbackslash}:beers: & Clinking Beer Mugs \\ \hline
U+1F37C & {\EmojiFont 🍼} & {\textbackslash}:baby\_bottle: & Baby Bottle \\ \hline
U+1F380 & {\EmojiFont 🎀} & {\textbackslash}:ribbon: & Ribbon \\ \hline
U+1F381 & {\EmojiFont 🎁} & {\textbackslash}:gift: & Wrapped Present \\ \hline
U+1F382 & {\EmojiFont 🎂} & {\textbackslash}:birthday: & Birthday Cake \\ \hline
U+1F383 & {\EmojiFont 🎃} & {\textbackslash}:jack\_o\_lantern: & Jack-O-Lantern \\ \hline
U+1F384 & {\EmojiFont 🎄} & {\textbackslash}:christmas\_tree: & Christmas Tree \\ \hline
U+1F385 & {\EmojiFont 🎅} & {\textbackslash}:santa: & Father Christmas \\ \hline
U+1F386 & {\EmojiFont 🎆} & {\textbackslash}:fireworks: & Fireworks \\ \hline
U+1F387 & {\EmojiFont 🎇} & {\textbackslash}:sparkler: & Firework Sparkler \\ \hline
U+1F388 & {\EmojiFont 🎈} & {\textbackslash}:balloon: & Balloon \\ \hline
U+1F389 & {\EmojiFont 🎉} & {\textbackslash}:tada: & Party Popper \\ \hline
U+1F38A & {\EmojiFont 🎊} & {\textbackslash}:confetti\_ball: & Confetti Ball \\ \hline
U+1F38B & {\EmojiFont 🎋} & {\textbackslash}:tanabata\_tree: & Tanabata Tree \\ \hline
U+1F38C & {\EmojiFont 🎌} & {\textbackslash}:crossed\_flags: & Crossed Flags \\ \hline
U+1F38D & {\EmojiFont 🎍} & {\textbackslash}:bamboo: & Pine Decoration \\ \hline
U+1F38E & {\EmojiFont 🎎} & {\textbackslash}:dolls: & Japanese Dolls \\ \hline
U+1F38F & {\EmojiFont 🎏} & {\textbackslash}:flags: & Carp Streamer \\ \hline
U+1F390 & {\EmojiFont 🎐} & {\textbackslash}:wind\_chime: & Wind Chime \\ \hline
U+1F391 & {\EmojiFont 🎑} & {\textbackslash}:rice\_scene: & Moon Viewing Ceremony \\ \hline
U+1F392 & {\EmojiFont 🎒} & {\textbackslash}:school\_satchel: & School Satchel \\ \hline
U+1F393 & {\EmojiFont 🎓} & {\textbackslash}:mortar\_board: & Graduation Cap \\ \hline
U+1F3A0 & {\EmojiFont 🎠} & {\textbackslash}:carousel\_horse: & Carousel Horse \\ \hline
U+1F3A1 & {\EmojiFont 🎡} & {\textbackslash}:ferris\_wheel: & Ferris Wheel \\ \hline
U+1F3A2 & {\EmojiFont 🎢} & {\textbackslash}:roller\_coaster: & Roller Coaster \\ \hline
U+1F3A3 & {\EmojiFont 🎣} & {\textbackslash}:fishing\_pole\_and\_fish: & Fishing Pole And Fish \\ \hline
U+1F3A4 & {\EmojiFont 🎤} & {\textbackslash}:microphone: & Microphone \\ \hline
U+1F3A5 & {\EmojiFont 🎥} & {\textbackslash}:movie\_camera: & Movie Camera \\ \hline
U+1F3A6 & {\EmojiFont 🎦} & {\textbackslash}:cinema: & Cinema \\ \hline
U+1F3A7 & {\EmojiFont 🎧} & {\textbackslash}:headphones: & Headphone \\ \hline
U+1F3A8 & {\EmojiFont 🎨} & {\textbackslash}:art: & Artist Palette \\ \hline
U+1F3A9 & {\EmojiFont 🎩} & {\textbackslash}:tophat: & Top Hat \\ \hline
U+1F3AA & {\EmojiFont 🎪} & {\textbackslash}:circus\_tent: & Circus Tent \\ \hline
U+1F3AB & {\EmojiFont 🎫} & {\textbackslash}:ticket: & Ticket \\ \hline
U+1F3AC & {\EmojiFont 🎬} & {\textbackslash}:clapper: & Clapper Board \\ \hline
U+1F3AD & {\EmojiFont 🎭} & {\textbackslash}:performing\_arts: & Performing Arts \\ \hline
U+1F3AE & {\EmojiFont 🎮} & {\textbackslash}:video\_game: & Video Game \\ \hline
U+1F3AF & {\EmojiFont 🎯} & {\textbackslash}:dart: & Direct Hit \\ \hline
U+1F3B0 & {\EmojiFont 🎰} & {\textbackslash}:slot\_machine: & Slot Machine \\ \hline
U+1F3B1 & {\EmojiFont 🎱} & {\textbackslash}:8ball: & Billiards \\ \hline
U+1F3B2 & {\EmojiFont 🎲} & {\textbackslash}:game\_die: & Game Die \\ \hline
U+1F3B3 & {\EmojiFont 🎳} & {\textbackslash}:bowling: & Bowling \\ \hline
U+1F3B4 & {\EmojiFont 🎴} & {\textbackslash}:flower\_playing\_cards: & Flower Playing Cards \\ \hline
U+1F3B5 & {\EmojiFont 🎵} & {\textbackslash}:musical\_note: & Musical Note \\ \hline
U+1F3B6 & {\EmojiFont 🎶} & {\textbackslash}:notes: & Multiple Musical Notes \\ \hline
U+1F3B7 & {\EmojiFont 🎷} & {\textbackslash}:saxophone: & Saxophone \\ \hline
U+1F3B8 & {\EmojiFont 🎸} & {\textbackslash}:guitar: & Guitar \\ \hline
U+1F3B9 & {\EmojiFont 🎹} & {\textbackslash}:musical\_keyboard: & Musical Keyboard \\ \hline
U+1F3BA & {\EmojiFont 🎺} & {\textbackslash}:trumpet: & Trumpet \\ \hline
U+1F3BB & {\EmojiFont 🎻} & {\textbackslash}:violin: & Violin \\ \hline
U+1F3BC & {\EmojiFont 🎼} & {\textbackslash}:musical\_score: & Musical Score \\ \hline
U+1F3BD & {\EmojiFont 🎽} & {\textbackslash}:running\_shirt\_with\_sash: & Running Shirt With Sash \\ \hline
U+1F3BE & {\EmojiFont 🎾} & {\textbackslash}:tennis: & Tennis Racquet And Ball \\ \hline
U+1F3BF & {\EmojiFont 🎿} & {\textbackslash}:ski: & Ski And Ski Boot \\ \hline
U+1F3C0 & {\EmojiFont 🏀} & {\textbackslash}:basketball: & Basketball And Hoop \\ \hline
U+1F3C1 & {\EmojiFont 🏁} & {\textbackslash}:checkered\_flag: & Chequered Flag \\ \hline
U+1F3C2 & {\EmojiFont 🏂} & {\textbackslash}:snowboarder: & Snowboarder \\ \hline
U+1F3C3 & {\EmojiFont 🏃} & {\textbackslash}:runner: & Runner \\ \hline
U+1F3C4 & {\EmojiFont 🏄} & {\textbackslash}:surfer: & Surfer \\ \hline
U+1F3C6 & {\EmojiFont 🏆} & {\textbackslash}:trophy: & Trophy \\ \hline
U+1F3C7 & {\EmojiFont 🏇} & {\textbackslash}:horse\_racing: & Horse Racing \\ \hline
U+1F3C8 & {\EmojiFont 🏈} & {\textbackslash}:football: & American Football \\ \hline
U+1F3C9 & {\EmojiFont 🏉} & {\textbackslash}:rugby\_football: & Rugby Football \\ \hline
U+1F3CA & {\EmojiFont 🏊} & {\textbackslash}:swimmer: & Swimmer \\ \hline
U+1F3E0 & {\EmojiFont 🏠} & {\textbackslash}:house: & House Building \\ \hline
U+1F3E1 & {\EmojiFont 🏡} & {\textbackslash}:house\_with\_garden: & House With Garden \\ \hline
U+1F3E2 & {\EmojiFont 🏢} & {\textbackslash}:office: & Office Building \\ \hline
U+1F3E3 & {\EmojiFont 🏣} & {\textbackslash}:post\_office: & Japanese Post Office \\ \hline
U+1F3E4 & {\EmojiFont 🏤} & {\textbackslash}:european\_post\_office: & European Post Office \\ \hline
U+1F3E5 & {\EmojiFont 🏥} & {\textbackslash}:hospital: & Hospital \\ \hline
U+1F3E6 & {\EmojiFont 🏦} & {\textbackslash}:bank: & Bank \\ \hline
U+1F3E7 & {\EmojiFont 🏧} & {\textbackslash}:atm: & Automated Teller Machine \\ \hline
U+1F3E8 & {\EmojiFont 🏨} & {\textbackslash}:hotel: & Hotel \\ \hline
U+1F3E9 & {\EmojiFont 🏩} & {\textbackslash}:love\_hotel: & Love Hotel \\ \hline
U+1F3EA & {\EmojiFont 🏪} & {\textbackslash}:convenience\_store: & Convenience Store \\ \hline
U+1F3EB & {\EmojiFont 🏫} & {\textbackslash}:school: & School \\ \hline
U+1F3EC & {\EmojiFont 🏬} & {\textbackslash}:department\_store: & Department Store \\ \hline
U+1F3ED & {\EmojiFont 🏭} & {\textbackslash}:factory: & Factory \\ \hline
U+1F3EE & {\EmojiFont 🏮} & {\textbackslash}:izakaya\_lantern: & Izakaya Lantern \\ \hline
U+1F3EF & {\EmojiFont 🏯} & {\textbackslash}:japanese\_castle: & Japanese Castle \\ \hline
U+1F3F0 & {\EmojiFont 🏰} & {\textbackslash}:european\_castle: & European Castle \\ \hline
U+1F3FB & {\EmojiFont 🏻} & {\textbackslash}:skin-tone-2: & Emoji Modifier Fitzpatrick Type-1-2 \\ \hline
U+1F3FC & {\EmojiFont 🏼} & {\textbackslash}:skin-tone-3: & Emoji Modifier Fitzpatrick Type-3 \\ \hline
U+1F3FD & {\EmojiFont 🏽} & {\textbackslash}:skin-tone-4: & Emoji Modifier Fitzpatrick Type-4 \\ \hline
U+1F3FE & {\EmojiFont 🏾} & {\textbackslash}:skin-tone-5: & Emoji Modifier Fitzpatrick Type-5 \\ \hline
U+1F3FF & {\EmojiFont 🏿} & {\textbackslash}:skin-tone-6: & Emoji Modifier Fitzpatrick Type-6 \\ \hline
U+1F400 & {\EmojiFont 🐀} & {\textbackslash}:rat: & Rat \\ \hline
U+1F401 & {\EmojiFont 🐁} & {\textbackslash}:mouse2: & Mouse \\ \hline
U+1F402 & {\EmojiFont 🐂} & {\textbackslash}:ox: & Ox \\ \hline
U+1F403 & {\EmojiFont 🐃} & {\textbackslash}:water\_buffalo: & Water Buffalo \\ \hline
U+1F404 & {\EmojiFont 🐄} & {\textbackslash}:cow2: & Cow \\ \hline
U+1F405 & {\EmojiFont 🐅} & {\textbackslash}:tiger2: & Tiger \\ \hline
U+1F406 & {\EmojiFont 🐆} & {\textbackslash}:leopard: & Leopard \\ \hline
U+1F407 & {\EmojiFont 🐇} & {\textbackslash}:rabbit2: & Rabbit \\ \hline
U+1F408 & {\EmojiFont 🐈} & {\textbackslash}:cat2: & Cat \\ \hline
U+1F409 & {\EmojiFont 🐉} & {\textbackslash}:dragon: & Dragon \\ \hline
U+1F40A & {\EmojiFont 🐊} & {\textbackslash}:crocodile: & Crocodile \\ \hline
U+1F40B & {\EmojiFont 🐋} & {\textbackslash}:whale2: & Whale \\ \hline
U+1F40C & {\EmojiFont 🐌} & {\textbackslash}:snail: & Snail \\ \hline
U+1F40D & {\EmojiFont 🐍} & {\textbackslash}:snake: & Snake \\ \hline
U+1F40E & {\EmojiFont 🐎} & {\textbackslash}:racehorse: & Horse \\ \hline
U+1F40F & {\EmojiFont 🐏} & {\textbackslash}:ram: & Ram \\ \hline
U+1F410 & {\EmojiFont 🐐} & {\textbackslash}:goat: & Goat \\ \hline
U+1F411 & {\EmojiFont 🐑} & {\textbackslash}:sheep: & Sheep \\ \hline
U+1F412 & {\EmojiFont 🐒} & {\textbackslash}:monkey: & Monkey \\ \hline
U+1F413 & {\EmojiFont 🐓} & {\textbackslash}:rooster: & Rooster \\ \hline
U+1F414 & {\EmojiFont 🐔} & {\textbackslash}:chicken: & Chicken \\ \hline
U+1F415 & {\EmojiFont 🐕} & {\textbackslash}:dog2: & Dog \\ \hline
U+1F416 & {\EmojiFont 🐖} & {\textbackslash}:pig2: & Pig \\ \hline
U+1F417 & {\EmojiFont 🐗} & {\textbackslash}:boar: & Boar \\ \hline
U+1F418 & {\EmojiFont 🐘} & {\textbackslash}:elephant: & Elephant \\ \hline
U+1F419 & {\EmojiFont 🐙} & {\textbackslash}:octopus: & Octopus \\ \hline
U+1F41A & {\EmojiFont 🐚} & {\textbackslash}:shell: & Spiral Shell \\ \hline
U+1F41B & {\EmojiFont 🐛} & {\textbackslash}:bug: & Bug \\ \hline
U+1F41C & {\EmojiFont 🐜} & {\textbackslash}:ant: & Ant \\ \hline
U+1F41D & {\EmojiFont 🐝} & {\textbackslash}:bee: & Honeybee \\ \hline
U+1F41E & {\EmojiFont 🐞} & {\textbackslash}:beetle: & Lady Beetle \\ \hline
U+1F41F & {\EmojiFont 🐟} & {\textbackslash}:fish: & Fish \\ \hline
U+1F420 & {\EmojiFont 🐠} & {\textbackslash}:tropical\_fish: & Tropical Fish \\ \hline
U+1F421 & {\EmojiFont 🐡} & {\textbackslash}:blowfish: & Blowfish \\ \hline
U+1F422 & {\EmojiFont 🐢} & {\textbackslash}:turtle: & Turtle \\ \hline
U+1F423 & {\EmojiFont 🐣} & {\textbackslash}:hatching\_chick: & Hatching Chick \\ \hline
U+1F424 & {\EmojiFont 🐤} & {\textbackslash}:baby\_chick: & Baby Chick \\ \hline
U+1F425 & {\EmojiFont 🐥} & {\textbackslash}:hatched\_chick: & Front-Facing Baby Chick \\ \hline
U+1F426 & {\EmojiFont 🐦} & {\textbackslash}:bird: & Bird \\ \hline
U+1F427 & {\EmojiFont 🐧} & {\textbackslash}:penguin: & Penguin \\ \hline
U+1F428 & {\EmojiFont 🐨} & {\textbackslash}:koala: & Koala \\ \hline
U+1F429 & {\EmojiFont 🐩} & {\textbackslash}:poodle: & Poodle \\ \hline
U+1F42A & {\EmojiFont 🐪} & {\textbackslash}:dromedary\_camel: & Dromedary Camel \\ \hline
U+1F42B & {\EmojiFont 🐫} & {\textbackslash}:camel: & Bactrian Camel \\ \hline
U+1F42C & {\EmojiFont 🐬} & {\textbackslash}:dolphin: & Dolphin \\ \hline
U+1F42D & {\EmojiFont 🐭} & {\textbackslash}:mouse: & Mouse Face \\ \hline
U+1F42E & {\EmojiFont 🐮} & {\textbackslash}:cow: & Cow Face \\ \hline
U+1F42F & {\EmojiFont 🐯} & {\textbackslash}:tiger: & Tiger Face \\ \hline
U+1F430 & {\EmojiFont 🐰} & {\textbackslash}:rabbit: & Rabbit Face \\ \hline
U+1F431 & {\EmojiFont 🐱} & {\textbackslash}:cat: & Cat Face \\ \hline
U+1F432 & {\EmojiFont 🐲} & {\textbackslash}:dragon\_face: & Dragon Face \\ \hline
U+1F433 & {\EmojiFont 🐳} & {\textbackslash}:whale: & Spouting Whale \\ \hline
U+1F434 & {\EmojiFont 🐴} & {\textbackslash}:horse: & Horse Face \\ \hline
U+1F435 & {\EmojiFont 🐵} & {\textbackslash}:monkey\_face: & Monkey Face \\ \hline
U+1F436 & {\EmojiFont 🐶} & {\textbackslash}:dog: & Dog Face \\ \hline
U+1F437 & {\EmojiFont 🐷} & {\textbackslash}:pig: & Pig Face \\ \hline
U+1F438 & {\EmojiFont 🐸} & {\textbackslash}:frog: & Frog Face \\ \hline
U+1F439 & {\EmojiFont 🐹} & {\textbackslash}:hamster: & Hamster Face \\ \hline
U+1F43A & {\EmojiFont 🐺} & {\textbackslash}:wolf: & Wolf Face \\ \hline
U+1F43B & {\EmojiFont 🐻} & {\textbackslash}:bear: & Bear Face \\ \hline
U+1F43C & {\EmojiFont 🐼} & {\textbackslash}:panda\_face: & Panda Face \\ \hline
U+1F43D & {\EmojiFont 🐽} & {\textbackslash}:pig\_nose: & Pig Nose \\ \hline
U+1F43E & {\EmojiFont 🐾} & {\textbackslash}:feet: & Paw Prints \\ \hline
U+1F440 & {\EmojiFont 👀} & {\textbackslash}:eyes: & Eyes \\ \hline
U+1F442 & {\EmojiFont 👂} & {\textbackslash}:ear: & Ear \\ \hline
U+1F443 & {\EmojiFont 👃} & {\textbackslash}:nose: & Nose \\ \hline
U+1F444 & {\EmojiFont 👄} & {\textbackslash}:lips: & Mouth \\ \hline
U+1F445 & {\EmojiFont 👅} & {\textbackslash}:tongue: & Tongue \\ \hline
U+1F446 & {\EmojiFont 👆} & {\textbackslash}:point\_up\_2: & White Up Pointing Backhand Index \\ \hline
U+1F447 & {\EmojiFont 👇} & {\textbackslash}:point\_down: & White Down Pointing Backhand Index \\ \hline
U+1F448 & {\EmojiFont 👈} & {\textbackslash}:point\_left: & White Left Pointing Backhand Index \\ \hline
U+1F449 & {\EmojiFont 👉} & {\textbackslash}:point\_right: & White Right Pointing Backhand Index \\ \hline
U+1F44A & {\EmojiFont 👊} & {\textbackslash}:facepunch: & Fisted Hand Sign \\ \hline
U+1F44B & {\EmojiFont 👋} & {\textbackslash}:wave: & Waving Hand Sign \\ \hline
U+1F44C & {\EmojiFont 👌} & {\textbackslash}:ok\_hand: & Ok Hand Sign \\ \hline
U+1F44D & {\EmojiFont 👍} & {\textbackslash}:+1: & Thumbs Up Sign \\ \hline
U+1F44E & {\EmojiFont 👎} & {\textbackslash}:-1: & Thumbs Down Sign \\ \hline
U+1F44F & {\EmojiFont 👏} & {\textbackslash}:clap: & Clapping Hands Sign \\ \hline
U+1F450 & {\EmojiFont 👐} & {\textbackslash}:open\_hands: & Open Hands Sign \\ \hline
U+1F451 & {\EmojiFont 👑} & {\textbackslash}:crown: & Crown \\ \hline
U+1F452 & {\EmojiFont 👒} & {\textbackslash}:womans\_hat: & Womans Hat \\ \hline
U+1F453 & {\EmojiFont 👓} & {\textbackslash}:eyeglasses: & Eyeglasses \\ \hline
U+1F454 & {\EmojiFont 👔} & {\textbackslash}:necktie: & Necktie \\ \hline
U+1F455 & {\EmojiFont 👕} & {\textbackslash}:shirt: & T-Shirt \\ \hline
U+1F456 & {\EmojiFont 👖} & {\textbackslash}:jeans: & Jeans \\ \hline
U+1F457 & {\EmojiFont 👗} & {\textbackslash}:dress: & Dress \\ \hline
U+1F458 & {\EmojiFont 👘} & {\textbackslash}:kimono: & Kimono \\ \hline
U+1F459 & {\EmojiFont 👙} & {\textbackslash}:bikini: & Bikini \\ \hline
U+1F45A & {\EmojiFont 👚} & {\textbackslash}:womans\_clothes: & Womans Clothes \\ \hline
U+1F45B & {\EmojiFont 👛} & {\textbackslash}:purse: & Purse \\ \hline
U+1F45C & {\EmojiFont 👜} & {\textbackslash}:handbag: & Handbag \\ \hline
U+1F45D & {\EmojiFont 👝} & {\textbackslash}:pouch: & Pouch \\ \hline
U+1F45E & {\EmojiFont 👞} & {\textbackslash}:mans\_shoe: & Mans Shoe \\ \hline
U+1F45F & {\EmojiFont 👟} & {\textbackslash}:athletic\_shoe: & Athletic Shoe \\ \hline
U+1F460 & {\EmojiFont 👠} & {\textbackslash}:high\_heel: & High-Heeled Shoe \\ \hline
U+1F461 & {\EmojiFont 👡} & {\textbackslash}:sandal: & Womans Sandal \\ \hline
U+1F462 & {\EmojiFont 👢} & {\textbackslash}:boot: & Womans Boots \\ \hline
U+1F463 & {\EmojiFont 👣} & {\textbackslash}:footprints: & Footprints \\ \hline
U+1F464 & {\EmojiFont 👤} & {\textbackslash}:bust\_in\_silhouette: & Bust In Silhouette \\ \hline
U+1F465 & {\EmojiFont 👥} & {\textbackslash}:busts\_in\_silhouette: & Busts In Silhouette \\ \hline
U+1F466 & {\EmojiFont 👦} & {\textbackslash}:boy: & Boy \\ \hline
U+1F467 & {\EmojiFont 👧} & {\textbackslash}:girl: & Girl \\ \hline
U+1F468 & {\EmojiFont 👨} & {\textbackslash}:man: & Man \\ \hline
U+1F469 & {\EmojiFont 👩} & {\textbackslash}:woman: & Woman \\ \hline
U+1F46A & {\EmojiFont 👪} & {\textbackslash}:family: & Family \\ \hline
U+1F46B & {\EmojiFont 👫} & {\textbackslash}:couple: & Man And Woman Holding Hands \\ \hline
U+1F46C & {\EmojiFont 👬} & {\textbackslash}:two\_men\_holding\_hands: & Two Men Holding Hands \\ \hline
U+1F46D & {\EmojiFont 👭} & {\textbackslash}:two\_women\_holding\_hands: & Two Women Holding Hands \\ \hline
U+1F46E & {\EmojiFont 👮} & {\textbackslash}:cop: & Police Officer \\ \hline
U+1F46F & {\EmojiFont 👯} & {\textbackslash}:dancers: & Woman With Bunny Ears \\ \hline
U+1F470 & {\EmojiFont 👰} & {\textbackslash}:bride\_with\_veil: & Bride With Veil \\ \hline
U+1F471 & {\EmojiFont 👱} & {\textbackslash}:person\_with\_blond\_hair: & Person With Blond Hair \\ \hline
U+1F472 & {\EmojiFont 👲} & {\textbackslash}:man\_with\_gua\_pi\_mao: & Man With Gua Pi Mao \\ \hline
U+1F473 & {\EmojiFont 👳} & {\textbackslash}:man\_with\_turban: & Man With Turban \\ \hline
U+1F474 & {\EmojiFont 👴} & {\textbackslash}:older\_man: & Older Man \\ \hline
U+1F475 & {\EmojiFont 👵} & {\textbackslash}:older\_woman: & Older Woman \\ \hline
U+1F476 & {\EmojiFont 👶} & {\textbackslash}:baby: & Baby \\ \hline
U+1F477 & {\EmojiFont 👷} & {\textbackslash}:construction\_worker: & Construction Worker \\ \hline
U+1F478 & {\EmojiFont 👸} & {\textbackslash}:princess: & Princess \\ \hline
U+1F479 & {\EmojiFont 👹} & {\textbackslash}:japanese\_ogre: & Japanese Ogre \\ \hline
U+1F47A & {\EmojiFont 👺} & {\textbackslash}:japanese\_goblin: & Japanese Goblin \\ \hline
U+1F47B & {\EmojiFont 👻} & {\textbackslash}:ghost: & Ghost \\ \hline
U+1F47C & {\EmojiFont 👼} & {\textbackslash}:angel: & Baby Angel \\ \hline
U+1F47D & {\EmojiFont 👽} & {\textbackslash}:alien: & Extraterrestrial Alien \\ \hline
U+1F47E & {\EmojiFont 👾} & {\textbackslash}:space\_invader: & Alien Monster \\ \hline
U+1F47F & {\EmojiFont 👿} & {\textbackslash}:imp: & Imp \\ \hline
U+1F480 & {\EmojiFont 💀} & {\textbackslash}:skull: & Skull \\ \hline
U+1F481 & {\EmojiFont 💁} & {\textbackslash}:information\_desk\_person: & Information Desk Person \\ \hline
U+1F482 & {\EmojiFont 💂} & {\textbackslash}:guardsman: & Guardsman \\ \hline
U+1F483 & {\EmojiFont 💃} & {\textbackslash}:dancer: & Dancer \\ \hline
U+1F484 & {\EmojiFont 💄} & {\textbackslash}:lipstick: & Lipstick \\ \hline
U+1F485 & {\EmojiFont 💅} & {\textbackslash}:nail\_care: & Nail Polish \\ \hline
U+1F486 & {\EmojiFont 💆} & {\textbackslash}:massage: & Face Massage \\ \hline
U+1F487 & {\EmojiFont 💇} & {\textbackslash}:haircut: & Haircut \\ \hline
U+1F488 & {\EmojiFont 💈} & {\textbackslash}:barber: & Barber Pole \\ \hline
U+1F489 & {\EmojiFont 💉} & {\textbackslash}:syringe: & Syringe \\ \hline
U+1F48A & {\EmojiFont 💊} & {\textbackslash}:pill: & Pill \\ \hline
U+1F48B & {\EmojiFont 💋} & {\textbackslash}:kiss: & Kiss Mark \\ \hline
U+1F48C & {\EmojiFont 💌} & {\textbackslash}:love\_letter: & Love Letter \\ \hline
U+1F48D & {\EmojiFont 💍} & {\textbackslash}:ring: & Ring \\ \hline
U+1F48E & {\EmojiFont 💎} & {\textbackslash}:gem: & Gem Stone \\ \hline
U+1F48F & {\EmojiFont 💏} & {\textbackslash}:couplekiss: & Kiss \\ \hline
U+1F490 & {\EmojiFont 💐} & {\textbackslash}:bouquet: & Bouquet \\ \hline
U+1F491 & {\EmojiFont 💑} & {\textbackslash}:couple\_with\_heart: & Couple With Heart \\ \hline
U+1F492 & {\EmojiFont 💒} & {\textbackslash}:wedding: & Wedding \\ \hline
U+1F493 & {\EmojiFont 💓} & {\textbackslash}:heartbeat: & Beating Heart \\ \hline
U+1F494 & {\EmojiFont 💔} & {\textbackslash}:broken\_heart: & Broken Heart \\ \hline
U+1F495 & {\EmojiFont 💕} & {\textbackslash}:two\_hearts: & Two Hearts \\ \hline
U+1F496 & {\EmojiFont 💖} & {\textbackslash}:sparkling\_heart: & Sparkling Heart \\ \hline
U+1F497 & {\EmojiFont 💗} & {\textbackslash}:heartpulse: & Growing Heart \\ \hline
U+1F498 & {\EmojiFont 💘} & {\textbackslash}:cupid: & Heart With Arrow \\ \hline
U+1F499 & {\EmojiFont 💙} & {\textbackslash}:blue\_heart: & Blue Heart \\ \hline
U+1F49A & {\EmojiFont 💚} & {\textbackslash}:green\_heart: & Green Heart \\ \hline
U+1F49B & {\EmojiFont 💛} & {\textbackslash}:yellow\_heart: & Yellow Heart \\ \hline
U+1F49C & {\EmojiFont 💜} & {\textbackslash}:purple\_heart: & Purple Heart \\ \hline
U+1F49D & {\EmojiFont 💝} & {\textbackslash}:gift\_heart: & Heart With Ribbon \\ \hline
U+1F49E & {\EmojiFont 💞} & {\textbackslash}:revolving\_hearts: & Revolving Hearts \\ \hline
U+1F49F & {\EmojiFont 💟} & {\textbackslash}:heart\_decoration: & Heart Decoration \\ \hline
U+1F4A0 & {\EmojiFont 💠} & {\textbackslash}:diamond\_shape\_with\_a\_dot\_inside: & Diamond Shape With A Dot Inside \\ \hline
U+1F4A1 & {\EmojiFont 💡} & {\textbackslash}:bulb: & Electric Light Bulb \\ \hline
U+1F4A2 & {\EmojiFont 💢} & {\textbackslash}:anger: & Anger Symbol \\ \hline
U+1F4A3 & {\EmojiFont 💣} & {\textbackslash}:bomb: & Bomb \\ \hline
U+1F4A4 & {\EmojiFont 💤} & {\textbackslash}:zzz: & Sleeping Symbol \\ \hline
U+1F4A5 & {\EmojiFont 💥} & {\textbackslash}:boom: & Collision Symbol \\ \hline
U+1F4A6 & {\EmojiFont 💦} & {\textbackslash}:sweat\_drops: & Splashing Sweat Symbol \\ \hline
U+1F4A7 & {\EmojiFont 💧} & {\textbackslash}:droplet: & Droplet \\ \hline
U+1F4A8 & {\EmojiFont 💨} & {\textbackslash}:dash: & Dash Symbol \\ \hline
U+1F4A9 & {\EmojiFont 💩} & {\textbackslash}:hankey: & Pile Of Poo \\ \hline
U+1F4AA & {\EmojiFont 💪} & {\textbackslash}:muscle: & Flexed Biceps \\ \hline
U+1F4AB & {\EmojiFont 💫} & {\textbackslash}:dizzy: & Dizzy Symbol \\ \hline
U+1F4AC & {\EmojiFont 💬} & {\textbackslash}:speech\_balloon: & Speech Balloon \\ \hline
U+1F4AD & {\EmojiFont 💭} & {\textbackslash}:thought\_balloon: & Thought Balloon \\ \hline
U+1F4AE & {\EmojiFont 💮} & {\textbackslash}:white\_flower: & White Flower \\ \hline
U+1F4AF & {\EmojiFont 💯} & {\textbackslash}:100: & Hundred Points Symbol \\ \hline
U+1F4B0 & {\EmojiFont 💰} & {\textbackslash}:moneybag: & Money Bag \\ \hline
U+1F4B1 & {\EmojiFont 💱} & {\textbackslash}:currency\_exchange: & Currency Exchange \\ \hline
U+1F4B2 & {\EmojiFont 💲} & {\textbackslash}:heavy\_dollar\_sign: & Heavy Dollar Sign \\ \hline
U+1F4B3 & {\EmojiFont 💳} & {\textbackslash}:credit\_card: & Credit Card \\ \hline
U+1F4B4 & {\EmojiFont 💴} & {\textbackslash}:yen: & Banknote With Yen Sign \\ \hline
U+1F4B5 & {\EmojiFont 💵} & {\textbackslash}:dollar: & Banknote With Dollar Sign \\ \hline
U+1F4B6 & {\EmojiFont 💶} & {\textbackslash}:euro: & Banknote With Euro Sign \\ \hline
U+1F4B7 & {\EmojiFont 💷} & {\textbackslash}:pound: & Banknote With Pound Sign \\ \hline
U+1F4B8 & {\EmojiFont 💸} & {\textbackslash}:money\_with\_wings: & Money With Wings \\ \hline
U+1F4B9 & {\EmojiFont 💹} & {\textbackslash}:chart: & Chart With Upwards Trend And Yen Sign \\ \hline
U+1F4BA & {\EmojiFont 💺} & {\textbackslash}:seat: & Seat \\ \hline
U+1F4BB & {\EmojiFont 💻} & {\textbackslash}:computer: & Personal Computer \\ \hline
U+1F4BC & {\EmojiFont 💼} & {\textbackslash}:briefcase: & Briefcase \\ \hline
U+1F4BD & {\EmojiFont 💽} & {\textbackslash}:minidisc: & Minidisc \\ \hline
U+1F4BE & {\EmojiFont 💾} & {\textbackslash}:floppy\_disk: & Floppy Disk \\ \hline
U+1F4BF & {\EmojiFont 💿} & {\textbackslash}:cd: & Optical Disc \\ \hline
U+1F4C0 & {\EmojiFont 📀} & {\textbackslash}:dvd: & Dvd \\ \hline
U+1F4C1 & {\EmojiFont 📁} & {\textbackslash}:file\_folder: & File Folder \\ \hline
U+1F4C2 & {\EmojiFont 📂} & {\textbackslash}:open\_file\_folder: & Open File Folder \\ \hline
U+1F4C3 & {\EmojiFont 📃} & {\textbackslash}:page\_with\_curl: & Page With Curl \\ \hline
U+1F4C4 & {\EmojiFont 📄} & {\textbackslash}:page\_facing\_up: & Page Facing Up \\ \hline
U+1F4C5 & {\EmojiFont 📅} & {\textbackslash}:date: & Calendar \\ \hline
U+1F4C6 & {\EmojiFont 📆} & {\textbackslash}:calendar: & Tear-Off Calendar \\ \hline
U+1F4C7 & {\EmojiFont 📇} & {\textbackslash}:card\_index: & Card Index \\ \hline
U+1F4C8 & {\EmojiFont 📈} & {\textbackslash}:chart\_with\_upwards\_trend: & Chart With Upwards Trend \\ \hline
U+1F4C9 & {\EmojiFont 📉} & {\textbackslash}:chart\_with\_downwards\_trend: & Chart With Downwards Trend \\ \hline
U+1F4CA & {\EmojiFont 📊} & {\textbackslash}:bar\_chart: & Bar Chart \\ \hline
U+1F4CB & {\EmojiFont 📋} & {\textbackslash}:clipboard: & Clipboard \\ \hline
U+1F4CC & {\EmojiFont 📌} & {\textbackslash}:pushpin: & Pushpin \\ \hline
U+1F4CD & {\EmojiFont 📍} & {\textbackslash}:round\_pushpin: & Round Pushpin \\ \hline
U+1F4CE & {\EmojiFont 📎} & {\textbackslash}:paperclip: & Paperclip \\ \hline
U+1F4CF & {\EmojiFont 📏} & {\textbackslash}:straight\_ruler: & Straight Ruler \\ \hline
U+1F4D0 & {\EmojiFont 📐} & {\textbackslash}:triangular\_ruler: & Triangular Ruler \\ \hline
U+1F4D1 & {\EmojiFont 📑} & {\textbackslash}:bookmark\_tabs: & Bookmark Tabs \\ \hline
U+1F4D2 & {\EmojiFont 📒} & {\textbackslash}:ledger: & Ledger \\ \hline
U+1F4D3 & {\EmojiFont 📓} & {\textbackslash}:notebook: & Notebook \\ \hline
U+1F4D4 & {\EmojiFont 📔} & {\textbackslash}:notebook\_with\_decorative\_cover: & Notebook With Decorative Cover \\ \hline
U+1F4D5 & {\EmojiFont 📕} & {\textbackslash}:closed\_book: & Closed Book \\ \hline
U+1F4D6 & {\EmojiFont 📖} & {\textbackslash}:book: & Open Book \\ \hline
U+1F4D7 & {\EmojiFont 📗} & {\textbackslash}:green\_book: & Green Book \\ \hline
U+1F4D8 & {\EmojiFont 📘} & {\textbackslash}:blue\_book: & Blue Book \\ \hline
U+1F4D9 & {\EmojiFont 📙} & {\textbackslash}:orange\_book: & Orange Book \\ \hline
U+1F4DA & {\EmojiFont 📚} & {\textbackslash}:books: & Books \\ \hline
U+1F4DB & {\EmojiFont 📛} & {\textbackslash}:name\_badge: & Name Badge \\ \hline
U+1F4DC & {\EmojiFont 📜} & {\textbackslash}:scroll: & Scroll \\ \hline
U+1F4DD & {\EmojiFont 📝} & {\textbackslash}:memo: & Memo \\ \hline
U+1F4DE & {\EmojiFont 📞} & {\textbackslash}:telephone\_receiver: & Telephone Receiver \\ \hline
U+1F4DF & {\EmojiFont 📟} & {\textbackslash}:pager: & Pager \\ \hline
U+1F4E0 & {\EmojiFont 📠} & {\textbackslash}:fax: & Fax Machine \\ \hline
U+1F4E1 & {\EmojiFont 📡} & {\textbackslash}:satellite: & Satellite Antenna \\ \hline
U+1F4E2 & {\EmojiFont 📢} & {\textbackslash}:loudspeaker: & Public Address Loudspeaker \\ \hline
U+1F4E3 & {\EmojiFont 📣} & {\textbackslash}:mega: & Cheering Megaphone \\ \hline
U+1F4E4 & {\EmojiFont 📤} & {\textbackslash}:outbox\_tray: & Outbox Tray \\ \hline
U+1F4E5 & {\EmojiFont 📥} & {\textbackslash}:inbox\_tray: & Inbox Tray \\ \hline
U+1F4E6 & {\EmojiFont 📦} & {\textbackslash}:package: & Package \\ \hline
U+1F4E7 & {\EmojiFont 📧} & {\textbackslash}:e-mail: & E-Mail Symbol \\ \hline
U+1F4E8 & {\EmojiFont 📨} & {\textbackslash}:incoming\_envelope: & Incoming Envelope \\ \hline
U+1F4E9 & {\EmojiFont 📩} & {\textbackslash}:envelope\_with\_arrow: & Envelope With Downwards Arrow Above \\ \hline
U+1F4EA & {\EmojiFont 📪} & {\textbackslash}:mailbox\_closed: & Closed Mailbox With Lowered Flag \\ \hline
U+1F4EB & {\EmojiFont 📫} & {\textbackslash}:mailbox: & Closed Mailbox With Raised Flag \\ \hline
U+1F4EC & {\EmojiFont 📬} & {\textbackslash}:mailbox\_with\_mail: & Open Mailbox With Raised Flag \\ \hline
U+1F4ED & {\EmojiFont 📭} & {\textbackslash}:mailbox\_with\_no\_mail: & Open Mailbox With Lowered Flag \\ \hline
U+1F4EE & {\EmojiFont 📮} & {\textbackslash}:postbox: & Postbox \\ \hline
U+1F4EF & {\EmojiFont 📯} & {\textbackslash}:postal\_horn: & Postal Horn \\ \hline
U+1F4F0 & {\EmojiFont 📰} & {\textbackslash}:newspaper: & Newspaper \\ \hline
U+1F4F1 & {\EmojiFont 📱} & {\textbackslash}:iphone: & Mobile Phone \\ \hline
U+1F4F2 & {\EmojiFont 📲} & {\textbackslash}:calling: & Mobile Phone With Rightwards Arrow At Left \\ \hline
U+1F4F3 & {\EmojiFont 📳} & {\textbackslash}:vibration\_mode: & Vibration Mode \\ \hline
U+1F4F4 & {\EmojiFont 📴} & {\textbackslash}:mobile\_phone\_off: & Mobile Phone Off \\ \hline
U+1F4F5 & {\EmojiFont 📵} & {\textbackslash}:no\_mobile\_phones: & No Mobile Phones \\ \hline
U+1F4F6 & {\EmojiFont 📶} & {\textbackslash}:signal\_strength: & Antenna With Bars \\ \hline
U+1F4F7 & {\EmojiFont 📷} & {\textbackslash}:camera: & Camera \\ \hline
U+1F4F9 & {\EmojiFont 📹} & {\textbackslash}:video\_camera: & Video Camera \\ \hline
U+1F4FA & {\EmojiFont 📺} & {\textbackslash}:tv: & Television \\ \hline
U+1F4FB & {\EmojiFont 📻} & {\textbackslash}:radio: & Radio \\ \hline
U+1F4FC & {\EmojiFont 📼} & {\textbackslash}:vhs: & Videocassette \\ \hline
U+1F500 & {\EmojiFont 🔀} & {\textbackslash}:twisted\_rightwards\_arrows: & Twisted Rightwards Arrows \\ \hline
U+1F501 & {\EmojiFont 🔁} & {\textbackslash}:repeat: & Clockwise Rightwards And Leftwards Open Circle Arrows \\ \hline
U+1F502 & {\EmojiFont 🔂} & {\textbackslash}:repeat\_one: & Clockwise Rightwards And Leftwards Open Circle Arrows With Circled One Overlay \\ \hline
U+1F503 & {\EmojiFont 🔃} & {\textbackslash}:arrows\_clockwise: & Clockwise Downwards And Upwards Open Circle Arrows \\ \hline
U+1F504 & {\EmojiFont 🔄} & {\textbackslash}:arrows\_counterclockwise: & Anticlockwise Downwards And Upwards Open Circle Arrows \\ \hline
U+1F505 & {\EmojiFont 🔅} & {\textbackslash}:low\_brightness: & Low Brightness Symbol \\ \hline
U+1F506 & {\EmojiFont 🔆} & {\textbackslash}:high\_brightness: & High Brightness Symbol \\ \hline
U+1F507 & {\EmojiFont 🔇} & {\textbackslash}:mute: & Speaker With Cancellation Stroke \\ \hline
U+1F508 & {\EmojiFont 🔈} & {\textbackslash}:speaker: & Speaker \\ \hline
U+1F509 & {\EmojiFont 🔉} & {\textbackslash}:sound: & Speaker With One Sound Wave \\ \hline
U+1F50A & {\EmojiFont 🔊} & {\textbackslash}:loud\_sound: & Speaker With Three Sound Waves \\ \hline
U+1F50B & {\EmojiFont 🔋} & {\textbackslash}:battery: & Battery \\ \hline
U+1F50C & {\EmojiFont 🔌} & {\textbackslash}:electric\_plug: & Electric Plug \\ \hline
U+1F50D & {\EmojiFont 🔍} & {\textbackslash}:mag: & Left-Pointing Magnifying Glass \\ \hline
U+1F50E & {\EmojiFont 🔎} & {\textbackslash}:mag\_right: & Right-Pointing Magnifying Glass \\ \hline
U+1F50F & {\EmojiFont 🔏} & {\textbackslash}:lock\_with\_ink\_pen: & Lock With Ink Pen \\ \hline
U+1F510 & {\EmojiFont 🔐} & {\textbackslash}:closed\_lock\_with\_key: & Closed Lock With Key \\ \hline
U+1F511 & {\EmojiFont 🔑} & {\textbackslash}:key: & Key \\ \hline
U+1F512 & {\EmojiFont 🔒} & {\textbackslash}:lock: & Lock \\ \hline
U+1F513 & {\EmojiFont 🔓} & {\textbackslash}:unlock: & Open Lock \\ \hline
U+1F514 & {\EmojiFont 🔔} & {\textbackslash}:bell: & Bell \\ \hline
U+1F515 & {\EmojiFont 🔕} & {\textbackslash}:no\_bell: & Bell With Cancellation Stroke \\ \hline
U+1F516 & {\EmojiFont 🔖} & {\textbackslash}:bookmark: & Bookmark \\ \hline
U+1F517 & {\EmojiFont 🔗} & {\textbackslash}:link: & Link Symbol \\ \hline
U+1F518 & {\EmojiFont 🔘} & {\textbackslash}:radio\_button: & Radio Button \\ \hline
U+1F519 & {\EmojiFont 🔙} & {\textbackslash}:back: & Back With Leftwards Arrow Above \\ \hline
U+1F51A & {\EmojiFont 🔚} & {\textbackslash}:end: & End With Leftwards Arrow Above \\ \hline
U+1F51B & {\EmojiFont 🔛} & {\textbackslash}:on: & On With Exclamation Mark With Left Right Arrow Above \\ \hline
U+1F51C & {\EmojiFont 🔜} & {\textbackslash}:soon: & Soon With Rightwards Arrow Above \\ \hline
U+1F51D & {\EmojiFont 🔝} & {\textbackslash}:top: & Top With Upwards Arrow Above \\ \hline
U+1F51E & {\EmojiFont 🔞} & {\textbackslash}:underage: & No One Under Eighteen Symbol \\ \hline
U+1F51F & {\EmojiFont 🔟} & {\textbackslash}:keycap\_ten: & Keycap Ten \\ \hline
U+1F520 & {\EmojiFont 🔠} & {\textbackslash}:capital\_abcd: & Input Symbol For Latin Capital Letters \\ \hline
U+1F521 & {\EmojiFont 🔡} & {\textbackslash}:abcd: & Input Symbol For Latin Small Letters \\ \hline
U+1F522 & {\EmojiFont 🔢} & {\textbackslash}:1234: & Input Symbol For Numbers \\ \hline
U+1F523 & {\EmojiFont 🔣} & {\textbackslash}:symbols: & Input Symbol For Symbols \\ \hline
U+1F524 & {\EmojiFont 🔤} & {\textbackslash}:abc: & Input Symbol For Latin Letters \\ \hline
U+1F525 & {\EmojiFont 🔥} & {\textbackslash}:fire: & Fire \\ \hline
U+1F526 & {\EmojiFont 🔦} & {\textbackslash}:flashlight: & Electric Torch \\ \hline
U+1F527 & {\EmojiFont 🔧} & {\textbackslash}:wrench: & Wrench \\ \hline
U+1F528 & {\EmojiFont 🔨} & {\textbackslash}:hammer: & Hammer \\ \hline
U+1F529 & {\EmojiFont 🔩} & {\textbackslash}:nut\_and\_bolt: & Nut And Bolt \\ \hline
U+1F52A & {\EmojiFont 🔪} & {\textbackslash}:hocho: & Hocho \\ \hline
U+1F52B & {\EmojiFont 🔫} & {\textbackslash}:gun: & Pistol \\ \hline
U+1F52C & {\EmojiFont 🔬} & {\textbackslash}:microscope: & Microscope \\ \hline
U+1F52D & {\EmojiFont 🔭} & {\textbackslash}:telescope: & Telescope \\ \hline
U+1F52E & {\EmojiFont 🔮} & {\textbackslash}:crystal\_ball: & Crystal Ball \\ \hline
U+1F52F & {\EmojiFont 🔯} & {\textbackslash}:six\_pointed\_star: & Six Pointed Star With Middle Dot \\ \hline
U+1F530 & {\EmojiFont 🔰} & {\textbackslash}:beginner: & Japanese Symbol For Beginner \\ \hline
U+1F531 & {\EmojiFont 🔱} & {\textbackslash}:trident: & Trident Emblem \\ \hline
U+1F532 & {\EmojiFont 🔲} & {\textbackslash}:black\_square\_button: & Black Square Button \\ \hline
U+1F533 & {\EmojiFont 🔳} & {\textbackslash}:white\_square\_button: & White Square Button \\ \hline
U+1F534 & {\EmojiFont 🔴} & {\textbackslash}:red\_circle: & Large Red Circle \\ \hline
U+1F535 & {\EmojiFont 🔵} & {\textbackslash}:large\_blue\_circle: & Large Blue Circle \\ \hline
U+1F536 & {\EmojiFont 🔶} & {\textbackslash}:large\_orange\_diamond: & Large Orange Diamond \\ \hline
U+1F537 & {\EmojiFont 🔷} & {\textbackslash}:large\_blue\_diamond: & Large Blue Diamond \\ \hline
U+1F538 & {\EmojiFont 🔸} & {\textbackslash}:small\_orange\_diamond: & Small Orange Diamond \\ \hline
U+1F539 & {\EmojiFont 🔹} & {\textbackslash}:small\_blue\_diamond: & Small Blue Diamond \\ \hline
U+1F53A & {\EmojiFont 🔺} & {\textbackslash}:small\_red\_triangle: & Up-Pointing Red Triangle \\ \hline
U+1F53B & {\EmojiFont 🔻} & {\textbackslash}:small\_red\_triangle\_down: & Down-Pointing Red Triangle \\ \hline
U+1F53C & {\EmojiFont 🔼} & {\textbackslash}:arrow\_up\_small: & Up-Pointing Small Red Triangle \\ \hline
U+1F53D & {\EmojiFont 🔽} & {\textbackslash}:arrow\_down\_small: & Down-Pointing Small Red Triangle \\ \hline
U+1F550 & {\EmojiFont 🕐} & {\textbackslash}:clock1: & Clock Face One Oclock \\ \hline
U+1F551 & {\EmojiFont 🕑} & {\textbackslash}:clock2: & Clock Face Two Oclock \\ \hline
U+1F552 & {\EmojiFont 🕒} & {\textbackslash}:clock3: & Clock Face Three Oclock \\ \hline
U+1F553 & {\EmojiFont 🕓} & {\textbackslash}:clock4: & Clock Face Four Oclock \\ \hline
U+1F554 & {\EmojiFont 🕔} & {\textbackslash}:clock5: & Clock Face Five Oclock \\ \hline
U+1F555 & {\EmojiFont 🕕} & {\textbackslash}:clock6: & Clock Face Six Oclock \\ \hline
U+1F556 & {\EmojiFont 🕖} & {\textbackslash}:clock7: & Clock Face Seven Oclock \\ \hline
U+1F557 & {\EmojiFont 🕗} & {\textbackslash}:clock8: & Clock Face Eight Oclock \\ \hline
U+1F558 & {\EmojiFont 🕘} & {\textbackslash}:clock9: & Clock Face Nine Oclock \\ \hline
U+1F559 & {\EmojiFont 🕙} & {\textbackslash}:clock10: & Clock Face Ten Oclock \\ \hline
U+1F55A & {\EmojiFont 🕚} & {\textbackslash}:clock11: & Clock Face Eleven Oclock \\ \hline
U+1F55B & {\EmojiFont 🕛} & {\textbackslash}:clock12: & Clock Face Twelve Oclock \\ \hline
U+1F55C & {\EmojiFont 🕜} & {\textbackslash}:clock130: & Clock Face One-Thirty \\ \hline
U+1F55D & {\EmojiFont 🕝} & {\textbackslash}:clock230: & Clock Face Two-Thirty \\ \hline
U+1F55E & {\EmojiFont 🕞} & {\textbackslash}:clock330: & Clock Face Three-Thirty \\ \hline
U+1F55F & {\EmojiFont 🕟} & {\textbackslash}:clock430: & Clock Face Four-Thirty \\ \hline
U+1F560 & {\EmojiFont 🕠} & {\textbackslash}:clock530: & Clock Face Five-Thirty \\ \hline
U+1F561 & {\EmojiFont 🕡} & {\textbackslash}:clock630: & Clock Face Six-Thirty \\ \hline
U+1F562 & {\EmojiFont 🕢} & {\textbackslash}:clock730: & Clock Face Seven-Thirty \\ \hline
U+1F563 & {\EmojiFont 🕣} & {\textbackslash}:clock830: & Clock Face Eight-Thirty \\ \hline
U+1F564 & {\EmojiFont 🕤} & {\textbackslash}:clock930: & Clock Face Nine-Thirty \\ \hline
U+1F565 & {\EmojiFont 🕥} & {\textbackslash}:clock1030: & Clock Face Ten-Thirty \\ \hline
U+1F566 & {\EmojiFont 🕦} & {\textbackslash}:clock1130: & Clock Face Eleven-Thirty \\ \hline
U+1F567 & {\EmojiFont 🕧} & {\textbackslash}:clock1230: & Clock Face Twelve-Thirty \\ \hline
U+1F5FB & {\EmojiFont 🗻} & {\textbackslash}:mount\_fuji: & Mount Fuji \\ \hline
U+1F5FC & {\EmojiFont 🗼} & {\textbackslash}:tokyo\_tower: & Tokyo Tower \\ \hline
U+1F5FD & {\EmojiFont 🗽} & {\textbackslash}:statue\_of\_liberty: & Statue Of Liberty \\ \hline
U+1F5FE & {\EmojiFont 🗾} & {\textbackslash}:japan: & Silhouette Of Japan \\ \hline
U+1F5FF & {\EmojiFont 🗿} & {\textbackslash}:moyai: & Moyai \\ \hline
U+1F600 & {\EmojiFont 😀} & {\textbackslash}:grinning: & Grinning Face \\ \hline
U+1F601 & {\EmojiFont 😁} & {\textbackslash}:grin: & Grinning Face With Smiling Eyes \\ \hline
U+1F602 & {\EmojiFont 😂} & {\textbackslash}:joy: & Face With Tears Of Joy \\ \hline
U+1F603 & {\EmojiFont 😃} & {\textbackslash}:smiley: & Smiling Face With Open Mouth \\ \hline
U+1F604 & {\EmojiFont 😄} & {\textbackslash}:smile: & Smiling Face With Open Mouth And Smiling Eyes \\ \hline
U+1F605 & {\EmojiFont 😅} & {\textbackslash}:sweat\_smile: & Smiling Face With Open Mouth And Cold Sweat \\ \hline
U+1F606 & {\EmojiFont 😆} & {\textbackslash}:laughing: & Smiling Face With Open Mouth And Tightly-Closed Eyes \\ \hline
U+1F607 & {\EmojiFont 😇} & {\textbackslash}:innocent: & Smiling Face With Halo \\ \hline
U+1F608 & {\EmojiFont 😈} & {\textbackslash}:smiling\_imp: & Smiling Face With Horns \\ \hline
U+1F609 & {\EmojiFont 😉} & {\textbackslash}:wink: & Winking Face \\ \hline
U+1F60A & {\EmojiFont 😊} & {\textbackslash}:blush: & Smiling Face With Smiling Eyes \\ \hline
U+1F60B & {\EmojiFont 😋} & {\textbackslash}:yum: & Face Savouring Delicious Food \\ \hline
U+1F60C & {\EmojiFont 😌} & {\textbackslash}:relieved: & Relieved Face \\ \hline
U+1F60D & {\EmojiFont 😍} & {\textbackslash}:heart\_eyes: & Smiling Face With Heart-Shaped Eyes \\ \hline
U+1F60E & {\EmojiFont 😎} & {\textbackslash}:sunglasses: & Smiling Face With Sunglasses \\ \hline
U+1F60F & {\EmojiFont 😏} & {\textbackslash}:smirk: & Smirking Face \\ \hline
U+1F610 & {\EmojiFont 😐} & {\textbackslash}:neutral\_face: & Neutral Face \\ \hline
U+1F611 & {\EmojiFont 😑} & {\textbackslash}:expressionless: & Expressionless Face \\ \hline
U+1F612 & {\EmojiFont 😒} & {\textbackslash}:unamused: & Unamused Face \\ \hline
U+1F613 & {\EmojiFont 😓} & {\textbackslash}:sweat: & Face With Cold Sweat \\ \hline
U+1F614 & {\EmojiFont 😔} & {\textbackslash}:pensive: & Pensive Face \\ \hline
U+1F615 & {\EmojiFont 😕} & {\textbackslash}:confused: & Confused Face \\ \hline
U+1F616 & {\EmojiFont 😖} & {\textbackslash}:confounded: & Confounded Face \\ \hline
U+1F617 & {\EmojiFont 😗} & {\textbackslash}:kissing: & Kissing Face \\ \hline
U+1F618 & {\EmojiFont 😘} & {\textbackslash}:kissing\_heart: & Face Throwing A Kiss \\ \hline
U+1F619 & {\EmojiFont 😙} & {\textbackslash}:kissing\_smiling\_eyes: & Kissing Face With Smiling Eyes \\ \hline
U+1F61A & {\EmojiFont 😚} & {\textbackslash}:kissing\_closed\_eyes: & Kissing Face With Closed Eyes \\ \hline
U+1F61B & {\EmojiFont 😛} & {\textbackslash}:stuck\_out\_tongue: & Face With Stuck-Out Tongue \\ \hline
U+1F61C & {\EmojiFont 😜} & {\textbackslash}:stuck\_out\_tongue\_winking\_eye: & Face With Stuck-Out Tongue And Winking Eye \\ \hline
U+1F61D & {\EmojiFont 😝} & {\textbackslash}:stuck\_out\_tongue\_closed\_eyes: & Face With Stuck-Out Tongue And Tightly-Closed Eyes \\ \hline
U+1F61E & {\EmojiFont 😞} & {\textbackslash}:disappointed: & Disappointed Face \\ \hline
U+1F61F & {\EmojiFont 😟} & {\textbackslash}:worried: & Worried Face \\ \hline
U+1F620 & {\EmojiFont 😠} & {\textbackslash}:angry: & Angry Face \\ \hline
U+1F621 & {\EmojiFont 😡} & {\textbackslash}:rage: & Pouting Face \\ \hline
U+1F622 & {\EmojiFont 😢} & {\textbackslash}:cry: & Crying Face \\ \hline
U+1F623 & {\EmojiFont 😣} & {\textbackslash}:persevere: & Persevering Face \\ \hline
U+1F624 & {\EmojiFont 😤} & {\textbackslash}:triumph: & Face With Look Of Triumph \\ \hline
U+1F625 & {\EmojiFont 😥} & {\textbackslash}:disappointed\_relieved: & Disappointed But Relieved Face \\ \hline
U+1F626 & {\EmojiFont 😦} & {\textbackslash}:frowning: & Frowning Face With Open Mouth \\ \hline
U+1F627 & {\EmojiFont 😧} & {\textbackslash}:anguished: & Anguished Face \\ \hline
U+1F628 & {\EmojiFont 😨} & {\textbackslash}:fearful: & Fearful Face \\ \hline
U+1F629 & {\EmojiFont 😩} & {\textbackslash}:weary: & Weary Face \\ \hline
U+1F62A & {\EmojiFont 😪} & {\textbackslash}:sleepy: & Sleepy Face \\ \hline
U+1F62B & {\EmojiFont 😫} & {\textbackslash}:tired\_face: & Tired Face \\ \hline
U+1F62C & {\EmojiFont 😬} & {\textbackslash}:grimacing: & Grimacing Face \\ \hline
U+1F62D & {\EmojiFont 😭} & {\textbackslash}:sob: & Loudly Crying Face \\ \hline
U+1F62E & {\EmojiFont 😮} & {\textbackslash}:open\_mouth: & Face With Open Mouth \\ \hline
U+1F62F & {\EmojiFont 😯} & {\textbackslash}:hushed: & Hushed Face \\ \hline
U+1F630 & {\EmojiFont 😰} & {\textbackslash}:cold\_sweat: & Face With Open Mouth And Cold Sweat \\ \hline
U+1F631 & {\EmojiFont 😱} & {\textbackslash}:scream: & Face Screaming In Fear \\ \hline
U+1F632 & {\EmojiFont 😲} & {\textbackslash}:astonished: & Astonished Face \\ \hline
U+1F633 & {\EmojiFont 😳} & {\textbackslash}:flushed: & Flushed Face \\ \hline
U+1F634 & {\EmojiFont 😴} & {\textbackslash}:sleeping: & Sleeping Face \\ \hline
U+1F635 & {\EmojiFont 😵} & {\textbackslash}:dizzy\_face: & Dizzy Face \\ \hline
U+1F636 & {\EmojiFont 😶} & {\textbackslash}:no\_mouth: & Face Without Mouth \\ \hline
U+1F637 & {\EmojiFont 😷} & {\textbackslash}:mask: & Face With Medical Mask \\ \hline
U+1F638 & {\EmojiFont 😸} & {\textbackslash}:smile\_cat: & Grinning Cat Face With Smiling Eyes \\ \hline
U+1F639 & {\EmojiFont 😹} & {\textbackslash}:joy\_cat: & Cat Face With Tears Of Joy \\ \hline
U+1F63A & {\EmojiFont 😺} & {\textbackslash}:smiley\_cat: & Smiling Cat Face With Open Mouth \\ \hline
U+1F63B & {\EmojiFont 😻} & {\textbackslash}:heart\_eyes\_cat: & Smiling Cat Face With Heart-Shaped Eyes \\ \hline
U+1F63C & {\EmojiFont 😼} & {\textbackslash}:smirk\_cat: & Cat Face With Wry Smile \\ \hline
U+1F63D & {\EmojiFont 😽} & {\textbackslash}:kissing\_cat: & Kissing Cat Face With Closed Eyes \\ \hline
U+1F63E & {\EmojiFont 😾} & {\textbackslash}:pouting\_cat: & Pouting Cat Face \\ \hline
U+1F63F & {\EmojiFont 😿} & {\textbackslash}:crying\_cat\_face: & Crying Cat Face \\ \hline
U+1F640 & {\EmojiFont 🙀} & {\textbackslash}:scream\_cat: & Weary Cat Face \\ \hline
U+1F645 & {\EmojiFont 🙅} & {\textbackslash}:no\_good: & Face With No Good Gesture \\ \hline
U+1F646 & {\EmojiFont 🙆} & {\textbackslash}:ok\_woman: & Face With Ok Gesture \\ \hline
U+1F647 & {\EmojiFont 🙇} & {\textbackslash}:bow: & Person Bowing Deeply \\ \hline
U+1F648 & {\EmojiFont 🙈} & {\textbackslash}:see\_no\_evil: & See-No-Evil Monkey \\ \hline
U+1F649 & {\EmojiFont 🙉} & {\textbackslash}:hear\_no\_evil: & Hear-No-Evil Monkey \\ \hline
U+1F64A & {\EmojiFont 🙊} & {\textbackslash}:speak\_no\_evil: & Speak-No-Evil Monkey \\ \hline
U+1F64B & {\EmojiFont 🙋} & {\textbackslash}:raising\_hand: & Happy Person Raising One Hand \\ \hline
U+1F64C & {\EmojiFont 🙌} & {\textbackslash}:raised\_hands: & Person Raising Both Hands In Celebration \\ \hline
U+1F64D & {\EmojiFont 🙍} & {\textbackslash}:person\_frowning: & Person Frowning \\ \hline
U+1F64E & {\EmojiFont 🙎} & {\textbackslash}:person\_with\_pouting\_face: & Person With Pouting Face \\ \hline
U+1F64F & {\EmojiFont 🙏} & {\textbackslash}:pray: & Person With Folded Hands \\ \hline
U+1F680 & {\EmojiFont 🚀} & {\textbackslash}:rocket: & Rocket \\ \hline
U+1F681 & {\EmojiFont 🚁} & {\textbackslash}:helicopter: & Helicopter \\ \hline
U+1F682 & {\EmojiFont 🚂} & {\textbackslash}:steam\_locomotive: & Steam Locomotive \\ \hline
U+1F683 & {\EmojiFont 🚃} & {\textbackslash}:railway\_car: & Railway Car \\ \hline
U+1F684 & {\EmojiFont 🚄} & {\textbackslash}:bullettrain\_side: & High-Speed Train \\ \hline
U+1F685 & {\EmojiFont 🚅} & {\textbackslash}:bullettrain\_front: & High-Speed Train With Bullet Nose \\ \hline
U+1F686 & {\EmojiFont 🚆} & {\textbackslash}:train2: & Train \\ \hline
U+1F687 & {\EmojiFont 🚇} & {\textbackslash}:metro: & Metro \\ \hline
U+1F688 & {\EmojiFont 🚈} & {\textbackslash}:light\_rail: & Light Rail \\ \hline
U+1F689 & {\EmojiFont 🚉} & {\textbackslash}:station: & Station \\ \hline
U+1F68A & {\EmojiFont 🚊} & {\textbackslash}:tram: & Tram \\ \hline
U+1F68B & {\EmojiFont 🚋} & {\textbackslash}:train: & Tram Car \\ \hline
U+1F68C & {\EmojiFont 🚌} & {\textbackslash}:bus: & Bus \\ \hline
U+1F68D & {\EmojiFont 🚍} & {\textbackslash}:oncoming\_bus: & Oncoming Bus \\ \hline
U+1F68E & {\EmojiFont 🚎} & {\textbackslash}:trolleybus: & Trolleybus \\ \hline
U+1F68F & {\EmojiFont 🚏} & {\textbackslash}:busstop: & Bus Stop \\ \hline
U+1F690 & {\EmojiFont 🚐} & {\textbackslash}:minibus: & Minibus \\ \hline
U+1F691 & {\EmojiFont 🚑} & {\textbackslash}:ambulance: & Ambulance \\ \hline
U+1F692 & {\EmojiFont 🚒} & {\textbackslash}:fire\_engine: & Fire Engine \\ \hline
U+1F693 & {\EmojiFont 🚓} & {\textbackslash}:police\_car: & Police Car \\ \hline
U+1F694 & {\EmojiFont 🚔} & {\textbackslash}:oncoming\_police\_car: & Oncoming Police Car \\ \hline
U+1F695 & {\EmojiFont 🚕} & {\textbackslash}:taxi: & Taxi \\ \hline
U+1F696 & {\EmojiFont 🚖} & {\textbackslash}:oncoming\_taxi: & Oncoming Taxi \\ \hline
U+1F697 & {\EmojiFont 🚗} & {\textbackslash}:car: & Automobile \\ \hline
U+1F698 & {\EmojiFont 🚘} & {\textbackslash}:oncoming\_automobile: & Oncoming Automobile \\ \hline
U+1F699 & {\EmojiFont 🚙} & {\textbackslash}:blue\_car: & Recreational Vehicle \\ \hline
U+1F69A & {\EmojiFont 🚚} & {\textbackslash}:truck: & Delivery Truck \\ \hline
U+1F69B & {\EmojiFont 🚛} & {\textbackslash}:articulated\_lorry: & Articulated Lorry \\ \hline
U+1F69C & {\EmojiFont 🚜} & {\textbackslash}:tractor: & Tractor \\ \hline
U+1F69D & {\EmojiFont 🚝} & {\textbackslash}:monorail: & Monorail \\ \hline
U+1F69E & {\EmojiFont 🚞} & {\textbackslash}:mountain\_railway: & Mountain Railway \\ \hline
U+1F69F & {\EmojiFont 🚟} & {\textbackslash}:suspension\_railway: & Suspension Railway \\ \hline
U+1F6A0 & {\EmojiFont 🚠} & {\textbackslash}:mountain\_cableway: & Mountain Cableway \\ \hline
U+1F6A1 & {\EmojiFont 🚡} & {\textbackslash}:aerial\_tramway: & Aerial Tramway \\ \hline
U+1F6A2 & {\EmojiFont 🚢} & {\textbackslash}:ship: & Ship \\ \hline
U+1F6A3 & {\EmojiFont 🚣} & {\textbackslash}:rowboat: & Rowboat \\ \hline
U+1F6A4 & {\EmojiFont 🚤} & {\textbackslash}:speedboat: & Speedboat \\ \hline
U+1F6A5 & {\EmojiFont 🚥} & {\textbackslash}:traffic\_light: & Horizontal Traffic Light \\ \hline
U+1F6A6 & {\EmojiFont 🚦} & {\textbackslash}:vertical\_traffic\_light: & Vertical Traffic Light \\ \hline
U+1F6A7 & {\EmojiFont 🚧} & {\textbackslash}:construction: & Construction Sign \\ \hline
U+1F6A8 & {\EmojiFont 🚨} & {\textbackslash}:rotating\_light: & Police Cars Revolving Light \\ \hline
U+1F6A9 & {\EmojiFont 🚩} & {\textbackslash}:triangular\_flag\_on\_post: & Triangular Flag On Post \\ \hline
U+1F6AA & {\EmojiFont 🚪} & {\textbackslash}:door: & Door \\ \hline
U+1F6AB & {\EmojiFont 🚫} & {\textbackslash}:no\_entry\_sign: & No Entry Sign \\ \hline
U+1F6AC & {\EmojiFont 🚬} & {\textbackslash}:smoking: & Smoking Symbol \\ \hline
U+1F6AD & {\EmojiFont 🚭} & {\textbackslash}:no\_smoking: & No Smoking Symbol \\ \hline
U+1F6AE & {\EmojiFont 🚮} & {\textbackslash}:put\_litter\_in\_its\_place: & Put Litter In Its Place Symbol \\ \hline
U+1F6AF & {\EmojiFont 🚯} & {\textbackslash}:do\_not\_litter: & Do Not Litter Symbol \\ \hline
U+1F6B0 & {\EmojiFont 🚰} & {\textbackslash}:potable\_water: & Potable Water Symbol \\ \hline
U+1F6B1 & {\EmojiFont 🚱} & {\textbackslash}:non-potable\_water: & Non-Potable Water Symbol \\ \hline
U+1F6B2 & {\EmojiFont 🚲} & {\textbackslash}:bike: & Bicycle \\ \hline
U+1F6B3 & {\EmojiFont 🚳} & {\textbackslash}:no\_bicycles: & No Bicycles \\ \hline
U+1F6B4 & {\EmojiFont 🚴} & {\textbackslash}:bicyclist: & Bicyclist \\ \hline
U+1F6B5 & {\EmojiFont 🚵} & {\textbackslash}:mountain\_bicyclist: & Mountain Bicyclist \\ \hline
U+1F6B6 & {\EmojiFont 🚶} & {\textbackslash}:walking: & Pedestrian \\ \hline
U+1F6B7 & {\EmojiFont 🚷} & {\textbackslash}:no\_pedestrians: & No Pedestrians \\ \hline
U+1F6B8 & {\EmojiFont 🚸} & {\textbackslash}:children\_crossing: & Children Crossing \\ \hline
U+1F6B9 & {\EmojiFont 🚹} & {\textbackslash}:mens: & Mens Symbol \\ \hline
U+1F6BA & {\EmojiFont 🚺} & {\textbackslash}:womens: & Womens Symbol \\ \hline
U+1F6BB & {\EmojiFont 🚻} & {\textbackslash}:restroom: & Restroom \\ \hline
U+1F6BC & {\EmojiFont 🚼} & {\textbackslash}:baby\_symbol: & Baby Symbol \\ \hline
U+1F6BD & {\EmojiFont 🚽} & {\textbackslash}:toilet: & Toilet \\ \hline
U+1F6BE & {\EmojiFont 🚾} & {\textbackslash}:wc: & Water Closet \\ \hline
U+1F6BF & {\EmojiFont 🚿} & {\textbackslash}:shower: & Shower \\ \hline
U+1F6C0 & {\EmojiFont 🛀} & {\textbackslash}:bath: & Bath \\ \hline
U+1F6C1 & {\EmojiFont 🛁} & {\textbackslash}:bathtub: & Bathtub \\ \hline
U+1F6C2 & {\EmojiFont 🛂} & {\textbackslash}:passport\_control: & Passport Control \\ \hline
U+1F6C3 & {\EmojiFont 🛃} & {\textbackslash}:customs: & Customs \\ \hline
U+1F6C4 & {\EmojiFont 🛄} & {\textbackslash}:baggage\_claim: & Baggage Claim \\ \hline
U+1F6C5 & {\EmojiFont 🛅} & {\textbackslash}:left\_luggage: & Left Luggage \\ \hline

  U+03030 & {\EmojiFont 〰} & {\textbackslash}:wavy\_dash: & Wavy Dash \\ \hline
  U+0303D & {\EmojiFont 〽} & {\textbackslash}:part\_alternation\_mark: & Part Alternation Mark \\ \hline
  U+03297 & {\EmojiFont ㊗} & {\textbackslash}:congratulations: & Circled Ideograph Congratulation \\ \hline
  U+03299 & {\EmojiFont ㊙} & {\textbackslash}:secret: & Circled Ideograph Secret \\ \hline
  
  \bottomrule
\end{longtable}

Julia 提供了非常灵活的变量命名策略。
变量名是大小写敏感的,且不包含语义,意思是说,Julia 不会根据变量的名字来区别对待它们。 
(译者注:Julia \textbf{不会}自动将全大写的变量识别为常量,
也\textbf{不会}将有特定前后缀的变量自动识别为某种特定类型的变量,即不会根据变量名字,自动判断变量的任何属性。)

\end{document} % 文档结尾