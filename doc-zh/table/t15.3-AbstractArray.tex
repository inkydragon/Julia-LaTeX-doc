

%% 旋转页面

\newgeometry{hmargin=1.5cm,vmargin=2cm}
\begin{landscape}

\begin{table}[h] 
  \centering 
\begin{tabulary}{\linewidth}{LLL}
  \toprule
  \textbf{需要实现的方法} &  & \textbf{简短描述} \\
  \midrule
  
  \texttt{size(A)} &  & 返回包含 \texttt{A} 各维度大小的元组 \\ \midrule
  \texttt{getindex(A, i::Int)} &  & (若为 \texttt{IndexLinear})线性标量索引 \\ \midrule
  \texttt{getindex(A, I::Vararg\{Int, N\})} &  & (若为 \texttt{IndexCartesian},其中 \texttt{N = ndims(A)})N 维标量索引 \\ \midrule
  \texttt{setindex!(A, v, i::Int)} &  & (若为 \texttt{IndexLinear})线性索引元素赋值 \\ \midrule
  \texttt{setindex!(A, v, I::Vararg\{Int, N\})} &  & (若为 \texttt{IndexCartesian},其中 \texttt{N = ndims(A)})N 维标量索引元素赋值 \\ 
  
  \toprule
  \textbf{可选方法} & \textbf{默认定义} & \textbf{简短描述} \\ 
  \midrule
  \texttt{IndexStyle(::Type)} & \texttt{IndexCartesian()} & 返回 \texttt{IndexLinear()} 或 \texttt{IndexCartesian()}。请参阅下文描述。 \\ \midrule
  \texttt{getindex(A, I...)} & 基于标量 \texttt{getindex} 定义 & \hyperlink{14469287548874312017}{多维非标量索引} \\ \midrule
  \texttt{setindex!(A, X, I...)} & 基于标量 \texttt{setindex!} 定义 & \hyperlink{14469287548874312017}{多维非标量索引元素赋值} \\ \midrule
  \texttt{iterate} & 基于标量 \texttt{getindex} 定义 & Iteration \\ \midrule
  \texttt{length(A)} & \texttt{prod(size(A))} & 元素数 \\ \midrule
  \texttt{similar(A)} & \texttt{similar(A, eltype(A), size(A))} & 返回具有相同形状和元素类型的可变数组 \\ \midrule
  \texttt{similar(A, ::Type\{S\})} & \texttt{similar(A, S, size(A))} & 返回具有相同形状和指定元素类型的可变数组 \\ \midrule
  \texttt{similar(A, dims::Dims)} & \texttt{similar(A, eltype(A), dims)} & 返回具有相同元素类型和大小为 \emph{dims} 的可变数组 \\ \midrule
  \texttt{similar(A, ::Type\{S\}, dims::Dims)} & \texttt{Array\{S\}(undef, dims)} & 返回具有指定元素类型及大小的可变数组 \\ 
  
  \toprule
  \textbf{不遵循惯例的索引} & \textbf{默认定义} & \textbf{简短描述} \\ 
  \midrule
  \texttt{axes(A)} & \texttt{map(OneTo, size(A))} & 返回有效索引的 \texttt{AbstractUnitRange} \\ \midrule
  \texttt{similar(A, ::Type\{S\}, inds)} & \texttt{similar(A, S, Base.to\_shape(inds))} & 返回使用特殊索引 \texttt{inds} 的可变数组(详见下文) \\ \midrule
  \texttt{similar(T::Union\{Type,Function\}, inds)} & \texttt{T(Base.to\_shape(inds))} & 返回类似于 \texttt{T} 的使用特殊索引 \texttt{inds} 的数组(详见下文) \\ 
  \bottomrule
\end{tabulary}
\end{table}

\end{landscape}
\restoregeometry


%% 正常版式 -------------------------------------------------------------------------------
% \begin{table}[h] 
%   \centering 
% \begin{tabulary}{\linewidth}{LLL}
%   \toprule
%   \textbf{需要实现的方法} &  & \textbf{简短描述} \\
%   \midrule

%   \texttt{size(A)} &  & 返回包含 \texttt{A} 各维度大小的元组 \\ \midrule
%   \texttt{getindex(A, i::Int)} &  & (若为 \texttt{IndexLinear})线性标量索引 \\ \midrule
%   \texttt{getindex(A, I::Vararg\{Int, N\})} &  & (若为 \texttt{IndexCartesian},其中 \texttt{N = ndims(A)})N 维标量索引 \\ \midrule
%   \texttt{setindex!(A, v, i::Int)} &  & (若为 \texttt{IndexLinear})线性索引元素赋值 \\ \midrule
%   \texttt{setindex!(A, v, I::Vararg\{Int, N\})} &  & (若为 \texttt{IndexCartesian},其中 \texttt{N = ndims(A)})N 维标量索引元素赋值 \\ 

%   \toprule
%   \textbf{可选方法} & \textbf{默认定义} & \textbf{简短描述} \\ 
%   \midrule
%   \texttt{IndexStyle(::Type)} & \texttt{IndexCartesian()} & 返回 \texttt{IndexLinear()} 或 \texttt{IndexCartesian()}。请参阅下文描述。 \\ \midrule
%   \texttt{getindex(A, I...)} & 基于标量 \texttt{getindex} 定义 & \hyperlink{14469287548874312017}{多维非标量索引} \\ \midrule
%   \texttt{setindex!(A, X, I...)} & 基于标量 \texttt{setindex!} 定义 & \hyperlink{14469287548874312017}{多维非标量索引元素赋值} \\ \midrule
%   \texttt{iterate} & 基于标量 \texttt{getindex} 定义 & Iteration \\ \midrule
%   \texttt{length(A)} & \texttt{prod(size(A))} & 元素数 \\ \midrule
%   \texttt{similar(A)} & \texttt{similar(A, eltype(A), size(A))} & 返回具有相同形状和元素类型的可变数组 \\ \midrule
%   \texttt{similar(A, ::Type\{S\})} & \texttt{similar(A, S, size(A))} & 返回具有相同形状和指定元素类型的可变数组 \\ \midrule
%   \texttt{similar(A, dims::Dims)} & \texttt{similar(A, eltype(A), dims)} & 返回具有相同元素类型和大小为 \emph{dims} 的可变数组 \\ \midrule
%   \texttt{similar(A, ::Type\{S\}, dims::Dims)} & \texttt{Array\{S\}(undef, dims)} & 返回具有指定元素类型及大小的可变数组 \\ 

%   \toprule
%   \textbf{不遵循惯例的索引} & \textbf{默认定义} & \textbf{简短描述} \\ 
%   \midrule
%   \texttt{axes(A)} & \texttt{map(OneTo, size(A))} & 返回有效索引的 \texttt{AbstractUnitRange} \\ \midrule
%   \texttt{similar(A, ::Type\{S\}, inds)} & \texttt{similar(A, S, Base.to\_shape(inds))} & 返回使用特殊索引 \texttt{inds} 的可变数组(详见下文) \\ \midrule
%   \texttt{similar(T::Union\{Type,Function\}, inds)} & \texttt{T(Base.to\_shape(inds))} & 返回类似于 \texttt{T} 的使用特殊索引 \texttt{inds} 的数组(详见下文) \\ 
%   \bottomrule
% \end{tabulary}
% \end{table}
