
\part{手册}


\hypertarget{14943148626325101976}{}


\chapter{入门}



无论是使用预编译好的二进制程序,还是自己从源码编译,安装 Julia 都是一件很简单的事情。 请按照 \href{https://julialang.org/downloads/}{https://julialang.org/downloads/} 的提示来下载并安装 Julia。



启动一个交互式会话(也叫 REPL)是学习和尝试 Julia 最简单的方法。双击 Julia 的可执行文件或是从命令行运行 \texttt{julia} 就可以启动:




\begin{lstlisting}
$ julia

               _
   _       _ _(_)_     |  Documentation: https://docs.julialang.org
  (_)     | (_) (_)    |
   _ _   _| |_  __ _   |  Type "?" for help, "]?" for Pkg help.
  | | | | | | |/ _` |  |
  | | |_| | | | (_| |  |  Version 1.4.2 (2020-05-23)
 _/ |\__'_|_|_|\__'_|  |  Official https://julialang.org/ release
|__/                   |


julia> 1 + 2
3

julia> ans
3
\end{lstlisting}



输入 \texttt{CTRL-D}(同时按 \texttt{Ctrl} 键和 \texttt{d} 键)或 \texttt{exit()} 便可以退出交互式会话。在交互式模式中,\texttt{julia} 会显示一条横幅并提示用户输入。一旦用户输入了一段完整的代码(表达式),例如 \texttt{1 + 2},然后按回车,交互式会话就会执行这段代码,并将结果显示出来。如果输入的代码以分号结尾,那么结果将不会显示出来。然而不管结果显示与否,变量 \texttt{ans} 总会存储上一次执行代码的结果,需要注意的是,变量 \texttt{ans} 只在交互式会话中才有。



在交互式会话中,要运行写在源文件 \texttt{file.jl} 中的代码,只需输入 \texttt{include({\textquotedbl}file.jl{\textquotedbl})}。



如果想非交互式地执行文件中的代码,可以把文件名作为 \texttt{julia} 命令的第一个参数:




\begin{lstlisting}
$ julia script.jl arg1 arg2...
\end{lstlisting}



如这个例子所示,\texttt{julia} 后跟着的命令行参数会被作为程序 \texttt{script.jl} 的命令行参数。这些参数使用全局常量 \texttt{ARGS} 来传递,脚本自身的名字会以全局变量 \texttt{PROGRAM\_FILE} 传入。注意当脚本以命令行里的 \texttt{-e} 选项输入时,\texttt{ARGS} 也会被设定(详见此页末尾列表)但是 \texttt{PROGRAM\_FILE} 会是空的。例如,要把一个脚本的输入参数显示出来,你可以:




\begin{lstlisting}
$ julia -e 'println(PROGRAM_FILE); for x in ARGS; println(x); end' foo bar

foo
bar
\end{lstlisting}



或者你可以把代码写到一个脚本文件中再执行它:




\begin{lstlisting}
$ echo 'println(PROGRAM_FILE); for x in ARGS; println(x); end' > script.jl
$ julia script.jl foo bar
script.jl
foo
bar
\end{lstlisting}



可以使用 \texttt{--} 分隔符来将传给脚本文件的参数和 Julia 本身的命令行参数区分开:




\begin{lstlisting}
$ julia --color=yes -O -- foo.jl arg1 arg2..
\end{lstlisting}



有关编写 Julia 脚本的更多信息,请参阅 \hyperlink{9384335615524550131}{脚本}。



使用选项 \texttt{-p} 或者 \texttt{--machine-file} 可以在并行模式下启动 Julia。 \texttt{-p n} 会启动额外的 \texttt{n} 个 worker,使用 \texttt{--machine-file file} 会为 \texttt{file} 文件中的每一行启动一个 worker。 定义在 \texttt{file} 中的机器必须能够通过一个不需要密码的 \texttt{ssh} 登陆访问到,且 Julia 的安装位置需要和当前主机相同。 定义机器的格式为 \texttt{[count*][user@]host[:port] [bind\_addr[:port]]}。 \texttt{user} 默认值是当前用户; \texttt{port} 默认值是标准 ssh 端口; \texttt{count} 是在这个节点上的 worker 的数量,默认是 1; 可选的 \texttt{bind-to bind\_addr[:port]} 指定了其它 worker 访问当前 worker 应当使用的 IP 地址与端口。



要让 Julia 每次启动都自动执行一些代码,你可以把它们放在 \texttt{{\textasciitilde}/.julia/config/startup.jl} 中:




\begin{lstlisting}
$ echo 'println("Greetings! 你好! 안녕하세요?")' > ~/.julia/config/startup.jl
$ julia
Greetings! 你好! 안녕하세요?

...
\end{lstlisting}



Note that although you should have a \texttt{{\textasciitilde}/.julia} directory once you{\textquotesingle}ve run Julia for the first time, you may need to create the \texttt{{\textasciitilde}/.julia/config} folder and the \texttt{{\textasciitilde}/.julia/config/startup.jl} file if you use it.



和 \texttt{perl} 和 \texttt{ruby} 程序类似,还有很多种运行 Julia 代码的方式,运行代码时也有很多选项:




\begin{lstlisting}
julia [switches] -- [programfile] [args...]
\end{lstlisting}


\begin{longtable}{lp{0.55\linewidth}}
    \caption{julia 命令行参数} \label{julia_cmd_arg} \\
    %% 所有表格 表头
    \toprule
    \bf{选项} & \bf{描述} \\ 
    \hline \endhead
    %% 所有表格 页脚
    \multicolumn{2}{r}{下页待续} \\ 
    \midrule \endfoot
    %% 表格最后一页 页脚
    \bottomrule \endlastfoot

    %% 表格正文
    \texttt{-v}, \texttt{--version} & 显示版本信息 \\ \hline
    \texttt{-h}, \texttt{--help} & Print command-line options (this message). \\ \hline
    \texttt{--project[=\{<dir>|@.\}]} & 将 <dir> 设置为主项目/环境。默认的 @. 选项将搜索父目录,直至找到 Project.toml 或 JuliaProject.toml 文件。 \\ \hline
    \texttt{-J}, \texttt{--sysimage <file>} & 用指定的镜像文件(system image file)启动 \\ \hline
    \texttt{-H}, \texttt{--home <dir>} & 设置 \texttt{julia} 可执行文件的路径 \\ \hline
    \texttt{--startup-file=\{yes|no\}} & 是否载入 \texttt{{\textasciitilde}/.julia/config/startup.jl} \\ \hline
    \texttt{--handle-signals=\{yes|no\}} & 开启或关闭 Julia 默认的 signal handlers \\ \hline
    \texttt{--sysimage-native-code=\{yes|no\}} & 在可能的情况下,使用系统镜像里的原生代码 \\ \hline
    \texttt{--compiled-modules=\{yes|no\}} & 开启或关闭 module 的增量预编译功能 \\ \hline
    \texttt{-e}, \texttt{--eval <expr>} & 执行 \texttt{<expr>} \\ \hline
    \texttt{-E}, \texttt{--print <expr>} & 执行 \texttt{<expr>} 并显示结果 \\ \hline
    \texttt{-L}, \texttt{--load <file>} & 立即在所有进程中载入 \texttt{<file>} \\ \hline
    \texttt{-p}, \texttt{--procs \{N|auto\}} & 这里的整数 N 表示启动 N 个额外的工作进程;\texttt{auto} 表示启动与 CPU 线程数目(logical cores)一样多的进程 \\ \hline
    \texttt{--machine-file <file>} & 在 \texttt{<file>} 中列出的主机上运行进程 \\ \hline
    \texttt{-i} & 交互式模式;REPL 运行且 \texttt{isinteractive()} 为 true \\ \hline
    \texttt{-q}, \texttt{--quiet} & 安静的启动;REPL 启动时无横幅,不显示警告 \\ \hline
    \texttt{--banner=\{yes|no|auto\}} & 开启或关闭 REPL 横幅 \\ \hline
    \texttt{--color=\{yes|no|auto\}} & 开启或关闭文字颜色 \\ \hline
    \texttt{--history-file=\{yes|no\}} & 载入或导出历史记录 \\ \hline
    \texttt{--depwarn=\{yes|no|error\}} & 开启或关闭语法弃用警告,\texttt{error} 表示将弃用警告转换为错误。 \\ \hline
    \texttt{--warn-overwrite=\{yes|no\}} & 开启或关闭“method overwrite”警告 \\ \hline
    \texttt{-C}, \texttt{--cpu-target <target>} & 设置 \texttt{<target>} 来限制使用 CPU 的某些特性;设置为 \texttt{help} 可以查看可用的选项 \\ \hline
    \texttt{-O}, \texttt{--optimize=\{0,1,2,3\}} & 设置编译器优化级别(若未配置此选项,则默认等级为2;若配置了此选项却没指定具体级别,则默认级别为3)。 \\ \hline
    \texttt{-g}, \texttt{-g <level>} & 开启或设置 debug 信息的生成等级。若未配置此选项,则默认 debug 信息的级别为 1;若配置了此选项却没指定具体级别,则默认级别为 2。 \\ \hline
    \texttt{--inline=\{yes|no\}} & 控制是否允许函数内联,此选项会覆盖源文件中的 \texttt{@inline} 声明 \\ \hline
    \texttt{--check-bounds=\{yes|no\}} & 设置边界检查状态:始终检查或永不检查。永不检查时会忽略源文件中的相应声明 \\ \hline
    \texttt{--math-mode=\{ieee,fast\}} & 开启或关闭非安全的浮点数代数计算优化,此选项会覆盖源文件中的 \texttt{@fastmath} 声明 \\ \hline
    \texttt{--code-coverage=\{none|user|all\}} & 对源文件中每行代码执行的次数计数 \\ \hline
    \texttt{--code-coverage} & 等价于 \texttt{--code-coverage=user} \\ \hline
    \texttt{--track-allocation=\{none|user|all\}} & 对源文件中每行代码的内存分配计数,单位 byte \\ \hline
    \texttt{--track-allocation} & 等价于 \texttt{--track-allocation=user} \\
    
    \bottomrule
\end{longtable}

% \begin{table}[h]

% \begin{tabulary}{\linewidth}{|L|L|}
% \hline
% 选项 & 描述 \\
% \hline
% \texttt{-v}, \texttt{--version} & 显示版本信息 \\
% \hline
% \texttt{-h}, \texttt{--help} & Print command-line options (this message). \\
% \hline
% \texttt{--project[=\{<dir>|@.\}]} & 将 <dir> 设置为主项目/环境。默认的 @. 选项将搜索父目录,直至找到 Project.toml 或 JuliaProject.toml 文件。 \\
% \hline
% \texttt{-J}, \texttt{--sysimage <file>} & 用指定的镜像文件(system image file)启动 \\
% \hline
% \texttt{-H}, \texttt{--home <dir>} & 设置 \texttt{julia} 可执行文件的路径 \\
% \hline
% \texttt{--startup-file=\{yes|no\}} & 是否载入 \texttt{{\textasciitilde}/.julia/config/startup.jl} \\
% \hline
% \texttt{--handle-signals=\{yes|no\}} & 开启或关闭 Julia 默认的 signal handlers \\
% \hline
% \texttt{--sysimage-native-code=\{yes|no\}} & 在可能的情况下,使用系统镜像里的原生代码 \\
% \hline
% \texttt{--compiled-modules=\{yes|no\}} & 开启或关闭 module 的增量预编译功能 \\
% \hline
% \texttt{-e}, \texttt{--eval <expr>} & 执行 \texttt{<expr>} \\
% \hline
% \texttt{-E}, \texttt{--print <expr>} & 执行 \texttt{<expr>} 并显示结果 \\
% \hline
% \texttt{-L}, \texttt{--load <file>} & 立即在所有进程中载入 \texttt{<file>} \\
% \hline
% \texttt{-t}, \texttt{--threads \{N|auto\}} & Enable N threads; \texttt{auto} currently sets N to the number of local CPU threads but this might change in the future \\
% \hline
% \texttt{-p}, \texttt{--procs \{N|auto\}} & 这里的整数 N 表示启动 N 个额外的工作进程;\texttt{auto} 表示启动与 CPU 线程数目(logical cores)一样多的进程 \\
% \hline
% \texttt{--machine-file <file>} & 在 \texttt{<file>} 中列出的主机上运行进程 \\
% \hline
% \texttt{-i} & 交互式模式;REPL 运行且 \texttt{isinteractive()} 为 true \\
% \hline
% \texttt{-q}, \texttt{--quiet} & 安静的启动;REPL 启动时无横幅,不显示警告 \\
% \hline
% \texttt{--banner=\{yes|no|auto\}} & 开启或关闭 REPL 横幅 \\
% \hline
% \texttt{--color=\{yes|no|auto\}} & 开启或关闭文字颜色 \\
% \hline
% \texttt{--history-file=\{yes|no\}} & 载入或导出历史记录 \\
% \hline
% \texttt{--depwarn=\{yes|no|error\}} & 开启或关闭语法弃用警告,\texttt{error} 表示将弃用警告转换为错误。 \\
% \hline
% \texttt{--warn-overwrite=\{yes|no\}} & 开启或关闭“method overwrite”警告 \\
% \hline
% \texttt{-C}, \texttt{--cpu-target <target>} & 设置 \texttt{<target>} 来限制使用 CPU 的某些特性;设置为 \texttt{help} 可以查看可用的选项 \\
% \hline
% \texttt{-O}, \texttt{--optimize=\{0,1,2,3\}} & 设置编译器优化级别(若未配置此选项,则默认等级为2;若配置了此选项却没指定具体级别,则默认级别为3)。 \\
% \hline
% \texttt{-g}, \texttt{-g <level>} & 开启或设置 debug 信息的生成等级。若未配置此选项,则默认 debug 信息的级别为 1;若配置了此选项却没指定具体级别,则默认级别为 2。 \\
% \hline
% \texttt{--inline=\{yes|no\}} & 控制是否允许函数内联,此选项会覆盖源文件中的 \texttt{@inline} 声明 \\
% \hline
% \texttt{--check-bounds=\{yes|no\}} & 设置边界检查状态:始终检查或永不检查。永不检查时会忽略源文件中的相应声明 \\
% \hline
% \texttt{--math-mode=\{ieee,fast\}} & 开启或关闭非安全的浮点数代数计算优化,此选项会覆盖源文件中的 \texttt{@fastmath} 声明 \\
% \hline
% \texttt{--code-coverage=\{none|user|all\}} & 对源文件中每行代码执行的次数计数 \\
% \hline
% \texttt{--code-coverage} & 等价于 \texttt{--code-coverage=user} \\
% \hline
% \texttt{--track-allocation=\{none|user|all\}} & 对源文件中每行代码的内存分配计数,单位 byte \\
% \hline
% \texttt{--track-allocation} & 等价于 \texttt{--track-allocation=user} \\
% \hline
% \end{tabulary}

% \end{table}



\begin{quote}
\textbf{Julia 1.1}

在 Julia 1.0 中,默认的 \texttt{--project=@.} 选项不会在 Git 仓库的根目录中寻找 \texttt{Project.toml} 文件。从 Julia 1.1 开始,此选项会在其中寻找该文件。

\end{quote}


\hypertarget{17073013932993739054}{}


\section{资源}



除了本手册以外,官方网站还提供了一个有用的\textbf{\href{https://julialang.org/learning/}{学习资源列表}}来帮助新用户学习 Julia。



\hypertarget{10731958648755981077}{}


\chapter{Variables}



Julia 语言中,变量是与某个值相关联(或绑定)的名字。你可以用它来保存一个值(例如某些计算得到的结果),供之后的代码使用。例如:




\begin{minted}{jlcon}
# 将 10 赋值给变量 x
julia> x = 10
10

# 使用 x 的值做计算
julia> x + 1
11

# 重新给 x 赋值
julia> x = 1 + 1
2

# 也可以给 x 赋其它类型的值, 比如字符串文本
julia> x = "Hello World!"
"Hello World!"
\end{minted}



Julia 提供了非常灵活的变量命名策略。变量名是大小写敏感的,且不包含语义,意思是说,Julia 不会根据变量的名字来区别对待它们。 (译者注:Julia \textbf{不会}自动将全大写的变量识别为常量,也\textbf{不会}将有特定前后缀的变量自动识别为某种特定类型的变量,即不会根据变量名字,自动判断变量的任何属性。)




\begin{minted}{jlcon}
julia> x = 1.0
1.0

julia> y = -3
-3

julia> Z = "My string"
"My string"

julia> customary_phrase = "Hello world!"
"Hello world!"

julia> UniversalDeclarationOfHumanRightsStart = "人人生而自由,在尊严和权利上一律平等。"
"人人生而自由,在尊严和权利上一律平等。"
\end{minted}



你还可以使用 UTF-8 编码的 Unicode 字符作为变量名:




\begin{minted}{jlcon}
julia> δ = 0.00001
1.0e-5

julia> 안녕하세요 = "Hello"
"Hello"
\end{minted}



在 Julia REPL 和一些其它的 Julia 编辑环境中,很多 Unicode 数学符号可以使用反斜杠加 LaTeX 符号名再按 \emph{tab} 健打出。 例如:变量名 \texttt{δ} 可以通过 \texttt{{\textbackslash}delta} \emph{tab} 来输入,甚至可以用 \texttt{{\textbackslash}alpha} \emph{tab} \texttt{{\textbackslash}hat} \emph{tab} \texttt{{\textbackslash}\_2} \emph{tab} 来输入 \texttt{α̂₂}  这种复杂的变量名。 如果你在某个地方(比如别人的代码里)看到了一个不知道怎么输入的符号,你可以在REPL中输入 \texttt{?},然后粘贴那个符号,帮助文档会告诉你输入方法。



如果有需要的话,Julia 甚至允许你重定义内置常量和函数。(这样做可能引发潜在的混淆,所以并不推荐)




\begin{minted}{jlcon}
julia> pi = 3
3

julia> pi
3

julia> sqrt = 4
4
\end{minted}



然而,如果你试图重定义一个已经在使用中的内置常量或函数,Julia 会报错:




\begin{minted}{jlcon}
julia> pi
π = 3.1415926535897...

julia> pi = 3
ERROR: cannot assign a value to variable MathConstants.pi from module Main

julia> sqrt(100)
10.0

julia> sqrt = 4
ERROR: cannot assign a value to variable Base.sqrt from module Main
\end{minted}



\hypertarget{8427534705431371449}{}


\section{合法的变量名}



变量名字必须以英文字母(A-Z 或 a-z)、下划线或编码大于 00A0 的 Unicode 字符的一个子集开头。 具体来说指的是,\href{http://www.fileformat.info/info/unicode/category/index.htm}{Unicode字符分类}中的 Lu/Ll/Lt/Lm/Lo/Nl(字母)、Sc/So(货币和其他符号)以及一些其它像字母的符号(例如 Sm 类别数学符号中的一部分)。 变量名的非首字符还允许使用惊叹号 \texttt{!}、数字(包括 0-9 和其他 Nd/No 类别中的 Unicode 字符)以及其它 Unicode 字符:变音符号和其他修改标记(Mn/Mc/Me/Sk 类别)、标点和连接符(Pc 类别)、引号和少许其他字符。



像 \texttt{+} 这样的运算符也是合法的标识符,但是它们会被特别地解析。 在一些语境中,运算符可以像变量一样使用,比如 \texttt{(+)} 表示加函数,语句 \texttt{(+) = f} 会把它重新赋值。 大部分 Sm 类别中的 Unicode 中缀运算符,像 \texttt{⊕},则会被解析成真正的中缀运算符,并且支持用户自定义方法(举个例子,你可以使用语句 \texttt{const ⊗ = kron} 将 \texttt{⊗} 定义为中缀的 Kronecker 积)。 运算符也可以使用修改标记、引号和上标/下标进行加缀,例如 \texttt{+̂ₐ″} 被解析成一个与 \texttt{+} 具有相同优先级的中缀运算符。



The only explicitly disallowed names for variables are the names of the built-in \href{@ref}{Keywords}:




\begin{minted}{jlcon}
julia> else = false
ERROR: syntax: unexpected "else"

julia> try = "No"
ERROR: syntax: unexpected "="
\end{minted}



Some Unicode characters are considered to be equivalent in identifiers. Different ways of entering Unicode combining characters (e.g., accents) are treated as equivalent (specifically, Julia identifiers are \href{http://www.macchiato.com/unicode/nfc-faq}{NFC}-normalized). The Unicode characters \texttt{ɛ} (U+025B: Latin small letter open e) and \texttt{µ} (U+00B5: micro sign) are treated as equivalent to the corresponding Greek letters, because the former are easily accessible via some input methods.



\hypertarget{1519367584459167025}{}


\section{命名规范}



虽然 Julia 语言对合法名字的限制非常少,但是遵循以下这些命名规范是非常有用的:



\begin{itemize}
\item 变量的名字采用小写。


\item 用下划线(\texttt{\_})分隔名字中的单词,但是不鼓励使用下划线, 除非在不使用下划线时名字会非常难读。


\item 类型(Type)和模块(Module)的名字使用大写字母开头,并且用大写字母 而不是用下划线分隔单词。


\item 函数(Function)和宏(Macro)的名字使用小写,不使用下划线。


\item 会对输入参数进行更改的函数要使用 \texttt{!} 结尾。这些函数有时叫做 “mutating” 或 “in-place” 函数,因为它们在被调用后,不仅仅会返回一些值 还会更改输入参数的内容。

\end{itemize}


关于命名规范的更多信息,可查看\hyperlink{12507952184948113283}{代码风格指南}。



\hypertarget{9431281250101057989}{}


\chapter{整数和浮点数}



整数和浮点值是算术和计算的基础。这些数值的内置表示被称作原始数值类型(numeric primitive),且整数和浮点数在代码中作为立即数时称作数值字面量(numeric literal)。例如,\texttt{1} 是个整型字面量,\texttt{1.0} 是个浮点型字面量,它们在内存中作为对象的二进制表示就是原始数值类型。



Julia 提供了很丰富的原始数值类型,并基于它们定义了一整套算术操作,还提供按位运算符以及一些标准数学函数。这些函数能够直接映射到现代计算机原生支持的数值类型及运算上,因此 Julia 可以充分地利用运算资源。此外,Julia 还为\hyperlink{7537478913062818871}{任意精度算术}提供了软件支持,对于无法使用原生硬件表示的数值类型,Julia 也能够高效地处理其数值运算。当然,这需要相对的牺牲一些性能。



以下是 Julia 的原始数值类型:



\begin{itemize}
\item \textbf{整数类型:}
\end{itemize}
%% re 替换
% find:     r"(-){0,1}(2)(\{\\textasciicircum\})(\d+)( - 1){0,1}"
% replace:  r"$ $1$2^{$4}$5 $"

\begin{table}[h]
  \centering
\begin{tabulary}{\linewidth}{LLLLL}
  \toprule
  类型 & 带符号? & 比特数 & 最小值 & 最大值 \\
  \midrule

  \hyperlink{5857518405103968275}{\texttt{Int8}} & ✓ & 8 & $ -2^{7} $ & $2^{7} - 1$ \\ \midrule
  \hyperlink{6609065134969660118}{\texttt{UInt8}} &  & 8 & 0 & $ 2^{8} - 1 $ \\ \midrule
  \hyperlink{6667287249103968645}{\texttt{Int16}} & ✓ & 16 & $ -2^{15} $ & $ 2^{15} - 1 $ \\ \midrule
  \hyperlink{7018610346698168012}{\texttt{UInt16}} &  & 16 & 0 & $ 2^{16} - 1 $ \\ \midrule
  \hyperlink{10103694114785108551}{\texttt{Int32}} & ✓ & 32 & $ -2^{31} $ & $ 2^{31} - 1 $ \\ \midrule
  \hyperlink{8690996847580776341}{\texttt{UInt32}} &  & 32 & 0 & $ 2^{32} - 1 $ \\ \midrule
  \hyperlink{7720564657383125058}{\texttt{Int64}} & ✓ & 64 & $ -2^{63} $ & $ 2^{63} - 1 $ \\ \midrule
  \hyperlink{5500998675195555601}{\texttt{UInt64}} &  & 64 & 0 & $ 2^{64} - 1 $ \\ \midrule
  \hyperlink{8012327724714767060}{\texttt{Int128}} & ✓ & 128 & $ -2^{127} $ & $ 2^{127} - 1 $ \\ \midrule
  \hyperlink{14811222188335428522}{\texttt{UInt128}} &  & 128 & 0 & $ 2^{128} - 1 $ \\ \midrule
  \hyperlink{46725311238864537}{\texttt{Bool}} & N/A & 8 & \texttt{false} (0) & \texttt{true} (1) \\
  \bottomrule
\end{tabulary}
\end{table}



% \begin{table}[h]

% \begin{tabulary}{\linewidth}{|L|L|L|L|L|}
% \hline
% 类型 & 带符号? & 比特数 & 最小值 & 最大值 \\
% \hline
% \hyperlink{5857518405103968275}{\texttt{Int8}} & ✓ & 8 & -2{\textasciicircum}7 & 2{\textasciicircum}7 - 1 \\
% \hline
% \hyperlink{6609065134969660118}{\texttt{UInt8}} &  & 8 & 0 & 2{\textasciicircum}8 - 1 \\
% \hline
% \hyperlink{6667287249103968645}{\texttt{Int16}} & ✓ & 16 & -2{\textasciicircum}15 & 2{\textasciicircum}15 - 1 \\
% \hline
% \hyperlink{7018610346698168012}{\texttt{UInt16}} &  & 16 & 0 & 2{\textasciicircum}16 - 1 \\
% \hline
% \hyperlink{10103694114785108551}{\texttt{Int32}} & ✓ & 32 & -2{\textasciicircum}31 & 2{\textasciicircum}31 - 1 \\
% \hline
% \hyperlink{8690996847580776341}{\texttt{UInt32}} &  & 32 & 0 & 2{\textasciicircum}32 - 1 \\
% \hline
% \hyperlink{7720564657383125058}{\texttt{Int64}} & ✓ & 64 & -2{\textasciicircum}63 & 2{\textasciicircum}63 - 1 \\
% \hline
% \hyperlink{5500998675195555601}{\texttt{UInt64}} &  & 64 & 0 & 2{\textasciicircum}64 - 1 \\
% \hline
% \hyperlink{8012327724714767060}{\texttt{Int128}} & ✓ & 128 & -2{\textasciicircum}127 & 2{\textasciicircum}127 - 1 \\
% \hline
% \hyperlink{14811222188335428522}{\texttt{UInt128}} &  & 128 & 0 & 2{\textasciicircum}128 - 1 \\
% \hline
% \hyperlink{46725311238864537}{\texttt{Bool}} & N/A & 8 & \texttt{false} (0) & \texttt{true} (1) \\
% \hline
% \end{tabulary}
% \end{table}



\begin{itemize}
\item \textbf{浮点类型:}
\end{itemize}

\begin{table}[h]
  \centering
\begin{tabulary}{\linewidth}{|L|L|L|}
  \hline
  类型 & 精度 & 比特数 \\
  \hline

  \hyperlink{2727296760866702904}{\texttt{Float16}} & \href{https://en.wikipedia.org/wiki/Half-precision\_floating-point\_format}{half} & 16 \\ \hline
  \hyperlink{8101639384272933082}{\texttt{Float32}} & \href{https://en.wikipedia.org/wiki/Single\_precision\_floating-point\_format}{single} & 32 \\ \hline
  \hyperlink{5027751419500983000}{\texttt{Float64}} & \href{https://en.wikipedia.org/wiki/Double\_precision\_floating-point\_format}{double} & 64 \\
  \hline
\end{tabulary}
\end{table}



% \begin{table}[h]
% \begin{tabulary}{\linewidth}{|L|L|L|}
% \hline
% 类型 & 精度 & 比特数 \\
% \hline
% \hyperlink{2727296760866702904}{\texttt{Float16}} & \href{https://en.wikipedia.org/wiki/Half-precision\_floating-point\_format}{half} & 16 \\
% \hline
% \hyperlink{8101639384272933082}{\texttt{Float32}} & \href{https://en.wikipedia.org/wiki/Single\_precision\_floating-point\_format}{single} & 32 \\
% \hline
% \hyperlink{5027751419500983000}{\texttt{Float64}} & \href{https://en.wikipedia.org/wiki/Double\_precision\_floating-point\_format}{double} & 64 \\
% \hline
% \end{tabulary}
% \end{table}



此外,对\hyperlink{13366825053081777829}{复数和有理数}的完整支持是在这些原始数据类型之上建立起来的。多亏了 Julia 有一个很灵活的、用户可扩展的\hyperlink{10374023657104680331}{类型提升系统},所有的数值类型都无需显式转换就可以很自然地相互进行运算。



\hypertarget{1329060658000677295}{}


\section{整数}



整数字面量以标准形式表示:




\begin{minted}{jlcon}
julia> 1
1

julia> 1234
1234
\end{minted}



整型字面量的默认类型取决于目标系统是 32 位还是 64 位架构:




\begin{minted}{jlcon}
# 32 位系统:
julia> typeof(1)
Int32

# 64 位系统:
julia> typeof(1)
Int64
\end{minted}



Julia 的内置变量 \hyperlink{6553323097149877235}{\texttt{Sys.WORD\_SIZE}} 表明了目标系统是 32 位还是 64 位架构:




\begin{minted}{jlcon}
# 32 位系统:
julia> Sys.WORD_SIZE
32

# 64 位系统:
julia> Sys.WORD_SIZE
64
\end{minted}



Julia 也定义了 \texttt{Int} 与 \texttt{UInt} 类型,它们分别是系统有符号和无符号的原生整数类型的别名。




\begin{minted}{jlcon}
# 32 位系统:
julia> Int
Int32
julia> UInt
UInt32

# 64 位系统:
julia> Int
Int64
julia> UInt
UInt64
\end{minted}



那些超过 32 位表示范围的大整数,如果能用 64 位表示,那么无论是什么系统都会用 64 位表示:




\begin{minted}{jlcon}
# 32 位或 64 位系统:
julia> typeof(3000000000)
Int64
\end{minted}



无符号整数会通过 \texttt{0x} 前缀以及十六进制数 \texttt{0-9a-f} 来输入和输出(输入也可以使用大写的 \texttt{A-F})。无符号值的位数取决于十六进制数字使用的数量:




\begin{minted}{jlcon}
julia> 0x1
0x01

julia> typeof(ans)
UInt8

julia> 0x123
0x0123

julia> typeof(ans)
UInt16

julia> 0x1234567
0x01234567

julia> typeof(ans)
UInt32

julia> 0x123456789abcdef
0x0123456789abcdef

julia> typeof(ans)
UInt64

julia> 0x11112222333344445555666677778888
0x11112222333344445555666677778888

julia> typeof(ans)
UInt128
\end{minted}



采用这种做法是因为,当人们使用无符号十六进制字面量表示整数值的时候,通常会用它们来表示一个固定的数值字节序列,而不仅仅是个整数值。



还记得这个 \hyperlink{11288188119698492222}{\texttt{ans}} 变量吗?它存着交互式会话中上一个表达式的运算结果,但以其他方式运行的 Julia 代码中没有这个变量。



二进制和八进制字面量也是支持的:




\begin{minted}{jlcon}
julia> 0b10
0x02

julia> typeof(ans)
UInt8

julia> 0o010
0x08

julia> typeof(ans)
UInt8

julia> 0x00000000000000001111222233334444
0x00000000000000001111222233334444

julia> typeof(ans)
UInt128
\end{minted}



二进制、八进制和十六进制的字面量都会产生无符号的整数类型。当字面量不是开头全是 0 时,它们二进制数据项的位数会是最少需要的位数。当开头都是 \texttt{0} 时,位数取决于一个字面量需要的最少位数,这里的字面量指的是一个有着同样长度但开头都为 \texttt{1} 的数。这样用户就可以控制位数了。那些无法使用 \texttt{UInt128} 类型存储下的值无法写成这样的字面量。



二进制、八进制和十六进制的字面量可以在前面紧接着加一个负号 \texttt{-},这样可以产生一个和原字面量有着同样位数而值为原数的补码的数(二补数):




\begin{minted}{jlcon}
julia> -0x2
0xfe

julia> -0x0002
0xfffe
\end{minted}



整型等原始数值类型的最小和最大可表示的值可用 \hyperlink{3613894539247233488}{\texttt{typemin}} 和 \hyperlink{17760305803764597758}{\texttt{typemax}} 函数得到:




\begin{minted}{jlcon}
julia> (typemin(Int32), typemax(Int32))
(-2147483648, 2147483647)

julia> for T in [Int8,Int16,Int32,Int64,Int128,UInt8,UInt16,UInt32,UInt64,UInt128]
           println("$(lpad(T,7)): [$(typemin(T)),$(typemax(T))]")
       end
   Int8: [-128,127]
  Int16: [-32768,32767]
  Int32: [-2147483648,2147483647]
  Int64: [-9223372036854775808,9223372036854775807]
 Int128: [-170141183460469231731687303715884105728,170141183460469231731687303715884105727]
  UInt8: [0,255]
 UInt16: [0,65535]
 UInt32: [0,4294967295]
 UInt64: [0,18446744073709551615]
UInt128: [0,340282366920938463463374607431768211455]
\end{minted}



The values returned by \hyperlink{3613894539247233488}{\texttt{typemin}} and \hyperlink{17760305803764597758}{\texttt{typemax}} are always of the given argument type. (The above expression uses several features that have yet to be introduced, including \hyperlink{9034109510149997190}{for loops}, \hyperlink{205866387929607333}{Strings}, and \hyperlink{4452850363638134205}{Interpolation}, but should be easy enough to understand for users with some existing programming experience.)



\hypertarget{7600249066838051055}{}


\subsection{溢出行为}



Julia 中,超出一个类型可表示的最大值会导致循环行为:




\begin{minted}{jlcon}
julia> x = typemax(Int64)
9223372036854775807

julia> x + 1
-9223372036854775808

julia> x + 1 == typemin(Int64)
true
\end{minted}



因此,Julia 的整数算术实际上是\href{https://zh.wikipedia.org/wiki/\%E6\%A8\%A1\%E7\%AE\%97\%E6\%95\%B8}{模算数}的一种形式,它反映了现代计算机实现底层算术的特点。在可能有溢出产生的程序中,对最值边界出现循环进行显式检查是必要的。否则,推荐使用\hyperlink{7537478913062818871}{任意精度算术}中的 \hyperlink{423405808990690832}{\texttt{BigInt}} 类型作为替代。



An example of overflow behavior and how to potentially resolve it is as follows:




\begin{minted}{jlcon}
julia> 10^19
-8446744073709551616

julia> big(10)^19
10000000000000000000
\end{minted}



\hypertarget{18099425100953658872}{}


\subsection{除法错误}



\texttt{div} 函数的整数除法有两种异常情况:除以零,以及使用 -1 去除最小的负数(\hyperlink{3613894539247233488}{\texttt{typemin}})。 这两种情况都会抛出一个 \hyperlink{4168463413201806292}{\texttt{DivideError}} 错误。 \texttt{rem} 取余函数和 \texttt{mod} 取模函数在除零时抛出 \hyperlink{4168463413201806292}{\texttt{DivideError}} 错误。



\hypertarget{7313324545649063110}{}


\section{浮点数}



浮点数字面量也使用标准格式表示,必要时可使用 \href{https://en.wikipedia.org/wiki/Scientific\_notation\#E-notation}{E-表示法}:




\begin{minted}{jlcon}
julia> 1.0
1.0

julia> 1.
1.0

julia> 0.5
0.5

julia> .5
0.5

julia> -1.23
-1.23

julia> 1e10
1.0e10

julia> 2.5e-4
0.00025
\end{minted}



上面的结果都是 \hyperlink{5027751419500983000}{\texttt{Float64}} 值。使用 \texttt{f} 替代 \texttt{e} 可以得到 \hyperlink{8101639384272933082}{\texttt{Float32}} 的字面量:




\begin{minted}{jlcon}
julia> 0.5f0
0.5f0

julia> typeof(ans)
Float32

julia> 2.5f-4
0.00025f0
\end{minted}



数值容易就能转换成 \hyperlink{8101639384272933082}{\texttt{Float32}}:




\begin{minted}{jlcon}
julia> Float32(-1.5)
-1.5f0

julia> typeof(ans)
Float32
\end{minted}



也存在十六进制的浮点数字面量,但只适用于 \hyperlink{5027751419500983000}{\texttt{Float64}} 值。一般使用 \texttt{p} 前缀及以 2 为底的指数来表示:




\begin{minted}{jlcon}
julia> 0x1p0
1.0

julia> 0x1.8p3
12.0

julia> 0x.4p-1
0.125

julia> typeof(ans)
Float64
\end{minted}



Julia 也支持半精度浮点数(\hyperlink{2727296760866702904}{\texttt{Float16}}),但它们是使用 \hyperlink{8101639384272933082}{\texttt{Float32}} 进行模拟实现的。




\begin{minted}{jlcon}
julia> sizeof(Float16(4.))
2

julia> 2*Float16(4.)
Float16(8.0)
\end{minted}



下划线 \texttt{\_} 可用作数字分隔符:




\begin{minted}{jlcon}
julia> 10_000, 0.000_000_005, 0xdead_beef, 0b1011_0010
(10000, 5.0e-9, 0xdeadbeef, 0xb2)
\end{minted}



\hypertarget{3917895508430327726}{}


\subsection{浮点数中的零}



浮点数有\href{https://zh.wikipedia.org/wiki/\%E2\%88\%920}{两个零},正零和负零。它们相互相等但有着不同的二进制表示,可以使用 \hyperlink{9171163989026657457}{\texttt{bitstring}} 函数来查看:




\begin{minted}{jlcon}
julia> 0.0 == -0.0
true

julia> bitstring(0.0)
"0000000000000000000000000000000000000000000000000000000000000000"

julia> bitstring(-0.0)
"1000000000000000000000000000000000000000000000000000000000000000"
\end{minted}



\hypertarget{16626704755049875766}{}


\subsection{特殊的浮点值}



有三种特定的标准浮点值不和实数轴上任何一点对应:


\begin{table}[h]
  \centering
  \tymin = 0.1\textwidth
\begin{tabulary}{\linewidth}{|L|L|L|L|L|}
  \hline
  \texttt{Float16} & \texttt{Float32} & \texttt{Float64} & 名称 & 描述 \\ \hline
  \texttt{Inf16} & \texttt{Inf32} & \texttt{Inf} & 正无穷 & 一个大于所有有限浮点数的数 \\ \hline
  \texttt{-Inf16} & \texttt{-Inf32} & \texttt{-Inf} & 负无穷 & 一个小于所有有限浮点数的数 \\ \hline
  \texttt{NaN16} & \texttt{NaN32} & \texttt{NaN} & 不是数\newline (Not a Number) & 一个不和任何浮点值(包括自己)相等(\texttt{==})的值 \\
  \hline
\end{tabulary}
\end{table}



% \begin{table}[h]

% \begin{tabulary}{\linewidth}{|L|L|L|L|L|}
% \hline
% \texttt{Float16} & \texttt{Float32} & \texttt{Float64} & 名称 & 描述 \\
% \hline
% \texttt{Inf16} & \texttt{Inf32} & \texttt{Inf} & 正无穷 & 一个大于所有有限浮点数的数 \\
% \hline
% \texttt{-Inf16} & \texttt{-Inf32} & \texttt{-Inf} & 负无穷 & 一个小于所有有限浮点数的数 \\
% \hline
% \texttt{NaN16} & \texttt{NaN32} & \texttt{NaN} & 不是数(Not a Number) & 一个不和任何浮点值(包括自己)相等(\texttt{==})的值 \\
% \hline
% \end{tabulary}

% \end{table}



对于这些非有限浮点值相互之间以及关于其它浮点值的顺序的更多讨论,请参见\hyperlink{7125151170457482788}{数值比较}。根据 \href{https://en.wikipedia.org/wiki/IEEE\_754\_revision}{IEEE 754 标准},这些浮点值是某些算术运算的结果:




\begin{minted}{jlcon}
julia> 1/Inf
0.0

julia> 1/0
Inf

julia> -5/0
-Inf

julia> 0.000001/0
Inf

julia> 0/0
NaN

julia> 500 + Inf
Inf

julia> 500 - Inf
-Inf

julia> Inf + Inf
Inf

julia> Inf - Inf
NaN

julia> Inf * Inf
Inf

julia> Inf / Inf
NaN

julia> 0 * Inf
NaN
\end{minted}



\hyperlink{3613894539247233488}{\texttt{typemin}} 和 \hyperlink{17760305803764597758}{\texttt{typemax}} 函数同样适用于浮点类型:




\begin{minted}{jlcon}
julia> (typemin(Float16),typemax(Float16))
(-Inf16, Inf16)

julia> (typemin(Float32),typemax(Float32))
(-Inf32, Inf32)

julia> (typemin(Float64),typemax(Float64))
(-Inf, Inf)
\end{minted}



\hypertarget{7614874233242990296}{}


\subsection{机器精度}



大多数实数都无法用浮点数准确地表示,因此有必要知道两个相邻可表示的浮点数间的距离。它通常被叫做\href{https://en.wikipedia.org/wiki/Machine\_epsilon}{机器精度}。



Julia 提供了 \hyperlink{4594213520310841636}{\texttt{eps}} 函数,它可以给出 \texttt{1.0} 与下一个 Julia 能表示的浮点数之间的差值:




\begin{minted}{jlcon}
julia> eps(Float32)
1.1920929f-7

julia> eps(Float64)
2.220446049250313e-16

julia> eps() # 与 eps(Float64) 相同
2.220446049250313e-16
\end{minted}



这些值分别是 \hyperlink{8101639384272933082}{\texttt{Float32}} 中的 \texttt{2.0{\textasciicircum}-23} 和 \hyperlink{5027751419500983000}{\texttt{Float64}} 中的 \texttt{2.0{\textasciicircum}-52}。\hyperlink{4594213520310841636}{\texttt{eps}} 函数也可以接受一个浮点值作为参数,然后给出这个值与下一个可表示的值直接的绝对差。也就是说,\texttt{eps(x)} 产生一个和 \texttt{x} 类型相同的值使得 \texttt{x + eps(x)} 是比 \texttt{x} 更大的下一个可表示的浮点值:




\begin{minted}{jlcon}
julia> eps(1.0)
2.220446049250313e-16

julia> eps(1000.)
1.1368683772161603e-13

julia> eps(1e-27)
1.793662034335766e-43

julia> eps(0.0)
5.0e-324
\end{minted}



两个相邻可表示的浮点数之间的距离并不是常数,数值越小,间距越小,数值越大,间距越大。换句话说,可表示的浮点数在实数轴上的零点附近最稠密,并沿着远离零点的方向以指数型的速度变得越来越稀疏。根据定义,\texttt{eps(1.0)} 与 \texttt{eps(Float64)} 相等,因为 \texttt{1.0} 是个 64 位浮点值。



Julia 也提供了 \hyperlink{8339500090035450608}{\texttt{nextfloat}} 和 \hyperlink{14035790731013288499}{\texttt{prevfloat}} 两个函数分别返回基于参数的下一个更大或更小的可表示的浮点数:




\begin{minted}{jlcon}
julia> x = 1.25f0
1.25f0

julia> nextfloat(x)
1.2500001f0

julia> prevfloat(x)
1.2499999f0

julia> bitstring(prevfloat(x))
"00111111100111111111111111111111"

julia> bitstring(x)
"00111111101000000000000000000000"

julia> bitstring(nextfloat(x))
"00111111101000000000000000000001"
\end{minted}



这个例子体现了一般原则,即相邻可表示的浮点数也有着相邻的二进制整数表示。



\hypertarget{5125794393878787955}{}


\subsection{舍入模式}



一个数如果没有精确的浮点表示,就必须被舍入到一个合适的可表示的值。然而,如果想的话,可以根据舍入模式改变舍入的方式,如 \href{https://en.wikipedia.org/wiki/IEEE\_754-2008}{IEEE 754 标准} 所述。 



Julia 所使用的默认模式总是 \hyperlink{868115654703135309}{\texttt{RoundNearest}},指舍入到最接近的可表示的值,这个被舍入的值会使用尽量少的有效位数。



\hypertarget{6428114724945706317}{}


\subsection{基础知识与参考文献}



浮点算术带来了很多微妙之处,它们可能对于那些不熟悉底层实现细节的用户会是很出人意料的。然而,这些微妙之处在大部分科学计算的书籍中以及以下的参考资料中都有详细介绍:



\begin{itemize}
\item 浮点数算术的权威指南是 \href{https://standards.ieee.org/standard/754-2008.html}{IEEE 754-2008 标准}; 然而这篇标准在网上无法免费获得。


\item 关于浮点数是如何表示的,想要一个简单而明白的介绍的话,可以看 John D. Cook 的\href{https://www.johndcook.com/blog/2009/04/06/anatomy-of-a-floating-point-number/}{文章}以及他关于从这种表示与实数理想的抽象化的差别中产生的一些问题的\href{https://www.johndcook.com/blog/2009/04/06/numbers-are-a-leaky-abstraction/}{介绍}


\item 同样推荐 Bruce Dawson 的\href{https://randomascii.wordpress.com/2012/05/20/thats-not-normalthe-performance-of-odd-floats}{一系列关于浮点数的博客文章}。


\item 想要一个对浮点数和使用浮点数计算时产生的数值精度问题的极好的、有深度的讨论,可以参见 David Goldberg 的文章 \href{http://citeseerx.ist.psu.edu/viewdoc/download?doi=10.1.1.22.6768\&rep=rep1\&type=pdf}{What Every Computer Scientist Should Know About Floating-Point Arithmetic}。


\item 更多延伸文档,包括浮点数的历史、基础理论、问题以及数值计算中很多其它主题的讨论,可以参见 \href{https://en.wikipedia.org/wiki/William\_Kahan}{William Kahan} 的\href{https://people.eecs.berkeley.edu/{\textasciitilde}wkahan/}{写作集}。他以“浮点数之父”闻名。特别感兴趣的话可以看 \href{https://people.eecs.berkeley.edu/{\textasciitilde}wkahan/ieee754status/754story.html}{An Interview with the Old Man of Floating-Point}。

\end{itemize}


\hypertarget{5997763766467609329}{}


\section{任意精度算术}



为了允许使用任意精度的整数与浮点数,Julia 分别包装了 \href{https://gmplib.org}{GNU Multiple Precision Arithmetic Library (GMP)} 以及 \href{https://www.mpfr.org}{GNU MPFR Library}。Julia 中的 \hyperlink{423405808990690832}{\texttt{BigInt}} 与 \hyperlink{749816618809421837}{\texttt{BigFloat}} 两种类型分别提供了任意精度的整数和浮点数。



Constructors exist to create these types from primitive numerical types, and the \href{@ref non-standard-string-literals}{string literal} \hyperlink{4226571565562941917}{\texttt{@big\_str}} or \hyperlink{14207407853646164654}{\texttt{parse}} can be used to construct them from \texttt{AbstractString}s.  Once created, they participate in arithmetic with all other numeric types thanks to Julia{\textquotesingle}s \hyperlink{10374023657104680331}{type promotion and conversion mechanism}:




\begin{minted}{jlcon}
julia> BigInt(typemax(Int64)) + 1
9223372036854775808

julia> big"123456789012345678901234567890" + 1
123456789012345678901234567891

julia> parse(BigInt, "123456789012345678901234567890") + 1
123456789012345678901234567891

julia> big"1.23456789012345678901"
1.234567890123456789010000000000000000000000000000000000000000000000000000000004

julia> parse(BigFloat, "1.23456789012345678901")
1.234567890123456789010000000000000000000000000000000000000000000000000000000004

julia> BigFloat(2.0^66) / 3
2.459565876494606882133333333333333333333333333333333333333333333333333333333344e+19

julia> factorial(BigInt(40))
815915283247897734345611269596115894272000000000
\end{minted}



然而,上面的原始类型与 \hyperlink{423405808990690832}{\texttt{BigInt}}/\hyperlink{749816618809421837}{\texttt{BigFloat}} 之间的类型提升并不是自动的,需要明确地指定:




\begin{minted}{jlcon}
julia> x = typemin(Int64)
-9223372036854775808

julia> x = x - 1
9223372036854775807

julia> typeof(x)
Int64

julia> y = BigInt(typemin(Int64))
-9223372036854775808

julia> y = y - 1
-9223372036854775809

julia> typeof(y)
BigInt
\end{minted}



\hyperlink{749816618809421837}{\texttt{BigFloat}} 的默认精度(有效数字的位数)和舍入模式可以通过调用 \hyperlink{3543074496498234209}{\texttt{setprecision}} 和 \hyperlink{12025922235501343815}{\texttt{setrounding}} 来全局地改变,所有之后的计算都会根据这些改变进行。还有一种方法,可以使用同样的函数以及 \texttt{do}-block 来只在运行一个特定代码块时改变精度和舍入模式:




\begin{minted}{jlcon}
julia> setrounding(BigFloat, RoundUp) do
           BigFloat(1) + parse(BigFloat, "0.1")
       end
1.100000000000000000000000000000000000000000000000000000000000000000000000000003

julia> setrounding(BigFloat, RoundDown) do
           BigFloat(1) + parse(BigFloat, "0.1")
       end
1.099999999999999999999999999999999999999999999999999999999999999999999999999986

julia> setprecision(40) do
           BigFloat(1) + parse(BigFloat, "0.1")
       end
1.1000000000004
\end{minted}



\hypertarget{14058350023597195643}{}


\section{数值字面量系数}



为了让常见的数值公式和表达式更清楚,Julia 允许变量直接跟在一个数值字面量后,暗指乘法。这可以让写多项式变得很清楚:




\begin{minted}{jlcon}
julia> x = 3
3

julia> 2x^2 - 3x + 1
10

julia> 1.5x^2 - .5x + 1
13.0
\end{minted}



也会让写指数函数变得更加优雅:




\begin{minted}{jlcon}
julia> 2^2x
64
\end{minted}



数值字面量系数的优先级跟一元运算符相同,比如说取相反数。所以 \texttt{2{\textasciicircum}3x} 会被解析成 \texttt{2{\textasciicircum}(3x)},而 \texttt{2x{\textasciicircum}3} 会被解析成 \texttt{2*(x{\textasciicircum}3)}。



数值字面量也能作为被括号表达式的系数:




\begin{minted}{jlcon}
julia> 2(x-1)^2 - 3(x-1) + 1
3
\end{minted}



\begin{quote}
\textbf{Note}

用于隐式乘法的数值字面量系数的优先级高于其它的二元运算符,例如乘法(\texttt{*})和除法(\texttt{/}、\texttt{{\textbackslash}} 以及 \texttt{//})。这意味着,比如说,\texttt{1 / 2im} 等于 \texttt{-0.5im} 以及 \texttt{6 // 2(2+1)} 等于 \texttt{1 // 1}。

\end{quote}


此外,括号表达式可以被用作变量的系数,暗指表达式与变量相乘:




\begin{minted}{jlcon}
julia> (x-1)x
6
\end{minted}



但是,无论是把两个括号表达式并列,还是把变量放在括号表达式之前,都不会被用作暗指乘法:




\begin{minted}{jlcon}
julia> (x-1)(x+1)
ERROR: MethodError: objects of type Int64 are not callable

julia> x(x+1)
ERROR: MethodError: objects of type Int64 are not callable
\end{minted}



这两种表达式都会被解释成函数调用:所有不是数值字面量的表达式,后面紧跟一个括号,就会被解释成使用括号内的值来调用函数(更多关于函数的信息请参见\hyperlink{16900494992832782120}{函数})。因此,在这两种情况中,都会因为左手边的值并不是函数而产生错误。



上述的语法糖显著地降低了在写普通数学公式时的视觉干扰。注意数值字面量系数和后面用来相乘的标识符或括号表达式之间不能有空格。



\hypertarget{5522643503764211248}{}


\subsection{语法冲突}



并列的字面量系数语法可能和两种数值字面量语法产生冲突:十六进制整数字面量以及浮点字面量的工程表示法。下面是几种会产生语法冲突的情况:



\begin{itemize}
\item 十六进制整数字面量 \texttt{0xff} 可能被解释成数值字面量 \texttt{0} 乘以变量 \texttt{xff}。


\item 浮点字面量表达式 \texttt{1e10} 可以被解释成数值字面量 \texttt{1} 乘以变量 \texttt{e10},与之等价的 \texttt{E}-表示法也存在类似的情况。


\item 32-bit 的浮点数字面量 \texttt{1.5f22} 被解释成数值字面量 \texttt{1.5} 乘以变量 \texttt{f22}。

\end{itemize}


在这些所有的情况中,都使用这样的解释方式来解决歧义:



\begin{itemize}
\item \texttt{0x} 开头的表达式总是十六进制字面量。


\item 数值开头跟着 \texttt{e} 和 \texttt{E} 的表达式总是浮点字面量。


\item 数值开头跟着 \texttt{f} 的表达式总是 32-bit 浮点字面量。

\end{itemize}


由于历史原因 \texttt{E} 和 \texttt{e} 在数值字面量上是等价的,与之不同的是,\texttt{F} 只是一个行为和 \texttt{f} 不同的字母。因此开头为 \texttt{F} 的表达式将会被 解析为一个数值字面量乘以一个变量,例如 \texttt{1.5F22}等价于 \texttt{1.5 * F22}。



\hypertarget{15171481192117197140}{}


\section{零和一的字面量}



Julia 提供了 0 和 1 的字面量函数,可以返回特定类型或所给变量的类型。




\begin{table}[h]

\begin{tabulary}{\linewidth}{|L|L|}
\hline
函数 & 描述 \\
\hline
\hyperlink{240596739242881814}{\texttt{zero(x)}} & \texttt{x} 类型或变量 \texttt{x} 的类型的零字面量 \\
\hline
\hyperlink{11395333326208453101}{\texttt{one(x)}} & \texttt{x} 类型或变量 \texttt{x} 的类型的一字面量 \\
\hline
\end{tabulary}

\end{table}



这些函数在\hyperlink{7125151170457482788}{数值比较}中可以用来避免不必要的\hyperlink{10374023657104680331}{类型转换}带来的开销。



例如:




\begin{minted}{jlcon}
julia> zero(Float32)
0.0f0

julia> zero(1.0)
0.0

julia> one(Int32)
1

julia> one(BigFloat)
1.0
\end{minted}



\hypertarget{7792257531330504645}{}


\chapter{数学运算和初等函数}



Julia 为它所有的基础数值类型,提供了整套的基础算术和位运算,也提供了一套高效、可移植的标准数学函数。



\hypertarget{11578520796345832337}{}


\section{算术运算符}



以下\href{https://en.wikipedia.org/wiki/Arithmetic\#Arithmetic\_operations}{算术运算符}支持所有的原始数值类型:




\begin{table}[h]

\begin{tabulary}{\linewidth}{|L|L|L|}
\hline
表达式 & 名称 & 描述 \\
\hline
\texttt{+x} & 一元加法运算符 & 全等操作 \\
\hline
\texttt{-x} & 一元减法运算符 & 将值变为其相反数 \\
\hline
\texttt{x + y} & 二元加法运算符 & 执行加法 \\
\hline
\texttt{x - y} & 二元减法运算符 & 执行减法 \\
\hline
\texttt{x * y} & 乘法运算符 & 执行乘法 \\
\hline
\texttt{x / y} & 除法运算符 & 执行除法 \\
\hline
\texttt{x ÷ y} & 整除 & 取 x / y 的整数部分 \\
\hline
\texttt{x {\textbackslash} y} & 反向除法 & 等价于 \texttt{y / x} \\
\hline
\texttt{x {\textasciicircum} y} & 幂操作符 & \texttt{x} 的 \texttt{y} 次幂 \\
\hline
\texttt{x \% y} & 取余 & 等价于 \texttt{rem(x,y)} \\
\hline
\end{tabulary}

\end{table}



以及对 \hyperlink{46725311238864537}{\texttt{Bool}} 类型的否定:




\begin{table}[h]

\begin{tabulary}{\linewidth}{|L|L|L|}
\hline
表达式 & 名称 & 描述 \\
\hline
\texttt{!x} & 否定 & 将 \texttt{true} 和 \texttt{false} 互换 \\
\hline
\end{tabulary}

\end{table}



除了优先级比二元操作符高以外,直接放在标识符或括号前的数字,如 \texttt{2x} 或 \texttt{2(x+y)} 还会被视为乘法。详见\hyperlink{7285052708387693199}{数值字面量系数}。



Julia 的类型提升系统使得混合参数类型上的代数运算也能顺其自然的工作,请参考\hyperlink{10374023657104680331}{类型提升系统}来了解更多内容。



这里是使用算术运算符的一些简单例子:




\begin{minted}{jlcon}
julia> 1 + 2 + 3
6

julia> 1 - 2
-1

julia> 3*2/12
0.5
\end{minted}



习惯上我们会把优先运算的操作符紧邻操作数,比如 \texttt{-x + 2} 表示先要给 \texttt{x}  取反,然后再加 \texttt{2} 。



When used in multiplication, \texttt{false} acts as a \emph{strong zero}:




\begin{minted}{jlcon}
julia> NaN * false
0.0

julia> false * Inf
0.0
\end{minted}



This is useful for preventing the propagation of \texttt{NaN} values in quantities that are known to be zero. See \href{https://arxiv.org/abs/math/9205211}{Knuth (1992)} for motivation.



\hypertarget{17155361622718020970}{}


\section{位运算符}



所有原始整数类型都支持以下\href{https://en.wikipedia.org/wiki/Bitwise\_operation\#Bitwise\_operators}{位运算符}:




\begin{table}[h]

\begin{tabulary}{\linewidth}{|L|L|}
\hline
表达式 & 名称 \\
\hline
\texttt{{\textasciitilde}x} & 按位取反 \\
\hline
\texttt{x \& y} & 按位与 \\
\hline
\texttt{x | y} & 按位或 \\
\hline
\texttt{x \unicodeveebar{} y} & 按位异或(逻辑异或) \\
\hline
\texttt{x >>> y} & \href{https://en.wikipedia.org/wiki/Logical\_shift}{逻辑右移} \\
\hline
\texttt{x >> y} & \href{https://en.wikipedia.org/wiki/Arithmetic\_shift}{算术右移} \\
\hline
\texttt{x << y} & 逻辑/算术左移 \\
\hline
\end{tabulary}

\end{table}



以下是位运算符的一些示例:




\begin{minted}[escapeinside=\#\%]{jlcon}
julia> ~123
-124

julia> 123 & 234
106

julia> 123 | 234
251

julia> 123 #\unicodeveebar% 234
145

julia> xor(123, 234)
145

julia> ~UInt32(123)
0xffffff84

julia> ~UInt8(123)
0x84
\end{minted}



\hypertarget{7921616576688360161}{}


\section{复合赋值操作符}



每一个二元运算符和位运算符都可以给左操作数复合赋值:方法是把 \texttt{=} 直接放在二元运算符后面。比如,\texttt{x += 3} 等价于 \texttt{x = x + 3} 。




\begin{minted}{jlcon}
julia> x = 1
1

julia> x += 3
4

julia> x
4
\end{minted}



二元运算和位运算的复合赋值操作符有下面几种:




\begin{lstlisting}[escapeinside=\%\%]
+=  -=  *=  /=  \=  ÷=  %\%%=  ^=  &=  |=  %\unicodeveebar%=  >>>=  >>=  <<=
\end{lstlisting}



\begin{quote}
\textbf{Note}

复合赋值后会把变量重新绑定到左操作数上,所以变量的类型可能会改变。


\begin{minted}{jlcon}
julia> x = 0x01; typeof(x)
UInt8

julia> x *= 2 # 与 x = x * 2 相同
2

julia> typeof(x)
Int64
\end{minted}

\end{quote}


\hypertarget{6173297391052343261}{}


\section{向量化 \texttt{dot} 运算符}



Julia 中,\textbf{每个}二元运算符都有一个 \texttt{dot} 运算符与之对应,例如 \texttt{{\textasciicircum}} 就有对应的 \texttt{.{\textasciicircum}} 存在。这个对应的 \texttt{.{\textasciicircum}} 被 Julia \textbf{自动地}定义为逐元素地执行 \texttt{{\textasciicircum}} 运算。比如 \texttt{[1,2,3] {\textasciicircum} 3} 是非法的,因为数学上没有给(长宽不一样的)数组的立方下过定义。但是 \texttt{[1,2,3] .{\textasciicircum} 3} 在 Julia 里是合法的,它会逐元素地执行 \texttt{{\textasciicircum}} 运算(或称向量化运算),得到 \texttt{[1{\textasciicircum}3, 2{\textasciicircum}3, 3{\textasciicircum}3]}。类似地,\texttt{!} 或 \texttt{√} 这样的一元运算符,也都有一个对应的 \texttt{.√} 用于执行逐元素运算。




\begin{minted}{jlcon}
julia> [1,2,3] .^ 3
3-element Array{Int64,1}:
  1
  8
 27
\end{minted}



具体来说,\texttt{a .{\textasciicircum} b} 被解析为 \hyperlink{17801130558550430478}{\texttt{dot} 调用} \texttt{({\textasciicircum}).(a,b)},这会执行 \href{@ref Broadcasting}{broadcast} 操作:该操作能结合数组和标量、相同大小的数组(元素之间的运算)、甚至不同形状的数组(例如行、列向量结合生成矩阵)。更进一步,就像所有向量化的 \texttt{dot} 调用一样,这些 \texttt{dot} 运算符是\textbf{融合}的(fused)。例如,在计算表达式 \texttt{2 .* A.{\textasciicircum}2 .+ sin.(A)} 时,Julia 只对 \texttt{A} 进行做\textbf{一次}循环,遍历 \texttt{A} 中的每个元素 a 并计算 \texttt{2a{\textasciicircum}2 + sin(a)}。上述表达式也可以用\hyperlink{16688502228717894452}{\texttt{@.}} 宏简写为 \texttt{@. 2A{\textasciicircum}2 + sin(A)}。特别的,类似 \texttt{f.(g.(x))} 的嵌套 \texttt{dot} 调用也是\textbf{融合}的,并且“相邻的”二元运算符表达式 \texttt{x .+ 3 .* x.{\textasciicircum}2} 可以等价转换为嵌套 \texttt{dot} 调用:\texttt{(+).(x, (*).(3, ({\textasciicircum}).(x, 2)))}。



除了 \texttt{dot} 运算符,我们还有 \texttt{dot} 复合赋值运算符,类似 \texttt{a .+= b}(或者 \texttt{@. a += b})会被解析成 \texttt{a .= a .+ b},这里的 \texttt{.=} 是一个\textbf{融合}的 in-place 运算,更多信息请查看 \hyperlink{17801130558550430478}{\texttt{dot} 文档})。



这个点语法,也能用在用户自定义的运算符上。例如,通过定义 \texttt{⊗(A,B) = kron(A,B)} 可以为 Kronecker 积(\hyperlink{14153417388267953812}{\texttt{kron}})提供一个方便的中缀语法 \texttt{A ⊗ B},那么配合点语法 \texttt{[A,B] .⊗ [C,D]} 就等价于 \texttt{[A⊗C, B⊗D]}。



将点运算符用于数值字面量可能会导致歧义。例如,\texttt{1.+x} 到底是表示 \texttt{1. + x} 还是 \texttt{1 .+ x}?这会令人疑惑。因此不允许使用这种语法,遇到这种情况时,必须明确地用空格消除歧义。



\hypertarget{2028216132575181376}{}


\section{数值比较}



标准的比较操作对所有原始数值类型有定义:




\begin{table}[h]

\begin{tabulary}{\linewidth}{|L|L|}
\hline
操作符 & 名称 \\
\hline
\hyperlink{15143149452920304570}{\texttt{==}} & 相等 \\
\hline
\hyperlink{3046079188653285114}{\texttt{!=}}, \hyperlink{3046079188653285114}{\texttt{≠}} & 不等 \\
\hline
\hyperlink{702782232449268329}{\texttt{<}} & 小于 \\
\hline
\hyperlink{11411050964021316526}{\texttt{<=}}, \hyperlink{11411050964021316526}{\texttt{≤}} & 小于等于 \\
\hline
\hyperlink{8677991761303191103}{\texttt{>}} & 大于 \\
\hline
\hyperlink{7019639580556993898}{\texttt{>=}}, \hyperlink{7019639580556993898}{\texttt{≥}} & 大于等于 \\
\hline
\end{tabulary}

\end{table}



下面是一些简单的例子:




\begin{minted}{jlcon}
julia> 1 == 1
true

julia> 1 == 2
false

julia> 1 != 2
true

julia> 1 == 1.0
true

julia> 1 < 2
true

julia> 1.0 > 3
false

julia> 1 >= 1.0
true

julia> -1 <= 1
true

julia> -1 <= -1
true

julia> -1 <= -2
false

julia> 3 < -0.5
false
\end{minted}



整数的比较方式是标准的按位比较,而浮点数的比较方式则遵循 \href{https://en.wikipedia.org/wiki/IEEE\_754-2008}{IEEE 754 标准}。



\begin{itemize}
\item 有限数的大小顺序,和我们所熟知的相同。


\item \texttt{+0} 等于但不大于 \texttt{-0}.


\item \texttt{Inf} 等于自身,并且大于除了 \texttt{NaN} 外的所有数。


\item \texttt{-Inf} is equal to itself and less than everything else except \texttt{NaN}.


\item \texttt{NaN} 不等于、不小于且不大于任何数值,包括它自己。

\end{itemize}


\texttt{NaN} 不等于它自己这一点可能会令人感到惊奇,所以需要注意:




\begin{minted}{jlcon}
julia> NaN == NaN
false

julia> NaN != NaN
true

julia> NaN < NaN
false

julia> NaN > NaN
false
\end{minted}



当你将 \texttt{NaN} 和 \hyperlink{16720099245556932994}{数组} 一起连用时,你就会感到头疼:




\begin{minted}{jlcon}
julia> [1 NaN] == [1 NaN]
false
\end{minted}



为此,Julia 给这些特别的数提供了下面几个额外的测试函数。这些函数在某些情况下很有用处,比如在做 hash 比较时。




\begin{table}[h]

\begin{tabulary}{\linewidth}{|L|L|}
\hline
函数 & 测试是否满足如下性质 \\
\hline
\hyperlink{269533589463185031}{\texttt{isequal(x, y)}} & \texttt{x} 与 \texttt{y} 是完全相同的 \\
\hline
\hyperlink{2906021895910968108}{\texttt{isfinite(x)}} & \texttt{x} 是有限大的数字 \\
\hline
\hyperlink{4492113908831448207}{\texttt{isinf(x)}} & \texttt{x} 是(正/负)无穷大 \\
\hline
\hyperlink{6770390199496851634}{\texttt{isnan(x)}} & \texttt{x} 是 \texttt{NaN} \\
\hline
\end{tabulary}

\end{table}



\hyperlink{269533589463185031}{\texttt{isequal}} 认为 \texttt{NaN} 之间是相等的:




\begin{minted}{jlcon}
julia> isequal(NaN, NaN)
true

julia> isequal([1 NaN], [1 NaN])
true

julia> isequal(NaN, NaN32)
true
\end{minted}



\texttt{isequal} 也能用来区分带符号的零:




\begin{minted}{jlcon}
julia> -0.0 == 0.0
true

julia> isequal(-0.0, 0.0)
false
\end{minted}



有符号整数、无符号整数以及浮点数之间的混合类型比较是很棘手的。开发者费了很大精力来确保 Julia 在这个问题上做的是正确的。



对于其它类型,\texttt{isequal} 会默认调用 \hyperlink{15143149452920304570}{\texttt{==}},所以如果你想给自己的类型定义相等,那么就只需要为 \hyperlink{15143149452920304570}{\texttt{==}} 增加一个方法。如果你想定义一个你自己的相等函数,你可能需要定义一个对应的 \hyperlink{13797072367283572032}{\texttt{hash}} 方法,用于确保 \texttt{isequal(x,y)} 隐含着 \texttt{hash(x) == hash(y)}。



\hypertarget{9107485161550737856}{}


\subsection{链式比较}



与其他多数语言不同,就像 \href{https://en.wikipedia.org/wiki/Python\_syntax\_and\_semantics\#Comparison\_operators}{notable exception of Python} 一样,Julia 允许链式比较:




\begin{minted}{jlcon}
julia> 1 < 2 <= 2 < 3 == 3 > 2 >= 1 == 1 < 3 != 5
true
\end{minted}



链式比较在写数值代码时特别方便,它使用 \texttt{\&\&} 运算符比较标量,数组则用 \hyperlink{1494761116451616317}{\texttt{\&}} 进行按元素比较。比如,\texttt{0 .< A .< 1} 会得到一个 boolean 数组,如果 \texttt{A} 的元素都在 0 和 1 之间则数组元素就都是 true。



注意链式比较的执行顺序:




\begin{minted}{jlcon}
julia> v(x) = (println(x); x)
v (generic function with 1 method)

julia> v(1) < v(2) <= v(3)
2
1
3
true

julia> v(1) > v(2) <= v(3)
2
1
false
\end{minted}



中间的表达式只会计算一次,而如果写成 \texttt{v(1) < v(2) \&\& v(2) <= v(3)} 是计算了两次的。然而,链式比较中的顺序是不确定的。强烈建议不要在表达式中使用有副作用(比如 printing)的函数。如果的确需要,请使用短路运算符 \texttt{\&\&}(请参考\hyperlink{7551496361738057869}{短路求值})。



\hypertarget{6116621209452494602}{}


\subsection{初等函数}



Julia 提供了强大的数学函数和运算符集合。这些数学运算定义在各种合理的数值上,包括整型、浮点数、分数和复数,只要这些定义有数学意义就行。



而且,和其它 Julia 函数一样,这些函数也能通过 \hyperlink{17801130558550430478}{点语法} \texttt{f.(A)} 以“向量化”的方式作用于数组和其它集合上。 比如,\texttt{sin.(A)} 会计算 \texttt{A} 中每个元素的 sin 值。



\hypertarget{1086128937891391302}{}


\section{运算符的优先级与结合性}



从高到低,Julia 运算符的优先级与结合性为:




\begin{table}[h]

\begin{tabulary}{\linewidth}{|L|L|L|}
\hline
分类 & 运算符 & 结合性 \\
\hline
语法 & \texttt{.} followed by \texttt{::} & 左结合 \\
\hline
幂运算 & \texttt{{\textasciicircum}} & 右结合 \\
\hline
一元运算符 & \texttt{+ - √} & 右结合\footnotemark[1] \\
\hline
移位运算 & \texttt{<< >> >>>} & 左结合 \\
\hline
除法 & \texttt{//} & 左结合 \\
\hline
乘法 & \texttt{* / \% \& {\textbackslash} ÷} & 左结合\footnotemark[2] \\
\hline
加法 & \texttt{+ - | \unicodeveebar{}} & 左结合\footnotemark[2] \\
\hline
语法 & \texttt{: ..} & 左结合 \\
\hline
语法 & \texttt{|>} & 左结合 \\
\hline
语法 & \texttt{<|} & 右结合 \\
\hline
比较 & \texttt{> < >= <= == === != !== <:} & 无结合性 \\
\hline
流程控制 & \texttt{\&\&} followed by \texttt{||} followed by \texttt{?} & 右结合 \\
\hline
Pair 操作 & \texttt{=>} & 右结合 \\
\hline
赋值 & \texttt{= += -= *= /= //= {\textbackslash}= {\textasciicircum}= ÷= \%= |= \&= \unicodeveebar{}= <<= >>= >>>=} & 右结合 \\
\hline
\end{tabulary}

\end{table}



\footnotetext[1]{一元运算符 \texttt{+} 和 \texttt{-} 需要显式调用,即给它们的参数加上括号,以免和 \texttt{++} 等运算符混淆。其它一元运算符的混合使用都被解析为右结合的,比如 \texttt{√√-a} 解析为 \texttt{√(√(-a))}。

}


\footnotetext[2]{The operators \texttt{+}, \texttt{++} and \texttt{*} are non-associative. \texttt{a + b + c} is parsed as \texttt{+(a, b, c)} not \texttt{+(+(a, b), c)}. However, the fallback methods for \texttt{+(a, b, c, d...)} and \texttt{*(a, b, c, d...)} both default to left-associative evaluation.

}


要看\textbf{全部} Julia 运算符的优先级关系,可以看这个文件的最上面部分:\href{https://github.com/JuliaLang/julia/blob/master/src/julia-parser.scm}{\texttt{src/julia-parser.scm}}



\hyperlink{7285052708387693199}{数字字面量系数},例如 \texttt{2x} 中的 2,它的优先级比二元运算符高,因此会当作乘法,并且它的优先级也比 \texttt{{\textasciicircum}} 高。



你也可以通过内置函数 \texttt{Base.operator\_precedence} 查看任何给定运算符的优先级数值,数值越大优先级越高:




\begin{minted}{jlcon}
julia> Base.operator_precedence(:+), Base.operator_precedence(:*), Base.operator_precedence(:.)
(11, 12, 17)

julia> Base.operator_precedence(:sin), Base.operator_precedence(:+=), Base.operator_precedence(:(=))  # (Note the necessary parens on `:(=)`)
(0, 1, 1)
\end{minted}



另外,内置函数 \texttt{Base.operator\_associativity} 可以返回运算符结合性的符号表示:




\begin{minted}{jlcon}
julia> Base.operator_associativity(:-), Base.operator_associativity(:+), Base.operator_associativity(:^)
(:left, :none, :right)

julia> Base.operator_associativity(:⊗), Base.operator_associativity(:sin), Base.operator_associativity(:→)
(:left, :none, :right)
\end{minted}



注意诸如 \texttt{:sin} 这样的符号返回优先级 \texttt{0},此值代表无效的运算符或非最低优先级运算符。类似地,它们的结合性被认为是 \texttt{:none}。



\hypertarget{1678218620254251806}{}


\section{数值转换}



Julia 支持三种数值转换,它们在处理不精确转换上有所不同。



\begin{itemize}
\item \texttt{T(x)} 和 \texttt{convert(T,x)} 都会把 \texttt{x} 转换为 \texttt{T}类型。

\begin{itemize}
\item 如果 \texttt{T} 是浮点类型,转换的结果就是最近的可表示值, 可能会是正负无穷大。


\item 如果 \texttt{T} 为整数类型,当 \texttt{x} 不能由 \texttt{T} 类型表示时,会抛出 \texttt{InexactError}。

\end{itemize}

\item \texttt{x \% T} 将整数 \texttt{x} 转换为整型 \texttt{T},与 \texttt{x} 模 \texttt{2{\textasciicircum}n} 的结果一致,其中 \texttt{n} 是 \texttt{T} 的位数。换句话说,在二进制表示下被截掉了一部分。


\item \hyperlink{9997236062216946610}{舍入函数} 接收一个 \texttt{T} 类型的可选参数。比如,\texttt{round(Int,x)} 是 \texttt{Int(round(x))} 的简写版。

\end{itemize}


下面的例子展示了不同的形式




\begin{minted}{jlcon}
julia> Int8(127)
127

julia> Int8(128)
ERROR: InexactError: trunc(Int8, 128)
Stacktrace:
[...]

julia> Int8(127.0)
127

julia> Int8(3.14)
ERROR: InexactError: Int8(3.14)
Stacktrace:
[...]

julia> Int8(128.0)
ERROR: InexactError: Int8(128.0)
Stacktrace:
[...]

julia> 127 % Int8
127

julia> 128 % Int8
-128

julia> round(Int8,127.4)
127

julia> round(Int8,127.6)
ERROR: InexactError: trunc(Int8, 128.0)
Stacktrace:
[...]
\end{minted}



请参考\hyperlink{10374023657104680331}{类型转换与类型提升}一节来定义你自己的类型转换和提升规则。



\hypertarget{10733784297691347404}{}


\subsection{舍入函数}




\begin{table}[h]

\begin{tabulary}{\linewidth}{|L|L|L|}
\hline
函数 & 描述 & 返回类型 \\
\hline
\hyperlink{12930779325193350739}{\texttt{round(x)}} & \texttt{x} 舍到最接近的整数 & \texttt{typeof(x)} \\
\hline
\hyperlink{12930779325193350739}{\texttt{round(T, x)}} & \texttt{x} 舍到最接近的整数 & \texttt{T} \\
\hline
\hyperlink{11115257331910840693}{\texttt{floor(x)}} & \texttt{x} 向 \texttt{-Inf} 舍入 & \texttt{typeof(x)} \\
\hline
\hyperlink{11115257331910840693}{\texttt{floor(T, x)}} & \texttt{x} 向 \texttt{-Inf} 舍入 & \texttt{T} \\
\hline
\hyperlink{10519509038312853061}{\texttt{ceil(x)}} & \texttt{x} 向 \texttt{+Inf} 方向取整 & \texttt{typeof(x)} \\
\hline
\hyperlink{10519509038312853061}{\texttt{ceil(T, x)}} & \texttt{x} 向 \texttt{+Inf} 方向取整 & \texttt{T} \\
\hline
\hyperlink{1728363361565303194}{\texttt{trunc(x)}} & \texttt{x} 向 0 取整 & \texttt{typeof(x)} \\
\hline
\hyperlink{1728363361565303194}{\texttt{trunc(T, x)}} & \texttt{x} 向 0 取整 & \texttt{T} \\
\hline
\end{tabulary}

\end{table}



\hypertarget{3613448754755213273}{}


\subsection{除法函数}




\begin{table}[h]

\begin{tabulary}{\linewidth}{|L|L|}
\hline
函数 & 描述 \\
\hline
\hyperlink{8020976424566491334}{\texttt{div(x,y)}}, \texttt{x÷y} & 截断除法;商向零近似 \\
\hline
\hyperlink{15067916827074788527}{\texttt{fld(x,y)}} & 向下取整除法;商向 \texttt{-Inf} 近似 \\
\hline
\hyperlink{7922388465305816555}{\texttt{cld(x,y)}} & 向上取整除法;商向 \texttt{+Inf} 近似 \\
\hline
\hyperlink{3827563084771191385}{\texttt{rem(x,y)}} & 取余;满足 \texttt{x == div(x,y)*y + rem(x,y)};符号与 \texttt{x} 一致 \\
\hline
\hyperlink{2082041235715276573}{\texttt{mod(x,y)}} & 取模;满足 \texttt{x == fld(x,y)*y + mod(x,y)};符号与 \texttt{y} 一致 \\
\hline
\hyperlink{13778479217547823795}{\texttt{mod1(x,y)}} & 偏移 1 的 \texttt{mod};若 \texttt{y>0},则返回 \texttt{r∈(0,y]},若 \texttt{y<0},则 \texttt{r∈[y,0)} 且满足 \texttt{mod(r, y) == mod(x, y)} \\
\hline
\hyperlink{15322754370885673769}{\texttt{mod2pi(x)}} & 以 2pi 为基取模;\texttt{0 <= mod2pi(x) < 2pi} \\
\hline
\hyperlink{6106909621813654214}{\texttt{divrem(x,y)}} & 返回 \texttt{(div(x,y),rem(x,y))} \\
\hline
\hyperlink{2806360720034558325}{\texttt{fldmod(x,y)}} & 返回 \texttt{(fld(x,y),mod(x,y))} \\
\hline
\hyperlink{15906911311436241979}{\texttt{gcd(x,y...)}} & \texttt{x}, \texttt{y},... 的最大公约数 \\
\hline
\hyperlink{12975400110924105221}{\texttt{lcm(x,y...)}} & \texttt{x}, \texttt{y},... 的最小公倍数 \\
\hline
\end{tabulary}

\end{table}



\hypertarget{1398763230003382412}{}


\subsection{符号和绝对值函数}




\begin{table}[h]

\begin{tabulary}{\linewidth}{|L|L|}
\hline
函数 & 描述 \\
\hline
\hyperlink{9614495866226399167}{\texttt{abs(x)}} & \texttt{x} 的模 \\
\hline
\hyperlink{15686257922156163743}{\texttt{abs2(x)}} & \texttt{x} 的模的平方 \\
\hline
\hyperlink{14349105033929355161}{\texttt{sign(x)}} & 表示 \texttt{x} 的符号,返回 -1,0,或 +1 \\
\hline
\hyperlink{9457038569823603490}{\texttt{signbit(x)}} & 表示符号位是 true 或 false \\
\hline
\hyperlink{6024566200716053110}{\texttt{copysign(x,y)}} & 返回一个数,其值等于 \texttt{x} 的模,符号与 \texttt{y} 一致 \\
\hline
\hyperlink{2689022981470151558}{\texttt{flipsign(x,y)}} & 返回一个数,其值等于 \texttt{x} 的模,符号与 \texttt{x*y} 一致 \\
\hline
\end{tabulary}

\end{table}



\hypertarget{15750140405864720482}{}


\subsection{幂、对数与平方根}




\begin{table}[h]

\begin{tabulary}{\linewidth}{|L|L|}
\hline
函数 & 描述 \\
\hline
\hyperlink{4551113327515323898}{\texttt{sqrt(x)}}, \texttt{√x} & \texttt{x} 的平方根 \\
\hline
\hyperlink{15104025502404840355}{\texttt{cbrt(x)}}, \texttt{∛x} & \texttt{x} 的立方根 \\
\hline
\hyperlink{18304489571285447949}{\texttt{hypot(x,y)}} & 当直角边的长度为 \texttt{x} 和 \texttt{y}时,直角三角形斜边的长度 \\
\hline
\hyperlink{5801729597955756107}{\texttt{exp(x)}} & 自然指数函数在 \texttt{x} 处的值 \\
\hline
\hyperlink{4939309737829480377}{\texttt{expm1(x)}} & 当 \texttt{x} 接近 0 时的 \texttt{exp(x)-1} 的精确值 \\
\hline
\hyperlink{14721177606508229464}{\texttt{ldexp(x,n)}} & \texttt{x*2{\textasciicircum}n} 的高效算法,\texttt{n} 为整数 \\
\hline
\hyperlink{17317607370922767936}{\texttt{log(x)}} & \texttt{x} 的自然对数 \\
\hline
\hyperlink{17317607370922767936}{\texttt{log(b,x)}} & 以 \texttt{b} 为底 \texttt{x} 的对数 \\
\hline
\hyperlink{18341149201477905713}{\texttt{log2(x)}} & 以 2 为底 \texttt{x} 的对数 \\
\hline
\hyperlink{3481560771470480868}{\texttt{log10(x)}} & 以 10 为底 \texttt{x} 的对数 \\
\hline
\hyperlink{5533050447473188877}{\texttt{log1p(x)}} & 当 \texttt{x}接近 0 时的 \texttt{log(1+x)} 的精确值 \\
\hline
\hyperlink{39736318364195845}{\texttt{exponent(x)}} & \texttt{x} 的二进制指数 \\
\hline
\hyperlink{11312242195671521747}{\texttt{significand(x)}} & 浮点数 \texttt{x} 的二进制有效数(也就是尾数) \\
\hline
\end{tabulary}

\end{table}



想大概了解一下为什么诸如 \hyperlink{18304489571285447949}{\texttt{hypot}}、\hyperlink{4939309737829480377}{\texttt{expm1}}和 \hyperlink{5533050447473188877}{\texttt{log1p}} 函数是必要和有用的,可以看一下 John D. Cook 关于这些主题的两篇优秀博文:\href{https://www.johndcook.com/blog/2010/06/07/math-library-functions-that-seem-unnecessary/}{expm1, log1p, erfc}, 和 \href{https://www.johndcook.com/blog/2010/06/02/whats-so-hard-about-finding-a-hypotenuse/}{hypot}。



\hypertarget{16706884854236336909}{}


\subsection{三角和双曲函数}



所有标准的三角函数和双曲函数也都已经定义了:




\begin{lstlisting}
sin    cos    tan    cot    sec    csc
sinh   cosh   tanh   coth   sech   csch
asin   acos   atan   acot   asec   acsc
asinh  acosh  atanh  acoth  asech  acsch
sinc   cosc
\end{lstlisting}



所有这些函数都是单参数函数,不过 \hyperlink{16445804261034090556}{\texttt{atan}} 也可以接收两个参数 来表示传统的 \href{https://en.wikipedia.org/wiki/Atan2}{\texttt{atan2}} 函数。



另外,\hyperlink{16554510911661822298}{\texttt{sinpi(x)}} 和 \hyperlink{2974547424856180253}{\texttt{cospi(x)}} 分别用来对 \hyperlink{10540279982054240733}{\texttt{sin(pi*x)}} 和 \hyperlink{10355926621556840804}{\texttt{cos(pi*x)}} 进行更精确的计算。



要计算角度而非弧度的三角函数,以 \texttt{d} 做后缀。 比如,\hyperlink{38337471195460170}{\texttt{sind(x)}} 计算 \texttt{x} 的 sine 值,其中 \texttt{x} 是一个角度值。 下面是角度变量的三角函数完整列表:




\begin{lstlisting}
sind   cosd   tand   cotd   secd   cscd
asind  acosd  atand  acotd  asecd  acscd
\end{lstlisting}



\hypertarget{17986622034630654775}{}


\subsection{特殊函数}



\href{https://github.com/JuliaMath/SpecialFunctions.jl}{SpecialFunctions.jl} 提供了许多其他的特殊数学函数。



\hypertarget{9739813100592614250}{}


\chapter{复数和有理数}



Julia 语言包含了预定义的复数和有理数类型,并且支持它们的各种标准\hyperlink{16865688524696028421}{数学运算和初等函数}。由于也定义了复数与分数的\hyperlink{10374023657104680331}{类型转换与类型提升},因此对预定义数值类型(无论是原始的还是复合的)的任意组合进行的操作都会表现得如预期的一样。



\hypertarget{5868123017618904517}{}


\section{复数}



在Julia中,全局常量 \hyperlink{15097910740298861288}{\texttt{im}} 被绑定到复数 \emph{i},表示 -1 的主平方根(不应使用数学家习惯的 \texttt{i} 或工程师习惯的 \texttt{j} 来表示此全局常量,因为它们是非常常用的索引变量名)。由于 Julia 允许数值字面量\hyperlink{7285052708387693199}{作为系数与标识符并置},这种绑定就足够为复数提供很方便的语法,类似于传统的数学记法:




\begin{minted}{jlcon}
julia> 1+2im
1 + 2im
\end{minted}



你可以对复数进行各种标准算术操作:




\begin{minted}{jlcon}
julia> (1 + 2im)*(2 - 3im)
8 + 1im

julia> (1 + 2im)/(1 - 2im)
-0.6 + 0.8im

julia> (1 + 2im) + (1 - 2im)
2 + 0im

julia> (-3 + 2im) - (5 - 1im)
-8 + 3im

julia> (-1 + 2im)^2
-3 - 4im

julia> (-1 + 2im)^2.5
2.729624464784009 - 6.9606644595719im

julia> (-1 + 2im)^(1 + 1im)
-0.27910381075826657 + 0.08708053414102428im

julia> 3(2 - 5im)
6 - 15im

julia> 3(2 - 5im)^2
-63 - 60im

julia> 3(2 - 5im)^-1.0
0.20689655172413796 + 0.5172413793103449im
\end{minted}



类型提升机制也确保你可以使用不同类型的操作数的组合:




\begin{minted}{jlcon}
julia> 2(1 - 1im)
2 - 2im

julia> (2 + 3im) - 1
1 + 3im

julia> (1 + 2im) + 0.5
1.5 + 2.0im

julia> (2 + 3im) - 0.5im
2.0 + 2.5im

julia> 0.75(1 + 2im)
0.75 + 1.5im

julia> (2 + 3im) / 2
1.0 + 1.5im

julia> (1 - 3im) / (2 + 2im)
-0.5 - 1.0im

julia> 2im^2
-2 + 0im

julia> 1 + 3/4im
1.0 - 0.75im
\end{minted}



注意 \texttt{3/4im == 3/(4*im) == -(3/4*im)},因为系数比除法的优先级更高。



Julia 提供了一些操作复数的标准函数:




\begin{minted}{jlcon}
julia> z = 1 + 2im
1 + 2im

julia> real(1 + 2im) # z 的实部
1

julia> imag(1 + 2im) # z 的虚部
2

julia> conj(1 + 2im) # z 的复共轭
1 - 2im

julia> abs(1 + 2im) # z 的绝对值
2.23606797749979

julia> abs2(1 + 2im) # 取平方后的绝对值
5

julia> angle(1 + 2im) # 以弧度为单位的相位角
1.1071487177940904
\end{minted}



按照惯例,复数的绝对值(\hyperlink{9614495866226399167}{\texttt{abs}})是从零点到它的距离。\hyperlink{15686257922156163743}{\texttt{abs2}} 给出绝对值的平方,作用于复数上时非常有用,因为它避免了取平方根。\hyperlink{9465547375318501186}{\texttt{angle}} 返回以弧度为单位的相位角(也被称为辐角函数)。所有其它的\hyperlink{7194628046456539104}{初等函数}在复数上也都有完整的定义:




\begin{minted}{jlcon}
julia> sqrt(1im)
0.7071067811865476 + 0.7071067811865475im

julia> sqrt(1 + 2im)
1.272019649514069 + 0.7861513777574233im

julia> cos(1 + 2im)
2.0327230070196656 - 3.0518977991518im

julia> exp(1 + 2im)
-1.1312043837568135 + 2.4717266720048188im

julia> sinh(1 + 2im)
-0.4890562590412937 + 1.4031192506220405im
\end{minted}



注意数学函数通常应用于实数就返回实数值,应用于复数就返回复数值。例如,当 \hyperlink{4551113327515323898}{\texttt{sqrt}} 应用于 \texttt{-1} 与 \texttt{-1 + 0im} 会有不同的表现,虽然 \texttt{-1 == -1 + 0im}:




\begin{minted}{jlcon}
julia> sqrt(-1)
ERROR: DomainError with -1.0:
sqrt will only return a complex result if called with a complex argument. Try sqrt(Complex(x)).
Stacktrace:
[...]

julia> sqrt(-1 + 0im)
0.0 + 1.0im
\end{minted}



从变量构建复数时,\hyperlink{7285052708387693199}{文本型数值系数记法}不再适用。相反地,乘法必须显式地写出:




\begin{minted}{jlcon}
julia> a = 1; b = 2; a + b*im
1 + 2im
\end{minted}



然而,我们\textbf{并不}推荐这样做,而应改为使用更高效的 \hyperlink{16014240202095271744}{\texttt{complex}} 函数直接通过实部与虚部构建一个复数值:




\begin{minted}{jlcon}
julia> a = 1; b = 2; complex(a, b)
1 + 2im
\end{minted}



这种构建避免了乘法和加法操作。



\hyperlink{1907914141659611007}{\texttt{Inf}} 和 \hyperlink{11449618129446476597}{\texttt{NaN}} 可能出现在复数的实部和虚部,正如\hyperlink{17731750208832839264}{特殊的浮点值}章节所描述的:




\begin{minted}{jlcon}
julia> 1 + Inf*im
1.0 + Inf*im

julia> 1 + NaN*im
1.0 + NaN*im
\end{minted}



\hypertarget{8440700082217486421}{}


\section{有理数}



Julia 有一个用于表示整数精确比值的分数类型。分数通过 \hyperlink{17539582191808611917}{\texttt{//}} 运算符构建:




\begin{minted}{jlcon}
julia> 2//3
2//3
\end{minted}



如果一个分数的分子和分母含有公因子,它们会被约分到最简形式且分母非负:




\begin{minted}{jlcon}
julia> 6//9
2//3

julia> -4//8
-1//2

julia> 5//-15
-1//3

julia> -4//-12
1//3
\end{minted}



整数比值的这种标准化形式是唯一的,所以分数值的相等性可由校验分子与分母都相等来测试。分数值的标准化分子和分母可以使用 \hyperlink{7885506453580572157}{\texttt{numerator}} 和 \hyperlink{12407209279719593434}{\texttt{denominator}} 函数得到:




\begin{minted}{jlcon}
julia> numerator(2//3)
2

julia> denominator(2//3)
3
\end{minted}



分子和分母的直接比较通常是不必要的,因为标准算术和比较操作对分数值也有定义:




\begin{minted}{jlcon}
julia> 2//3 == 6//9
true

julia> 2//3 == 9//27
false

julia> 3//7 < 1//2
true

julia> 3//4 > 2//3
true

julia> 2//4 + 1//6
2//3

julia> 5//12 - 1//4
1//6

julia> 5//8 * 3//12
5//32

julia> 6//5 / 10//7
21//25
\end{minted}



分数可以很容易地转换成浮点数:




\begin{minted}{jlcon}
julia> float(3//4)
0.75
\end{minted}



对任意整数值 \texttt{a} 和 \texttt{b}(除了 \texttt{a == 0} 且 \texttt{b == 0} 时),从分数到浮点数的转换遵从以下的一致性:




\begin{minted}{jlcon}
julia> a = 1; b = 2;

julia> isequal(float(a//b), a/b)
true
\end{minted}



Julia接受构建无穷分数值:




\begin{minted}{jlcon}
julia> 5//0
1//0

julia> -3//0
-1//0

julia> typeof(ans)
Rational{Int64}
\end{minted}



但不接受试图构建一个 \hyperlink{11449618129446476597}{\texttt{NaN}} 分数值:




\begin{minted}{jlcon}
julia> 0//0
ERROR: ArgumentError: invalid rational: zero(Int64)//zero(Int64)
Stacktrace:
[...]
\end{minted}



像往常一样,类型提升系统使得分数可以轻松地同其它数值类型进行交互:




\begin{minted}{jlcon}
julia> 3//5 + 1
8//5

julia> 3//5 - 0.5
0.09999999999999998

julia> 2//7 * (1 + 2im)
2//7 + 4//7*im

julia> 2//7 * (1.5 + 2im)
0.42857142857142855 + 0.5714285714285714im

julia> 3//2 / (1 + 2im)
3//10 - 3//5*im

julia> 1//2 + 2im
1//2 + 2//1*im

julia> 1 + 2//3im
1//1 - 2//3*im

julia> 0.5 == 1//2
true

julia> 0.33 == 1//3
false

julia> 0.33 < 1//3
true

julia> 1//3 - 0.33
0.0033333333333332993
\end{minted}



\hypertarget{3772396547767597421}{}


\chapter{字符串}



字符串是由有限个字符组成的序列。而字符在英文中一般包括字母 \texttt{A},\texttt{B}, \texttt{C} 等、数字和常用的标点符号。这些字符由 \href{https://en.wikipedia.org/wiki/ASCII}{ASCII} 标准统一标准化并且与 0 到 127 范围内的整数一一对应。当然,还有很多非英文字符,包括 ASCII 字符在注音或其他方面的变体,例如西里尔字母和希腊字母,以及与 ASCII 和英文均完全无关的字母系统,包括阿拉伯语,中文, 希伯来语,印度语, 日本语, 和韩语。\href{https://en.wikipedia.org/wiki/Unicode}{Unicode} 标准对这些复杂的字符做了统一的定义,是一种大家普遍接受标准。 根据需求,写代码时可以忽略这种复杂性而只处理 ASCII 字符,也可针对可能出现的非 ASCII 文本而处理所有的字符或编码。Julia 可以简单高效地处理纯粹的 ASCII 文本以及 Unicode 文本。 甚至,在 Julia 中用 C 语言风格的代码来处理 ASCII 字符串,可以在不失性能和易读性的前提下达到预期效果。当遇到非 ASCII 文本时,Julia会优雅明确地提示错误信息而不是引入乱码。 这时,直接修改代码使其可以处理非 ASCII 数据即可。



关于 Julia 的字符串类型有一些值得注意的高级特性:



\begin{itemize}
\item Julia 中用于字符串(和字符串文字)的内置具体类型是 \hyperlink{2825695355940841177}{\texttt{String}}。 它支持全部 \href{https://en.wikipedia.org/wiki/Unicode}{Unicode} 字符 通过 \href{https://en.wikipedia.org/wiki/UTF-8}{UTF-8} 编码。(\hyperlink{11147209877072452260}{\texttt{transcode}} 函数是 提供 Unicode 编码和其他编码转换的函数。)


\item 所有的字符串类型都是抽象类型 \texttt{AbstractString} 的子类型,而一些外部包定义了别的 \texttt{AbstractString} 子类型(例如为其它的编码定义的子类型)。若要定义需要字符串参数的函数,你应当声明此类型为 \texttt{AbstractString} 来让这函数接受任何字符串类型。


\item 类似 C 和 Java,但是和大多数动态语言不同的是,Julia 有优秀的表示单字符的类型,即 \hyperlink{17842511721012314372}{\texttt{AbstractChar}}。\hyperlink{3463806064296245385}{\texttt{Char}} 是 \texttt{AbstractChar} 的内置子类型,它能表示任何 Unicode 字符的 32 位原始类型(基于 UTF-8 编码)。


\item 如 Java 中那样,字符串不可改——任何 \texttt{AbstractString} 对象的值不可改变。 若要构造不同的字符串值,应当从其它字符串的部分构造一个新的字符串。


\item 从概念上讲,字符串是从索引到字符的\emph{部分函数}:对于某些索引值,它不返回字符值,而是引发异常。这允许通过编码表示形式的字节索引来实现高效的字符串索引,而不是通过字符索引——它不能简单高效地实现可变宽度的 Unicode 字符串编码。

\end{itemize}


\hypertarget{11743000381881707413}{}


\section{字符}



\texttt{Char} 类型的值代表单个字符:它只是带有特殊文本表示法和适当算术行为的 32 位原始类型,不能转化为代表 \href{https://en.wikipedia.org/wiki/Code\_point}{Unicode 代码} 的数值。(Julia 的包可能会定义别的 \texttt{AbstractChar} 子类型,比如当为了优化对其它 \href{https://en.wikipedia.org/wiki/Character\_encoding}{字符编码} 的操作时)\texttt{Char} 类型的值以这样的方式输入和显示:




\begin{minted}{jlcon}
julia> 'x'
'x': ASCII/Unicode U+0078 (category Ll: Letter, lowercase)

julia> typeof(ans)
Char
\end{minted}



你可以轻松地将 \texttt{Char} 转换为其对应的整数值,即 Unicode 代码:




\begin{minted}{jlcon}
julia> Int('x')
120

julia> typeof(ans)
Int64
\end{minted}



在 32 位架构中,\hyperlink{13440452181855594120}{\texttt{typeof(ans)}} 将显示为 \hyperlink{10103694114785108551}{\texttt{Int32}}。你可以轻松地将一个整数值转回 \texttt{Char}。




\begin{minted}{jlcon}
julia> Char(120)
'x': ASCII/Unicode U+0078 (category Ll: Letter, lowercase)
\end{minted}



并非所有的整数值都是有效的 Unicode 代码,但是为了性能,\texttt{Char} 的转化不会检查每个值是否有效。如果你想检查每个转换的值是否为有效值,请使用 \hyperlink{9678448882095016755}{\texttt{isvalid}} 函数:




\begin{minted}{jlcon}
julia> Char(0x110000)
'\U110000': Unicode U+110000 (category In: Invalid, too high)

julia> isvalid(Char, 0x110000)
false
\end{minted}



As of this writing, the valid Unicode code points are \texttt{U+0000} through \texttt{U+D7FF} and \texttt{U+E000} through \texttt{U+10FFFF}. These have not all been assigned intelligible meanings yet, nor are they necessarily interpretable by applications, but all of these values are considered to be valid Unicode characters.



你可以在单引号中输入任何 Unicode 字符,通过使用 \texttt{{\textbackslash}u} 加上至多 4 个十六进制数字或者 \texttt{{\textbackslash}U} 加上至多 8 个十六进制数(最长的有效值也只需要 6 个):




\begin{minted}{jlcon}
julia> '\u0'
'\0': ASCII/Unicode U+0000 (category Cc: Other, control)

julia> '\u78'
'x': ASCII/Unicode U+0078 (category Ll: Letter, lowercase)

julia> '\u2200'
'∀': Unicode U+2200 (category Sm: Symbol, math)

julia> '\U10ffff'
'\U10ffff': Unicode U+10FFFF (category Cn: Other, not assigned)
\end{minted}



Julia 使用系统默认的区域和语言设置来确定,哪些字符可以被正确显示,哪些需要用 \texttt{{\textbackslash}u} 或 \texttt{{\textbackslash}U} 的转义来显示。除 Unicode 转义格式之外,还可以使用所有的\href{https://en.wikipedia.org/wiki/C\_syntax\#Backslash\_escapes}{传统 C 语言转义输入形式}:




\begin{minted}{jlcon}
julia> Int('\0')
0

julia> Int('\t')
9

julia> Int('\n')
10

julia> Int('\e')
27

julia> Int('\x7f')
127

julia> Int('\177')
127
\end{minted}



你可以对 \texttt{Char} 的值进行比较和有限的算术运算:




\begin{minted}{jlcon}
julia> 'A' < 'a'
true

julia> 'A' <= 'a' <= 'Z'
false

julia> 'A' <= 'X' <= 'Z'
true

julia> 'x' - 'a'
23

julia> 'A' + 1
'B': ASCII/Unicode U+0042 (category Lu: Letter, uppercase)
\end{minted}



\hypertarget{6723865345393966445}{}


\section{字符串基础}



字符串字面量由双引号或三重双引号分隔:




\begin{minted}{jlcon}
julia> str = "Hello, world.\n"
"Hello, world.\n"

julia> """Contains "quote" characters"""
"Contains \"quote\" characters"
\end{minted}



If you want to extract a character from a string, you index into it:




\begin{minted}{jlcon}
julia> str[begin]
'H': ASCII/Unicode U+0048 (category Lu: Letter, uppercase)

julia> str[1]
'H': ASCII/Unicode U+0048 (category Lu: Letter, uppercase)

julia> str[6]
',': ASCII/Unicode U+002C (category Po: Punctuation, other)

julia> str[end]
'\n': ASCII/Unicode U+000A (category Cc: Other, control)
\end{minted}



Many Julia objects, including strings, can be indexed with integers. The index of the first element (the first character of a string) is returned by \hyperlink{16943669671291374223}{\texttt{firstindex(str)}}, and the index of the last element (character) with \hyperlink{15780929618270241785}{\texttt{lastindex(str)}}. The keywords \texttt{begin} and \texttt{end} can be used inside an indexing operation as shorthand for the first and last indices, respectively, along the given dimension. String indexing, like most indexing in Julia, is 1-based: \texttt{firstindex} always returns \texttt{1} for any \texttt{AbstractString}. As we will see below, however, \texttt{lastindex(str)} is \emph{not} in general the same as \texttt{length(str)} for a string, because some Unicode characters can occupy multiple {\textquotedbl}code units{\textquotedbl}.



你可以用 \hyperlink{11574363005673055470}{\texttt{end}} 进行算术以及其它操作,就像普通值一样:




\begin{minted}{jlcon}
julia> str[end-1]
'.': ASCII/Unicode U+002E (category Po: Punctuation, other)

julia> str[end÷2]
' ': ASCII/Unicode U+0020 (category Zs: Separator, space)
\end{minted}



Using an index less than \texttt{begin} (\texttt{1}) or greater than \texttt{end} raises an error:




\begin{minted}{jlcon}
julia> str[begin-1]
ERROR: BoundsError: attempt to access String
  at index [0]
[...]

julia> str[end+1]
ERROR: BoundsError: attempt to access String
  at index [15]
[...]
\end{minted}



你也可以用范围索引来提取子字符串:




\begin{minted}{jlcon}
julia> str[4:9]
"lo, wo"
\end{minted}



Notice that the expressions \texttt{str[k]} and \texttt{str[k:k]} do not give the same result:




\begin{minted}{jlcon}
julia> str[6]
',': ASCII/Unicode U+002C (category Po: Punctuation, other)

julia> str[6:6]
","
\end{minted}



前者是 \texttt{Char} 类型的单个字符值,后者是碰巧只有单个字符的字符串值。在 Julia 里面两者大不相同。



范围索引复制了原字符串的选定部分。此外,也可以用 \hyperlink{2624824381693370630}{\texttt{SubString}} 类型创建字符串的 \texttt{view},例如:




\begin{minted}{jlcon}
julia> str = "long string"
"long string"

julia> substr = SubString(str, 1, 4)
"long"

julia> typeof(substr)
SubString{String}
\end{minted}



几个标准函数,像 \hyperlink{18002354026785919806}{\texttt{chop}}, \hyperlink{5360081372937794006}{\texttt{chomp}} 或者 \hyperlink{7002432768371197450}{\texttt{strip}} 都会返回一个 \hyperlink{2624824381693370630}{\texttt{SubString}}。



\hypertarget{12357763399910926447}{}


\section{Unicode 和 UTF-8}



Julia 完全支持 Unicode 字符和字符串。\hyperlink{16744269384625214739}{如上所述},在字符字面量中,Unicode 代码可以用 Unicode \texttt{{\textbackslash}u} 和 \texttt{{\textbackslash}U} 转义序列表示,也可以用所有标准 C 转义序列表示。这些同样可以用来写字符串字面量:




\begin{minted}{jlcon}
julia> s = "\u2200 x \u2203 y"
"∀ x ∃ y"
\end{minted}



Whether these Unicode characters are displayed as escapes or shown as special characters depends on your terminal{\textquotesingle}s locale settings and its support for Unicode. String literals are encoded using the UTF-8 encoding. UTF-8 is a variable-width encoding, meaning that not all characters are encoded in the same number of bytes ({\textquotedbl}code units{\textquotedbl}). In UTF-8, ASCII characters — i.e. those with code points less than 0x80 (128) – are encoded as they are in ASCII, using a single byte, while code points 0x80 and above are encoded using multiple bytes — up to four per character.



String indices in Julia refer to code units (= bytes for UTF-8), the fixed-width building blocks that are used to encode arbitrary characters (code points). This means that not every index into a \texttt{String} is necessarily a valid index for a character. If you index into a string at such an invalid byte index, an error is thrown:




\begin{minted}{jlcon}
julia> s[1]
'∀': Unicode U+2200 (category Sm: Symbol, math)

julia> s[2]
ERROR: StringIndexError("∀ x ∃ y", 2)
[...]

julia> s[3]
ERROR: StringIndexError("∀ x ∃ y", 3)
Stacktrace:
[...]

julia> s[4]
' ': ASCII/Unicode U+0020 (category Zs: Separator, space)
\end{minted}



在这种情况下,字符 \texttt{∀} 是一个三字节字符,因此索引 2 和 3 都是无效的,而下一个字符的索引是 4;这个接下来的有效索引可以用 \hyperlink{7455293228649070526}{\texttt{nextind(s,1)}} 来计算,再接下来的用 \texttt{nextind(s,4)},依此类推。



Since \texttt{end} is always the last valid index into a collection, \texttt{end-1} references an invalid byte index if the second-to-last character is multibyte.




\begin{minted}{jlcon}
julia> s[end-1]
' ': ASCII/Unicode U+0020 (category Zs: Separator, space)

julia> s[end-2]
ERROR: StringIndexError("∀ x ∃ y", 9)
Stacktrace:
[...]

julia> s[prevind(s, end, 2)]
'∃': Unicode U+2203 (category Sm: Symbol, math)
\end{minted}



The first case works, because the last character \texttt{y} and the space are one-byte characters, whereas \texttt{end-2} indexes into the middle of the \texttt{∃} multibyte representation. The correct way for this case is using \texttt{prevind(s, lastindex(s), 2)} or, if you{\textquotesingle}re using that value to index into \texttt{s} you can write \texttt{s[prevind(s, end, 2)]} and \texttt{end} expands to \texttt{lastindex(s)}.



Extraction of a substring using range indexing also expects valid byte indices or an error is thrown:




\begin{minted}{jlcon}
julia> s[1:1]
"∀"

julia> s[1:2]
ERROR: StringIndexError("∀ x ∃ y", 2)
Stacktrace:
[...]

julia> s[1:4]
"∀ "
\end{minted}



Because of variable-length encodings, the number of characters in a string (given by \hyperlink{3699181304419743826}{\texttt{length(s)}}) is not always the same as the last index. If you iterate through the indices 1 through \hyperlink{15780929618270241785}{\texttt{lastindex(s)}} and index into \texttt{s}, the sequence of characters returned when errors aren{\textquotesingle}t thrown is the sequence of characters comprising the string \texttt{s}. Thus we have the identity that \texttt{length(s) <= lastindex(s)}, since each character in a string must have its own index. The following is an inefficient and verbose way to iterate through the characters of \texttt{s}:




\begin{minted}{jlcon}
julia> for i = firstindex(s):lastindex(s)
           try
               println(s[i])
           catch
               # ignore the index error
           end
       end
∀

x

∃

y
\end{minted}



The blank lines actually have spaces on them. Fortunately, the above awkward idiom is unnecessary for iterating through the characters in a string, since you can just use the string as an iterable object, no exception handling required:




\begin{minted}{jlcon}
julia> for c in s
           println(c)
       end
∀

x

∃

y
\end{minted}



If you need to obtain valid indices for a string, you can use the \hyperlink{7455293228649070526}{\texttt{nextind}} and \hyperlink{15871508897466976220}{\texttt{prevind}} functions to increment/decrement to the next/previous valid index, as mentioned above. You can also use the \hyperlink{4701773772897287974}{\texttt{eachindex}} function to iterate over the valid character indices:




\begin{minted}{jlcon}
julia> collect(eachindex(s))
7-element Array{Int64,1}:
  1
  4
  5
  6
  7
 10
 11
\end{minted}



To access the raw code units (bytes for UTF-8) of the encoding, you can use the \hyperlink{16983098119361955361}{\texttt{codeunit(s,i)}} function, where the index \texttt{i} runs consecutively from \texttt{1} to \hyperlink{1775518749150675445}{\texttt{ncodeunits(s)}}.  The \hyperlink{17283482973786973382}{\texttt{codeunits(s)}} function returns an \texttt{AbstractVector\{UInt8\}} wrapper that lets you access these raw codeunits (bytes) as an array.



Strings in Julia can contain invalid UTF-8 code unit sequences. This convention allows to treat any byte sequence as a \texttt{String}. In such situations a rule is that when parsing a sequence of code units from left to right characters are formed by the longest sequence of 8-bit code units that matches the start of one of the following bit patterns (each \texttt{x} can be \texttt{0} or \texttt{1}):



\begin{itemize}
\item \texttt{0xxxxxxx};


\item \texttt{110xxxxx} \texttt{10xxxxxx};


\item \texttt{1110xxxx} \texttt{10xxxxxx} \texttt{10xxxxxx};


\item \texttt{11110xxx} \texttt{10xxxxxx} \texttt{10xxxxxx} \texttt{10xxxxxx};


\item \texttt{10xxxxxx};


\item \texttt{11111xxx}.

\end{itemize}


In particular this means that overlong and too-high code unit sequences and prefixes thereof are treated as a single invalid character rather than multiple invalid characters. This rule may be best explained with an example:




\begin{minted}{jlcon}
julia> s = "\xc0\xa0\xe2\x88\xe2|"
"\xc0\xa0\xe2\x88\xe2|"

julia> foreach(display, s)
'\xc0\xa0': [overlong] ASCII/Unicode U+0020 (category Zs: Separator, space)
'\xe2\x88': Malformed UTF-8 (category Ma: Malformed, bad data)
'\xe2': Malformed UTF-8 (category Ma: Malformed, bad data)
'|': ASCII/Unicode U+007C (category Sm: Symbol, math)

julia> isvalid.(collect(s))
4-element BitArray{1}:
 0
 0
 0
 1

julia> s2 = "\xf7\xbf\xbf\xbf"
"\U1fffff"

julia> foreach(display, s2)
'\U1fffff': Unicode U+1FFFFF (category In: Invalid, too high)
\end{minted}



我们可以看到字符串 \texttt{s} 中的前两个代码单元形成了一个过长的空格字符编码。这是无效的,但是在字符串中作为单个字符是可以接受的。接下来的两个代码单元形成了一个有效的 3 位 UTF-8 序列开头。然而,第五个代码单元 \texttt{{\textbackslash}xe2} 不是它的有效延续,所以代码单元 3 和 4 在这个字符串中也被解释为格式错误的字符。同理,由于 \texttt{|} 不是它的有效延续,代码单元 5 形成了一个格式错误的字符。最后字符串 \texttt{s2} 包含了一个太高的代码。



Julia 默认使用 UTF-8 编码,对于新编码的支持可以通过包加上。例如,\href{https://github.com/JuliaStrings/LegacyStrings.jl}{LegacyStrings.jl} 包实现了 \texttt{UTF16String} 和 \texttt{UTF32String} 类型。关于其它编码的额外讨论以及如何实现对它们的支持暂时超过了这篇文档的讨论范围。UTF-8 编码相关问题的进一步讨论参见下面的\href{@ref man-byte-array-literals}{字节数组字面量}章节。\hyperlink{11147209877072452260}{\texttt{transcode}} 函数可在各种 UTF-xx 编码之间转换,主要用于外部数据和包。



\hypertarget{3486870924145745190}{}


\section{拼接}



最常见最有用的字符串操作是级联:




\begin{minted}{jlcon}
julia> greet = "Hello"
"Hello"

julia> whom = "world"
"world"

julia> string(greet, ", ", whom, ".\n")
"Hello, world.\n"
\end{minted}



意识到像对无效 UTF-8 字符进行级联这样的潜在危险情形是非常重要的。生成的字符串可能会包含和输入字符串不同的字符,并且其中字符的数目也可能少于被级联字符串中字符数目之和,例如:




\begin{minted}{jlcon}
julia> a, b = "\xe2\x88", "\x80"
("\xe2\x88", "\x80")

julia> c = a*b
"∀"

julia> collect.([a, b, c])
3-element Array{Array{Char,1},1}:
 ['\xe2\x88']
 ['\x80']
 ['∀']

julia> length.([a, b, c])
3-element Array{Int64,1}:
 1
 1
 1
\end{minted}



这种情形只可能发生于无效 UTF-8 字符串上。对于有效 UTF-8 字符串,级联保留字符串中的所有字符和字符串的总长度。



Julia 也提供 \hyperlink{7592762607639177347}{\texttt{*}} 用于字符串级联:




\begin{minted}{jlcon}
julia> greet * ", " * whom * ".\n"
"Hello, world.\n"
\end{minted}



尽管对于提供 \texttt{+} 函数用于字符串拼接的语言使用者而言,\texttt{*} 似乎是一个令人惊讶的选择,但 \texttt{*} 的这种用法在数学中早有先例,尤其是在抽象代数中。



在数学上,\texttt{+} 通常表示可交换运算(\emph{commutative} operation)——运算对象的顺序不重要。一个例子是矩阵加法:对于任何形状相同的矩阵 \texttt{A} 和 \texttt{B},都有 \texttt{A + B == B + A}。与之相反,\texttt{*} 通常表示不可交换运算——运算对象的顺序很重要。例如,对于矩阵乘法,一般 \texttt{A * B != B * A}。同矩阵乘法类似,字符串拼接是不可交换的:\texttt{greet * whom != whom * greet}。在这一点上,对于插入字符串的拼接操作,\texttt{*} 是一个自然而然的选择,与它在数学中的用法一致。



更确切地说,有限长度字符串集合 \emph{S} 和字符串拼接操作 \texttt{*} 构成了一个\href{https://en.wikipedia.org/wiki/Free\_monoid}{自由幺半群} (\emph{S}, \texttt{*})。该集合的单位元是空字符串,\texttt{{\textquotedbl}{\textquotedbl}}。当一个自由幺半群不是交换的时,它的运算通常表示为 \texttt{{\textbackslash}cdot},\texttt{*},或者类似的符号,而非暗示交换性的 \texttt{+}。



\hypertarget{12583298261221600612}{}


\section{插值}



拼接构造字符串的方式有时有些麻烦。为了减少对于 \hyperlink{7919678712989769360}{\texttt{string}} 的冗余调用或者重复地做乘法,Julia 允许像 Perl 中一样使用 \texttt{\$} 对字符串字面量进行插值:




\begin{minted}{jlcon}
julia> "$greet, $whom.\n"
"Hello, world.\n"
\end{minted}



This is more readable and convenient and equivalent to the above string concatenation – the system rewrites this apparent single string literal into the call \texttt{string(greet, {\textquotedbl}, {\textquotedbl}, whom, {\textquotedbl}.{\textbackslash}n{\textquotedbl})}.



在 \texttt{\$} 之后最短的完整表达式被视为插入其值于字符串中的表达式。因此,你可以用括号向字符串中插入任何表达式:




\begin{minted}{jlcon}
julia> "1 + 2 = $(1 + 2)"
"1 + 2 = 3"
\end{minted}



Both concatenation and string interpolation call \hyperlink{7919678712989769360}{\texttt{string}} to convert objects into string form. However, \texttt{string} actually just returns the output of \hyperlink{8248717042415202230}{\texttt{print}}, so new types should add methods to \hyperlink{8248717042415202230}{\texttt{print}} or \hyperlink{14071376285304310153}{\texttt{show}} instead of \texttt{string}.



Most non-\texttt{AbstractString} objects are converted to strings closely corresponding to how they are entered as literal expressions:




\begin{minted}{jlcon}
julia> v = [1,2,3]
3-element Array{Int64,1}:
 1
 2
 3

julia> "v: $v"
"v: [1, 2, 3]"
\end{minted}



\hyperlink{7919678712989769360}{\texttt{string}} 是 \texttt{AbstractString} 和 \texttt{AbstractChar} 值的标识,所以它们作为自身被插入字符串,无需引用,无需转义:




\begin{minted}{jlcon}
julia> c = 'x'
'x': ASCII/Unicode U+0078 (category Ll: Letter, lowercase)

julia> "hi, $c"
"hi, x"
\end{minted}



若要在字符串字面量中包含文本 \texttt{\$},就用反斜杠转义:




\begin{minted}{jlcon}
julia> print("I have \$100 in my account.\n")
I have $100 in my account.
\end{minted}



\hypertarget{6215712550513853493}{}


\section{三引号字符串字面量}



当使用三引号(\texttt{{\textquotedbl}{\textquotedbl}{\textquotedbl}...{\textquotedbl}{\textquotedbl}{\textquotedbl}})创建字符串时,它们有一些在创建更长文本块时可能用到的特殊行为。



首先,三引号字符串也被反缩进到最小缩进线的水平。这在定义包含缩进的字符串时很有用。例如:




\begin{minted}{jlcon}
julia> str = """
           Hello,
           world.
         """
"  Hello,\n  world.\n"
\end{minted}



在这里,后三引号 \texttt{{\textquotedbl}{\textquotedbl}{\textquotedbl}} 前面的最后一(空)行设置了缩进级别。



反缩进级别被确定为所有行中空格或制表符的最大公共起始序列,不包括前三引号 \texttt{{\textquotedbl}{\textquotedbl}{\textquotedbl}} 后面的一行以及只包含空格或制表符的行(总包含结尾 \texttt{{\textquotedbl}{\textquotedbl}{\textquotedbl}} 的行)。那么对于所有不包括前三引号 \texttt{{\textquotedbl}{\textquotedbl}{\textquotedbl}} 后面文本的行而言,公共起始序列就被移除了(包括只含空格和制表符而以此序列开始的行),例如:




\begin{minted}{jlcon}
julia> """    This
         is
           a test"""
"    This\nis\n  a test"
\end{minted}



接下来,如果前三引号 \texttt{{\textquotedbl}{\textquotedbl}{\textquotedbl}} 后面紧跟换行符,那么换行符就从生成的字符串中被剥离。




\begin{minted}{julia}
"""hello"""
\end{minted}



等价于




\begin{minted}{julia}
"""
hello"""
\end{minted}



但是




\begin{minted}{julia}
"""

hello"""
\end{minted}



将在开头包含一个文本换行符。



换行符的移除是在反缩进之后进行的。例如:




\begin{minted}{jlcon}
julia> """
         Hello,
         world."""
"Hello,\nworld."
\end{minted}



尾随空格保持不变。



Triple-quoted string literals can contain \texttt{{\textquotedbl}} characters without escaping.



注意,无论是用单引号还是三引号,在文本字符串中换行符都会生成一个换行 (LF) 字符 \texttt{{\textbackslash}n},即使你的编辑器使用回车组合符 \texttt{{\textbackslash}r} (CR) 或 CRLF 来结束行。为了在字符串中包含 CR,总是应该使用显式转义符 \texttt{{\textbackslash}r};比如,可以输入文本字符串 \texttt{{\textquotedbl}a CRLF line ending{\textbackslash}r{\textbackslash}n{\textquotedbl}}。



\hypertarget{2767013232051989875}{}


\section{常见操作}



你可以使用标准的比较操作符按照字典顺序比较字符串:




\begin{minted}{jlcon}
julia> "abracadabra" < "xylophone"
true

julia> "abracadabra" == "xylophone"
false

julia> "Hello, world." != "Goodbye, world."
true

julia> "1 + 2 = 3" == "1 + 2 = $(1 + 2)"
true
\end{minted}



你可以使用 \hyperlink{13752961745140943082}{\texttt{findfirst}} 与 \hyperlink{16601358451866933976}{\texttt{findlast}} 函数搜索特定字符的索引:




\begin{minted}{jlcon}
julia> findfirst(isequal('o'), "xylophone")
4

julia> findlast(isequal('o'), "xylophone")
7

julia> findfirst(isequal('z'), "xylophone")
\end{minted}



你可以带上第三个参数,用 \hyperlink{9906000186778518011}{\texttt{findnext}} 与 \hyperlink{3864667477361062614}{\texttt{findprev}} 函数来在给定偏移量处搜索字符:




\begin{minted}{jlcon}
julia> findnext(isequal('o'), "xylophone", 1)
4

julia> findnext(isequal('o'), "xylophone", 5)
7

julia> findprev(isequal('o'), "xylophone", 5)
4

julia> findnext(isequal('o'), "xylophone", 8)
\end{minted}



你可以用 \hyperlink{7988132114328914630}{\texttt{occursin}} 函数检查在字符串中某子字符串可否找到。




\begin{minted}{jlcon}
julia> occursin("world", "Hello, world.")
true

julia> occursin("o", "Xylophon")
true

julia> occursin("a", "Xylophon")
false

julia> occursin('o', "Xylophon")
true
\end{minted}



最后那个例子表明 \hyperlink{7988132114328914630}{\texttt{occursin}} 也可用于搜寻字符字面量。



另外还有两个方便的字符串函数 \hyperlink{15426606278434194584}{\texttt{repeat}} 和 \hyperlink{18064910688022011979}{\texttt{join}}:




\begin{minted}{jlcon}
julia> repeat(".:Z:.", 10)
".:Z:..:Z:..:Z:..:Z:..:Z:..:Z:..:Z:..:Z:..:Z:..:Z:."

julia> join(["apples", "bananas", "pineapples"], ", ", " and ")
"apples, bananas and pineapples"
\end{minted}



其它有用的函数还包括:



\begin{itemize}
\item \hyperlink{16943669671291374223}{\texttt{firstindex(str)}} 给出可用来索引到 \texttt{str} 的最小(字节)索引(对字符串来说这总是 1,对于别的容器来说却不一定如此)。


\item \hyperlink{15780929618270241785}{\texttt{lastindex(str)}} 给出可用来索引到 \texttt{str} 的最大(字节)索引。


\item \hyperlink{3699181304419743826}{\texttt{length(str)}},\texttt{str} 中的字符个数。


\item \hyperlink{3699181304419743826}{\texttt{length(str, i, j)}},\texttt{str} 中从 \texttt{i} 到 \texttt{j} 的有效字符索引个数。


\item \hyperlink{1775518749150675445}{\texttt{ncodeunits(str)}},字符串中\href{https://en.wikipedia.org/wiki/Character\_encoding\#Terminology}{代码单元}(\href{https://zh.wikipedia.org/wiki/字符编码\#字符集、代码页,与字符映射}{码元})的数目。


\item \hyperlink{16983098119361955361}{\texttt{codeunit(str, i)}} 给出在字符串 \texttt{str} 中索引为 \texttt{i} 的代码单元值。


\item \hyperlink{11299403048911786045}{\texttt{thisind(str, i)}},给定一个字符串的任意索引,查找索引点所在的首个索引。


\item \hyperlink{7455293228649070526}{\texttt{nextind(str, i, n=1)}} 查找在索引 \texttt{i} 之后第 \texttt{n} 个字符的开头。


\item \hyperlink{15871508897466976220}{\texttt{prevind(str, i, n=1)}} 查找在索引 \texttt{i} 之前第 \texttt{n} 个字符的开始。

\end{itemize}


\hypertarget{7550171062631975520}{}


\section{非标准字符串字面量}



有时当你想构造字符串或者使用字符串语义,标准的字符串构造却不能很好的满足需求。Julia 为这种情形提供了非标准字符串字面量。非标准字符串字面量看似常规双引号字符串字面量,但却直接加上了标识符前缀因而并不那么像普通的字符串字面量。下面将提到,正则表达式,字节数组字面量和版本号字面量都是非标准字符串字面量的例子。其它例子见\hyperlink{12781685063176814936}{元编程}章。



\hypertarget{2492267677934939291}{}


\section{正则表达式}



Julia has Perl-compatible regular expressions (regexes), as provided by the \href{http://www.pcre.org/}{PCRE} library (a description of the syntax can be found \href{http://www.pcre.org/current/doc/html/pcre2syntax.html}{here}). Regular expressions are related to strings in two ways: the obvious connection is that regular expressions are used to find regular patterns in strings; the other connection is that regular expressions are themselves input as strings, which are parsed into a state machine that can be used to efficiently search for patterns in strings. In Julia, regular expressions are input using non-standard string literals prefixed with various identifiers beginning with \texttt{r}. The most basic regular expression literal without any options turned on just uses \texttt{r{\textquotedbl}...{\textquotedbl}}:




\begin{minted}{jlcon}
julia> r"^\s*(?:#|$)"
r"^\s*(?:#|$)"

julia> typeof(ans)
Regex
\end{minted}



若要检查正则表达式是否匹配某字符串,就用 \hyperlink{7988132114328914630}{\texttt{occursin}}:




\begin{minted}{jlcon}
julia> occursin(r"^\s*(?:#|$)", "not a comment")
false

julia> occursin(r"^\s*(?:#|$)", "# a comment")
true
\end{minted}



可以看到,\hyperlink{7988132114328914630}{\texttt{occursin}} 只返回正确或错误,表明给定正则表达式是否在该字符串中出现。然而,通常我们不只想知道字符串是否匹配,更想了解它是如何匹配的。要捕获匹配的信息,可以改用 \hyperlink{2695862412477105800}{\texttt{match}} 函数:




\begin{minted}{jlcon}
julia> match(r"^\s*(?:#|$)", "not a comment")

julia> match(r"^\s*(?:#|$)", "# a comment")
RegexMatch("#")
\end{minted}



若正则表达式与给定字符串不匹配,\hyperlink{2695862412477105800}{\texttt{match}} 返回 \hyperlink{9331422207248206047}{\texttt{nothing}}——在交互式提示框中不打印任何东西的特殊值。除了不打印,它是一个完全正常的值,这可以用程序来测试:




\begin{minted}{julia}
m = match(r"^\s*(?:#|$)", line)
if m === nothing
    println("not a comment")
else
    println("blank or comment")
end
\end{minted}



如果正则表达式匹配,\hyperlink{2695862412477105800}{\texttt{match}} 的返回值是 \texttt{RegexMatch} 对象。这些对象记录了表达式是如何匹配的,包括该模式匹配的子字符串和任何可能被捕获的子字符串。上面的例子仅仅捕获了匹配的部分子字符串,但也许我们想要捕获的是公共字符后面的任何非空文本。我们可以这样做:




\begin{minted}{jlcon}
julia> m = match(r"^\s*(?:#\s*(.*?)\s*$|$)", "# a comment ")
RegexMatch("# a comment ", 1="a comment")
\end{minted}



当调用 \hyperlink{2695862412477105800}{\texttt{match}} 时,你可以选择指定开始搜索的索引。例如:




\begin{minted}{jlcon}
julia> m = match(r"[0-9]","aaaa1aaaa2aaaa3",1)
RegexMatch("1")

julia> m = match(r"[0-9]","aaaa1aaaa2aaaa3",6)
RegexMatch("2")

julia> m = match(r"[0-9]","aaaa1aaaa2aaaa3",11)
RegexMatch("3")
\end{minted}



你可以从 \texttt{RegexMatch} 对象中提取如下信息:



\begin{itemize}
\item 匹配的整个子字符串:\texttt{m.match}


\item 作为字符串数组捕获的子字符串:\texttt{m.captures}


\item 整个匹配开始处的偏移:\texttt{m.offset}


\item 作为向量的捕获子字符串的偏移:\texttt{m.offsets}

\end{itemize}


当捕获不匹配时,\texttt{m.captures} 在该处不再包含一个子字符串,而是 \texttt{什么也不} 包含;此外,\texttt{m.offsets} 的偏移量为 0(回想一下,Julia 的索引是从 1 开始的,因此字符串的零偏移是无效的)。下面是两个有些牵强的例子:




\begin{minted}{jlcon}
julia> m = match(r"(a|b)(c)?(d)", "acd")
RegexMatch("acd", 1="a", 2="c", 3="d")

julia> m.match
"acd"

julia> m.captures
3-element Array{Union{Nothing, SubString{String}},1}:
 "a"
 "c"
 "d"

julia> m.offset
1

julia> m.offsets
3-element Array{Int64,1}:
 1
 2
 3

julia> m = match(r"(a|b)(c)?(d)", "ad")
RegexMatch("ad", 1="a", 2=nothing, 3="d")

julia> m.match
"ad"

julia> m.captures
3-element Array{Union{Nothing, SubString{String}},1}:
 "a"
 nothing
 "d"

julia> m.offset
1

julia> m.offsets
3-element Array{Int64,1}:
 1
 0
 2
\end{minted}



让捕获作为数组返回是很方便的,这样就可以用解构语法把它们和局域变量绑定起来:




\begin{minted}{jlcon}
julia> first, second, third = m.captures; first
"a"
\end{minted}



通过使用捕获组的编号或名称对 \texttt{RegexMatch} 对象进行索引,也可实现对捕获的访问:




\begin{minted}{jlcon}
julia> m=match(r"(?<hour>\d+):(?<minute>\d+)","12:45")
RegexMatch("12:45", hour="12", minute="45")

julia> m[:minute]
"45"

julia> m[2]
"45"
\end{minted}



使用 \hyperlink{17608641146794059481}{\texttt{replace}} 时利用 \texttt{{\textbackslash}n} 引用第 n 个捕获组和给替换字符串加上 \texttt{s} 的前缀,可以实现替换字符串中对捕获的引用。捕获组 0 指的是整个匹配对象。可在替换中用 \texttt{{\textbackslash}g<groupname>} 对命名捕获组进行引用。例如:




\begin{minted}{jlcon}
julia> replace("first second", r"(\w+) (?<agroup>\w+)" => s"\g<agroup> \1")
"second first"
\end{minted}



为明确起见,编号捕获组也可用 \texttt{{\textbackslash}g<n>} 进行引用,例如:




\begin{minted}{jlcon}
julia> replace("a", r"." => s"\g<0>1")
"a1"
\end{minted}



你可以在后双引号的后面加上 \texttt{i}, \texttt{m}, \texttt{s} 和 \texttt{x} 等标志对正则表达式进行修改。这些标志和 Perl 里面的含义一样,详见以下对 \href{http://perldoc.perl.org/perlre.html\#Modifiers}{perlre 手册}的摘录:




\begin{lstlisting}
i   不区分大小写的模式匹配。

    若区域设置规则有效,相应映射中代码点小于 255 的部分取自当前区域设置,更大代码点的部分取自 Unicode 规则。然而,跨越 Unicode 规则(ords 255/256)和 非 Unicode 规则边界的匹配将失败。

m   将字符串视为多行。也即更改 "^" 和 "$", 使其从匹配字符串的开头和结尾变为匹配字符串中任意一行的开头或结尾。

s   将字符串视为单行。也即更改 "." 以匹配任何字符,即使是通常不能匹配的换行符。

    像这样一起使用,r""ms,它们让 "." 匹配任何字符,同时也支持分别在字符串中换行符的后面和前面用 "^" 和 "$" 进行匹配。

x   令正则表达式解析器忽略多数既不是反斜杠也不属于字符类的空白。它可以用来把正则表达式分解成(略为)更易读的部分。和普通代码中一样,`#` 字符也被当作引入注释的元字符。
\end{lstlisting}



例如,下面的正则表达式已打开所有三个标志:




\begin{minted}{jlcon}
julia> r"a+.*b+.*?d$"ism
r"a+.*b+.*?d$"ims

julia> match(r"a+.*b+.*?d$"ism, "Goodbye,\nOh, angry,\nBad world\n")
RegexMatch("angry,\nBad world")
\end{minted}



\texttt{r{\textquotedbl}...{\textquotedbl}} 文本的构造没有插值和转义(除了引号 \texttt{{\textquotedbl}} 仍然需要转义)。下面例子展示了它和标准字符串字面量之间的差别:




\begin{minted}{jlcon}
julia> x = 10
10

julia> r"$x"
r"$x"

julia> "$x"
"10"

julia> r"\x"
r"\x"

julia> "\x"
ERROR: syntax: invalid escape sequence
\end{minted}



Julia 也支持 \texttt{r{\textquotedbl}{\textquotedbl}{\textquotedbl}...{\textquotedbl}{\textquotedbl}{\textquotedbl}} 形式的三引号正则表达式字符串(或许便于处理包含引号和换行符的正则表达式)。



\texttt{Regex()} 构造函数可以用于以编程方式创建合法的正则表达式字符串。这允许在构造正则表达式字符串时使用字符串变量的内容和其他字符串操作。上面的任何正则表达式代码可以在 \texttt{Regex()} 的单字符串参数中使用。下面是一些例子:




\begin{minted}{jlcon}
julia> using Dates

julia> d = Date(1962,7,10)
1962-07-10

julia> regex_d = Regex("Day " * string(day(d)))
r"Day 10"

julia> match(regex_d, "It happened on Day 10")
RegexMatch("Day 10")

julia> name = "Jon"
"Jon"

julia> regex_name = Regex("[\"( ]$name[\") ]")  # 插入 name 的值
r"[\"( ]Jon[\") ]"

julia> match(regex_name," Jon ")
RegexMatch(" Jon ")

julia> match(regex_name,"[Jon]") === nothing
true
\end{minted}



\hypertarget{6357751790368299582}{}


\section{字节数组字面量}



另一个有用的非标准字符串字面量是字节数组字面量:\texttt{b{\textquotedbl}...{\textquotedbl}}。这种形式使你能够用字符串表示法来表达只读字面量字节数组,也即 \hyperlink{6609065134969660118}{\texttt{UInt8}} 值的数组。字节数组字面量的规则如下:



\begin{itemize}
\item ASCII 字符和 ASCII 转义生成单个字节。


\item \texttt{{\textbackslash}x} 和八进制转义序列生成与转义值对应的\emph{字节}。


\item Unicode 转义序列生成编码 UTF-8 中该代码点的字节序列。

\end{itemize}


这些规则有一些重叠,这是因为 \texttt{{\textbackslash}x} 的行为和小于 0x80(128) 的八进制转义被前两个规则同时包括了;但这两个规则又是一致的。通过这些规则可以方便地同时使用 ASCII 字符,任意字节值,以及 UTF-8 序列来生成字节数组。下面是一个用到全部三个规则的例子:




\begin{minted}{jlcon}
julia> b"DATA\xff\u2200"
8-element Base.CodeUnits{UInt8,String}:
 0x44
 0x41
 0x54
 0x41
 0xff
 0xe2
 0x88
 0x80
\end{minted}



其中,ASCII 字符串 {\textquotedbl}DATA{\textquotedbl} 对应于字节 68, 65, 84, 65。\texttt{{\textbackslash}xff} 生成单个字节 255。Unicode 转义 \texttt{{\textbackslash}u2200} 在 UTF-8 中被编码为三个字节 226, 136, 128。注意生成的字节数组不对应任何有效 UTF-8 字符串。




\begin{minted}{jlcon}
julia> isvalid("DATA\xff\u2200")
false
\end{minted}



如前所述,\texttt{CodeUnits\{UInt8,String\}} 类型的行为类似于只读 \texttt{UInt8} 数组。如果需要标准数组,你可以 `Vector\{UInt8\} 进行转换。




\begin{minted}{jlcon}
julia> x = b"123"
3-element Base.CodeUnits{UInt8,String}:
 0x31
 0x32
 0x33

julia> x[1]
0x31

julia> x[1] = 0x32
ERROR: setindex! not defined for Base.CodeUnits{UInt8,String}
[...]

julia> Vector{UInt8}(x)
3-element Array{UInt8,1}:
 0x31
 0x32
 0x33
\end{minted}



同时,要注意到 \texttt{{\textbackslash}xff} 和 \texttt{{\textbackslash}uff} 之间的显著差别:前面的转义序列编码为\emph{字节 255},而后者代表 \emph{代码 255},它在 UTF-8 中编码为两个字节:




\begin{minted}{jlcon}
julia> b"\xff"
1-element Base.CodeUnits{UInt8,String}:
 0xff

julia> b"\uff"
2-element Base.CodeUnits{UInt8,String}:
 0xc3
 0xbf
\end{minted}



字符字面量也用到了相同的行为。



对于小于 \texttt{{\textbackslash}u80} 的代码,每个代码的 UTF-8 编码恰好只是由相应 \texttt{{\textbackslash}x} 转义产生的单个字节,因此忽略两者的差别无伤大雅。然而,从 \texttt{x80} 到 \texttt{{\textbackslash}xff} 的转义比起从 \texttt{u80} 到 \texttt{{\textbackslash}uff} 的转义来,就有一个主要的差别:前者都只编码为一个字节,它没有形成任何有效 UTF-8 数据,除非它后面有非常特殊的连接字节;而后者则都代表 2 字节编码的 Unicode 代码。



如果这些还是太难理解,试着读一下 \href{https://www.joelonsoftware.com/2003/10/08/the-absolute-minimum-every-software-developer-absolutely-positively-must-know-about-unicode-and-character-sets-no-excuses/}{{\textquotedbl}每个软件开发人员绝对必须知道的最基础 Unicode 和字符集知识{\textquotedbl}}。它是一个优质的 Unicode 和 UTF-8 指南,或许能帮助解除一些这方面的疑惑。



\hypertarget{10038787961375920908}{}


\section{版本号字面量}



版本号很容易用 \hyperlink{8914616188788850763}{\texttt{v{\textquotedbl}...{\textquotedbl}}} 形式的非标准字符串字面量表示。版本号字面量生成遵循\href{https://semver.org/}{语义版本}规范的 \hyperlink{16653194174751393225}{\texttt{VersionNumber}} 对象,因此由主、次、补丁号构成,后跟预发行 (pre-release) 和生成阿尔法数注释(build alpha-numeric)。例如,\texttt{v{\textquotedbl}0.2.1-rc1+win64{\textquotedbl}} 可分为主版本号 \texttt{0},次版本号 \texttt{2},补丁版本号 \texttt{1},预发行版号 \texttt{rc1},以及生成版本 \texttt{win64}。输入版本字面量时,除了主版本号以外所有内容都是可选的,因此 \texttt{v{\textquotedbl}0.2{\textquotedbl}} 等效于 \texttt{v{\textquotedbl}0.2.0{\textquotedbl}}(预发行号和生成注释为空),\texttt{v{\textquotedbl}2{\textquotedbl}} 等效于 \texttt{v{\textquotedbl}2.0.0{\textquotedbl}},等等。



\texttt{VersionNumber} 对象在轻松正确地比较两个(或更多)版本时非常有用。例如,常数 \texttt{VERSION} 把 Julia 的版本号保留为一个 \texttt{VersionNumber} 对象,因此可以像下面这样用简单的声明定义一些特定版本的行为:




\begin{minted}{julia}
if v"0.2" <= VERSION < v"0.3-"
    # 针对 0.2 发行版系列做些事情
end
\end{minted}



注意在上例中用到了非标准版本号 \texttt{v{\textquotedbl}0.3-{\textquotedbl}},其中有尾随符 \texttt{-}:这个符号是 Julia 标准的扩展,它可以用来表明低于任何 \texttt{0.3} 发行版的版本,包括所有的预发行版。所以上例中代码只能在稳定版本 \texttt{0.2} 上运行,而不能在 \texttt{v{\textquotedbl}0.3.0-rc1{\textquotedbl}} 这样的版本上运行。为了支持非稳定(即预发行)的 \texttt{0.2} 版本,下限检查应像这样应该改为:\texttt{v{\textquotedbl}0.2-{\textquotedbl} <= VERSION}。



另一个非标准版本规范扩展使得能够使用 \texttt{+} 来表示生成版本的上限,例如 \texttt{VERSION > v{\textquotedbl}0.2-rc1+{\textquotedbl}} 可以用来表示任意高于 \texttt{0.2-rc1} 和其任意生成版本的版本:它对 \texttt{v{\textquotedbl}0.2-rc1+win64{\textquotedbl}} 返回 \texttt{false} 而对 \texttt{v{\textquotedbl}0.2-rc2{\textquotedbl}} 返回 \texttt{true}。



在比较中使用这样的特殊版本是个好办法(特别是,总是应该对高版本使用尾随 \texttt{-},除非有好理由不这样),但它们不应该被用作任何内容的实际版本,因为它们在语义版本控制方案中无效。



除了用于定义常数 \hyperlink{12605722316331458198}{\texttt{VERSION}},\texttt{VersionNumber} 对象在 \texttt{Pkg} 模块应用广泛,常用于指定软件包的版本及其依赖。



\hypertarget{5096496368362976040}{}


\section{原始字符串字面量}



无插值和非转义的原始字符串可用 \texttt{raw{\textquotedbl}...{\textquotedbl}} 形式的非标准字符串字面量表示。原始字符串字面量生成普通的 \texttt{String} 对象,它无需插值和非转义地包含和输入完全一样的封闭式内容。这对于包含其他语言中使用 {\textquotedbl} 或 {\textbackslash}{\textquotedbl} 作为特殊字符的代码或标记的字符串很有用。



例外的是,引号仍必须转义,例如 \texttt{raw{\textquotedbl}{\textbackslash}{\textquotedbl}{\textquotedbl}} 等效于 \texttt{{\textquotedbl}{\textbackslash}{\textquotedbl}{\textquotedbl}}。为了能够表达所有字符串,反斜杠也必须转义,不过只是当它刚好出现在引号前面时。




\begin{minted}{jlcon}
julia> println(raw"\\ \\\"")
\\ \"
\end{minted}



请注意,前两个反斜杠在输出中逐字显示,这是因为它们不是在引号前面。然而,接下来的一个反斜杠字符转义了后面的一个反斜杠;又由于这些反斜杠出现在引号前面,最后一个反斜杠转义了一个引号。



\hypertarget{11836327794581856778}{}


\chapter{函数}



In Julia, a function is an object that maps a tuple of argument values to a return value. Julia functions are not pure mathematical functions, because they can alter and be affected by the global state of the program. The basic syntax for defining functions in Julia is:




\begin{minted}{jlcon}
julia> function f(x,y)
           x + y
       end
f (generic function with 1 method)
\end{minted}



This function accepts two arguments \texttt{x} and \texttt{y} and returns the value of the last expression evaluated, which is \texttt{x + y}.



在 Julia 中定义函数还有第二种更简洁的语法。上述的传统函数声明语法等效于以下紧凑性的“赋值形式”:




\begin{minted}{jlcon}
julia> f(x,y) = x + y
f (generic function with 1 method)
\end{minted}



尽管函数可以是复合表达式 (见 \hyperlink{14178955634857450370}{复合表达式}),但在赋值形式下,函数体必须是一个一行的表达式。简短的函数定义在 Julia 中是很常见的。非常惯用的短函数语法大大减少了打字和视觉方面的干扰。



使用传统的括号语法调用函数:




\begin{minted}{jlcon}
julia> f(2,3)
5
\end{minted}



Without parentheses, the expression \texttt{f} refers to the function object, and can be passed around like any other value:




\begin{minted}{jlcon}
julia> g = f;

julia> g(2,3)
5
\end{minted}



和变量名一样,Unicode 字符也可以用作函数名:




\begin{minted}{jlcon}
julia> ∑(x,y) = x + y
∑ (generic function with 1 method)

julia> ∑(2, 3)
5
\end{minted}



\hypertarget{7147748673565495429}{}


\section{参数传递行为}



Julia 函数参数遵循有时称为 “pass-by-sharing” 的约定,这意味着变量在被传递给函数时其值并不会被复制。函数参数本身充当新的变量绑定(指向变量值的新地址),它们所指向的值与所传递变量的值完全相同。调用者可以看到对函数内可变值(如数组)的修改。这与 Scheme,大多数 Lisps,Python,Ruby 和 Perl 以及其他动态语言中的行为相同。



\hypertarget{8101712267958021215}{}


\section{\texttt{return} 关键字}



The value returned by a function is the value of the last expression evaluated, which, by default, is the last expression in the body of the function definition. In the example function, \texttt{f}, from the previous section this is the value of the expression \texttt{x + y}. As an alternative, as in many other languages, the \texttt{return} keyword causes a function to return immediately, providing an expression whose value is returned:




\begin{minted}{julia}
function g(x,y)
    return x * y
    x + y
end
\end{minted}



由于函数定义可以输入到交互式会话中,因此可以很容易的比较这些定义:




\begin{minted}{jlcon}
julia> f(x,y) = x + y
f (generic function with 1 method)

julia> function g(x,y)
           return x * y
           x + y
       end
g (generic function with 1 method)

julia> f(2,3)
5

julia> g(2,3)
6
\end{minted}



当然,在一个单纯的线性执行的函数体内,例如 \texttt{g},使用 \texttt{return} 是没有意义的,因为表达式 \texttt{x + y} 永远不会被执行到,我们可以简单地把 \texttt{x * y} 写为最后一个表达式从而省略掉 \texttt{return}。 然而在使用其他控制流程的函数体内,\texttt{return} 却是有用的。 例如,在计算两条边长分别为 \texttt{x} 和 \texttt{y} 的三角形的斜边长度时可以避免溢出:




\begin{minted}{jlcon}
julia> function hypot(x,y)
           x = abs(x)
           y = abs(y)
           if x > y
               r = y/x
               return x*sqrt(1+r*r)
           end
           if y == 0
               return zero(x)
           end
           r = x/y
           return y*sqrt(1+r*r)
       end
hypot (generic function with 1 method)

julia> hypot(3, 4)
5.0
\end{minted}



这个函数有三个可能的返回处,返回三个不同表达式的值,具体取决于 \texttt{x} 和 \texttt{y} 的值。 最后一行的 \texttt{return} 可以省略,因为它是最后一个表达式。



\hypertarget{3145756716608098346}{}


\subsection{返回类型}



A return type can be specified in the function declaration using the \texttt{::} operator. This converts the return value to the specified type.




\begin{minted}{jlcon}
julia> function g(x, y)::Int8
           return x * y
       end;

julia> typeof(g(1, 2))
Int8
\end{minted}



这个函数将忽略 \texttt{x} 和 \texttt{y} 的类型,返回 \texttt{Int8} 类型的值。有关返回类型的更多信息,请参见\href{@ref}{类型声明}。



\hypertarget{720555615078928815}{}


\subsection{Returning nothing}



For functions that do not need to return a value (functions used only for some side effects), the Julia convention is to return the value \hyperlink{9331422207248206047}{\texttt{nothing}}:




\begin{minted}{julia}
function printx(x)
    println("x = $x")
    return nothing
end
\end{minted}



This is a \emph{convention} in the sense that \texttt{nothing} is not a Julia keyword but a only singleton object of type \texttt{Nothing}. Also, you may notice that the \texttt{printx} function example above is contrived, because \texttt{println} already returns \texttt{nothing}, so that the \texttt{return} line is redundant.



There are two possible shortened forms for the \texttt{return nothing} expression. On the one hand, the \texttt{return} keyword implicitly returns \texttt{nothing}, so it can be used alone. On the other hand, since functions implicitly return their last expression evaluated, \texttt{nothing} can be used alone when it{\textquotesingle}s the last expression. The preference for the expression \texttt{return nothing} as opposed to \texttt{return} or \texttt{nothing} alone is a matter of coding style.



\hypertarget{6095699413160695994}{}


\section{操作符也是函数}



在 Julia中,大多数操作符只不过是支持特殊语法的函数( \texttt{\&\&} 和\texttt{||} 等具有特殊评估语义的操作符除外,他们不能是函数,因为\hyperlink{7551496361738057869}{短路求值}要求在计算整个表达式的值之前不计算每个操作数)。因此,您也可以使用带括号的参数列表来使用它们,就和任何其他函数一样:




\begin{minted}{jlcon}
julia> 1 + 2 + 3
6

julia> +(1,2,3)
6
\end{minted}



中缀表达式和函数形式完全等价。—— 事实上,前一种形式会被编译器转换为函数调用。这也意味着你可以对操作符,例如 \hyperlink{3677358729494553841}{\texttt{+}} 和 \hyperlink{7592762607639177347}{\texttt{*}} ,进行赋值和传参,就像其它函数传参一样。




\begin{minted}{jlcon}
julia> f = +;

julia> f(1,2,3)
6
\end{minted}



然而,函数以\texttt{f}命名时不再支持中缀表达式。



\hypertarget{11001836393552529826}{}


\section{具有特殊名称的操作符}



有一些特殊的表达式对应的函数调用没有显示的函数名称,它们是:




\begin{table}[h]

\begin{tabulary}{\linewidth}{|L|L|}
\hline
表达式 & 函数调用 \\
\hline
\texttt{[A B C ...]} & \hyperlink{8862791894748483563}{\texttt{hcat}} \\
\hline
\texttt{[A; B; C; ...]} & \hyperlink{14691815416955507876}{\texttt{vcat}} \\
\hline
\texttt{[A B; C D; ...]} & \hyperlink{16279083053557795116}{\texttt{hvcat}} \\
\hline
\texttt{A{\textquotesingle}} & \hyperlink{10565518144285607255}{\texttt{adjoint}} \\
\hline
\texttt{A[i]} & \hyperlink{13720608614876840481}{\texttt{getindex}} \\
\hline
\texttt{A[i] = x} & \hyperlink{1309244355901386657}{\texttt{setindex!}} \\
\hline
\texttt{A.n} & \hyperlink{11040282462516403506}{\texttt{getproperty}} \\
\hline
\texttt{A.n = x} & \hyperlink{9055518433069578344}{\texttt{setproperty!}} \\
\hline
\end{tabulary}

\end{table}



\hypertarget{8300730259363458305}{}


\section{匿名函数}



函数在Julia里是\href{https://en.wikipedia.org/wiki/First-class\_citizen}{一等公民}:可以指定给变量,并使用标准函数调用语法通过被指定的变量调用。函数可以用作参数,也可以当作返回值。函数也可以不带函数名称地匿名创建,使用语法如下:




\begin{minted}{jlcon}
julia> x -> x^2 + 2x - 1
#1 (generic function with 1 method)

julia> function (x)
           x^2 + 2x - 1
       end
#3 (generic function with 1 method)
\end{minted}



这样就创建了一个接受一个参数 \texttt{x} 并返回当前值的多项式 \texttt{x{\textasciicircum}2+2x-1} 的函数。注意结果是个泛型函数,但是带了编译器生成的连续编号的名字。



匿名函数最主要的用法是传递给接收函数作为参数的函数。一个经典的例子是 \hyperlink{11483231213869150535}{\texttt{map}} ,为数组的每个元素应用一次函数,然后返回一个包含结果值的新数组:




\begin{minted}{jlcon}
julia> map(round, [1.2, 3.5, 1.7])
3-element Array{Float64,1}:
 1.0
 4.0
 2.0
\end{minted}



如果做为第一个参数传递给 \hyperlink{11483231213869150535}{\texttt{map}} 的转换函数已经存在,那直接使用函数名称是没问题的。但是通常要使用的函数还没有定义好,这样使用匿名函数就更加方便:




\begin{minted}{jlcon}
julia> map(x -> x^2 + 2x - 1, [1, 3, -1])
3-element Array{Int64,1}:
  2
 14
 -2
\end{minted}



接受多个参数的匿名函数写法可以使用语法 \texttt{(x,y,z)->2x+y-z},而无参匿名函数写作 \texttt{()->3} 。无参函数的这种写法看起来可能有些奇怪,不过它对于延迟计算很有必要。这种用法会把代码块包进一个无参函数中,后续把它当做 \texttt{f} 调用。



As an example, consider this call to \hyperlink{282460992333585641}{\texttt{get}}:




\begin{minted}{julia}
get(dict, key) do
    # default value calculated here
    time()
end
\end{minted}



上面的代码等效于使用包含代码的匿名函数调用\texttt{get}。 被包围在do和end之间,如下所示




\begin{minted}{julia}
get(()->time(), dict, key)
\end{minted}



The call to \hyperlink{2441622941271736623}{\texttt{time}} is delayed by wrapping it in a 0-argument anonymous function that is called only when the requested key is absent from \texttt{dict}.



\hypertarget{9218398227562398910}{}


\section{元组}



Julia 有一个和函数参数与返回值密切相关的内置数据结构叫做元组(\emph{tuple})。 一个元组是一个固定长度的容器,可以容纳任何值,但不可以被修改(是\emph{immutable}的)。 元组通过圆括号和逗号来构造,其内容可以通过索引来访问:




\begin{minted}{jlcon}
julia> (1, 1+1)
(1, 2)

julia> (1,)
(1,)

julia> x = (0.0, "hello", 6*7)
(0.0, "hello", 42)

julia> x[2]
"hello"
\end{minted}



注意,长度为1的元组必须使用逗号 \texttt{(1,)},而 \texttt{(1)} 只是一个带括号的值。\texttt{()} 表示空元组(长度为0)。



\hypertarget{14936898299796428859}{}


\section{具名元组}



元组的元素可以有名字,这时候就有了\emph{具名元组}:




\begin{minted}{jlcon}
julia> x = (a=2, b=1+2)
(a = 2, b = 3)

julia> x[1]
2

julia> x.a
2
\end{minted}



Named tuples are very similar to tuples, except that fields can additionally be accessed by name using dot syntax (\texttt{x.a}) in addition to the regular indexing syntax (\texttt{x[1]}).



\hypertarget{14329153377204363380}{}


\section{多返回值}



Julia 中,一个函数可以返回一个元组来实现返回多个值。不过,元组的创建和消除都不一定要用括号,这时候给人的感觉就是返回了多个值而非一个元组。比如下面这个例子,函数返回了两个值:




\begin{minted}{jlcon}
julia> function foo(a,b)
           a+b, a*b
       end
foo (generic function with 1 method)
\end{minted}



如果你在交互式会话中调用它且不把返回值赋值给任何变量,你会看到返回的元组:




\begin{minted}{jlcon}
julia> foo(2,3)
(5, 6)
\end{minted}



这种值对的典型用法是把每个值抽取为一个变量。Julia 支持简洁的元组“解构”:




\begin{minted}{jlcon}
julia> x, y = foo(2,3)
(5, 6)

julia> x
5

julia> y
6
\end{minted}



You can also return multiple values using the \texttt{return} keyword:




\begin{minted}{julia}
function foo(a,b)
    return a+b, a*b
end
\end{minted}



这与之前的定义的\texttt{foo}函数具有完全相同的效果。



\hypertarget{13913778167558632631}{}


\section{参数解构}



析构特性也可以被用在函数参数中。 如果一个函数的参数被写成了元组形式 (如  \texttt{(x, y)}) 而不是简单的符号,那么一个赋值运算 \texttt{(x, y) = argument} 将会被默认插入:




\begin{minted}{julia}
julia> minmax(x, y) = (y < x) ? (y, x) : (x, y)

julia> gap((min, max)) = max - min

julia> gap(minmax(10, 2))
8
\end{minted}



Notice the extra set of parentheses in the definition of \texttt{gap}. Without those, \texttt{gap} would be a two-argument function, and this example would not work.



\hypertarget{2609189760420802889}{}


\section{变参函数}



It is often convenient to be able to write functions taking an arbitrary number of arguments. Such functions are traditionally known as {\textquotedbl}varargs{\textquotedbl} functions, which is short for {\textquotedbl}variable number of arguments{\textquotedbl}. You can define a varargs function by following the last positional argument with an ellipsis:




\begin{minted}{jlcon}
julia> bar(a,b,x...) = (a,b,x)
bar (generic function with 1 method)
\end{minted}



变量 \texttt{a} 和 \texttt{b} 和以前一样被绑定给前两个参数,后面的参数整个做为迭代集合被绑定到变量 \texttt{x} 上 :




\begin{minted}{jlcon}
julia> bar(1,2)
(1, 2, ())

julia> bar(1,2,3)
(1, 2, (3,))

julia> bar(1, 2, 3, 4)
(1, 2, (3, 4))

julia> bar(1,2,3,4,5,6)
(1, 2, (3, 4, 5, 6))
\end{minted}



在所有这些情况下,\texttt{x} 被绑定到传递给 \texttt{bar} 的尾随值的元组。



也可以限制可以传递给函数的参数的数量,这部分内容稍后在  \hyperlink{14394864568540094383}{参数化约束的可变参数方法}  中讨论。



另一方面,将可迭代集中包含的值拆解为单独的参数进行函数调用通常很方便。 要实现这一点,需要在函数调用中额外使用 \texttt{...} 而不仅仅只是变量:




\begin{minted}{jlcon}
julia> x = (3, 4)
(3, 4)

julia> bar(1,2,x...)
(1, 2, (3, 4))
\end{minted}



在这个情况下一组值会被精确切片成一个可变参数调用,这里参数的数量是可变的。但是并不需要成为这种情况:




\begin{minted}{jlcon}
julia> x = (2, 3, 4)
(2, 3, 4)

julia> bar(1,x...)
(1, 2, (3, 4))

julia> x = (1, 2, 3, 4)
(1, 2, 3, 4)

julia> bar(x...)
(1, 2, (3, 4))
\end{minted}



进一步,拆解给函数调用中的可迭代对象不需要是个元组:




\begin{minted}{jlcon}
julia> x = [3,4]
2-element Array{Int64,1}:
 3
 4

julia> bar(1,2,x...)
(1, 2, (3, 4))

julia> x = [1,2,3,4]
4-element Array{Int64,1}:
 1
 2
 3
 4

julia> bar(x...)
(1, 2, (3, 4))
\end{minted}



另外,参数可拆解的函数也不一定就是变参函数 —— 尽管一般都是:




\begin{minted}{jlcon}
julia> baz(a,b) = a + b;

julia> args = [1,2]
2-element Array{Int64,1}:
1
2

julia> baz(args...)
3

julia> args = [1,2,3]
3-element Array{Int64,1}:
1
2
3

julia> baz(args...)
ERROR: MethodError: no method matching baz(::Int64, ::Int64, ::Int64)
Closest candidates are:
baz(::Any, ::Any) at none:1
\end{minted}



正如你所见,如果要拆解的容器(比如元组或数组)元素数量不匹配就会报错,和直接给多个参数报错一样。



\hypertarget{16207703858977287144}{}


\section{可选参数}



It is often possible to provide sensible default values for function arguments. This can save users from having to pass every argument on every call. For example, the function \hyperlink{4488183467971164548}{\texttt{Date(y, [m, d])}} from \texttt{Dates} module constructs a \texttt{Date} type for a given year \texttt{y}, month \texttt{m} and day \texttt{d}. However, \texttt{m} and \texttt{d} arguments are optional and their default value is \texttt{1}. This behavior can be expressed concisely as:




\begin{minted}{julia}
function Date(y::Int64, m::Int64=1, d::Int64=1)
    err = validargs(Date, y, m, d)
    err === nothing || throw(err)
    return Date(UTD(totaldays(y, m, d)))
end
\end{minted}



Observe, that this definition calls another method of the \texttt{Date} function that takes one argument of type \texttt{UTInstant\{Day\}}.



With this definition, the function can be called with either one, two or three arguments, and \texttt{1} is automatically passed when only one or two of the arguments are specified:




\begin{minted}{jlcon}
julia> using Dates

julia> Date(2000, 12, 12)
2000-12-12

julia> Date(2000, 12)
2000-12-01

julia> Date(2000)
2000-01-01
\end{minted}



可选参数实际上只是一种方便的语法,用于编写多种具有不同数量参数的方法定义(请参阅 \hyperlink{15680937628543940678}{可选参数和关键字的参数的注意事项})。这可通过调用 \texttt{methods} 函数来检查我们的 \texttt{Date} 函数示例。



\hypertarget{8084690442149965313}{}


\section{关键字参数}



某些函数需要大量参数,或者具有大量行为。记住如何调用这样的函数可能很困难。关键字参数允许通过名称而不是仅通过位置来识别参数,使得这些复杂接口易于使用和扩展。



例如,考虑绘制一条线的函数 \texttt{plot}。这个函数可能有很多选项,用来控制线条的样式、宽度、颜色等。如果它接受关键字参数,一个可行的调用可能看起来像 \texttt{plot(x, y, width=2)},这里我们仅指定线的宽度。请注意,这样做有两个目的。调用更可读,因为我们能以其意义标记参数。也使得大量参数的任意子集都能以任意次序传递。



具有关键字参数的函数在签名中使用分号定义:




\begin{minted}{julia}
function plot(x, y; style="solid", width=1, color="black")
    ###
end
\end{minted}



在函数调用时,分号是可选的:可以调用 \texttt{plot(x, y, width=2)} 或 \texttt{plot(x, y; width=2)},但前者的风格更为常见。显式的分号只有在传递可变参数或下文中描述的需计算的关键字时是必要的。



关键字参数的默认值只在必需时求值(当相应的关键字参数没有被传入),并且按从左到右的顺序求值,因为默认值的表达式可能会参照先前的关键字参数。



关键字参数的类型可以通过如下的方式显式指定:




\begin{minted}{julia}
function f(;x::Int=1)
    ###
end
\end{minted}



Keyword arguments can also be used in varargs functions:




\begin{minted}{julia}
function plot(x...; style="solid")
    ###
end
\end{minted}



附加的关键字参数可用 \texttt{...} 收集,正如在变参函数中:




\begin{minted}{julia}
function f(x; y=0, kwargs...)
    ###
end
\end{minted}



在 \texttt{f} 内部,\texttt{kwargs} 会是一个具名元组。具名元组(以及键类型为 \texttt{Symbol} 的字典)可作为关键字参数传递,这通过在调用中使用分号,例如 \texttt{f(x, z=1; kwargs...)}。



如果一个关键字参数在方法定义中未指定默认值,那么它就是\emph{必需的}:如果调用者没有为其赋值,那么将会抛出一个 \hyperlink{14325831233857471256}{\texttt{UndefKeywordError}} 异常:




\begin{minted}{julia}
function f(x; y)
    ###
end
f(3, y=5) # ok, y is assigned
f(3)      # throws UndefKeywordError(:y)
\end{minted}



在分号后也可传递 \texttt{key => value} 表达式。例如,\texttt{plot(x, y; :width => 2)} 等价于 \texttt{plot(x, y, width=2)}。当关键字名称需要在运行时被计算时,这就很实用了。



When a bare identifier or dot expression occurs after a semicolon, the keyword argument name is implied by the identifier or field name. For example \texttt{plot(x, y; width)} is equivalent to \texttt{plot(x, y; width=width)} and \texttt{plot(x, y; options.width)} is equivalent to \texttt{plot(x, y; width=options.width)}.



可选参数的性质使得可以多次指定同一参数的值。例如,在调用 \texttt{plot(x, y; options..., width=2)} 的过程中,\texttt{options} 结构也能包含一个 \texttt{width} 的值。在这种情况下,最右边的值优先级最高;在此例中,\texttt{width} 的值可以确定是 \texttt{2}。但是,显式地多次指定同一参数的值是不允许的,例如 \texttt{plot(x, y, width=2, width=3)},这会导致语法错误。



\hypertarget{5412457164020493169}{}


\section{默认值作用域的计算}



当计算可选和关键字参数的默认值表达式时,只有\emph{先前}的参数才在作用域内。例如,给出以下定义:




\begin{minted}{julia}
function f(x, a=b, b=1)
    ###
end
\end{minted}



\texttt{a=b} 中的 \texttt{b} 指的是外部作用域内的 \texttt{b},而不是后续参数中的 \texttt{b}。



\hypertarget{6290715684433505788}{}


\section{函数参数中的 Do 结构}



把函数作为参数传递给其他函数是一种强大的技术,但它的语法并不总是很方便。当函数参数占据多行时,这样的调用便特别难以编写。例如,考虑在具有多种情况的函数上调用 \hyperlink{11483231213869150535}{\texttt{map}}:




\begin{minted}{julia}
map(x->begin
           if x < 0 && iseven(x)
               return 0
           elseif x == 0
               return 1
           else
               return x
           end
       end,
    [A, B, C])
\end{minted}



Julia 提供了一个保留字 \texttt{do},用于更清楚地重写此代码:




\begin{minted}{julia}
map([A, B, C]) do x
    if x < 0 && iseven(x)
        return 0
    elseif x == 0
        return 1
    else
        return x
    end
end
\end{minted}



\texttt{do x} 语法创建一个带有参数 \texttt{x} 的匿名函数,并将其作为第一个参数传递 \hyperlink{11483231213869150535}{\texttt{map}}。类似地,\texttt{do a,b} 会创建一个双参数匿名函数,而一个简单的 \texttt{do} 会声明一个满足形式 \texttt{() -> ...} 的匿名函数。



这些参数如何初始化取决于「外部」函数;在这里,\hyperlink{11483231213869150535}{\texttt{map}} 将会依次将 \texttt{x} 设置为 \texttt{A}、\texttt{B}、\texttt{C},再分别调用调用匿名函数,正如在 \texttt{map(func, [A, B, C])} 语法中所发生的。



这种语法使得更容易使用函数来有效地扩展语言,因为调用看起来就像普通代码块。有许多可能的用法与 \hyperlink{11483231213869150535}{\texttt{map}} 完全不同,比如管理系统状态。例如,有一个版本的 \hyperlink{300818094931158296}{\texttt{open}} 可以通过运行代码来确保已经打开的文件最终会被关闭:




\begin{minted}{julia}
open("outfile", "w") do io
    write(io, data)
end
\end{minted}



这是通过以下定义实现的:




\begin{minted}{julia}
function open(f::Function, args...)
    io = open(args...)
    try
        f(io)
    finally
        close(io)
    end
end
\end{minted}



在这里,\hyperlink{300818094931158296}{\texttt{open}} 首先打开要写入的文件,接着将结果输出流传递给你在 \texttt{do ... end} 代码快中定义的匿名函数。在你的函数退出后,\hyperlink{300818094931158296}{\texttt{open}} 将确保流被正确关闭,无论你的函数是正常退出还是抛出了一个异常(\texttt{try/finally} 结构会在 \hyperlink{6880586223574224557}{流程控制} 中描述)。



使用 \texttt{do} 代码块语法时,查阅文档或实现有助于了解用户函数的参数是如何初始化的。



A \texttt{do} block, like any other inner function, can {\textquotedbl}capture{\textquotedbl} variables from its enclosing scope. For example, the variable \texttt{data} in the above example of \texttt{open...do} is captured from the outer scope. Captured variables can create performance challenges as discussed in \hyperlink{627547588659365489}{performance tips}.



\hypertarget{8022546101791804390}{}


\section{Function composition and piping}



Functions in Julia can be combined by composing or piping (chaining) them together.



Function composition is when you combine functions together and apply the resulting composition to arguments. You use the function composition operator (\texttt{∘}) to compose the functions, so \texttt{(f ∘ g)(args...)} is the same as \texttt{f(g(args...))}.



You can type the composition operator at the REPL and suitably-configured editors using \texttt{{\textbackslash}circ<tab>}.



For example, the \texttt{sqrt} and \texttt{+} functions can be composed like this:




\begin{minted}{jlcon}
julia> (sqrt ∘ +)(3, 6)
3.0
\end{minted}



这个语句先把数字相加,再对结果求平方根。



The next example composes three functions and maps the result over an array of strings:




\begin{minted}{jlcon}
julia> map(first ∘ reverse ∘ uppercase, split("you can compose functions like this"))
6-element Array{Char,1}:
 'U': ASCII/Unicode U+0055 (category Lu: Letter, uppercase)
 'N': ASCII/Unicode U+004E (category Lu: Letter, uppercase)
 'E': ASCII/Unicode U+0045 (category Lu: Letter, uppercase)
 'S': ASCII/Unicode U+0053 (category Lu: Letter, uppercase)
 'E': ASCII/Unicode U+0045 (category Lu: Letter, uppercase)
 'S': ASCII/Unicode U+0053 (category Lu: Letter, uppercase)
\end{minted}



Function chaining (sometimes called {\textquotedbl}piping{\textquotedbl} or {\textquotedbl}using a pipe{\textquotedbl} to send data to a subsequent function) is when you apply a function to the previous function{\textquotesingle}s output:




\begin{minted}{jlcon}
julia> 1:10 |> sum |> sqrt
7.416198487095663
\end{minted}



Here, the total produced by \texttt{sum} is passed to the \texttt{sqrt} function. The equivalent composition would be:




\begin{minted}{jlcon}
julia> (sqrt ∘ sum)(1:10)
7.416198487095663
\end{minted}



The pipe operator can also be used with broadcasting, as \texttt{.|>}, to provide a useful combination of the chaining/piping and dot vectorization syntax (described next).




\begin{minted}{jlcon}
julia> ["a", "list", "of", "strings"] .|> [uppercase, reverse, titlecase, length]
4-element Array{Any,1}:
  "A"
  "tsil"
  "Of"
 7
\end{minted}



\hypertarget{13590013989415065742}{}


\section{向量化函数的点语法}



在科学计算语言中,通常会有函数的「向量化」版本,它简单地将给定函数 \texttt{f(x)} 作用于数组 \texttt{A} 的每个元素,接着通过 \texttt{f(A)} 生成一个新数组。这种语法便于数据处理,但在其它语言中,向量化通常也是性能所需要的:如果循环很慢,函数的「向量化」版本可以调用由低级语言编写的、快速的库代码。在 Julia 中,向量化函数\emph{不}是性能所必需的,实际上编写自己的循环通常也是有益的(请参阅 \hyperlink{818954303942149020}{Performance Tips}),但它们仍然很方便。因此,\emph{任何} Julia 函数 \texttt{f} 能够以元素方式作用于任何数组(或者其它集合),这通过语法 \texttt{f.(A)} 实现。例如,\texttt{sin} 可以作用于向量 \texttt{A} 中的所有元素,如下所示:




\begin{minted}{jlcon}
julia> A = [1.0, 2.0, 3.0]
3-element Array{Float64,1}:
 1.0
 2.0
 3.0

julia> sin.(A)
3-element Array{Float64,1}:
 0.8414709848078965
 0.9092974268256817
 0.1411200080598672
\end{minted}



Of course, you can omit the dot if you write a specialized {\textquotedbl}vector{\textquotedbl} method of \texttt{f}, e.g. via \texttt{f(A::AbstractArray) = map(f, A)}, and this is just as efficient as \texttt{f.(A)}. The advantage of the \texttt{f.(A)} syntax is that which functions are vectorizable need not be decided upon in advance by the library writer.



更一般地,\texttt{f.(args...)} 实际上等价于 \texttt{broadcast(f, args...)},它允许你操作多个数组(甚至是不同形状的),或是数组和标量的混合(请参阅 \href{@ref}{Broadcasting})。例如,如果有 \texttt{f(x,y) = 3x + 4y},那么 \texttt{f.(pi,A)} 将为 \texttt{A} 中的每个 \texttt{a} 返回一个由 \texttt{f(pi,a)} 组成的新数组,而 \texttt{f.(vector1,vector2)} 将为每个索引 \texttt{i} 返回一个由 \texttt{f(vector1[i],vector2[i])} 组成的新向量(如果向量具有不同的长度则会抛出异常)。




\begin{minted}{jlcon}
julia> f(x,y) = 3x + 4y;

julia> A = [1.0, 2.0, 3.0];

julia> B = [4.0, 5.0, 6.0];

julia> f.(pi, A)
3-element Array{Float64,1}:
 13.42477796076938
 17.42477796076938
 21.42477796076938

julia> f.(A, B)
3-element Array{Float64,1}:
 19.0
 26.0
 33.0
\end{minted}



此外,\emph{嵌套的} \texttt{f.(args...)} 调用会被\emph{融合}到一个 \texttt{broadcast} 循环中。例如,\texttt{sin.(cos.(X))} 等价于 \texttt{broadcast(x -> sin(cos(x)), X)},类似于 \texttt{[sin(cos(x)) for x in X]}:在 \texttt{X} 上只有一个循环,并且只为结果分配了一个数组。[ 相反,在典型的「向量化」语言中,\texttt{sin(cos(X))} 首先会为 \texttt{tmp=cos(X)} 分配第一个临时数组,然后在单独的循环中计算 \texttt{sin(tmp)},再分配第二个数组。] 这种循环融合不是可能发生也可能不发生的编译器优化,只要遇到了嵌套的 \texttt{f.(args...)} 调用,它就是一个\emph{语法保证}。技术上,一旦遇到「非点」函数调用,融合就会停止;例如,在 \texttt{sin.(sort(cos.(X)))} 中,由于插入的 \texttt{sort} 函数,\texttt{sin} 和 \texttt{cos} 无法被合并。



Finally, the maximum efficiency is typically achieved when the output array of a vectorized operation is \emph{pre-allocated}, so that repeated calls do not allocate new arrays over and over again for the results (see \href{@ref}{Pre-allocating outputs}). A convenient syntax for this is \texttt{X .= ...}, which is equivalent to \texttt{broadcast!(identity, X, ...)} except that, as above, the \texttt{broadcast!} loop is fused with any nested {\textquotedbl}dot{\textquotedbl} calls. For example, \texttt{X .= sin.(Y)} is equivalent to \texttt{broadcast!(sin, X, Y)}, overwriting \texttt{X} with \texttt{sin.(Y)} in-place. If the left-hand side is an array-indexing expression, e.g. \texttt{X[begin+1:end] .= sin.(Y)}, then it translates to \texttt{broadcast!} on a \texttt{view}, e.g. \texttt{broadcast!(sin, view(X, firstindex(X)+1:lastindex(X)), Y)}, so that the left-hand side is updated in-place.



由于在表达式中为许多操作和函数调用添加点可能很乏味并导致难以阅读的代码,宏 \hyperlink{16688502228717894452}{\texttt{@.}} 用于将表达式中的\emph{每个}函数调用、操作和赋值转换为「点」版本。




\begin{minted}{jlcon}
julia> Y = [1.0, 2.0, 3.0, 4.0];

julia> X = similar(Y); # pre-allocate output array

julia> @. X = sin(cos(Y)) # equivalent to X .= sin.(cos.(Y))
4-element Array{Float64,1}:
  0.5143952585235492
 -0.4042391538522658
 -0.8360218615377305
 -0.6080830096407656
\end{minted}



像 \texttt{.+} 这样的二元(或一元)运算符使用相同的机制进行管理:它们等价于 \texttt{broadcast} 调用且可与其它嵌套的「点」调用融合。\texttt{X .+= Y} 等等价于 \texttt{X .= X .+ Y},结果为一个融合的 in-place 赋值;另见 \hyperlink{15967322336376951940}{dot operators}。



您也可以使用 \hyperlink{5135459825603202944}{\texttt{|>}} 将点操作与函数链组合在一起,如本例所示:




\begin{minted}{jlcon}
julia> [1:5;] .|> [x->x^2, inv, x->2*x, -, isodd]
5-element Array{Real,1}:
    1
    0.5
    6
   -4
 true
\end{minted}



\hypertarget{17965614568943116111}{}


\section{更多阅读}



我们应该在这里提到,这远不是定义函数的完整图景。Julia 拥有一个复杂的类型系统并且允许对参数类型进行多重分派。这里给出的示例都没有为它们的参数提供任何类型注释,意味着它们可以作用于任何类型的参数。类型系统在\hyperlink{8510890508040013186}{类型}中描述,而\hyperlink{3842379394166369470}{方法}则描述了根据运行时参数类型上的多重分派所选择的方法定义函数。



\hypertarget{8001618391799989953}{}


\chapter{流程控制}



Julia 提供了大量的流程控制构件:



\begin{itemize}
\item \hyperlink{14178955634857450370}{Compound Expressions}: \texttt{begin} and \texttt{;}.


\item \hyperlink{14451148373001501733}{条件表达式}:\texttt{if}-\texttt{elseif}-\texttt{else} 和 \texttt{?:} (三元运算符)。


\item \hyperlink{7551496361738057869}{短路求值}:\texttt{\&\&}、\texttt{||} 和链式比较。


\item \hyperlink{9034109510149997190}{重复执行:循环}:\texttt{while} 和 \texttt{for}。


\item \hyperlink{17887694433469406627}{异常处理}:\texttt{try}-\texttt{catch}、\hyperlink{17992125292605951734}{\texttt{error}} 和 \hyperlink{16410366672587017456}{\texttt{throw}}。


\item \hyperlink{17473131347184639576}{\texttt{Task}(协程)}:\hyperlink{4920987536368477483}{\texttt{yieldto}}。

\end{itemize}


前五个流程控制机制是高级编程语言的标准。\hyperlink{7131243650304654155}{\texttt{Task}} 不是那么的标准:它提供了非局部的流程控制,这使得在暂时挂起的计算任务之间进行切换成为可能。这是一个功能强大的构件:Julia 中的异常处理和协同多任务都是通过 \texttt{Task} 实现的。虽然日常编程并不需要直接使用 \texttt{Task},但某些问题用 \texttt{Task} 处理会更加简单。



\hypertarget{16096814372489430927}{}


\section{复合表达式}



Sometimes it is convenient to have a single expression which evaluates several subexpressions in order, returning the value of the last subexpression as its value. There are two Julia constructs that accomplish this: \texttt{begin} blocks and \texttt{;} chains. The value of both compound expression constructs is that of the last subexpression. Here{\textquotesingle}s an example of a \texttt{begin} block:




\begin{minted}{jlcon}
julia> z = begin
           x = 1
           y = 2
           x + y
       end
3
\end{minted}



Since these are fairly small, simple expressions, they could easily be placed onto a single line, which is where the \texttt{;} chain syntax comes in handy:




\begin{minted}{jlcon}
julia> z = (x = 1; y = 2; x + y)
3
\end{minted}



This syntax is particularly useful with the terse single-line function definition form introduced in \hyperlink{645008301484218813}{Functions}. Although it is typical, there is no requirement that \texttt{begin} blocks be multiline or that \texttt{;} chains be single-line:




\begin{minted}{jlcon}
julia> begin x = 1; y = 2; x + y end
3

julia> (x = 1;
        y = 2;
        x + y)
3
\end{minted}



\hypertarget{9876835618453764646}{}


\section{条件表达式}



条件表达式(Conditional evaluation)可以根据布尔表达式的值,让部分代码被执行或者不被执行。下面是对 \texttt{if}-\texttt{elseif}-\texttt{else} 条件语法的分析:




\begin{minted}{julia}
if x < y
    println("x is less than y")
elseif x > y
    println("x is greater than y")
else
    println("x is equal to y")
end
\end{minted}



如果表达式 \texttt{x < y} 是 \texttt{true},那么对应的代码块会被执行;否则判断条件表达式 \texttt{x > y},如果它是 \texttt{true},则执行对应的代码块;如果没有表达式是 true,则执行 \texttt{else} 代码块。下面是一个例子:




\begin{minted}{jlcon}
julia> function test(x, y)
           if x < y
               println("x is less than y")
           elseif x > y
               println("x is greater than y")
           else
               println("x is equal to y")
           end
       end
test (generic function with 1 method)

julia> test(1, 2)
x is less than y

julia> test(2, 1)
x is greater than y

julia> test(1, 1)
x is equal to y
\end{minted}



\texttt{elseif} 和 \texttt{else} 代码块是可选的,并且可以使用任意多个 \texttt{elseif} 代码块。 \texttt{if}-\texttt{elseif}-\texttt{else} 组件中的第一个条件表达式为 \texttt{true} 时,其他条件表达式才会被执行,当对应的代码块被执行后,其余的表达式或者代码块将不会被执行。



\texttt{if} 代码块是{\textquotedbl}有渗漏的{\textquotedbl},也就是说它们不会引入局部作用域。这意味着在 \texttt{if} 语句中新定义的变量依然可以在 \texttt{if} 代码块之后使用,尽管这些变量没有在 \texttt{if} 语句之前定义过。所以,我们可以将上面的 \texttt{test} 函数定义为




\begin{minted}{jlcon}
julia> function test(x,y)
           if x < y
               relation = "less than"
           elseif x == y
               relation = "equal to"
           else
               relation = "greater than"
           end
           println("x is ", relation, " y.")
       end
test (generic function with 1 method)

julia> test(2, 1)
x is greater than y.
\end{minted}



变量 \texttt{relation} 是在 \texttt{if} 代码块内部声明的,但可以在外部使用。然而,在利用这种行为的时候,要保证变量在所有的分支下都进行了定义。对上述函数做如下修改会导致运行时错误




\begin{minted}{jlcon}
julia> function test(x,y)
           if x < y
               relation = "less than"
           elseif x == y
               relation = "equal to"
           end
           println("x is ", relation, " y.")
       end
test (generic function with 1 method)

julia> test(1,2)
x is less than y.

julia> test(2,1)
ERROR: UndefVarError: relation not defined
Stacktrace:
 [1] test(::Int64, ::Int64) at ./none:7
\end{minted}



\texttt{if} 代码块也会返回一个值,这可能对于一些从其他语言转过来的用户来说不是很直观。 这个返回值就是被执行的分支中最后一个被执行的语句的返回值。 所以




\begin{minted}{jlcon}
julia> x = 3
3

julia> if x > 0
           "positive!"
       else
           "negative..."
       end
"positive!"
\end{minted}



需要注意的是,在 Julia 中,经常会用短路求值来表示非常短的条件表达式(单行),这会在下一节中介绍。



与 C, MATLAB, Perl, Python,以及 Ruby 不同,但跟 Java,还有一些别的严谨的类型语言类似:一个条件表达式的值如果不是 \texttt{true} 或者 \texttt{false} 的话,会返回错误:




\begin{minted}{jlcon}
julia> if 1
           println("true")
       end
ERROR: TypeError: non-boolean (Int64) used in boolean context
\end{minted}



这个错误是说,条件判断结果的类型:\hyperlink{7720564657383125058}{\texttt{Int64}} 是错的,而不是期望的 \hyperlink{46725311238864537}{\texttt{Bool}}。



所谓的 {\textquotedbl}三元运算符{\textquotedbl}, \texttt{?:},很类似 \texttt{if}-\texttt{elseif}-\texttt{else} 语法,它用于选择性获取单个表达式的值,而不是选择性执行大段的代码块。它因在很多语言中是唯一一个有三个操作数的运算符而得名:




\begin{minted}{julia}
a ? b : c
\end{minted}



在 \texttt{?} 之前的表达式 \texttt{a}, 是一个条件表达式,如果条件 \texttt{a} 是 \texttt{true},三元运算符计算在 \texttt{:} 之前的表达式 \texttt{b};如果条件 \texttt{a} 是 \texttt{false},则执行 \texttt{:} 后面的表达式 \texttt{c}。注意,\texttt{?} 和 \texttt{:} 旁边的空格是强制的,像 \texttt{a?b:c} 这种表达式不是一个有效的三元表达式(但在\texttt{?} 和 \texttt{:} 之后的换行是允许的)。



理解这种行为的最简单方式是看一个实际的例子。在前一个例子中,虽然在三个分支中都有调用 \texttt{println},但实质上是选择打印哪一个字符串。在这种情况下,我们可以用三元运算符更紧凑地改写。为了简明,我们先尝试只有两个分支的版本:




\begin{minted}{jlcon}
julia> x = 1; y = 2;

julia> println(x < y ? "less than" : "not less than")
less than

julia> x = 1; y = 0;

julia> println(x < y ? "less than" : "not less than")
not less than
\end{minted}



如果表达式 \texttt{x < y} 为真,整个三元运算符会执行字符串 \texttt{{\textquotedbl}less than{\textquotedbl}},否则执行字符串 \texttt{{\textquotedbl}not less than{\textquotedbl}}。原本的三个分支的例子需要链式嵌套使用三元运算符:




\begin{minted}{jlcon}
julia> test(x, y) = println(x < y ? "x is less than y"    :
                            x > y ? "x is greater than y" : "x is equal to y")
test (generic function with 1 method)

julia> test(1, 2)
x is less than y

julia> test(2, 1)
x is greater than y

julia> test(1, 1)
x is equal to y
\end{minted}



为了方便链式传值,运算符从右到左连接到一起。



重要地是,与 \texttt{if}-\texttt{elseif}-\texttt{else} 类似,\texttt{:} 之前和之后的表达式只有在条件表达式为 \texttt{true} 或者 \texttt{false} 时才会被相应地执行:




\begin{minted}{jlcon}
julia> v(x) = (println(x); x)
v (generic function with 1 method)

julia> 1 < 2 ? v("yes") : v("no")
yes
"yes"

julia> 1 > 2 ? v("yes") : v("no")
no
"no"
\end{minted}



\hypertarget{736299957672008833}{}


\section{短路求值}



短路求值非常类似条件求值。这种行为在多数有 \texttt{\&\&} 和 \texttt{||} 布尔运算符地命令式编程语言里都可以找到:在一系列由这些运算符连接的布尔表达式中,为了得到整个链的最终布尔值,仅仅只有最小数量的表达式被计算。更明确的说,这意味着:



\begin{itemize}
\item 在表达式 \texttt{a \&\& b} 中,子表达式 \texttt{b} 仅当 \texttt{a} 为 \texttt{true} 的时候才会被执行。


\item 在表达式 \texttt{a || b} 中,子表达式 \texttt{b} 仅在 \texttt{a} 为 \texttt{false} 的时候才会被执行。

\end{itemize}


这里的原因是:如果 \texttt{a} 是 \texttt{false},那么无论 \texttt{b} 的值是多少,\texttt{a \&\& b} 一定是 \texttt{false}。同理,如果 \texttt{a} 是 \texttt{true},那么无论 \texttt{b} 的值是多少,\texttt{a || b} 的值一定是 true。\texttt{\&\&} 和 \texttt{||} 都依赖于右边,但是 \texttt{\&\&} 比 \texttt{||} 有更高的优先级。我们可以简单地测试一下这个行为:




\begin{minted}{jlcon}
julia> t(x) = (println(x); true)
t (generic function with 1 method)

julia> f(x) = (println(x); false)
f (generic function with 1 method)

julia> t(1) && t(2)
1
2
true

julia> t(1) && f(2)
1
2
false

julia> f(1) && t(2)
1
false

julia> f(1) && f(2)
1
false

julia> t(1) || t(2)
1
true

julia> t(1) || f(2)
1
true

julia> f(1) || t(2)
1
2
true

julia> f(1) || f(2)
1
2
false
\end{minted}



你可以用同样的方式测试不同 \texttt{\&\&} 和 \texttt{||} 运算符的组合条件下的关联和优先级。



这种行为在 Julia 中经常被用来作为简短 \texttt{if} 语句的替代。 可以用 \texttt{<cond> \&\& <statement>} (可读为: <cond> \emph{and then} <statement>)来替换 \texttt{if <cond> <statement> end}。 类似的, 可以用 \texttt{<cond> || <statement>} (可读为: <cond> \emph{or else} <statement>)来替换 \texttt{if ! <cond> <statement> end}.



例如,可以像这样定义递归阶乘:




\begin{minted}{jlcon}
julia> function fact(n::Int)
           n >= 0 || error("n must be non-negative")
           n == 0 && return 1
           n * fact(n-1)
       end
fact (generic function with 1 method)

julia> fact(5)
120

julia> fact(0)
1

julia> fact(-1)
ERROR: n must be non-negative
Stacktrace:
 [1] error at ./error.jl:33 [inlined]
 [2] fact(::Int64) at ./none:2
 [3] top-level scope
\end{minted}



\textbf{无}短路求值的布尔运算可以用位布尔运算符来完成,见\hyperlink{16865688524696028421}{数学运算和初等函数}:\texttt{\&} 和 \texttt{|}。这些是普通的函数,同时也刚好支持中缀运算符语法,但总是会计算它们的所有参数:




\begin{minted}{jlcon}
julia> f(1) & t(2)
1
2
false

julia> t(1) | t(2)
1
2
true
\end{minted}



与 \texttt{if}, \texttt{elseif} 或者三元运算符中的条件表达式相同,\texttt{\&\&} 或者 \texttt{||} 的操作数必须是布尔值(\texttt{true} 或者 \texttt{false})。在链式嵌套的条件表达式中, 除最后一项外,使用非布尔值会导致错误:




\begin{minted}{jlcon}
julia> 1 && true
ERROR: TypeError: non-boolean (Int64) used in boolean context
\end{minted}



但在链的末尾允许使用任意类型的表达式,此表达式会根据前面的条件被执行并返回:




\begin{minted}{jlcon}
julia> true && (x = (1, 2, 3))
(1, 2, 3)

julia> false && (x = (1, 2, 3))
false
\end{minted}



\hypertarget{6800841505698205300}{}


\section{重复执行:循环}



有两个用于重复执行表达式的组件:\texttt{while} 循环和 \texttt{for} 循环。下面是一个 \texttt{while} 循环的例子:




\begin{minted}{jlcon}
julia> i = 1;

julia> while i <= 5
           println(i)
           global i += 1
       end
1
2
3
4
5
\end{minted}



\texttt{while} 循环会执行条件表达式(例子中为 \texttt{i <= 5}),只要它为 \texttt{true},就一直执行\texttt{while} 循环的主体部分。当 \texttt{while} 循环第一次执行时,如果条件表达式为 \texttt{false},那么主体代码就一次也不会被执行。



\texttt{for} 循环使得常见的重复执行代码写起来更容易。 像之前 \texttt{while} 循环中用到的向上和向下计数是可以用 \texttt{for} 循环更简明地表达:




\begin{minted}{jlcon}
julia> for i = 1:5
           println(i)
       end
1
2
3
4
5
\end{minted}



这里的 \texttt{1:5} 是一个范围对象,代表数字 1, 2, 3, 4, 5 的序列。\texttt{for} 循环在这些值之中迭代,对每一个变量 \texttt{i} 进行赋值。\texttt{for} 循环与之前 \texttt{while} 循环的一个非常重要区别是作用域,即变量的可见性。如果变量 \texttt{i} 没有在另一个作用域里引入,在 \texttt{for} 循环内,它就只在 \texttt{for} 循环内部可见,在外部和后面均不可见。你需要一个新的交互式会话实例或者一个新的变量名来测试这个特性:




\begin{minted}{jlcon}
julia> for j = 1:5
           println(j)
       end
1
2
3
4
5

julia> j
ERROR: UndefVarError: j not defined
\end{minted}



参见\hyperlink{11957539949537805757}{变量作用域}中对变量作用域的详细解释以及它在 Julia 中是如何工作的。



一般来说,\texttt{for} 循环组件可以用于迭代任一个容器。在这种情况下,相比 \texttt{=},另外的(但完全相同)关键字 \texttt{in} 或者 \texttt{∈} 则更常用,因为它使得代码更清晰:




\begin{minted}{jlcon}
julia> for i in [1,4,0]
           println(i)
       end
1
4
0

julia> for s ∈ ["foo","bar","baz"]
           println(s)
       end
foo
bar
baz
\end{minted}



在手册后面的章节中会介绍和讨论各种不同的迭代容器(比如,\hyperlink{16720099245556932994}{多维数组})。



为了方便,我们可能会在测试条件不成立之前终止一个 \texttt{while} 循环,或者在访问到迭代对象的结尾之前停止一个 \texttt{for} 循环,这可以用关键字 \texttt{break} 来完成:




\begin{minted}{jlcon}
julia> i = 1;

julia> while true
           println(i)
           if i >= 5
               break
           end
           global i += 1
       end
1
2
3
4
5

julia> for j = 1:1000
           println(j)
           if j >= 5
               break
           end
       end
1
2
3
4
5
\end{minted}



没有关键字 \texttt{break} 的话,上面的 \texttt{while} 循环永远不会自己结束,而 \texttt{for} 循环会迭代到 1000,这些循环都可以使用 \texttt{break} 来提前结束。



在某些场景下,需要直接结束此次迭代,并立刻进入下次迭代,\texttt{continue} 关键字可以用来完成此功能:




\begin{minted}{jlcon}
julia> for i = 1:10
           if i % 3 != 0
               continue
           end
           println(i)
       end
3
6
9
\end{minted}



这是一个有点做作的例子,因为我们可以通过否定这个条件,把 \texttt{println} 调用放到 \texttt{if} 代码块里来更简洁的实现同样的功能。在实际应用中,在 \texttt{continue} 后面还会有更多的代码要运行,并且调用 \texttt{continue} 的地方可能会有多个。



多个嵌套的 \texttt{for} 循环可以合并到一个外部循环,可以用来创建其迭代对象的笛卡尔积:




\begin{minted}{jlcon}
julia> for i = 1:2, j = 3:4
           println((i, j))
       end
(1, 3)
(1, 4)
(2, 3)
(2, 4)
\end{minted}



有了这个语法,迭代变量依然可以正常使用循环变量来进行索引,例如 \texttt{for i = 1:n, j = 1:i} 是合法的,但是在一个循环里面使用 \texttt{break} 语句则会跳出整个嵌套循环,不仅仅是内层循环。每次内层循环运行的时候,变量(\texttt{i} 和 \texttt{j})会被赋值为他们当前的迭代变量值。所以对 \texttt{i} 的赋值对于接下来的迭代是不可见的:




\begin{minted}{jlcon}
julia> for i = 1:2, j = 3:4
           println((i, j))
           i = 0
       end
(1, 3)
(1, 4)
(2, 3)
(2, 4)
\end{minted}



如果这个例子给每个变量一个关键字 \texttt{for} 来重写,那么输出会不一样:第二个和第四个变量包含 \texttt{0}。



\hypertarget{11112426789887598982}{}


\section{异常处理}



当一个意外条件发生时,一个函数可能无法向调用者返回一个合理的值。在这种情况下,最好让意外条件终止程序并打印出调试的错误信息,或者根据程序员预先提供的异常处理代码来采取恰当的措施。



\hypertarget{18218681852389631555}{}


\subsection{内置的 \texttt{Exception}}



当一个意外的情况发生时,会抛出 \texttt{Exception}。下面列出的内置 \texttt{Exception} 都会中断正常的控制流程。




\begin{table}[h]

\begin{tabulary}{\linewidth}{|L|}
\hline
\texttt{Exception} \\
\hline
\hyperlink{9721838137887538764}{\texttt{ArgumentError}} \\
\hline
\hyperlink{9731558909100893938}{\texttt{BoundsError}} \\
\hline
\hyperlink{15047752250898038281}{\texttt{CompositeException}} \\
\hline
\hyperlink{13752533629496758140}{\texttt{DimensionMismatch}} \\
\hline
\hyperlink{4168463413201806292}{\texttt{DivideError}} \\
\hline
\hyperlink{14085880504701688639}{\texttt{DomainError}} \\
\hline
\hyperlink{2683611566077490148}{\texttt{EOFError}} \\
\hline
\hyperlink{12102596058483452470}{\texttt{ErrorException}} \\
\hline
\hyperlink{5399118524830636312}{\texttt{InexactError}} \\
\hline
\hyperlink{15248096136337910028}{\texttt{InitError}} \\
\hline
\hyperlink{11255134339055983338}{\texttt{InterruptException}} \\
\hline
\texttt{InvalidStateException} \\
\hline
\hyperlink{12862287453053981792}{\texttt{KeyError}} \\
\hline
\hyperlink{15548397364092946520}{\texttt{LoadError}} \\
\hline
\hyperlink{9656432107553099418}{\texttt{OutOfMemoryError}} \\
\hline
\hyperlink{5617183776424836760}{\texttt{ReadOnlyMemoryError}} \\
\hline
\hyperlink{10250718604436154991}{\texttt{RemoteException}} \\
\hline
\hyperlink{68769522931907606}{\texttt{MethodError}} \\
\hline
\hyperlink{10461069697702909970}{\texttt{OverflowError}} \\
\hline
\hyperlink{6896679243086513948}{\texttt{Meta.ParseError}} \\
\hline
\hyperlink{16303515589950241655}{\texttt{SystemError}} \\
\hline
\hyperlink{2622693721821893139}{\texttt{TypeError}} \\
\hline
\hyperlink{7764749529861419421}{\texttt{UndefRefError}} \\
\hline
\hyperlink{4452889246677411554}{\texttt{UndefVarError}} \\
\hline
\hyperlink{414193743931514144}{\texttt{StringIndexError}} \\
\hline
\end{tabulary}

\end{table}



例如,当输入参数为负实数时,\hyperlink{4551113327515323898}{\texttt{sqrt}} 函数会抛出一个 \hyperlink{14085880504701688639}{\texttt{DomainError}} :




\begin{minted}{jlcon}
julia> sqrt(-1)
ERROR: DomainError with -1.0:
sqrt will only return a complex result if called with a complex argument. Try sqrt(Complex(x)).
Stacktrace:
[...]
\end{minted}



你可能需要根据下面的方式来定义你自己的异常:




\begin{minted}{jlcon}
julia> struct MyCustomException <: Exception end
\end{minted}



\hypertarget{11451631637715363921}{}


\subsection{\texttt{throw} 函数}



我们可以用 \hyperlink{16410366672587017456}{\texttt{throw}} 显式地创建异常。例如,若一个函数只对非负数有定义,当输入参数是负数的时候,可以用 \hyperlink{16410366672587017456}{\texttt{throw}} 抛出一个 \hyperlink{14085880504701688639}{\texttt{DomainError}}。




\begin{minted}{jlcon}
julia> f(x) = x>=0 ? exp(-x) : throw(DomainError(x, "argument must be nonnegative"))
f (generic function with 1 method)

julia> f(1)
0.36787944117144233

julia> f(-1)
ERROR: DomainError with -1:
argument must be nonnegative
Stacktrace:
 [1] f(::Int64) at ./none:1
\end{minted}



注意 \hyperlink{14085880504701688639}{\texttt{DomainError}} 后面不接括号的话不是一个异常,而是一个异常类型。我们需要调用它来获得一个 \texttt{Exception} 对象:




\begin{minted}{jlcon}
julia> typeof(DomainError(nothing)) <: Exception
true

julia> typeof(DomainError) <: Exception
false
\end{minted}



另外,一些异常类型会接受一个或多个参数来进行错误报告:




\begin{minted}{jlcon}
julia> throw(UndefVarError(:x))
ERROR: UndefVarError: x not defined
\end{minted}



我们可以仿照 \hyperlink{4452889246677411554}{\texttt{UndefVarError}} 的写法,用自定义异常类型来轻松实现这个机制:




\begin{minted}{jlcon}
julia> struct MyUndefVarError <: Exception
           var::Symbol
       end

julia> Base.showerror(io::IO, e::MyUndefVarError) = print(io, e.var, " not defined")
\end{minted}



\begin{quote}
\textbf{Note}

When writing an error message, it is preferred to make the first word lowercase. For example,

\texttt{size(A) == size(B) || throw(DimensionMismatch({\textquotedbl}size of A not equal to size of B{\textquotedbl}))}

is preferred over

\texttt{size(A) == size(B) || throw(DimensionMismatch({\textquotedbl}Size of A not equal to size of B{\textquotedbl}))}.

However, sometimes it makes sense to keep the uppercase first letter, for instance if an argument to a function is a capital letter:

\texttt{size(A,1) == size(B,2) || throw(DimensionMismatch({\textquotedbl}A has first dimension...{\textquotedbl}))}.

\end{quote}


\hypertarget{18278096073868016389}{}


\subsection{错误}



我们可以用 \hyperlink{17992125292605951734}{\texttt{error}} 函数生成一个 \hyperlink{12102596058483452470}{\texttt{ErrorException}} 来中断正常的控制流程。



假设我们希望在计算负数的平方根时让程序立即停止执行。为了实现它,我们可以定义一个挑剔的 \hyperlink{4551113327515323898}{\texttt{sqrt}} 函数,当它的参数是负数时,产生一个错误:




\begin{minted}{jlcon}
julia> fussy_sqrt(x) = x >= 0 ? sqrt(x) : error("negative x not allowed")
fussy_sqrt (generic function with 1 method)

julia> fussy_sqrt(2)
1.4142135623730951

julia> fussy_sqrt(-1)
ERROR: negative x not allowed
Stacktrace:
 [1] error at ./error.jl:33 [inlined]
 [2] fussy_sqrt(::Int64) at ./none:1
 [3] top-level scope
\end{minted}



如果另一个函数调用 \texttt{fussy\_sqrt} 和一个负数, 它会立马返回, 在交互会话中显示错误信息,而不会继续执行调用的函数:




\begin{minted}{jlcon}
julia> function verbose_fussy_sqrt(x)
           println("before fussy_sqrt")
           r = fussy_sqrt(x)
           println("after fussy_sqrt")
           return r
       end
verbose_fussy_sqrt (generic function with 1 method)

julia> verbose_fussy_sqrt(2)
before fussy_sqrt
after fussy_sqrt
1.4142135623730951

julia> verbose_fussy_sqrt(-1)
before fussy_sqrt
ERROR: negative x not allowed
Stacktrace:
 [1] error at ./error.jl:33 [inlined]
 [2] fussy_sqrt at ./none:1 [inlined]
 [3] verbose_fussy_sqrt(::Int64) at ./none:3
 [4] top-level scope
\end{minted}



\hypertarget{11842305126309838851}{}


\subsection{\texttt{try/catch} 语句}



通过 \texttt{try / catch} 语句,可以测试 Exception 并 优雅处理可能会破坏应用程序的事情。 例如, 在下面的代码中,平方根函数会引发异常。 通过 在其周围放置 \texttt{try / catch} 块可以缓解。 您可以选择如何 处理此异常,无论是记录它,返回占位符值还是 就像下面仅打印一句话。 要注意的是 在决定如何处理异常时,使用\texttt{try / catch} 块 比使用条件分支处理要慢得多。 以下是使用\texttt{try / catch} 块处理异常的更多示例:




\begin{minted}{jlcon}
julia> try
sqrt("ten")
catch e
println("You should have entered a numeric value")
end
You should have entered a numeric value
\end{minted}



\texttt{try/catch} 语句允许保存 \texttt{Exception} 到一个变量中。在下面这个做作的例子中,如果 \texttt{x} 是可索引的,则计算 \texttt{x} 的第二项的平方根,否则就假设 \texttt{x} 是一个实数,并返回它的平方根:




\begin{minted}{jlcon}
julia> sqrt_second(x) = try
           sqrt(x[2])
       catch y
           if isa(y, DomainError)
               sqrt(complex(x[2], 0))
           elseif isa(y, BoundsError)
               sqrt(x)
           end
       end
sqrt_second (generic function with 1 method)

julia> sqrt_second([1 4])
2.0

julia> sqrt_second([1 -4])
0.0 + 2.0im

julia> sqrt_second(9)
3.0

julia> sqrt_second(-9)
ERROR: DomainError with -9.0:
sqrt will only return a complex result if called with a complex argument. Try sqrt(Complex(x)).
Stacktrace:
[...]
\end{minted}



注意 \texttt{catch} 后面的字符会被一直认为是异常的名字,所以在写 \texttt{try/catch} 单行表达式时,需要特别小心。下面的代码\textbf{不会}在错误的情况下返回 \texttt{x} 的值:




\begin{minted}{julia}
try bad() catch x end
\end{minted}



正确的做法是在 \texttt{catch} 后添加一个分号或者直接换行:




\begin{minted}{julia}
try bad() catch; x end

try bad()
catch
    x
end
\end{minted}



\texttt{try/catch} 组件的强大之处在于能够将高度嵌套的计算立刻解耦成更高层次地调用函数。有时没有错误产生,但需要能够解耦堆栈,并传值到上层。Julia 提供了 \hyperlink{2102349972401293064}{\texttt{rethrow}}、\hyperlink{6187626674327343338}{\texttt{backtrace}}、\hyperlink{98342946516168163}{\texttt{catch\_backtrace}} 和 \hyperlink{5950075931444385711}{\texttt{Base.catch\_stack}} 函数进行更高级的错误处理。



\hypertarget{13560047635024894791}{}


\subsection{\texttt{finally} 子句}



在进行状态改变或者使用类似文件的资源的编程时,经常需要在代码结束的时候进行必要的清理工作(比如关闭文件)。由于异常会使得部分代码块在正常结束之前退出,所以可能会让上述工作变得复杂。\texttt{finally} 关键字提供了一种方式,无论代码块是如何退出的,都能够让代码块在退出时运行某段代码。



这里是一个确保一个打开的文件被关闭的例子:




\begin{minted}{julia}
f = open("file")
try
    # operate on file f
finally
    close(f)
end
\end{minted}



当控制流离开 \texttt{try} 代码块(例如,遇到 \texttt{return},或者正常结束),\texttt{close(f)} 就会被执行。如果 \texttt{try} 代码块由于异常退出,这个异常会继续传递。\texttt{catch} 代码块可以和 \texttt{try} 还有 \texttt{finally} 配合使用。这时 \texttt{finally} 代码块会在 \texttt{catch} 处理错误之后才运行。



\hypertarget{15038256797533490288}{}


\section{\texttt{Task}(协程)}



Tasks are a control flow feature that allows computations to be suspended and resumed in a flexible manner. We mention them here only for completeness; for a full discussion see \href{@ref man-asynchronous}{Asynchronous Programming}.



\hypertarget{14993622729045334657}{}


\chapter{变量作用域}



The \emph{scope} of a variable is the region of code within which a variable is visible. Variable scoping helps avoid variable naming conflicts. The concept is intuitive: two functions can both have arguments called \texttt{x} without the two \texttt{x}{\textquotesingle}s referring to the same thing. Similarly, there are many other cases where different blocks of code can use the same name without referring to the same thing. The rules for when the same variable name does or doesn{\textquotesingle}t refer to the same thing are called scope rules; this section spells them out in detail.



Certain constructs in the language introduce \emph{scope blocks}, which are regions of code that are eligible to be the scope of some set of variables. The scope of a variable cannot be an arbitrary set of source lines; instead, it will always line up with one of these blocks. There are two main types of scopes in Julia, \emph{global scope} and \emph{local scope}. The latter can be nested. There is also a distinction in Julia between constructs which introduce a {\textquotedbl}hard scope{\textquotedbl} and those which only introduce a {\textquotedbl}soft scope{\textquotedbl}, which affects whether shadowing a global variable by the same name is allowed or not.



\hypertarget{10787034693073583413}{}


\subsection{作用域结构}



The constructs introducing scope blocks are:




\begin{table}[h]

\begin{tabulary}{\linewidth}{|R|R|R|}
\hline
结构 & 作用域类型 & Allowed within \\
\hline
\hyperlink{16285380181904025577}{\texttt{module}}, \hyperlink{13329108222158426840}{\texttt{baremodule}} & 全局 & 全局 \\
\hline
\hyperlink{4119979838407461137}{\texttt{struct}} & local (soft) & 全局 \\
\hline
\hyperlink{9105224580875818383}{\texttt{for}}, \hyperlink{15133348314455964692}{\texttt{while}}, \hyperlink{16338536928035025961}{\texttt{try}} & local (soft) & 全局或局部 \\
\hline
\hyperlink{4625593635027008869}{\texttt{macro}} & local (hard) & 全局 \\
\hline
\hyperlink{4956741936243461891}{\texttt{let}}, functions, comprehensions, generators & local (hard) & 全局或局部 \\
\hline
\end{tabulary}

\end{table}



Notably missing from this table are \hyperlink{14178955634857450370}{begin blocks} and \hyperlink{14451148373001501733}{if blocks} which do \emph{not} introduce new scopes. The three types of scopes follow somewhat different rules which will be explained below.



Julia使用\href{https://en.wikipedia.org/wiki/Scope\_\%28computer\_science\%29\#Lexical\_scoping\_vs.\_dynamic\_scoping}{词法作用域},也就是说一个函数的作用域不会从其调用者的作用域继承,而从函数定义处的作用域继承。举个例子,在下列的代码中\texttt{foo}中的\texttt{x}指向的是模块\texttt{Bar}的全局作用域中的\texttt{x}。




\begin{minted}{jlcon}
julia> module Bar
           x = 1
           foo() = x
       end;
\end{minted}



并且在\texttt{foo}被使用的地方\texttt{x}并不在作用域中:




\begin{minted}{jlcon}
julia> import .Bar

julia> x = -1;

julia> Bar.foo()
1
\end{minted}



Thus \emph{lexical scope} means that what a variable in a particular piece of code refers to can be deduced from the code in which it appears alone and does not depend on how the program executes. A scope nested inside another scope can {\textquotedbl}see{\textquotedbl} variables in all the outer scopes in which it is contained. Outer scopes, on the other hand, cannot see variables in inner scopes.



\hypertarget{4662569705122519760}{}


\section{全局作用域}



Each module introduces a new global scope, separate from the global scope of all other modules—there is no all-encompassing global scope. Modules can introduce variables of other modules into their scope through the \hyperlink{16725527896995457152}{using or import} statements or through qualified access using the dot-notation, i.e. each module is a so-called \emph{namespace} as well as a first-class data structure associating names with values. Note that while variable bindings can be read externally, they can only be changed within the module to which they belong. As an escape hatch, you can always evaluate code inside that module to modify a variable; this guarantees, in particular, that module bindings cannot be modified externally by code that never calls \texttt{eval}.




\begin{minted}{jlcon}
julia> module A
           a = 1 # a global in A's scope
       end;

julia> module B
           module C
               c = 2
           end
           b = C.c    # can access the namespace of a nested global scope
                      # through a qualified access
           import ..A # makes module A available
           d = A.a
       end;

julia> module D
           b = a # errors as D's global scope is separate from A's
       end;
ERROR: UndefVarError: a not defined

julia> module E
           import ..A # make module A available
           A.a = 2    # throws below error
       end;
ERROR: cannot assign variables in other modules
\end{minted}



注意交互式提示行(即REPL)是在模块\texttt{Main}的全局作用域中。



\hypertarget{8604224695833880734}{}


\section{局部作用域}



A new local scope is introduced by most code blocks (see above \hyperlink{8072811582823893323}{table} for a complete list). Some programming languages require explicitly declaring new variables before using them. Explicit declaration works in Julia too: in any local scope, writing \texttt{local x} declares a new local variable in that scope, regardless of whether there is already a variable named \texttt{x} in an outer scope or not. Declaring each new local like this is somewhat verbose and tedious, however, so Julia, like many other languages, considers assignment to a new variable in a local scope to implicitly declare that variable as a new local. Mostly this is pretty intuitive, but as with many things that behave intuitively, the details are more subtle than one might naïvely imagine.



When \texttt{x = <value>} occurs in a local scope, Julia applies the following rules to decide what the expression means based on where the assignment expression occurs and what \texttt{x} already refers to at that location:



\begin{itemize}
\item[1. ] \textbf{Existing local:} If \texttt{x} is \emph{already a local variable}, then the existing local \texttt{x} is assigned;


\item[2. ] \textbf{Hard scope:} If \texttt{x} is \emph{not already a local variable} and assignment occurs inside of any hard scope construct (i.e. within a let block, function or macro body, comprehension, or generator), a new local named \texttt{x} is created in the scope of the assignment;


\item[3. ] \textbf{Soft scope:} If \texttt{x} is \emph{not already a local variable} and all of the scope constructs containing the assignment are soft scopes (loops, \texttt{try}/\texttt{catch} blocks, or \texttt{struct} blocks), the behavior depends on whether the global variable \texttt{x} is defined:

\begin{itemize}
\item if global \texttt{x} is \emph{undefined}, a new local named \texttt{x} is created in the scope of the assignment;


\item if global \texttt{x} is \emph{defined}, the assignment is considered ambiguous:

\begin{itemize}
\item in \emph{non-interactive} contexts (files, eval), an ambiguity warning is printed and a new local is created;


\item in \emph{interactive} contexts (REPL, notebooks), the global variable \texttt{x} is assigned.

\end{itemize}
\end{itemize}
\end{itemize}


You may note that in non-interactive contexts the hard and soft scope behaviors are identical except that a warning is printed when an implicitly local variable (i.e. not declared with \texttt{local x}) shadows a global. In interactive contexts, the rules follow a more complex heuristic for the sake of convenience. This is covered in depth in examples that follow.



Now that you know the rules, let{\textquotesingle}s look at some examples. Each example is assumed to be evaluated in a fresh REPL session so that the only globals in each snippet are the ones that are assigned in that block of code.



We{\textquotesingle}ll begin with a nice and clear-cut situation—assignment inside of a hard scope, in this case a function body, when no local variable by that name already exists:




\begin{minted}{jlcon}
julia> function greet()
           x = "hello" # new local
           println(x)
       end
greet (generic function with 1 method)

julia> greet()
hello

julia> x # global
ERROR: UndefVarError: x not defined
\end{minted}



Inside of the \texttt{greet} function, the assignment \texttt{x = {\textquotedbl}hello{\textquotedbl}} causes \texttt{x} to be a new local variable in the function{\textquotesingle}s scope. There are two relevant facts: the assignment occurs in local scope and there is no existing local \texttt{x} variable. Since \texttt{x} is local, it doesn{\textquotesingle}t matter if there is a global named \texttt{x} or not. Here for example we define \texttt{x = 123} before defining and calling \texttt{greet}:




\begin{minted}{jlcon}
julia> x = 123 # global
123

julia> function greet()
           x = "hello" # new local
           println(x)
       end
greet (generic function with 1 method)

julia> greet()
hello

julia> x # global
123
\end{minted}



Since the \texttt{x} in \texttt{greet} is local, the value (or lack thereof) of the global \texttt{x} is unaffected by calling \texttt{greet}. The hard scope rule doesn{\textquotesingle}t care whether a global named \texttt{x} exists or not: assignment to \texttt{x} in a hard scope is local (unless \texttt{x} is declared global).



The next clear cut situation we{\textquotesingle}ll consider is when there is already a local variable named \texttt{x}, in which case \texttt{x = <value>} always assigns to this existing local \texttt{x}.  The function \texttt{sum\_to} computes the sum of the numbers from one up to \texttt{n}:




\begin{minted}{julia}
function sum_to(n)
    s = 0 # new local
    for i = 1:n
        s = s + i # assign existing local
    end
    return s # same local
end
\end{minted}



As in the previous example, the first assignment to \texttt{s} at the top of \texttt{sum\_to} causes \texttt{s} to be a new local variable in the body of the function. The \texttt{for} loop has its own inner local scope within the function scope. At the point where \texttt{s = s + i} occurs, \texttt{s} is already a local variable, so the assignment updates the existing \texttt{s} instead of creating a new local. We can test this out by calling \texttt{sum\_to} in the REPL:




\begin{minted}{jlcon}
julia> function sum_to(n)
           s = 0 # new local
           for i = 1:n
               s = s + i # assign existing local
           end
           return s # same local
       end
sum_to (generic function with 1 method)

julia> sum_to(10)
55

julia> s # global
ERROR: UndefVarError: s not defined
\end{minted}



Since \texttt{s} is local to the function \texttt{sum\_to}, calling the function has no effect on the global variable \texttt{s}. We can also see that the update \texttt{s = s + i} in the \texttt{for} loop must have updated the same \texttt{s} created by the initialization \texttt{s = 0} since we get the correct sum of 55 for the integers 1 through 10.



Let{\textquotesingle}s dig into the fact that the \texttt{for} loop body has its own scope for a second by writing a slightly more verbose variation which we{\textquotesingle}ll call \texttt{sum\_to′}, in which we save the sum \texttt{s + i} in a variable \texttt{t} before updating \texttt{s}:




\begin{minted}{jlcon}
julia> function sum_to′(n)
           s = 0 # new local
           for i = 1:n
               t = s + i # new local `t`
               s = t # assign existing local `s`
           end
           return s, @isdefined(t)
       end
sum_to′ (generic function with 1 method)

julia> sum_to′(10)
(55, false)
\end{minted}



This version returns \texttt{s} as before but it also uses the \texttt{@isdefined} macro to return a boolean indicating whether there is a local variable named \texttt{t} defined in the function{\textquotesingle}s outermost local scope. As you can see, there is no \texttt{t} defined outside of the \texttt{for} loop body. This is because of the hard scope rule again: since the assignment to \texttt{t} occurs inside of a function, which introduces a hard scope, the assignment causes \texttt{t} to become a new local variable in the local scope where it appears, i.e. inside of the loop body. Even if there were a global named \texttt{t}, it would make no difference—the hard scope rule isn{\textquotesingle}t affected by anything in global scope.



Let{\textquotesingle}s move onto some more ambiguous cases covered by the soft scope rule. We{\textquotesingle}ll explore this by extracting the bodies of the \texttt{greet} and \texttt{sum\_to′} functions into soft scope contexts. First, let{\textquotesingle}s put the body of \texttt{greet} in a \texttt{for} loop—which is soft, rather than hard—and evaluate it in the REPL:




\begin{minted}{jlcon}
julia> for i = 1:3
           x = "hello" # new local
           println(x)
       end
hello
hello
hello

julia> x
ERROR: UndefVarError: x not defined
\end{minted}



Since the global \texttt{x} is not defined when the \texttt{for} loop is evaluated, the first clause of the soft scope rule applies and \texttt{x} is created as local to the \texttt{for} loop and therefore global \texttt{x} remains undefined after the loop executes. Next, let{\textquotesingle}s consider the body of \texttt{sum\_to′} extracted into global scope, fixing its argument to \texttt{n = 10}




\begin{minted}{julia}
s = 0
for i = 1:10
    t = s + i
    s = t
end
s
@isdefined(t)
\end{minted}



What does this code do? Hint: it{\textquotesingle}s a trick question. The answer is {\textquotedbl}it depends.{\textquotedbl} If this code is entered interactively, it behaves the same way it does in a function body. But if the code appears in a file, it  prints an ambiguity warning and throws an undefined variable error. Let{\textquotesingle}s see it working in the REPL first:




\begin{minted}{jlcon}
julia> s = 0 # global
0

julia> for i = 1:10
           t = s + i # new local `t`
           s = t # assign global `s`
       end

julia> s # global
55

julia> @isdefined(t) # global
false
\end{minted}



The REPL approximates being in the body of a function by deciding whether assignment inside the loop assigns to a global or creates new local based on whether a global variable by that name is defined or not. If a global by the name exists, then the assignment updates it. If no global exists, then the assignment creates a new local variable. In this example we see both cases in action:



\begin{itemize}
\item There is no global named \texttt{t}, so \texttt{t = s + i} creates a new \texttt{t} that is local to the \texttt{for} loop;


\item There is a global named \texttt{s}, so \texttt{s = t} assigns to it.

\end{itemize}


The second fact is why execution of the loop changes the global value of \texttt{s} and the first fact is why \texttt{t} is still undefined after the loop executes. Now, let{\textquotesingle}s try evaluating this same code as though it were in a file instead:




\begin{minted}{jlcon}
julia> code = """
       s = 0 # global
       for i = 1:10
           t = s + i # new local `t`
           s = t # new local `s` with warning
       end
       s, # global
       @isdefined(t) # global
       """;

julia> include_string(Main, code)
┌ Warning: Assignment to `s` in soft scope is ambiguous because a global variable by the same name exists: `s` will be treated as a new local. Disambiguate by using `local s` to suppress this warning or `global s` to assign to the existing global variable.
└ @ string:4
ERROR: LoadError: UndefVarError: s not defined
\end{minted}



Here we use \hyperlink{2796348696499086186}{\texttt{include\_string}}, to evaluate \texttt{code} as though it were the contents of a file. We could also save \texttt{code} to a file and then call \texttt{include} on that file—the result would be the same. As you can see, this behaves quite different from evaluating the same code in the REPL. Let{\textquotesingle}s break down what{\textquotesingle}s happening here:



\begin{itemize}
\item global \texttt{s} is defined with the value \texttt{0} before the loop is evaluated


\item the assignment \texttt{s = t} occurs in a soft scope—a \texttt{for} loop outside of any function body or other hard scope construct


\item therefore the second clause of the soft scope rule applies, and the assignment is ambiguous so a warning is emitted


\item execution continues, making \texttt{s} local to the \texttt{for} loop body


\item since \texttt{s} is local to the \texttt{for} loop, it is undefined when \texttt{t = s + i} is evaluated, causing an error


\item evaluation stops there, but if it got to \texttt{s} and \texttt{@isdefined(t)}, it would return \texttt{0} and \texttt{false}.

\end{itemize}


This demonstrates some important aspects of scope: in a scope, each variable can only have one meaning, and that meaning is determined regardless of the order of expressions. The presence of the expression \texttt{s = t} in the loop causes \texttt{s} to be local to the loop, which means that it is also local when it appears on the right hand side of \texttt{t = s + i}, even though that expression appears first and is evaluated first. One might imagine that the \texttt{s} on the first line of the loop could be global while the \texttt{s} on the second line of the loop is local, but that{\textquotesingle}s not possible since the two lines are in the same scope block and each variable can only mean one thing in a given scope.



\hypertarget{1371816050938225763}{}


\subsubsection{On Soft Scope}



We have now covered all the local scope rules, but before wrapping up this section, perhaps a few words should be said about why the ambiguous soft scope case is handled differently in interactive and non-interactive contexts. There are two obvious questions one could ask:



\begin{itemize}
\item[1. ] Why doesn{\textquotesingle}t it just work like the REPL everywhere?


\item[2. ] Why doesn{\textquotesingle}t it just work like in files everywhere? And maybe skip the warning?

\end{itemize}


In Julia ≤ 0.6, all global scopes did work like the current REPL: when \texttt{x = <value>} occurred in a loop (or \texttt{try}/\texttt{catch}, or \texttt{struct} body) but outside of a function body (or \texttt{let} block or comprehension), it was decided based on whether a global named \texttt{x} was defined or not whether \texttt{x} should be local to the loop. This behavior has the advantage of being intuitive and convenient since it approximates the behavior inside of a function body as closely as possible. In particular, it makes it easy to move code back and forth between a function body and the REPL when trying to debug the behavior of a function. However, it has some downsides. First, it{\textquotesingle}s quite a complex behavior: many people over the years were confused about this behavior and complained that it was complicated and hard both to explain and understand. Fair point. Second, and arguably worse, is that it{\textquotesingle}s bad for programming {\textquotedbl}at scale.{\textquotedbl} When you see a small piece of code in one place like this, it{\textquotesingle}s quite clear what{\textquotesingle}s going on:




\begin{minted}{julia}
s = 0
for i = 1:10
    s += i
end
\end{minted}



Obviously the intention is to modify the existing global variable \texttt{s}. What else could it mean? However, not all real world code is so short or so clear. We found that code like the following often occurs in the wild:




\begin{minted}{julia}
x = 123

# much later
# maybe in a different file

for i = 1:10
    x = "hello"
    println(x)
end

# much later
# maybe in yet another file
# or maybe back in the first one where `x = 123`

y = x + 234
\end{minted}



It{\textquotesingle}s far less clear what should happen here. Since \texttt{x + {\textquotedbl}hello{\textquotedbl}} is a method error, it seems probable that the intention is for \texttt{x} to be local to the \texttt{for} loop. But runtime values and what methods happen to exist cannot be used to determine the scopes of variables. With the Julia ≤ 0.6 behavior, it{\textquotesingle}s especially concerning that someone might have written the \texttt{for} loop first, had it working just fine, but later when someone else adds a new global far away—possibly in a different file—the code suddenly changes meaning and either breaks noisily or, worse still, silently does the wrong thing. This kind of \href{https://en.wikipedia.org/wiki/Action\_at\_a\_distance\_(computer\_programming)}{{\textquotedbl}spooky action at a distance{\textquotedbl}} is something that good programming language designs should prevent.



So in Julia 1.0, we simplified the rules for scope: in any local scope, assignment to a name that wasn{\textquotesingle}t already a local variable created a new local variable. This eliminated the notion of soft scope entirely as well as removing the potential for spooky action. We uncovered and fixed a significant number of bugs due to the removal of soft scope, vindicating the choice to get rid of it. And there was much rejoicing! Well, no, not really. Because some people were angry that they now had to write:




\begin{minted}{julia}
s = 0
for i = 1:10
    global s += i
end
\end{minted}



Do you see that \texttt{global} annotation in there? Hideous. Obviously this situation could not be tolerated. But seriously, there are two main issues with requiring \texttt{global} for this kind of top-level code:



\begin{itemize}
\item[1. ] It{\textquotesingle}s no longer convenient to copy and paste the code from inside a function body into the REPL to debug it—you have to add \texttt{global} annotations and then remove them again to go back;


\item[2. ] Beginners will write this kind of code without the \texttt{global} and have no idea why their code doesn{\textquotesingle}t work—the error that they get is that \texttt{s} is undefined, which does not seem to enlighten anyone who happens to make this mistake.

\end{itemize}


As of Julia 1.5, this code works without the \texttt{global} annotation in interactive contexts like the REPL or Jupyter notebooks (just like Julia 0.6) and in files and other non-interactive contexts, it prints this very direct warning:



\begin{quote}
Assignment to \texttt{s} in soft scope is ambiguous because a global variable by the same name exists: \texttt{s} will be treated as a new local. Disambiguate by using \texttt{local s} to suppress this warning or \texttt{global s} to assign to the existing global variable.

\end{quote}


This addresses both issues while preserving the {\textquotedbl}programming at scale{\textquotedbl} benefits of the 1.0 behavior: global variables have no spooky effect on the meaning of code that may be far away; in the REPL copy-and-paste debugging works and beginners don{\textquotesingle}t have any issues; any time someone either forgets a \texttt{global} annotation or accidentally shadows an existing global with a local in a soft scope, which would be confusing anyway, they get a nice clear warning.



An important property of this design is that any code that executes in a file without a warning will behave the same way in a fresh REPL. And on the flip side, if you take a REPL session and save it to file, if it behaves differently than it did in the REPL, then you will get a warning.



\hypertarget{5946450612527580652}{}


\subsection{let块}



不像局部变量的赋值行为,\texttt{let}语句每次运行都新建一个新的变量绑定。赋值改变的是已存在值的位置,\texttt{let}会新建新的位置。这个区别通常都不重要,只会在通过闭包跳出作用域的变量的情况下能探测到。\texttt{let}语法接受由逗号隔开的一系列的赋值和变量名:




\begin{minted}{jlcon}
julia> x, y, z = -1, -1, -1;

julia> let x = 1, z
           println("x: $x, y: $y") # x is local variable, y the global
           println("z: $z") # errors as z has not been assigned yet but is local
       end
x: 1, y: -1
ERROR: UndefVarError: z not defined
\end{minted}



这个赋值会按顺序评估,在左边的新变量被引入之前右边的每隔两都会在作用域中被评估。所以编写像\texttt{let x = x}这样的东西是有意义的,因为两个\texttt{x}变量是不一样的,拥有不同的存储位置。这里有个例子,在例子中\texttt{let}的行为是必须的:




\begin{minted}{jlcon}
julia> Fs = Vector{Any}(undef, 2); i = 1;

julia> while i <= 2
           Fs[i] = ()->i
           global i += 1
       end

julia> Fs[1]()
3

julia> Fs[2]()
3
\end{minted}



这里我创建并存储了两个返回变量\texttt{i}的闭包。但是这两个始终是同一个变量\texttt{i}。所以这两个闭包行为是相同的。我们可以使用\texttt{let}来为\texttt{i}创建新的绑定:




\begin{minted}{jlcon}
julia> Fs = Vector{Any}(undef, 2); i = 1;

julia> while i <= 2
           let i = i
               Fs[i] = ()->i
           end
           global i += 1
       end

julia> Fs[1]()
1

julia> Fs[2]()
2
\end{minted}



因为 \texttt{begin} 结构不会引入新的作用域,使用没有参数的 \texttt{let} 来只引进一个新的作用域块而不创建新的绑定可能是有用的:




\begin{minted}{jlcon}
julia> let
           local x = 1
           let
               local x = 2
           end
           x
       end
1
\end{minted}



因为\texttt{let}引进了一个新的作用域块,内部的局部\texttt{x}与外部的局部\texttt{x}是不同的变量。



\hypertarget{17142085918864004510}{}


\subsection{Loops and Comprehensions}



In loops and \hyperlink{3967134426571365188}{comprehensions}, new variables introduced in their body scopes are freshly allocated for each loop iteration, as if the loop body were surrounded by a \texttt{let} block, as demonstrated by this example:




\begin{minted}{jlcon}
julia> Fs = Vector{Any}(undef, 2);

julia> for j = 1:2
           Fs[j] = ()->j
       end

julia> Fs[1]()
1

julia> Fs[2]()
2
\end{minted}



\texttt{for}循环或者推导式的迭代变量始终是个新的变量:




\begin{lstlisting}
julia> function f()
           i = 0
           for i = 1:3
               # empty
           end
           return i
       end;

julia> f()
0
\end{lstlisting}



However, it is occasionally useful to reuse an existing local variable as the iteration variable. This can be done conveniently by adding the keyword \texttt{outer}:




\begin{minted}{jlcon}
julia> function f()
           i = 0
           for outer i = 1:3
               # empty
           end
           return i
       end;

julia> f()
3
\end{minted}



\hypertarget{5407065244463689569}{}


\section{常量}



变量的经常的一个使用方式是给一个特定的不变的值一个名字。这样的变量只会被赋值一次。这个想法可以通过使用 \hyperlink{8421885763632484758}{\texttt{const}} 关键字传递给编译器:




\begin{minted}{jlcon}
julia> const e  = 2.71828182845904523536;

julia> const pi = 3.14159265358979323846;
\end{minted}



多个变量可以使用单个\texttt{const}语句进行声明:




\begin{minted}{jlcon}
julia> const a, b = 1, 2
(1, 2)
\end{minted}



\texttt{const}声明只应该在全局作用域中对全局变量使用。编译器很难为包含全局变量的代码优化,因为它们的值(甚至它们的类型)可以任何时候改变。如果一个全局变量不会改变,添加\texttt{const}声明会解决这个问题。



局部常量却大有不同。编译器能够自动确定一个局部变量什么时候是不变的,所以局部常量声明是不必要的,其实现在也并不支持。



特别的顶层赋值,比如使用\texttt{function}和\texttt{structure}关键字进行的,默认是不变的。



注意 \texttt{const} 只会影响变量绑定;变量可能会绑定到一个可变的对象上(比如一个数组)使得其仍然能被改变。另外当尝试给一个声明为常量的变量赋值时下列情景是可能的:



\begin{itemize}
\item 如果一个新值的类型与常量类型不一样时会扔出一个错误:

\end{itemize}



\begin{minted}{jlcon}
julia> const x = 1.0
1.0

julia> x = 1
ERROR: invalid redefinition of constant x
\end{minted}



\begin{itemize}
\item if a new value has the same type as the constant then a warning is printed:

\end{itemize}



\begin{minted}{jlcon}
julia> const y = 1.0
1.0

julia> y = 2.0
WARNING: redefinition of constant y. This may fail, cause incorrect answers, or produce other errors.
2.0
\end{minted}



\begin{itemize}
\item 如果赋值不会导致变量值的变化,不会给出任何信息:

\end{itemize}



\begin{minted}{jlcon}
julia> const z = 100
100

julia> z = 100
100
\end{minted}



最后一条规则适用于不可变对象,即使变量绑定会改变,例如:




\begin{minted}{jlcon}
julia> const s1 = "1"
"1"

julia> s2 = "1"
"1"

julia> pointer.([s1, s2], 1)
2-element Array{Ptr{UInt8},1}:
 Ptr{UInt8} @0x00000000132c9638
 Ptr{UInt8} @0x0000000013dd3d18

julia> s1 = s2
"1"

julia> pointer.([s1, s2], 1)
2-element Array{Ptr{UInt8},1}:
 Ptr{UInt8} @0x0000000013dd3d18
 Ptr{UInt8} @0x0000000013dd3d18
\end{minted}



但是对于可变对象,警告会如预期出现:




\begin{minted}{jlcon}
julia> const a = [1]
1-element Array{Int64,1}:
 1

julia> a = [1]
WARNING: redefinition of constant a. This may fail, cause incorrect answers, or produce other errors.
1-element Array{Int64,1}:
 1
\end{minted}



Note that although sometimes possible, changing the value of a \texttt{const} variable is strongly discouraged, and is intended only for convenience during interactive use. Changing constants can cause various problems or unexpected behaviors. For instance, if a method references a constant and is already compiled before the constant is changed, then it might keep using the old value:




\begin{minted}{jlcon}
julia> const x = 1
1

julia> f() = x
f (generic function with 1 method)

julia> f()
1

julia> x = 2
WARNING: redefinition of constant x. This may fail, cause incorrect answers, or produce other errors.
2

julia> f()
1
\end{minted}



\hypertarget{2141690262982725665}{}


\chapter{类型}



通常,我们把程序语言中的类型系统划分成两类:静态类型和动态类型。对于静态类型系统,在程序运行之前,我们就可计算每一个表达式的类型。而对于动态类型系统,我们只有通过运行那个程序,得到表达式具体的值,才能确定其具体的类型。通过让编写的代码无需在编译时知道值的确切类型,面向对象允许静态类型语言具有一定的灵活性。可以编写在不同类型上都能运行的代码的能力被称为多态。在经典的动态类型语言中,所有的代码都是多态的,这意味着这些代码对于其中值的类型没有约束,除非在代码中去具体的判断一个值的类型,或者对对象做一些它不支持的操作。



Julia 类型系统是动态的,但通过允许指出某些变量具有特定类型,获得了静态类型系统的一些优点。这对于生成高效的代码非常有帮助,但更重要的是,它允许针对函数参数类型的方法派发与语言深度集成。方法派发将在\hyperlink{3842379394166369470}{方法}中详细探讨,但它根植于此处提供的类型系统。



在类型被省略时,Julia 的默认行为是允许变量为任何类型。因此,可以编写许多有用的 Julia 函数,而无需显式使用类型。然而,当需要额外的表达力时,很容易逐渐将显式的类型注释引入先前的「无类型」代码中。添加类型注释主要有三个目的:利用 Julia 强大的多重派发机制、提高代码可读性以及捕获程序错误。



Julia 用\href{https://en.wikipedia.org/wiki/Type\_system}{类型系统}的术语描述是动态(dynamic)、主格(nominative)和参数(parametric)的。泛型可以被参数化,并且类型之间的层次关系可以被\href{https://en.wikipedia.org/wiki/Nominal\_type\_system}{显式地声明},而不是\href{https://en.wikipedia.org/wiki/Structural\_type\_system}{隐含地通过兼容的结构}。Julia 类型系统的一个特别显著的特征是具体类型相互之间不能是子类型:所有具体类型都是最终的类型,并且只有抽象类型可以作为其超类型。虽然起初看起来这可能过于严格,但它有许多有益的结果,但缺点却少得出奇。事实证明,能够继承行为比继承结构更重要,同时继承两者在传统的面向对象语言中导致了重大困难。Julia 类型系统的其它高级方面应当在先言明:



\begin{itemize}
\item 对象值和非对象值之间没有分别:Julia 中的所有值都是具有类型的真实对象其类型属于一个单独的、完全连通的类型图,该类型图的所有节点作为类型一样都是头等的。


\item 「编译期类型」是没有任何意义的概念:变量所具有的唯一类型是程序运行时的实际类型。这在面向对象被称为「运行时类型」,其中静态编译和多态的组合使得这种区别变得显著。


\item 值有类型,变量没有类型——变量仅仅是绑定给值的名字而已。


\item 抽象类型和具体类型都可以通过其它类型进行参数化。它们的参数化还可通过符号、使得 \hyperlink{12980593021531333073}{\texttt{isbits}} 返回 true 的任意类型的值(实质上,也就是像数字或布尔变量这样的东西,存储方式像 C 类型或不包含指向其它对象的指针的 \texttt{struct})和其元组。类型参数在不需要被引用或限制时可以省略。

\end{itemize}


Julia 的类型系统设计得强大而富有表现力,却清晰、直观且不引人注目。许多 Julia 程序员可能从未感觉需要编写明确使用类型的代码。但是,某些场景的编程可通过声明类型变得更加清晰、简单、快速和健壮。



\hypertarget{11271598028486730305}{}


\section{类型声明}



\texttt{::} 运算符可以用来在程序中给表达式和变量附加类型注释。这有两个主要原因:



\begin{itemize}
\item[1. ] 作为断言,帮助程序确认能是否正常运行,


\item[2. ] 给编译器提供额外的类型信息,这可能帮助程序提升性能,在某些情况下

\end{itemize}


当被附加到一个计算值的表达式时,\texttt{::} 操作符读作「是······的实例」。在任何地方都可以用它来断言左侧表达式的值是右侧类型的实例。当右侧类型是具体类型时,左侧的值必须能够以该类型作为其实现——回想一下,所有具体类型都是最终的,因此没有任何实现是任何其它具体类型的子类型。当右侧类型是抽象类型时,值是由该抽象类型子类型中的某个具体类型实现的才能满足该断言。如果类型断言非真,抛出一个异常,否则返回左侧的值:




\begin{minted}{jlcon}
julia> (1+2)::AbstractFloat
ERROR: TypeError: in typeassert, expected AbstractFloat, got a value of type Int64

julia> (1+2)::Int
3
\end{minted}



可以在任何表达式的所在位置做类型断言。



当被附加到赋值左侧的变量或作为 \texttt{local} 声明的一部分时,\texttt{::} 操作符的意义有所不同:它声明变量始终具有指定的类型,就像静态类型语言(如 C)中的类型声明。每个被赋给该变量的值都将使用 \hyperlink{1846942650946171605}{\texttt{convert}} 转换为被声明的类型:




\begin{minted}{jlcon}
julia> function foo()
           x::Int8 = 100
           x
       end
foo (generic function with 1 method)

julia> foo()
100

julia> typeof(ans)
Int8
\end{minted}



这个特性用于避免性能「陷阱」,即给一个变量赋值时意外更改了类型。



此「声明」行为仅发生在特定上下文中:




\begin{minted}{julia}
local x::Int8  # in a local declaration
x::Int8 = 10   # as the left-hand side of an assignment
\end{minted}



并应用于整个当前作用域,甚至在该声明之前。目前,类型声明不能在全局作用域中使用,例如在 REPL 中就不可以,因为 Julia 还没有常量类型的全局变量。



声明也可以附加到函数定义:




\begin{minted}{julia}
function sinc(x)::Float64
    if x == 0
        return 1
    end
    return sin(pi*x)/(pi*x)
end
\end{minted}



此函数的返回值就像赋值给了一个类型已被声明的变量:返回值始终转换为\texttt{Float64}。



\hypertarget{5929227155379782502}{}


\section{Abstract Types}



抽象类型不能实例化,只能作为类型图中的节点使用,从而描述由相关具体类型组成的集合:那些作为其后代的具体类型。我们从抽象类型开始,即使它们没有实例,因为它们是类型系统的主干:它们形成了概念的层次结构,这使得 Julia 的类型系统不只是对象实现的集合。



回想一下,在\hyperlink{8249022581856827126}{整数和浮点数}中,我们介绍了各种数值的具体类型:\hyperlink{5857518405103968275}{\texttt{Int8}}、\hyperlink{6609065134969660118}{\texttt{UInt8}}、\hyperlink{6667287249103968645}{\texttt{Int16}}、\hyperlink{7018610346698168012}{\texttt{UInt16}}、\hyperlink{10103694114785108551}{\texttt{Int32}}、\hyperlink{8690996847580776341}{\texttt{UInt32}}、\hyperlink{7720564657383125058}{\texttt{Int64}}、\hyperlink{5500998675195555601}{\texttt{UInt64}}、\hyperlink{8012327724714767060}{\texttt{Int128}}、\hyperlink{14811222188335428522}{\texttt{UInt128}}、\hyperlink{2727296760866702904}{\texttt{Float16}}、\hyperlink{8101639384272933082}{\texttt{Float32}} 和 \hyperlink{5027751419500983000}{\texttt{Float64}}。尽管 \texttt{Int8}、\texttt{Int16}、\texttt{Int32}、\texttt{Int64} 和 \texttt{Int128} 具有不同的表示大小,但都具有共同的特征,即它们都是带符号的整数类型。类似地,\texttt{UInt8}、\texttt{UInt16}、\texttt{UInt32}、\texttt{UInt64} 和 \texttt{UInt128} 都是无符号整数类型,而 \texttt{Float16}、\texttt{Float32} 和 \texttt{Float64} 是不同的浮点数类型而非整数类型。一段代码只对某些类型有意义是很常见的,比如,只在其参数是某种类型的整数,而不真正取决于特定\emph{类型}的整数时有意义。例如,最大公分母算法适用于所有类型的整数,但不适用于浮点数。抽象类型允许构造类型的层次结构,提供了具体类型可以适应的上下文。例如,这允许你轻松地为任何类型的整数编程,而不用将算法限制为某种特殊类型的整数。



使用 \hyperlink{12403756508738429935}{\texttt{abstract type}} 关键字来声明抽象类型。声明抽象类型的一般语法是:




\begin{lstlisting}
abstract type «name» end
abstract type «name» <: «supertype» end
\end{lstlisting}



该 \texttt{abstract type} 关键字引入了一个新的抽象类型,\texttt{«name»} 为其名称。此名称后面可以跟 \hyperlink{6254591906563366276}{\texttt{<:}} 和一个已存在的类型,表示新声明的抽象类型是此「父」类型的子类型。



如果没有给出超类型,则默认超类型为 \texttt{Any}——一个预定义的抽象类型,所有对象都是它的实例并且所有类型都是它的子类型。在类型理论中,\texttt{Any} 通常称为「top」,因为它位于类型图的顶点。Julia 还有一个预定义的抽象「bottom」类型,在类型图的最低点,写成 \texttt{Union\{\}}。这与 \texttt{Any} 完全相反:任何对象都不是 \texttt{Union\{\}} 的实例,所有的类型都是 \texttt{Union\{\}} 的超类型。



让我们考虑一些构成 Julia 数值类型层次结构的抽象类型:




\begin{minted}{julia}
abstract type Number end
abstract type Real     <: Number end
abstract type AbstractFloat <: Real end
abstract type Integer  <: Real end
abstract type Signed   <: Integer end
abstract type Unsigned <: Integer end
\end{minted}



\hyperlink{1990584313715697055}{\texttt{Number}} 类型为 \texttt{Any} 类型的直接子类型,并且 \hyperlink{6175959395021454412}{\texttt{Real}} 为它的子类型。反过来,\texttt{Real} 有两个子类型(它还有更多的子类型,但这里只展示了两个,稍后将会看到其它的子类型): \hyperlink{8469131683393450448}{\texttt{Integer}} 和 \hyperlink{11465394427882483091}{\texttt{AbstractFloat}},将世界分为整数的表示和实数的表示。实数的表示当然包括浮点类型,但也包括其他类型,例如有理数。因此,\texttt{AbstractFloat} 是一个 \texttt{Real} 的子类型,仅包括实数的浮点表示。整数被进一步细分为 \hyperlink{14154866400772377486}{\texttt{Signed}} 和 \hyperlink{4780971278803506664}{\texttt{Unsigned}} 两类。



\texttt{<:} 运算符的通常意义为「是······的子类型」,并被用于像这样的声明右侧类型是新声明类型的直接超类型。它也可以在表达式中用作子类型运算符,在其左操作数为其右操作数的子类型时返回 \texttt{true}:




\begin{minted}{jlcon}
julia> Integer <: Number
true

julia> Integer <: AbstractFloat
false
\end{minted}



抽象类型的一个重要用途是为具体类型提供默认实现。举个简单的例子,考虑:




\begin{minted}{julia}
function myplus(x,y)
    x+y
end
\end{minted}



首先需要注意的是上述的参数声明等价于 \texttt{x::Any} 和 \texttt{y::Any}。当函数被调用时,例如 \texttt{myplus(2,5)},派发器会选择与给定参数相匹配的名称为 \texttt{myplus} 的最具体方法。(有关多重派发的更多信息,请参阅\hyperlink{3842379394166369470}{方法}。)



假设没有找到比上述方法更具体的方法,Julia 接下来会在内部定义并编译一个名为 \texttt{myplus} 的方法,专门用于基于上面给出的泛型函数的两个 \texttt{Int} 参数,即它定义并编译:




\begin{minted}{julia}
function myplus(x::Int,y::Int)
    x+y
end
\end{minted}



最后,调用这个具体的方法。



因此,抽象类型允许程序员编写泛型函数,之后可以通过许多具体类型的组合将其用作默认方法。多亏了多重分派,程序员可以完全控制是使用默认方法还是更具体的方法。



An important point to note is that there is no loss in performance if the programmer relies on a function whose arguments are abstract types, because it is recompiled for each tuple of argument concrete types with which it is invoked. (There may be a performance issue, however, in the case of function arguments that are containers of abstract types; see \hyperlink{16419743784254835624}{Performance Tips}.)



\hypertarget{7048513132833584013}{}


\section{原始类型}



\begin{quote}
\textbf{Warning}

\end{quote}


It is almost always preferable to wrap an existing primitive type in a new   composite type than to define your own primitive type.



This functionality exists to allow Julia to bootstrap the standard primitive   types that LLVM supports. Once they are defined, there is very little reason   to define more.



原始类型是具体类型,其数据是由简单的位组成。原始类型的经典示例是整数和浮点数。与大多数语言不同,Julia 允许你声明自己的原始类型,而不是只提供一组固定的内置原始类型。实际上,标准原始类型都是在语言本身中定义的:




\begin{minted}{julia}
primitive type Float16 <: AbstractFloat 16 end
primitive type Float32 <: AbstractFloat 32 end
primitive type Float64 <: AbstractFloat 64 end

primitive type Bool <: Integer 8 end
primitive type Char <: AbstractChar 32 end

primitive type Int8    <: Signed   8 end
primitive type UInt8   <: Unsigned 8 end
primitive type Int16   <: Signed   16 end
primitive type UInt16  <: Unsigned 16 end
primitive type Int32   <: Signed   32 end
primitive type UInt32  <: Unsigned 32 end
primitive type Int64   <: Signed   64 end
primitive type UInt64  <: Unsigned 64 end
primitive type Int128  <: Signed   128 end
primitive type UInt128 <: Unsigned 128 end
\end{minted}



声明原始类型的一般语法是:




\begin{lstlisting}
primitive type «name» «bits» end
primitive type «name» <: «supertype» «bits» end
\end{lstlisting}



The number of bits indicates how much storage the type requires and the name gives the new type a name. A primitive type can optionally be declared to be a subtype of some supertype. If a supertype is omitted, then the type defaults to having \texttt{Any} as its immediate supertype. The declaration of \hyperlink{46725311238864537}{\texttt{Bool}} above therefore means that a boolean value takes eight bits to store, and has \hyperlink{8469131683393450448}{\texttt{Integer}} as its immediate supertype. Currently, only sizes that are multiples of 8 bits are supported and you are likely to experience LLVM bugs with sizes other than those used above. Therefore, boolean values, although they really need just a single bit, cannot be declared to be any smaller than eight bits.



\hyperlink{46725311238864537}{\texttt{Bool}},\hyperlink{5857518405103968275}{\texttt{Int8}} 和 \hyperlink{6609065134969660118}{\texttt{UInt8}} 类型都具有相同的表现形式:它们都是 8 位内存块。然而,由于 Julia 的类型系统是主格的,它们尽管具有相同的结构,但不是通用的。它们之间的一个根本区别是它们具有不同的超类型:\hyperlink{46725311238864537}{\texttt{Bool}} 的直接超类型是 \hyperlink{8469131683393450448}{\texttt{Integer}}、\hyperlink{5857518405103968275}{\texttt{Int8}} 的是 \hyperlink{14154866400772377486}{\texttt{Signed}} 而 \hyperlink{6609065134969660118}{\texttt{UInt8}} 的是 \hyperlink{4780971278803506664}{\texttt{Unsigned}}。\hyperlink{46725311238864537}{\texttt{Bool}},\hyperlink{5857518405103968275}{\texttt{Int8}} 和 \hyperlink{6609065134969660118}{\texttt{UInt8}} 的所有其它差异是行为上的——定义函数的方式在这些类型的对象作为参数给定时起作用。这也是为什么主格的类型系统是必须的:如果结构确定类型,类型决定行为,就不可能使 \hyperlink{46725311238864537}{\texttt{Bool}} 的行为与 \hyperlink{5857518405103968275}{\texttt{Int8}} 或 \hyperlink{6609065134969660118}{\texttt{UInt8}} 有任何不同。



\hypertarget{805665046800217201}{}


\section{复合类型}



\href{https://en.wikipedia.org/wiki/Composite\_data\_type}{复合类型}在各种语言中被称为 record、struct 和 object。复合类型是命名字段的集合,其实例可以视为单个值。复合类型在许多语言中是唯一一种用户可定义的类型,也是 Julia 中最常用的用户定义类型。



在主流的面向对象语言中,比如 C++、Java、Python 和 Ruby,复合类型也具有与它们相关的命名函数,并且该组合称为「对象」。在纯粹的面向对象语言中,例如 Ruby 或 Smalltalk,所有值都是对象,无论它们是否为复合类型。在不太纯粹的面向对象语言中,包括 C++ 和 Java,一些值,比如整数和浮点值,不是对象,而用户定义的复合类型是具有相关方法的真实对象。在 Julia 中,所有值都是对象,但函数不与它们操作的对象捆绑在一起。这是必要的,因为 Julia 通过多重派发选择函数使用的方法,这意味着在选择方法时考虑\emph{所有}函数参数的类型,而不仅仅是第一个(有关方法和派发的更多信息,请参阅\hyperlink{3842379394166369470}{方法})。因此,函数仅仅「属于」它们的第一个参数是不合适的。将方法组织到函数对象中而不是在每个对象「内部」命名方法最终成为语言设计中一个非常有益的方面。



\hyperlink{4119979838407461137}{\texttt{struct}} 关键字与复合类型一起引入,后跟一个字段名称的块,可选择使用 \texttt{::} 运算符注释类型:




\begin{minted}{jlcon}
julia> struct Foo
           bar
           baz::Int
           qux::Float64
       end
\end{minted}



没有类型注释的字段默认为 \texttt{Any} 类型,所以可以包含任何类型的值。



类型为 \texttt{Foo} 的新对象通过将 \texttt{Foo} 类型对象像函数一样应用于其字段的值来创建:




\begin{minted}{jlcon}
julia> foo = Foo("Hello, world.", 23, 1.5)
Foo("Hello, world.", 23, 1.5)

julia> typeof(foo)
Foo
\end{minted}



当像函数一样使用类型时,它被称为\emph{构造函数}。有两个构造函数会被自动生成(这些构造函数称为\emph{默认构造函数})。一个接受任何参数并通过调用 \hyperlink{1846942650946171605}{\texttt{convert}} 函数将它们转换为字段的类型,另一个接受与字段类型完全匹配的参数。两者都生成的原因是,这使得更容易添加新定义而不会在无意中替换默认构造函数。



由于 \texttt{bar} 字段在类型上不受限制,因此任何值都可以。但是 \texttt{baz} 的值必须可转换为 \texttt{Int} 类型:




\begin{minted}{jlcon}
julia> Foo((), 23.5, 1)
ERROR: InexactError: Int64(23.5)
Stacktrace:
[...]
\end{minted}



可以使用 \hyperlink{17481253338332315021}{\texttt{fieldnames}} 函数找到字段名称列表。




\begin{minted}{jlcon}
julia> fieldnames(Foo)
(:bar, :baz, :qux)
\end{minted}



可以使用传统的 \texttt{foo.bar} 表示法访问复合对象的字段值:




\begin{minted}{jlcon}
julia> foo.bar
"Hello, world."

julia> foo.baz
23

julia> foo.qux
1.5
\end{minted}



使用 \texttt{struct} 声明的对象都是\emph{不可变的},它们在构造后无法修改。一开始看来这很奇怪,但它有几个优点:



\begin{itemize}
\item 它可以更高效。某些 struct 可以被高效地打包到数组中,并且在某些情况下,编译器可以避免完全分配不可变对象。


\item 不可能违反由类型的构造函数提供的不变性。


\item 使用不可变对象的代码更容易推理。

\end{itemize}


不可变对象可以包含可变对象(比如数组)作为字段。那些被包含的对象将保持可变;只是不可变对象本身的字段不能更改为指向不同的对象。



如果需要,可以使用关键字 \hyperlink{15383430693516362700}{\texttt{mutable struct}} 声明可变复合对象,这将在下一节中讨论



没有字段的不可变复合类型是单态类型;这种类型只能有一个实例:




\begin{minted}{jlcon}
julia> struct NoFields
       end

julia> NoFields() === NoFields()
true
\end{minted}



\hyperlink{7974744969331231272}{\texttt{===}} 函数用来确认构造出来的「两个」\texttt{NoFields} 实例实际上是同一个。单态类型将在\hyperlink{14008188290941962431}{下面}进一步详细描述。



关于如何构造复合类型的实例还有很多要说的,但这种讨论依赖于\hyperlink{5603543911318150609}{参数类型}和\hyperlink{3842379394166369470}{方法},并且这是非常重要的,应该在专门的章节中讨论:\hyperlink{1489967485005487723}{构造函数}。



\hypertarget{17783679803569553227}{}


\section{可变复合类型}



如果使用 \texttt{mutable struct} 而不是 \texttt{struct} 声明复合类型,则它的实例可以被修改:




\begin{minted}{jlcon}
julia> mutable struct Bar
           baz
           qux::Float64
       end

julia> bar = Bar("Hello", 1.5);

julia> bar.qux = 2.0
2.0

julia> bar.baz = 1//2
1//2
\end{minted}



为了支持修改,这种对象通常分配在堆上,并且具有稳定的内存地址。可变对象就像一个小容器,随着时间的推移,可能保持不同的值,因此只能通过其地址可靠地识别。相反地,不可变类型的实例与特定字段值相关——仅字段值就告诉你该对象的所有内容。在决定是否使类型为可变类型时,请询问具有相同字段值的两个实例是否被视为相同,或者它们是否可能需要随时间独立更改。如果它们被认为是相同的,该类型就应该是不可变的。



总结一下,Julia 的两个基本属性定义了不变性:



\begin{itemize}
\item 不允许修改不可变类型的值。

\begin{itemize}
\item 对于位类型,这意味着值的位模式一旦设置将不再改变,并且该值是位类型的标识。


\item 对于复合类型,这意味着其字段值的标识将不再改变。当字段是位类型时,这意味着它们的位将不再改变,对于其值是可变类型(如数组)的字段,这意味着字段将始终引用相同的可变值,尽管该可变值的内容本身可能被修改。

\end{itemize}

\item 具有不可变类型的对象可以被编译器自由复制,因为其不可变性使得不可能以编程方式区分原始对象和副本。

\begin{itemize}
\item 特别地,这意味着足够小的不可变值(如整数和浮点数)通常在寄存器(或栈分配)中传递给函数。


\item 另一方面,可变值是堆分配的,并作为指向堆分配值的指针传递给函数,除非编译器确定没有办法知道这不是正在发生的事情。

\end{itemize}
\end{itemize}


\hypertarget{11930598139930328403}{}


\section{已声明的类型}



前面章节中讨论的三种类型(抽象、原始、复合)实际上都是密切相关的。它们共有相同的关键属性:



\begin{itemize}
\item 它们都是显式声明的。


\item 它们都具有名称。


\item 它们都已经显式声明超类型。


\item 它们可以有参数。

\end{itemize}


由于这些共有属性,它们在内部表现为相同概念 \texttt{DataType} 的实例,其是任何这些类型的类型:




\begin{minted}{jlcon}
julia> typeof(Real)
DataType

julia> typeof(Int)
DataType
\end{minted}



\texttt{DataType} 可以是抽象的或具体的。它如果是具体的,就具有指定的大小、存储布局和字段名称(可选)。因此,原始类型是具有非零大小的 \texttt{DataType},但没有字段名称。复合类型是具有字段名称或者为空(大小为零)的 \texttt{DataType}。



每一个具体的值在系统里都是某个 \texttt{DataType} 的实例。



\hypertarget{5785180493438363730}{}


\section{类型共用体}



类型共用体是一种特殊的抽象类型,它包含作为对象的任何参数类型的所有实例,使用特殊\hyperlink{5087820771052303592}{\texttt{Union}}关键字构造:




\begin{minted}{jlcon}
julia> IntOrString = Union{Int,AbstractString}
Union{Int64, AbstractString}

julia> 1 :: IntOrString
1

julia> "Hello!" :: IntOrString
"Hello!"

julia> 1.0 :: IntOrString
ERROR: TypeError: in typeassert, expected Union{Int64, AbstractString}, got a value of type Float64
\end{minted}



许多语言都有内建的共用体结构来推导类型;Julia 简单地将它暴露给程序员。Julia 编译器能在 \texttt{Union} 类型只具有少量类型\footnotemark[1]的情况下生成高效的代码,方法是为每个可能类型的不同分支都生成专用代码。



\texttt{Union} 类型的一种特别有用的情况是 \texttt{Union\{T, Nothing\}},其中 \texttt{T} 可以是任何类型,\hyperlink{13508459519898889544}{\texttt{Nothing}} 是单态类型,其唯一实例是对象 \hyperlink{9331422207248206047}{\texttt{nothing}}。此模式是其它语言中 \href{https://en.wikipedia.org/wiki/Nullable\_type}{\texttt{Nullable}、\texttt{Option} 或 \texttt{Maybe}} 类型在 Julia 的等价。通过将函数参数或字段声明为 \texttt{Union\{T, Nothing\}},可以将其设置为类型为 \texttt{T} 的值,或者 \texttt{nothing} 来表示没有值。有关详细信息,请参阅\hyperlink{11397816795210039176}{常见问题的此条目}。



\hypertarget{8667374522381748142}{}


\section{参数类型}



Julia 类型系统的一个重要和强大的特征是它是参数的:类型可以接受参数,因此类型声明实际上引入了一整套新类型——每一个参数值的可能组合引入一个新类型。许多语言支持某种版本的\href{https://en.wikipedia.org/wiki/Generic\_programming}{泛型编程},其中,可以指定操作泛型的数据结构和算法,而无需指定所涉及的确切类型。例如,某些形式的泛型编程存在于 ML、Haskell、Ada、Eiffel、C++、Java、C\#、F\#、和 Scala 中,这只是其中的一些例子。这些语言中的一些支持真正的参数多态(例如 ML、Haskell、Scala),而其它语言基于模板的泛型编程风格(例如 C++、Java)。由于在不同语言中有多种不同种类的泛型编程和参数类型,我们甚至不会尝试将 Julia 的参数类型与其它语言的进行比较,而是专注于解释 Julia 系统本身。然而,我们将注意到,因为 Julia 是动态类型语言并且不需要在编译时做出所有类型决定,所以许多在静态参数类型系统中遇到的传统困难可以被相对容易地处理。



所有已声明的类型(\texttt{DataType} 类型)都可被参数化,在每种情况下都使用一样的语法。我们将按一下顺序讨论它们:首先是参数复合类型,接着是参数抽象类型,最后是参数原始类型。



\hypertarget{4169334510369531373}{}


\subsection{参数复合类型}



类型参数在类型名称后引入,用大括号扩起来:




\begin{minted}{jlcon}
julia> struct Point{T}
           x::T
           y::T
       end
\end{minted}



此声明定义了一个新的参数类型,\texttt{Point\{T\}},拥有类型为 \texttt{T} 的两个「坐标」。有人可能会问 \texttt{T} 是什么?嗯,这恰恰是参数类型的重点:它可以是任何类型(或者任何位类型值,虽然它实际上在这里显然用作类型)。\texttt{Point\{Float64\}} 是一个具体类型,该类型等价于通过用 \hyperlink{5027751419500983000}{\texttt{Float64}} 替换 \texttt{Point} 的定义中的 \texttt{T} 所定义的类型。因此,单独这一个声明实际上声明了无限个类型:\texttt{Point\{Float64\}},\texttt{Point\{AbstractString\}},\texttt{Point\{Int64\}},等等。这些类型中的每一个类型现在都是可用的具体类型:




\begin{minted}{jlcon}
julia> Point{Float64}
Point{Float64}

julia> Point{AbstractString}
Point{AbstractString}
\end{minted}



\texttt{Point\{Float64\}} 类型是坐标为 64 位浮点值的点,而 \texttt{Point\{AbstractString\}} 类型是「坐标」为字符串对象(请参阅 \href{@id man-strings}{Strings})的「点」。



\texttt{Point} 本身也是一个有效的类型对象,包括所有实例 \texttt{Point\{Float64\}}、\texttt{Point\{AbstractString\}} 等作为子类型:




\begin{minted}{jlcon}
julia> Point{Float64} <: Point
true

julia> Point{AbstractString} <: Point
true
\end{minted}



当然,其他类型不是它的子类型:




\begin{minted}{jlcon}
julia> Float64 <: Point
false

julia> AbstractString <: Point
false
\end{minted}



\texttt{Point} 不同 \texttt{T} 值所声明的具体类型之间,不能互相作为子类型:




\begin{minted}{jlcon}
julia> Point{Float64} <: Point{Int64}
false

julia> Point{Float64} <: Point{Real}
false
\end{minted}



\begin{quote}
\textbf{Warning}

最后一点\emph{非常}重要:即使 \texttt{Float64 <: Real} 也\textbf{没有} \texttt{Point\{Float64\} <: Point\{Real\}}。

\end{quote}


换成类型理论说法,Julia 的类型参数是\emph{不变的},而不是\href{https://en.wikipedia.org/wiki/Covariance\_and\_contravariance\_\%28computer\_science\%29}{协变的(或甚至是逆变的)}。这是出于实际原因:虽然任何 \texttt{Point\{Float64\}} 的实例在概念上也可能像是 \texttt{Point\{Real\}} 的实例,但这两种类型在内存中有不同的表示:



\begin{itemize}
\item \texttt{Point\{Float64\}} 的实例可以紧凑而高效地表示为一对 64 位立即数;


\item \texttt{Point\{Real\}} 的实例必须能够保存任何一对 \hyperlink{6175959395021454412}{\texttt{Real}} 的实例。由于 \texttt{Real} 实例的对象可以具有任意的大小和结构,\texttt{Point\{Real\}} 的实例实际上必须表示为一对指向单独分配的 \texttt{Real} 对象的指针。

\end{itemize}


在数组的情况下,能够以立即数存储 \texttt{Point\{Float64\}} 对象会极大地提高效率:\texttt{Array\{Float64\}} 可以存储为一段 64 位浮点值组成的连续内存块,而 \texttt{Array\{Real\}} 必须是一个由指向单独分配的 \hyperlink{6175959395021454412}{\texttt{Real}} 的指针组成的数组——这可能是 \href{https://en.wikipedia.org/wiki/Object\_type\_\%28object-oriented\_programming\%29\#Boxing}{boxed} 64 位浮点值,但也可能是任意庞大和复杂的对象,且其被声明为 \texttt{Real} 抽象类型的表示。



由于 \texttt{Point\{Float64\}} 不是 \texttt{Point\{Real\}} 的子类型,下面的方法不适用于类型为 \texttt{Point\{Float64\}} 的参数:




\begin{minted}{julia}
function norm(p::Point{Real})
    sqrt(p.x^2 + p.y^2)
end
\end{minted}



一种正确的方法来定义一个接受类型的所有参数的方法,\texttt{Point\{T\}}其中\texttt{T}是一个子类型\hyperlink{6175959395021454412}{\texttt{Real}}:




\begin{minted}{julia}
function norm(p::Point{<:Real})
    sqrt(p.x^2 + p.y^2)
end
\end{minted}



(等效地,另一种定义方法 \texttt{function norm(p::Point\{T\} where T<:Real)} 或 \texttt{function norm(p::Point\{T\}) where T<:Real};查看 \hyperlink{11072845175692859046}{UnionAll 类型}。)



稍后将在\hyperlink{3842379394166369470}{方法}中讨论更多示例。



如何构造一个 \texttt{Point} 对象?可以为复合类型定义自定义的构造函数,这将在\hyperlink{1489967485005487723}{构造函数}中详细讨论,但在没有任何特别的构造函数声明的情况下,有两种默认方式可以创建新的复合对象,一种是显式地给出类型参数,另一种是通过传给对象构造函数的参数隐含地给出。



由于 \texttt{Point\{Float64\}} 类型等价于在 \texttt{Point} 声明时用 \hyperlink{5027751419500983000}{\texttt{Float64}} 替换 \texttt{T} 得到的具体类型,它可以相应地作为构造函数使用:




\begin{minted}{jlcon}
julia> Point{Float64}(1.0, 2.0)
Point{Float64}(1.0, 2.0)

julia> typeof(ans)
Point{Float64}
\end{minted}



对于默认的构造函数,必须为每个字段提供一个参数:




\begin{minted}{jlcon}
julia> Point{Float64}(1.0)
ERROR: MethodError: no method matching Point{Float64}(::Float64)
[...]

julia> Point{Float64}(1.0,2.0,3.0)
ERROR: MethodError: no method matching Point{Float64}(::Float64, ::Float64, ::Float64)
[...]
\end{minted}



参数类型只生成一个默认的构造函数,因为它无法覆盖。这个构造函数接受任何参数并将它们转换为字段的类型。



在许多情况下,提供想要构造的 \texttt{Point} 对象的类型是多余的,因为构造函数调用参数的类型已经隐式地提供了类型信息。因此,你也可以将 \texttt{Point} 本身用作构造函数,前提是参数类型 \texttt{T} 的隐含值是确定的:




\begin{minted}{jlcon}
julia> Point(1.0,2.0)
Point{Float64}(1.0, 2.0)

julia> typeof(ans)
Point{Float64}

julia> Point(1,2)
Point{Int64}(1, 2)

julia> typeof(ans)
Point{Int64}
\end{minted}



在 \texttt{Point} 的例子中,当且仅当 \texttt{Point} 的两个参数类型相同时,\texttt{T} 的类型才确实是隐含的。如果不是这种情况,构造函数将失败并出现 \hyperlink{68769522931907606}{\texttt{MethodError}}:




\begin{minted}{jlcon}
julia> Point(1,2.5)
ERROR: MethodError: no method matching Point(::Int64, ::Float64)
Closest candidates are:
  Point(::T, !Matched::T) where T at none:2
\end{minted}



可以定义适当处理此类混合情况的函数构造方法,将在后面的\hyperlink{1489967485005487723}{构造函数}中讨论。



\hypertarget{2167251626535554270}{}


\subsection{参数抽象类型}



参数抽象类型声明以非常相似的方式声明了一族抽象类型:




\begin{minted}{jlcon}
julia> abstract type Pointy{T} end
\end{minted}



在此声明中,对于每个类型或整数值 \texttt{T},\texttt{Pointy\{T\}} 都是不同的抽象类型。与参数复合类型一样,每个此类型的实例都是 \texttt{Pointy} 的子类型:




\begin{minted}{jlcon}
julia> Pointy{Int64} <: Pointy
true

julia> Pointy{1} <: Pointy
true
\end{minted}



参数抽象类型是不变的,就像参数复合类型:




\begin{minted}{jlcon}
julia> Pointy{Float64} <: Pointy{Real}
false

julia> Pointy{Real} <: Pointy{Float64}
false
\end{minted}



符号 \texttt{Pointy\{<:Real\}} 可用于表示\emph{协变}类型的 Julia 类似物,而 \texttt{Pointy\{>:Int\}} 类似于\emph{逆变}类型,但从技术上讲,它们都代表了类型的\emph{集合}(参见 \hyperlink{11072845175692859046}{UnionAll 类型})。




\begin{minted}{jlcon}
julia> Pointy{Float64} <: Pointy{<:Real}
true

julia> Pointy{Real} <: Pointy{>:Int}
true
\end{minted}



正如之前的普通抽象类型用于在具体类型上创建实用的类型层次结构一样,参数抽象类型在参数复合类型上具有相同的用途。例如,我们可以将 \texttt{Point\{T\}} 声明为 \texttt{Pointy\{T\}} 的子类型,如下所示:




\begin{minted}{jlcon}
julia> struct Point{T} <: Pointy{T}
           x::T
           y::T
       end
\end{minted}



鉴于此类声明,对每个 \texttt{T},都有 \texttt{Point\{T\}} 是 \texttt{Pointy\{T\}} 的子类型:




\begin{minted}{jlcon}
julia> Point{Float64} <: Pointy{Float64}
true

julia> Point{Real} <: Pointy{Real}
true

julia> Point{AbstractString} <: Pointy{AbstractString}
true
\end{minted}



下面的关系依然不变:




\begin{minted}{jlcon}
julia> Point{Float64} <: Pointy{Real}
false

julia> Point{Float64} <: Pointy{<:Real}
true
\end{minted}



参数抽象类型(比如 \texttt{Pointy})的用途是什么?考虑一下如果点都在对角线 \emph{x = y} 上,那我们创建的点的实现可以只有一个坐标:




\begin{minted}{jlcon}
julia> struct DiagPoint{T} <: Pointy{T}
           x::T
       end
\end{minted}



现在,\texttt{Point\{Float64\}} 和 \texttt{DiagPoint\{Float64\}} 都是抽象 \texttt{Pointy\{Float64\}} 的实现,每个类型 \texttt{T} 的其它可能选择与之类似。这允许对被所有 \texttt{Pointy} 对象共享的公共接口进行编程,接口都由 \texttt{Point} 和 \texttt{DiagPoint} 实现。但是,直到我们在下一节\hyperlink{3842379394166369470}{方法}中引入方法和分派前,这无法完全证明。



有时,类型参数取遍所有可能类型也许是无意义的。在这种情况下,可以像这样约束 \texttt{T} 的范围:




\begin{minted}{jlcon}
julia> abstract type Pointy{T<:Real} end
\end{minted}



With such a declaration, it is acceptable to use any type that is a subtype of \hyperlink{6175959395021454412}{\texttt{Real}} in place of \texttt{T}, but not types that are not subtypes of \texttt{Real}:




\begin{minted}{jlcon}
julia> Pointy{Float64}
Pointy{Float64}

julia> Pointy{Real}
Pointy{Real}

julia> Pointy{AbstractString}
ERROR: TypeError: in Pointy, in T, expected T<:Real, got Type{AbstractString}

julia> Pointy{1}
ERROR: TypeError: in Pointy, in T, expected T<:Real, got a value of type Int64
\end{minted}



参数化复合类型的类型参数可用相同的方式限制:




\begin{minted}{julia}
struct Point{T<:Real} <: Pointy{T}
    x::T
    y::T
end
\end{minted}



在这里给出一个真实示例,展示了所有这些参数类型机制如何发挥作用,下面是 Julia 的不可变类型 \hyperlink{8304566144531167610}{\texttt{Rational}} 的实际定义(除了我们为了简单起见省略了的构造函数),用来表示准确的整数比例:




\begin{minted}{julia}
struct Rational{T<:Integer} <: Real
    num::T
    den::T
end
\end{minted}



只有接受整数值的比例才是有意义的,因此参数类型 \texttt{T} 被限制为 \hyperlink{8469131683393450448}{\texttt{Integer}} 的子类型,又整数的比例代表实数轴上的值,因此任何 \hyperlink{8304566144531167610}{\texttt{Rational}} 都是抽象 \hyperlink{6175959395021454412}{\texttt{Real}} 的实现。



\hypertarget{5158816437121320312}{}


\subsection{元组类型}



元组类型是函数参数的抽象——不是函数本身的。函数参数的突出特征是它们的顺序和类型。因此,元组类型类似于参数化的不可变类型,其中每个参数都是一个字段的类型。例如,二元元组类型类似于以下不可变类型:




\begin{minted}{julia}
struct Tuple2{A,B}
    a::A
    b::B
end
\end{minted}



然而,有三个主要差异:



\begin{itemize}
\item 元组类型可以具有任意数量的参数。


\item 元组类型的参数是\emph{协变的}:\texttt{Tuple\{Int\}} 是 \texttt{Tuple\{Any\}} 的子类型。因此,\texttt{Tuple\{Any\}} 被认为是一种抽象类型,且元组类型只有在它们的参数都是具体类型时才是具体类型。 


\item 元组没有字段名称; 字段只能通过索引访问。

\end{itemize}


元组值用括号和逗号书写。构造元组时,会根据需要生成适当的元组类型:




\begin{minted}{jlcon}
julia> typeof((1,"foo",2.5))
Tuple{Int64,String,Float64}
\end{minted}



请注意协变性的含义:




\begin{minted}{jlcon}
julia> Tuple{Int,AbstractString} <: Tuple{Real,Any}
true

julia> Tuple{Int,AbstractString} <: Tuple{Real,Real}
false

julia> Tuple{Int,AbstractString} <: Tuple{Real,}
false
\end{minted}



直观地,这对应于函数参数的类型是函数签名(当函数签名匹配时)的子类型。



\hypertarget{5746534767732668628}{}


\subsection{变参元组类型}



元组类型的最后一个参数可以是特殊类型 \hyperlink{5941806424098279588}{\texttt{Vararg}},它表示任意数量的尾随参数:




\begin{minted}{jlcon}
julia> mytupletype = Tuple{AbstractString,Vararg{Int}}
Tuple{AbstractString,Vararg{Int64,N} where N}

julia> isa(("1",), mytupletype)
true

julia> isa(("1",1), mytupletype)
true

julia> isa(("1",1,2), mytupletype)
true

julia> isa(("1",1,2,3.0), mytupletype)
false
\end{minted}



请注意,\texttt{Vararg\{T\}} 对应于零个或更多的类型为 \texttt{T} 的元素。变参元组类型被用来表示变参方法接受的参数(请参阅\hyperlink{9965084594348935329}{变参函数})。



类型 \texttt{Vararg\{T,N\}} 对应于正好 \texttt{N} 个类型为 \texttt{T} 的元素。\texttt{NTuple\{N,T\}} 是 \texttt{Tuple\{Vararg\{T,N\}\}} 的别名,即包含正好 \texttt{N} 个类型为 \texttt{T} 元素的元组类型。



\hypertarget{6997324644254770141}{}


\subsection{具名元组类型}



具名元组是 \hyperlink{3845731488275720657}{\texttt{NamedTuple}} 类型的实例,该类型有两个参数:一个给出字段名称的符号元组,和一个给出字段类型的元组类型。




\begin{minted}{jlcon}
julia> typeof((a=1,b="hello"))
NamedTuple{(:a, :b),Tuple{Int64,String}}
\end{minted}



The \href{@ref}{\texttt{@NamedTuple}} macro provides a more convenient \texttt{struct}-like syntax for declaring \texttt{NamedTuple} types via \texttt{key::Type} declarations, where an omitted \texttt{::Type} corresponds to \texttt{::Any}.




\begin{minted}{jlcon}
julia> @NamedTuple{a::Int, b::String}
NamedTuple{(:a, :b),Tuple{Int64,String}}

julia> @NamedTuple begin
           a::Int
           b::String
       end
NamedTuple{(:a, :b),Tuple{Int64,String}}
\end{minted}



\texttt{NamedTuple} 类型可以用作构造函数,接受一个单独的元组作为参数。构造出来的 \texttt{NamedTuple} 类型可以是具体类型,如果参数都被指定,也可以是只由字段名称所指定的类型:




\begin{minted}{jlcon}
julia> @NamedTuple{a::Float32,b::String}((1,""))
(a = 1.0f0, b = "")

julia> NamedTuple{(:a, :b)}((1,""))
(a = 1, b = "")
\end{minted}



如果指定了字段类型,参数会被转换。否则,就直接使用参数的类型。



\hypertarget{647919389478144252}{}


\subsection{单态类型}



这里必须提到一种特殊的抽象类型:单态类型。对于每个类型 \texttt{T},「单态类型」\texttt{Type\{T\}} 是个抽象类型且唯一的实例就是对象 \texttt{T}。由于定义有点难以解释,让我们看一些例子:




\begin{minted}{jlcon}
julia> isa(Float64, Type{Float64})
true

julia> isa(Real, Type{Float64})
false

julia> isa(Real, Type{Real})
true

julia> isa(Float64, Type{Real})
false
\end{minted}



换种说法,\hyperlink{7066325108767373297}{\texttt{isa(A,Type\{B\})}} 为真当且仅当 \texttt{A} 与 \texttt{B} 是同一对象且该对象是一个类型。不带参数时,\texttt{Type} 是个抽象类型,所有类型对象都是它的实例,当然也包括单态类型:




\begin{minted}{jlcon}
julia> isa(Type{Float64}, Type)
true

julia> isa(Float64, Type)
true

julia> isa(Real, Type)
true
\end{minted}



只有对象是类型时,才是 \texttt{Type} 的实例:




\begin{minted}{jlcon}
julia> isa(1, Type)
false

julia> isa("foo", Type)
false
\end{minted}



在我们讨论\hyperlink{5820282638415739482}{参数方法}和\hyperlink{10374023657104680331}{类型转换}之前,很难解释单态类型的作用,但简而言之,它允许针对特定类型\emph{值}专门指定函数行为。这对于编写方法(尤其是参数方法)很有用,这些方法的行为取决于作为显式参数给出的类型,而不是隐含在它的某个参数的类型中。



一些流行的语言有单态类型,比如 Haskell、Scala 和 Ruby。在一般用法中,术语「单态类型」指的是唯一实例为单个值的类型。这定义适用于 Julia 的单态类型,但需要注意的是 Julia 里只有类型对象具有对应的单态类型。



\hypertarget{2858316496912626713}{}


\subsection{参数原始类型}



原始类型也可以参数化声明,例如,指针都能表示为原始类型,其在 Julia 中以如下方式声明:




\begin{minted}{julia}
# 32-bit system:
primitive type Ptr{T} 32 end

# 64-bit system:
primitive type Ptr{T} 64 end
\end{minted}



与典型的参数复合类型相比,此声明中略显奇怪的特点是类型参数 \texttt{T} 并未在类型本身的定义里使用——它实际上只是一个抽象的标记,定义了一整族具有相同结构的类型,类型间仅由它们的类型参数来区分。因此,\texttt{Ptr\{Float64\}} 和 \texttt{Ptr\{Int64\}} 是不同的类型,就算它们具有相同的表示。当然,所有特定的指针类型都是总类型 \hyperlink{10630331440513004826}{\texttt{Ptr}} 的子类型:




\begin{minted}{jlcon}
julia> Ptr{Float64} <: Ptr
true

julia> Ptr{Int64} <: Ptr
true
\end{minted}



\hypertarget{1421020605730253291}{}


\section{UnionAll 类型}



我们已经说过像 \texttt{Ptr} 这样的参数类型充当它所有实例(\texttt{Ptr\{Int64\}} 等)的超类型。这是如何工作的?\texttt{Ptr} 本身不能是普通的数据类型,因为在不知道引用数据的类型时,该类型显然不能用于存储器操作。答案是 \texttt{Ptr}(或其它参数类型像 \texttt{Array})是一种不同种类的类型,称为 \hyperlink{13291956087044414878}{\texttt{UnionAll}} 类型。这种类型表示某些参数的所有值的类型的\emph{迭代并集}。



\texttt{UnionAll} 类型通常使用关键字 \texttt{where} 编写。例如,\texttt{Ptr} 可以更精确地写为 \texttt{Ptr\{T\} where T},也就是对于 \texttt{T} 的某些值,所有类型为 \texttt{Ptr\{T\}} 的值。在这种情况下,参数 \texttt{T} 也常被称为「类型变量」,因为它就像一个取值范围为类型的变量。每个 \texttt{where} 只引入一个类型变量,因此在具有多个参数的类型中这些表达式会被嵌套,例如 \texttt{Array\{T,N\} where N where T}。



类型应用语法 \texttt{A\{B,C\}} 要求 \texttt{A} 是个 \texttt{UnionAll} 类型,并先把 \texttt{B} 替换为 \texttt{A} 中最外层的类型变量。结果应该是另一个 \texttt{UnionAll} 类型,然后把 \texttt{C} 替换为该类型的类型变量。所以 \texttt{A\{B,C\}} 等价于 \texttt{A\{B\}\{C\}}。这解释了为什么可以部分实例化一个类型,比如 \texttt{Array\{Float64\}}:第一个参数已经被固定,但第二个参数仍取遍所有可能值。通过使用 \texttt{where} 语法,任何参数子集都能被固定。例如,所有一维数组的类型可以写为 \texttt{Array\{T,1\} where T}。



类型变量可以用子类型关系来加以限制。\texttt{Array\{T\} where T<:Integer} 指的是元素类型是某种 \hyperlink{8469131683393450448}{\texttt{Integer}} 的所有数组。语法 \texttt{Array\{<:Integer\}} 是 \texttt{Array\{T\} where T<:Integer} 的便捷的缩写。类型变量可同时具有上下界。\texttt{Array\{T\} where Int<:T<:Number} 指的是元素类型为能够包含 \texttt{Int} 的 \hyperlink{1990584313715697055}{\texttt{Number}} 的所有数组(因为 \texttt{T} 至少和 \texttt{Int} 一样大)。语法 \texttt{where T>:Int} 也能用来只指定类型变量的下界,且 \texttt{Array\{>:Int\}} 等价于 \texttt{Array\{T\} where T>:Int}。



由于 \texttt{where} 表达式可以嵌套,类型变量界可以引用更外层的类型变量。比如 \texttt{Tuple\{T,Array\{S\}\} where S<:AbstractArray\{T\} where T<:Real} 指的是二元元组,其第一个元素是某个 \hyperlink{6175959395021454412}{\texttt{Real}},而第二个元素是任意种类的数组 \texttt{Array},且该数组的元素类型包含于第一个元组元素的类型。



\texttt{where} 关键字本身可以嵌套在更复杂的声明里。例如,考虑由以下声明创建的两个类型:




\begin{minted}{jlcon}
julia> const T1 = Array{Array{T,1} where T, 1}
Array{Array{T,1} where T,1}

julia> const T2 = Array{Array{T,1}, 1} where T
Array{Array{T,1},1} where T
\end{minted}



类型 \texttt{T1} 定义了由一维数组组成的一维数组;每个内部数组由相同类型的对象组成,但此类型对于不同内部数组可以不同。另一方面,类型 \texttt{T2} 定义了由一维数组组成的一维数组,其中的每个内部数组必须具有相同的类型。请注意,\texttt{T2} 是个抽象类型,比如 \texttt{Array\{Array\{Int,1\},1\} <: T2},而 \texttt{T1} 是个具体类型。因此,\texttt{T1} 可由零参数构造函数 \texttt{a=T1()} 构造,但 \texttt{T2} 不行。



命名此类型有一种方便的语法,类似于函数定义语法的简短形式:




\begin{minted}{julia}
Vector{T} = Array{T,1}
\end{minted}



这等价于 \texttt{const Vector = Array\{T,1\} where T}。编写 \texttt{Vector\{Float64\}} 等价于编写 \texttt{Array\{Float64,1\}},总类型 \texttt{Vector} 具有所有 \texttt{Array} 对象的实例,其中 \texttt{Array} 对象的第二个参数——数组维数——是 1,而不考虑元素类型是什么。在参数类型必须总被完整指定的语言中,这不是特别有用,但在 Julia 中,这允许只编写 \texttt{Vector} 来表示包含任何元素类型的所有一维密集数组的抽象类型。



\hypertarget{11325503690546832900}{}


\section{类型别名}



有时为一个已经可表达的类型引入新名称是很方便的。这可通过一个简单的赋值语句完成。例如,\texttt{UInt} 是 \hyperlink{8690996847580776341}{\texttt{UInt32}} 或 \hyperlink{5500998675195555601}{\texttt{UInt64}} 的别名,因为它的大小与系统上的指针大小是相适应的。




\begin{minted}{jlcon}
# 32-bit system:
julia> UInt
UInt32

# 64-bit system:
julia> UInt
UInt64
\end{minted}



在 \texttt{base/boot.jl} 中,通过以下代码实现:




\begin{minted}{julia}
if Int === Int64
    const UInt = UInt64
else
    const UInt = UInt32
end
\end{minted}



当然,这依赖于 \texttt{Int} 的别名,但它被预定义成正确的类型—— \hyperlink{10103694114785108551}{\texttt{Int32}} 或 \hyperlink{7720564657383125058}{\texttt{Int64}}。



(注意,与 \texttt{Int} 不同,\texttt{Float} 不作为特定大小的 \hyperlink{11465394427882483091}{\texttt{AbstractFloat}} 类型的别名而存在。与整数寄存器不同,浮点数寄存器大小由 IEEE-754 标准指定,而 \texttt{Int} 的大小反映了该机器上本地指针的大小。)



\hypertarget{17381545984694686313}{}


\section{类型操作}



因为 Julia 中的类型本身就是对象,所以一般的函数可以对它们进行操作。已经引入了一些对于使用或探索类型特别有用的函数,例如 \texttt{<:} 运算符,它表示其左操作数是否为其右操作数的子类型。



\hyperlink{7066325108767373297}{\texttt{isa}} 函数测试对象是否具有给定类型并返回 true 或 false:




\begin{minted}{jlcon}
julia> isa(1, Int)
true

julia> isa(1, AbstractFloat)
false
\end{minted}



已经在手册各处的示例中使用的 \hyperlink{13440452181855594120}{\texttt{typeof}} 函数返回其参数的类型。如上所述,因为类型都是对象,所以它们也有类型,我们可以询问它们的类型:




\begin{minted}{jlcon}
julia> typeof(Rational{Int})
DataType

julia> typeof(Union{Real,String})
Union
\end{minted}



如果我们重复这个过程会怎样?一个类型的类型是什么?碰巧,每个类型都是复合值,因此都具有 \texttt{DataType} 类型:




\begin{minted}{jlcon}
julia> typeof(DataType)
DataType

julia> typeof(Union)
DataType
\end{minted}



\texttt{DataType} 是它自己的类型。



另一个适用于某些类型的操作是 \hyperlink{12192788431675298651}{\texttt{supertype}},它显示了类型的超类型。只有已声明的类型(\texttt{DataType})才有明确的超类型:




\begin{minted}{jlcon}
julia> supertype(Float64)
AbstractFloat

julia> supertype(Number)
Any

julia> supertype(AbstractString)
Any

julia> supertype(Any)
Any
\end{minted}



如果将 \hyperlink{12192788431675298651}{\texttt{supertype}} 应用于其它类型对象(或非类型对象),则会引发 \hyperlink{68769522931907606}{\texttt{MethodError}}:




\begin{minted}{jlcon}
julia> supertype(Union{Float64,Int64})
ERROR: MethodError: no method matching supertype(::Type{Union{Float64, Int64}})
Closest candidates are:
[...]
\end{minted}



\hypertarget{10200728548672135026}{}


\section{自定义 pretty-printing}



通常,人们会想要自定义显示类型实例的方式。这可通过重载 \hyperlink{14071376285304310153}{\texttt{show}} 函数来完成。举个例子,假设我们定义一个类型来表示极坐标形式的复数:




\begin{minted}{jlcon}
julia> struct Polar{T<:Real} <: Number
           r::T
           Θ::T
       end

julia> Polar(r::Real,Θ::Real) = Polar(promote(r,Θ)...)
Polar
\end{minted}



在这里,我们添加了一个自定义的构造函数,这样就可以接受不同 \hyperlink{6175959395021454412}{\texttt{Real}} 类型的参数并将它们类型提升为共同类型(请参阅\hyperlink{1489967485005487723}{构造函数}和\hyperlink{10374023657104680331}{类型转换和类型提升})。(当然,为了让它表现地像个 \hyperlink{1990584313715697055}{\texttt{Number}},我们需要定义许多其它方法,例如 \texttt{+}、\texttt{*}、\texttt{one}、\texttt{zero} 及类型提升规则等。)默认情况下,此类型的实例只是相当简单地显示有关类型名称和字段值的信息,比如,\texttt{Polar\{Float64\}(3.0,4.0)}。



如果我们希望它显示为 \texttt{3.0 * exp(4.0im)},我们将定义以下方法来将对象打印到给定的输出对象 \texttt{io}(其代表文件、终端、及缓冲区等;请参阅\hyperlink{4176621353987521289}{网络和流}):




\begin{minted}{jlcon}
julia> Base.show(io::IO, z::Polar) = print(io, z.r, " * exp(", z.Θ, "im)")
\end{minted}



\texttt{Polar} 对象的输出可以被更精细地控制。特别是,人们有时想要啰嗦的多行打印格式,用于在 REPL 和其它交互式环境中显示单个对象,以及一个更紧凑的单行格式,用于 \hyperlink{8248717042415202230}{\texttt{print}} 函数或在作为其它对象(比如一个数组)的部分是显示该对象。虽然在两种情况下默认都会调用 \texttt{show(io, z)} 函数,你仍可以定义一个\emph{不同}的多行格式来显示单个对象,这通过重载三参数形式的 \texttt{show} 函数,该函数接收 \texttt{text/plain} MIME 类型(请参阅 \hyperlink{9485638019478733873}{多媒体 I/O})作为它的第二个参数,举个例子:




\begin{minted}{jlcon}
julia> Base.show(io::IO, ::MIME"text/plain", z::Polar{T}) where{T} =
           print(io, "Polar{$T} complex number:\n   ", z)
\end{minted}



(请注意 \texttt{print(..., z)} 在这里调用的是双参数的 \texttt{show(io, z)} 方法。)这导致:




\begin{minted}{jlcon}
julia> Polar(3, 4.0)
Polar{Float64} complex number:
   3.0 * exp(4.0im)

julia> [Polar(3, 4.0), Polar(4.0,5.3)]
2-element Array{Polar{Float64},1}:
 3.0 * exp(4.0im)
 4.0 * exp(5.3im)
\end{minted}



其中单行格式的 \texttt{show(io, z)} 仍用于由 \texttt{Polar} 值组成的数组。从技术上讲,REPL 调用 \texttt{display(z)} 来显示单行的执行结果,其默认为 \texttt{show(stdout, MIME({\textquotedbl}text/plain{\textquotedbl}), z)},而后者又默认为 \texttt{show(stdout, z)},但是你\emph{不应该}定义新的 \hyperlink{12073120410747960438}{\texttt{display}} 方法,除非你正在定义新的多媒体显示管理器(请参阅\hyperlink{9485638019478733873}{多媒体 I/O})。



此外,你还可以为其它 MIME 类型定义 \texttt{show} 方法,以便在支持的环境(比如 IJulia)中实现更丰富的对象显示(HTML、图像等)。例如,我们可以定义 \texttt{Polar} 对象的 HTML 显示格式,使其带有上标和斜体:




\begin{minted}{jlcon}
julia> Base.show(io::IO, ::MIME"text/html", z::Polar{T}) where {T} =
           println(io, "<code>Polar{$T}</code> complex number: ",
                   z.r, " <i>e</i><sup>", z.Θ, " <i>i</i></sup>")
\end{minted}



之后会在支持 HTML 显示的环境中自动使用 HTML 显示 \texttt{Polar} 对象,但如果你想,也可以手动调用 \texttt{show} 来获取 HTML 输出:




\begin{minted}{jlcon}
julia> show(stdout, "text/html", Polar(3.0,4.0))
<code>Polar{Float64}</code> complex number: 3.0 <i>e</i><sup>4.0 <i>i</i></sup>
\end{minted}





根据经验,单行 \texttt{show} 方法应为创建的显示对象打印有效的 Julia 表达式。当这个 \texttt{show} 方法包含中缀运算符时,比如上面的 \texttt{Polar} 的单行 \texttt{show} 方法里的乘法运算符(\texttt{*}),在作为另一个对象的部分打印时,它可能无法被正确解析。要查看此问题,请考虑下面的表达式对象(请参阅\hyperlink{10559372927865899180}{程序表示}),它代表 \texttt{Polar} 类型的特定实例的平方:




\begin{minted}{jlcon}
julia> a = Polar(3, 4.0)
Polar{Float64} complex number:
   3.0 * exp(4.0im)

julia> print(:($a^2))
3.0 * exp(4.0im) ^ 2
\end{minted}



因为运算符 \texttt{{\textasciicircum}} 的优先级高于 \texttt{*}(请参阅\hyperlink{1006859879084707050}{运算符的优先级与结合性}),所以此输出不忠实地表示了表达式 \texttt{a {\textasciicircum} 2},而该表达式等价于 \texttt{(3.0 * exp(4.0im)) {\textasciicircum} 2}。为了解决这个问题,我们必须为 \texttt{Base.show\_unquoted(io::IO, z::Polar, indent::Int, precedence::Int)} 创建一个自定义方法,在打印时,表达式对象会在内部调用它:




\begin{minted}{jlcon}
julia> function Base.show_unquoted(io::IO, z::Polar, ::Int, precedence::Int)
           if Base.operator_precedence(:*) <= precedence
               print(io, "(")
               show(io, z)
               print(io, ")")
           else
               show(io, z)
           end
       end

julia> :($a^2)
:((3.0 * exp(4.0im)) ^ 2)
\end{minted}



当正在调用的运算符的优先级大于等于乘法的优先级时,上面定义的方法会在 \texttt{show} 调用的两侧加上括号。这个检查允许在没有括号的情况下被正确解析的表达式(例如 \texttt{:(\$a + 2)} 和 \texttt{:(\$a == 2)})在打印时省略括号:




\begin{minted}{jlcon}
julia> :($a + 2)
:(3.0 * exp(4.0im) + 2)

julia> :($a == 2)
:(3.0 * exp(4.0im) == 2)
\end{minted}



在某些情况下,根据上下文调整 \texttt{show} 方法的行为是很有用的。这可通过 \hyperlink{13454403377667762339}{\texttt{IOContext}} 类型实现,它允许一起传递上下文属性和封装后的 IO 流。例如,我们可以在 \texttt{:compact} 属性设置为 \texttt{true} 时创建一个更短的表示,而在该属性为 \texttt{false} 或不存在时返回长的表示:




\begin{minted}{jlcon}
julia> function Base.show(io::IO, z::Polar)
           if get(io, :compact, false)
               print(io, z.r, "ℯ", z.Θ, "im")
           else
               print(io, z.r, " * exp(", z.Θ, "im)")
           end
       end
\end{minted}



当传入的 IO 流是设置了 \texttt{:compact}(译注:该属性还应当设置为 \texttt{true})属性的 \texttt{IOContext} 对象时,将使用这个新的紧凑表示。特别地,当打印具有多列的数组(由于水平空间有限)时就是这种情况:




\begin{minted}{jlcon}
julia> show(IOContext(stdout, :compact=>true), Polar(3, 4.0))
3.0ℯ4.0im

julia> [Polar(3, 4.0) Polar(4.0,5.3)]
1×2 Array{Polar{Float64},2}:
 3.0ℯ4.0im  4.0ℯ5.3im
\end{minted}



有关调整打印效果的常用属性列表,请参阅文档 \hyperlink{13454403377667762339}{\texttt{IOContext}}。



\hypertarget{764608991862279222}{}


\section{值类型}



在 Julia 中,你无法根据诸如 \texttt{true} 或 \texttt{false} 之类的\emph{值}进行分派。然而,你可以根据参数类型进行分派,Julia 允许你包含「plain bits」值(类型、符号、整数、浮点数和元组等)作为类型参数。\texttt{Array\{T,N\}} 里的维度参数就是一个常见的例子,在那里 \texttt{T} 是类型(比如 \hyperlink{5027751419500983000}{\texttt{Float64}}),而 \texttt{N} 只是个 \texttt{Int}。



你可以创建把值作为参数的自定义类型,并使用它们控制自定义类型的分派。为了说明这个想法,让我们引入参数类型 \texttt{Val\{x\}} 和构造函数 \texttt{Val(x) = Val\{x\}()},它可以作为一种习惯的方式来利用这种技术需要更精细的层次结构。这可以作为利用这种技术的惯用方式,而且不需要更精细的层次结构。



\hyperlink{1312938105781775871}{\texttt{Val}} 的定义为:




\begin{minted}{jlcon}
julia> struct Val{x}
       end

julia> Val(x) = Val{x}()
Val
\end{minted}



\texttt{Val} 的实现就只需要这些。一些 Julia 标准库里的函数接收 \texttt{Val} 的实例作为参数,你也可以使用它来编写你自己的函数,例如:




\begin{minted}{jlcon}
julia> firstlast(::Val{true}) = "First"
firstlast (generic function with 1 method)

julia> firstlast(::Val{false}) = "Last"
firstlast (generic function with 2 methods)

julia> firstlast(Val(true))
"First"

julia> firstlast(Val(false))
"Last"
\end{minted}



For consistency across Julia, the call site should always pass a \texttt{Val} \emph{instance} rather than using a \emph{type}, i.e., use \texttt{foo(Val(:bar))} rather than \texttt{foo(Val\{:bar\})}.



It{\textquotesingle}s worth noting that it{\textquotesingle}s extremely easy to mis-use parametric {\textquotedbl}value{\textquotedbl} types, including \texttt{Val}; in unfavorable cases, you can easily end up making the performance of your code much \emph{worse}.  In particular, you would never want to write actual code as illustrated above.  For more information about the proper (and improper) uses of \texttt{Val}, please read \hyperlink{17259605703392147735}{the more extensive discussion in the performance tips}.



\footnotetext[1]{「少数」由常数 \texttt{MAX\_UNION\_SPLITTING} 定义,目前设置为 4。

}


\hypertarget{12379207465798704957}{}


\chapter{方法}



我们回想一下,在\hyperlink{645008301484218813}{函数}中我们知道函数是这么一个对象,它把一组参数映射成一个返回值,或者当没有办法返回恰当的值时扔出一个异常。具有相同概念的函数或者运算,经常会根据参数类型的不同而进行有很大差异的实现:两个整数的加法与两个浮点数的加法是相当不一样的,整数与浮点数之间的加法也不一样。除了它们实现上的不同,这些运算都归在{\textquotedbl}加法{\textquotedbl}这么一个广义的概念之下,因此在 Julia 中这些行为都属于同一个对象:\texttt{+} 函数。



为了让对同样的概念使用许多不同的实现这件事更顺畅,函数没有必要马上全部都被定义,反而应该是一块一块地定义,为特定的参数类型和数量的组合提供指定的行为。对于一个函数的一个可能行为的定义叫做\emph{方法}。直到这里,我们只展示了那些只定了一个方法的,对参数的所有类型都适用的函数。但是方法定义的特征是不仅能表明参数的数量,也能表明参数的类型,并且能提供多个方法定义。当一个函数被应用于特殊的一组参数时,能用于这一组参数的最特定的方法会被使用。所以,函数的全体行为是他的不同的方法定义的行为的组合。如果这个组合被设计得好,即使方法们的实现之间会很不一样,函数的外部行为也会显得无缝而自洽。



当一个函数被应用时执行方法的选择被称为\emph{分派}。Julia 允许分派过程来基于给的参数的个数和所有的参数的类型来选择调用函数的哪个方法。这与传统的面对对象的语言不一样,面对对象语言的分派只基于第一参数,经常有特殊的参数语法并且有时是暗含而非显式写成一个参数。\footnotemark[1]使用函数的所有参数,而非只用第一个,来决定调用哪个方法被称为\href{https://en.wikipedia.org/wiki/Multiple\_dispatch}{多重分派}。多重分派对于数学代码来说特别有用,人工地将运算视为对于其中一个参数的属于程度比其他所有的参数都强的这个概念对于数学代码是几乎没有意义的:\texttt{x + y} 中的加法运算对 \texttt{x} 的属于程度比对 \texttt{y} 更强?一个数学运算符的实现普遍基于它所有的参数的类型。即使跳出数学运算,多重分派是对于结构和组织程序来说也是一个强大而方便的范式。



\footnotetext[1]{In C++ or Java, for example, in a method call like \texttt{obj.meth(arg1,arg2)}, the object obj {\textquotedbl}receives{\textquotedbl} the method call and is implicitly passed to the method via the \texttt{this} keyword, rather than as an explicit method argument. When the current \texttt{this} object is the receiver of a method call, it can be omitted altogether, writing just \texttt{meth(arg1,arg2)}, with \texttt{this} implied as the receiving object.

}


\hypertarget{17361934126771506898}{}


\section{定义方法}



直到这里,在我们的例子中,我们定义的函数只有一个不限制参数类型的方法。这种函数的行为就与传统动态类型语言中的函数一样。不过,我们已经在没有意识到的情况下已经使用了多重分派和方法:所有 Julia 标准函数和运算符,就像之前提到的 \texttt{+} 函数,都根据参数的类型和数量的不同组合而定义了大量方法。



当定义一个函数时,可以根据需要使用在\hyperlink{4168730090950432836}{复合类型}中介绍的 \texttt{::} 类型断言运算符来限制参数类型,




\begin{minted}{jlcon}
julia> f(x::Float64, y::Float64) = 2x + y
f (generic function with 1 method)
\end{minted}



这个函数只在 \texttt{x} 和 \texttt{y} 的类型都是 \hyperlink{5027751419500983000}{\texttt{Float64}} 的情况下才会被调用:




\begin{minted}{jlcon}
julia> f(2.0, 3.0)
7.0
\end{minted}



用其它任意的参数类型则会导致 \hyperlink{68769522931907606}{\texttt{MethodError}}:




\begin{minted}{jlcon}
julia> f(2.0, 3)
ERROR: MethodError: no method matching f(::Float64, ::Int64)
Closest candidates are:
  f(::Float64, !Matched::Float64) at none:1

julia> f(Float32(2.0), 3.0)
ERROR: MethodError: no method matching f(::Float32, ::Float64)
Closest candidates are:
  f(!Matched::Float64, ::Float64) at none:1

julia> f(2.0, "3.0")
ERROR: MethodError: no method matching f(::Float64, ::String)
Closest candidates are:
  f(::Float64, !Matched::Float64) at none:1

julia> f("2.0", "3.0")
ERROR: MethodError: no method matching f(::String, ::String)
\end{minted}



如同你所看到的,参数必须精确地是 \hyperlink{5027751419500983000}{\texttt{Float64}} 类型。其它数字类型,比如整数或者 32 位浮点数值,都不会自动转化成 64 位浮点数,字符串也不会解析成数字。由于 \texttt{Float64} 是一个具体类型,且在 Julia 中具体类型无法拥有子类,所以这种定义方式只能适用于函数的输入类型精确地是 \texttt{Float64} 的情况,但一个常见的做法是用抽象类型来定义通用的方法:




\begin{minted}{jlcon}
julia> f(x::Number, y::Number) = 2x - y
f (generic function with 2 methods)

julia> f(2.0, 3)
1.0
\end{minted}



用上面这种方式定义的方法可以接收任意一对 \hyperlink{1990584313715697055}{\texttt{Number}} 的实例参数,且它们不需要是同一类型的,只要求都是数值。如何根据不同的类型来做相应的处理就可以委托给表达式 \texttt{2x - y} 中的代数运算。



为了定义一个有多个方法的函数,只需简单定义这个函数多次,使用不同的参数数量和类型。函数的第一个方法定义会建立这个函数对象,后续的方法定义会添加新的方法到存在的函数对象中去。当函数被应用时,最符合参数的数量和类型的特定方法会被执行。所以,上面的两个方法定义在一起定义了函数\texttt{f}对于所有的一对虚拟类型\texttt{Number}实例的行为 – 但是针对一对\hyperlink{5027751419500983000}{\texttt{Float64}}值有不同的行为。如果一个参数是64位浮点数而另一个不是,\texttt{f(Float64,Float64)}方法不会被调用,而一定使用更加通用的\texttt{f(Number,Number)}方法:




\begin{minted}{jlcon}
julia> f(2.0, 3.0)
7.0

julia> f(2, 3.0)
1.0

julia> f(2.0, 3)
1.0

julia> f(2, 3)
1
\end{minted}



\texttt{2x + y} 定义只用于第一个情况,\texttt{2x - y} 定义用于其他的情况。没有使用任何自动的函数参数的指派或者类型转换:Julia中的所有转换都不是 magic 的,都是完全显式的。然而\hyperlink{10374023657104680331}{类型转换和类型提升}显示了足够先进的技术的智能应用能够与 magic 不可分辨到什么程度。\footnotemark[2] 对于非数字值,和比两个参数更多或者更少的情况,函数 \texttt{f} 并没有定义,应用会导致 \hyperlink{68769522931907606}{\texttt{MethodError}}:




\begin{minted}{jlcon}
julia> f("foo", 3)
ERROR: MethodError: no method matching f(::String, ::Int64)
Closest candidates are:
  f(!Matched::Number, ::Number) at none:1

julia> f()
ERROR: MethodError: no method matching f()
Closest candidates are:
  f(!Matched::Float64, !Matched::Float64) at none:1
  f(!Matched::Number, !Matched::Number) at none:1
\end{minted}



可以简单地看到对于函数存在哪些方法,通过在交互式会话中键入函数对象本身:




\begin{minted}{jlcon}
julia> f
f (generic function with 2 methods)
\end{minted}



这个输出告诉我们\texttt{f}是有两个方法的函数对象。为了找出那些方法的特征是什么,使用 \hyperlink{3025953302266245919}{\texttt{methods}}函数:




\begin{minted}{jlcon}
julia> methods(f)
# 2 methods for generic function "f":
[1] f(x::Float64, y::Float64) in Main at none:1
[2] f(x::Number, y::Number) in Main at none:1
\end{minted}



这表示\texttt{f}有两个方法,一个接受两个\texttt{Float64}参数一个接受两个\texttt{Number}类型的参数。它也显示了这些方法定义所在的文件和行数:因为这些方法是在REPL中定义的,我们得到了表面上的行数\texttt{none:1}.



没有\texttt{::}的类型声明,方法参数的类型默认为\texttt{Any},这就意味着没有约束,因为Julia中的所有的值都是抽象类型\texttt{Any}的实例。所以,我们可以为\texttt{f}定义一个接受所有的方法,像这样:




\begin{minted}{jlcon}
julia> f(x,y) = println("Whoa there, Nelly.")
f (generic function with 3 methods)

julia> f("foo", 1)
Whoa there, Nelly.
\end{minted}



这个接受所有的方法比其他的对一堆参数值的其他任意可能的方法定义更不专用。所以他只会被没有其他方法定义应用的一对参数调用。



虽然这像是一个简单的概念,基于值的类型的多重分派可能是Julia语言的一个最强大和中心特性。核心运算符都典型地含有很多方法:




\begin{minted}{jlcon}
julia> methods(+)
# 180 methods for generic function "+":
[1] +(x::Bool, z::Complex{Bool}) in Base at complex.jl:227
[2] +(x::Bool, y::Bool) in Base at bool.jl:89
[3] +(x::Bool) in Base at bool.jl:86
[4] +(x::Bool, y::T) where T<:AbstractFloat in Base at bool.jl:96
[5] +(x::Bool, z::Complex) in Base at complex.jl:234
[6] +(a::Float16, b::Float16) in Base at float.jl:373
[7] +(x::Float32, y::Float32) in Base at float.jl:375
[8] +(x::Float64, y::Float64) in Base at float.jl:376
[9] +(z::Complex{Bool}, x::Bool) in Base at complex.jl:228
[10] +(z::Complex{Bool}, x::Real) in Base at complex.jl:242
[11] +(x::Char, y::Integer) in Base at char.jl:40
[12] +(c::BigInt, x::BigFloat) in Base.MPFR at mpfr.jl:307
[13] +(a::BigInt, b::BigInt, c::BigInt, d::BigInt, e::BigInt) in Base.GMP at gmp.jl:392
[14] +(a::BigInt, b::BigInt, c::BigInt, d::BigInt) in Base.GMP at gmp.jl:391
[15] +(a::BigInt, b::BigInt, c::BigInt) in Base.GMP at gmp.jl:390
[16] +(x::BigInt, y::BigInt) in Base.GMP at gmp.jl:361
[17] +(x::BigInt, c::Union{UInt16, UInt32, UInt64, UInt8}) in Base.GMP at gmp.jl:398
...
[180] +(a, b, c, xs...) in Base at operators.jl:424
\end{minted}



多重分派和灵活的参数类型系统让Julia有能力抽象地表达高层级算法,而与实现细节解耦,也能生成高效而专用的代码来在运行中处理每个情况。



\hypertarget{8405464025999625028}{}


\section{方法歧义}



在一系列的函数方法定义时有可能没有单独的最专用的方法能适用于参数的某些组合:




\begin{minted}{jlcon}
julia> g(x::Float64, y) = 2x + y
g (generic function with 1 method)

julia> g(x, y::Float64) = x + 2y
g (generic function with 2 methods)

julia> g(2.0, 3)
7.0

julia> g(2, 3.0)
8.0

julia> g(2.0, 3.0)
ERROR: MethodError: g(::Float64, ::Float64) is ambiguous. Candidates:
  g(x::Float64, y) in Main at none:1
  g(x, y::Float64) in Main at none:1
Possible fix, define
  g(::Float64, ::Float64)
\end{minted}



这里\texttt{g(2.0,3.0)}的调用使用\texttt{g(Float64, Any)}和\texttt{g(Any, Float64)}都能处理,并且两个都不更加专用。在这样的情况下,Julia会扔出\hyperlink{68769522931907606}{\texttt{MethodError}}而非任意选择一个方法。你可以通过对交叉情况指定一个合适的方法来避免方法歧义:




\begin{minted}{jlcon}
julia> g(x::Float64, y::Float64) = 2x + 2y
g (generic function with 3 methods)

julia> g(2.0, 3)
7.0

julia> g(2, 3.0)
8.0

julia> g(2.0, 3.0)
10.0
\end{minted}



建议先定义没有歧义的方法,因为不这样的话,歧义就会存在,即使是暂时性的,直到更加专用的方法被定义。



在更加复杂的情况下,解决方法歧义会会涉及到设计的某一个元素;这个主题将会在\hyperlink{15846346227037149553}{下面}进行进一步的探索。



\hypertarget{14064657876301533350}{}


\section{参数方法}



方法定义可以视需要存在限定特征的类型参数:




\begin{minted}{jlcon}
julia> same_type(x::T, y::T) where {T} = true
same_type (generic function with 1 method)

julia> same_type(x,y) = false
same_type (generic function with 2 methods)
\end{minted}



第一个方法应用于两个参数都是同一个具体类型时,不管类型是什么,而第二个方法接受一切,涉及其他所有情况。所以,总得来说,这个定义了一个布尔函数来检查两个参数是否是同样的类型:




\begin{minted}{jlcon}
julia> same_type(1, 2)
true

julia> same_type(1, 2.0)
false

julia> same_type(1.0, 2.0)
true

julia> same_type("foo", 2.0)
false

julia> same_type("foo", "bar")
true

julia> same_type(Int32(1), Int64(2))
false
\end{minted}



这样的定义对应着那些类型签名是 \texttt{UnionAll} 类型的方法(参见 \hyperlink{11072845175692859046}{UnionAll 类型})。



在Julia中这种通过分派进行函数行为的定义是十分常见的,甚至是惯用的。方法类型参数并不局限于用作参数的类型:他们可以用在任意地方,只要值会在函数或者函数体的特征中。这里有个例子,例子中方法类型参数\texttt{T}用作方法特征中的参数类型\texttt{Vector\{T\}}的类型参数:




\begin{minted}{jlcon}
julia> myappend(v::Vector{T}, x::T) where {T} = [v..., x]
myappend (generic function with 1 method)

julia> myappend([1,2,3],4)
4-element Array{Int64,1}:
 1
 2
 3
 4

julia> myappend([1,2,3],2.5)
ERROR: MethodError: no method matching myappend(::Array{Int64,1}, ::Float64)
Closest candidates are:
  myappend(::Array{T,1}, !Matched::T) where T at none:1

julia> myappend([1.0,2.0,3.0],4.0)
4-element Array{Float64,1}:
 1.0
 2.0
 3.0
 4.0

julia> myappend([1.0,2.0,3.0],4)
ERROR: MethodError: no method matching myappend(::Array{Float64,1}, ::Int64)
Closest candidates are:
  myappend(::Array{T,1}, !Matched::T) where T at none:1
\end{minted}



如你所看到的,追加的元素的类型必须匹配它追加到的向量的元素类型,否则会引起\hyperlink{68769522931907606}{\texttt{MethodError}}。在下面的例子中,方法类型参量\texttt{T}用作返回值:




\begin{minted}{jlcon}
julia> mytypeof(x::T) where {T} = T
mytypeof (generic function with 1 method)

julia> mytypeof(1)
Int64

julia> mytypeof(1.0)
Float64
\end{minted}



就像你能在类型声明时通过类型参数对子类型进行约束一样(参见\hyperlink{5603543911318150609}{参数类型}),你也可以约束方法的类型参数:




\begin{minted}{jlcon}
julia> same_type_numeric(x::T, y::T) where {T<:Number} = true
same_type_numeric (generic function with 1 method)

julia> same_type_numeric(x::Number, y::Number) = false
same_type_numeric (generic function with 2 methods)

julia> same_type_numeric(1, 2)
true

julia> same_type_numeric(1, 2.0)
false

julia> same_type_numeric(1.0, 2.0)
true

julia> same_type_numeric("foo", 2.0)
ERROR: MethodError: no method matching same_type_numeric(::String, ::Float64)
Closest candidates are:
  same_type_numeric(!Matched::T, ::T) where T<:Number at none:1
  same_type_numeric(!Matched::Number, ::Number) at none:1

julia> same_type_numeric("foo", "bar")
ERROR: MethodError: no method matching same_type_numeric(::String, ::String)

julia> same_type_numeric(Int32(1), Int64(2))
false
\end{minted}



\texttt{same\_type\_numeric}函数的行为与上面定义的\texttt{same\_type}函数基本相似,但是它只对一对数定义。



参数方法允许与 \texttt{where} 表达式同样的语法用来写类型(参见 \hyperlink{11072845175692859046}{UnionAll 类型})。如果只有一个参数,封闭的大括号(在 \texttt{where \{T\}} 中)可以省略,但是为了清楚起见推荐写上。多个参数可以使用逗号隔开,例如 \texttt{where \{T, S <: Real\}},或者使用嵌套的 \texttt{where} 来写,例如 \texttt{where S<:Real where T}。



\hypertarget{18141577506112006209}{}


\section{重定义方法}



当重定义一个方法或者增加一个方法时,知道这个变化不会立即生效很重要。这是Julia能够静态推断和编译代码使其运行很快而没有惯常的JIT技巧和额外开销的关键。实际上,任意新的方法定义不会对当前运行环境可见,包括Tasks和线程(和所有的之前定义的\texttt{@generated}函数)。让我们通过一个例子说明这意味着什么:




\begin{minted}{jlcon}
julia> function tryeval()
           @eval newfun() = 1
           newfun()
       end
tryeval (generic function with 1 method)

julia> tryeval()
ERROR: MethodError: no method matching newfun()
The applicable method may be too new: running in world age xxxx1, while current world is xxxx2.
Closest candidates are:
  newfun() at none:1 (method too new to be called from this world context.)
 in tryeval() at none:1
 ...

julia> newfun()
1
\end{minted}



在这个例子中看到\texttt{newfun}的新定义已经被创建,但是并不能立即调用。新的全局变量立即对\texttt{tryeval}函数可见,所以你可以写\texttt{return newfun}(没有小括号)。但是你,你的调用器,和他们调用的函数等等都不能调用这个新的方法定义!



但是这里有个例外:之后的\emph{在 REPL 中}的 \texttt{newfun} 的调用会按照预期工作,能够见到并调用\texttt{newfun} 的新定义。



但是,之后的 \texttt{tryeval} 的调用将会继续看到 \texttt{newfun} 的定义,因为该定义\emph{位于 REPL 的前一个语句中}并因此在之后的 \texttt{tryeval} 的调用之前。



你可以试试这个来让自己了解这是如何工作的。



这个行为的实现通过一个「world age 计数器」。这个单调递增的值会跟踪每个方法定义操作。此计数器允许用单个数字描述「对于给定运行时环境可见的方法定义集」,或者说「world age」。它还允许仅仅通过其序数值来比较在两个 world 中可用的方法。在上例中,我们看到(方法 \texttt{newfun} 所存在的)「current world」比局部于任务的「runtime world」大一,后者在 \texttt{tryeval} 开始执行时是固定的。



有时规避这个是必要的(例如,如果你在实现上面的REPL)。幸运的是这里有个简单地解决方法:使用\hyperlink{15240876280767285272}{\texttt{Base.invokelatest}}调用函数:




\begin{minted}{jlcon}
julia> function tryeval2()
           @eval newfun2() = 2
           Base.invokelatest(newfun2)
       end
tryeval2 (generic function with 1 method)

julia> tryeval2()
2
\end{minted}



最后,让我们看一些这个规则生效的更复杂的例子。 定义一个函数\texttt{f(x)},最开始有一个方法:




\begin{minted}{jlcon}
julia> f(x) = "original definition"
f (generic function with 1 method)
\end{minted}



开始一些使用\texttt{f(x)}的运算:




\begin{minted}{jlcon}
julia> g(x) = f(x)
g (generic function with 1 method)

julia> t = @async f(wait()); yield();
\end{minted}



现在我们给\texttt{f(x)}加上一些新的方法:




\begin{minted}{jlcon}
julia> f(x::Int) = "definition for Int"
f (generic function with 2 methods)

julia> f(x::Type{Int}) = "definition for Type{Int}"
f (generic function with 3 methods)
\end{minted}



比较一下这些结果如何不同:




\begin{minted}{jlcon}
julia> f(1)
"definition for Int"

julia> g(1)
"definition for Int"

julia> fetch(schedule(t, 1))
"original definition"

julia> t = @async f(wait()); yield();

julia> fetch(schedule(t, 1))
"definition for Int"
\end{minted}



\hypertarget{1696520674414442593}{}


\section{使用参数方法设计样式}



虽然复杂的分派逻辑对于性能或者可用性并不是必须的,但是有时这是表达某些算法的最好的方法。 这里有一些常见的设计样式,在以这个方法使用分派时有时会出现。



\hypertarget{13813570673284607064}{}


\subsection{从超类型中提取出类型参数}



这里是一个正确地代码模板,它返回\texttt{AbstractArray}的任意子类型的元素类型\texttt{T}:




\begin{minted}{julia}
abstract type AbstractArray{T, N} end
eltype(::Type{<:AbstractArray{T}}) where {T} = T
\end{minted}



using so-called triangular dispatch.  Note that if \texttt{T} is a \texttt{UnionAll} type, as e.g. \texttt{eltype(Array\{T\} where T <: Integer)}, then \texttt{Any} is returned (as does the version of \texttt{eltype} in \texttt{Base}).



另外一个方法,这是在Julia v0.6中的三角分派到来之前的唯一正确方法,是:




\begin{minted}{julia}
abstract type AbstractArray{T, N} end
eltype(::Type{AbstractArray}) = Any
eltype(::Type{AbstractArray{T}}) where {T} = T
eltype(::Type{AbstractArray{T, N}}) where {T, N} = T
eltype(::Type{A}) where {A<:AbstractArray} = eltype(supertype(A))
\end{minted}



另外一个可能性如下例,这可以对适配那些参数\texttt{T}需要更严格匹配的情况有用:




\begin{minted}{julia}
eltype(::Type{AbstractArray{T, N} where {T<:S, N<:M}}) where {M, S} = Any
eltype(::Type{AbstractArray{T, N} where {T<:S}}) where {N, S} = Any
eltype(::Type{AbstractArray{T, N} where {N<:M}}) where {M, T} = T
eltype(::Type{AbstractArray{T, N}}) where {T, N} = T
eltype(::Type{A}) where {A <: AbstractArray} = eltype(supertype(A))
\end{minted}



一个常见的错误是试着使用内省来得到元素类型:




\begin{minted}{julia}
eltype_wrong(::Type{A}) where {A<:AbstractArray} = A.parameters[1]
\end{minted}



但是创建一个这个方法会失败的情况不难:




\begin{minted}{julia}
struct BitVector <: AbstractArray{Bool, 1}; end
\end{minted}



这里我们已经创建了一个没有参数的类型\texttt{BitVector},但是元素类型已经完全指定了,\texttt{T}等于\texttt{Bool}!



\hypertarget{7211262558234007578}{}


\subsection{用不同的类型参数构建相似的类型}



当构建通用代码时,通常需要创建一些类似对象,在类型的布局上有一些变化,这就也让类型参数的变化变得必要。 例如,你会有一些任意元素类型的抽象数组,想使用特定的元素类型来编写你基于它的计算。你必须实现为每个\texttt{AbstractArray\{T\}}的子类型实现方法,这些方法描述了如何计算类型转换。从一个子类型转化成拥有一个不同参数的另一个子类型的通用方法在这里不存在。(快速复习:你明白为什么吗?)



\texttt{AbstractArray}的子类型典型情况下会实现两个方法来完成这个: 一个方法把输入输入转换成特定的\texttt{AbstractArray\{T,N\}}抽象类型的子类型;一个方法用特定的元素类型构建一个新的未初始化的数组。这些的样例实现可以在Julia Base里面找到。这里是一个基础的样例使用,保证\texttt{输入}与\texttt{输出}是同一种类型:




\begin{minted}{julia}
input = convert(AbstractArray{Eltype}, input)
output = similar(input, Eltype)
\end{minted}



作为这个的扩展,在算法需要输入数组的拷贝的情况下,\hyperlink{1846942650946171605}{\texttt{convert}}使无法胜任的,因为返回值可能只是原始输入的别名。把\hyperlink{15525808546723795098}{\texttt{similar}}(构建输出数组)和\hyperlink{12476124489163612623}{\texttt{copyto!}}(用输入数据填满)结合起来是需要给出输入参数的可变拷贝的一个范用方法:




\begin{minted}{julia}
copy_with_eltype(input, Eltype) = copyto!(similar(input, Eltype), input)
\end{minted}



\hypertarget{6940352566710127049}{}


\subsection{迭代分派}



为了分派一个多层的参数参量列表,将每一层分派分开到不同的函数中常常是最好的。这可能听起来跟单分派的方法相似,但是你会在下面见到,这个更加灵活。



例如,尝试按照数组的元素类型进行分派常常会引起歧义。相反地,常见的代码会首先按照容易类型分派,然后基于eltype递归到更加更加专用的方法。在大部分情况下,算法会很方便地就屈从与这个分层方法,在其他情况下,这种严苛的工作必须手动解决。这个分派分支能被观察到,例如在两个矩阵的加法的逻辑中:




\begin{minted}{julia}
# 首先分派选择了逐元素相加的map算法。
+(a::Matrix, b::Matrix) = map(+, a, b)
# 然后分派处理了每个元素然后选择了计算的
# 恰当的常见元素类型。
+(a, b) = +(promote(a, b)...)
# 一旦元素有了相同类型,它们就可以相加。
# 例如,通过处理器暴露出的原始运算。
+(a::Float64, b::Float64) = Core.add(a, b)
\end{minted}



\hypertarget{5085783943422577248}{}


\subsection{基于 Trait 的分派}



对于上面的可迭代分派的一个自然扩展是给方法选择加一个内涵层,这个层允许按照那些与类型层级定义的集合相独立的类型的集合来分派。我们可以通过写出问题中的类型的一个\texttt{Union}来创建这个一个集合,但是这不能够扩展,因为\texttt{Union}类型在创建之后无法改变。但是这么一个可扩展的集合可以通过一个叫做\href{https://github.com/JuliaLang/julia/issues/2345\#issuecomment-54537633}{{\textquotedbl}Holy-trait{\textquotedbl}}的一个设计样式来实现。



这个样式是通过定义一个范用函数来实现,这个函数为函数参数可能属于的每个trait集合都计算出不同的单例值(或者类型)。如果这个函数是单纯的,这与通常的分派对于性能没有任何影响。



上一节的例子掩盖了\hyperlink{11483231213869150535}{\texttt{map}}和\hyperlink{1760874576431605095}{\texttt{promote}}的实现细节,这两个都是依据trait来进行运算的。当对一个矩阵进行迭代,比如\texttt{map}的实现中,一个重要的问题是按照什么顺序去遍历数据。当\texttt{AbstractArray}的子类型实现了\hyperlink{7782790551324367092}{\texttt{Base.IndexStyle}}trait,其他函数,比如\texttt{map}就可以根据这个信息进行分派,以选择最好的算法(参见\hyperlink{9718377734213742156}{抽象数组接口})。这意味着每个子类型就没有必要去实现对应的\texttt{map}版本,因为通用的定义加trait类就能让系统选择最快的版本。这里一个玩具似的\texttt{map}实现说明了基于trait的分派:




\begin{minted}{julia}
map(f, a::AbstractArray, b::AbstractArray) = map(Base.IndexStyle(a, b), f, a, b)
# generic implementation:
map(::Base.IndexCartesian, f, a::AbstractArray, b::AbstractArray) = ...
# linear-indexing implementation (faster)
map(::Base.IndexLinear, f, a::AbstractArray, b::AbstractArray) = ...
\end{minted}



这个基于trait的方法也出现在\hyperlink{1760874576431605095}{\texttt{promote}}机制中,被标量\texttt{+}使用。 它使用了\hyperlink{15048881762587391286}{\texttt{promote\_type}},这在知道两个计算对象的类型的情况下返回计算这个运算的最佳的常用类型。这就使得我们不用为每一对可能的类型参数实现每一个函数,而把问题简化为对于每个类型实现一个类型转换运算这样一个小很多的问题,还有一个优选的逐对的类型提升规则的表格。



\hypertarget{8273619992944815153}{}


\subsection{输出类型计算}



基于trait的类型提升的讨论可以过渡到我们的下一个设计样式:为矩阵运算计算输出元素类型。



为了实现像加法这样的原始运算,我们使用\hyperlink{15048881762587391286}{\texttt{promote\_type}}函数来计算想要的输出类型。(像之前一样,我们在\texttt{+}调用中的\texttt{promote}调用中见到了这个工作)。



对于矩阵的更加复杂的函数,对于更加复杂的运算符序列来计算预期的返回类型是必要的。这经常按下列步骤进行:



\begin{itemize}
\item[1. ] 编写一个小函数\texttt{op}来表示算法核心中使用的运算的集合。


\item[2. ] 使用\texttt{promote\_op(op, argument\_types...)}计算结果矩阵的元素类型\texttt{R}, 这里\texttt{argument\_types}是通过应用到每个输入数组的\texttt{eltype}计算的。


\item[3. ] 创建类似于\texttt{similar(R, dims)}的输出矩阵,这里\texttt{dims}是输出矩阵的预期维度数。

\end{itemize}


作为一个更加具体的例子,一个范用的方阵乘法的伪代码是:




\begin{minted}{julia}
function matmul(a::AbstractMatrix, b::AbstractMatrix)
    op = (ai, bi) -> ai * bi + ai * bi

    ## this is insufficient because it assumes `one(eltype(a))` is constructable:
    # R = typeof(op(one(eltype(a)), one(eltype(b))))

    ## this fails because it assumes `a[1]` exists and is representative of all elements of the array
    # R = typeof(op(a[1], b[1]))

    ## this is incorrect because it assumes that `+` calls `promote_type`
    ## but this is not true for some types, such as Bool:
    # R = promote_type(ai, bi)

    # this is wrong, since depending on the return value
    # of type-inference is very brittle (as well as not being optimizable):
    # R = Base.return_types(op, (eltype(a), eltype(b)))

    ## but, finally, this works:
    R = promote_op(op, eltype(a), eltype(b))
    ## although sometimes it may give a larger type than desired
    ## it will always give a correct type

    output = similar(b, R, (size(a, 1), size(b, 2)))
    if size(a, 2) > 0
        for j in 1:size(b, 2)
            for i in 1:size(a, 1)
                ## here we don't use `ab = zero(R)`,
                ## since `R` might be `Any` and `zero(Any)` is not defined
                ## we also must declare `ab::R` to make the type of `ab` constant in the loop,
                ## since it is possible that typeof(a * b) != typeof(a * b + a * b) == R
                ab::R = a[i, 1] * b[1, j]
                for k in 2:size(a, 2)
                    ab += a[i, k] * b[k, j]
                end
                output[i, j] = ab
            end
        end
    end
    return output
end
\end{minted}



\hypertarget{1067448927897925814}{}


\subsection{分离转换和内核逻辑}



能有效减少编译时间和测试复杂度的一个方法是将预期的类型和计算转换的逻辑隔离。这会让编译器将与大型内核的其他部分相独立的类型转换逻辑特别化并内联。



将更大的类型类转换成被算法实际支持的特定参数类是一个常见的设计样式:




\begin{minted}{julia}
complexfunction(arg::Int) = ...
complexfunction(arg::Any) = complexfunction(convert(Int, arg))

matmul(a::T, b::T) = ...
matmul(a, b) = matmul(promote(a, b)...)
\end{minted}



\hypertarget{3661198273813942193}{}


\section{参数化约束的可变参数方法}



函数参数也可以用于约束应用于{\textquotedbl}可变参数{\textquotedbl}函数(\hyperlink{9965084594348935329}{变参函数})的参数的数量。\texttt{Vararg\{T,N\}} 可用于表明这么一个约束。举个例子:




\begin{minted}{jlcon}
julia> bar(a,b,x::Vararg{Any,2}) = (a,b,x)
bar (generic function with 1 method)

julia> bar(1,2,3)
ERROR: MethodError: no method matching bar(::Int64, ::Int64, ::Int64)
Closest candidates are:
  bar(::Any, ::Any, ::Any, !Matched::Any) at none:1

julia> bar(1,2,3,4)
(1, 2, (3, 4))

julia> bar(1,2,3,4,5)
ERROR: MethodError: no method matching bar(::Int64, ::Int64, ::Int64, ::Int64, ::Int64)
Closest candidates are:
  bar(::Any, ::Any, ::Any, ::Any) at none:1
\end{minted}



更加有用的是,用一个参数就约束可变参数的方法是可能的。例如:




\begin{minted}{julia}
function getindex(A::AbstractArray{T,N}, indices::Vararg{Number,N}) where {T,N}
\end{minted}



只会在\texttt{indices}的个数与数组的维数相同时才会调用。



当只有提供的参数的类型需要被约束时,\texttt{Vararg\{T\}}可以写成\texttt{T...}。例如\texttt{f(x::Int...) = x}是\texttt{f(x::Vararg\{Int\}) = x}的简便写法。



\hypertarget{6114459498123607412}{}


\section{可选参数和关键字的参数的注意事项}



与在\hyperlink{645008301484218813}{函数}中简要提到的一样,可选参数是使用多方法定义语法来实现的。例如,这个定义:




\begin{minted}{julia}
f(a=1,b=2) = a+2b
\end{minted}



翻译成下列三个方法:




\begin{minted}{julia}
f(a,b) = a+2b
f(a) = f(a,2)
f() = f(1,2)
\end{minted}



这就意味着调用\texttt{f()}等于调用\texttt{f(1,2)}。在这个情况下结果是\texttt{5},因为\texttt{f(1,2)}使用的是上面\texttt{f}的第一个方法。但是,不总是需要是这种情况。如果你定义了第四个对于整数更加专用的方法:




\begin{minted}{julia}
f(a::Int,b::Int) = a-2b
\end{minted}



此时\texttt{f()}和\texttt{f(1,2)}的结果都是\texttt{-3}。换句话说,可选参数只与函数捆绑,而不是函数的任意一个特定的方法。这个决定于使用的方法的可选参数的类型。当可选参数是用全局变量的形式定义时,可选参数的类型甚至会在运行时改变。



关键字参数与普通的位置参数的行为很不一样。特别地,他们不参与到方法分派中。方法只基于位置参数分派,在匹配得方法确定之后关键字参数才会被处理。



\hypertarget{12721694880938168924}{}


\section{类函数对象}



方法与类型相关,所以可以通过给类型加方法使得任意一个Julia类型变得{\textquotedbl}可被调用{\textquotedbl}。(这个{\textquotedbl}可调用{\textquotedbl}的对象有时称为{\textquotedbl}函子{\textquotedbl}。)



例如,你可以定义一个类型,存储着多项式的系数,但是行为像是一个函数,可以为多项式求值:




\begin{minted}{jlcon}
julia> struct Polynomial{R}
           coeffs::Vector{R}
       end

julia> function (p::Polynomial)(x)
           v = p.coeffs[end]
           for i = (length(p.coeffs)-1):-1:1
               v = v*x + p.coeffs[i]
           end
           return v
       end

julia> (p::Polynomial)() = p(5)
\end{minted}



注意函数是通过类型而非名字来指定的。如同普通函数一样这里有一个简洁的语法形式。在函数体内,\texttt{p}会指向被调用的对象。\texttt{Polynomial}会按如下方式使用:




\begin{minted}{jlcon}
julia> p = Polynomial([1,10,100])
Polynomial{Int64}([1, 10, 100])

julia> p(3)
931

julia> p()
2551
\end{minted}



这个机制也是Julia中类型构造函数和闭包(指向其环境的内部函数)的工作原理。



\hypertarget{117540638029415517}{}


\section{空泛型函数}



有时引入一个没有添加方法的范用函数是有用的。这会用于分离实现与接口定义。这也可为了文档或者代码可读性。为了这个的语法是没有参数组的一个空\texttt{函数}块:




\begin{minted}{julia}
function emptyfunc
end
\end{minted}



\hypertarget{11088607530909626670}{}


\section{方法设计与避免歧义}



Julia的方法多态性是其最有力的特性之一,利用这个功能会带来设计上的挑战。特别地,在更加复杂的方法层级中出现\hyperlink{1524461975045594238}{歧义}不能说不常见。



在上面我们曾经指出我们可以像这样解决歧义




\begin{minted}{julia}
f(x, y::Int) = 1
f(x::Int, y) = 2
\end{minted}



靠定义一个方法




\begin{minted}{julia}
f(x::Int, y::Int) = 3
\end{minted}



这是经常使用的对的方案;但是有些环境下盲目地遵从这个建议会适得其反。特别地,范用函数有的方法越多,出现歧义的可能性越高。当你的方法层级比这些简单的例子更加复杂时,就值得你花时间去仔细想想其他的方案。



下面我们会讨论特别的一些挑战和解决这些挑战的一些可选方法。



\hypertarget{17193104641044635373}{}


\subsection{元组和N元组参数}



\texttt{Tuple}(和\texttt{NTuple})参数会带来特别的挑战。例如,




\begin{minted}{julia}
f(x::NTuple{N,Int}) where {N} = 1
f(x::NTuple{N,Float64}) where {N} = 2
\end{minted}



是有歧义的,因为存在\texttt{N == 0}的可能性:没有元素去确定\texttt{Int}还是\texttt{Float64}变体应该被调用。为了解决歧义,一个方法是为空元组定义方法:




\begin{minted}{julia}
f(x::Tuple{}) = 3
\end{minted}



作为一种选择,对于其中一个方法之外的所有的方法可以坚持元组中至少有一个元素:




\begin{minted}{julia}
f(x::NTuple{N,Int}) where {N} = 1           # this is the fallback
f(x::Tuple{Float64, Vararg{Float64}}) = 2   # this requires at least one Float64
\end{minted}



\hypertarget{2934107525015609338}{}


\subsection{正交化你的设计}



当你打算根据两个或更多的参数进行分派时,考虑一下,一个「包裹」函数是否会让设计简单一些。举个例子,与其编写多变量:




\begin{minted}{julia}
f(x::A, y::A) = ...
f(x::A, y::B) = ...
f(x::B, y::A) = ...
f(x::B, y::B) = ...
\end{minted}



不如考虑定义




\begin{minted}{julia}
f(x::A, y::A) = ...
f(x, y) = f(g(x), g(y))
\end{minted}



这里\texttt{g}把参数转变为类型\texttt{A}。这是更加普遍的\href{https://en.wikipedia.org/wiki/Orthogonality\_(programming)}{正交设计}原理的一个特别特殊的例子,在正交设计中不同的概念被分配到不同的方法中去。这里\texttt{g}最可能需要一个fallback定义




\begin{minted}{julia}
g(x::A) = x
\end{minted}



一个相关的方案使用\texttt{promote}来把\texttt{x}和\texttt{y}变成常见的类型:




\begin{minted}{julia}
f(x::T, y::T) where {T} = ...
f(x, y) = f(promote(x, y)...)
\end{minted}



One risk with this design is the possibility that if there is no suitable promotion method converting \texttt{x} and \texttt{y} to the same type, the second method will recurse on itself infinitely and trigger a stack overflow.



\hypertarget{2211158616759162982}{}


\subsection{一次只根据一个参数分派}



如果你你需要根据多个参数进行分派,并且有太多的为了能定义所有可能的变量而存在的组合,而存在很多回退函数,你可以考虑引入{\textquotedbl}名字级联{\textquotedbl},这里(例如)你根据第一个参数分配然后调用一个内部的方法:




\begin{minted}{julia}
f(x::A, y) = _fA(x, y)
f(x::B, y) = _fB(x, y)
\end{minted}



接着内部方法\texttt{\_fA}和\texttt{\_fB}可以根据\texttt{y}进行分派,而不考虑有关\texttt{x}的歧义存在。



需要意识到这个方案至少有一个主要的缺点:在很多情况下,用户没有办法通过进一步定义你的输出函数\texttt{f}的具体行为来进一步定制\texttt{f}的行为。相反,他们需要去定义你的内部方法\texttt{\_fA}和\texttt{\_fB}的具体行为,这会模糊输出方法和内部方法之间的界线。



\hypertarget{7624993517486343641}{}


\subsection{抽象容器与元素类型}



在可能的情况下要试图避免定义根据抽象容器的具体元素类型来分派的方法。举个例子,




\begin{minted}{julia}
-(A::AbstractArray{T}, b::Date) where {T<:Date}
\end{minted}



会引起歧义,当定义了这个方法:




\begin{minted}{julia}
-(A::MyArrayType{T}, b::T) where {T}
\end{minted}



最好的方法是不要定义这些方法中的\emph{任何一个}。相反,使用范用方法\texttt{-(A::AbstractArray, b)}并确认这个方法是使用\emph{分别}对于每个容器类型和元素类型都是适用的通用调用(像\texttt{similar}和\texttt{-})实现的。这只是建议\hyperlink{1356336112225694303}{正交化}你的方法的一个更加复杂的变种而已。



当这个方法不可行时,这就值得与其他开发者开始讨论如果解决歧义;只是因为一个函数先定义并不总是意味着他不能改变或者被移除。作为最后一个手段,开发者可以定义{\textquotedbl}创可贴{\textquotedbl}方法




\begin{minted}{julia}
-(A::MyArrayType{T}, b::Date) where {T<:Date} = ...
\end{minted}



可以暴力解决歧义。



\hypertarget{5733743229509139145}{}


\subsection{与默认参数的复杂方法{\textquotedbl}级联{\textquotedbl}}



如果你定义了提供默认的方法{\textquotedbl}级联{\textquotedbl},要小心去掉对应着潜在默认的任何参数。例如,假设你在写一个数字过滤算法,你有一个通过应用padding来出来信号的边的方法:




\begin{minted}{julia}
function myfilter(A, kernel, ::Replicate)
    Apadded = replicate_edges(A, size(kernel))
    myfilter(Apadded, kernel)  # now perform the "real" computation
end
\end{minted}



这会与提供默认padding的方法产生冲突:




\begin{minted}{julia}
myfilter(A, kernel) = myfilter(A, kernel, Replicate()) # replicate the edge by default
\end{minted}



这两个方法一起会生成无限的递归,\texttt{A}会不断变大。



更好的设计是像这样定义你的调用层级:




\begin{minted}{julia}
struct NoPad end # indicate that no padding is desired, or that it's already applied

myfilter(A, kernel) = myfilter(A, kernel, Replicate()) # default boundary conditions

function myfilter(A, kernel, ::Replicate)
 Apadded = replicate_edges(A, size(kernel))
 myfilter(Apadded, kernel, NoPad()) # indicate the new boundary conditions
end

# other padding methods go here

function myfilter(A, kernel, ::NoPad)
 # Here's the "real" implementation of the core computation
end
\end{minted}



\texttt{NoPad} 被置于与其他 padding 类型一致的参数位置上,这保持了分派层级的良好组织,同时降低了歧义的可能性。而且,它扩展了「公开」的 \texttt{myfilter} 接口:想要显式控制 padding 的用户可以直接调用 \texttt{NoPad} 变量。



\footnotetext[2]{Arthur C. Clarke, \emph{Profiles of the Future} (1961): Clarke{\textquotesingle}s Third Law.

}


\hypertarget{17317810227993044854}{}


\chapter{构造函数}



构造函数 \footnotemark[1] 是用来创建新对象的函数 – 确切地说,它创建的是\hyperlink{4168730090950432836}{复合类型}的实例。在 Julia 中,类型对象也同时充当构造函数的角色:可以用类名加参数元组的方式像函数调用一样来创建新实例。这一点在介绍复合类型(Composite Types)时已经大致谈过了。例如:




\begin{minted}{jlcon}
julia> struct Foo
           bar
           baz
       end

julia> foo = Foo(1, 2)
Foo(1, 2)

julia> foo.bar
1

julia> foo.baz
2
\end{minted}



For many types, forming new objects by binding their field values together is all that is ever needed to create instances. However, in some cases more functionality is required when creating composite objects. Sometimes invariants must be enforced, either by checking arguments or by transforming them. \href{https://en.wikipedia.org/wiki/Recursion\_\%28computer\_science\%29\#Recursive\_data\_structures\_.28structural\_recursion.29}{Recursive data structures}, especially those that may be self-referential, often cannot be constructed cleanly without first being created in an incomplete state and then altered programmatically to be made whole, as a separate step from object creation. Sometimes, it{\textquotesingle}s just convenient to be able to construct objects with fewer or different types of parameters than they have fields. Julia{\textquotesingle}s system for object construction addresses all of these cases and more.



\footnotetext[1]{Nomenclature: while the term {\textquotedbl}constructor{\textquotedbl} generally refers to the entire function which constructs objects of a type, it is common to abuse terminology slightly and refer to specific constructor methods as {\textquotedbl}constructors{\textquotedbl}. In such situations, it is generally clear from the context that the term is used to mean {\textquotedbl}constructor method{\textquotedbl} rather than {\textquotedbl}constructor function{\textquotedbl}, especially as it is often used in the sense of singling out a particular method of the constructor from all of the others.

}


\hypertarget{8095711241800911617}{}


\section{Outer Constructor Methods}



A constructor is just like any other function in Julia in that its overall behavior is defined by the combined behavior of its methods. Accordingly, you can add functionality to a constructor by simply defining new methods. For example, let{\textquotesingle}s say you want to add a constructor method for \texttt{Foo} objects that takes only one argument and uses the given value for both the \texttt{bar} and \texttt{baz} fields. This is simple:




\begin{minted}{jlcon}
julia> Foo(x) = Foo(x,x)
Foo

julia> Foo(1)
Foo(1, 1)
\end{minted}



你也可以为 \texttt{Foo} 添加新的零参数构造方法,它为 \texttt{bar} 和 \texttt{baz} 提供默认值:




\begin{minted}{jlcon}
julia> Foo() = Foo(0)
Foo

julia> Foo()
Foo(0, 0)
\end{minted}



Here the zero-argument constructor method calls the single-argument constructor method, which in turn calls the automatically provided two-argument constructor method. For reasons that will become clear very shortly, additional constructor methods declared as normal methods like this are called \emph{outer} constructor methods. Outer constructor methods can only ever create a new instance by calling another constructor method, such as the automatically provided default ones.



\hypertarget{3020780065533340945}{}


\section{Inner Constructor Methods}



While outer constructor methods succeed in addressing the problem of providing additional convenience methods for constructing objects, they fail to address the other two use cases mentioned in the introduction of this chapter: enforcing invariants, and allowing construction of self-referential objects. For these problems, one needs \emph{inner} constructor methods. An inner constructor method is like an outer constructor method, except for two differences:



\begin{itemize}
\item[1. ] It is declared inside the block of a type declaration, rather than outside of it like normal methods.


\item[2. ] It has access to a special locally existent function called \hyperlink{13888762393600028594}{\texttt{new}} that creates objects of the block{\textquotesingle}s type.

\end{itemize}


For example, suppose one wants to declare a type that holds a pair of real numbers, subject to the constraint that the first number is not greater than the second one. One could declare it like this:




\begin{minted}{jlcon}
julia> struct OrderedPair
           x::Real
           y::Real
           OrderedPair(x,y) = x > y ? error("out of order") : new(x,y)
       end
\end{minted}



现在 \texttt{OrderedPair} 对象只能在 \texttt{x <= y} 时被成功构造:




\begin{minted}{jlcon}
julia> OrderedPair(1, 2)
OrderedPair(1, 2)

julia> OrderedPair(2,1)
ERROR: out of order
Stacktrace:
 [1] error at ./error.jl:33 [inlined]
 [2] OrderedPair(::Int64, ::Int64) at ./none:4
 [3] top-level scope
\end{minted}



如果类型被声明为 \texttt{mutable},你可以直接更改字段值来打破这个固有属性,然而,在未经允许的情况下,随意摆弄对象的内核一般都是不好的行为。你(或者其他人)可以在以后任何时候提供额外的外部构造方法,但一旦类型被声明了,就没有办法来添加更多的内部构造方法了。由于外部构造方法只能通过调用其它的构造方法来创建对象,所以最终构造对象的一定是某个内部构造函数。这保证了已声明类型的对象必须通过调用该类型的内部构造方法才得已存在,从而在某种程度上保证了类型的固有属性。



只要定义了任何一个内部构造方法,Julia 就不会再提供默认的构造方法:它会假定你已经为自己提供了所需的所有内部构造方法。默认构造方法等效于一个你自己编写的内部构造函数,该函数将所有成员作为参数(如果相应的字段具有类型,则约束为正确的类型),并将它们传递给 \texttt{new},最后返回结果对象:




\begin{minted}{jlcon}
julia> struct Foo
           bar
           baz
           Foo(bar,baz) = new(bar,baz)
       end

\end{minted}



这个声明与前面没有显式内部构造方法的 \texttt{Foo} 类型的定义效果相同。 以下两个类型是等价的 – 一个具有默认构造方法,另一个具有显式构造方法:




\begin{minted}{jlcon}
julia> struct T1
           x::Int64
       end

julia> struct T2
           x::Int64
           T2(x) = new(x)
       end

julia> T1(1)
T1(1)

julia> T2(1)
T2(1)

julia> T1(1.0)
T1(1)

julia> T2(1.0)
T2(1)
\end{minted}



提供尽可能少的内部构造方法是一种良好的形式:仅在需要显式地处理所有参数,以及强制执行必要的错误检查和转换时候才使用内部构造。其它用于提供便利的构造方法,比如提供默认值或辅助转换,应该定义为外部构造函数,然后再通过调用内部构造函数来执行繁重的工作。这种解耦是很自然的。



\hypertarget{7871769496419060352}{}


\section{不完整初始化}



最后一个还没提到的问题是,如何构造具有自引用的对象,更广义地来说是构造递归数据结构。由于这其中的困难并不是那么显而易见,这里我们来简单解释一下,考虑如下的递归类型声明:




\begin{minted}{jlcon}
julia> mutable struct SelfReferential
           obj::SelfReferential
       end

\end{minted}



这种类型可能看起来没什么大不了,直到我们考虑如何来构造它的实例。 如果 \texttt{a} 是 \texttt{SelfReferential} 的一个实例,则第二个实例可以用如下的调用来创建:




\begin{minted}{jlcon}
julia> b = SelfReferential(a)
\end{minted}



但是,当没有实例存在的情况下,即没有可以传递给 \texttt{obj} 成员变量的有效值时,如何构造第一个实例?唯一的解决方案是允许使用未初始化的 \texttt{obj} 成员来创建一个未完全初始化的 \texttt{SelfReferential} 实例,并使用该不完整的实例作为另一个实例的 \texttt{obj} 成员的有效值,例如,它本身。



为了允许创建未完全初始化的对象,Julia 允许使用少于该类型成员数的参数来调用 \href{@ ref}{\texttt{new}} 函数,并返回一个具有某个未初始化成员的对象。然后,内部构造函数可以使用不完整的对象,在返回之前完成初始化。例如,我们在定义 \texttt{SelfReferential} 类型时采用了另一个方法,使用零参数内部构造函数来返回一个实例,此实例的 \texttt{obj} 成员指向其自身:




\begin{minted}{jlcon}
julia> mutable struct SelfReferential
           obj::SelfReferential
           SelfReferential() = (x = new(); x.obj = x)
       end

\end{minted}



我们可以验证这一构造函数有效性,且由其构造的对象确实是自引用的:




\begin{minted}{jlcon}
julia> x = SelfReferential();

julia> x === x
true

julia> x === x.obj
true

julia> x === x.obj.obj
true
\end{minted}



虽然从一个内部构造函数中返回一个完全初始化的对象是很好的,但是也可以返回未完全初始化的对象:




\begin{minted}{jlcon}
julia> mutable struct Incomplete
           data
           Incomplete() = new()
       end

julia> z = Incomplete();
\end{minted}



尽管允许创建含有未初始化成员的对象,然而任何对未初始化引用的访问都会立即报错:




\begin{minted}{jlcon}
julia> z.data
ERROR: UndefRefError: access to undefined reference
\end{minted}



这避免了不断地检测 \texttt{null} 值的需要。然而,并不是所有的对象成员都是引用。Julia 会将一些类型当作纯数据({\textquotedbl}plain data{\textquotedbl}),这意味着它们的数据是自包含的,并且没有引用其它对象。这些纯数据包括原始类型(比如 \texttt{Int} )和由其它纯数据类型构成的不可变结构体。纯数据类型的初始值是未定义的:




\begin{minted}{jlcon}
julia> struct HasPlain
           n::Int
           HasPlain() = new()
       end

julia> HasPlain()
HasPlain(438103441441)
\end{minted}



由纯数据组成的数组也具有一样的行为。



在内部构造函数中,你可以将不完整的对象传递给其它函数来委托其补全构造:




\begin{minted}{jlcon}
julia> mutable struct Lazy
           data
           Lazy(v) = complete_me(new(), v)
       end
\end{minted}



与构造函数返回的不完整对象一样,如果 \texttt{complete\_me} 或其任何被调用者尝试在初始化之前访问 \texttt{Lazy} 对象的 \texttt{data} 字段,就会立刻报错。



\hypertarget{12775137678629941390}{}


\section{参数类型的构造函数}



参数类型的存在为构造函数增加了更多的复杂性。首先,让我们回顾一下\hyperlink{5603543911318150609}{参数类型}。在默认情况下,我们可以用两种方法来实例化参数复合类型,一种是显式地提供类型参数,另一种是让 Julia 根据构造函数输入参数的类型来隐式地推导类型参数。这里有一些例子:




\begin{minted}{jlcon}
julia> struct Point{T<:Real}
           x::T
           y::T
       end

julia> Point(1,2) ## 隐式的 T ##
Point{Int64}(1, 2)

julia> Point(1.0,2.5) ## 隐式的 T ##
Point{Float64}(1.0, 2.5)

julia> Point(1,2.5) ## implicit T ##
ERROR: MethodError: no method matching Point(::Int64, ::Float64)
Closest candidates are:
  Point(::T, ::T) where T<:Real at none:2

julia> Point{Int64}(1, 2) ## 显式的 T ##
Point{Int64}(1, 2)

julia> Point{Int64}(1.0,2.5) ## 显式的 T ##
ERROR: InexactError: Int64(2.5)
Stacktrace:
[...]

julia> Point{Float64}(1.0, 2.5) ## 显式的 T ##
Point{Float64}(1.0, 2.5)

julia> Point{Float64}(1,2) ## 显式的 T ##
Point{Float64}(1.0, 2.0)
\end{minted}



就像你看到的那样,用类型参数显式地调用构造函数,其参数会被转换为指定的类型:\texttt{Point\{Int64\}(1,2)} 可以正常工作,但是 \texttt{Point\{Int64\}(1.0,2.5)} 则会在将 \texttt{2.5} 转换为 \hyperlink{7720564657383125058}{\texttt{Int64}} 的时候报一个 \hyperlink{5399118524830636312}{\texttt{InexactError}}。当类型是从构造函数的参数隐式推导出来的时候,比如在例子 \texttt{Point(1,2)} 中,输入参数的类型必须一致,否则就无法确定 \texttt{T} 是什么,但 \texttt{Point} 的构造函数仍可以适配任意同类型的实数对。



实际上,这里的 \texttt{Point},\texttt{Point\{Float64\}} 以及 \texttt{Point\{Int64\}} 是不同的构造函数。\texttt{Point\{T\}} 表示对于每个类型 \texttt{T} 都存在一个不同的构造函数。如果不显式提供内部构造函数,在声明复合类型 \texttt{Point\{T<:Real\}} 的时候,Julia 会对每个满足 \texttt{T<:Real} 条件的类型都提供一个默认的内部构造函数 \texttt{Point\{T\}},它们的行为与非参数类型的默认内部构造函数一致。Julia 同时也会提供了一个通用的外部构造函数 \texttt{Point},用于适配任意同类型的实数对。Julia 默认提供的构造函数等价于下面这种显式的声明:




\begin{minted}{jlcon}
julia> struct Point{T<:Real}
           x::T
           y::T
           Point{T}(x,y) where {T<:Real} = new(x,y)
       end

julia> Point(x::T, y::T) where {T<:Real} = Point{T}(x,y);
\end{minted}



注意,每个构造函数定义的方式与调用它们的方式是一样的。调用 \texttt{Point\{Int64\}(1,2)} 会触发 \texttt{struct} 块内部的 \texttt{Point\{T\}(x,y)}。另一方面,外部构造函数声明的 \texttt{Point} 构造函数只会被同类型的实数对触发,它使得我们可以直接以 \texttt{Point(1,2)} 和 \texttt{Point(1.0,2.5)} 这种方式来创建实例,而不需要显示地使用类型参数。由于此方法的声明方式已经对输入参数的类型施加了约束,像 \texttt{Point(1,2.5)} 这种调用自然会导致 {\textquotedbl}no method{\textquotedbl} 错误。



假如我们想让 \texttt{Point(1,2.5)} 这种调用方式正常工作,比如,通过将整数 \texttt{1} 自动「提升」为浮点数 \texttt{1.0},最简单的方法是像下面这样定义一个额外的外部构造函数:




\begin{minted}{jlcon}
julia> Point(x::Int64, y::Float64) = Point(convert(Float64,x),y);
\end{minted}



此方法采用了 \hyperlink{1846942650946171605}{\texttt{convert}} 函数,显式地将 \texttt{x} 转化成了 \hyperlink{5027751419500983000}{\texttt{Float64}} 类型,之后再委托前面讲到的那个通用的外部构造函数来进行具体的构造工作,经过转化,两个参数的类型都是 \hyperlink{5027751419500983000}{\texttt{Float64}},所以可以正确构造出一个 \texttt{Point\{Float64\}} 对象,而不会像之前那样触发 \hyperlink{68769522931907606}{\texttt{MethodError}}。




\begin{minted}{jlcon}
julia> Point(1,2.5)
Point{Float64}(1.0, 2.5)

julia> typeof(ans)
Point{Float64}
\end{minted}



然而,其它类似的调用依然有问题:




\begin{minted}{jlcon}
julia> Point(1.5,2)
ERROR: MethodError: no method matching Point(::Float64, ::Int64)
Closest candidates are:
  Point(::T, !Matched::T) where T<:Real at none:1
\end{minted}



如果你想要找到一种方法可以使类似的调用都可以正常工作,请参阅\hyperlink{10374023657104680331}{类型转换与类型提升}。这里稍稍“剧透”一下,我们可以利用下面的这个外部构造函数来满足需求,无论输入参数的类型如何,它都可以触发通用的 \texttt{Point} 构造函数:




\begin{minted}{jlcon}
julia> Point(x::Real, y::Real) = Point(promote(x,y)...);
\end{minted}



这里的 \texttt{promote} 函数会将它的输入转化为同一类型,在此例中是 \hyperlink{5027751419500983000}{\texttt{Float64}}。定义了这个方法,\texttt{Point} 构造函数会自动提升输入参数的类型,且提升机制与算术运算符相同,比如 \hyperlink{3677358729494553841}{\texttt{+}},因此对所有的实数输入参数都适用:




\begin{minted}{jlcon}
julia> Point(1.5,2)
Point{Float64}(1.5, 2.0)

julia> Point(1,1//2)
Point{Rational{Int64}}(1//1, 1//2)

julia> Point(1.0,1//2)
Point{Float64}(1.0, 0.5)
\end{minted}



所以,即使 Julia 提供的默认内部构造函数对于类型参数的要求非常严格,我们也有方法将其变得更加易用。正因为构造函数可以充分发挥类型系统、方法以及多重分派的作用,定义复杂的行为也会变得非常简单。



\hypertarget{4347124195691244322}{}


\section{案例分析:分数的实现}



上文主要讲了关于参数复合类型及其构造函数的一些零散内容,或许将这些内容结合起来的一个最佳方法是分析一个真实的案例。为此,我们来实现一个我们自己的分数类型 \texttt{OurRational},它与 Julia 内置的分数类型 \hyperlink{8304566144531167610}{\texttt{Rational}} 很相似,它的定义在 \href{https://github.com/JuliaLang/julia/blob/master/base/rational.jl}{\texttt{rational.jl}} 里:




\begin{minted}{jlcon}
julia> struct OurRational{T<:Integer} <: Real
           num::T
           den::T
           function OurRational{T}(num::T, den::T) where T<:Integer
               if num == 0 && den == 0
                    error("invalid rational: 0//0")
               end
               g = gcd(den, num)
               num = div(num, g)
               den = div(den, g)
               new(num, den)
           end
       end

julia> OurRational(n::T, d::T) where {T<:Integer} = OurRational{T}(n,d)
OurRational

julia> OurRational(n::Integer, d::Integer) = OurRational(promote(n,d)...)
OurRational

julia> OurRational(n::Integer) = OurRational(n,one(n))
OurRational

julia> ⊘(n::Integer, d::Integer) = OurRational(n,d)
⊘ (generic function with 1 method)

julia> ⊘(x::OurRational, y::Integer) = x.num ⊘ (x.den*y)
⊘ (generic function with 2 methods)

julia> ⊘(x::Integer, y::OurRational) = (x*y.den) ⊘ y.num
⊘ (generic function with 3 methods)

julia> ⊘(x::Complex, y::Real) = complex(real(x) ⊘ y, imag(x) ⊘ y)
⊘ (generic function with 4 methods)

julia> ⊘(x::Real, y::Complex) = (x*y') ⊘ real(y*y')
⊘ (generic function with 5 methods)

julia> function ⊘(x::Complex, y::Complex)
           xy = x*y'
           yy = real(y*y')
           complex(real(xy) ⊘ yy, imag(xy) ⊘ yy)
       end
⊘ (generic function with 6 methods)
\end{minted}



第一行 – \texttt{struct OurRational\{T<:Integer\} <: Real} – 声明了 \texttt{OurRational} 会接收一个整数类型的类型参数,且它自己属于实数类型。它声明了两个成员:\texttt{num::T} 和 \texttt{den::T}。这表明一个 \texttt{OurRational\{T\}} 的实例中会包含一对整数,且类型为 \texttt{T},其中一个表示分子,另一个表示分母。



现在事情开始变得有意思了,\texttt{OurRational} 只有一个内部构造函数,它的作用是检查 \texttt{num} 和 \texttt{den} 是否为 0,并确保构建的每个分数都是经过约分化简的形式,且分母为非负数。这可以令分子和分母同时除以它们的最大公约数来实现,最大公约数可以用 Julia 内置的 \texttt{gcd} 函数计算。由于 \texttt{gcd} 返回的最大公约数的符号是跟第一个参数 \texttt{den} 一致的,所以约分后一定会保证 \texttt{den} 的值为非负数。因为这是 \texttt{OurRational} 的唯一一个内部构造函数,所以我们可以确保构建出的 \texttt{OurRational} 对象一定是这种化简的形式。



为了方便,\texttt{OurRational} 也提供了一些其它的外部构造函数。第一个外部构造函数是“标准的”通用构造函数,当分子和分母的类型一致时,它就可以推导出类型参数 \texttt{T}。第二个外部构造函数可以用于分子和分母的类型不一致的情景,它会将分子和分母的类型提升至一个共同的类型,然后再委托第一个外部构造函数进行构造。第三个构造函数会将一个整数转化为分数,方法是将 1 当作分母。



在定义了外部构造函数之后,我们为 \texttt{⊘} 算符定义了一系列的方法,之后就可以使用 \texttt{⊘} 算符来写分数,比如 \texttt{1 ⊘ 2}。Julia 的 \texttt{Rational} 类型采用的是 \hyperlink{17539582191808611917}{\texttt{//}} 算符。在做上述定义之前,\texttt{⊘} 是一个无意的且未被定义的算符。它的行为与在\hyperlink{30883190695696392}{有理数}一节中描述的一致,注意它的所有行为都是那短短几行定义的。第一个也是最基础的定义只是将 \texttt{a ⊘ b} 中的 \texttt{a} 和 \texttt{b} 当作参数传递给 \texttt{OurRational} 的构造函数来实例化 \texttt{OurRational},当然这要求 \texttt{a} 和 \texttt{b} 分别都是整数。在 \texttt{⊘} 的某个操作数已经是分数的情况下,我们采用了一个有点不一样的方法来构建新的分数,这实际上等价于用分数除以一个整数。最后,我们也可以让 \texttt{⊘} 作用于复数,用来创建一个类型为 \texttt{Complex\{OurRational\}} 的对象,即一个实部和虚部都是分数的复数:




\begin{minted}{jlcon}
julia> z = (1 + 2im) ⊘ (1 - 2im);

julia> typeof(z)
Complex{OurRational{Int64}}

julia> typeof(z) <: Complex{OurRational}
false
\end{minted}



因此,尽管 \texttt{⊘} 算符通常会返回一个 \texttt{OurRational} 的实例,但倘若其中一个操作数是复整数,那么就会返回 \texttt{Complex\{OurRational\}}。感兴趣的话可以读一读 \href{https://github.com/JuliaLang/julia/blob/master/base/rational.jl}{\texttt{rational.jl}}:它实现了一个完整的 Julia 基本类型,但却非常的简短,而且是自包涵的。



\hypertarget{11047482593881141721}{}


\section{Outer-only constructors}



正如我们所看到的,典型的参数类型都有一个内部构造函数,它仅在全部的类型参数都已知的情况下才会被调用。例如,可以用 \texttt{Point\{Int\}} 调用,但\texttt{Point} 就不行。我们可以选择性的添加外部构造函数来自动推导并添加类型参数,比如,调用 \texttt{Point(1,2)} 来构造 \texttt{Point\{Int\}}。外部构造函数调用内部构造函数来实际创建实例。然而,在某些情况下,我们可能并不想要内部构造函数,从而达到禁止手动指定类型参数的目的。



例如,假设我们要定义一个类型用于存储数组以及其累加和:




\begin{minted}{jlcon}
julia> struct SummedArray{T<:Number,S<:Number}
           data::Vector{T}
           sum::S
       end

julia> SummedArray(Int32[1; 2; 3], Int32(6))
SummedArray{Int32,Int32}(Int32[1, 2, 3], 6)
\end{minted}



问题在于我们想让 \texttt{S} 的类型始终比 \texttt{T} 大,这样做是为了确保累加过程不会丢失信息。例如,当 \texttt{T} 是 \hyperlink{10103694114785108551}{\texttt{Int32}} 时,我们想让 \texttt{S} 是 \hyperlink{7720564657383125058}{\texttt{Int64}}。所以我们想要一种接口来禁止用户创建像 \texttt{SummedArray\{Int32,Int32\}} 这种类型的实例。一种实现方式是只提供一个 \texttt{SummedArray} 构造函数,当需要将其放入 \texttt{struct}-block 中,从而不让 Julia 提供默认的构造函数:




\begin{minted}{jlcon}
julia> struct SummedArray{T<:Number,S<:Number}
           data::Vector{T}
           sum::S
           function SummedArray(a::Vector{T}) where T
               S = widen(T)
               new{T,S}(a, sum(S, a))
           end
       end

julia> SummedArray(Int32[1; 2; 3], Int32(6))
ERROR: MethodError: no method matching SummedArray(::Array{Int32,1}, ::Int32)
Closest candidates are:
  SummedArray(::Array{T,1}) where T at none:4
\end{minted}



此构造函数将会被 \texttt{SummedArray(a)} 这种写法触发。\texttt{new\{T,S\}} 的这种写法允许指定待构建类型的参数,也就是说调用它会返回一个 \texttt{SummedArray\{T,S\}} 的实例。\texttt{new\{T,S\}} 也可以用于其它构造函数的定义中,但为了方便,Julia 会根据正在构造的类型自动推导出 \texttt{new\{\}} 花括号里的参数(如果可行的话)。



\hypertarget{10686378388163930476}{}


\chapter{类型转换和类型提升}



Julia 有一个提升系统,可以将数学运算符的参数提升为通用类型,如在前面章节中提到的\hyperlink{8249022581856827126}{整数和浮点数}、\hyperlink{16865688524696028421}{数学运算和初等函数}、\hyperlink{8510890508040013186}{类型}和\hyperlink{3842379394166369470}{方法}。在本节中,我们将解释类型提升系统如何工作,以及如何将其扩展到新的类型,并将其应用于除内置数学运算符之外的其他函数。传统上,编程语言在参数的类型提升上分为两大阵营:



\begin{itemize}
\item \textbf{内置数学类型和运算符的自动类型提升。}大多数语言中,内置数值类型,当作为带有中缀语法的算术运算符的操作数时,例如 \texttt{+}、\texttt{-}、\texttt{*} 和 \texttt{/} 将自动提升为通用类型,以产生预期的结果。举例来说,C、Java、Perl 和 Python,都将 \texttt{1 + 1.5} 的和作为浮点值 \texttt{2.5},即使 \texttt{+} 的一个操作数是整数。这些系统非常方便且设计得足够精细,以至于它对于程序员来讲通常是不可见的:在编写这样的表达式时,几乎没有人有意识地想到这种类型提升,但编译器和解释器必须在相加前执行转换,因为整数和浮点值无法按原样相加。因此,这种自动类型转换的复杂规则不可避免地是这些语言的规范和实现的一部分。


\item \textbf{没有自动类型提升。}这个阵营包括 Ada 和 ML——非常「严格的」 静态类型语言。在这些语言中,每个类型转换都必须由程序员明确指定。因此,示例表达式 \texttt{1 + 1.5} 在 Ada 和 ML 中都会导致编译错误。相反地,必须编写 \texttt{real(1) + 1.5},来在执行加法前将整数 \texttt{1} 显式转换为浮点值。然而,处处都显式转换是如此地不方便,以至于连 Ada 也有一定程度的自动类型转换:整数字面量被类型提升为预期的整数类型,浮点字面量同样被类型提升为适当的浮点类型。

\end{itemize}


在某种意义上,Julia 属于「无自动类型提升」类别:数学操作符只是具有特殊语法的函数,函数的参数永远不会自动转换。然而,人们可能会发现数学运算能应用于各种混合的参数类型,但这只是多态的多重分派的极端情况——这是 Julia 的分派和类型系统特别适合处理的情况。数学操作数的「自动」类型提升只是作为一个特殊的应用出现:Julia 带有预定义的数学运算符的 catch-all 分派规则,其在某些操作数类型的组合没有特定实现时调用。这些 catch-all 分派规则首先使用用户可定义的类型提升规则将所有操作数提升到一个通用的类型,然后针对结果值(现在已属于相同类型)调用相关运算符的特定实现。用户定义的类型可简单地加入这个类型提升系统,这需要先定义与其它类型进行相互类型转换的方法,接着提供一些类型提升规则来定义与其它类型混合时应该提升到什么类型。



\hypertarget{5183188243565893084}{}


\section{类型转换}



The standard way to obtain a value of a certain type \texttt{T} is to call the type{\textquotesingle}s constructor, \texttt{T(x)}. However, there are cases where it{\textquotesingle}s convenient to convert a value from one type to another without the programmer asking for it explicitly. One example is assigning a value into an array: if \texttt{A} is a \texttt{Vector\{Float64\}}, the expression \texttt{A[1] = 2} should work by automatically converting the \texttt{2} from \texttt{Int} to \texttt{Float64}, and storing the result in the array. This is done via the \hyperlink{1846942650946171605}{\texttt{convert}} function.



\texttt{convert} 函数通常接受两个参数:第一个是类型对象,第二个是需要转换为该类型的值。返回的是已转换后的值。理解这个函数最简单的办法就是尝试:




\begin{minted}{jlcon}
julia> x = 12
12

julia> typeof(x)
Int64

julia> convert(UInt8, x)
0x0c

julia> typeof(ans)
UInt8

julia> convert(AbstractFloat, x)
12.0

julia> typeof(ans)
Float64

julia> a = Any[1 2 3; 4 5 6]
2×3 Array{Any,2}:
 1  2  3
 4  5  6

julia> convert(Array{Float64}, a)
2×3 Array{Float64,2}:
 1.0  2.0  3.0
 4.0  5.0  6.0
\end{minted}



Conversion isn{\textquotesingle}t always possible, in which case a \hyperlink{68769522931907606}{\texttt{MethodError}} is thrown indicating that \texttt{convert} doesn{\textquotesingle}t know how to perform the requested conversion:




\begin{minted}{jlcon}
julia> convert(AbstractFloat, "foo")
ERROR: MethodError: Cannot `convert` an object of type String to an object of type AbstractFloat
[...]
\end{minted}



Some languages consider parsing strings as numbers or formatting numbers as strings to be conversions (many dynamic languages will even perform conversion for you automatically). This is not the case in Julia. Even though some strings can be parsed as numbers, most strings are not valid representations of numbers, and only a very limited subset of them are. Therefore in Julia the dedicated \hyperlink{14207407853646164654}{\texttt{parse}} function must be used to perform this operation, making it more explicit.



\hypertarget{7039772400060801175}{}


\subsection{什么时候使用 \texttt{convert} 函数?}



构造以下语言结构时需要调用 \texttt{convert} 函数:



\begin{itemize}
\item 对一个数组赋值会转换为数组元素的类型。


\item 对一个对象的字段赋值会转换为已声明的字段类型。


\item Constructing an object with \hyperlink{13888762393600028594}{\texttt{new}} converts to the object{\textquotesingle}s declared field types.


\item 对已声明类型的变量赋值(例如 \texttt{local x::T})会转换为该类型。


\item 已声明返回类型的函数会转换其返回值为该类型。


\item Passing a value to \hyperlink{14245046751182637566}{\texttt{ccall}} converts it to the corresponding argument type.

\end{itemize}


\hypertarget{11541557100875125079}{}


\subsection{类型转换与构造}



注意到 \texttt{convert(T, x)} 的行为似乎与 \texttt{T(x)} 几乎相同,它的确通常是这样。但是,有一个关键的语义差别:因为 \texttt{convert} 能被隐式调用,所以它的方法仅限于被认为是「安全」或「意料之内」的情况。\texttt{convert} 只会在表示事物的相同基本种类的类型之间进行转换(例如,不同的数字表示和不同的字符串编码)。它通常也是无损的;将值转换为其它类型并再次转换回去应该产生完全相同的值。



这是四种一般的构造函数与 \texttt{convert} 不同的情况:



\hypertarget{17931790162561418365}{}


\subsubsection{与其参数类型无关的类型的构造函数}



一些构造函数没有体现「转换」的概念。例如,\texttt{Timer(2)} 创建一个时长 2 秒的定时器,它实际上并不是从整数到定时器的「转换」。



\hypertarget{8367796589961400339}{}


\subsubsection{可变的集合}



如果 \texttt{x} 类型已经为 \texttt{T},\texttt{convert(T, x)} 应该返回原本的 \texttt{x}。相反地,如果 \texttt{T} 是一个可变的集合类型,那么 \texttt{T(x)} 应该总是创建一个新的集合(从 \texttt{x} 复制元素)。



\hypertarget{13864114731640700485}{}


\subsubsection{封装器类型}



对于某些「封装」其它值的类型,构造函数可能会将其参数封装在一个新对象中,即使它已经是所请求的类型。例如,用 \texttt{Some(x)} 表示封装了一个 \texttt{x} 值(在上下文中,其结果可能是一个 \texttt{Some} 或 \texttt{nothing})。但是,\texttt{x} 本身可能是对象 \texttt{Some(y)},在这种情况下,结果为 \texttt{Some(Some(y))},封装了两层。然而,\texttt{convert(Some, x)} 只会返回 \texttt{x},因为它已经是 \texttt{Some} 的实例了。



\hypertarget{1440871335026577968}{}


\subsubsection{不返回自身类型的实例的构造函数}



在\emph{极少见}的情况下,构造函数 \texttt{T(x)} 返回一个类型不为 \texttt{T} 的对象是有意义的。如果封装器类型是它自身的反转(例如 \texttt{Flip(Flip(x)) === x}),或者在重构库时为了支持某个旧的调用语法以实现向后兼容,则可能发生这种情况。但是,\texttt{convert(T, x)} 应该总是返回一个类型为 \texttt{T} 的值。



\hypertarget{16585115796165922811}{}


\subsection{定义新的类型转换}



在定义新类型时,最初创建它的所有方法都应定义为构造函数。如果隐式类型转换很明显是有用的,并且某些构造函数满足上面的「安全」标准,那么可以考虑添加 \texttt{convert} 方法。这些方法通常非常简单,因为它们只需要调用适当的构造函数。此类定义可能会像这样:




\begin{minted}{julia}
convert(::Type{MyType}, x) = MyType(x)
\end{minted}



此方法的第一个参数的类型是\hyperlink{14008188290941962431}{单态类型} \texttt{Type\{MyType\}},其唯一实例是 \texttt{MyType}。因此,此方法仅在第一个参数是类型值 \texttt{MyType} 时才被调用。请注意第一个参数使用的语法:在 \texttt{::} 符号之前省略了参数名,只是给出了参数类型。这是 Julia 中用于函数参数的语法,该参数的类型已经指定,但其值无需通过名称引用。在此例中,由于参数类型是单态类型,我们已经知道其值而无需引用参数名称。



某些抽象类型的所有实例默认都被认为是「足够相似的」,在 Julia Base 中也提供了通用的 \texttt{convert} 定义。例如,这个定义声明通过调用单参数构造函数将任何 \texttt{Number} 类型 \texttt{convert} 为其它任何 \texttt{Number} 类型是有效的:




\begin{minted}{julia}
convert(::Type{T}, x::Number) where {T<:Number} = T(x)
\end{minted}



这意味着新的 \texttt{Number} 类型只需要定义构造函数,因为此定义将为它们处理 \texttt{convert}。在参数已经是所请求的类型的情况下,用恒同变换来处理 \texttt{convert}。




\begin{minted}{julia}
convert(::Type{T}, x::T) where {T<:Number} = x
\end{minted}



Similar definitions exist for \texttt{AbstractString}, \hyperlink{6514416309183787338}{\texttt{AbstractArray}}, and \hyperlink{6373987858401217649}{\texttt{AbstractDict}}.



\hypertarget{701866407360133120}{}


\section{类型提升}



类型提升是指将一组混合类型的值转换为单个通用类型。尽管不是绝对必要的,但一般暗示被转换的值的通用类型可以忠实地表示所有原始值。此意义下,术语「类型提升」是合适的,因为值被转换为「更大」的类型——即能用一个通用类型表示所有输入值的类型。但重要的是,不要将它与面向对象(结构)超类或 Julia 的抽象超类型混淆:类型提升与类型层次结构无关,而与备选的表示之间的转换有关。例如,尽管每个 \hyperlink{10103694114785108551}{\texttt{Int32}} 值可以表示为 \hyperlink{5027751419500983000}{\texttt{Float64}} 值,但 \texttt{Int32} 不是 \texttt{Float64} 的子类型。



Promotion to a common {\textquotedbl}greater{\textquotedbl} type is performed in Julia by the \hyperlink{1760874576431605095}{\texttt{promote}} function, which takes any number of arguments, and returns a tuple of the same number of values, converted to a common type, or throws an exception if promotion is not possible. The most common use case for promotion is to convert numeric arguments to a common type:




\begin{minted}{jlcon}
julia> promote(1, 2.5)
(1.0, 2.5)

julia> promote(1, 2.5, 3)
(1.0, 2.5, 3.0)

julia> promote(2, 3//4)
(2//1, 3//4)

julia> promote(1, 2.5, 3, 3//4)
(1.0, 2.5, 3.0, 0.75)

julia> promote(1.5, im)
(1.5 + 0.0im, 0.0 + 1.0im)

julia> promote(1 + 2im, 3//4)
(1//1 + 2//1*im, 3//4 + 0//1*im)
\end{minted}



浮点值被提升为最大的浮点参数类型。整数值会被提升为本机机器字大小或最大的整数参数类型中较大的一个。整数和浮点值的混合会被提升为一个足以包含所有值的浮点类型。与有理数混合的整数会被提升有理数。与浮点数混合的有理数会被提升为浮点数。与实数值混合的复数值会被提升为合适类型的复数值。



这就是使用类型提升的全部内容。剩下的只是聪明的应用,最典型的「聪明」应用是数值操作(如 \texttt{+}、\texttt{-}、\texttt{*} 和 \texttt{/})的 catch-all 方法的定义。以下是在 \href{https://github.com/JuliaLang/julia/blob/master/base/promotion.jl}{\texttt{promotion.jl}} 中给出的几个 catch-all 方法的定义:




\begin{minted}{julia}
+(x::Number, y::Number) = +(promote(x,y)...)
-(x::Number, y::Number) = -(promote(x,y)...)
*(x::Number, y::Number) = *(promote(x,y)...)
/(x::Number, y::Number) = /(promote(x,y)...)
\end{minted}



这些方法的定义表明,如果没有更特殊的规则来加、减、乘及除一对数值,则将这些值提升为通用类型并再试一次。这就是它的全部内容:在其它任何地方都不需要为数值操作担心到通用数值类型的类型提升——它会自动进行。许多算术和数学函数的 catch-all 类型提升方法的定义在 \href{https://github.com/JuliaLang/julia/blob/master/base/promotion.jl}{\texttt{promotion.jl}} 中,但除此之外,Julia Base 中几乎不再需要调用 \texttt{promote}。\texttt{promote} 最常用于外部构造方法中,为了更方便,可允许使用混合类型的构造函数调用委托给一个内部构造函数,并将字段提升为适当的通用类型。例如,回想一下,\href{https://github.com/JuliaLang/julia/blob/master/base/rational.jl}{\texttt{rational.jl}} 提供了以下外部构造方法:




\begin{minted}{julia}
Rational(n::Integer, d::Integer) = Rational(promote(n,d)...)
\end{minted}



这允许像下面这样的调用正常工作:




\begin{minted}{jlcon}
julia> Rational(Int8(15),Int32(-5))
-3//1

julia> typeof(ans)
Rational{Int32}
\end{minted}



对于大多数用户定义的类型,最好要求程序员明确地向构造函数提供期待的类型,但有时,尤其是对于数值问题,自动进行类型提升会很方便。



\hypertarget{13578530399861722948}{}


\subsection{定义类型提升规则}



Although one could, in principle, define methods for the \texttt{promote} function directly, this would require many redundant definitions for all possible permutations of argument types. Instead, the behavior of \texttt{promote} is defined in terms of an auxiliary function called \hyperlink{16547112220540026290}{\texttt{promote\_rule}}, which one can provide methods for. The \texttt{promote\_rule} function takes a pair of type objects and returns another type object, such that instances of the argument types will be promoted to the returned type. Thus, by defining the rule:




\begin{minted}{julia}
promote_rule(::Type{Float64}, ::Type{Float32}) = Float64
\end{minted}



one declares that when 64-bit and 32-bit floating-point values are promoted together, they should be promoted to 64-bit floating-point. The promotion type does not need to be one of the argument types. For example, the following promotion rules both occur in Julia Base:




\begin{minted}{julia}
promote_rule(::Type{BigInt}, ::Type{Float64}) = BigFloat
promote_rule(::Type{BigInt}, ::Type{Int8}) = BigInt
\end{minted}



在后一种情况下,输出类型是 \hyperlink{423405808990690832}{\texttt{BigInt}},因为 \texttt{BigInt} 是唯一一个足以容纳任意精度整数运算结果的类型。还要注意,不需要同时定义 \texttt{promote\_rule(::Type\{A\}, ::Type\{B\})} 和 \texttt{promote\_rule(::Type\{B\}, ::Type\{A\})}——对称性隐含在类型提升过程中使用 \texttt{promote\_rule} 的方式。



The \texttt{promote\_rule} function is used as a building block to define a second function called \hyperlink{15048881762587391286}{\texttt{promote\_type}}, which, given any number of type objects, returns the common type to which those values, as arguments to \texttt{promote} should be promoted. Thus, if one wants to know, in absence of actual values, what type a collection of values of certain types would promote to, one can use \texttt{promote\_type}:




\begin{minted}{jlcon}
julia> promote_type(Int8, Int64)
Int64
\end{minted}



在内部,\texttt{promote\_type} 在 \texttt{promote} 中用于确定参数值应被转换为什么类型以便进行类型提升。但是,它本身可能是有用的。好奇的读者可以阅读 \href{https://github.com/JuliaLang/julia/blob/master/base/promotion.jl}{\texttt{promotion.jl}},该文件用大概 35 行定义了完整的类型提升规则。



\hypertarget{8231690970182763012}{}


\subsection{案例研究:有理数的类型提升}



最后,我们来完成关于 Julia 的有理数类型的案例研究,该案例通过以下类型提升规则相对复杂地使用了类型提升机制:




\begin{minted}{julia}
promote_rule(::Type{Rational{T}}, ::Type{S}) where {T<:Integer,S<:Integer} = Rational{promote_type(T,S)}
promote_rule(::Type{Rational{T}}, ::Type{Rational{S}}) where {T<:Integer,S<:Integer} = Rational{promote_type(T,S)}
promote_rule(::Type{Rational{T}}, ::Type{S}) where {T<:Integer,S<:AbstractFloat} = promote_type(T,S)
\end{minted}



第一条规则说,使用其它整数类型类型提升有理数类型会得到个有理数类型,其分子/分母类型是使用其它整数类型提升该有理数分子/分母类型的结果。第二条规则将相同的逻辑应用于两种不同的有理数类型,它们进行类型提升会得到有理数类型,其分子/分母类型是它们各自的分子/分母类型进行提升的结果。第三个也是最后一个规则规定,使用浮点数类型提升有理数类型与使用该浮点数类型提升其分子/分母类型会产生相同的类型。



这一小部分的类型提升规则,连同该类型的构造函数和数字的默认 \texttt{convert} 方法,便足以使有理数与 Julia 的其它数值类型——整数、浮点数和复数——完全自然地互操作。通过以相同的方式提供类型转换方法和类型提升规则,任何用户定义的数值类型都可像 Julia 的预定义数值类型一样自然地进行互操作。



\hypertarget{5506253490972465797}{}


\chapter{接口}



Julia 的很多能力和扩展性都来自于一些非正式的接口。通过为自定义的类型扩展一些特定的方法,自定义类型的对象不但获得那些方法的功能,而且也能够用于其它的基于那些行为而定义的通用方法中。



\hypertarget{5510379658285713272}{}


\section{迭代}



\begin{table}[h] 
  \centering 
  \tymin=0.1\textwidth
  \tymax=0.5\textwidth
\begin{tabulary}{\linewidth}{LLL}
  \toprule
  \textbf{必需方法} &  & \textbf{简短描述} \\
  \midrule
  
  \texttt{iterate(iter)} &  & 通常返回由第一项及其初始状态组成的元组,但如果为空,则返回 \hyperlink{9331422207248206047}{\texttt{nothing}} \\ \midrule
  \texttt{iterate(iter, state)} &  & 通常返回由下一项及其状态组成的元组,或者在没有下一项存在时返回 \texttt{nothing}。 \\ 
  
  \toprule
  \textbf{重要可选方法} & \textbf{默认定义} & \textbf{简短描述} \\ \midrule
  \texttt{IteratorSize(IterType)} & \texttt{HasLength()} & \texttt{HasLength()},\texttt{HasShape\{N\}()},\texttt{IsInfinite()} 或者 \texttt{SizeUnknown()} 中合适的一个 \\ \midrule
  \texttt{IteratorEltype(IterType)} & \texttt{HasEltype()} & \texttt{EltypeUnknown()} 或 \texttt{HasEltype()} 中合适的一个 \\ \midrule
  \texttt{eltype(IterType)} & \texttt{Any} & 由 \texttt{iterate()} 返回元组中第一项的类型。 \\ \midrule
  \texttt{length(iter)} & (\emph{未定义}) & 项数,如果已知 \\ \midrule
  \texttt{size(iter, [dim])} & (\emph{未定义}) & 在各个维度上项数,如果已知 \\
  \bottomrule
\end{tabulary}
\end{table}



% \begin{table}[h]

% \begin{tabulary}{\linewidth}{|L|L|L|}
% \hline
% 必需方法 &  & 简短描述 \\
% \hline
% \texttt{iterate(iter)} &  & 通常返回由第一项及其初始状态组成的元组,但如果为空,则返回 \hyperlink{9331422207248206047}{\texttt{nothing}} \\
% \hline
% \texttt{iterate(iter, state)} &  & 通常返回由下一项及其状态组成的元组,或者在没有下一项存在时返回 \texttt{nothing}。 \\
% \hline
% \textbf{重要可选方法} & \textbf{默认定义} & \textbf{简短描述} \\
% \hline
% \texttt{IteratorSize(IterType)} & \texttt{HasLength()} & \texttt{HasLength()},\texttt{HasShape\{N\}()},\texttt{IsInfinite()} 或者 \texttt{SizeUnknown()} 中合适的一个 \\
% \hline
% \texttt{IteratorEltype(IterType)} & \texttt{HasEltype()} & \texttt{EltypeUnknown()} 或 \texttt{HasEltype()} 中合适的一个 \\
% \hline
% \texttt{eltype(IterType)} & \texttt{Any} & 由 \texttt{iterate()} 返回元组中第一项的类型。 \\
% \hline
% \texttt{length(iter)} & (\emph{未定义}) & 项数,如果已知 \\
% \hline
% \texttt{size(iter, [dim])} & (\emph{未定义}) & 在各个维度上项数,如果已知 \\
% \hline
% \end{tabulary}

% \end{table}




\begin{table}[h]

\begin{tabulary}{\linewidth}{|L|L|}
\hline
由 \texttt{IteratorSize(IterType)} 返回的值 & 必需方法 \\
\hline
\texttt{HasLength()} & \hyperlink{3699181304419743826}{\texttt{length(iter)}} \\
\hline
\texttt{HasShape\{N\}()} & \texttt{length(iter)} 和 \texttt{size(iter, [dim])} \\
\hline
\texttt{IsInfinite()} & (\emph{无}) \\
\hline
\texttt{SizeUnknown()} & (\emph{无}) \\
\hline
\end{tabulary}

\end{table}




\begin{table}[h]

\begin{tabulary}{\linewidth}{|L|L|}
\hline
由 \texttt{IteratorEltype(IterType)} 返回的值 & 必需方法 \\
\hline
\texttt{HasEltype()} & \texttt{eltype(IterType)} \\
\hline
\texttt{EltypeUnknown()} & (\emph{none}) \\
\hline
\end{tabulary}

\end{table}



顺序迭代由 \hyperlink{1722534687975587846}{\texttt{iterate}} 函数实现。 Julia 的迭代器可以从对象外部跟踪迭代状态,而不是在迭代过程中改变对象本身。 迭代过程中的返回一个包含了当前迭代值及其状态的元组,或者在没有元素存在的情况下返回 \texttt{nothing}。 状态对象将在下一次迭代时传递回 iterate 函数,并且通常被认为是可迭代对象的私有实现细节。



任何定义了这个函数的对象都是可迭代的,并且可以被应用到\hyperlink{16454089156260356769}{许多依赖迭代的函数上} 。 也可以直接被应用到  \hyperlink{9105224580875818383}{\texttt{for}} 循环中,因为根据语法:




\begin{minted}{julia}
for i in iter   # or  "for i = iter"
    # body
end
\end{minted}



以上代码被解释为:




\begin{minted}{julia}
next = iterate(iter)
while next !== nothing
    (i, state) = next
    # body
    next = iterate(iter, state)
end
\end{minted}



举一个简单的例子:一组定长数据的平方数迭代序列:




\begin{minted}{jlcon}
julia> struct Squares
           count::Int
       end

julia> Base.iterate(S::Squares, state=1) = state > S.count ? nothing : (state*state, state+1)
\end{minted}



仅仅定义了 \hyperlink{1722534687975587846}{\texttt{iterate}} 函数的 \texttt{Squares} 类型就已经很强大了。我们现在可以迭代所有的元素了:




\begin{minted}{jlcon}
julia> for i in Squares(7)
           println(i)
       end
1
4
9
16
25
36
49
\end{minted}



我们可以利用许多内置方法来处理迭代,比如标准库 \texttt{Statistics}  中的 \hyperlink{17277603976666670638}{\texttt{in}},\hyperlink{15061550543970113934}{\texttt{mean}} 和 \hyperlink{1955374586742019663}{\texttt{std}} 。




\begin{minted}{jlcon}
julia> 25 in Squares(10)
true

julia> using Statistics

julia> mean(Squares(100))
3383.5

julia> std(Squares(100))
3024.355854282583
\end{minted}



我们可以扩展一些其它的方法,为 Julia 提供有关此可迭代集合的更多信息。我们知道 \texttt{Squares} 序列中的元素总是 \texttt{Int} 型的。通过扩展 \hyperlink{6396209842929672718}{\texttt{eltype}} 方法,我们可以给 Julia 更多信息来帮助其在更复杂的方法中生成更具体的代码。我们同时也知道该序列中的元素数目,故同样地也可以扩展 \hyperlink{9362803119463040896}{\texttt{length}}:




\begin{minted}{jlcon}
julia> Base.eltype(::Type{Squares}) = Int # Note that this is defined for the type

julia> Base.length(S::Squares) = S.count
\end{minted}



现在,当我们让 Julia 去 \hyperlink{6278865767444641812}{\texttt{collect}} 所有元素到一个数组中时,Julia 可以预分配一个适当大小的 \texttt{Vector\{Int\}},而不是盲目地 \hyperlink{18026893834387542681}{\texttt{push!}} 每一个元素到 \texttt{Vector\{Any\}}:




\begin{minted}{jlcon}
julia> collect(Squares(4))
4-element Array{Int64,1}:
  1
  4
  9
 16
\end{minted}



尽管大多时候我们都可以依赖一些通用的实现,但某些时候,如果我们知道一个更简单的算法,可以用其扩展具体方法。例如,计算平方和有公式,因此可以扩展出一个更高效的解法来替代通用方法:




\begin{minted}{jlcon}
julia> Base.sum(S::Squares) = (n = S.count; return n*(n+1)*(2n+1)÷6)

julia> sum(Squares(1803))
1955361914
\end{minted}



这种模式在 Julia Base 中很常见,一些必须实现的方法构成了一个小的集合,从而定义出一个非正式的接口,用于实现一些更加炫酷的操作。某些应用场景中,一些类型有更高效的算法,故可以扩展出额外的专用方法。



能以\emph{逆序}迭代集合也很有用,这可由 \hyperlink{12943296479800134710}{\texttt{Iterators.reverse(iterator)}} 迭代实现。但是,为了实际支持逆序迭代,迭代器类型 \texttt{T} 需要为 \texttt{Iterators.Reverse\{T\}} 实现 \texttt{iterate}。(给定 \texttt{r::Iterators.Reverse\{T\}},类型 \texttt{T} 的底层迭代器是 \texttt{r.itr}。)在我们的 \texttt{Squares} 示例中,我们可以实现 \texttt{Iterators.Reverse\{Squares\}} 方法:




\begin{minted}{jlcon}
julia> Base.iterate(rS::Iterators.Reverse{Squares}, state=rS.itr.count) = state < 1 ? nothing : (state*state, state-1)

julia> collect(Iterators.reverse(Squares(4)))
4-element Array{Int64,1}:
 16
  9
  4
  1
\end{minted}



\hypertarget{14566118977838625303}{}


\section{Indexing}




\begin{table}[h]

\begin{tabulary}{\linewidth}{|L|L|}
\hline
Methods to implement & Brief description \\
\hline
\texttt{getindex(X, i)} & \texttt{X[i]}, indexed element access \\
\hline
\texttt{setindex!(X, v, i)} & \texttt{X[i] = v}, indexed assignment \\
\hline
\texttt{firstindex(X)} & The first index, used in \texttt{X[begin]} \\
\hline
\texttt{lastindex(X)} & The last index, used in \texttt{X[end]} \\
\hline
\end{tabulary}

\end{table}



For the \texttt{Squares} iterable above, we can easily compute the \texttt{i}th element of the sequence by squaring it.  We can expose this as an indexing expression \texttt{S[i]}. To opt into this behavior, \texttt{Squares} simply needs to define \hyperlink{13720608614876840481}{\texttt{getindex}}:




\begin{minted}{jlcon}
julia> function Base.getindex(S::Squares, i::Int)
           1 <= i <= S.count || throw(BoundsError(S, i))
           return i*i
       end

julia> Squares(100)[23]
529
\end{minted}



Additionally, to support the syntax \texttt{S[begin]} and \texttt{S[end]}, we must define \hyperlink{16943669671291374223}{\texttt{firstindex}} and \hyperlink{15780929618270241785}{\texttt{lastindex}} to specify the first and last valid indices, respectively:




\begin{minted}{jlcon}
julia> Base.firstindex(S::Squares) = 1

julia> Base.lastindex(S::Squares) = length(S)

julia> Squares(23)[end]
529
\end{minted}



但请注意,上面只定义了带有一个整数索引的 \hyperlink{13720608614876840481}{\texttt{getindex}}。使用除 \texttt{Int} 外的任何值进行索引会抛出 \hyperlink{68769522931907606}{\texttt{MethodError}},表示没有匹配的方法。为了支持使用某个范围内的 \texttt{Int} 或 \texttt{Int} 向量进行索引,必须编写单独的方法:




\begin{minted}{jlcon}
julia> Base.getindex(S::Squares, i::Number) = S[convert(Int, i)]

julia> Base.getindex(S::Squares, I) = [S[i] for i in I]

julia> Squares(10)[[3,4.,5]]
3-element Array{Int64,1}:
  9
 16
 25
\end{minted}



虽然这开始支持更多\hyperlink{16717190941363337071}{某些内置类型支持的索引操作},但仍然有很多行为不支持。因为我们为 \texttt{Squares} 序列所添加的行为,它开始看起来越来越像向量。我们可以正式定义其为 \hyperlink{6514416309183787338}{\texttt{AbstractArray}} 的子类型,而不是自己定义所有这些行为。



\hypertarget{522338241536202486}{}


\section{抽象数组}




%% 旋转页面

\newgeometry{hmargin=1.5cm,vmargin=2cm}
\begin{landscape}

\begin{table}[h] 
  \centering 
\begin{tabulary}{\linewidth}{LLL}
  \toprule
  \textbf{需要实现的方法} &  & \textbf{简短描述} \\
  \midrule
  
  \texttt{size(A)} &  & 返回包含 \texttt{A} 各维度大小的元组 \\ \midrule
  \texttt{getindex(A, i::Int)} &  & (若为 \texttt{IndexLinear})线性标量索引 \\ \midrule
  \texttt{getindex(A, I::Vararg\{Int, N\})} &  & (若为 \texttt{IndexCartesian},其中 \texttt{N = ndims(A)})N 维标量索引 \\ \midrule
  \texttt{setindex!(A, v, i::Int)} &  & (若为 \texttt{IndexLinear})线性索引元素赋值 \\ \midrule
  \texttt{setindex!(A, v, I::Vararg\{Int, N\})} &  & (若为 \texttt{IndexCartesian},其中 \texttt{N = ndims(A)})N 维标量索引元素赋值 \\ 
  
  \toprule
  \textbf{可选方法} & \textbf{默认定义} & \textbf{简短描述} \\ 
  \midrule
  \texttt{IndexStyle(::Type)} & \texttt{IndexCartesian()} & 返回 \texttt{IndexLinear()} 或 \texttt{IndexCartesian()}。请参阅下文描述。 \\ \midrule
  \texttt{getindex(A, I...)} & 基于标量 \texttt{getindex} 定义 & \hyperlink{14469287548874312017}{多维非标量索引} \\ \midrule
  \texttt{setindex!(A, X, I...)} & 基于标量 \texttt{setindex!} 定义 & \hyperlink{14469287548874312017}{多维非标量索引元素赋值} \\ \midrule
  \texttt{iterate} & 基于标量 \texttt{getindex} 定义 & Iteration \\ \midrule
  \texttt{length(A)} & \texttt{prod(size(A))} & 元素数 \\ \midrule
  \texttt{similar(A)} & \texttt{similar(A, eltype(A), size(A))} & 返回具有相同形状和元素类型的可变数组 \\ \midrule
  \texttt{similar(A, ::Type\{S\})} & \texttt{similar(A, S, size(A))} & 返回具有相同形状和指定元素类型的可变数组 \\ \midrule
  \texttt{similar(A, dims::Dims)} & \texttt{similar(A, eltype(A), dims)} & 返回具有相同元素类型和大小为 \emph{dims} 的可变数组 \\ \midrule
  \texttt{similar(A, ::Type\{S\}, dims::Dims)} & \texttt{Array\{S\}(undef, dims)} & 返回具有指定元素类型及大小的可变数组 \\ 
  
  \toprule
  \textbf{不遵循惯例的索引} & \textbf{默认定义} & \textbf{简短描述} \\ 
  \midrule
  \texttt{axes(A)} & \texttt{map(OneTo, size(A))} & 返回有效索引的 \texttt{AbstractUnitRange} \\ \midrule
  \texttt{similar(A, ::Type\{S\}, inds)} & \texttt{similar(A, S, Base.to\_shape(inds))} & 返回使用特殊索引 \texttt{inds} 的可变数组(详见下文) \\ \midrule
  \texttt{similar(T::Union\{Type,Function\}, inds)} & \texttt{T(Base.to\_shape(inds))} & 返回类似于 \texttt{T} 的使用特殊索引 \texttt{inds} 的数组(详见下文) \\ 
  \bottomrule
\end{tabulary}
\end{table}

\end{landscape}
\restoregeometry


%% 正常版式 -------------------------------------------------------------------------------
% \begin{table}[h] 
%   \centering 
% \begin{tabulary}{\linewidth}{LLL}
%   \toprule
%   \textbf{需要实现的方法} &  & \textbf{简短描述} \\
%   \midrule

%   \texttt{size(A)} &  & 返回包含 \texttt{A} 各维度大小的元组 \\ \midrule
%   \texttt{getindex(A, i::Int)} &  & (若为 \texttt{IndexLinear})线性标量索引 \\ \midrule
%   \texttt{getindex(A, I::Vararg\{Int, N\})} &  & (若为 \texttt{IndexCartesian},其中 \texttt{N = ndims(A)})N 维标量索引 \\ \midrule
%   \texttt{setindex!(A, v, i::Int)} &  & (若为 \texttt{IndexLinear})线性索引元素赋值 \\ \midrule
%   \texttt{setindex!(A, v, I::Vararg\{Int, N\})} &  & (若为 \texttt{IndexCartesian},其中 \texttt{N = ndims(A)})N 维标量索引元素赋值 \\ 

%   \toprule
%   \textbf{可选方法} & \textbf{默认定义} & \textbf{简短描述} \\ 
%   \midrule
%   \texttt{IndexStyle(::Type)} & \texttt{IndexCartesian()} & 返回 \texttt{IndexLinear()} 或 \texttt{IndexCartesian()}。请参阅下文描述。 \\ \midrule
%   \texttt{getindex(A, I...)} & 基于标量 \texttt{getindex} 定义 & \hyperlink{14469287548874312017}{多维非标量索引} \\ \midrule
%   \texttt{setindex!(A, X, I...)} & 基于标量 \texttt{setindex!} 定义 & \hyperlink{14469287548874312017}{多维非标量索引元素赋值} \\ \midrule
%   \texttt{iterate} & 基于标量 \texttt{getindex} 定义 & Iteration \\ \midrule
%   \texttt{length(A)} & \texttt{prod(size(A))} & 元素数 \\ \midrule
%   \texttt{similar(A)} & \texttt{similar(A, eltype(A), size(A))} & 返回具有相同形状和元素类型的可变数组 \\ \midrule
%   \texttt{similar(A, ::Type\{S\})} & \texttt{similar(A, S, size(A))} & 返回具有相同形状和指定元素类型的可变数组 \\ \midrule
%   \texttt{similar(A, dims::Dims)} & \texttt{similar(A, eltype(A), dims)} & 返回具有相同元素类型和大小为 \emph{dims} 的可变数组 \\ \midrule
%   \texttt{similar(A, ::Type\{S\}, dims::Dims)} & \texttt{Array\{S\}(undef, dims)} & 返回具有指定元素类型及大小的可变数组 \\ 

%   \toprule
%   \textbf{不遵循惯例的索引} & \textbf{默认定义} & \textbf{简短描述} \\ 
%   \midrule
%   \texttt{axes(A)} & \texttt{map(OneTo, size(A))} & 返回有效索引的 \texttt{AbstractUnitRange} \\ \midrule
%   \texttt{similar(A, ::Type\{S\}, inds)} & \texttt{similar(A, S, Base.to\_shape(inds))} & 返回使用特殊索引 \texttt{inds} 的可变数组(详见下文) \\ \midrule
%   \texttt{similar(T::Union\{Type,Function\}, inds)} & \texttt{T(Base.to\_shape(inds))} & 返回类似于 \texttt{T} 的使用特殊索引 \texttt{inds} 的数组(详见下文) \\ 
%   \bottomrule
% \end{tabulary}
% \end{table}


% \begin{table}[h]

% \begin{tabulary}{\linewidth}{|L|L|L|}
% \hline
% 需要实现的方法 &  & 简短描述 \\
% \hline
% \texttt{size(A)} &  & 返回包含 \texttt{A} 各维度大小的元组 \\
% \hline
% \texttt{getindex(A, i::Int)} &  & (若为 \texttt{IndexLinear})线性标量索引 \\
% \hline
% \texttt{getindex(A, I::Vararg\{Int, N\})} &  & (若为 \texttt{IndexCartesian},其中 \texttt{N = ndims(A)})N 维标量索引 \\
% \hline
% \texttt{setindex!(A, v, i::Int)} &  & (若为 \texttt{IndexLinear})线性索引元素赋值 \\
% \hline
% \texttt{setindex!(A, v, I::Vararg\{Int, N\})} &  & (若为 \texttt{IndexCartesian},其中 \texttt{N = ndims(A)})N 维标量索引元素赋值 \\
% \hline
% \textbf{可选方法} & \textbf{默认定义} & \textbf{简短描述} \\
% \hline
% \texttt{IndexStyle(::Type)} & \texttt{IndexCartesian()} & 返回 \texttt{IndexLinear()} 或 \texttt{IndexCartesian()}。请参阅下文描述。 \\
% \hline
% \texttt{getindex(A, I...)} & 基于标量 \texttt{getindex} 定义 & \hyperlink{16717190941363337071}{多维非标量索引} \\
% \hline
% \texttt{setindex!(A, X, I...)} & 基于标量 \texttt{setindex!} 定义 & \hyperlink{16717190941363337071}{多维非标量索引元素赋值} \\
% \hline
% \texttt{iterate} & 基于标量 \texttt{getindex} 定义 & Iteration \\
% \hline
% \texttt{length(A)} & \texttt{prod(size(A))} & 元素数 \\
% \hline
% \texttt{similar(A)} & \texttt{similar(A, eltype(A), size(A))} & 返回具有相同形状和元素类型的可变数组 \\
% \hline
% \texttt{similar(A, ::Type\{S\})} & \texttt{similar(A, S, size(A))} & 返回具有相同形状和指定元素类型的可变数组 \\
% \hline
% \texttt{similar(A, dims::Dims)} & \texttt{similar(A, eltype(A), dims)} & 返回具有相同元素类型和大小为 \emph{dims} 的可变数组 \\
% \hline
% \texttt{similar(A, ::Type\{S\}, dims::Dims)} & \texttt{Array\{S\}(undef, dims)} & 返回具有指定元素类型及大小的可变数组 \\
% \hline
% \textbf{不遵循惯例的索引} & \textbf{默认定义} & \textbf{简短描述} \\
% \hline
% \texttt{axes(A)} & \texttt{map(OneTo, size(A))} & 返回有效索引的 \texttt{AbstractUnitRange} \\
% \hline
% \texttt{similar(A, ::Type\{S\}, inds)} & \texttt{similar(A, S, Base.to\_shape(inds))} & 返回使用特殊索引 \texttt{inds} 的可变数组(详见下文) \\
% \hline
% \texttt{similar(T::Union\{Type,Function\}, inds)} & \texttt{T(Base.to\_shape(inds))} & 返回类似于 \texttt{T} 的使用特殊索引 \texttt{inds} 的数组(详见下文) \\
% \hline
% \end{tabulary}

% \end{table}



如果一个类型被定义为 \texttt{AbstractArray} 的子类型,那它就继承了一大堆丰富的行为,包括构建在单元素访问之上的迭代和多维索引。有关更多支持的方法,请参阅文档 \hyperlink{16720099245556932994}{多维数组} 及 \hyperlink{3951748617092839742}{Julia Base}。



定义 \texttt{AbstractArray} 子类型的关键部分是 \hyperlink{7782790551324367092}{\texttt{IndexStyle}}。由于索引是数组的重要部分且经常出现在 hot loops 中,使索引和索引赋值尽可能高效非常重要。数组数据结构通常以两种方式定义:要么仅使用一个索引(即线性索引)来最高效地访问其元素,要么实际上使用由各个维度确定的索引访问其元素。这两种方式被 Julia 标记为 \texttt{IndexLinear()} 和 \texttt{IndexCartesian()}。把线性索引转换为多重索引下标通常代价高昂,因此这提供了基于 traits 机制,以便能为所有矩阵类型提供高效的通用代码。



此区别决定了该类型必须定义的标量索引方法。\texttt{IndexLinear()} 很简单:只需定义 \texttt{getindex(A::ArrayType, i::Int)}。当数组后用多维索引集进行索引时,回退 \texttt{getindex(A::AbstractArray, I...)()} 高效地将该索引转换为线性索引,然后调用上述方法。另一方面,\texttt{IndexCartesian()} 数组需要为每个支持的、使用 \texttt{ndims(A)} 个 \texttt{Int} 索引的维度定义方法。例如,\texttt{SparseArrays} 标准库里的 \hyperlink{15099699527958384292}{\texttt{SparseMatrixCSC}} 只支持二维,所以它只定义了 \texttt{getindex(A::SparseMatrixCSC, i::Int, j::Int)}。\hyperlink{1309244355901386657}{\texttt{setindex!}} 也是如此。



回到上面的平方数序列,我们可以将它定义为 \texttt{AbstractArray\{Int, 1\}} 的子类型:




\begin{minted}{jlcon}
julia> struct SquaresVector <: AbstractArray{Int, 1}
           count::Int
       end

julia> Base.size(S::SquaresVector) = (S.count,)

julia> Base.IndexStyle(::Type{<:SquaresVector}) = IndexLinear()

julia> Base.getindex(S::SquaresVector, i::Int) = i*i
\end{minted}



请注意,指定 \texttt{AbstractArray} 的两个参数非常重要;第一个参数定义了 \hyperlink{6396209842929672718}{\texttt{eltype}},第二个则定义了 \hyperlink{1688406579181746010}{\texttt{ndims}}。该超类型和这三个方法就足以使 \texttt{SquaresVector} 变成一个可迭代、可索引且功能齐全的数组:




\begin{minted}{jlcon}
julia> s = SquaresVector(4)
4-element SquaresVector:
  1
  4
  9
 16

julia> s[s .> 8]
2-element Array{Int64,1}:
  9
 16

julia> s + s
4-element Array{Int64,1}:
  2
  8
 18
 32

julia> sin.(s)
4-element Array{Float64,1}:
  0.8414709848078965
 -0.7568024953079282
  0.4121184852417566
 -0.2879033166650653
\end{minted}



作为一个更复杂的例子,让我们在 \hyperlink{3089397136845322041}{\texttt{Dict}} 之上定义自己的玩具性质的 N 维稀疏数组类型。




\begin{minted}{jlcon}
julia> struct SparseArray{T,N} <: AbstractArray{T,N}
           data::Dict{NTuple{N,Int}, T}
           dims::NTuple{N,Int}
       end

julia> SparseArray(::Type{T}, dims::Int...) where {T} = SparseArray(T, dims);

julia> SparseArray(::Type{T}, dims::NTuple{N,Int}) where {T,N} = SparseArray{T,N}(Dict{NTuple{N,Int}, T}(), dims);

julia> Base.size(A::SparseArray) = A.dims

julia> Base.similar(A::SparseArray, ::Type{T}, dims::Dims) where {T} = SparseArray(T, dims)

julia> Base.getindex(A::SparseArray{T,N}, I::Vararg{Int,N}) where {T,N} = get(A.data, I, zero(T))

julia> Base.setindex!(A::SparseArray{T,N}, v, I::Vararg{Int,N}) where {T,N} = (A.data[I] = v)
\end{minted}



请注意,这是个 \texttt{IndexCartesian} 数组,因此我们必须在数组的维度上手动定义 \hyperlink{13720608614876840481}{\texttt{getindex}} 和 \hyperlink{1309244355901386657}{\texttt{setindex!}}。与 \texttt{SquaresVector} 不同,我们可以定义 \hyperlink{1309244355901386657}{\texttt{setindex!}},这样便能更改数组:




\begin{minted}{jlcon}
julia> A = SparseArray(Float64, 3, 3)
3×3 SparseArray{Float64,2}:
 0.0  0.0  0.0
 0.0  0.0  0.0
 0.0  0.0  0.0

julia> fill!(A, 2)
3×3 SparseArray{Float64,2}:
 2.0  2.0  2.0
 2.0  2.0  2.0
 2.0  2.0  2.0

julia> A[:] = 1:length(A); A
3×3 SparseArray{Float64,2}:
 1.0  4.0  7.0
 2.0  5.0  8.0
 3.0  6.0  9.0
\end{minted}



索引 \texttt{AbstractArray} 的结果本身可以是数组(例如,在使用 \texttt{AbstractRange} 时)。\texttt{AbstractArray} 回退方法使用 \hyperlink{15525808546723795098}{\texttt{similar}} 来分配具有适当大小和元素类型的 \texttt{Array},该数组使用上述的基本索引方法填充。但是,在实现数组封装器时,你通常希望也封装结果:




\begin{minted}{jlcon}
julia> A[1:2,:]
2×3 SparseArray{Float64,2}:
 1.0  4.0  7.0
 2.0  5.0  8.0
\end{minted}



在此例中,创建合适的封装数组通过定义 \texttt{Base.similar\{T\}(A::SparseArray, ::Type\{T\}, dims::Dims)} 来实现。(请注意,虽然 \texttt{similar} 支持 1 参数和 2 参数形式,但在大多数情况下,你只需要专门定义 3 参数形式。)为此,\texttt{SparseArray} 是可变的(支持 \texttt{setindex!})便很重要。为 \texttt{SparseArray} 定义 \texttt{similar}、\texttt{getindex} 和 \texttt{setindex!} 也使得该数组能够 \hyperlink{15665284441316555522}{\texttt{copy}} 。




\begin{minted}{jlcon}
julia> copy(A)
3×3 SparseArray{Float64,2}:
 1.0  4.0  7.0
 2.0  5.0  8.0
 3.0  6.0  9.0
\end{minted}



除了上面的所有可迭代和可索引方法之外,这些类型还能相互交互,并使用在 Julia Base 中为 \texttt{AbstractArray} 定义的大多数方法:




\begin{minted}{jlcon}
julia> A[SquaresVector(3)]
3-element SparseArray{Float64,1}:
 1.0
 4.0
 9.0

julia> sum(A)
45.0
\end{minted}



如果要定义允许非传统索引(索引以 1 之外的数字开始)的数组类型,你应该专门指定 \hyperlink{7074821531920287868}{\texttt{axes}}。你也应该专门指定 \hyperlink{15525808546723795098}{\texttt{similar}},以便 \texttt{dims} 参数(通常是大小为 \texttt{Dims} 的元组)可以接收 \texttt{AbstractUnitRange} 对象,它也许是你自己设计的 range 类型 \texttt{Ind}。有关更多信息,请参阅\hyperlink{1238988360302116626}{使用自定义索引的数组}。



\hypertarget{2800090857858949975}{}


\section{Strided 数组}




\begin{table}[h]

\begin{tabulary}{\linewidth}{|L|L|L|}
\hline
需要实现的方法 &  & 简短描述 \\
\hline
\texttt{strides(A)} &  & 返回每个维度中相邻元素之间的内存距离(以内存元素数量的形式)组成的元组。如果 \texttt{A} 是 \texttt{AbstractArray\{T,0\}},这应该返回空元组。 \\
\hline
\texttt{Base.unsafe\_convert(::Type\{Ptr\{T\}\}, A)} &  & 返回数组的本地内存地址。 \\
\hline
\textbf{可选方法} & \textbf{默认定义} & \textbf{简短描述} \\
\hline
\texttt{stride(A, i::Int)} & \texttt{strides(A)[i]} & 返回维度 i(译注:原文为 k)上相邻元素之间的内存距离(以内存元素数量的形式)。 \\
\hline
\end{tabulary}

\end{table}



Strided 数组是 \texttt{AbstractArray} 的子类型,其条目以固定步长储存在内存中。如果数组的元素类型与 BLAS 兼容,则 strided 数组可以利用 BLAS 和 LAPACK 例程来实现更高效的线性代数例程。用户定义的 strided 数组的典型示例是把标准 \texttt{Array} 用附加结构进行封装的数组。



警告:如果底层存储实际上不是 strided,则不要实现这些方法,因为这可能导致错误的结果或段错误。



下面是一些示例,用来演示哪些数组类型是 strided 数组,哪些不是:




\begin{minted}{julia}
1:5   # not strided (there is no storage associated with this array.)
Vector(1:5)  # is strided with strides (1,)
A = [1 5; 2 6; 3 7; 4 8]  # is strided with strides (1,4)
V = view(A, 1:2, :)   # is strided with strides (1,4)
V = view(A, 1:2:3, 1:2)   # is strided with strides (2,4)
V = view(A, [1,2,4], :)   # is not strided, as the spacing between rows is not fixed.
\end{minted}



\hypertarget{8927705294232715192}{}


\section{自定义广播}




\begin{table}[h]

\begin{tabulary}{\linewidth}{|L|L|}
\hline
需要实现的方法 & 简短描述 \\
\hline
\texttt{Base.BroadcastStyle(::Type\{SrcType\}) = SrcStyle()} & \texttt{SrcType} 的广播行为 \\
\hline
\texttt{Base.similar(bc::Broadcasted\{DestStyle\}, ::Type\{ElType\})} & 输出容器的分配 \\
\hline
\textbf{可选方法} &  \\
\hline
\texttt{Base.BroadcastStyle(::Style1, ::Style2) = Style12()} & 混合广播风格的优先级规则 \\
\hline
\texttt{Base.axes(x)} & 用于广播的 \texttt{x} 的索引的声明(默认为 \hyperlink{7074821531920287868}{\texttt{axes(x)}}) \\
\hline
\texttt{Base.broadcastable(x)} & 将 \texttt{x} 转换为一个具有 \texttt{axes} 且支持索引的对象 \\
\hline
\textbf{绕过默认机制} &  \\
\hline
\texttt{Base.copy(bc::Broadcasted\{DestStyle\})} & \texttt{broadcast} 的自定义实现 \\
\hline
\texttt{Base.copyto!(dest, bc::Broadcasted\{DestStyle\})} & 专门针对 \texttt{DestStyle} 的自定义 \texttt{broadcast!} 实现 \\
\hline
\texttt{Base.copyto!(dest::DestType, bc::Broadcasted\{Nothing\})} & 专门针对 \texttt{DestStyle} 的自定义 \texttt{broadcast!} 实现 \\
\hline
\texttt{Base.Broadcast.broadcasted(f, args...)} & 覆盖融合表达式中的默认惰性行为 \\
\hline
\texttt{Base.Broadcast.instantiate(bc::Broadcasted\{DestStyle\})} & 覆盖惰性广播的 axes 的计算 \\
\hline
\end{tabulary}

\end{table}



\hyperlink{10888979137852348176}{广播}可由 \texttt{broadcast} 或 \texttt{broadcast!} 的显式调用、或者像 \texttt{A .+ b} 或 \texttt{f.(x, y)} 这样的「点」操作隐式触发。任何具有 \hyperlink{7074821531920287868}{\texttt{axes}} 且支持索引的对象都可作为参数参与广播,默认情况下,广播结果储存在 \texttt{Array} 中。这个基本框架可通过三个主要方式扩展:



\begin{itemize}
\item 确保所有参数都支持广播


\item 为给定参数集选择合适的输出数组


\item 为给定参数集选择高效的实现

\end{itemize}


不是所有类型都支持 \texttt{axes} 和索引,但许多类型便于支持广播。\hyperlink{3229213625072672556}{\texttt{Base.broadcastable}} 函数会在每个广播参数上调用,它能返回与广播参数不同的支持 \texttt{axes} 和索引的对象。默认情况下,对于所有 \texttt{AbstractArray} 和 \texttt{Number} 来说这是 identity 函数——因为它们已经支持 \texttt{axes} 和索引了。少数其它类型(包括但不限于类型本身、函数、像 \hyperlink{14596725676261444434}{\texttt{missing}} 和 \hyperlink{9331422207248206047}{\texttt{nothing}} 这样的特殊单态类型以及日期)为了能被广播,\texttt{Base.broadcastable} 会返回封装在 \texttt{Ref} 的参数来充当 0 维「标量」。自定义类型可以类似地指定 \texttt{Base.broadcastable} 来定义其形状,但是它们应当遵循 \texttt{collect(Base.broadcastable(x)) == collect(x)} 的约定。一个值得注意的例外是 \texttt{AbstractString};字符串是个特例,为了能被广播其表现为标量,尽管它们是其字符的可迭代集合(详见 \href{@id man-strings}{字符串})。



接下来的两个步骤(选择输出数组和实现)依赖于如何确定给定参数集的唯一解。广播必须接受其参数的所有不同类型,并把它们折叠到一个输出数组和实现。广播称此唯一解为「风格」。每个可广播对象都有自己的首选风格,并使用类似于类型提升的系统将这些风格组合成一个唯一解——「目标风格」。



\hypertarget{5448969838863032993}{}


\subsection{广播风格}



抽象类型 \texttt{Base.BroadcastStyle} 派生了所有的广播风格。其在用作函数时有两种可能的形式,分别为一元形式(单参数)和二元形式。使用一元形式表明你打算实现特定的广播行为和/或输出类型,并且不希望依赖于默认的回退 \hyperlink{5203521679854231580}{\texttt{Broadcast.DefaultArrayStyle}}。



为了覆盖这些默认值,你可以为对象自定义 \texttt{BroadcastStyle}:




\begin{minted}{julia}
struct MyStyle <: Broadcast.BroadcastStyle end
Base.BroadcastStyle(::Type{<:MyType}) = MyStyle()
\end{minted}



在某些情况下,无需定义 \texttt{MyStyle} 也许很方便,在这些情况下,你可以利用一个通用的广播封装器:



\begin{itemize}
\item \texttt{Base.BroadcastStyle(::Type\{<:MyType\}) = Broadcast.Style\{MyType\}()} 可用于任意类型。


\item 如果 \texttt{MyType} 是一个 \texttt{AbstractArray},首选是 \texttt{Base.BroadcastStyle(::Type\{<:MyType\}) = Broadcast.ArrayStyle\{MyType\}()}。


\item 对于只支持某个具体维度的 \texttt{AbstractArrays},请创建 \texttt{Broadcast.AbstractArrayStyle\{N\}} 的子类型(请参阅下文)。

\end{itemize}


当你的广播操作涉及多个参数,各个广播风格将合并,来确定唯一一个 \texttt{DestStyle} 以控制输出容器的类型。有关更多详细信息,请参阅\hyperlink{17567878480973592299}{下文}。



\hypertarget{4710253435053989143}{}


\subsection{选择合适的输出数组}



每个广播操作都会计算广播风格以便支持派发和专门化。结果数组的实际分配由 \texttt{similar} 处理,其使用 Broadcasted 对象作为其第一个参数。




\begin{minted}{julia}
Base.similar(bc::Broadcasted{DestStyle}, ::Type{ElType})
\end{minted}



回退定义是




\begin{minted}{julia}
similar(bc::Broadcasted{DefaultArrayStyle{N}}, ::Type{ElType}) where {N,ElType} =
    similar(Array{ElType}, axes(bc))
\end{minted}



但是,如果需要,你可以专门化任何或所有这些参数。最后的参数 \texttt{bc} 是(还可能是融合的)广播操作的惰性表示,即 \texttt{Broadcasted} 对象。出于这些目的,该封装器中最重要的字段是 \texttt{f} 和 \texttt{args},分别描述函数和参数列表。请注意,参数列表可以——并且经常——包含其它嵌套的 \texttt{Broadcasted} 封装器。



举个完整的例子,假设你创建了类型 \texttt{ArrayAndChar},该类型存储一个数组和单个字符:




\begin{minted}{julia}
struct ArrayAndChar{T,N} <: AbstractArray{T,N}
    data::Array{T,N}
    char::Char
end
Base.size(A::ArrayAndChar) = size(A.data)
Base.getindex(A::ArrayAndChar{T,N}, inds::Vararg{Int,N}) where {T,N} = A.data[inds...]
Base.setindex!(A::ArrayAndChar{T,N}, val, inds::Vararg{Int,N}) where {T,N} = A.data[inds...] = val
Base.showarg(io::IO, A::ArrayAndChar, toplevel) = print(io, typeof(A), " with char '", A.char, "'")
\end{minted}



你可能想要保留「元数据」\texttt{char}。为此,我们首先定义




\begin{minted}{julia}
Base.BroadcastStyle(::Type{<:ArrayAndChar}) = Broadcast.ArrayStyle{ArrayAndChar}()
\end{minted}



这意味着我们还必须定义相应的 \texttt{similar} 方法:




\begin{minted}{julia}
function Base.similar(bc::Broadcast.Broadcasted{Broadcast.ArrayStyle{ArrayAndChar}}, ::Type{ElType}) where ElType
    # Scan the inputs for the ArrayAndChar:
    A = find_aac(bc)
    # Use the char field of A to create the output
    ArrayAndChar(similar(Array{ElType}, axes(bc)), A.char)
end

"`A = find_aac(As)` returns the first ArrayAndChar among the arguments."
find_aac(bc::Base.Broadcast.Broadcasted) = find_aac(bc.args)
find_aac(args::Tuple) = find_aac(find_aac(args[1]), Base.tail(args))
find_aac(x) = x
find_aac(::Tuple{}) = nothing
find_aac(a::ArrayAndChar, rest) = a
find_aac(::Any, rest) = find_aac(rest)
\end{minted}



在这些定义中,可以得到以下行为:




\begin{minted}{jlcon}
julia> a = ArrayAndChar([1 2; 3 4], 'x')
2×2 ArrayAndChar{Int64,2} with char 'x':
 1  2
 3  4

julia> a .+ 1
2×2 ArrayAndChar{Int64,2} with char 'x':
 2  3
 4  5

julia> a .+ [5,10]
2×2 ArrayAndChar{Int64,2} with char 'x':
  6   7
 13  14
\end{minted}



\hypertarget{5201970122303370123}{}


\subsection{使用自定义实现扩展广播}



一般来说,广播操作由一个惰性 \texttt{Broadcasted} 容器表示,该容器保存要应用的函数及其参数。这些参数可能本身是嵌套得更深的 \texttt{Broadcasted} 容器,并一起形成了一个待求值的大型表达式树。嵌套的 \texttt{Broadcasted} 容器树可由隐式的点语法直接构造;例如,\texttt{5 .+ 2.*x} 由 \texttt{Broadcasted(+, 5, Broadcasted(*, 2, x))} 暂时表示。这对于用户是不可见的,因为它是通过调用 \texttt{copy} 立即实现的,但是此容器为自定义类型的作者提供了广播可扩展性的基础。然后,内置的广播机制将根据参数确定结果的类型和大小,为它分配内存,并最终通过默认的 \texttt{copyto!(::AbstractArray, ::Broadcasted)} 方法将 \texttt{Broadcasted} 对象复制到其中。内置的回退 \texttt{broadcast} 和 \texttt{broadcast!} 方法类似地构造操作的暂时 \texttt{Broadcasted} 表示,因此它们共享相同的代码路径。这便允许自定义的数组实现通过提供它们自己的专门化 \texttt{copyto!} 来定义和优化广播。这再次由计算后的广播风格确定。此广播风格在广播操作中非常重要,以至于它被存储为 \texttt{Broadcasted} 类型的第一个类型参数,且允许派发和专门化。



对于某些类型,跨越层层嵌套的广播的「融合」操作无法实现,或者无法更高效地逐步完成。在这种情况下,你可能需要或者想要求值 \texttt{x .* (x .+ 1)},就好像该式已被编写成 \texttt{broadcast(*, x, broadcast(+, x, 1))},其中内部广播操作会在处理外部广播操作前进行求值。这种直接的操作以有点间接的方式得到直接支持;Julia 不会直接构造 \texttt{Broadcasted} 对象,而会将 待融合的表达式 \texttt{x .* (x .+ 1)} 降低为 \texttt{Broadcast.broadcasted(*, x, Broadcast.broadcasted(+, x, 1))}。现在,默认情况下,\texttt{broadcasted} 只会调用 \texttt{Broadcasted} 构造函数来创建待融合表达式树的惰性表示,但是你可以选择为函数和参数的特定组合覆盖它。



举个例子,内置的 \texttt{AbstractRange} 对象使用此机制优化广播表达式的片段,这些表达式片段可以只根据 start、step 和 length(或 stop)直接进行求值,而无需计算每个元素。与所有其它机制一样,\texttt{broadcasted} 也会计算并暴露其参数的组合广播风格,所以你可以为广播风格、函数和参数的任意组合专门化 \texttt{broadcasted(::DestStyle, f, args...)},而不是专门化 \texttt{broadcasted(f, args...)}。



例如,以下定义支持 range 的负运算:




\begin{minted}{julia}
broadcasted(::DefaultArrayStyle{1}, ::typeof(-), r::OrdinalRange) = range(-first(r), step=-step(r), length=length(r))
\end{minted}



\hypertarget{13437862742050583795}{}


\subsection{扩展 in-place 广播}



In-place 广播可通过定义合适的 \texttt{copyto!(dest, bc::Broadcasted)} 方法来支持。由于你可能想要专门化 \texttt{dest} 或 \texttt{bc} 的特定子类型,为了避免包之间的歧义,我们建议采用以下约定。



如果你想要专门化特定的广播风格 \texttt{DestStyle},请为其定义一个方法




\begin{minted}{julia}
copyto!(dest, bc::Broadcasted{DestStyle})
\end{minted}



你可选择使用此形式,如果使用,你还可以专门化 \texttt{dest} 的类型。



如果你想专门化目标类型 \texttt{DestType} 而不专门化 \texttt{DestStyle},那么你应该定义一个带有以下签名的方法:




\begin{minted}{julia}
copyto!(dest::DestType, bc::Broadcasted{Nothing})
\end{minted}



这利用了 \texttt{copyto!} 的回退实现,它将该封装器转换为一个 \texttt{Broadcasted\{Nothing\}} 对象。因此,专门化 \texttt{DestType} 的方法优先级低于专门化 \texttt{DestStyle} 的方法。



同样,你可以使用 \texttt{copy(::Broadcasted)} 方法完全覆盖 out-of-place 广播。



\hypertarget{2171470859232296256}{}


\subsubsection{使用 \texttt{Broadcasted} 对象}



当然,为了实现这样的 \texttt{copy} 或 \texttt{copyto!} 方法,你必须使用 \texttt{Broadcasted} 封装器来计算每个元素。这主要有两种方式:



\begin{itemize}
\item \texttt{Broadcast.flatten} 将可能的嵌套操作重新计算为单个函数并平铺参数列表。你自己负责实现广播形状规则,但这在有限的情况下可能会有所帮助。


\item 迭代 \texttt{axes(::Broadcasted)} 的 \texttt{CartesianIndices} 并使用所生成的 \texttt{CartesianIndex} 对象的索引来计算结果。

\end{itemize}


\hypertarget{13308248870533973226}{}


\subsection{编写二元广播规则}



广播风格的优先级规则由二元 \texttt{BroadcastStyle} 调用定义:




\begin{minted}{julia}
Base.BroadcastStyle(::Style1, ::Style2) = Style12()
\end{minted}



其中,\texttt{Style12} 是你要为输出所选择的 \texttt{BroadcastStyle},所涉及的参数具有 \texttt{Style1} 及 \texttt{Style2}。例如,




\begin{minted}{julia}
Base.BroadcastStyle(::Broadcast.Style{Tuple}, ::Broadcast.AbstractArrayStyle{0}) = Broadcast.Style{Tuple}()
\end{minted}



表示 \texttt{Tuple}「胜过」零维数组(输出容器将是元组)。值得注意的是,你不需要(也不应该)为此调用的两个参数顺序下定义;无论用户提供的以何种顺序提供参数,定义一个就够了。



对于 \texttt{AbstractArray} 类型,定义 \texttt{BroadcastStyle} 将取代回退选择 \hyperlink{5203521679854231580}{\texttt{Broadcast.DefaultArrayStyle}}。\texttt{DefaultArrayStyle} 及其抽象超类型 \texttt{AbstractArrayStyle} 将维度存储为类型参数,以支持具有固定维度需求的特定数组类型。



由于以下方法,\texttt{DefaultArrayStyle}「输给」任何其它已定义的 \texttt{AbstractArrayStyle}:




\begin{minted}{julia}
BroadcastStyle(a::AbstractArrayStyle{Any}, ::DefaultArrayStyle) = a
BroadcastStyle(a::AbstractArrayStyle{N}, ::DefaultArrayStyle{N}) where N = a
BroadcastStyle(a::AbstractArrayStyle{M}, ::DefaultArrayStyle{N}) where {M,N} =
    typeof(a)(_max(Val(M),Val(N)))
\end{minted}



除非你想要为两个或多个非 \texttt{DefaultArrayStyle} 的类型建立优先级,否则不需要编写二元 \texttt{BroadcastStyle} 规则。



如果你的数组类型确实有固定的维度需求,那么你应该定义一个 \texttt{AbstractArrayStyle} 的子类型。例如,稀疏数组的代码中有以下定义:




\begin{minted}{julia}
struct SparseVecStyle <: Broadcast.AbstractArrayStyle{1} end
struct SparseMatStyle <: Broadcast.AbstractArrayStyle{2} end
Base.BroadcastStyle(::Type{<:SparseVector}) = SparseVecStyle()
Base.BroadcastStyle(::Type{<:SparseMatrixCSC}) = SparseMatStyle()
\end{minted}



每当你定义一个 \texttt{AbstractArrayStyle} 的子类型,你还需要定义用于组合维度的规则,这通过为你的广播风格创建带有一个 \texttt{Val(N)} 参数的构造函数。例如:




\begin{minted}{julia}
SparseVecStyle(::Val{0}) = SparseVecStyle()
SparseVecStyle(::Val{1}) = SparseVecStyle()
SparseVecStyle(::Val{2}) = SparseMatStyle()
SparseVecStyle(::Val{N}) where N = Broadcast.DefaultArrayStyle{N}()
\end{minted}



这些规则表明 \texttt{SparseVecStyle} 与 0 维或 1 维数组的组合会产生另一个 \texttt{SparseVecStyle},与 2 维数组的组合会产生 \texttt{SparseMatStyle},而与维度更高的数组则回退到任意维密集矩阵的框架中。这些规则允许广播为产生一维或二维输出的操作保持其稀疏表示,但为任何其它维度生成 \texttt{Array}。



\hypertarget{10438697863683890874}{}


\chapter{模块}



Julia 中的模块(module)是一些互相隔离的可变工作空间,也就是说它们会引入新的全局作用域。它们在语法上以 \texttt{module Name ... end} 界定。模块允许你创建顶层定义(也称为全局变量),而无需担心命名冲突。在模块中,利用导入(importing),你可以控制其它模块中的哪些名称是可见的;利用导出(exporting),你可以控制你自己的模块中的哪些名称是公开的。



下面的示例演示了模块的主要功能。它不是为了运行,只是为了方便说明:




\begin{minted}{julia}
module MyModule
using Lib

using BigLib: thing1, thing2

import Base.show

export MyType, foo

struct MyType
    x
end

bar(x) = 2x
foo(a::MyType) = bar(a.x) + 1

show(io::IO, a::MyType) = print(io, "MyType $(a.x)")
end
\end{minted}



注意,模块中的代码样式不需要缩进,否则的话,会导致整个文件缩进。



上面的模块定义了一个 \texttt{MyType} 类型,以及两个函数,其中,函数 \texttt{foo} 和类型 \texttt{MyType} 被导出了,因而可以被导入到其它模块,而函数 \texttt{bar} 是模块 \texttt{MyModule} 的私有函数。



\texttt{using Lib} 意味着一个名称为 \texttt{Lib} 的模块会在需要的时候用于解释变量名。当一个全局变量在当前模块中没有定义时,系统就会从 \texttt{Lib} 中导出的变量中搜索该变量,如果找到了的话,就导入进来。也就是说,当前模块中,所有使用该全局变量的地方都会解释为 \texttt{Lib} 中对应的变量。



代码 \texttt{using BigLib: thing1, thing2} 显式地将标识符 \texttt{thing1} 和 \texttt{thing2} 从模块 \texttt{BigLib} 中引入到当前作用域。如果这两个变量是函数的话,则\textbf{不允许}给它们增加新的方法,毕竟代码里写的是 {\textquotedbl}using{\textquotedbl}(使用)它们,而不是扩展它们。



\hyperlink{16252475688663093021}{\texttt{import}} 关键字所支持的语法与 \hyperlink{169458112978175560}{\texttt{using}} 一致。 它并不会像 \texttt{using} 那样将模块添加到搜索空间中。 与 \texttt{using} 不同,\texttt{import} 引入的函数 \textbf{可以} 为其增加新的方法。



前面的 \texttt{MyModule} 模块中,我们希望给 \hyperlink{14071376285304310153}{\texttt{show}} 函数增加一个方法,需要写成 \texttt{import Base.show}。如果用 \texttt{using} 的话,就不能扩展 \texttt{show} 函数。通过 \texttt{using} 导入才可见的名字是不能被扩展的。



一旦一个变量通过 \texttt{using} 或 \texttt{import} 引入,当前模块就不能创建同名的变量了。而且导入的变量是只读的,给全局变量赋值只能影响到由当前模块拥有的变量,否则会报错。



\hypertarget{14242284106617888119}{}


\section{模块用法摘要}



要导入一个模块,可以用 \texttt{using} 或 \texttt{import} 关键字。为了更好地理解它们的区别,请参考下面的例子:




\begin{minted}{julia}
module MyModule

export x, y

x() = "x"
y() = "y"
p() = "p"

end
\end{minted}



这个模块用关键字 \texttt{export} 导出了 \texttt{x} 和 \texttt{y} 函数,此外还有一个没有被导出的函数 \texttt{p}。想要将该模块及其内部的函数导入当前模块有以下方法:




\begin{table}[h]

\begin{tabulary}{\linewidth}{|L|L|L|}
\hline
导入代码 & 当前作用域导入了哪些变量? & 可增加新方法的名字 \\
\hline
\texttt{using MyModule} & All \texttt{export}ed names (\texttt{x} and \texttt{y}), \texttt{MyModule.x}, \texttt{MyModule.y} and \texttt{MyModule.p} & \texttt{MyModule.x}, \texttt{MyModule.y} and \texttt{MyModule.p} \\
\hline
\texttt{using MyModule: x, p} & \texttt{x} and \texttt{p} &  \\
\hline
\texttt{import MyModule} & \texttt{MyModule.x}、\texttt{MyModule.y} 和 \texttt{MyModule.p} & \texttt{MyModule.x}、\texttt{MyModule.y} 和 \texttt{MyModule.p} \\
\hline
\texttt{import MyModule.x, MyModule.p} & \texttt{x} 和 \texttt{p} & \texttt{x} 和 \texttt{p} \\
\hline
\texttt{import MyModule: x, p} & \texttt{x} 和 \texttt{p} & \texttt{x} 和 \texttt{p} \\
\hline
\end{tabulary}

\end{table}



\hypertarget{14341239136076041524}{}


\subsection{模块和文件}



模块与文件和文件名无关;模块只与模块表达式有关。一个模块可以有多个文件,一个文件也可以有多个模块。




\begin{minted}{julia}
module Foo

include("file1.jl")
include("file2.jl")

end
\end{minted}



在不同的模块中引入同一段代码,可以提供一种类似 mixin 的行为。我们可以利用这个特性来观察,在不同的定义下,执行同一段代码会有什么结果。例如,在测试的时候,可以使用某些「安全」的运算符。




\begin{minted}{julia}
module Normal
include("mycode.jl")
end

module Testing
include("safe_operators.jl")
include("mycode.jl")
end
\end{minted}



\hypertarget{2129272965593313585}{}


\subsection{标准模块}



有三个重要的标准模块:



\begin{itemize}
\item \hyperlink{14876339894285762624}{\texttt{Core}} 包含了语言“内置”的所有功能。


\item \hyperlink{464144976511314225}{\texttt{Base}} 包含了绝大多数情况下都会用到的基本功能。


\item \hyperlink{7094459820733568273}{\texttt{Main}} 是顶层模块,当 julia 启动时,也是当前模块。

\end{itemize}


\hypertarget{3653095448809961286}{}


\subsection{默认顶层定义以及裸模块}



除了默认包含 \texttt{using Base} 之外,所有模块都还包含 \hyperlink{7507639810592563424}{\texttt{eval}} 和 \hyperlink{7507443674556842580}{\texttt{include}} 函数。这两个函数用于在对应模块的全局环境中,执行表达式或文件。



如果连这些默认的定义都不需要,那么可以用 \hyperlink{13329108222158426840}{\texttt{baremodule}} 定义裸模块(不过 \texttt{Core} 模块仍然会被引入,否则啥也干不了)。用裸模块表达的标准模块定义如下:




\begin{lstlisting}
baremodule Mod

using Base

eval(x) = Core.eval(Mod, x)
include(p) = Base.include(Mod, p)

...

end
\end{lstlisting}



\hypertarget{15942908766340569567}{}


\subsection{模块的绝对路径和相对路径}



给定语句 \texttt{using Foo},系统在顶层模块的内部表中查找名为 \texttt{Foo} 的包。如果模块不存在,系统会尝试 \texttt{require(:Foo)},这通常会从已安装的包中加载代码。



但是,某些模块包含子模块,这意味着你有时需要访问非顶层模块。有两种方法可以做到这一点。第一种是使用绝对路径,例如 \texttt{using Base.Sort}。第二种是使用相对路径,这样可以更容易地导入当前模块或其任何封闭模块的子模块:




\begin{lstlisting}
module Parent

module Utils
...
end

using .Utils

...
end
\end{lstlisting}



这里的模块 \texttt{Parent} 包含一个子模块 \texttt{Utils},而 \texttt{Parent} 中的代码希望 \texttt{Utils} 的内容可见,这是可以使用 \texttt{using} 加点 \texttt{.} 这种相对路径来实现。添加更多的点会移动到模块层次结构中的更上级别。例如,\texttt{using ..Utils} 会在 \texttt{Parent} 的上级模块中查找 \texttt{Utils} 而不是在 \texttt{Parent} 中查找。



请注意,相对导入符号 \texttt{.} 仅在 \texttt{using} 和 \texttt{import} 语句中有效。



\hypertarget{5019601776205271706}{}


\subsection{命名空间的相关话题}



如果名称是限定的(例如 \texttt{Base.sin}),那么即使它没有被导出,我们也可以访问它。这通常在调试时很有用。若函数名也使用这种限定的方式,就可以为其添加方法。但是,对于函数名仅包含符号的情况,例如一个运算符,\texttt{Base.+},由于会出现语法歧义,所以必须使用 \texttt{Base.:+} 来引用它。如果运算符的字符不止一个,则必须用括号括起来,例如:\texttt{Base.:(==)}。



宏名称在导入和导出语句中用 \texttt{@} 编写,例如:\texttt{import Mod.@mac}。其它模块中的宏可以用 \texttt{Mod.@mac} 或 \texttt{@Mod.mac} 触发。



不允许使用 \texttt{M.x = y} 这种写法给另一个模块中的全局变量赋值;必须在模块内部才能进行全局变量的赋值。



用 \texttt{global x} 声明变量可以仅“保留”名称而不赋值。有些全局变量需要在代码加载后才初始化,这样做可以防止命名冲突。



\hypertarget{10308651053456408379}{}


\subsection{模块初始化和预编译}



因为执行模块中的所有语句通常需要编译大量代码,大型模块可能需要几秒钟才能加载。Julia 会创建模块的预编译缓存以减少这个时间。



当用 \texttt{import} 或 \texttt{using} 加载一个模块时,模块增量预编译文件会自动创建并使用。这会让模块在第一次加载时自动编译。 另外,你也可以手工调用 \hyperlink{15403934372637978246}{\texttt{Base.compilecache(modulename)}},产生的缓存文件会放在 \texttt{DEPOT\_PATH[1]/compiled/} 目录下。 之后,当该模块的任何一个依赖发生变更时,该模块会在 \texttt{using} 或 \texttt{import} 时自动重新编译; 模块的依赖指的是:任何它导入的模块、Julia 自身、include 的文件或由 \hyperlink{13423629850785876688}{\texttt{include\_dependency(path)}} 显式声明的依赖。



对于文件依赖,判断是否有变动的方法是:在 \texttt{include} 或 \texttt{include\_dependency} 的时候检查每个文件的变更时间(\texttt{mtime})是否没变,或等于截断变更时间。截断变更时间是指将变更时间截断到最近的一秒,这是由于在某些操作系统中,用 \texttt{mtime} 无法获取亚秒级的精度。此外,也会考虑到 \texttt{require} 搜索到的文件路径与之前预编译文件中的是否匹配。对于已经加载到当前进程的依赖,即使它们的文件发成了变更,甚至是丢失,Julia 也不会重新编译这些模块,这是为了避免正在运行的系统与预编译缓存之间的不兼容性。



如果你认为预编译自己的模块是\textbf{不}安全的(基于下面所说的各种原因),那么你应该在模块文件中添加 \texttt{\_\_precompile\_\_(false)},一般会将其写在文件的最上面。这就可以触发 \texttt{Base.compilecache} 报错,并且在直接使用 \texttt{using} / \texttt{import} 加载的时候跳过预编译和缓存。这样做同时也可以防止其它开启预编译的模块加载此模块。



在开发模块的时候,你可能需要了解一些与增量编译相关的固有行为。例如,外部状态不会被保留。为了解决这个问题,需要显式分离运行时与编译期的部分。Julia 允许你定义一个 \texttt{\_\_init\_\_()} 函数来执行任何需要在运行时发生的初始化。在编译期(\texttt{--output-*}),此函数将不会被调用。你可以假设在代码的生存周期中,此函数只会被运行一次。当然,如果有必要,你也可以手动调用它,但在默认的情况下,请假定此函数是为了处理与本机状态相关的信息,注意这些信息不需要,更不应该存入预编译镜像。此函数会在模块被导入到当前进程之后被调用,这包括在一个增量编译中导入该模块的时候(\texttt{--output-incremental=yes}),但在完整编译时该函数不会被调用。



特别的,如果你在模块里定义了一个名为 \texttt{\_\_init\_\_()} 的函数,那么 Julia 在加载这个模块之后会在第一次运行时(runtime)立刻调用这个函数(例如,通过 \texttt{import},\texttt{using},或者 \texttt{require} 加载时),也就是说 \texttt{\_\_init\_\_} 只会在模块中所有其它命令都执行完以后被调用一次。因为这个函数将在模块完全载入后被调用,任何子模块或者已经载入的模块都将在当前模块调用 \texttt{\_\_init\_\_} \textbf{之前} 调用自己的 \texttt{\_\_init\_\_} 函数。



\texttt{\_\_init\_\_}的典型用法有二,一是用于调用外部 C 库的运行时初始化函数,二是用于初始化涉及到外部库所返回的指针的全局常量。例如,假设我们正在调用一个 C 库 \texttt{libfoo},它要求我们在运行时调用\texttt{foo\_init()} 这个初始化函数。假设我们还想定义一个全局常量 \texttt{foo\_data\_ptr},它保存 \texttt{libfoo} 所定义的 \texttt{void *foo\_data()} 函数的返回值——必须在运行时(而非编译时)初始化这个常量,因为指针地址不是固定的。可以通过在模块中定义 \texttt{\_\_init\_\_} 函数来完成这个操作。




\begin{minted}{julia}
const foo_data_ptr = Ref{Ptr{Cvoid}}(0)
function __init__()
    ccall((:foo_init, :libfoo), Cvoid, ())
    foo_data_ptr[] = ccall((:foo_data, :libfoo), Ptr{Cvoid}, ())
    nothing
end
\end{minted}



注意,在像 \texttt{\_\_init\_\_} 这样的函数里定义一个全局变量是完全可以的,这是动态语言的优点之一。但是把全局作用域的值定义成常量,可以让编译器能确定该值的类型,并且能让编译器生成更好的优化过的代码。显然,你的模块(Module)中,任何其他依赖于 \texttt{foo\_data\_ptr} 的全局量也必须在 \texttt{\_\_init\_\_} 中被初始化。



涉及大多数不是由 \hyperlink{14245046751182637566}{\texttt{ccall}} 生成的 Julia 对象的常量,不需要放在 \texttt{\_\_init\_\_} 中:它们的定义可以预编译并从缓存的模块映像中加载。这包括复杂的堆分配对象,如数组。 但是,任何返回原始指针值的例程都必须在运行时调用,以便进行预编译(除非将 \hyperlink{10630331440513004826}{\texttt{Ptr}} 对象隐藏在 \hyperlink{12980593021531333073}{\texttt{isbits}} 对象中,否则它们将转换为空指针)。 这包括 Julia 函数 \texttt{cfunction} 和 \hyperlink{8901246211940014300}{\texttt{pointer}} 的返回值。



字典和集合类型,或者通常任何依赖于 \texttt{hash(key)} 方法的类型,都是比较棘手的情况。 通常当键是数字、字符串、符号、范围、\texttt{Expr} 或这些类型的组合(通过数组、元组、集合、映射对等)时,可以安全地预编译它们。但是,对于一些其它的键类型,例如 \texttt{Function} 或 \texttt{DataType}、以及还没有定义散列方法的通用用户定义类型,回退(fallback)的散列(\texttt{hash})方法依赖于对象的内存地址(通过 \texttt{objectid}),因此可能会在每次运行时发生变化。 如果您有这些关键类型中的一种,或者您不确定,为了安全起见,您可以在您的 \texttt{\_\_init\_\_} 函数中初始化这个字典。或者,您可以使用 \hyperlink{14088500196255451490}{\texttt{IdDict}} 字典类型,它是由预编译专门处理的,因此在编译时初始化是安全的。



当使用预编译时,我们必须要清楚地区分代码的编译阶段和运行阶段。在此模式下,我们会更清楚发现 Julia 的编译器可以执行任何 Julia 代码,而不是一个用于生成编译后代码的独立的解释器。



其它已知的潜在失败场景包括:



\begin{itemize}
\item[1. ] 全局计数器,例如:为了试图唯一的标识对象。考虑以下代码片段:


\begin{minted}{julia}
mutable struct UniquedById
    myid::Int
    let counter = 0
        UniquedById() = new(counter += 1)
    end
end
\end{minted}

尽管这段代码的目标是给每个实例赋一个唯一的 ID,但计数器的值会在代码编译结束时被记录。任何对此增量编译模块的后续使用,计数器都将从同一个值开始计数。

注意 \texttt{objectid} (工作原理是取内存指针的 hash)也有类似的问题,请查阅下面关于 \texttt{Dict} 的用法。

一种解决方案是用宏捕捉 \hyperlink{8796901235206560169}{\texttt{@\_\_MODULE\_\_}},并将它与目前的 \texttt{counter} 值一起保存。然而,更好的方案是对代码进行重新设计,不要依赖这种全局状态变量。


\item[2. ] 像 \texttt{Dict} 和 \texttt{Set} 这种关联集合需要在 \texttt{\_\_init\_\_} 中 re-hash。Julia 在未来很可能会提供一个机制来注册初始化函数。


\item[3. ] 依赖编译期的副作用会在加载时蔓延。例子包括:更改其它 Julia 模块里的数组或变量,操作文件或设备的句柄,保存指向其它系统资源(包括内存)的指针。


\item[4. ] 无意中从其它模块中“拷贝”了全局状态:通过直接引用的方式而不是通过查找的方式。例如,在全局作用域下:


\begin{minted}{julia}
#mystdout = Base.stdout #= will not work correctly, since this will copy Base.stdout into this module =#
# instead use accessor functions:
getstdout() = Base.stdout #= best option =#
# or move the assignment into the runtime:
__init__() = global mystdout = Base.stdout #= also works =#
\end{minted}

\end{itemize}


此处为预编译中的操作附加了若干限制,以帮助用户避免其他误操作:



\begin{itemize}
\item[1. ] 调用 \hyperlink{7507639810592563424}{\texttt{eval}} 来在另一个模块中引发副作用。当增量预编译被标记时,该操作同时会导致抛出一个警告。


\item[2. ] 当 \texttt{\_\_init\_\_()} 已经开始执行后,在局部作用域中声明 \texttt{global const}(见 issue \#12010,计划为此情况添加一个错误提示)


\item[3. ] 在增量预编译时替换模块是一个运行时错误。

\end{itemize}


一些其他需要注意的点:



\begin{itemize}
\item[1. ] 在源代码文件本身被修改之后,不会执行代码重载或缓存失效化处理(包括由 \texttt{Pkg.update} 执行的修改,此外在 \texttt{Pkg.rm} 执行后也没有清理操作)


\item[2. ] 变形数组的内存共享特性会被预编译忽略(每个数组样貌都会获得一个拷贝)


\item[3. ] 文件系统在编译期间和运行期间被假设为不变的,比如使用 \hyperlink{1518763743618824993}{\texttt{@\_\_FILE\_\_}}/\texttt{source\_path()} 在运行期间寻找资源、或使用 BinDeps 宏 \texttt{@checked\_lib}。有时这是不可避免的。但是可能的话,在编译期将资源复制到模块里面是个好做法,这样在运行期间,就不需要去寻找它们了。


\item[4. ] \texttt{WeakRef} 对象和完成器目前在序列化器中无法被恰当地处理(在接下来的发行版中将修复)。


\item[5. ] 通常,最好避免去捕捉内部元数据对象的引用,如 \texttt{Method}、\texttt{MethodInstance}、\texttt{TypeMapLevel}、\texttt{TypeMapEntry} 及这些对象的字段,因为这会迷惑序列化器,且可能会引发你不想要的结果。此操作不足以成为一个错误,但你需做好准备:系统会尝试拷贝一部分,然后创建其余部分的单个独立实例。

\end{itemize}


在开发模块时,关闭增量预编译可能会有所帮助。命令行标记 \texttt{--compiled-modules=\{yes|no\}} 可以让你切换预编译的开启和关闭。当 Julia 附加 \texttt{--compiled-modules=no} 启动,在载入模块和模块依赖时,编译缓存中的序列化模块会被忽略。\texttt{Base.compilecache} 仍可以被手动调用。此命令行标记的状态会被传递给 \texttt{Pkg.build},禁止其在安装、更新、显式构建包时触发自动预编译。



\hypertarget{11221734891767592985}{}


\chapter{文档}



自Julia 0.4 开始,Julia 允许开发者和用户,使用其内置的文档系统更加便捷地为函数、类型以及其他对象编写文档。



The basic syntax is simple: any string appearing at the toplevel right before an object (function, macro, type or instance) will be interpreted as documenting it (these are called \emph{docstrings}). Note that no blank lines or comments may intervene between a docstring and the documented object. Here is a basic example:




\begin{minted}{julia}
"Tell whether there are too foo items in the array."
foo(xs::Array) = ...
\end{minted}



文档会被翻译成 \href{https://en.wikipedia.org/wiki/Markdown}{Markdown},所以你可以 使用缩进和代码块来分隔代码示例和文本。从技术上来说,任何对象 都可以作为 metadata 与任何其他对象关联;Markdown 是默认的,但是可以创建 其它字符串宏并传递给 \texttt{@doc} 宏来使用其他格式。



\begin{quote}
\textbf{Note}

Markdown support is implemented in the \texttt{Markdown} standard library and for a full list of supported syntax see the \hyperlink{4003493111480691691}{documentation}.

\end{quote}


这里是一个更加复杂的例子,但仍然使用 Markdown:




\begin{minted}{julia}
"""
    bar(x[, y])

Compute the Bar index between `x` and `y`. If `y` is missing, compute
the Bar index between all pairs of columns of `x`.

# Examples
```julia-repl
julia> bar([1, 2], [1, 2])
1
```
"""
function bar(x, y) ...
\end{minted}



如上例所示,我们推荐在写文档时遵守一些简单约定:



\begin{itemize}
\item[1.  ] 始终在文档顶部显示函数的签名并带有四空格缩进,以便能够显示成 Julia 代码。

这和在 Julia 代码中的签名是一样的(比如 \texttt{mean(x::AbstractArray)}),或是简化版。可选参数应该尽可能与默认值一同显示(例如 \texttt{f(x, y=1)}),这与实际的 Julia 语法一致。没有默认值的可选参数应该放在括号中(例如 \texttt{f(x[, y])} 和 \texttt{f(x[, y[, z]])})。可选的解决方法是使用多行:一个没有可选参数,其他的拥有可选参数(或者多个可选参数)。这个解决方案也可以用作给某个函数的多个方法来写文档。当一个函数接收到多个关键字参数,只在签名中包含占位符 \texttt{<keyword arguments>}(例如 \texttt{f(x; <keyword arguments>)}),并在 \texttt{\# Arguments} 章节给出完整列表(参照下列第 4 点)。


\item[2.  ] 在简化的签名块后请包含一个描述函数能做什么或者对象代表什么的单行句。如果需要的话,在一个空行之后,在第二段提供更详细的信息。

撰写函数的文档时,单行语句应使用祈使结构(比如「Do this」、「Return that」)而非第三人称(不要写「Returns the length...」)。并且应以句号结尾。如果函数的意义不能简单地总结,更好的方法是分成分开的组合句(虽然这不应被看做是对于每种情况下的绝对要求)。


\item[3.  ] 不要自我重复。

因为签名给出了函数名,所以没有必要用「The function \texttt{bar}...」开始文档:直接说要点。类似地,如果签名指定了参数的类型,在描述中提到这些是多余的。


\item[4.  ] 只在确实必要时提供参数列表。

对于简单函数,直接在函数目的的描述中提到参数的作用常常更加清楚。参数列表只会重复再其他地方提供过的信息。但是,对于拥有多个参数的(特别是含有关键字参数的)复杂函数来说,提供一个参数列表是个好主意。在这种情况下,请在函数的一般描述之后、标题 \texttt{\# Arguments} 之下插入参数列表,并在每个参数前加个着重号 \texttt{-}。参数列表应该提到参数的类型和默认值(如果有):


\begin{minted}{julia}
"""
...
# Arguments
- `n::Integer`: the number of elements to compute.
- `dim::Integer=1`: the dimensions along which to perform the computation.
...
"""
\end{minted}


\item[5.  ] 给相关函数提供提示。

有时会存在具有功能相联系的函数。为了更易于发现相关函数,请在段落 \texttt{See also:} 中为其提供一个小列表。


\begin{lstlisting}
See also: [`bar!`](@ref), [`baz`](@ref), [`baaz`](@ref)
\end{lstlisting}


\item[6.  ] 请在 \texttt{\# Examples} 中包含一些代码例子。

例子应尽可能按照 \emph{doctest} 来写。\emph{doctest} 是一个栅栏分隔开的代码块(请参阅\href{@ref}{代码块}),其以 \texttt{```jldoctest} 开头并包含任意数量的提示符 \texttt{julia>} 以及用来模拟 Julia REPL 的输入和预期输出。

\begin{quote}
\textbf{Note}

Doctest 由 \href{https://github.com/JuliaDocs/Documenter.jl}{\texttt{Documenter.jl}} 支持。有关更详细的文档,请参阅 Documenter 的\href{https://juliadocs.github.io/Documenter.jl/}{手册}。

\end{quote}
例如在下面的 docstring 中定义了变量 \texttt{a},预期的输出,跟在 Julia REPL 中打印的一样,出现在后面。


\begin{minted}{julia}
"""
Some nice documentation here.

# Examples
```jldoctest
julia> a = [1 2; 3 4]
2×2 Array{Int64,2}:
 1  2
 3  4
```
"""
\end{minted}

\begin{quote}
\textbf{Warning}

Calling \texttt{rand} and other RNG-related functions should be avoided in doctests since they will not produce consistent outputs during different Julia sessions. If you would like to show some random number generation related functionality, one option is to explicitly construct and seed your own \hyperlink{4960058165975837552}{\texttt{MersenneTwister}} (or other pseudorandom number generator) and pass it to the functions you are doctesting.

Operating system word size (\hyperlink{10103694114785108551}{\texttt{Int32}} or \hyperlink{7720564657383125058}{\texttt{Int64}}) as well as path separator differences (\texttt{/} or \texttt{{\textbackslash}}) will also affect the reproducibility of some doctests.

Note that whitespace in your doctest is significant! The doctest will fail if you misalign the output of pretty-printing an array, for example.

\end{quote}
你可以运行 \texttt{make -C doc doctest=true} 来运行在 Julia 手册和 API 文档中的 doctests,这样可以确保你的例子都能正常运行。

为了表示输出结果被截断了,你应该在校验应该停止的一行写上 \texttt{[...]}。这个在当 doctest 显示有个异常被抛出时隐藏堆栈跟踪时很有用(堆栈跟踪包含对 julia 代码的行的非永久引用),例如:


\begin{minted}{julia}
```jldoctest
julia> div(1, 0)
ERROR: DivideError: integer division error
[...]
```
\end{minted}

那些不能进行测试的例子应该写在以 \texttt{```julia} 开头的栅栏分隔的代码块中,以便在生成的文档中正确地高亮显示。

\begin{quote}
\textbf{Tip}

例子应尽可能\textbf{独立}和\textbf{可运行}以便读者可以在不需要引入任何依赖的情况下对它们进行实验。

\end{quote}

\item[7.  ] 使用倒引号来标识代码和方程。

Julia 标识符和代码摘录应该出现在倒引号 \texttt{`} 之间来使其能高亮显示。LaTeX 语法下的方程应该插入到双倒引号 \texttt{``} 之间。请使用 Unicode 字符而非 LaTeX 转义序列,比如 \texttt{``α = 1``} 而非 \texttt{``{\textbackslash}{\textbackslash}alpha = 1``}。


\item[8.  ] 请将起始和结束的\texttt{{\textquotedbl}{\textquotedbl}{\textquotedbl}}符号单独成行。

也就是说,请写:


\begin{minted}{julia}
"""
...

...
"""
f(x, y) = ...
\end{minted}

而非:


\begin{minted}{julia}
"""...

..."""
f(x, y) = ...
\end{minted}

This makes it clearer where docstrings start and end.


\item[9.  ] 请在代码中遵守单行长度限制。

Docstring 是使用与代码相同的工具编辑的。所以应运用同样的约定。 建议一行 92 个字符后换行。 It is recommended that lines are at most 92 characters wide.


\item[10. ] 请在 \texttt{\# Implementation} 章节中提供自定义类型如何实现该函数的信息。这些实现细节是针对开发者而非用户的,解释了例如哪些函数应该被重写、哪些函数自动使用恰当的回退函数等信息,最好与描述函数的主体描述分开。 \texttt{\# Implementation} section. These implementation details are intended for developers rather than users, explaining e.g. which functions should be overridden and which functions automatically use appropriate fallbacks. Such details are best kept separate from the main description of the function{\textquotesingle}s behavior.


\item[11. ] For long docstrings, consider splitting the documentation with an \texttt{\# Extended help} header. The typical help-mode will show only the material above the header; you can access the full help by adding a {\textquotesingle}?{\textquotesingle} at the beginning of the expression (i.e., {\textquotedbl}??foo{\textquotedbl} rather than {\textquotedbl}?foo{\textquotedbl}).

\end{itemize}


\hypertarget{10486321714157126961}{}


\section{访问文档}



文档可以在REPL中访问,也可以在 \href{https://github.com/JuliaLang/IJulia.jl}{IJulia} 中通过键入\texttt{?}紧接函数或者宏的名字并按下\texttt{Enter}访问。例如,




\begin{minted}{julia}
?cos
?@time
?r""
\end{minted}



will show documentation for the relevant function, macro or string macro respectively. In \href{http://junolab.org}{Juno} using \texttt{Ctrl-J, Ctrl-D} will show the documentation for the object under the cursor.



\hypertarget{5384792468741396562}{}


\section{函数与方法}



在Julia中函数可能有多种实现,被称为方法。虽然通用函数 一般只有一个目的,Julia允许在必要时可以对方法独立写文档。 通常,应该只有最通用的方法才有文档,或者甚至只是函数本身 (也就是在\texttt{function bar end}之前没有任何方法的对象)。特定方法应该 只因为其行为与其他通用方法有所区别才写文档。在任何情况下都不应 重复其他地方有的信息。例如




\begin{minted}{julia}
"""
    *(x, y, z...)

Multiplication operator. `x * y * z *...` calls this function with multiple
arguments, i.e. `*(x, y, z...)`.
"""
function *(x, y, z...)
    # ... [implementation sold separately] ...
end

"""
    *(x::AbstractString, y::AbstractString, z::AbstractString...)

When applied to strings, concatenates them.
"""
function *(x::AbstractString, y::AbstractString, z::AbstractString...)
    # ... [insert secret sauce here] ...
end

help?> *
search: * .*

  *(x, y, z...)

  Multiplication operator. x * y * z *... calls this function with multiple
  arguments, i.e. *(x,y,z...).

  *(x::AbstractString, y::AbstractString, z::AbstractString...)

  When applied to strings, concatenates them.
\end{minted}



当从通用函数里抽取文档时,每个方法的元数据会用函数\texttt{catdoc}拼接,其当然可以被自定义类型重写。



\hypertarget{9947466122062338539}{}


\section{进阶用法}



The \texttt{@doc} macro associates its first argument with its second in a per-module dictionary called \texttt{META}.



为了让写文档更加简单,语法分析器对宏名\texttt{@doc}特殊对待:如果\texttt{@doc}的调用只有一个参数,但是在下一行出现了另外一个表达式,那么这个表达式就会追加为宏的参数。所以接下来的语法会被分析成\texttt{@doc}的2个参数的调用:




\begin{minted}{julia}
@doc raw"""
...
"""
f(x) = x
\end{minted}



This makes it possible to use expressions other than normal string literals (such as the \texttt{raw{\textquotedbl}{\textquotedbl}} string macro) as a docstring.



当\texttt{@doc}宏(或者\texttt{doc}函数)用作抽取文档时,他会在所有的\texttt{META}字典寻找与对象相关的元数据并且返回。返回的对象(例如一些Markdown内容)会默认智能地显示。这个设计也让以编程方法使用文档系统变得容易;例如,在一个函数的不同版本中重用文档:




\begin{minted}{julia}
@doc "..." foo!
@doc (@doc foo!) foo
\end{minted}



或者与Julia的元编程功能一起使用:




\begin{minted}{julia}
for (f, op) in ((:add, :+), (:subtract, :-), (:multiply, :*), (:divide, :/))
    @eval begin
        $f(a,b) = $op(a,b)
    end
end
@doc "`add(a,b)` adds `a` and `b` together" add
@doc "`subtract(a,b)` subtracts `b` from `a`" subtract
\end{minted}



写在非顶级块,比如\texttt{begin}, \texttt{if}, \texttt{for}, 和 \texttt{let},中的文档会根据块的评估情况加入文档系统中,例如:




\begin{minted}{julia}
if condition()
    "..."
    f(x) = x
end
\end{minted}



会被加到\texttt{f(x)}的文档中,当\texttt{condition()}是\texttt{true}的时候。注意即使\texttt{f(x)}在块的末尾离开了作用域,他的文档还会保留。



It is possible to make use of metaprogramming to assist in the creation of documentation. When using string-interpolation within the docstring you will need to use an extra \texttt{\$} as shown with \texttt{\$(\$name)}:




\begin{minted}{julia}
for func in (:day, :dayofmonth)
    name = string(func)
    @eval begin
        @doc """
            $($name)(dt::TimeType) -> Int64

        The day of month of a `Date` or `DateTime` as an `Int64`.
        """ $func(dt::Dates.TimeType)
    end
end
\end{minted}



\hypertarget{7588661187190124361}{}


\subsection{动态写文档}



有些时候类型的实例的合适的文档并非只取决于类型本身,也取决于实例的值。在这些情况下,你可以添加一个方法给自定义类型的\texttt{Docs.getdoc}函数,返回基于每个实例的文档。例如,




\begin{minted}{julia}
struct MyType
    value::String
end

Docs.getdoc(t::MyType) = "Documentation for MyType with value $(t.value)"

x = MyType("x")
y = MyType("y")
\end{minted}



输入\texttt{?x}会显示{\textquotedbl}Documentation for MyType with value x{\textquotedbl},输入\texttt{?y}则会显示{\textquotedbl}Documentation for MyType with value y{\textquotedbl}。



\hypertarget{12360783788480513081}{}


\section{语法指南}



This guide provides a comprehensive overview of how to attach documentation to all Julia syntax constructs for which providing documentation is possible.



在下述例子中\texttt{{\textquotedbl}...{\textquotedbl}}用来表示任意的docstring。



\hypertarget{4617511478986608909}{}


\subsection{\texttt{\$} and \texttt{{\textbackslash}} characters}



The \texttt{\$} and \texttt{{\textbackslash}} characters are still parsed as string interpolation or start of an escape sequence in docstrings too. The \texttt{raw{\textquotedbl}{\textquotedbl}} string macro together with the \texttt{@doc} macro can be used to avoid having to escape them. This is handy when the docstrings include LaTeX or Julia source code examples containing interpolation:




\begin{minted}{julia}
@doc raw"""
```math
\LaTeX
```
"""
function f end
\end{minted}



\hypertarget{198065463690830309}{}


\subsection{函数与方法}




\begin{minted}{julia}
"..."
function f end

"..."
f
\end{minted}



把 docstring \texttt{{\textquotedbl}...{\textquotedbl}} 添加给了函数 \texttt{f}。首选的语法是第一种,虽然两者是等价的。




\begin{minted}{julia}
"..."
f(x) = x

"..."
function f(x)
    x
end

"..."
f(x)
\end{minted}



把 docstring \texttt{{\textquotedbl}...{\textquotedbl}} 添加给了方法 \texttt{f(::Any)}。




\begin{minted}{julia}
"..."
f(x, y = 1) = x + y
\end{minted}



把 docstring \texttt{{\textquotedbl}...{\textquotedbl}} 添加给了两个方法,分别为 \texttt{f(::Any)} 和 \texttt{f(::Any, ::Any)}。



\hypertarget{958098821246331521}{}


\subsection{宏}




\begin{minted}{julia}
"..."
macro m(x) end
\end{minted}



把 docstring \texttt{{\textquotedbl}...{\textquotedbl}} 添加给了宏 \texttt{@m(::Any)} 的定义。




\begin{minted}{julia}
"..."
:(@m)
\end{minted}



把 docstring \texttt{{\textquotedbl}...{\textquotedbl}} 添加给了名为 \texttt{@m} 的宏。



\hypertarget{5145884598618557277}{}


\subsection{类型}




\begin{lstlisting}
"..."
abstract type T1 end

"..."
mutable struct T2
    ...
end

"..."
struct T3
    ...
end
\end{lstlisting}



把 docstring \texttt{{\textquotedbl}...{\textquotedbl}} 添加给了类型 \texttt{T1}、\texttt{T2} 和 \texttt{T3}。




\begin{minted}{julia}
"..."
struct T
    "x"
    x
    "y"
    y
end
\end{minted}



把 docstring \texttt{{\textquotedbl}...{\textquotedbl}} 添加给了类型 \texttt{T},\texttt{{\textquotedbl}x{\textquotedbl}} 添加给字段 \texttt{T.x},\texttt{{\textquotedbl}y{\textquotedbl}} 添加给字段 \texttt{T.y}。也可以运用于\texttt{mutable struct} 类型。



\hypertarget{18307601734386005108}{}


\subsection{模块}




\begin{minted}{julia}
"..."
module M end

module M

"..."
M

end
\end{minted}



Adds docstring \texttt{{\textquotedbl}...{\textquotedbl}} to the \texttt{Module} \texttt{M}. Adding the docstring above the \texttt{Module} is the preferred syntax, however both are equivalent.




\begin{minted}{julia}
"..."
baremodule M
# ...
end

baremodule M

import Base: @doc

"..."
f(x) = x

end
\end{minted}



通过把 docstring 放在表达式之上来给一个 \texttt{baremodule} 写文档会在模块中自动引入 \texttt{@doc}。它在模块表达式并没有文档时必须手动引入。空的 \texttt{baremodule} 不能有文档。



\hypertarget{17299710876257759093}{}


\subsection{全局变量}




\begin{minted}{julia}
"..."
const a = 1

"..."
b = 2

"..."
global c = 3
\end{minted}



把docstring\texttt{{\textquotedbl}...{\textquotedbl}}添加给了\texttt{绑定} \texttt{a},\texttt{b}和\texttt{c}。



\texttt{绑定}是用来在\texttt{模块}中存储对于特定\texttt{符号}的引用而非存储被引用的值本身。



\begin{quote}
\textbf{Note}

当一个 \texttt{const} 定义只是用作定义另外一个定义的别名时,比如函数 \texttt{div} 和其在 \texttt{Base} 中的别名 \texttt{÷},并不要为别名写文档,转而去为实际的函数写文档。

如果别名写了文档而实际定义没有,那么文档系统(\texttt{?} 模式)在寻找实际定义的文档时将不会返回别名的对应文档。

比如你应该写


\begin{minted}{julia}
"..."
f(x) = x + 1
const alias = f
\end{minted}

而非


\begin{minted}{julia}
f(x) = x + 1
"..."
const alias = f
\end{minted}

\end{quote}



\begin{minted}{julia}
"..."
sym
\end{minted}



Adds docstring \texttt{{\textquotedbl}...{\textquotedbl}} to the value associated with \texttt{sym}. However, it is preferred that \texttt{sym} is documented where it is defined.



\hypertarget{6306013858572164490}{}


\subsection{多重对象}




\begin{minted}{julia}
"..."
a, b
\end{minted}



把docstring \texttt{{\textquotedbl}...{\textquotedbl}} 添加给\texttt{a}和\texttt{b},两个都应该是可以写文档的表达式。这个语法等价于




\begin{minted}{julia}
"..."
a

"..."
b
\end{minted}



这种方法可以给任意数量的表达式写文档。当两个函数相关,比如非变版本\texttt{f}和可变版本\texttt{f!},这个语法是有用的。



\hypertarget{5068022185116881179}{}


\subsection{宏生成代码}




\begin{minted}{julia}
"..."
@m expression
\end{minted}



Adds docstring \texttt{{\textquotedbl}...{\textquotedbl}} to the expression generated by expanding \texttt{@m expression}. This allows for expressions decorated with \texttt{@inline}, \texttt{@noinline}, \texttt{@generated}, or any other macro to be documented in the same way as undecorated expressions.



宏作者应该注意到只有只生成单个表达式的宏才会自动支持docstring。如果宏返回的是含有多个子表达式的块,需要写文档的子表达式应该使用宏 \hyperlink{8194145670752069829}{\texttt{@\_\_doc\_\_}} 标记。



The \hyperlink{18177775477210803027}{\texttt{@enum}} macro makes use of \texttt{@\_\_doc\_\_} to allow for documenting \hyperlink{12477318268908279491}{\texttt{Enum}}s. Examining its definition should serve as an example of how to use \texttt{@\_\_doc\_\_} correctly.


\hypertarget{8194145670752069829}{} 
\hyperlink{8194145670752069829}{\texttt{Core.@\_\_doc\_\_}}  -- {Macro.}

\begin{adjustwidth}{2em}{0pt}


\begin{minted}{julia}
@__doc__(ex)
\end{minted}

Low-level macro used to mark expressions returned by a macro that should be documented. If more than one expression is marked then the same docstring is applied to each expression.


\begin{lstlisting}
macro example(f)
    quote
        $(f)() = 0
        @__doc__ $(f)(x) = 1
        $(f)(x, y) = 2
    end |> esc
end
\end{lstlisting}

\texttt{@\_\_doc\_\_} has no effect when a macro that uses it is not documented.



\href{https://github.com/JuliaLang/julia/blob/44fa15b1502a45eac76c9017af94332d4557b251/base/docs/Docs.jl#L428-L443}{\texttt{source}}


\end{adjustwidth}

\hypertarget{899642791320764560}{}


\chapter{元编程}



Lisp 留给 Julia 最大的遗产就是它的元编程支持。和 Lisp 一样,Julia 把自己的代码表示为语言中的数据结构。既然代码被表示为了可以在语言中创建和操作的对象,程序就可以变换和生成自己的代码。这允许在没有额外构建步骤的情况下生成复杂的代码,并且还允许在 \href{https://en.wikipedia.org/wiki/Abstract\_syntax\_tree}{abstract syntax trees} 级别上运行的真正的 Lisp 风格的宏。与之相对的是预处理器“宏”系统,比如 C 和 C++ 中的,它们在解析和解释代码之前进行文本操作和变换。由于 Julia 中的所有数据类型和代码都被表示为 Julia 的 数据结构,强大的 \href{https://en.wikipedia.org/wiki/Reflection\_omputer\_\%28cprogramming\%29}{reflection} 功能可用于探索程序的内部及其类型,就像任何其他数据一样。



\hypertarget{13890173916214395200}{}


\section{程序表示}



每个 Julia 程序均以字符串开始:




\begin{minted}{jlcon}
julia> prog = "1 + 1"
"1 + 1"
\end{minted}



\textbf{What happens next?}



The next step is to \href{https://en.wikipedia.org/wiki/Parsing\#Computer\_languages}{parse} each string into an object called an expression, represented by the Julia type \hyperlink{17120496304147995299}{\texttt{Expr}}:




\begin{minted}{jlcon}
julia> ex1 = Meta.parse(prog)
:(1 + 1)

julia> typeof(ex1)
Expr
\end{minted}



\texttt{Expr} 对象包含两个部分:



\begin{itemize}
\item a \hyperlink{18332791376992528422}{\texttt{Symbol}} identifying the kind of expression. A symbol is an \href{https://en.wikipedia.org/wiki/String\_interning}{interned string} 标识符(下面会有更多讨论)

\end{itemize}



\begin{minted}{jlcon}
julia> ex1.head
:call
\end{minted}



\begin{itemize}
\item 表达式的参数,可能是符号、其他表达式或字面值:

\end{itemize}



\begin{minted}{jlcon}
julia> ex1.args
3-element Array{Any,1}:
  :+
 1
 1
\end{minted}



表达式也可能直接用 \href{https://en.wikipedia.org/wiki/Polish\_notation}{prefix notation} 构造:




\begin{minted}{jlcon}
julia> ex2 = Expr(:call, :+, 1, 1)
:(1 + 1)
\end{minted}



上面构造的两个表达式 – 一个通过解析构造一个通过直接构造 – 是等价的:




\begin{minted}{jlcon}
julia> ex1 == ex2
true
\end{minted}



\textbf{这里的关键点是 Julia 的代码在内部表示为可以从语言本身访问的数据结构}



函数 \hyperlink{15981569052160951906}{\texttt{dump}} 可以带有缩进和注释地显示 \texttt{Expr} 对象:




\begin{minted}{jlcon}
julia> dump(ex2)
Expr
  head: Symbol call
  args: Array{Any}((3,))
    1: Symbol +
    2: Int64 1
    3: Int64 1
\end{minted}



\texttt{Expr} 对象也可以嵌套:




\begin{minted}{jlcon}
julia> ex3 = Meta.parse("(4 + 4) / 2")
:((4 + 4) / 2)
\end{minted}



另外一个查看表达式的方法是使用 \texttt{Meta.show\_sexpr},它能显示给定 \texttt{Expr} 的 \href{https://en.wikipedia.org/wiki/S-expression}{S-expression},对 Lisp 用户来说,这看着很熟悉。下面是一个示例,阐释了如何显示嵌套的 \texttt{Expr}:




\begin{minted}{jlcon}
julia> Meta.show_sexpr(ex3)
(:call, :/, (:call, :+, 4, 4), 2)
\end{minted}



\hypertarget{17960433860062790097}{}


\subsection{符号}



字符 \texttt{:} 在 Julia 中有两个作用。第一种形式构造一个  \hyperlink{18332791376992528422}{\texttt{Symbol}},这是作为表达式组成部分的一个 \href{https://en.wikipedia.org/wiki/String\_interning}{interned string}:




\begin{minted}{jlcon}
julia> :foo
:foo

julia> typeof(ans)
Symbol
\end{minted}



构造函数 \hyperlink{18332791376992528422}{\texttt{Symbol}} 接受任意数量的参数并通过把它们的字符串表示连在一起创建一个新的符号:




\begin{minted}{jlcon}
julia> :foo == Symbol("foo")
true

julia> Symbol("func",10)
:func10

julia> Symbol(:var,'_',"sym")
:var_sym
\end{minted}



Note that to use \texttt{:} syntax, the symbol{\textquotesingle}s name must be a valid identifier. Otherwise the \texttt{Symbol(str)} constructor must be used.



在表达式的上下文中,符号用来表示对变量的访问;当一个表达式被求值时,符号会被替换为这个符号在合适的 \hyperlink{11957539949537805757}{scope} 中所绑定的值。



Sometimes extra parentheses around the argument to \texttt{:} are needed to avoid ambiguity in parsing:




\begin{minted}{jlcon}
julia> :(:)
:(:)

julia> :(::)
:(::)
\end{minted}



\hypertarget{3051242913122022314}{}


\section{表达式与求值}



\hypertarget{12430289445905702597}{}


\subsection{引用}



The second syntactic purpose of the \texttt{:} character is to create expression objects without using the explicit \hyperlink{17120496304147995299}{\texttt{Expr}} constructor. This is referred to as \emph{quoting}. The \texttt{:} character, followed by paired parentheses around a single statement of Julia code, produces an \texttt{Expr} object based on the enclosed code. Here is example of the short form used to quote an arithmetic expression:




\begin{minted}{jlcon}
julia> ex = :(a+b*c+1)
:(a + b * c + 1)

julia> typeof(ex)
Expr
\end{minted}



(为了查看这个表达式的结构,可以试一试 \texttt{ex.head} 和 \texttt{ex.args},或者使用 \hyperlink{15981569052160951906}{\texttt{dump}} 同时查看 \texttt{ex.head} 和 \texttt{ex.args} 或者 \hyperlink{11314997131411442967}{\texttt{Meta.@dump}})



注意等价的表达式也可以使用 \hyperlink{10422957797582368651}{\texttt{Meta.parse}} 或者直接用 \texttt{Expr} 构造:




\begin{minted}{jlcon}
julia>      :(a + b*c + 1)       ==
       Meta.parse("a + b*c + 1") ==
       Expr(:call, :+, :a, Expr(:call, :*, :b, :c), 1)
true
\end{minted}



解析器提供的表达式通常只有符号、其它表达式和字面量值作为其参数,而由 Julia 代码构造的表达式能以非字面量形式的任意运行期值作为其参数。在此特例中,\texttt{+} 和 \texttt{a} 都是符号,\texttt{*(b,c)} 是子表达式,而 \texttt{1} 是 64 位带符号整数字面量。



引用多个表达式有第二种语法形式:在 \texttt{quote ... end} 中包含代码块。




\begin{minted}{jlcon}
julia> ex = quote
           x = 1
           y = 2
           x + y
       end
quote
    #= none:2 =#
    x = 1
    #= none:3 =#
    y = 2
    #= none:4 =#
    x + y
end

julia> typeof(ex)
Expr
\end{minted}



\hypertarget{6473060285850683914}{}


\subsection{Interpolation}



Direct construction of \hyperlink{17120496304147995299}{\texttt{Expr}} objects with value arguments is powerful, but \texttt{Expr} constructors can be tedious compared to {\textquotedbl}normal{\textquotedbl} Julia syntax. As an alternative, Julia allows \emph{interpolation} of literals or expressions into quoted expressions. Interpolation is indicated by a prefix \texttt{\$}.



在此示例中,插入了变量 \texttt{a} 的值:




\begin{minted}{jlcon}
julia> a = 1;

julia> ex = :($a + b)
:(1 + b)
\end{minted}



对未被引用的表达式进行插值是不支持的,这会导致编译期错误:




\begin{minted}{jlcon}
julia> $a + b
ERROR: syntax: "$" expression outside quote
\end{minted}



在此示例中,元组 \texttt{(1,2,3)} 作为表达式插入到条件测试中:




\begin{minted}{jlcon}
julia> ex = :(a in $:((1,2,3)) )
:(a in (1, 2, 3))
\end{minted}



在表达式插值中使用 \texttt{\$} 是有意让人联想到\hyperlink{4452850363638134205}{字符串插值}和\hyperlink{3603331931999023419}{命令插值}。表达式插值使得复杂 Julia 表达式的程序化构造变得方便和易读。



\hypertarget{12062767751485347352}{}


\subsection{Splatting 插值}



请注意,\texttt{\$} 插值语法只允许插入单个表达式到包含它的表达式中。有时,你手头有个由表达式组成的数组,需要它们都变成其所处表达式的参数,而这可通过 \texttt{\$(xs...)} 语法做到。例如,下面的代码生成了一个函数调用,其参数数量通过编程确定:




\begin{minted}{jlcon}
julia> args = [:x, :y, :z];

julia> :(f(1, $(args...)))
:(f(1, x, y, z))
\end{minted}



\hypertarget{15842920146318002847}{}


\subsection{嵌套引用}



自然地,引用表达式可以包含在其它引用表达式中。插值在这些情形中的工作方式可能会有点难以理解。考虑这个例子:




\begin{minted}{jlcon}
julia> x = :(1 + 2);

julia> e = quote quote $x end end
quote
    #= none:1 =#
    $(Expr(:quote, quote
    #= none:1 =#
    $(Expr(:$, :x))
end))
end
\end{minted}



Notice that the result contains \texttt{\$x}, which means that \texttt{x} has not been evaluated yet. In other words, the \texttt{\$} expression {\textquotedbl}belongs to{\textquotedbl} the inner quote expression, and so its argument is only evaluated when the inner quote expression is:




\begin{minted}{jlcon}
julia> eval(e)
quote
    #= none:1 =#
    1 + 2
end
\end{minted}



但是,外部 \texttt{quote} 表达式可以把值插入到内部引用表达式的 \texttt{\$} 中去。这通过多个 \texttt{\$} 实现:




\begin{minted}{jlcon}
julia> e = quote quote $$x end end
quote
    #= none:1 =#
    $(Expr(:quote, quote
    #= none:1 =#
    $(Expr(:$, :(1 + 2)))
end))
end
\end{minted}



Notice that \texttt{(1 + 2)} now appears in the result instead of the symbol \texttt{x}. Evaluating this expression yields an interpolated \texttt{3}:




\begin{minted}{jlcon}
julia> eval(e)
quote
    #= none:1 =#
    3
end
\end{minted}



这种行为背后的直觉是每个 \texttt{\$} 都将 \texttt{x} 求值一遍:一个 \texttt{\$} 工作方式类似于 \texttt{eval(:x)},其返回 \texttt{x} 的值,而两个 \texttt{\$} 行为相当于 \texttt{eval(eval(:x))}。



\hypertarget{15194695339988120385}{}


\subsection{QuoteNode}



The usual representation of a \texttt{quote} form in an AST is an \hyperlink{17120496304147995299}{\texttt{Expr}} with head \texttt{:quote}:




\begin{minted}{jlcon}
julia> dump(Meta.parse(":(1+2)"))
Expr
  head: Symbol quote
  args: Array{Any}((1,))
    1: Expr
      head: Symbol call
      args: Array{Any}((3,))
        1: Symbol +
        2: Int64 1
        3: Int64 2
\end{minted}



As we have seen, such expressions support interpolation with \texttt{\$}. However, in some situations it is necessary to quote code \emph{without} performing interpolation. This kind of quoting does not yet have syntax, but is represented internally as an object of type \texttt{QuoteNode}:




\begin{minted}{jlcon}
julia> eval(Meta.quot(Expr(:$, :(1+2))))
3

julia> eval(QuoteNode(Expr(:$, :(1+2))))
:($(Expr(:$, :(1 + 2))))
\end{minted}



The parser yields \texttt{QuoteNode}s for simple quoted items like symbols:




\begin{minted}{jlcon}
julia> dump(Meta.parse(":x"))
QuoteNode
  value: Symbol x
\end{minted}



\texttt{QuoteNode} can also be used for certain advanced metaprogramming tasks.



\hypertarget{15751418549857902160}{}


\subsection{Evaluating expressions}



Given an expression object, one can cause Julia to evaluate (execute) it at global scope using \hyperlink{7507639810592563424}{\texttt{eval}}:




\begin{minted}{jlcon}
julia> :(1 + 2)
:(1 + 2)

julia> eval(ans)
3

julia> ex = :(a + b)
:(a + b)

julia> eval(ex)
ERROR: UndefVarError: b not defined
[...]

julia> a = 1; b = 2;

julia> eval(ex)
3
\end{minted}



Every \hyperlink{16725527896995457152}{module} has its own \hyperlink{7507639810592563424}{\texttt{eval}} function that evaluates expressions in its global scope. Expressions passed to \hyperlink{7507639810592563424}{\texttt{eval}} are not limited to returning values – they can also have side-effects that alter the state of the enclosing module{\textquotesingle}s environment:




\begin{minted}{jlcon}
julia> ex = :(x = 1)
:(x = 1)

julia> x
ERROR: UndefVarError: x not defined

julia> eval(ex)
1

julia> x
1
\end{minted}



这里,表达式对象的求值导致一个值被赋值给全局变量 \texttt{x}。



由于表达式只是 \texttt{Expr} 对象,而其可以通过编程方式构造然后对它求值,因此可以动态地生成任意代码,然后使用 \hyperlink{7507639810592563424}{\texttt{eval}} 运行所生成的代码。这是个简单的例子:




\begin{minted}{jlcon}
julia> a = 1;

julia> ex = Expr(:call, :+, a, :b)
:(1 + b)

julia> a = 0; b = 2;

julia> eval(ex)
3
\end{minted}



\texttt{a} 的值被用于构造表达式 \texttt{ex},该表达式将函数 \texttt{+} 作用于值 1 和变量 \texttt{b}。请注意 \texttt{a} 和 \texttt{b} 使用方式间的重要区别:



\begin{itemize}
\item \emph{变量} \texttt{a} 在表达式构造时的值在表达式中用作立即值。因此,在对表达式求值时,\texttt{a} 的值就无关紧要了:表达式中的值已经是 \texttt{1},与 \texttt{a} 的值无关。


\item 另一方面,因为在表达式构造时用的是符号 \texttt{:b},所以变量 \texttt{b} 的值无关紧要——\texttt{:b} 只是一个符号,变量 \texttt{b} 甚至无需被定义。然而,在表达式求值时,符号 \texttt{:b} 的值通过寻找变量 \texttt{b} 的值来解析。

\end{itemize}


\hypertarget{2578517908282982662}{}


\subsection{关于表达式的函数}



As hinted above, one extremely useful feature of Julia is the capability to generate and manipulate Julia code within Julia itself. We have already seen one example of a function returning \hyperlink{17120496304147995299}{\texttt{Expr}} objects: the \hyperlink{14207407853646164654}{\texttt{parse}} function, which takes a string of Julia code and returns the corresponding \texttt{Expr}. A function can also take one or more \texttt{Expr} objects as arguments, and return another \texttt{Expr}. Here is a simple, motivating example:




\begin{minted}{jlcon}
julia> function math_expr(op, op1, op2)
           expr = Expr(:call, op, op1, op2)
           return expr
       end
math_expr (generic function with 1 method)

julia>  ex = math_expr(:+, 1, Expr(:call, :*, 4, 5))
:(1 + 4 * 5)

julia> eval(ex)
21
\end{minted}



作为另一个例子,这个函数将数值参数加倍,但不处理表达式:




\begin{minted}{jlcon}
julia> function make_expr2(op, opr1, opr2)
           opr1f, opr2f = map(x -> isa(x, Number) ? 2*x : x, (opr1, opr2))
           retexpr = Expr(:call, op, opr1f, opr2f)
           return retexpr
       end
make_expr2 (generic function with 1 method)

julia> make_expr2(:+, 1, 2)
:(2 + 4)

julia> ex = make_expr2(:+, 1, Expr(:call, :*, 5, 8))
:(2 + 5 * 8)

julia> eval(ex)
42
\end{minted}



\hypertarget{11146454106624591870}{}


\section{宏}



宏提供了在程序的最终主体中包含所生成的代码的方法。宏将参数元组映射到所返回的\emph{表达式},且生成的表达式会被直接编译,并不需要运行时的 \hyperlink{7507639810592563424}{\texttt{eval}} 调用。宏的参数可以包括表达式、字面量值和符号。



\hypertarget{13022711559815737661}{}


\subsection{基础}



这是一个非常简单的宏:




\begin{minted}{jlcon}
julia> macro sayhello()
           return :( println("Hello, world!") )
       end
@sayhello (macro with 1 method)
\end{minted}



宏在Julia的语法中有一个专门的字符 \texttt{@} (at-sign),紧接着是其使用\texttt{macro NAME ... end} 形式来声明的唯一的宏名。在这个例子中,编译器会把所有的\texttt{@sayhello} 替换成:




\begin{minted}{julia}
:( println("Hello, world!") )
\end{minted}



当 \texttt{@sayhello} 在REPL中被输入时,解释器立即执行,因此我们只会看到计算后的结果:




\begin{minted}{jlcon}
julia> @sayhello()
Hello, world!
\end{minted}



现在,考虑一个稍微复杂一点的宏:




\begin{minted}{jlcon}
julia> macro sayhello(name)
 return :( println("Hello, ", $name) )
 end
@sayhello (macro with 1 method)
\end{minted}



这个宏接受一个参数\texttt{name}。当遇到\texttt{@sayhello}时,quoted 表达式会被\emph{展开}并将参数中的值插入到最终的表达式中:




\begin{minted}{jlcon}
julia> @sayhello("human")
Hello, human
\end{minted}



We can view the quoted return expression using the function \hyperlink{8018172489611994488}{\texttt{macroexpand}} (\textbf{important note:} this is an extremely useful tool for debugging macros):




\begin{lstlisting}
julia> ex = macroexpand(Main, :(@sayhello("human")) )
:(Main.println("Hello, ", "human"))

julia> typeof(ex)
Expr
\end{lstlisting}



我们可以看到 \texttt{{\textquotedbl}human{\textquotedbl}} 字面量已被插入到表达式中了。



还有一个宏 \href{@ ref}{\texttt{@ macroexpand}},它可能比 \texttt{macroexpand} 函数更方便:




\begin{minted}{jlcon}
julia> @macroexpand @sayhello "human"
:(println("Hello, ", "human"))
\end{minted}



\hypertarget{12261577225730588688}{}


\subsection{Hold up: why macros?}



We have already seen a function \texttt{f(::Expr...) -> Expr} in a previous section. In fact, \hyperlink{8018172489611994488}{\texttt{macroexpand}} is also such a function. So, why do macros exist?



Macros are necessary because they execute when code is parsed, therefore, macros allow the programmer to generate and include fragments of customized code \emph{before} the full program is run. To illustrate the difference, consider the following example:




\begin{lstlisting}
julia> macro twostep(arg)
           println("I execute at parse time. The argument is: ", arg)
           return :(println("I execute at runtime. The argument is: ", $arg))
       end
@twostep (macro with 1 method)

julia> ex = macroexpand(Main, :(@twostep :(1, 2, 3)) );
I execute at parse time. The argument is: :((1, 2, 3))
\end{lstlisting}



第一个 \hyperlink{783803254548423222}{\texttt{println}} 调用在调用 \hyperlink{8018172489611994488}{\texttt{macroexpand}} 时执行。生成的表达式\emph{只}包含第二个 \texttt{println}:




\begin{lstlisting}
julia> typeof(ex)
Expr

julia> ex
:(println("I execute at runtime. The argument is: ", $(Expr(:copyast, :($(QuoteNode(:((1, 2, 3)))))))))

julia> eval(ex)
I execute at runtime. The argument is: (1, 2, 3)
\end{lstlisting}



\hypertarget{14488850884072492512}{}


\subsection{宏的调用}



宏的通常调用语法如下:




\begin{minted}{julia}
@name expr1 expr2 ...
@name(expr1, expr2, ...)
\end{minted}



请注意,在宏名称前的标志 \texttt{@},且在第一种形式中参数表达式间没有逗号,而在第二种形式中 \texttt{@name} 后没有空格。这两种风格不应混淆。例如,下列语法不同于上述例子;它把元组 \texttt{(expr1, expr2, ...)} 作为参数传给宏:




\begin{minted}{julia}
@name (expr1, expr2, ...)
\end{minted}



在数组字面量(或推导式)上调用宏的另一种方法是不使用括号直接并列两者。在这种情况下,数组将是唯一的传给宏的表达式。以下语法等价(且与 \texttt{@name [a b] * v} 不同):




\begin{minted}{julia}
@name[a b] * v
@name([a b]) * v
\end{minted}



在这着重强调,宏把它们的参数作为表达式、字面量或符号接收。浏览宏参数的一种方法是在宏的内部调用 \hyperlink{14071376285304310153}{\texttt{show}} 函数:




\begin{minted}{jlcon}
julia> macro showarg(x)
           show(x)
           # ... remainder of macro, returning an expression
       end
@showarg (macro with 1 method)

julia> @showarg(a)
:a

julia> @showarg(1+1)
:(1 + 1)

julia> @showarg(println("Yo!"))
:(println("Yo!"))
\end{minted}



除了给定的参数列表,每个宏都会传递名为 \texttt{\_\_source\_\_} 和 \texttt{\_\_module\_\_} 的额外参数。



The argument \texttt{\_\_source\_\_} provides information (in the form of a \texttt{LineNumberNode} object) about the parser location of the \texttt{@} sign from the macro invocation. This allows macros to include better error diagnostic information, and is commonly used by logging, string-parser macros, and docs, for example, as well as to implement the \hyperlink{277452200962288519}{\texttt{@\_\_LINE\_\_}}, \hyperlink{1518763743618824993}{\texttt{@\_\_FILE\_\_}}, and \hyperlink{12719499456415901450}{\texttt{@\_\_DIR\_\_}} macros.



引用 \texttt{\_\_source\_\_.line} 和 \texttt{\_\_source\_\_.file} 即可访问位置信息:




\begin{minted}{jlcon}
julia> macro __LOCATION__(); return QuoteNode(__source__); end
@__LOCATION__ (macro with 1 method)

julia> dump(
            @__LOCATION__(
       ))
LineNumberNode
  line: Int64 2
  file: Symbol none
\end{minted}



参数 \texttt{\_\_module\_\_} 提供宏调用展开处的上下文相关信息(以 \texttt{Module} 对象的形式)。这允许宏查找上下文相关的信息,比如现有的绑定,或者将值作为附加参数插入到一个在当前模块中进行自我反射的运行时函数调用中。



\hypertarget{12271969370439662350}{}


\subsection{构建高级的宏}



Here is a simplified definition of Julia{\textquotesingle}s \hyperlink{4796942656392369899}{\texttt{@assert}} macro:




\begin{minted}{jlcon}
julia> macro assert(ex)
           return :( $ex ? nothing : throw(AssertionError($(string(ex)))) )
       end
@assert (macro with 1 method)
\end{minted}



这个宏可以像这样使用:




\begin{minted}{jlcon}
julia> @assert 1 == 1.0

julia> @assert 1 == 0
ERROR: AssertionError: 1 == 0
\end{minted}



宏调用在解析时扩展为其返回结果,并替代已编写的语法。这相当于编写:




\begin{minted}{julia}
1 == 1.0 ? nothing : throw(AssertionError("1 == 1.0"))
1 == 0 ? nothing : throw(AssertionError("1 == 0"))
\end{minted}



That is, in the first call, the expression \texttt{:(1 == 1.0)} is spliced into the test condition slot, while the value of \texttt{string(:(1 == 1.0))} is spliced into the assertion message slot. The entire expression, thus constructed, is placed into the syntax tree where the \texttt{@assert} macro call occurs. Then at execution time, if the test expression evaluates to true, then \hyperlink{9331422207248206047}{\texttt{nothing}} is returned, whereas if the test is false, an error is raised indicating the asserted expression that was false. Notice that it would not be possible to write this as a function, since only the \emph{value} of the condition is available and it would be impossible to display the expression that computed it in the error message.



The actual definition of \texttt{@assert} in Julia Base is more complicated. It allows the user to optionally specify their own error message, instead of just printing the failed expression. Just like in functions with a variable number of arguments (\href{@ref}{Varargs Functions}), this is specified with an ellipses following the last argument:




\begin{minted}{jlcon}
julia> macro assert(ex, msgs...)
           msg_body = isempty(msgs) ? ex : msgs[1]
           msg = string(msg_body)
           return :($ex ? nothing : throw(AssertionError($msg)))
       end
@assert (macro with 1 method)
\end{minted}



Now \texttt{@assert} has two modes of operation, depending upon the number of arguments it receives! If there{\textquotesingle}s only one argument, the tuple of expressions captured by \texttt{msgs} will be empty and it will behave the same as the simpler definition above. But now if the user specifies a second argument, it is printed in the message body instead of the failing expression. You can inspect the result of a macro expansion with the aptly named \hyperlink{14913190777653949565}{\texttt{@macroexpand}} macro:




\begin{lstlisting}
julia> @macroexpand @assert a == b
:(if Main.a == Main.b
        Main.nothing
    else
        Main.throw(Main.AssertionError("a == b"))
    end)

julia> @macroexpand @assert a==b "a should equal b!"
:(if Main.a == Main.b
        Main.nothing
    else
        Main.throw(Main.AssertionError("a should equal b!"))
    end)
\end{lstlisting}



实际的 \texttt{@assert} 宏还处理了另一种情形:我们如果除了打印「a should equal b」外还想打印它们的值?有人也许会天真地尝试在自定义消息中使用字符串插值,例如,\texttt{@assert a==b {\textquotedbl}a (\$a) should equal b (\$b)!{\textquotedbl}},但这不会像上面的宏一样按预期工作。你能想到为什么吗?回想一下\hyperlink{4452850363638134205}{字符串插值},内插字符串会被重写为 \hyperlink{7919678712989769360}{\texttt{string}} 的调用。比较:




\begin{minted}{jlcon}
julia> typeof(:("a should equal b"))
String

julia> typeof(:("a ($a) should equal b ($b)!"))
Expr

julia> dump(:("a ($a) should equal b ($b)!"))
Expr
  head: Symbol string
  args: Array{Any}((5,))
    1: String "a ("
    2: Symbol a
    3: String ") should equal b ("
    4: Symbol b
    5: String ")!"
\end{minted}



所以,现在宏在 \texttt{msg\_body} 中获得的不是单纯的字符串,其接收了一个完整的表达式,该表达式需进行求值才能按预期显示。这可作为 \hyperlink{7919678712989769360}{\texttt{string}} 调用的参数直接拼接到返回的表达式中;有关完整实现,请参阅 \href{https://github.com/JuliaLang/julia/blob/master/base/error.jl}{\texttt{error.jl}}。



\texttt{@assert} 宏充分利用拼接被引用的表达式,以便简化对宏内部表达式的操作。



\hypertarget{12123968542051578223}{}


\subsection{卫生宏}



在更复杂的宏中会出现关于\href{https://en.wikipedia.org/wiki/Hygienic\_macro}{卫生宏} 的问题。简而言之,宏必须确保在其返回表达式中引入的变量不会意外地与其展开处周围代码中的现有变量相冲突。相反,作为参数传递给宏的表达式通常被\emph{认为}在其周围代码的上下文中进行求值,与现有变量交互并修改之。另一个问题源于这样的事实:宏可以在不同于其定义所处模块的模块中调用。在这种情况下,我们需要确保所有全局变量都被解析到正确的模块中。Julia 比使用文本宏展开的语言(比如 C)具有更大的优势,因为它只需要考虑返回的表达式。所有其它变量(例如上面\texttt{@assert} 中的 \texttt{msg})遵循\hyperlink{11957539949537805757}{通常的作用域块规则}。



为了演示这些问题,让我们来编写宏 \texttt{@time},其以表达式为参数,记录当前时间,对表达式求值,再次记录当前时间,打印前后的时间差,然后以表达式的值作为其最终值。该宏可能看起来就像这样:




\begin{minted}{julia}
macro time(ex)
    return quote
        local t0 = time_ns()
        local val = $ex
        local t1 = time_ns()
        println("elapsed time: ", (t1-t0)/1e9, " seconds")
        val
    end
end
\end{minted}



在这里,我们希望 \texttt{t0}、\texttt{t1} 和 \texttt{val} 是私有的临时变量且 \texttt{time} 引用在 Julia Base 中的 \hyperlink{2441622941271736623}{\texttt{time}} 函数,而不是用户也许具有的任何 \texttt{time} 变量(对于 \texttt{println} 也是一样)。想象一下,如果用户表达式 \texttt{ex} 中也包含对名为 \texttt{t0} 的变量的赋值、或者定义了自己的 \texttt{time} 变量,则可能会出现问题,我们可能会得到错误或者诡异且不正确的行为。



Julia 的宏展开器以下列方式解决这些问题。首先,宏返回结果中的变量被分为局部变量或全局变量。如果一个变量被赋值(且未声明为全局变量)、声明为局部变量或者用作函数参数名称,则将其视为局部变量。否则,则认为它是全局变量。接着,局部变量重命名为唯一名称(通过生成新符号的 \hyperlink{3515345868651201289}{\texttt{gensym}} 函数),并在宏定义所处环境中解析全局变量。因此,上述两个问题都被解决了;宏的局部变量不会与任何用户变量相冲突,\texttt{time} 和 \texttt{println} 也将引用其在 Julia Base 中的定义。



然而,仍有另外的问题。考虑此宏的以下用法:




\begin{minted}{julia}
module MyModule
import Base.@time

time() = ... # compute something

@time time()
end
\end{minted}



在这里,用户表达式 \texttt{ex} 是对 \texttt{time} 的调用,但不是宏所使用的 \texttt{time} 函数。它明确地引用 \texttt{MyModule.time}。因此,我们必须将 \texttt{ex} 中的代码安排在宏调用所处环境中解析。这通过用 \hyperlink{17861659594346526773}{\texttt{esc}}「转义」表达式来完成:




\begin{minted}{julia}
macro time(ex)
    ...
    local val = $(esc(ex))
    ...
end
\end{minted}



以这种方式封装的表达式会被宏展开器单独保留,并将其简单地逐字粘贴到输出中。因此,它将在宏调用所处环境中解析。



这种转义机制可以在必要时用于「违反」卫生,以便于引入或操作用户变量。例如,以下宏在其调用所处环境中将 \texttt{x} 设置为零:




\begin{minted}{jlcon}
julia> macro zerox()
           return esc(:(x = 0))
       end
@zerox (macro with 1 method)

julia> function foo()
           x = 1
           @zerox
           return x # is zero
       end
foo (generic function with 1 method)

julia> foo()
0
\end{minted}



应当明智地使用这种变量操作,但它偶尔会很方便。



Getting the hygiene rules correct can be a formidable challenge. Before using a macro, you might want to consider whether a function closure would be sufficient. Another useful strategy is to defer as much work as possible to runtime. For example, many macros simply wrap their arguments in a \texttt{QuoteNode} or other similar \hyperlink{17120496304147995299}{\texttt{Expr}}. Some examples of this include \texttt{@task body} which simply returns \texttt{schedule(Task(() -> \$body))}, and \texttt{@eval expr}, which simply returns \texttt{eval(QuoteNode(expr))}.



为了演示,我们可以将上面的 \texttt{@time} 示例重新编写成:




\begin{minted}{julia}
macro time(expr)
    return :(timeit(() -> $(esc(expr))))
end
function timeit(f)
    t0 = time_ns()
    val = f()
    t1 = time_ns()
    println("elapsed time: ", (t1-t0)/1e9, " seconds")
    return val
end
\end{minted}



但是,我们不这样做也是有充分理由的:将 \texttt{expr} 封装在新的作用域块(该匿名函数)中也会稍微改变该表达式的含义(其中任何变量的作用域),而我们想要 \texttt{@time} 使用时对其封装的代码影响最小。



\hypertarget{11371303288264026631}{}


\subsection{宏与派发}



与 Julia 函数一样,宏也是泛型的。由于多重派发,这意味着宏也能有多个方法定义:




\begin{minted}{jlcon}
julia> macro m end
@m (macro with 0 methods)

julia> macro m(args...)
           println("$(length(args)) arguments")
       end
@m (macro with 1 method)

julia> macro m(x,y)
           println("Two arguments")
       end
@m (macro with 2 methods)

julia> @m "asd"
1 arguments

julia> @m 1 2
Two arguments
\end{minted}



但是应该记住,宏派发基于传递给宏的 AST 的类型,而不是 AST 在运行时进行求值的类型:




\begin{minted}{jlcon}
julia> macro m(::Int)
           println("An Integer")
       end
@m (macro with 3 methods)

julia> @m 2
An Integer

julia> x = 2
2

julia> @m x
1 arguments
\end{minted}



\hypertarget{4927517878935278303}{}


\section{代码生成}



当需要大量重复的样板代码时,为了避免冗余,通常以编程方式生成它。在大多数语言中,这需要一个额外的构建步骤以及生成重复代码的独立程序。在 Julia 中,表达式插值和 \hyperlink{7507639810592563424}{\texttt{eval}} 允许在通常的程序执行过程中生成这些代码。例如,考虑下列自定义类型




\begin{minted}{julia}
struct MyNumber
    x::Float64
end
# output

\end{minted}



我们想为该类型添加一些方法。在下面的循环中,我们以编程的方式完成此工作:




\begin{minted}{julia}
for op = (:sin, :cos, :tan, :log, :exp)
    eval(quote
        Base.$op(a::MyNumber) = MyNumber($op(a.x))
    end)
end
# output

\end{minted}



现在,我们对自定义类型调用这些函数:




\begin{minted}{jlcon}
julia> x = MyNumber(π)
MyNumber(3.141592653589793)

julia> sin(x)
MyNumber(1.2246467991473532e-16)

julia> cos(x)
MyNumber(-1.0)
\end{minted}



在这种方法中,Julia 充当了自己的\href{https://en.wikipedia.org/wiki/Preprocessor}{预处理器},并且允许从语言内部生成代码。使用 \texttt{:} 前缀的引用形式编写上述代码会使其更简洁:




\begin{minted}{julia}
for op = (:sin, :cos, :tan, :log, :exp)
    eval(:(Base.$op(a::MyNumber) = MyNumber($op(a.x))))
end
\end{minted}



不管怎样,这种使用 \texttt{eval(quote(...))} 模式生成语言内部的代码很常见,为此,Julia 自带了一个宏来缩写该模式:




\begin{minted}{julia}
for op = (:sin, :cos, :tan, :log, :exp)
    @eval Base.$op(a::MyNumber) = MyNumber($op(a.x))
end
\end{minted}



\hyperlink{12895501458291832858}{\texttt{@eval}} 重写此调用,使其与上面的较长版本完全等价。为了生成较长的代码块,可以把一个代码块作为表达式参数传给 \hyperlink{12895501458291832858}{\texttt{@eval}}:




\begin{minted}{julia}
@eval begin
    # multiple lines
end
\end{minted}



\hypertarget{7550171062631975520}{}


\section{非标准字符串字面量}



回想一下在\href{@ref non-standard-string-literals}{字符串}的文档中,以标识符为前缀的字符串字面量被称为非标准字符串字面量,它们可以具有与未加前缀的字符串字面量不同的语义。例如:



\begin{itemize}
\item \texttt{r{\textquotedbl}{\textasciicircum}{\textbackslash}s*(?:\#|\$){\textquotedbl}} 生成一个正则表达式对象而不是一个字符串


\item \texttt{b{\textquotedbl}DATA{\textbackslash}xff{\textbackslash}u2200{\textquotedbl}} 是字节数组 \texttt{[68,65,84,65,255,226,136,128]} 的字面量。

\end{itemize}


可能令人惊讶的是,这些行为并没有被硬编码到 Julia 的解释器或编译器中。相反,它们是由一个通用机制实现的自定义行为,且任何人都可以使用该机制:带前缀的字符串字面量被解析为特定名称的宏的调用。例如,正则表达式宏如下:




\begin{minted}{julia}
macro r_str(p)
    Regex(p)
end
\end{minted}



这便是全部代码。这个宏说的是字符串字面量 \texttt{r{\textquotedbl}{\textasciicircum}{\textbackslash}s*(?:\#|\$){\textquotedbl}} 的字面内容应该传给宏 \texttt{@r\_str},并且展开后的结果应当放在该字符串字面量出现处的语法树中。换句话说,表达式 \texttt{r{\textquotedbl}{\textasciicircum}{\textbackslash}s*(?:\#|\$){\textquotedbl}} 等价于直接把下列对象放进语法树中:




\begin{minted}{julia}
Regex("^\\s*(?:#|\$)")
\end{minted}



字符串字面量形式不仅更短、更方便,也更高效:因为正则表达式需要编译,\texttt{Regex} 对象实际上是\emph{在编译代码时}创建的,所以编译只发生一次,而不是每次执行代码时都再编译一次。请考虑如果正则表达式出现在循环中:




\begin{minted}{julia}
for line = lines
    m = match(r"^\s*(?:#|$)", line)
    if m === nothing
        # non-comment
    else
        # comment
    end
end
\end{minted}



因为正则表达式 \texttt{r{\textquotedbl}{\textasciicircum}{\textbackslash}s*(?:\#|\$){\textquotedbl}} 在这段代码解析时便已编译并被插入到语法树中,所以它只编译一次,而不是每次执行循环时都再编译一次。要在不使用宏的情况下实现此效果,必须像这样编写此循环:




\begin{minted}{julia}
re = Regex("^\\s*(?:#|\$)")
for line = lines
    m = match(re, line)
    if m === nothing
        # non-comment
    else
        # comment
    end
end
\end{minted}



此外,如果编译器无法确定在所有循环中正则表达式对象都是常量,可能无法进行某些优化,使得此版本的效率依旧低于上面的更方便的字面量形式。当然,在某些情况下,非字面量形式更方便:如果需要向正则表达式中插入变量,就必须采用这种更冗长的方法;如果正则表达式模式本身是动态的,可能在每次循环迭代时发生变化,就必须在每次迭代中构造新的正则表达式对象。然而,在绝大多数用例中,正则表达式不是基于运行时的数据构造的。在大多数情况下,将正则表达式编写为编译期值的能力是无法估量的。



与非标准字符串字面量一样,非标准命令字面量存在使用命令字面量语法的带前缀变种。命令字面量 \texttt{custom`literal`} 被解析为 \texttt{@custom\_cmd {\textquotedbl}literal{\textquotedbl}}。Julia 本身不包含任何非标准命令字面量,但包可以使用此语法。除了语法不同以及使用 \texttt{\_cmd} 而不是 \texttt{\_str} 后缀,非标准命令字面量的行为与非标准字符串字面量完全相同。



如果两个模块提供了同名的非标准字符串或命令字面量,能使用模块名限定该字符串或命令字面量。例如,如果 \texttt{Foo} 和 \texttt{Bar} 提供了相同的字符串字面量 \texttt{@x\_str},那么可以编写 \texttt{Foo.x{\textquotedbl}literal{\textquotedbl}} 或 \texttt{Bar.x{\textquotedbl}literal{\textquotedbl}} 来消除两者的歧义。



用户定义的字符串字面量的机制十分强大。不仅 Julia 的非标准字面量的实现使用它,而且命令字面量的语法(\texttt{`echo {\textquotedbl}Hello, \$person{\textquotedbl}`})用下面看起来人畜无害的宏实现:




\begin{minted}{julia}
macro cmd(str)
    :(cmd_gen($(shell_parse(str)[1])))
end
\end{minted}



当然,这个宏的定义中使用的函数隐藏了许多复杂性,但它们只是函数且完全用 Julia 编写。你可以阅读它们的源代码并精确地看到它们的行为——它们所做的一切就是构造要插入到你的程序的语法树的表达式对象。



\hypertarget{925665269920917597}{}


\section{生成函数}



有个非常特殊的宏叫 \hyperlink{11479538870805927749}{\texttt{@generated}},它允许你定义所谓的\emph{生成函数}。它们能根据其参数类型生成专用代码,与用多重派发所能实现的代码相比,其代码更灵活和/或少。虽然宏在解析时使用表达式且无法访问其输入值的类型,但是生成函数在参数类型已知时会被展开,但该函数尚未编译。



生成函数的声明不会执行某些计算或操作,而会返回一个被引用的表达式,接着该表达式构成参数类型所对应方法的主体。在调用生成函数时,其返回的表达式会被编译然后执行。为了提高效率,通常会缓存结果。为了能推断是否缓存结果,只能使用语言的受限子集。因此,生成函数提供了一个灵活的方式来将工作重运行时移到编译时,代价则是其构造能力受到更大的限制。



When defining generated functions, there are five main differences to ordinary functions:



\begin{itemize}
\item[1. ] 使用 \texttt{@generated} 标注函数声明。这会向 AST 附加一些信息,让编译器知道这个函数是生成函数。


\item[2. ] 在生成函数的主体中,你只能访问参数的\emph{类型},而不能访问其值,以及在生成函数的定义之前便已定义的任何函数。 not their values.


\item[3. ] 不应计算某些东西或执行某些操作,应返回一个\emph{被引用的}表达式,它会在被求值时执行你想要的操作。


\item[4. ] Generated functions are only permitted to call functions that were defined \emph{before} the definition of the generated function. (Failure to follow this may result in getting \texttt{MethodErrors} referring to functions from a future world-age.)


\item[5. ] 生成函数不能\emph{更改}或\emph{观察}任何非常量的全局状态。(例如,其包括 IO、锁、非局部的字典或者使用 \texttt{hasmethod})即它们只能读取全局常量,且没有任何副作用。换句话说,它们必须是纯函数。由于实现限制,这也意味着它们目前无法定义闭包或生成器。 for example, IO, locks, non-local dictionaries, or using \hyperlink{6562783328134837372}{\texttt{hasmethod}}).

\end{itemize}


举例子来说明这个是最简单的。我们可以将生成函数 \texttt{foo} 声明为




\begin{minted}{jlcon}
julia> @generated function foo(x)
           Core.println(x)
           return :(x * x)
       end
foo (generic function with 1 method)
\end{minted}



请注意,代码主体返回一个被引用的表达式,即 \texttt{:(x * x)},而不仅仅是 \texttt{x * x} 的值。



从调用者的角度看,这与通常的函数等价;实际上,你无需知道你所调用的是通常的函数还是生成函数。让我们看看 \texttt{foo} 的行为:




\begin{minted}{jlcon}
julia> x = foo(2); # note: output is from println() statement in the body
Int64

julia> x           # now we print x
4

julia> y = foo("bar");
String

julia> y
"barbar"
\end{minted}



因此,我们知道在生成函数的主体中,\texttt{x} 是所传递参数的\emph{类型},并且,生成函数的返回值是其定义所返回的被引用的表达式的求值结果,在该表达式求值时 \texttt{x} 表示其\emph{值}。



如果我们使用我们已经使用过的类型再次对 \texttt{foo} 求值会发生什么?




\begin{minted}{jlcon}
julia> foo(4)
16
\end{minted}



请注意,这里并没有打印 \hyperlink{7720564657383125058}{\texttt{Int64}}。我们可以看到对于特定的参数类型集来说,生成函数的主体只执行一次,且结果会被缓存。此后,对于此示例,生成函数首次调用返回的表达式被重新用作方法主体。但是,实际的缓存行为是由实现定义的性能优化,过于依赖此行为并不实际。



生成函数\emph{可能}只生成一次函数,但也\emph{可能}多次生成,或者看起来根本就没有生成过函数。因此,你应该\emph{从不}编写有副作用的生成函数——因为副作用发生的时间和频率是不确定的。(对于宏来说也是如此——跟宏一样,在生成函数中使用 \hyperlink{7507639810592563424}{\texttt{eval}} 也许意味着你正以错误的方式做某事。)但是,与宏不同,运行时系统无法正确处理对 \hyperlink{7507639810592563424}{\texttt{eval}} 的调用,所以不允许这样做。



理解 \texttt{@generated} 函数与方法的重定义间如何相互作用也很重要。遵循正确的 \texttt{@generated} 函数不能观察任何可变状态或导致全局状态的任何更改的原则,我们看到以下行为。观察到,生成函数\emph{不能}调用在生成函数本身的\emph{定义}之前未定义的任何方法。



一开始 \texttt{f(x)} 有一个定义




\begin{minted}{jlcon}
julia> f(x) = "original definition";
\end{minted}



定义使用 \texttt{f(x)} 的其它操作:




\begin{minted}{jlcon}
julia> g(x) = f(x);

julia> @generated gen1(x) = f(x);

julia> @generated gen2(x) = :(f(x));
\end{minted}



我们现在为 \texttt{f(x)} 添加几个新定义:




\begin{minted}{jlcon}
julia> f(x::Int) = "definition for Int";

julia> f(x::Type{Int}) = "definition for Type{Int}";
\end{minted}



并比较这些结果的差异:




\begin{minted}{jlcon}
julia> f(1)
"definition for Int"

julia> g(1)
"definition for Int"

julia> gen1(1)
"original definition"

julia> gen2(1)
"definition for Int"
\end{minted}



生成函数的每个方法都有自己的已定义函数视图:




\begin{minted}{jlcon}
julia> @generated gen1(x::Real) = f(x);

julia> gen1(1)
"definition for Type{Int}"
\end{minted}



上例中的生成函数 \texttt{foo} 能做的,通常的函数 \texttt{foo(x) = x * x} 也能做(除了在第一次调用时打印类型,并产生了更高的开销)。但是,生成函数的强大之处在于其能够根据传递给它的类型计算不同的被引用的表达式:




\begin{minted}{jlcon}
julia> @generated function bar(x)
           if x <: Integer
               return :(x ^ 2)
           else
               return :(x)
           end
       end
bar (generic function with 1 method)

julia> bar(4)
16

julia> bar("baz")
"baz"
\end{minted}



(当然,这个刻意的例子可以更简单地通过多重派发实现······)



滥用它会破坏运行时系统并导致未定义行为:




\begin{minted}{jlcon}
julia> @generated function baz(x)
           if rand() < .9
               return :(x^2)
           else
               return :("boo!")
           end
       end
baz (generic function with 1 method)
\end{minted}



由于生成函数的主体具有不确定性,其行为和\emph{所有后续代码的行为}并未定义。



\emph{不要复制这些例子!}



这些例子有助于说明生成函数定义和调用的工作方式;但是,\emph{不要复制它们},原因如下:



\begin{itemize}
\item \texttt{foo} 函数有副作用(对 \texttt{Core.println} 的调用),并且未确切定义这些副作用发生的时间、频率和次数。


\item \texttt{bar} 函数解决的问题可通过多重派发被更好地解决——定义 \texttt{bar(x) = x} 和 \texttt{bar(x::Integer) = x {\textasciicircum} 2} 会做同样的事,但它更简单和快捷。


\item \texttt{baz} 函数是病态的

\end{itemize}


请注意,不应在生成函数中尝试的操作并无严格限制,且运行时系统现在只能检测一部分无效操作。还有许多操作只会破坏运行时系统而没有通知,通常以微妙的方式而非显然地与错误的定义相关联。因为函数生成器是在类型推导期间运行的,所以它必须遵守该代码的所有限制。



一些不应该尝试的操作包括:



\begin{itemize}
\item[1. ] 缓存本地指针。


\item[2. ] 以任何方式与 \texttt{Core.Compiler} 的内容或方法交互。


\item[3. ] 观察任何可变状态。

\begin{itemize}
\item 生成函数的类型推导可以在\emph{任何}时候运行,包括你的代码正在尝试观察或更改此状态时。

\end{itemize}

\item[4. ] 采用任何锁:你调用的 C 代码可以在内部使用锁(例如,调用 \texttt{malloc} 不会有问题,即使大多数实现在内部需要锁),但是不要试图在执行 Julia 代码时保持或请求任何锁。


\item[5. ] 调用在生成函数的主体后定义的任何函数。对于增量加载的预编译模块,则放宽此条件,以允许调用模块中的任何函数。

\end{itemize}


那好,我们现在已经更好地理解了生成函数的工作方式,让我们使用它来构建一些更高级(和有效)的功能……



\hypertarget{13559476639927770222}{}


\subsection{一个高级的例子}



Julia{\textquotesingle}s base library has an internal \texttt{sub2ind} function to calculate a linear index into an n-dimensional array, based on a set of n multilinear indices - in other words, to calculate the index \texttt{i} that can be used to index into an array \texttt{A} using \texttt{A[i]}, instead of \texttt{A[x,y,z,...]}. One possible implementation is the following:




\begin{minted}{jlcon}
julia> function sub2ind_loop(dims::NTuple{N}, I::Integer...) where N
           ind = I[N] - 1
           for i = N-1:-1:1
               ind = I[i]-1 + dims[i]*ind
           end
           return ind + 1
       end
sub2ind_loop (generic function with 1 method)

julia> sub2ind_loop((3, 5), 1, 2)
4
\end{minted}



用递归可以完成同样的事情:




\begin{minted}{jlcon}
julia> sub2ind_rec(dims::Tuple{}) = 1;

julia> sub2ind_rec(dims::Tuple{}, i1::Integer, I::Integer...) =
           i1 == 1 ? sub2ind_rec(dims, I...) : throw(BoundsError());

julia> sub2ind_rec(dims::Tuple{Integer, Vararg{Integer}}, i1::Integer) = i1;

julia> sub2ind_rec(dims::Tuple{Integer, Vararg{Integer}}, i1::Integer, I::Integer...) =
           i1 + dims[1] * (sub2ind_rec(Base.tail(dims), I...) - 1);

julia> sub2ind_rec((3, 5), 1, 2)
4
\end{minted}



这两种实现虽然不同,但本质上做同样的事情:在数组维度上的运行时循环,将每个维度上的偏移量收集到最后的索引中。



然而,循环所需的信息都已嵌入到参数的类型信息中。因此,我们可以利用生成函数将迭代移动到编译期;用编译器的说法,我们用生成函数手动展开循环。代码主体变得几乎相同,但我们不是计算线性索引,而是建立计算索引的\emph{表达式}:




\begin{minted}{jlcon}
julia> @generated function sub2ind_gen(dims::NTuple{N}, I::Integer...) where N
           ex = :(I[$N] - 1)
           for i = (N - 1):-1:1
               ex = :(I[$i] - 1 + dims[$i] * $ex)
           end
           return :($ex + 1)
       end
sub2ind_gen (generic function with 1 method)

julia> sub2ind_gen((3, 5), 1, 2)
4
\end{minted}



\textbf{这会生成什么代码?}



找出所生成代码的一个简单方法是将生成函数的主体提取到另一个(通常的)函数中:




\begin{minted}{jlcon}
julia> @generated function sub2ind_gen(dims::NTuple{N}, I::Integer...) where N
           return sub2ind_gen_impl(dims, I...)
       end
sub2ind_gen (generic function with 1 method)

julia> function sub2ind_gen_impl(dims::Type{T}, I...) where T <: NTuple{N,Any} where N
           length(I) == N || return :(error("partial indexing is unsupported"))
           ex = :(I[$N] - 1)
           for i = (N - 1):-1:1
               ex = :(I[$i] - 1 + dims[$i] * $ex)
           end
           return :($ex + 1)
       end
sub2ind_gen_impl (generic function with 1 method)
\end{minted}



我们现在可以执行 \texttt{sub2ind\_gen\_impl} 并检查它所返回的表达式:




\begin{minted}{jlcon}
julia> sub2ind_gen_impl(Tuple{Int,Int}, Int, Int)
:(((I[1] - 1) + dims[1] * (I[2] - 1)) + 1)
\end{minted}



因此,这里使用的方法主体根本不包含循环——只有两个元组的索引、乘法和加法/减法。所有循环都是在编译期执行的,我们完全避免了在执行期间的循环。因此,我们只需对每个类型循环\emph{一次},在本例中每个 \texttt{N} 循环一次(除了在该函数被多次生成的边缘情况——请参阅上面的免责声明)。



\hypertarget{2054450996027567500}{}


\subsection{可选地生成函数}



生成函数可以在运行时实现高效率,但需要编译时间成本:必须为具体的参数类型的每个组合生成新的函数体。通常,Julia 能够编译函数的「泛型」版本,其适用于任何参数,但对于生成函数,这是不可能的。这意味着大量使用生成函数的程序可能无法静态编译。



为了解决这个问题,语言提供用于编写生成函数的通常、非生成的替代实现的语法。应用于上面的 \texttt{sub2ind} 示例,它看起来像这样:




\begin{minted}{julia}
function sub2ind_gen(dims::NTuple{N}, I::Integer...) where N
    if N != length(I)
        throw(ArgumentError("Number of dimensions must match number of indices."))
    end
    if @generated
        ex = :(I[$N] - 1)
        for i = (N - 1):-1:1
            ex = :(I[$i] - 1 + dims[$i] * $ex)
        end
        return :($ex + 1)
    else
        ind = I[N] - 1
        for i = (N - 1):-1:1
            ind = I[i] - 1 + dims[i]*ind
        end
        return ind + 1
    end
end
\end{minted}



在内部,这段代码创建了函数的两个实现:一个生成函数的实现,其使用 \texttt{if @generated} 中的第一个块,一个通常的函数的实现,其使用 \texttt{else} 块。在 \texttt{if @generated} 块的 \texttt{then} 部分中,代码与其它生成函数具有相同的语义:参数名称引用类型,且代码应返回表达式。可能会出现多个 \texttt{if @generated} 块,在这种情况下,生成函数的实现使用所有的 \texttt{then} 块,而替代实现使用所有的 \texttt{else} 块。



请注意,我们在函数顶部添加了错误检查。此代码对两个版本都是通用的,且是两个版本中的运行时代码(它将被引用并返回为生成函数版本中的表达式)。这意味着局部变量的值和类型在代码生成时不可用——用于代码生成的代码只能看到参数类型。



在这种定义方式中,代码生成功能本质上只是一种可选的优化。如果方便,编译器将使用它,否则可能选择使用通常的实现。这种方式是首选的,因为它允许编译器做出更多决策和以更多方式编译程序,还因为通常代码比由代码生成的代码更易读。但是,使用哪种实现取决于编译器实现细节,因此,两个实现的行为必须相同。



\hypertarget{12380164357355707963}{}


\chapter{多维数组}



与大多数技术计算语言一样,Julia 提供原生的数组实现。 大多数技术计算语言非常重视其数组实现,但需要付出使用其它容器的代价。Julia 用同样的方式来处理数组。就像和其它用 Julia 写的代码一样,Julia 的数组库几乎完全是用 Julia 自身实现的,它的性能源自编译器。这样一来,用户就可以通过继承 \hyperlink{6514416309183787338}{\texttt{AbstractArray}} 的方式来创建自定义数组类型。 实现自定义数组类型的更多详细信息,请参阅\hyperlink{9718377734213742156}{manual section on the AbstractArray interface}。



数组是存储在多维网格中对象的集合。在最一般的情况下, 数组中的对象可能是 \hyperlink{15014186392807667022}{\texttt{Any}} 类型。 对于大多数计算上的需求,数组中对象的类型应该更加具体,例如 \hyperlink{5027751419500983000}{\texttt{Float64}} 或 \hyperlink{10103694114785108551}{\texttt{Int32}}。



一般来说,与许多其他科学计算语言不同,Julia 不希望为了性能而以向量化的方式编写程序。Julia 的编译器使用类型推断,并为标量数组索引生成优化的代码,从而能够令用户方便地编写可读性良好的程序,而不牺牲性能,并且时常会减少内存使用。



在 Julia 中,所有函数的参数都是 \href{https://en.wikipedia.org/wiki/Evaluation\_strategy\#Call\_by\_sharing}{passed by sharing}。一些科学计算语言用传值的方式传递数组,尽管这样做可以防止数组在被调函数中被意外地篡改,但这也会导致不必要的数组拷贝。通常,以一个 \texttt{!} 结尾的函数名表示它会对自己的一个或者多个参数的值进行修改或者销毁(例如,请比较 \hyperlink{8473525809131227606}{\texttt{sort}} 和 \hyperlink{12296873681374954808}{\texttt{sort!}})。被调函数必须进行显式拷贝,以确保它们不会无意中修改输入参数。很多 “non-mutating” 函数在实现的时候,都会先进行显式拷贝,然后调用一个以 \texttt{!} 结尾的同名函数,最后返回之前拷贝的副本。



\hypertarget{3050591823172658870}{}


\section{基本函数}




\begin{table}[h]

\begin{tabulary}{\linewidth}{|L|L|}
\hline
函数 & 描述 \\
\hline
\hyperlink{6396209842929672718}{\texttt{eltype(A)}} & \texttt{A} 中元素的类型 \\
\hline
\hyperlink{3699181304419743826}{\texttt{length(A)}} & \texttt{A} 中元素的数量 \\
\hline
\hyperlink{1688406579181746010}{\texttt{ndims(A)}} & \texttt{A} 的维数 \\
\hline
\hyperlink{17888996102305087038}{\texttt{size(A)}} & 一个包含 \texttt{A} 各个维度上元素数量的元组 \\
\hline
\hyperlink{17888996102305087038}{\texttt{size(A,n)}} & \texttt{A} 第 \texttt{n} 维中的元素数量 \\
\hline
\hyperlink{7074821531920287868}{\texttt{axes(A)}} & 一个包含 \texttt{A} 有效索引的元组 \\
\hline
\hyperlink{7074821531920287868}{\texttt{axes(A,n)}} & 第 \texttt{n} 维有效索引的范围 \\
\hline
\hyperlink{4701773772897287974}{\texttt{eachindex(A)}} & 一个访问 \texttt{A} 中每一个位置的高效迭代器 \\
\hline
\hyperlink{97811245619734938}{\texttt{stride(A,k)}} & 在第 \texttt{k} 维上的间隔(stride)(相邻元素间的线性索引距离) \\
\hline
\hyperlink{13576557637670855932}{\texttt{strides(A)}} & 包含每一维上的间隔(stride)的元组 \\
\hline
\end{tabulary}

\end{table}



\hypertarget{10907259792659637782}{}


\section{构造和初始化}



Julia 提供了许多用于构造和初始化数组的函数。在下列函数中,参数 \texttt{dims ...} 可以是一个包含维数大小的元组,也可以表示用任意个参数传递的一系列维数大小值。大部分函数的第一个参数都表示数组的元素类型 \texttt{T} 。如果类型 \texttt{T} 被省略,那么将默认为 \hyperlink{5027751419500983000}{\texttt{Float64}}。




\begin{table}[h]

\begin{tabulary}{\linewidth}{|L|L|}
\hline
函数 & 描述 \\
\hline
\hyperlink{15492651498431872487}{\texttt{Array\{T\}(undef, dims...)}} & 一个没有初始化的密集 \hyperlink{15492651498431872487}{\texttt{Array}} \\
\hline
\hyperlink{13837674686090348619}{\texttt{zeros(T, dims...)}} & 一个全零 \texttt{Array} \\
\hline
\hyperlink{5858390260510292771}{\texttt{ones(T, dims...)}} & 一个元素均为 1 的 \texttt{Array} \\
\hline
\hyperlink{12844393578243965152}{\texttt{trues(dims...)}} & 一个每个元素都为 \texttt{true} 的 \hyperlink{18015155802543401629}{\texttt{BitArray}} \\
\hline
\hyperlink{12518029339635756199}{\texttt{falses(dims...)}} & 一个每个元素都为 \texttt{false} 的 \texttt{BitArray} \\
\hline
\hyperlink{3388738163419525310}{\texttt{reshape(A, dims...)}} & 一个包含跟 \texttt{A} 相同数据但维数不同的数组 \\
\hline
\hyperlink{15665284441316555522}{\texttt{copy(A)}} & 拷贝 \texttt{A} \\
\hline
\hyperlink{3259459540194502889}{\texttt{deepcopy(A)}} & 深拷贝,即拷贝 \texttt{A},并递归地拷贝其元素 \\
\hline
\hyperlink{15525808546723795098}{\texttt{similar(A, T, dims...)}} & 一个与\texttt{A}具有相同类型(这里指的是密集,稀疏等)的未初始化数组,但具有指定的元素类型和维数。第二个和第三个参数都是可选的,如果省略则默认为元素类型和 \texttt{A} 的维数。 \\
\hline
\hyperlink{293815781001952115}{\texttt{reinterpret(T, A)}} & 与 \texttt{A} 具有相同二进制数据的数组,但元素类型为 \texttt{T} \\
\hline
\hyperlink{7668863842145012694}{\texttt{rand(T, dims...)}} & 一个随机 \texttt{Array},元素值是  \([0, 1)\)  半开区间中的均匀分布且服从一阶独立同分布 \footnotemark[1] \\
\hline
\hyperlink{7347069443766288058}{\texttt{randn(T, dims...)}} & 一个随机 \texttt{Array},元素为标准正态分布,服从独立同分布 \\
\hline
\hyperlink{5448927444601277512}{\texttt{Matrix\{T\}(I, m, n)}} & \texttt{m}-by-\texttt{n} identity matrix. Requires \texttt{using LinearAlgebra} for \hyperlink{15346645596018210602}{\texttt{I}}. \\
\hline
\hyperlink{737600656772861535}{\texttt{range(start, stop=stop, length=n)}} & 从 \texttt{start} 到 \texttt{stop} 的带有 \texttt{n} 个线性间隔元素的范围 \\
\hline
\hyperlink{5162290739791026948}{\texttt{fill!(A, x)}} & 用值 \texttt{x} 填充数组 \texttt{A} \\
\hline
\hyperlink{2836152204730819918}{\texttt{fill(x, dims...)}} & 一个被值 \texttt{x} 填充的 \texttt{Array} \\
\hline
\end{tabulary}

\end{table}



\footnotetext[1]{\emph{iid},独立同分布

}


要查看各种方法,我们可以将不同维数传递给这些构造函数,请考虑以下示例:




\begin{minted}{jlcon}
julia> zeros(Int8, 2, 3)
2×3 Array{Int8,2}:
 0  0  0
 0  0  0

julia> zeros(Int8, (2, 3))
2×3 Array{Int8,2}:
 0  0  0
 0  0  0

julia> zeros((2, 3))
2×3 Array{Float64,2}:
 0.0  0.0  0.0
 0.0  0.0  0.0
\end{minted}



Here, \texttt{(2, 3)} is a \hyperlink{17462354060312563026}{\texttt{Tuple}} and the first argument — the element type — is optional, defaulting to \texttt{Float64}.



\hypertarget{15443953423472878802}{}


\section{Array literals}



Arrays can also be directly constructed with square braces; the syntax \texttt{[A, B, C, ...]} creates a one dimensional array (i.e., a vector) containing the comma-separated arguments as its elements. The element type (\hyperlink{6396209842929672718}{\texttt{eltype}}) of the resulting array is automatically determined by the types of the arguments inside the braces. If all the arguments are the same type, then that is its \texttt{eltype}. If they all have a common \hyperlink{10374023657104680331}{promotion type} then they get converted to that type using \hyperlink{1846942650946171605}{\texttt{convert}} and that type is the array{\textquotesingle}s \texttt{eltype}. Otherwise, a heterogeneous array that can hold anything — a \texttt{Vector\{Any\}} — is constructed; this includes the literal \texttt{[]} where no arguments are given.




\begin{minted}{jlcon}
julia> [1,2,3] # An array of `Int`s
3-element Array{Int64,1}:
 1
 2
 3

julia> promote(1, 2.3, 4//5) # This combination of Int, Float64 and Rational promotes to Float64
(1.0, 2.3, 0.8)

julia> [1, 2.3, 4//5] # Thus that's the element type of this Array
3-element Array{Float64,1}:
 1.0
 2.3
 0.8

julia> []
Any[]
\end{minted}



\hypertarget{8665822927896221545}{}


\subsection{Concatenation}



If the arguments inside the square brackets are separated by semicolons (\texttt{;}) or newlines instead of commas, then their contents are \emph{vertically concatenated} together instead of the arguments being used as elements themselves.




\begin{minted}{jlcon}
julia> [1:2, 4:5] # Has a comma, so no concatenation occurs. The ranges are themselves the elements
2-element Array{UnitRange{Int64},1}:
 1:2
 4:5

julia> [1:2; 4:5]
4-element Array{Int64,1}:
 1
 2
 4
 5

julia> [1:2
        4:5
        6]
5-element Array{Int64,1}:
 1
 2
 4
 5
 6
\end{minted}



Similarly, if the arguments are separated by tabs or spaces, then their contents are \emph{horizontally concatenated} together.




\begin{minted}{jlcon}
julia> [1:2  4:5  7:8]
2×3 Array{Int64,2}:
 1  4  7
 2  5  8

julia> [[1,2]  [4,5]  [7,8]]
2×3 Array{Int64,2}:
 1  4  7
 2  5  8

julia> [1 2 3] # Numbers can also be horizontally concatenated
1×3 Array{Int64,2}:
 1  2  3
\end{minted}



Using semicolons (or newlines) and spaces (or tabs) can be combined to concatenate both horizontally and vertically at the same time.




\begin{minted}{jlcon}
julia> [1 2
        3 4]
2×2 Array{Int64,2}:
 1  2
 3  4

julia> [zeros(Int, 2, 2) [1; 2]
        [3 4]            5]
3×3 Array{Int64,2}:
 0  0  1
 0  0  2
 3  4  5
\end{minted}



More generally, concatenation can be accomplished through the \hyperlink{9868138443525443234}{\texttt{cat}} function. These syntaxes are shorthands for function calls that themselves are convenience functions:




\begin{table}[h]

\begin{tabulary}{\linewidth}{|L|L|L|}
\hline
语法 & 函数 & 描述 \\
\hline
 & \hyperlink{9868138443525443234}{\texttt{cat}} & 沿着 s 的第 \texttt{k} 维拼接数组 \\
\hline
\texttt{[A; B; C; ...]} & \hyperlink{14691815416955507876}{\texttt{vcat}} & shorthand for `cat(A...; dims=1) \\
\hline
\texttt{[A B C ...]} & \hyperlink{8862791894748483563}{\texttt{hcat}} & shorthand for `cat(A...; dims=2) \\
\hline
\texttt{[A B; C D; ...]} & \hyperlink{16279083053557795116}{\texttt{hvcat}} & simultaneous vertical and horizontal concatenation \\
\hline
\end{tabulary}

\end{table}



\hypertarget{7617457072899643427}{}


\subsection{Typed array literals}



可以用 \texttt{T[A, B, C, ...]} 的方式声明一个元素为某种特定类型的数组。该方法定义一个元素类型为 \texttt{T} 的一维数组并且初始化元素为 \texttt{A}, \texttt{B}, \texttt{C}, ....。比如,\texttt{Any[x, y, z]} 会构建一个异构数组,该数组可以包含任意类型的元素。



类似的,拼接也可以用类型为前缀来指定结果的元素类型。




\begin{minted}{jlcon}
julia> [[1 2] [3 4]]
1×4 Array{Int64,2}:
 1  2  3  4

julia> Int8[[1 2] [3 4]]
1×4 Array{Int8,2}:
 1  2  3  4
\end{minted}



\hypertarget{12661687782855472919}{}


\section{Comprehensions}



(数组)推导提供了构造数组的通用且强大的方法。其语法类似于数学中的集合构造的写法:




\begin{lstlisting}
A = [ F(x,y,...) for x=rx, y=ry, ... ]
\end{lstlisting}



这种形式的含义是 \texttt{F(x,y,...)} 取其给定列表中变量 \texttt{x},\texttt{y} 等的每个值进行计算。值可以指定为任何可迭代对象,但通常是 \texttt{1:n} 或 \texttt{2:(n-1)} 之类的范围,或者像 \texttt{[1.2, 3.4, 5.7]} 这样的显式数组值。结果是一个 N 维密集数组,其维数是变量范围 \texttt{rx},\texttt{ry} 等的维数串联。每次 \texttt{F(x,y,...)} 计算返回一个标量。



下面的示例计算当前元素和沿一维网格其左,右相邻元素的加权平均值:




\begin{minted}{jlcon}
julia> x = rand(8)
8-element Array{Float64,1}:
 0.843025
 0.869052
 0.365105
 0.699456
 0.977653
 0.994953
 0.41084
 0.809411

julia> [ 0.25*x[i-1] + 0.5*x[i] + 0.25*x[i+1] for i=2:length(x)-1 ]
6-element Array{Float64,1}:
 0.736559
 0.57468
 0.685417
 0.912429
 0.8446
 0.656511
\end{minted}



The resulting array type depends on the types of the computed elements just like \hyperlink{13961675686342166416}{array literals} do. In order to control the type explicitly, a type can be prepended to the comprehension. For example, we could have requested the result in single precision by writing:




\begin{minted}{julia}
Float32[ 0.25*x[i-1] + 0.5*x[i] + 0.25*x[i+1] for i=2:length(x)-1 ]
\end{minted}



\hypertarget{5737546333215614116}{}


\section{生成器表达式}



也可以在没有方括号的情况下编写(数组)推导,从而产生称为生成器的对象。可以迭代此对象以按需生成值,而不是预先分配数组并存储它们(请参阅 \hyperlink{13048041929642713791}{迭代})。例如,以下表达式在不分配内存的情况下对一个序列进行求和:




\begin{minted}{jlcon}
julia> sum(1/n^2 for n=1:1000)
1.6439345666815615
\end{minted}



在参数列表中使用具有多个维度的生成器表达式时,需要使用括号将生成器与后续参数分开:




\begin{minted}{jlcon}
julia> map(tuple, 1/(i+j) for i=1:2, j=1:2, [1:4;])
ERROR: syntax: invalid iteration specification
\end{minted}



\texttt{for} 后面所有逗号分隔的表达式都被解释为范围。 添加括号让我们可以向 \hyperlink{11483231213869150535}{\texttt{map}} 中添加第三个参数:




\begin{minted}{jlcon}
julia> map(tuple, (1/(i+j) for i=1:2, j=1:2), [1 3; 2 4])
2×2 Array{Tuple{Float64,Int64},2}:
 (0.5, 1)       (0.333333, 3)
 (0.333333, 2)  (0.25, 4)
\end{minted}



Generators are implemented via inner functions. Just like inner functions used elsewhere in the language, variables from the enclosing scope can be {\textquotedbl}captured{\textquotedbl} in the inner function.  For example, \texttt{sum(p[i] - q[i] for i=1:n)} captures the three variables \texttt{p}, \texttt{q} and \texttt{n} from the enclosing scope. Captured variables can present performance challenges; see \hyperlink{627547588659365489}{performance tips}.



通过编写多个 \texttt{for} 关键字,生成器和推导中的范围可以取决于之前的范围:




\begin{minted}{jlcon}
julia> [(i,j) for i=1:3 for j=1:i]
6-element Array{Tuple{Int64,Int64},1}:
 (1, 1)
 (2, 1)
 (2, 2)
 (3, 1)
 (3, 2)
 (3, 3)
\end{minted}



在这些情况下,结果都是一维的。



可以使用 \texttt{if} 关键字过滤生成的值:




\begin{minted}{jlcon}
julia> [(i,j) for i=1:3 for j=1:i if i+j == 4]
2-element Array{Tuple{Int64,Int64},1}:
 (2, 2)
 (3, 1)
\end{minted}



\hypertarget{14469287548874312017}{}


\section{索引}



索引 n 维数组 \texttt{A} 的一般语法是:




\begin{lstlisting}
X = A[I_1, I_2, ..., I_n]
\end{lstlisting}



其中每个 \texttt{I\_k} 可以是标量整数,整数数组或任何其他\hyperlink{3335763678693018755}{支持的索引类型}。这包括 \hyperlink{13649361117037263099}{\texttt{Colon}} (\texttt{:}) 来选择整个维度中的所有索引,形式为 \texttt{a:c} 或 \texttt{a:b:c} 的范围来选择连续或跨步的子区间,以及布尔数组以选择索引为 \texttt{true} 的元素。



如果所有索引都是标量,则结果 \texttt{X} 是数组 \texttt{A} 中的单个元素。否则,\texttt{X} 是一个数组,其维数与所有索引的维数之和相同。



如果所有索引 \texttt{I\_k} 都是向量,则 \texttt{X} 的形状将是 \texttt{(length(I\_1), length(I\_2), ..., length(I\_n))},其中,\texttt{X} 中位于 \texttt{i\_1, i\_2, ..., i\_n} 处的元素为 \texttt{A[I\_1[i\_1], I\_2[i\_2], ..., I\_n[i\_n]]}。



例如:




\begin{minted}{jlcon}
julia> A = reshape(collect(1:16), (2, 2, 2, 2))
2×2×2×2 Array{Int64,4}:
[:, :, 1, 1] =
 1  3
 2  4

[:, :, 2, 1] =
 5  7
 6  8

[:, :, 1, 2] =
  9  11
 10  12

[:, :, 2, 2] =
 13  15
 14  16

julia> A[1, 2, 1, 1] # all scalar indices
3

julia> A[[1, 2], [1], [1, 2], [1]] # all vector indices
2×1×2×1 Array{Int64,4}:
[:, :, 1, 1] =
 1
 2

[:, :, 2, 1] =
 5
 6

julia> A[[1, 2], [1], [1, 2], 1] # a mix of index types
2×1×2 Array{Int64,3}:
[:, :, 1] =
 1
 2

[:, :, 2] =
 5
 6
\end{minted}



请注意最后两种情况下得到的数组大小为何是不同的。



如果 \texttt{I\_1} 是二维矩阵,则 \texttt{X} 是 \texttt{n+1} 维数组,其形状为 \texttt{(size(I\_1, 1), size(I\_1, 2), length(I\_2), ..., length(I\_n))}。矩阵会添加一个维度。



例如:




\begin{minted}{jlcon}
julia> A = reshape(collect(1:16), (2, 2, 2, 2));

julia> A[[1 2; 1 2]]
2×2 Array{Int64,2}:
 1  2
 1  2

julia> A[[1 2; 1 2], 1, 2, 1]
2×2 Array{Int64,2}:
 5  6
 5  6
\end{minted}



位于 \texttt{i\_1, i\_2, i\_3, ..., i\_\{n+1\}} 处的元素值是 \texttt{A[I\_1[i\_1, i\_2], I\_2[i\_3], ..., I\_n[i\_\{n+1\}]]}。所有使用标量索引的维度都将被丢弃,例如,假设 \texttt{J} 是索引数组,那么 \texttt{A[2,J,3]} 的结果是一个大小为 \texttt{size(J)} 的数组、其第 j 个元素由 \texttt{A[2, J[j], 3]} 填充。



作为此语法的特殊部分,\texttt{end} 关键字可用于表示索引括号内每个维度的最后一个索引,由索引的最内层数组的大小决定。没有 \texttt{end} 关键字的索引语法相当于调用\hyperlink{13720608614876840481}{\texttt{getindex}}:




\begin{lstlisting}
X = getindex(A, I_1, I_2, ..., I_n)
\end{lstlisting}



例如:




\begin{minted}{jlcon}
julia> x = reshape(1:16, 4, 4)
4×4 reshape(::UnitRange{Int64}, 4, 4) with eltype Int64:
 1  5   9  13
 2  6  10  14
 3  7  11  15
 4  8  12  16

julia> x[2:3, 2:end-1]
2×2 Array{Int64,2}:
 6  10
 7  11

julia> x[1, [2 3; 4 1]]
2×2 Array{Int64,2}:
  5  9
 13  1
\end{minted}



\hypertarget{7105044708769418916}{}


\section{Indexed Assignment}



在 n 维数组 \texttt{A} 中赋值的一般语法是:




\begin{lstlisting}
A[I_1, I_2, ..., I_n] = X
\end{lstlisting}



其中每个 \texttt{I\_k} 可以是标量整数,整数数组或任何其他\hyperlink{3335763678693018755}{支持的索引类型}。这包括 \hyperlink{13649361117037263099}{\texttt{Colon}} (\texttt{:}) 来选择整个维度中的所有索引,形式为 \texttt{a:c} 或 \texttt{a:b:c} 的范围来选择连续或跨步的子区间,以及布尔数组以选择索引为 \texttt{true} 的元素。



如果所有 \texttt{I\_k} 都为整数,则数组 \texttt{A} 中 \texttt{I\_1, I\_2, ..., I\_n} 位置的值将被 \texttt{X} 的值覆盖,必要时将 \hyperlink{1846942650946171605}{\texttt{convert}} 为数组 \texttt{A} 的 \hyperlink{6396209842929672718}{\texttt{eltype}}。



如果任一 \texttt{I\_k} 选择了一个以上的位置,则等号右侧的 \texttt{X} 必须为一个与 \texttt{A[I\_1, I\_2, ..., I\_n]} 形状一致的数组或一个具有相同元素数的向量。数组 \texttt{A} 中 \texttt{I\_1[i\_1], I\_2[i\_2], ..., I\_n[i\_n]} 位置的值将被 \texttt{X[I\_1, I\_2, ..., I\_n]} 的值覆盖,必要时会转换类型。逐元素的赋值运算符 \texttt{.=} 可以用于将 \texttt{X} 沿选择的位置 \hyperlink{10888979137852348176}{broadcast}:




\begin{lstlisting}
A[I_1, I_2, ..., I_n] .= X
\end{lstlisting}



就像在\hyperlink{16717190941363337071}{索引}中一样,\texttt{end}关键字可用于表示索引括号中每个维度的最后一个索引,由被赋值的数组大小决定。 没有\texttt{end}关键字的索引赋值语法相当于调用\hyperlink{1309244355901386657}{\texttt{setindex!}}:




\begin{lstlisting}
setindex!(A, X, I_1, I_2, ..., I_n)
\end{lstlisting}



例如:




\begin{minted}{jlcon}
julia> x = collect(reshape(1:9, 3, 3))
3×3 Array{Int64,2}:
 1  4  7
 2  5  8
 3  6  9

julia> x[3, 3] = -9;

julia> x[1:2, 1:2] = [-1 -4; -2 -5];

julia> x
3×3 Array{Int64,2}:
 -1  -4   7
 -2  -5   8
  3   6  -9
\end{minted}



\hypertarget{982887983034702059}{}


\section{支持的索引类型}



在表达式 \texttt{A[I\_1, I\_2, ..., I\_n]} 中,每个 \texttt{I\_k} 可以是标量索引,标量索引数组,或者用 \hyperlink{10027537986402266830}{\texttt{to\_indices}} 转换成的表示标量索引数组的对象:



\begin{itemize}
\item[1. ] 标量索引。默认情况下,这包括:

\begin{itemize}
\item 非布尔的整数


\item \href{@ref}{\href{@ref}{\texttt{CartesianIndex \{N\}}}s,其行为类似于跨越多个维度的 \texttt{N} 维整数元组(详见下文)}s, which behave like an \texttt{N}-tuple of integers spanning multiple dimensions (see below for more details)

\end{itemize}

\item[2. ] 标量索引数组。这包括:

\begin{itemize}
\item 整数向量和多维整数数组


\item 像 \texttt{[]} 这样的空数组,它不选择任何元素


\item 如 \texttt{a:c} 或 \texttt{a:b:c} 的范围,从 \texttt{a} 到 \texttt{c}(包括)选择连续或间隔的部分元素


\item 任何自定义标量索引数组,它是 \texttt{AbstractArray} 的子类型


\item \texttt{CartesianIndex\{N\}} 数组(详见下文)

\end{itemize}

\item[3. ] 一个表示标量索引数组的对象,可以通过\hyperlink{10027537986402266830}{\texttt{to\_indices}}转换为这样的对象。 默认情况下,这包括:

\begin{itemize}
\item \hyperlink{13649361117037263099}{\texttt{Colon()}} (\texttt{:}),表示整个维度内或整个数组中的所有索引


\item 布尔数组,选择其中值为 \texttt{true} 的索引对应的元素(更多细节见下文)

\end{itemize}
\end{itemize}


一些例子:




\begin{minted}{jlcon}
julia> A = reshape(collect(1:2:18), (3, 3))
3×3 Array{Int64,2}:
 1   7  13
 3   9  15
 5  11  17

julia> A[4]
7

julia> A[[2, 5, 8]]
3-element Array{Int64,1}:
  3
  9
 15

julia> A[[1 4; 3 8]]
2×2 Array{Int64,2}:
 1   7
 5  15

julia> A[[]]
Int64[]

julia> A[1:2:5]
3-element Array{Int64,1}:
 1
 5
 9

julia> A[2, :]
3-element Array{Int64,1}:
  3
  9
 15

julia> A[:, 3]
3-element Array{Int64,1}:
 13
 15
 17
\end{minted}



\hypertarget{6884198732360978942}{}


\subsection{笛卡尔索引}



特殊的 \texttt{CartesianIndex\{N\}} 对象表示一个标量索引,其行为类似于张成多个维度的 \texttt{N} 维整数元组。例如:




\begin{minted}{jlcon}
julia> A = reshape(1:32, 4, 4, 2);

julia> A[3, 2, 1]
7

julia> A[CartesianIndex(3, 2, 1)] == A[3, 2, 1] == 7
true
\end{minted}



如果单独考虑,这可能看起来相对微不足道;\texttt{CartesianIndex} 只是将多个整数聚合成一个表示单个多维索引的对象。 但是,当与其他索引形式和迭代器组合产生多个 \texttt{CartesianIndex} 时,这可以生成非常优雅和高效的代码。请参阅下面的\hyperlink{13048041929642713791}{迭代},有关更高级的示例,请参阅\href{https://julialang.org/blog/2016/02/iteration}{关于多维算法和迭代博客文章}。



也支持 \texttt{CartesianIndex \{N\}} 的数组。它们代表一组标量索引,每个索引都跨越 \texttt{N} 个维度,从而实现一种有时也称为逐点索引的索引形式。例如,它可以从上面的 \texttt{A} 的第一「页」访问对角元素:




\begin{minted}{jlcon}
julia> page = A[:,:,1]
4×4 Array{Int64,2}:
 1  5   9  13
 2  6  10  14
 3  7  11  15
 4  8  12  16

julia> page[[CartesianIndex(1,1),
             CartesianIndex(2,2),
             CartesianIndex(3,3),
             CartesianIndex(4,4)]]
4-element Array{Int64,1}:
  1
  6
 11
 16
\end{minted}



这可以通过 \hyperlink{17801130558550430478}{dot broadcasting} 以及普通整数索引(而不是把从 \texttt{A} 中提取第一“页”作为单独的步骤)更加简单地表达。它甚至可以与 \texttt{:} 结合使用,同时从两个页面中提取两个对角线:




\begin{minted}{jlcon}
julia> A[CartesianIndex.(axes(A, 1), axes(A, 2)), 1]
4-element Array{Int64,1}:
  1
  6
 11
 16

julia> A[CartesianIndex.(axes(A, 1), axes(A, 2)), :]
4×2 Array{Int64,2}:
  1  17
  6  22
 11  27
 16  32
\end{minted}



\begin{quote}
\textbf{Warning}

\texttt{CartesianIndex} and arrays of \texttt{CartesianIndex} are not compatible with the \texttt{end} keyword to represent the last index of a dimension. Do not use \texttt{end} in indexing expressions that may contain either \texttt{CartesianIndex} or arrays thereof.

\end{quote}


\hypertarget{5700513412579425383}{}


\subsection{Logical indexing}



Often referred to as logical indexing or indexing with a logical mask, indexing by a boolean array selects elements at the indices where its values are \texttt{true}. Indexing by a boolean vector \texttt{B} is effectively the same as indexing by the vector of integers that is returned by \hyperlink{16067208921941164599}{\texttt{findall(B)}}. Similarly, indexing by a \texttt{N}-dimensional boolean array is effectively the same as indexing by the vector of \texttt{CartesianIndex\{N\}}s where its values are \texttt{true}. A logical index must be a vector of the same length as the dimension it indexes into, or it must be the only index provided and match the size and dimensionality of the array it indexes into. It is generally more efficient to use boolean arrays as indices directly instead of first calling \hyperlink{16067208921941164599}{\texttt{findall}}.




\begin{minted}{jlcon}
julia> x = reshape(1:16, 4, 4)
4×4 reshape(::UnitRange{Int64}, 4, 4) with eltype Int64:
 1  5   9  13
 2  6  10  14
 3  7  11  15
 4  8  12  16

julia> x[[false, true, true, false], :]
2×4 Array{Int64,2}:
 2  6  10  14
 3  7  11  15

julia> mask = map(ispow2, x)
4×4 Array{Bool,2}:
 1  0  0  0
 1  0  0  0
 0  0  0  0
 1  1  0  1

julia> x[mask]
5-element Array{Int64,1}:
  1
  2
  4
  8
 16
\end{minted}



\hypertarget{8886266762373473264}{}


\subsection{Number of indices}



\hypertarget{740108784806621199}{}


\subsubsection{Cartesian indexing}



The ordinary way to index into an \texttt{N}-dimensional array is to use exactly \texttt{N} indices; each index selects the position(s) in its particular dimension. For example, in the three-dimensional array \texttt{A = rand(4, 3, 2)}, \texttt{A[2, 3, 1]} will select the number in the second row of the third column in the first {\textquotedbl}page{\textquotedbl} of the array. This is often referred to as \emph{cartesian indexing}.



\hypertarget{18413909182716267462}{}


\subsubsection{Linear indexing}



When exactly one index \texttt{i} is provided, that index no longer represents a location in a particular dimension of the array. Instead, it selects the \texttt{i}th element using the column-major iteration order that linearly spans the entire array. This is known as \emph{linear indexing}. It essentially treats the array as though it had been reshaped into a one-dimensional vector with \hyperlink{18435874855636770528}{\texttt{vec}}.




\begin{minted}{jlcon}
julia> A = [2 6; 4 7; 3 1]
3×2 Array{Int64,2}:
 2  6
 4  7
 3  1

julia> A[5]
7

julia> vec(A)[5]
7
\end{minted}



A linear index into the array \texttt{A} can be converted to a \texttt{CartesianIndex} for cartesian indexing with \texttt{CartesianIndices(A)[i]} (see \hyperlink{16831958174907250244}{\texttt{CartesianIndices}}), and a set of \texttt{N} cartesian indices can be converted to a linear index with \texttt{LinearIndices(A)[i\_1, i\_2, ..., i\_N]} (see \hyperlink{12250457823889413092}{\texttt{LinearIndices}}).




\begin{minted}{jlcon}
julia> CartesianIndices(A)[5]
CartesianIndex(2, 2)

julia> LinearIndices(A)[2, 2]
5
\end{minted}



It{\textquotesingle}s important to note that there{\textquotesingle}s a very large assymmetry in the performance of these conversions. Converting a linear index to a set of cartesian indices requires dividing and taking the remainder, whereas going the other way is just multiplies and adds. In modern processors, integer division can be 10-50 times slower than multiplication. While some arrays — like \hyperlink{15492651498431872487}{\texttt{Array}} itself — are implemented using a linear chunk of memory and directly use a linear index in their implementations, other arrays — like \hyperlink{3300114559258360989}{\texttt{Diagonal}} — need the full set of cartesian indices to do their lookup (see \hyperlink{7782790551324367092}{\texttt{IndexStyle}} to introspect which is which). As such, when iterating over an entire array, it{\textquotesingle}s much better to iterate over \hyperlink{4701773772897287974}{\texttt{eachindex(A)}} instead of \texttt{1:length(A)}. Not only will the former be much faster in cases where \texttt{A} is \texttt{IndexCartesian}, but it will also support OffsetArrays, too.



\hypertarget{3274469472431833212}{}


\subsubsection{Omitted and extra indices}



In addition to linear indexing, an \texttt{N}-dimensional array may be indexed with fewer or more than \texttt{N} indices in certain situations.



Indices may be omitted if the trailing dimensions that are not indexed into are all length one. In other words, trailing indices can be omitted only if there is only one possible value that those omitted indices could be for an in-bounds indexing expression. For example, a four-dimensional array with size \texttt{(3, 4, 2, 1)} may be indexed with only three indices as the dimension that gets skipped (the fourth dimension) has length one. Note that linear indexing takes precedence over this rule.




\begin{minted}{jlcon}
julia> A = reshape(1:24, 3, 4, 2, 1)
3×4×2×1 reshape(::UnitRange{Int64}, 3, 4, 2, 1) with eltype Int64:
[:, :, 1, 1] =
 1  4  7  10
 2  5  8  11
 3  6  9  12

[:, :, 2, 1] =
 13  16  19  22
 14  17  20  23
 15  18  21  24

julia> A[1, 3, 2] # Omits the fourth dimension (length 1)
19

julia> A[1, 3] # Attempts to omit dimensions 3 & 4 (lengths 2 and 1)
ERROR: BoundsError: attempt to access 3×4×2×1 reshape(::UnitRange{Int64}, 3, 4, 2, 1) with eltype Int64 at index [1, 3]

julia> A[19] # Linear indexing
19
\end{minted}



When omitting \emph{all} indices with \texttt{A[]}, this semantic provides a simple idiom to retrieve the only element in an array and simultaneously ensure that there was only one element.



Similarly, more than \texttt{N} indices may be provided if all the indices beyond the dimensionality of the array are \texttt{1} (or more generally are the first and only element of \texttt{axes(A, d)} where \texttt{d} is that particular dimension number). This allows vectors to be indexed like one-column matrices, for example:




\begin{minted}{jlcon}
julia> A = [8,6,7]
3-element Array{Int64,1}:
 8
 6
 7

julia> A[2,1]
6
\end{minted}



\hypertarget{5063194151918629111}{}


\section{迭代}



迭代整个数组的推荐方法是




\begin{minted}{julia}
for a in A
    # Do something with the element a
end

for i in eachindex(A)
    # Do something with i and/or A[i]
end
\end{minted}



当你需要每个元素的值而不是索引时,使用第一个构造。 在第二个构造中,如果 \texttt{A} 是具有快速线性索引的数组类型,\texttt{i} 将是 \texttt{Int}; 否则,它将是一个 \texttt{CartesianIndex}:




\begin{minted}{jlcon}
julia> A = rand(4,3);

julia> B = view(A, 1:3, 2:3);

julia> for i in eachindex(B)
           @show i
       end
i = CartesianIndex(1, 1)
i = CartesianIndex(2, 1)
i = CartesianIndex(3, 1)
i = CartesianIndex(1, 2)
i = CartesianIndex(2, 2)
i = CartesianIndex(3, 2)
\end{minted}



与 \texttt{for i = 1:length(A)} 相比,\hyperlink{4701773772897287974}{\texttt{eachindex}} 提供了一种迭代任何数组类型的有效方法。



\hypertarget{10166346050354504892}{}


\section{Array traits}



如果你编写一个自定义的 \hyperlink{6514416309183787338}{\texttt{AbstractArray}} 类型,你可以用以下代码指定它使用快速线性索引




\begin{minted}{julia}
Base.IndexStyle(::Type{<:MyArray}) = IndexLinear()
\end{minted}



此设置将导致 \texttt{myArray} 上的 \texttt{eachindex} 迭代使用整数。如果未指定此特征,则使用默认值 \texttt{IndexCartesian()}。



\hypertarget{1179855062476093403}{}


\section{Array and Vectorized Operators and Functions}



以下运算符支持对数组操作



\begin{itemize}
\item[1. ] 一元运算符 – \texttt{-}, \texttt{+}


\item[2. ] 二元运算符 – \texttt{-}, \texttt{+}, \texttt{*}, \texttt{/}, \texttt{{\textbackslash}}, \texttt{{\textasciicircum}}


\item[3. ] 比较操作符 – \texttt{==}, \texttt{!=}, \texttt{≈} (\hyperlink{12499503887608197213}{\texttt{isapprox}}), \texttt{≉}

\end{itemize}


另外,为了便于数学上和其他运算的向量化,Julia \hyperlink{17801130558550430478}{提供了点语法(dot syntax)} \texttt{f.(args...)},例如,\texttt{sin.(x)} 或 \texttt{min.(x,y)},用于数组或数组和标量的混合上的按元素运算(\hyperlink{10888979137852348176}{广播}运算);当与其他点调用(dot call)结合使用时,它们的额外优点是能「融合」到单个循环中,例如,\texttt{sin.(cos.(x))}。



此外,\emph{每个}二元运算符支持相应的\hyperlink{15967322336376951940}{点操作版本},可以应用于此类\hyperlink{17801130558550430478}{融合 broadcasting 操作}的数组(以及数组和标量的组合),例如 \texttt{z .== sin.(x .* y)}。



请注意,类似 \texttt{==} 的比较运算在作用于整个数组时,得到一个布尔结果。使用像 \texttt{.==} 这样的点运算符进行按元素的比较。(对于像 \texttt{<} 这样的比较操作,\emph{只有}按元素运算的版本 \texttt{.<} 适用于数组。)



还要注意 \texttt{max.(a,b)} 和 \hyperlink{14719513931696680717}{\texttt{maximum(a)}} 之间的区别,\texttt{max.(a,b)} 对 \texttt{a} 和 \texttt{b} 的每个元素 \hyperlink{616124539803111168}{\texttt{broadcast}}s \hyperlink{7839419811914289844}{\texttt{max}},\hyperlink{14719513931696680717}{\texttt{maximum(a)}} 寻找在 \texttt{a} 中的最大值。\texttt{min.(a,b)} 和 \texttt{minimum(a)} 也有同样的关系。



\hypertarget{7321350919619265679}{}


\section{广播}



有时需要在不同尺寸的数组上执行元素对元素的操作,例如将矩阵的每一列加一个向量。一种低效的方法是将向量复制成矩阵的大小:




\begin{minted}{jlcon}
julia> a = rand(2,1); A = rand(2,3);

julia> repeat(a,1,3)+A
2×3 Array{Float64,2}:
 1.20813  1.82068  1.25387
 1.56851  1.86401  1.67846
\end{minted}



当维度较大的时候,这种方法将会十分浪费,所以 Julia 提供了广播 \hyperlink{616124539803111168}{\texttt{broadcast}},它将会将参数中低维度的参数扩展,使得其与其他维度匹配,且不会使用额外的内存,并将所给的函数逐元素地应用。




\begin{minted}{jlcon}
julia> broadcast(+, a, A)
2×3 Array{Float64,2}:
 1.20813  1.82068  1.25387
 1.56851  1.86401  1.67846

julia> b = rand(1,2)
1×2 Array{Float64,2}:
 0.867535  0.00457906

julia> broadcast(+, a, b)
2×2 Array{Float64,2}:
 1.71056  0.847604
 1.73659  0.873631
\end{minted}



\hyperlink{15967322336376951940}{Dotted operators} such as \texttt{.+} and \texttt{.*} are equivalent to \texttt{broadcast} calls (except that they fuse, as \hyperlink{4802910107640435151}{described above}). There is also a \hyperlink{7631985657411687574}{\texttt{broadcast!}} function to specify an explicit destination (which can also be accessed in a fusing fashion by \texttt{.=} assignment). In fact, \texttt{f.(args...)} is equivalent to \texttt{broadcast(f, args...)}, providing a convenient syntax to broadcast any function (\hyperlink{17801130558550430478}{dot syntax}). Nested {\textquotedbl}dot calls{\textquotedbl} \texttt{f.(...)} (including calls to \texttt{.+} etcetera) \hyperlink{15967322336376951940}{automatically fuse} into a single \texttt{broadcast} call.



Additionally, \hyperlink{616124539803111168}{\texttt{broadcast}} is not limited to arrays (see the function documentation); it also handles scalars, tuples and other collections.  By default, only some argument types are considered scalars, including (but not limited to) \texttt{Number}s, \texttt{String}s, \texttt{Symbol}s, \texttt{Type}s, \texttt{Function}s and some common singletons like \texttt{missing} and \texttt{nothing}. All other arguments are iterated over or indexed into elementwise.




\begin{minted}{jlcon}
julia> convert.(Float32, [1, 2])
2-element Array{Float32,1}:
 1.0
 2.0

julia> ceil.(UInt8, [1.2 3.4; 5.6 6.7])
2×2 Array{UInt8,2}:
 0x02  0x04
 0x06  0x07

julia> string.(1:3, ". ", ["First", "Second", "Third"])
3-element Array{String,1}:
 "1. First"
 "2. Second"
 "3. Third"
\end{minted}



Sometimes, you want a container (like an array) that would normally participate in broadcast to be {\textquotedbl}protected{\textquotedbl} from broadcast{\textquotesingle}s behavior of iterating over all of its elements. By placing it inside another container (like a single element \hyperlink{17462354060312563026}{\texttt{Tuple}}) broadcast will treat it as a single value.




\begin{minted}{jlcon}
julia> ([1, 2, 3], [4, 5, 6]) .+ ([1, 2, 3],)
([2, 4, 6], [5, 7, 9])

julia> ([1, 2, 3], [4, 5, 6]) .+ tuple([1, 2, 3])
([2, 4, 6], [5, 7, 9])
\end{minted}



\hypertarget{2709595058891761459}{}


\section{实现}



Julia 中的基本数组类型是抽象类型 \hyperlink{6514416309183787338}{\texttt{AbstractArray\{T,N\}}}。它通过维数 \texttt{N} 和元素类型 \texttt{T} 进行参数化。\hyperlink{12517057979818647811}{\texttt{AbstractVector}} 和 \hyperlink{17966587371929951201}{\texttt{AbstractMatrix}} 是一维和二维情况下的别名。\texttt{AbstractArray} 对象的操作是使用更高级别的运算符和函数定义的,其方式独立于底层存储。这些操作可以正确地被用于任何特定数组实现的回退操作。



\texttt{AbstractArray} 类型包含任何模糊类似的东西,它的实现可能与传统数组完全不同。例如,可以根据请求而不是存储来计算元素。但是,任何具体的 \texttt{AbstractArray\{T,N\}} 类型通常应该至少实现 \hyperlink{17888996102305087038}{\texttt{size(A)}}(返回 \texttt{Int} 元组),\hyperlink{2839226020402435013}{\texttt{getindex(A,i)}} 和 \hyperlink{13720608614876840481}{\texttt{getindex(A,i1,...,iN)}};可变数组也应该实现 \hyperlink{1309244355901386657}{\texttt{setindex!}}。建议这些操作具有几乎为常数的时间复杂性,或严格说来 Õ(1) 复杂性,否则某些数组函数可能出乎意料的慢。具体类型通常还应提供 \hyperlink{15525808546723795098}{\texttt{similar(A,T=eltype(A),dims=size(A))}} 方法,用于为 \hyperlink{15665284441316555522}{\texttt{copy}} 分配类似的数组和其他位于当前数组空间外的操作。无论在内部如何表示 \texttt{AbstractArray\{T,N\}},\texttt{T} 是由 \emph{整数} 索引返回的对象类型(\texttt{A[1, ..., 1]},当 \texttt{A} 不为空),\texttt{N} 应该是 \hyperlink{17888996102305087038}{\texttt{size}} 返回的元组的长度。有关定义自定义 \texttt{AbstractArray} 实现的更多详细信息,请参阅\hyperlink{9718377734213742156}{接口章节中的数组接口导则}。



\texttt{DenseArray} is an abstract subtype of \texttt{AbstractArray} intended to include all arrays where elements are stored contiguously in column-major order (see \hyperlink{11239800376478112527}{additional notes in Performance Tips}). The \hyperlink{15492651498431872487}{\texttt{Array}} type is a specific instance of \texttt{DenseArray};  \hyperlink{10571362059486397014}{\texttt{Vector}} and \hyperlink{5448927444601277512}{\texttt{Matrix}} are aliases for the 1-d and 2-d cases. Very few operations are implemented specifically for \texttt{Array} beyond those that are required for all \texttt{AbstractArray}s; much of the array library is implemented in a generic manner that allows all custom arrays to behave similarly.



\texttt{SubArray} 是 \texttt{AbstractArray} 的特例,它通过与原始数组共享内存而不是复制它来执行索引。 使用\hyperlink{4861450464669906845}{\texttt{view}} 函数创建 \texttt{SubArray},它的调用方式与\hyperlink{13720608614876840481}{\texttt{getindex}} 相同(作用于数组和一系列索引参数)。 \hyperlink{4861450464669906845}{\texttt{view}} 的结果看起来与 \hyperlink{13720608614876840481}{\texttt{getindex}} 的结果相同,只是数据保持不变。 \hyperlink{4861450464669906845}{\texttt{view}} 将输入索引向量存储在 \texttt{SubArray} 对象中,该对象稍后可用于间接索引原始数组。 通过将  \hyperlink{4544474300423667148}{\texttt{@views}} 宏放在表达式或代码块之前,该表达式中的任何 \texttt{array [...]} 切片将被转换为创建一个 \texttt{SubArray} 视图。



\hyperlink{18015155802543401629}{\texttt{BitArray}} 是节省空间“压缩”的布尔数组,每个比特(bit)存储一个布尔值。 它们可以类似于 \texttt{Array\{Bool\}} 数组(每个字节(byte)存储一个布尔值),并且可以分别通过 \texttt{Array(bitarray)} 和 \texttt{BitArray(array)} 相互转换。



An array is {\textquotedbl}strided{\textquotedbl} if it is stored in memory with well-defined spacings (strides) between its elements. A strided array with a supported element type may be passed to an external (non-Julia) library like BLAS or LAPACK by simply passing its \hyperlink{8901246211940014300}{\texttt{pointer}} and the stride for each dimension. The \hyperlink{97811245619734938}{\texttt{stride(A, d)}} is the distance between elements along dimension \texttt{d}. For example, the builtin \texttt{Array} returned by \texttt{rand(5,7,2)} has its elements arranged contiguously in column major order. This means that the stride of the first dimension — the spacing between elements in the same column — is \texttt{1}:




\begin{minted}{jlcon}
julia> A = rand(5,7,2);

julia> stride(A,1)
1
\end{minted}



The stride of the second dimension is the spacing between elements in the same row, skipping as many elements as there are in a single column (\texttt{5}). Similarly, jumping between the two {\textquotedbl}pages{\textquotedbl} (in the third dimension) requires skipping \texttt{5*7 == 35} elements.  The \hyperlink{13576557637670855932}{\texttt{strides}} of this array is the tuple of these three numbers together:




\begin{minted}{jlcon}
julia> strides(A)
(1, 5, 35)
\end{minted}



In this particular case, the number of elements skipped \emph{in memory} matches the number of \emph{linear indices} skipped. This is only the case for contiguous arrays like \texttt{Array} (and other \texttt{DenseArray} subtypes) and is not true in general. Views with range indices are a good example of \emph{non-contiguous} strided arrays; consider \texttt{V = @view A[1:3:4, 2:2:6, 2:-1:1]}. This view \texttt{V} refers to the same memory as \texttt{A} but is skipping and re-arranging some of its elements. The stride of the first dimension of \texttt{V} is \texttt{3} because we{\textquotesingle}re only selecting every third row from our original array:




\begin{minted}{jlcon}
julia> V = @view A[1:3:4, 2:2:6, 2:-1:1];

julia> stride(V, 1)
3
\end{minted}



This view is similarly selecting every other column from our original \texttt{A} — and thus it needs to skip the equivalent of two five-element columns when moving between indices in the second dimension:




\begin{minted}{jlcon}
julia> stride(V, 2)
10
\end{minted}



The third dimension is interesting because its order is reversed! Thus to get from the first {\textquotedbl}page{\textquotedbl} to the second one it must go \emph{backwards} in memory, and so its stride in this dimension is negative!




\begin{minted}{jlcon}
julia> stride(V, 3)
-35
\end{minted}



This means that the \texttt{pointer} for \texttt{V} is actually pointing into the middle of \texttt{A}{\textquotesingle}s memory block, and it refers to elements both backwards and forwards in memory. See the \hyperlink{3010450308855105276}{interface guide for strided arrays} for more details on defining your own strided arrays. \hyperlink{18350706206094827862}{\texttt{StridedVector}} and \hyperlink{3855703768476610836}{\texttt{StridedMatrix}} are convenient aliases for many of the builtin array types that are considered strided arrays, allowing them to dispatch to select specialized implementations that call highly tuned and optimized BLAS and LAPACK functions using just the pointer and strides.



It is worth emphasizing that strides are about offsets in memory rather than indexing. If you are looking to convert between linear (single-index) indexing and cartesian (multi-index) indexing, see \hyperlink{12250457823889413092}{\texttt{LinearIndices}} and \hyperlink{16831958174907250244}{\texttt{CartesianIndices}}.



\hypertarget{16922701023279192482}{}


\chapter{缺失值}



Julia 支持表示统计意义上的缺失值,即某个变量在观察中没有可用值,但在理论上存在有效值的情况。缺失值由 \hyperlink{14596725676261444434}{\texttt{missing}} 对象表示,该对象是 \hyperlink{9306488559141158579}{\texttt{Missing}} 类型的唯一实例。\texttt{missing} 等价于 \href{https://en.wikipedia.org/wiki/NULL\_(SQL)}{SQL 中的 \texttt{NULL}} 以及 \href{https://cran.r-project.org/doc/manuals/r-release/R-lang.html\#NA-handling}{R 中的 \texttt{NA}},并在大多数情况下表现得与它们一样。



\hypertarget{16983381017967078050}{}


\section{缺失值的传播}



\texttt{missing} values \emph{propagate} automatically when passed to standard mathematical operators and functions. For these functions, uncertainty about the value of one of the operands induces uncertainty about the result. In practice, this means a math operation involving a \texttt{missing} value generally returns \texttt{missing}




\begin{minted}{jlcon}
julia> missing + 1
missing

julia> "a" * missing
missing

julia> abs(missing)
missing
\end{minted}



As \texttt{missing} is a normal Julia object, this propagation rule only works for functions which have opted in to implement this behavior. This can be achieved either via a specific method defined for arguments of type \texttt{Missing}, or simply by accepting arguments of this type, and passing them to functions which propagate them (like standard math operators). Packages should consider whether it makes sense to propagate missing values when defining new functions, and define methods appropriately if that is the case. Passing a \texttt{missing} value to a function for which no method accepting arguments of type \texttt{Missing} is defined throws a \hyperlink{68769522931907606}{\texttt{MethodError}}, just like for any other type.



Functions that do not propagate \texttt{missing} values can be made to do so by wrapping them in the \texttt{passmissing} function provided by the \href{https://github.com/JuliaData/Missings.jl}{Missings.jl} package. For example, \texttt{f(x)} becomes \texttt{passmissing(f)(x)}.



\hypertarget{13926030736377467708}{}


\section{相等和比较运算符}



标准相等和比较运算符遵循上面给出的传播规则:如果任何操作数是 \texttt{missing},那么结果是 \texttt{missing}。这是一些例子




\begin{minted}{jlcon}
julia> missing == 1
missing

julia> missing == missing
missing

julia> missing < 1
missing

julia> 2 >= missing
missing
\end{minted}



特别要注意,\texttt{missing == missing} 返回 \texttt{missing},所以 \texttt{==} 不能用于测试值是否为缺失值。要测试 \texttt{x} 是否为 \texttt{missing},请用 \hyperlink{3452327148507948899}{\texttt{ismissing(x)}}。



特殊的比较运算符 \hyperlink{269533589463185031}{\texttt{isequal}} 和 \hyperlink{7974744969331231272}{\texttt{===}} 是传播规则的例外:它们总返回一个 \texttt{Bool} 值,即使存在 \texttt{missing} 值,并认为 \texttt{missing} 与 \texttt{missing} 相等且其与任何其它值不同。因此,它们可用于测试某个值是否为 \texttt{missing}。




\begin{minted}{jlcon}
julia> missing === 1
false

julia> isequal(missing, 1)
false

julia> missing === missing
true

julia> isequal(missing, missing)
true
\end{minted}



\hyperlink{8062916604071842790}{\texttt{isless}} 运算符是另一个例外:\texttt{missing} 被认为比任何其它值大。此运算符被用于 \hyperlink{8473525809131227606}{\texttt{sort}},因此 \texttt{missing} 值被放置在所有其它值之后。




\begin{minted}{jlcon}
julia> isless(1, missing)
true

julia> isless(missing, Inf)
false

julia> isless(missing, missing)
false
\end{minted}



\hypertarget{2677689633086417735}{}


\section{逻辑运算符}



逻辑(或布尔)运算符 \hyperlink{9633687763646488853}{\texttt{|}}、\hyperlink{1494761116451616317}{\texttt{\&}} 和 \hyperlink{7071880015536674935}{\texttt{xor}} 是另一种特殊情况,因为它们只有在逻辑上是必需的时传递 \texttt{missing} 值。对于这些运算符来说,结果是否不确定取决于具体操作,其遵循\href{https://en.wikipedia.org/wiki/Three-valued\_logic}{\emph{三值逻辑}}的既定规则,这些规则也由 SQL 中的 \texttt{NULL} 以及 R 中的 \texttt{NA} 实现。这个抽象的定义实际上对应于一系列相对自然的行为,这最好通过具体的例子来解释。



让我们用逻辑「或」运算符 \hyperlink{9633687763646488853}{\texttt{|}} 来说明这个原理。按照布尔逻辑的规则,如果其中一个操作数是 \texttt{true},则另一个操作数对结果没影响,结果总是 \texttt{true}。




\begin{minted}{jlcon}
julia> true | true
true

julia> true | false
true

julia> false | true
true
\end{minted}



基于观察,我们可以得出结论,如果其中一个操作数是 \texttt{true} 而另一个是 \texttt{missing},我们知道结果为 \texttt{true},尽管另一个参数的实际值存在不确定性。如果我们能观察到第二个操作数的实际值,那么它只能是 \texttt{true} 或 \texttt{false},在两种情况下结果都是 \texttt{true}。因此,在这种特殊情况下,值的缺失不会传播




\begin{minted}{jlcon}
julia> true | missing
true

julia> missing | true
true
\end{minted}



相反地,如果其中一个操作数是 \texttt{false},结果可能是 \texttt{true} 或 \texttt{false},这取决于另一个操作数的值。因此,如果一个操作数是 \texttt{missing},那么结果也是 \texttt{missing}。




\begin{minted}{jlcon}
julia> false | true
true

julia> true | false
true

julia> false | false
false

julia> false | missing
missing

julia> missing | false
missing
\end{minted}



逻辑「且」运算符 \hyperlink{1494761116451616317}{\texttt{\&}} 的行为与 \texttt{|} 运算符相似,区别在于当其中一个操作数为 \texttt{false} 时,值的缺失不会传播。例如,当第一个操作数是 \texttt{false} 时




\begin{minted}{jlcon}
julia> false & false
false

julia> false & true
false

julia> false & missing
false
\end{minted}



另一方面,当其中一个操作数为 \texttt{true} 时,值的缺失会传播,例如,当第一个操作数是 \texttt{true} 时




\begin{minted}{jlcon}
julia> true & true
true

julia> true & false
false

julia> true & missing
missing
\end{minted}



最后,逻辑「异或」运算符 \hyperlink{7071880015536674935}{\texttt{xor}} 总传播 \texttt{missing} 值,因为两个操作数都总是对结果产生影响。还要注意,否定运算符 \hyperlink{4329035214952292986}{\texttt{!}} 在操作数是 \texttt{missing} 时返回 \texttt{missing},这就像其它一元运算符。



\hypertarget{354241034752728129}{}


\section{流程控制和短路运算符}



流程控制操作符,包括 \hyperlink{11624168233949720742}{\texttt{if}}、\hyperlink{15133348314455964692}{\texttt{while}} 和\hyperlink{14451148373001501733}{三元运算符} \texttt{x ? y : z},不允许缺失值。这是因为如果我们能够观察实际值,它是 \texttt{true} 还是 \texttt{false} 是不确定的,这意味着我们不知道程序应该如何运行。一旦在以下上下文中遇到 \texttt{missing} 值,就会抛出 \hyperlink{2622693721821893139}{\texttt{TypeError}}




\begin{minted}{jlcon}
julia> if missing
           println("here")
       end
ERROR: TypeError: non-boolean (Missing) used in boolean context
\end{minted}



出于同样的原因,并与上面给出的逻辑运算符相反,短路布尔运算符 \hyperlink{4714012140247170866}{\texttt{\&\&}} 和 \hyperlink{2053797086840563251}{\texttt{||}} 在当前操作数的值决定下一个操作数是否求值时不允许 \texttt{missing} 值。例如




\begin{minted}{jlcon}
julia> missing || false
ERROR: TypeError: non-boolean (Missing) used in boolean context

julia> missing && false
ERROR: TypeError: non-boolean (Missing) used in boolean context

julia> true && missing && false
ERROR: TypeError: non-boolean (Missing) used in boolean context
\end{minted}



另一方面,如果无需 \texttt{missing} 值即可确定结果,则不会引发错误。代码在对 \texttt{missing} 操作数求值前短路,以及 \texttt{missing} 是最后一个操作数都是这种情况。




\begin{minted}{jlcon}
julia> true && missing
missing

julia> false && missing
false
\end{minted}



\hypertarget{929400294062348079}{}


\section{包含缺失值的数组}



包含缺失值的数组的创建就像其它数组




\begin{minted}{jlcon}
julia> [1, missing]
2-element Array{Union{Missing, Int64},1}:
 1
  missing
\end{minted}



如此示例所示,此类数组的元素类型为 \texttt{Union\{Missing, T\}},其中 \texttt{T} 为非缺失值的类型。这简单地反映了以下事实:数组条目可以具有类型 \texttt{T}(在这是 \texttt{Int64})或类型 \texttt{Missing}。此类数组使用高效的内存存储,其等价于一个 \texttt{Array\{T\}} 组合一个 \texttt{Array\{UInt8\}},前者保存实际值,后者表示条目类型(即它是 \texttt{Missing} 还是 \texttt{T})。



允许缺失值的数组可以使用标准语法构造。使用 \texttt{Array\{Union\{Missing, T\}\}(missing, dims)} 来创建填充缺失值的数组:




\begin{minted}{jlcon}
julia> Array{Union{Missing, String}}(missing, 2, 3)
2×3 Array{Union{Missing, String},2}:
 missing  missing  missing
 missing  missing  missing
\end{minted}



允许但不包含 \texttt{missing} 值的数组可使用 \hyperlink{1846942650946171605}{\texttt{convert}} 转换回不允许缺失值的数组。如果该数组包含 \texttt{missing} 值,在类型转换时会抛出 \texttt{MethodError}




\begin{minted}{jlcon}
julia> x = Union{Missing, String}["a", "b"]
2-element Array{Union{Missing, String},1}:
 "a"
 "b"

julia> convert(Array{String}, x)
2-element Array{String,1}:
 "a"
 "b"

julia> y = Union{Missing, String}[missing, "b"]
2-element Array{Union{Missing, String},1}:
 missing
 "b"

julia> convert(Array{String}, y)
ERROR: MethodError: Cannot `convert` an object of type Missing to an object of type String
\end{minted}



\hypertarget{12164012210983849465}{}


\section{跳过缺失值}



由于 \texttt{missing} 会随着标准数学运算符传播,归约函数会在调用的数组包含缺失值时返回 \texttt{missing}




\begin{minted}{jlcon}
julia> sum([1, missing])
missing
\end{minted}



在这种情况下,使用 \hyperlink{2012470681884771400}{\texttt{skipmissing}} 即可跳过缺失值




\begin{minted}{jlcon}
julia> sum(skipmissing([1, missing]))
1
\end{minted}



This convenience function returns an iterator which filters out \texttt{missing} values efficiently. It can therefore be used with any function which supports iterators




\begin{minted}{jlcon}
julia> x = skipmissing([3, missing, 2, 1])
skipmissing(Union{Missing, Int64}[3, missing, 2, 1])

julia> maximum(x)
3

julia> mean(x)
2.0

julia> mapreduce(sqrt, +, x)
4.146264369941973
\end{minted}



Objects created by calling \texttt{skipmissing} on an array can be indexed using indices from the parent array. Indices corresponding to missing values are not valid for these objects and an error is thrown when trying to use them (they are also skipped by \texttt{keys} and \texttt{eachindex})




\begin{minted}{jlcon}
julia> x[1]
3

julia> x[2]
ERROR: MissingException: the value at index (2,) is missing
[...]
\end{minted}



This allows functions which operate on indices to work in combination with \texttt{skipmissing}. This is notably the case for search and find functions, which return indices valid for the object returned by \texttt{skipmissing} which are also the indices of the matching entries \emph{in the parent array}




\begin{minted}{jlcon}
julia> findall(==(1), x)
1-element Array{Int64,1}:
 4

julia> findfirst(!iszero, x)
1

julia> argmax(x)
1
\end{minted}



Use \hyperlink{6278865767444641812}{\texttt{collect}} to extract non-\texttt{missing} values and store them in an array




\begin{minted}{jlcon}
julia> collect(x)
3-element Array{Int64,1}:
 3
 2
 1
\end{minted}



\hypertarget{15378768208982642165}{}


\section{数组上的逻辑运算}



上面描述的逻辑运算符的三值逻辑也适用于针对数组的函数。因此,使用 \hyperlink{15143149452920304570}{\texttt{==}} 运算符的数组相等性测试中,若在未知 \texttt{missing} 条目实际值时无法确定结果,就返回 \texttt{missing}。在实际应用中意味着,在待比较数组中所有非缺失值都相等,且某个或全部数组包含缺失值(也许在不同位置)时会返回 \texttt{missing}。




\begin{minted}{jlcon}
julia> [1, missing] == [2, missing]
false

julia> [1, missing] == [1, missing]
missing

julia> [1, 2, missing] == [1, missing, 2]
missing
\end{minted}



对于单个值,\hyperlink{269533589463185031}{\texttt{isequal}} 会将 \texttt{missing} 值视为与其它 \texttt{missing} 值相等但与非缺失值不同。




\begin{minted}{jlcon}
julia> isequal([1, missing], [1, missing])
true

julia> isequal([1, 2, missing], [1, missing, 2])
false
\end{minted}



函数 \hyperlink{14612039032155203548}{\texttt{any}} 和 \hyperlink{7942004983516218646}{\texttt{all}} 遵循三值逻辑的规则,会在结果无法被确定时返回 \texttt{missing}。




\begin{minted}{jlcon}
julia> all([true, missing])
missing

julia> all([false, missing])
false

julia> any([true, missing])
true

julia> any([false, missing])
missing
\end{minted}



\hypertarget{15469988008975030780}{}


\chapter{网络和流}



Julia 提供了一个功能丰富的接口来处理流式 I/O 对象,如终端、管道和 TCP 套接字。此接口虽然在系统级是异步的,但是其以同步的方式展现给程序员,通常也不需要考虑底层的异步操作。这是通过大量使用 Julia 协作线程(\hyperlink{17473131347184639576}{协程})功能实现的。



\hypertarget{16725310763095786332}{}


\section{基础流 I/O}



所有 Julia stream 都暴露了 \hyperlink{8104134490906192097}{\texttt{read}} 和 \hyperlink{16947913578760238729}{\texttt{write}} 方法,将 stream 作为它们的第一个参数,如:




\begin{minted}{jlcon}
julia> write(stdout, "Hello World");  # suppress return value 11 with ;
Hello World
julia> read(stdin, Char)

'\n': ASCII/Unicode U+000a (category Cc: Other, control)
\end{minted}



注意,\hyperlink{16947913578760238729}{\texttt{write}} 返回 11,字节数(\texttt{{\textquotedbl}Hello World{\textquotedbl}})写入 \hyperlink{18181294266083891471}{\texttt{stdout}},但是返回值使用 \texttt{;} 抑制。



这里按了两次回车,以便 Julia 能够读取到换行符。正如你在这个例子中所看到的,\hyperlink{16947913578760238729}{\texttt{write}} 以待写入的数据作为其第二个参数,而 \hyperlink{8104134490906192097}{\texttt{read}} 以待读取的数据的类型作为其第二个参数。



例如,为了读取一个简单的字节数组,我们可以这样做:




\begin{minted}{jlcon}
julia> x = zeros(UInt8, 4)
4-element Array{UInt8,1}:
 0x00
 0x00
 0x00
 0x00

julia> read!(stdin, x)
abcd
4-element Array{UInt8,1}:
 0x61
 0x62
 0x63
 0x64
\end{minted}



但是,因为这有些繁琐,所以提供了几个方便的方法。例如,我们可以把上面的代码编写为:




\begin{minted}{jlcon}
julia> read(stdin, 4)
abcd
4-element Array{UInt8,1}:
 0x61
 0x62
 0x63
 0x64
\end{minted}



或者如果我们想要读取一整行:




\begin{minted}{jlcon}
julia> readline(stdin)
abcd
"abcd"
\end{minted}



请注意,根据你的终端设置,你的 TTY 可能是行缓冲的,因此在数据发送给 Julia 前可能需要额外的回车。



若要读取 \hyperlink{3330957653919693521}{\texttt{stdin}} 的每一行,可以使用 \hyperlink{3474649815265066504}{\texttt{eachline}}:




\begin{minted}{julia}
for line in eachline(stdin)
    print("Found $line")
end
\end{minted}



或者如果你想要按字符读取的话,使用 \hyperlink{8104134490906192097}{\texttt{read}} :




\begin{minted}{julia}
while !eof(stdin)
    x = read(stdin, Char)
    println("Found: $x")
end
\end{minted}



\hypertarget{14803809014545196748}{}


\section{文本 I/O}



请注意,上面提到的 \hyperlink{16947913578760238729}{\texttt{write}} 方法对二进制流进行操作。具体来说,值不会转换为任何规范的文本表示形式,而是按原样输出:




\begin{minted}{jlcon}
julia> write(stdout, 0x61);  # suppress return value 1 with ;
a
\end{minted}



请注意,\texttt{a} 被 \hyperlink{16947913578760238729}{\texttt{write}} 函数写入到 \hyperlink{18181294266083891471}{\texttt{stdout}} 并且返回值为 \texttt{1}(因为 \texttt{0x61} 为一个字节)。



对于文本 I/O,请根据需要使用 \hyperlink{8248717042415202230}{\texttt{print}} 或 \hyperlink{14071376285304310153}{\texttt{show}} 方法(有关这两个方法之间的差异的详细讨论,请参阅它们的文档):




\begin{minted}{jlcon}
julia> print(stdout, 0x61)
97
\end{minted}



有关如何实现自定义类型的显示方法的更多信息,请参阅 \hyperlink{5246022684399876238}{自定义 pretty-printing}。



\hypertarget{11059597754503252223}{}


\section{IO 输出的上下文信息}



有时,IO 输出可受益于将上下文信息传递到 show 方法的能力。\hyperlink{13454403377667762339}{\texttt{IOContext}} 对象提供了将任意元数据与 IO 对象相关联的框架。例如,\texttt{:compact => true} 向 IO 对象添加一个参数来提示调用的 show 方法应该打印一个较短的输出(如果适用)。有关常用属性的列表,请参阅 \hyperlink{13454403377667762339}{\texttt{IOContext}} 文档。



\hypertarget{17626527008259433393}{}


\section{使用文件}



Like many other environments, Julia has an \hyperlink{300818094931158296}{\texttt{open}} function, which takes a filename and returns an \hyperlink{12496894737220238417}{\texttt{IOStream}} object that you can use to read and write things from the file. For example, if we have a file, \texttt{hello.txt}, whose contents are \texttt{Hello, World!}:




\begin{minted}{jlcon}
julia> f = open("hello.txt")
IOStream(<file hello.txt>)

julia> readlines(f)
1-element Array{String,1}:
 "Hello, World!"
\end{minted}



若要写入文件,则可以带着 write(\texttt{{\textquotedbl}w{\textquotedbl}})标志来打开它:




\begin{minted}{jlcon}
julia> f = open("hello.txt","w")
IOStream(<file hello.txt>)

julia> write(f,"Hello again.")
12
\end{minted}



你如果在此刻检查 \texttt{hello.txt} 的内容,会注意到它是空的;改动实际上还没有写入到磁盘中。这是因为 \texttt{IOStream} 必须在写入实际刷新到磁盘前关闭:




\begin{minted}{jlcon}
julia> close(f)
\end{minted}



再次检查 \texttt{hello.txt} 将显示其内容已被更改。



打开文件,对其内容执行一些操作,并再次关闭它是一种非常常见的模式。为了使这更容易,\hyperlink{300818094931158296}{\texttt{open}} 还有另一种调用方式,它以一个函数作为其第一个参数,以文件名作为其第二个参数,以该文件为参数调用该函数,然后再次关闭它。例如,给定函数:




\begin{minted}{julia}
function read_and_capitalize(f::IOStream)
    return uppercase(read(f, String))
end
\end{minted}



可以调用:




\begin{minted}{jlcon}
julia> open(read_and_capitalize, "hello.txt")
"HELLO AGAIN."
\end{minted}



来打开 \texttt{hello.txt},对它调用 \texttt{read\_and\_capitalize},关闭 \texttt{hello.txt} 并返回大写的内容。



为了避免被迫定义一个命名函数,你可以使用 \texttt{do} 语法,它可以动态地创建匿名函数:




\begin{minted}{jlcon}
julia> open("hello.txt") do f
           uppercase(read(f, String))
       end
"HELLO AGAIN."
\end{minted}



\hypertarget{15377875155010529137}{}


\section{一个简单的 TCP 示例}



让我们直接进入一个 TCP 套接字相关的简单示例。此功能位于名为 \texttt{Sockets} 的标准库中。让我们先创建一个简单的服务器:




\begin{minted}{jlcon}
julia> using Sockets

julia> @async begin
           server = listen(2000)
           while true
               sock = accept(server)
               println("Hello World\n")
           end
       end
Task (runnable) @0x00007fd31dc11ae0
\end{minted}



对于那些熟悉 Unix 套接字 API 的人,这些方法名称会让人感觉很熟悉,可是它们的用法比原始的 Unix 套接字 API 要简单些。在本例中,首次调用 \hyperlink{780704944207038170}{\texttt{listen}} 会创建一个服务器,等待传入指定端口(2000)的连接。




\begin{minted}{jlcon}
julia> listen(2000) # 监听(IPv4 下的)localhost:2000
Sockets.TCPServer(active)

julia> listen(ip"127.0.0.1",2000) # 等价于第一个
Sockets.TCPServer(active)

julia> listen(ip"::1",2000) # 监听(IPv6 下的)localhost:2000
Sockets.TCPServer(active)

julia> listen(IPv4(0),2001) # 监听所有 IPv4 接口的端口 2001
Sockets.TCPServer(active)

julia> listen(IPv6(0),2001) # 监听所有 IPv6 接口的端口 2001
Sockets.TCPServer(active)

julia> listen("testsocket") # 监听 UNIX 域套接字
Sockets.PipeServer(active)

julia> listen("\\\\.\\pipe\\testsocket") # 监听 Windows 命名管道
Sockets.PipeServer(active)
\end{minted}



Note that the return type of the last invocation is different. This is because this server does not listen on TCP, but rather on a named pipe (Windows) or UNIX domain socket. Also note that Windows named pipe format has to be a specific pattern such that the name prefix (\texttt{{\textbackslash}{\textbackslash}.{\textbackslash}pipe{\textbackslash}}) uniquely identifies the \href{https://docs.microsoft.com/windows/desktop/ipc/pipe-names}{file type}. The difference between TCP and named pipes or UNIX domain sockets is subtle and has to do with the \hyperlink{1426793569216032849}{\texttt{accept}} and \hyperlink{9743233285520657275}{\texttt{connect}} methods. The \hyperlink{1426793569216032849}{\texttt{accept}} method retrieves a connection to the client that is connecting on the server we just created, while the \hyperlink{9743233285520657275}{\texttt{connect}} function connects to a server using the specified method. The \hyperlink{9743233285520657275}{\texttt{connect}} function takes the same arguments as \hyperlink{780704944207038170}{\texttt{listen}}, so, assuming the environment (i.e. host, cwd, etc.) is the same you should be able to pass the same arguments to \hyperlink{9743233285520657275}{\texttt{connect}} as you did to listen to establish the connection. So let{\textquotesingle}s try that out (after having created the server above):




\begin{minted}{jlcon}
julia> connect(2000)
TCPSocket(open, 0 bytes waiting)

julia> Hello World
\end{minted}



不出所料,我们看到「Hello World」被打印出来。那么,让我们分析一下幕后发生的事情。在我们调用 \hyperlink{9743233285520657275}{\texttt{connect}} 时,我们连接到刚刚创建的服务器。与此同时,accept 函数返回到新创建的套接字的服务器端连接,并打印「Hello World」来表明连接成功。



Julia 的强大优势在于,即使 I/O 实际上是异步发生的,API 也以同步方式暴露,我们不必担心回调,甚至不必确保服务器能够运行。在我们调用 \hyperlink{9743233285520657275}{\texttt{connect}} 时,当前任务等待建立连接,并在这之后才继续执行。在此暂停中,服务器任务恢复执行(因为现在有一个连接请求是可用的),接受该连接,打印信息并等待下一个客户端。读取和写入以同样的方式运行。为了理解这一点,请考虑以下简单的 echo 服务器:




\begin{minted}{jlcon}
julia> @async begin
           server = listen(2001)
           while true
               sock = accept(server)
               @async while isopen(sock)
                   write(sock, readline(sock, keep=true))
               end
           end
       end
Task (runnable) @0x00007fd31dc12e60

julia> clientside = connect(2001)
TCPSocket(RawFD(28) open, 0 bytes waiting)

julia> @async while isopen(clientside)
           write(stdout, readline(clientside, keep=true))
       end
Task (runnable) @0x00007fd31dc11870

julia> println(clientside,"Hello World from the Echo Server")
Hello World from the Echo Server
\end{minted}



与其他流一样,使用 \hyperlink{5331333469799487255}{\texttt{close}} 即可断开该套接字:




\begin{minted}{jlcon}
julia> close(clientside)
\end{minted}



\hypertarget{8798664283149579111}{}


\section{解析 IP 地址}



与 \hyperlink{780704944207038170}{\texttt{listen}} 方法不一致的 \hyperlink{9743233285520657275}{\texttt{connect}} 方法之一是 \texttt{connect(host::String,port)},它将尝试连接到由 \texttt{host} 参数给定的主机上的由 \texttt{port} 参数给定的端口。它允许你执行以下操作:




\begin{minted}{jlcon}
julia> connect("google.com", 80)
TCPSocket(RawFD(30) open, 0 bytes waiting)
\end{minted}



此功能的基础是 \hyperlink{10301989504197190983}{\texttt{getaddrinfo}},它将执行适当的地址解析:




\begin{minted}{jlcon}
julia> getaddrinfo("google.com")
ip"74.125.226.225"
\end{minted}



\hypertarget{6184253979084926}{}


\chapter{并行计算}



Julia supports three main categories of features for concurrent and parallel programming:



\begin{itemize}
\item[1. ] Asynchronous {\textquotedbl}tasks{\textquotedbl}, or coroutines


\item[2. ] Multi-threading


\item[3. ] Distributed computing

\end{itemize}


Julia Tasks allow suspending and resuming computations for I/O, event handling, producer-consumer processes, and similar patterns. Tasks can synchronize through operations like \hyperlink{13761789780433862250}{\texttt{wait}} and \hyperlink{11007884648860062495}{\texttt{fetch}}, and communicate via \hyperlink{12548845729684045604}{\texttt{Channel}}s.



Multi-threading functionality builds on tasks by allowing them to run simultaneously on more than one thread or CPU core, sharing memory.



Finally, distributed computing runs multiple processes with separate memory spaces, potentially on different machines. This functionality is provided by the \texttt{Distributed} standard library as well as external packages like \texttt{MPI.jl} and \texttt{DistributedArrays.jl}.



\hypertarget{12867141128563676518}{}


\chapter{运行外部程序}



Julia 中命令的反引号记法借鉴于 shell、Perl 和 Ruby。然而,在 Julia 中编写




\begin{minted}{jlcon}
julia> `echo hello`
`echo hello`
\end{minted}



在多个方面上与 shell、Perl 和 Ruby 中的行为有所不同:



\begin{itemize}
\item 反引号创建一个 \hyperlink{10541952265148699805}{\texttt{Cmd}} 对象来表示命令,而不是立即运行命令。 你可以使用此对象将命令通过管道连接到其它命令、\hyperlink{18309243184998755104}{\texttt{run}} 它以及对它进行 \hyperlink{8104134490906192097}{\texttt{read}} 或 \hyperlink{16947913578760238729}{\texttt{write}}。


\item 在命令运行时,Julia 不会捕获命令的输出结果,除非你对它专门安排。相反,在默认情况下,命令的输出会被定向到 \hyperlink{18181294266083891471}{\texttt{stdout}},因为它将使用 \texttt{libc} 的 \texttt{system} 调用。


\item 命令从不会在 shell 中运行。相反地,Julia 会直接解析命令语法,适当地插入变量并像 shell 那样拆分单词,同时遵从 shell 的引用语法。命令会作为 \texttt{julia} 的直接子进程运行,使用 \texttt{fork} 和 \texttt{exec} 调用。

\end{itemize}


这是运行外部程序的简单示例:




\begin{minted}{jlcon}
julia> mycommand = `echo hello`
`echo hello`

julia> typeof(mycommand)
Cmd

julia> run(mycommand);
hello
\end{minted}



\texttt{hello} 是 \texttt{echo} 命令的输出,会被发送到 \hyperlink{18181294266083891471}{\texttt{stdout}} 中去。run 方法本身返回 \texttt{nothing},如果外部命令未能成功运行,则抛出 \hyperlink{12102596058483452470}{\texttt{ErrorException}}。



如果要读取外部命令的输出,可以使用 \hyperlink{8104134490906192097}{\texttt{read}}:




\begin{minted}{jlcon}
julia> a = read(`echo hello`, String)
"hello\n"

julia> chomp(a) == "hello"
true
\end{minted}



更一般地,你可以使用 \hyperlink{300818094931158296}{\texttt{open}} 来读取或写入外部命令。




\begin{minted}{jlcon}
julia> open(`less`, "w", stdout) do io
           for i = 1:3
               println(io, i)
           end
       end
1
2
3
\end{minted}



命令中的程序名称和各个参数可以访问和迭代,这就好像命令也是一个字符串数组:




\begin{minted}{jlcon}
julia> collect(`echo "foo bar"`)
2-element Array{String,1}:
 "echo"
 "foo bar"

julia> `echo "foo bar"`[2]
"foo bar"
\end{minted}



\hypertarget{6373319844820183024}{}


\section{插值}



假设你想要做的事情更复杂,并使用以变量 \texttt{file} 表示的文件名作为命令的参数。那你可以像在字符串字面量中那样使用 \texttt{\$} 进行插值:




\begin{minted}{jlcon}
julia> file = "/etc/passwd"
"/etc/passwd"

julia> `sort $file`
`sort /etc/passwd`
\end{minted}



通过 shell 运行外部程序的一个常见陷阱是,如果文件名中包含 shell 中的特殊字符,那么可能会导致不希望出现的行为。例如,假设我们想要对其内容进行排序的文件是 \texttt{/Volumes/External HD/data.csv},而不是 \texttt{/etc/passwd}。让我们来试试:




\begin{minted}{jlcon}
julia> file = "/Volumes/External HD/data.csv"
"/Volumes/External HD/data.csv"

julia> `sort $file`
`sort '/Volumes/External HD/data.csv'`
\end{minted}



文件名是如何被引用的?Julia 知道 \texttt{file} 是作为单个参数插入的,因此它替你引用了此单词。事实上,这不太准确:\texttt{file} 的值始终不会被 shell 解释,因此并不需要实际引用;插入引号只是为了展现给用户。就算你把值作为 shell 单词的一部分插入,这也可以工作:




\begin{minted}{jlcon}
julia> path = "/Volumes/External HD"
"/Volumes/External HD"

julia> name = "data"
"data"

julia> ext = "csv"
"csv"

julia> `sort $path/$name.$ext`
`sort '/Volumes/External HD/data.csv'`
\end{minted}



如你所见,\texttt{path} 变量中的空格被恰当地转义了。但是,如果你\emph{想}插入多个单词怎么办?在此情况下,只需使用数组(或其它可迭代容器):




\begin{minted}{jlcon}
julia> files = ["/etc/passwd","/Volumes/External HD/data.csv"]
2-element Array{String,1}:
 "/etc/passwd"
 "/Volumes/External HD/data.csv"

julia> `grep foo $files`
`grep foo /etc/passwd '/Volumes/External HD/data.csv'`
\end{minted}



如果将数组作为 shell 单词的一部分插入,Julia 将模拟 shell 的 \texttt{\{a,b,c\}} 参数生成:




\begin{minted}{jlcon}
julia> names = ["foo","bar","baz"]
3-element Array{String,1}:
 "foo"
 "bar"
 "baz"

julia> `grep xylophone $names.txt`
`grep xylophone foo.txt bar.txt baz.txt`
\end{minted}



此外,若在同一单词中插入多个数组,则将模拟 shell 的笛卡尔积生成行为:




\begin{minted}{jlcon}
julia> names = ["foo","bar","baz"]
3-element Array{String,1}:
 "foo"
 "bar"
 "baz"

julia> exts = ["aux","log"]
2-element Array{String,1}:
 "aux"
 "log"

julia> `rm -f $names.$exts`
`rm -f foo.aux foo.log bar.aux bar.log baz.aux baz.log`
\end{minted}



因为可以插入字面量数组,所以你可以使用此生成功能,而无需先创建临时数组对象:




\begin{minted}{jlcon}
julia> `rm -rf $["foo","bar","baz","qux"].$["aux","log","pdf"]`
`rm -rf foo.aux foo.log foo.pdf bar.aux bar.log bar.pdf baz.aux baz.log baz.pdf qux.aux qux.log qux.pdf`
\end{minted}



\hypertarget{12430289445905702597}{}


\section{引用}



不可避免地,我们会想要编写不那么简单的命令,且有必要使用引号。下面是 shell 提示符下单行 Perl 程序的简单示例:




\begin{lstlisting}
sh$ perl -le '$|=1; for (0..3) { print }'
0
1
2
3
\end{lstlisting}



该 Perl 表达式需要使用单引号有两个原因:一是为了避免空格将表达式分解为多个 shell 单词,二是为了在使用像 \texttt{\$|}(是的,这在 Perl 中是变量名)这样的 Perl 变量时避免发生插值。在其它情况下,你可能想要使用双引号来\emph{真的}进行插值:




\begin{lstlisting}
sh$ first="A"
sh$ second="B"
sh$ perl -le '$|=1; print for @ARGV' "1: $first" "2: $second"
1: A
2: B
\end{lstlisting}



总之,Julia 反引号语法是经过精心设计的,因此你可以只是将 shell 命令剪切并粘贴到反引号中,接着它们将会工作:转义、引用和插值行为与 shell 相同。唯一的不同是,插值是集成的并且知道在 Julia 的概念中什么是单个字符串值、什么是多个值的容器。让我们在 Julia 中尝试上面的两个例子:




\begin{minted}{jlcon}
julia> A = `perl -le '$|=1; for (0..3) { print }'`
`perl -le '$|=1; for (0..3) { print }'`

julia> run(A);
0
1
2
3

julia> first = "A"; second = "B";

julia> B = `perl -le 'print for @ARGV' "1: $first" "2: $second"`
`perl -le 'print for @ARGV' '1: A' '2: B'`

julia> run(B);
1: A
2: B
\end{minted}



结果是相同的,且 Julia 的插值行为模仿了 shell 的并对其做了一些改进,因为 Julia 支持头等的可迭代对象,但大多数 shell 通过使用空格分隔字符串来实现这一点,而这又引入了歧义。在尝试将 shell 命令移植到 Julia 中时,请先试着剪切并粘贴它。因为 Julia 会在运行命令前向你显示命令,所以你可以在不造成任何破坏的前提下轻松并安全地检查命令的解释。



\hypertarget{565642647898476823}{}


\section{管道}



Shell 元字符,如 \texttt{|}、\texttt{\&} 和 \texttt{>},在 Julia 的反引号中需被引用(或转义):




\begin{minted}{jlcon}
julia> run(`echo hello '|' sort`);
hello | sort

julia> run(`echo hello \| sort`);
hello | sort
\end{minted}



此表达式调用 \texttt{echo} 命令并以三个单词作为其参数:\texttt{hello}、\texttt{|} 和 \texttt{sort}。结果是只打印了一行:\texttt{hello | sort}。那么,如何构造管道呢?为此,请使用 \hyperlink{17710887576380723118}{\texttt{pipeline}},而不是在反引号内使用 \texttt{{\textquotesingle}|{\textquotesingle}}:




\begin{minted}{jlcon}
julia> run(pipeline(`echo hello`, `sort`));
hello
\end{minted}



这将 \texttt{echo} 命令的输出传输到 \texttt{sort} 命令中。当然,这不是很有趣,因为只有一行要排序,但是我们的当然可以做更多、更有趣的事:




\begin{minted}{jlcon}
julia> run(pipeline(`cut -d: -f3 /etc/passwd`, `sort -n`, `tail -n5`))
210
211
212
213
214
\end{minted}



这将打印在 UNIX 系统上最高的五个用户 ID。\texttt{cut}、\texttt{sort} 和 \texttt{tail} 命令都是当前 \texttt{julia} 进程的直接子进程,这中间没有 shell 进程的干预。Julia 自己负责设置管道和连接文件描述符,而这通常由 shell 完成。因为 Julia 自己做了这些事,所以它能更好的控制并做 shell 做不到的一些事情。



Julia 可以并行地运行多个命令:




\begin{minted}{jlcon}
julia> run(`echo hello` & `echo world`);
world
hello
\end{minted}



这里的输出顺序是不确定的,因为两个 \texttt{echo} 进程几乎同时启动,并且争着先写入 \hyperlink{18181294266083891471}{\texttt{stdout}} 描述符和 \texttt{julia} 父进程。Julia 允许你将这两个进程的输出通过管道传输到另一个程序:




\begin{minted}{jlcon}
julia> run(pipeline(`echo world` & `echo hello`, `sort`));
hello
world
\end{minted}



在 UNIX 管道方面,这里发生的是,一个 UNIX 管道对象由两个 \texttt{echo} 进程创建和写入,管道的另一端由 \texttt{sort} 命令读取。



IO 重定向可以通过向 \texttt{pipeline} 函数传递关键字参数 \texttt{stdin}、\texttt{stdout} 和 \texttt{stderr} 来实现:




\begin{minted}{julia}
pipeline(`do_work`, stdout=pipeline(`sort`, "out.txt"), stderr="errs.txt")
\end{minted}



\hypertarget{10119727931012697003}{}


\subsection{避免管道中的死锁}



在单个进程中读取和写入管道的两端时,避免强制内核缓冲所有数据是很重要的。



例如,在读取命令的所有输出时,请调用 \texttt{read(out, String)},而非 \texttt{wait(process)},因为前者会积极地消耗由该进程写入的所有数据,而后者在等待读取者连接时会尝试将数据存储内核的缓冲区中。



另一个常见的解决方案是将读取者和写入者分离到单独的 \hyperlink{7131243650304654155}{\texttt{Task}} 中:




\begin{minted}{julia}
writer = @async write(process, "data")
reader = @async do_compute(read(process, String))
wait(writer)
fetch(reader)
\end{minted}



\hypertarget{14183945833806948978}{}


\subsection{复杂示例}



高级编程语言、头等的命令抽象以及进程间管道的自动设置,三者组合起来非常强大。为了更好地理解可被轻松创建的复杂管道,这里有一些更复杂的例子,以避免对单行 Perl 程序的滥用。




\begin{minted}{jlcon}
julia> prefixer(prefix, sleep) = `perl -nle '$|=1; print "'$prefix' ", $_; sleep '$sleep';'`;

julia> run(pipeline(`perl -le '$|=1; for(0..5){ print; sleep 1 }'`, prefixer("A",2) & prefixer("B",2)));
B 0
A 1
B 2
A 3
B 4
A 5
\end{minted}



这是一个经典的例子,一个生产者为两个并发的消费者提供内容:一个 \texttt{perl} 进程生成从数字 0 到 5 的行,而两个并行进程则使用该输出,一个行首加字母「A」,另一个行首加字母「B」。哪个进程使用第一行是不确定的,但是一旦赢得了竞争,这些行会先后被其中一个进程及另一个进程交替使用。(在 Perl 中设置 \texttt{\$|=1} 会导致每个 print 语句刷新 \hyperlink{18181294266083891471}{\texttt{stdout}} 句柄,这是本例工作所必需的。此外,所有输出将被缓存并一次性打印到管道中,以便只由一个消费者进程读取。)



这是一个更加复杂的多阶段生产者——消费者示例:




\begin{minted}{jlcon}
julia> run(pipeline(`perl -le '$|=1; for(0..5){ print; sleep 1 }'`,
           prefixer("X",3) & prefixer("Y",3) & prefixer("Z",3),
           prefixer("A",2) & prefixer("B",2)));
A X 0
B Y 1
A Z 2
B X 3
A Y 4
B Z 5
\end{minted}



此示例与前一个类似,不同之处在于本例中的消费者有两个阶段,并且阶段间有不同的延迟,因此它们使用不同数量的并行 worker 来维持饱和的吞吐量。



我们强烈建议你尝试所有这些例子,以便了解它们的工作原理。



\hypertarget{6259308371449123088}{}


\chapter{调用 C 和 Fortran 代码}



在数值计算领域,尽管有很多用 C 语言或 Fortran 写的高质量且成熟的库都可以用 Julia 重写,但为了便捷利用现有的 C 或 Fortran 代码,Julia 提供简洁且高效的调用方式。Julia 的哲学是 \texttt{no boilerplate}: Julia 可以直接调用 C/Fortran 的函数,不需要任何{\textquotedbl}胶水{\textquotedbl}代码,代码生成或其它编译过程 – 即使在交互式会话 (REPL/Jupyter notebook) 中使用也一样. 在 Julia 中,上述特性可以仅仅通过调用 \hyperlink{14245046751182637566}{\texttt{ccall}} 实现,它的语法看起来就像是普通的函数调用。



The code to be called must be available as a shared library. Most C and Fortran libraries ship compiled as shared libraries already, but if you are compiling the code yourself using GCC (or Clang), you will need to use the \texttt{-shared} and \texttt{-fPIC} options. The machine instructions generated by Julia{\textquotesingle}s JIT are the same as a native C call would be, so the resulting overhead is the same as calling a library function from C code. \footnotemark[1]



Shared libraries and functions are referenced by a tuple of the form \texttt{(:function, {\textquotedbl}library{\textquotedbl})} or \texttt{({\textquotedbl}function{\textquotedbl}, {\textquotedbl}library{\textquotedbl})} where \texttt{function} is the C-exported function name, and \texttt{library} refers to the shared library name.  Shared libraries available in the (platform-specific) load path will be resolved by name.  The full path to the library may also be specified.



可以单独使用函数名来代替元组(只用 \texttt{:function} 或 \texttt{{\textquotedbl}function{\textquotedbl}})。在这种情况下,函数名在当前进程中进行解析。这一调用形式可用于调用 C 库函数、Julia 运行时中的函数或链接到 Julia 的应用程序中的函数。



默认情况下,Fortran 编译器会\href{https://en.wikipedia.org/wiki/Name\_mangling\#Fortran}{进行名称修饰}(例如,将函数名转换为小写或大写,通常会添加下划线),要通过 \hyperlink{14245046751182637566}{\texttt{ccall}} 调用 Fortran 函数,传递的标识符必须与 Fortran 编译器名称修饰之后的一致。此外,在调用 Fortran 函数时,\textbf{所有}输入必须以指针形式传递,并已在堆或栈上分配内存。这不仅适用于通常是堆分配的数组及可变对象,而且适用于整数和浮点数等标量值,尽管这些值通常是栈分配的,且在使用 C 或 Julia 调用约定时通常是通过寄存器传递的。



Finally, you can use \hyperlink{14245046751182637566}{\texttt{ccall}} to actually generate a call to the library function. The arguments to \hyperlink{14245046751182637566}{\texttt{ccall}} are:



\begin{itemize}
\item[1. ] A \texttt{(:function, {\textquotedbl}library{\textquotedbl})} pair (most common),

或

a \texttt{:function} name symbol or \texttt{{\textquotedbl}function{\textquotedbl}} name string (for symbols in the current process or libc),

或

一个函数指针(例如,从 \texttt{dlsym} 获得的指针)。


\item[2. ] The function{\textquotesingle}s return type


\item[3. ] A tuple of input types, corresponding to the function signature


\item[4. ] The actual argument values to be passed to the function, if any; each is a separate parameter.

\end{itemize}


\begin{quote}
\textbf{Note}

The \texttt{(:function, {\textquotedbl}library{\textquotedbl})} pair, return type, and input types must be literal constants (i.e., they can{\textquotesingle}t be variables, but see \hyperlink{415091760485310867}{Non-constant Function Specifications} below).

The remaining parameters are evaluated at compile time, when the containing method is defined.

\end{quote}


\begin{quote}
\textbf{Note}

See below for how to \hyperlink{10872711251657367863}{map C types to Julia types}.

\end{quote}


As a complete but simple example, the following calls the \texttt{clock} function from the standard C library on most Unix-derived systems:




\begin{minted}{jlcon}
julia> t = ccall(:clock, Int32, ())
2292761

julia> t
2292761

julia> typeof(ans)
Int32
\end{minted}



\texttt{clock} takes no arguments and returns an \hyperlink{10103694114785108551}{\texttt{Int32}}. One common mistake is forgetting that a 1-tuple of argument types must be written with a trailing comma. For example, to call the \texttt{getenv} function to get a pointer to the value of an environment variable, one makes a call like this:




\begin{minted}{jlcon}
julia> path = ccall(:getenv, Cstring, (Cstring,), "SHELL")
Cstring(@0x00007fff5fbffc45)

julia> unsafe_string(path)
"/bin/bash"
\end{minted}



Note that the argument type tuple must be written as \texttt{(Cstring,)}, not \texttt{(Cstring)}. This is because \texttt{(Cstring)} is just the expression \texttt{Cstring} surrounded by parentheses, rather than a 1-tuple containing \texttt{Cstring}:




\begin{minted}{jlcon}
julia> (Cstring)
Cstring

julia> (Cstring,)
(Cstring,)
\end{minted}



In practice, especially when providing reusable functionality, one generally wraps \hyperlink{14245046751182637566}{\texttt{ccall}} uses in Julia functions that set up arguments and then check for errors in whatever manner the C or Fortran function specifies. And if an error occurs it is thrown as a normal Julia exception. This is especially important since C and Fortran APIs are notoriously inconsistent about how they indicate error conditions. For example, the \texttt{getenv} C library function is wrapped in the following Julia function, which is a simplified version of the actual definition from \href{https://github.com/JuliaLang/julia/blob/master/base/env.jl}{\texttt{env.jl}}:




\begin{minted}{julia}
function getenv(var::AbstractString)
    val = ccall(:getenv, Cstring, (Cstring,), var)
    if val == C_NULL
        error("getenv: undefined variable: ", var)
    end
    return unsafe_string(val)
end
\end{minted}



C 函数 \texttt{getenv} 通过返回 \texttt{NULL} 的方式进行报错,但是其他 C 标准库函数也会通过多种不同的方式来报错,这包括返回 -1,0,1 以及其它特殊值。此封装能够明确地抛出异常信息,即是否调用者在尝试获取一个不存在的环境变量:




\begin{minted}{jlcon}
julia> getenv("SHELL")
"/bin/bash"

julia> getenv("FOOBAR")
getenv: undefined variable: FOOBAR
\end{minted}



Here is a slightly more complex example that discovers the local machine{\textquotesingle}s hostname. In this example, the networking library code is assumed to be in a shared library named {\textquotedbl}libc{\textquotedbl}. In practice, this function is usually part of the C standard library, and so the {\textquotedbl}libc{\textquotedbl} portion should be omitted, but we wish to show here the usage of this syntax.




\begin{minted}{julia}
function gethostname()
    hostname = Vector{UInt8}(undef, 256) # MAXHOSTNAMELEN
    err = ccall((:gethostname, "libc"), Int32,
                (Ptr{UInt8}, Csize_t),
                hostname, sizeof(hostname))
    Base.systemerror("gethostname", err != 0)
    hostname[end] = 0 # ensure null-termination
    return unsafe_string(pointer(hostname))
end
\end{minted}



This example first allocates an array of bytes. It then calls the C library function \texttt{gethostname} to populate the array with the hostname. Finally, it takes a pointer to the hostname buffer, and converts the pointer to a Julia string, assuming that it is a NUL-terminated C string. It is common for C libraries to use this pattern of requiring the caller to allocate memory to be passed to the callee and populated. Allocation of memory from Julia like this is generally accomplished by creating an uninitialized array and passing a pointer to its data to the C function. This is why we don{\textquotesingle}t use the \texttt{Cstring} type here: as the array is uninitialized, it could contain NUL bytes. Converting to a \texttt{Cstring} as part of the \hyperlink{14245046751182637566}{\texttt{ccall}} checks for contained NUL bytes and could therefore throw a conversion error.



\hypertarget{17519979972664810183}{}


\section{创建和C兼容的Julia函数指针}



可以将Julia函数传递给接受函数指针参数的原生C函数。例如,要匹配满足下面的C原型:




\begin{lstlisting}
typedef returntype (*functiontype)(argumenttype, ...)
\end{lstlisting}



The macro \hyperlink{11617107520401351255}{\texttt{@cfunction}} generates the C-compatible function pointer for a call to a Julia function. The arguments to \hyperlink{11617107520401351255}{\texttt{@cfunction}} are:



\begin{itemize}
\item[1. ] A Julia function


\item[2. ] The function{\textquotesingle}s return type


\item[3. ] A tuple of input types, corresponding to the function signature

\end{itemize}


\begin{quote}
\textbf{Note}

As with \texttt{ccall}, the return type and tuple of input types must be literal constants.

\end{quote}


\begin{quote}
\textbf{Note}

Currently, only the platform-default C calling convention is supported. This means that \texttt{@cfunction}-generated pointers cannot be used in calls where WINAPI expects a \texttt{stdcall} function on 32-bit Windows, but can be used on WIN64 (where \texttt{stdcall} is unified with the C calling convention).

\end{quote}


一个典型的例子就是标准C库函数\texttt{qsort},定义为:




\begin{lstlisting}
void qsort(void *base, size_t nmemb, size_t size,
           int (*compare)(const void*, const void*));
\end{lstlisting}



The \texttt{base} argument is a pointer to an array of length \texttt{nmemb}, with elements of \texttt{size} bytes each. \texttt{compare} is a callback function which takes pointers to two elements \texttt{a} and \texttt{b} and returns an integer less/greater than zero if \texttt{a} should appear before/after \texttt{b} (or zero if any order is permitted).



Now, suppose that we have a 1-d array \texttt{A} of values in Julia that we want to sort using the \texttt{qsort} function (rather than Julia{\textquotesingle}s built-in \texttt{sort} function). Before we consider calling \texttt{qsort} and passing arguments, we need to write a comparison function:




\begin{minted}{jlcon}
julia> function mycompare(a, b)::Cint
           return (a < b) ? -1 : ((a > b) ? +1 : 0)
       end
mycompare (generic function with 1 method)
\end{minted}



\texttt{qsort} expects a comparison function that return a C \texttt{int}, so we annotate the return type to be \texttt{Cint}.



In order to pass this function to C, we obtain its address using the macro \texttt{@cfunction}:




\begin{minted}{jlcon}
julia> mycompare_c = @cfunction(mycompare, Cint, (Ref{Cdouble}, Ref{Cdouble}));
\end{minted}



\hyperlink{11617107520401351255}{\texttt{@cfunction}} 需要三个参数: Julia函数 (\texttt{mycompare}), 返回值类型(\texttt{Cint}), 和一个输入参数类型的值元组, 此处是要排序的\texttt{Cdouble}(\hyperlink{5027751419500983000}{\texttt{Float64}}) 元素的数组.



\texttt{qsort}的最终调用看起来是这样的:




\begin{minted}{jlcon}
julia> A = [1.3, -2.7, 4.4, 3.1]
4-element Array{Float64,1}:
  1.3
 -2.7
  4.4
  3.1

julia> ccall(:qsort, Cvoid, (Ptr{Cdouble}, Csize_t, Csize_t, Ptr{Cvoid}),
             A, length(A), sizeof(eltype(A)), mycompare_c)

julia> A
4-element Array{Float64,1}:
 -2.7
  1.3
  3.1
  4.4
\end{minted}



As the example shows, the original Julia array \texttt{A} has now been sorted: \texttt{[-2.7, 1.3, 3.1, 4.4]}. Note that Julia \hyperlink{3709811402862932682}{takes care of converting the array to a \texttt{Ptr\{Cdouble\}}}), computing the size of the element type in bytes, and so on.



For fun, try inserting a \texttt{println({\textquotedbl}mycompare(\$a, \$b){\textquotedbl})} line into \texttt{mycompare}, which will allow you to see the comparisons that \texttt{qsort} is performing (and to verify that it is really calling the Julia function that you passed to it).



\hypertarget{10747596823753409973}{}


\section{Mapping C Types to Julia}



It is critical to exactly match the declared C type with its declaration in Julia. Inconsistencies can cause code that works correctly on one system to fail or produce indeterminate results on a different system.



Note that no C header files are used anywhere in the process of calling C functions: you are responsible for making sure that your Julia types and call signatures accurately reflect those in the C header file.\footnotemark[2]



\hypertarget{15689505875890763631}{}


\subsection{Automatic Type Conversion}



Julia automatically inserts calls to the \hyperlink{16487788729383051927}{\texttt{Base.cconvert}} function to convert each argument to the specified type. For example, the following call:




\begin{minted}{julia}
ccall((:foo, "libfoo"), Cvoid, (Int32, Float64), x, y)
\end{minted}



will behave as if it were written like this:




\begin{minted}{julia}
ccall((:foo, "libfoo"), Cvoid, (Int32, Float64),
      Base.unsafe_convert(Int32, Base.cconvert(Int32, x)),
      Base.unsafe_convert(Float64, Base.cconvert(Float64, y)))
\end{minted}



\hyperlink{16487788729383051927}{\texttt{Base.cconvert}} normally just calls \hyperlink{1846942650946171605}{\texttt{convert}}, but can be defined to return an arbitrary new object more appropriate for passing to C. This should be used to perform all allocations of memory that will be accessed by the C code. For example, this is used to convert an \texttt{Array} of objects (e.g. strings) to an array of pointers.



\hyperlink{6011318385865707029}{\texttt{Base.unsafe\_convert}} handles conversion to \hyperlink{10630331440513004826}{\texttt{Ptr}} types. It is considered unsafe because converting an object to a native pointer can hide the object from the garbage collector, causing it to be freed prematurely.



\hypertarget{11183503169412025304}{}


\subsection{Type Correspondences}



First, let{\textquotesingle}s review some relevant Julia type terminology:




\begin{table}[h]

\begin{tabulary}{\linewidth}{|L|L|L|}
\hline
语法 / 关键字 & 例子 & 描述 \\
\hline
\texttt{mutable struct} & \texttt{BitSet} & \texttt{Leaf Type}:包含 \texttt{type-tag} 的一组相关数据,由 Julia GC 管理,通过 \texttt{object-identity} 来定义。为了保证实例可以被构造,\texttt{Leaf Type} 必须是完整定义的,即不允许使用 \texttt{TypeVars}。 \\
\hline
\texttt{abstract type} & \texttt{Any}, \texttt{AbstractArray\{T, N\}}, \texttt{Complex\{T\}} & \texttt{Super Type}:用于描述一组类型,它不是 \texttt{Leaf-Type},也无法被实例化。 \\
\hline
\texttt{T\{A\}} & \texttt{Vector\{Int\}} & \texttt{Type Parameter}:某种类型的一种具体化,通常用于分派或存储优化。 \\
\hline
 &  & \texttt{TypeVar}:\texttt{Type parameter} 声明中的 \texttt{T} 是一个 \texttt{TypeVar},它是类型变量的简称。 \\
\hline
\texttt{primitive type} & \texttt{Int}, \texttt{Float64} & \texttt{Primitive Type}:一种没有成员变量的类型,但是它有大小。It is stored and defined by-value. \\
\hline
\texttt{struct} & \texttt{Pair\{Int, Int\}} & {\textquotedbl}Struct{\textquotedbl} :: A type with all fields defined to be constant. It is defined by-value, and may be stored with a type-tag. \\
\hline
 & \texttt{ComplexF64} (\texttt{isbits}) & {\textquotedbl}Is-Bits{\textquotedbl}   :: A \texttt{primitive type}, or a \texttt{struct} type where all fields are other \texttt{isbits} types. It is defined by-value, and is stored without a type-tag. \\
\hline
\texttt{struct ...; end} & \texttt{nothing} & \texttt{Singleton}:没有成员变量的 \texttt{Leaf Type} 或 \texttt{Struct}。 \\
\hline
\texttt{(...)} or \texttt{tuple(...)} & \texttt{(1, 2, 3)} & {\textquotedbl}Tuple{\textquotedbl} :: an immutable data-structure similar to an anonymous struct type, or a constant array. Represented as either an array or a struct. \\
\hline
\end{tabulary}

\end{table}



\hypertarget{538651652486673311}{}


\subsection{Bits Types}



There are several special types to be aware of, as no other type can be defined to behave the same:



\begin{itemize}
\item \texttt{Float32}

和C语言中的 \texttt{float} 类型完全对应(以及Fortran中的 \texttt{REAL*4} )


\item \texttt{Float64}

和C语言中的 \texttt{double} 类型完全对应(以及Fortran中的 \texttt{REAL*8} )


\item \texttt{ComplexF32}

和C语言中的 \texttt{complex float} 类型完全对应(以及Fortran中的 \texttt{COMPLEX*8} )


\item \texttt{ComplexF64}

和C语言中的 \texttt{complex double} 类型完全对应(以及Fortran中的 \texttt{COMPLEX*16} )


\item \texttt{Signed}

和C语言中的 \texttt{signed} 类型标识完全对应(以及Fortran中的任意 \texttt{INTEGER} 类型) Julia中任何不是\hyperlink{14154866400772377486}{\texttt{Signed}} 的子类型的类型,都会被认为是unsigned类型。

\end{itemize}


\begin{itemize}
\item \texttt{Ref\{T\}}

和 \texttt{Ptr\{T\}} 行为相同,能通过Julia的GC管理其内存。

\end{itemize}


\begin{itemize}
\item \texttt{Array\{T,N\}}

When an array is passed to C as a \texttt{Ptr\{T\}} argument, it is not reinterpret-cast: Julia requires that the element type of the array matches \texttt{T}, and the address of the first element is passed.

因此,如果一个 \texttt{Array} 中的数据格式不正确,它必须被显式地转换 ,通过类似 \texttt{trunc(Int32, a)} 的函数。

若要将一个数组 \texttt{A} 以不同类型的指针传递,而\emph{不提前转换数据}, (比如,将一个 \texttt{Float64} 数组传给一个处理原生字节的函数时),你 可以将这一参数声明为 \texttt{Ptr\{Cvoid\}} 。

如果一个元素类型为 \texttt{Ptr\{T\}} 的数组作为 \texttt{Ptr\{Ptr\{T\}\}} 类型的参数传递, \hyperlink{16487788729383051927}{\texttt{Base.cconvert}}  将会首先尝试进行 null-terminated copy(即直到下一个元素为null才停止复制),并将每一个元素使用其通过 \hyperlink{16487788729383051927}{\texttt{Base.cconvert}} 转换后的版本替换。 这允许,比如,将一个 \texttt{argv} 的指针数组,其类型为 \texttt{Vector\{String\}} ,传递给一个类型为 \texttt{Ptr\{Ptr\{Cchar\}\}} 的参数。

\end{itemize}


On all systems we currently support, basic C/C++ value types may be translated to Julia types as follows. Every C type also has a corresponding Julia type with the same name, prefixed by C. This can help when writing portable code (and remembering that an \texttt{int} in C is not the same as an \texttt{Int} in Julia).



\textbf{System Independent Types}




\begin{table}[h]

\begin{tabulary}{\linewidth}{|L|L|L|L|}
\hline
C 类型 & Fortran 类型 & 标准 Julia 别名 & Julia 基本类型 \\
\hline
\texttt{unsigned char} & \texttt{CHARACTER} & \texttt{Cuchar} & \texttt{UInt8} \\
\hline
\texttt{bool} (\_Bool in C99+) &  & \texttt{Cuchar} & \texttt{UInt8} \\
\hline
\texttt{short} & \texttt{INTEGER*2}, \texttt{LOGICAL*2} & \texttt{Cshort} & \texttt{Int16} \\
\hline
\texttt{unsigned short} &  & \texttt{Cushort} & \texttt{UInt16} \\
\hline
\texttt{int}, \texttt{BOOL} (C, typical) & \texttt{INTEGER*4}, \texttt{LOGICAL*4} & \texttt{Cint} & \texttt{Int32} \\
\hline
\texttt{unsigned int} &  & \texttt{Cuint} & \texttt{UInt32} \\
\hline
\texttt{long long} & \texttt{INTEGER*8}, \texttt{LOGICAL*8} & \texttt{Clonglong} & \texttt{Int64} \\
\hline
\texttt{unsigned long long} &  & \texttt{Culonglong} & \texttt{UInt64} \\
\hline
\texttt{intmax\_t} &  & \texttt{Cintmax\_t} & \texttt{Int64} \\
\hline
\texttt{uintmax\_t} &  & \texttt{Cuintmax\_t} & \texttt{UInt64} \\
\hline
\texttt{float} & \texttt{REAL*4i} & \texttt{Cfloat} & \texttt{Float32} \\
\hline
\texttt{double} & \texttt{REAL*8} & \texttt{Cdouble} & \texttt{Float64} \\
\hline
\texttt{complex float} & \texttt{COMPLEX*8} & \texttt{ComplexF32} & \texttt{Complex\{Float32\}} \\
\hline
\texttt{complex double} & \texttt{COMPLEX*16} & \texttt{ComplexF64} & \texttt{Complex\{Float64\}} \\
\hline
\texttt{ptrdiff\_t} &  & \texttt{Cptrdiff\_t} & \texttt{Int} \\
\hline
\texttt{ssize\_t} &  & \texttt{Cssize\_t} & \texttt{Int} \\
\hline
\texttt{size\_t} &  & \texttt{Csize\_t} & \texttt{UInt} \\
\hline
\texttt{void} &  &  & \texttt{Cvoid} \\
\hline
\texttt{void} and \texttt{[[noreturn]]} or \texttt{\_Noreturn} &  &  & \texttt{Union\{\}} \\
\hline
\texttt{void*} &  &  & \texttt{Ptr\{Cvoid\}} \\
\hline
\texttt{T*} (where T represents an appropriately defined type) &  &  & \texttt{Ref\{T\}} \\
\hline
\texttt{char*} (or \texttt{char[]}, e.g. a string) & \texttt{CHARACTER*N} &  & \texttt{Cstring} if NUL-terminated, or \texttt{Ptr\{UInt8\}} if not \\
\hline
\texttt{char**} (or \texttt{*char[]}) &  &  & \texttt{Ptr\{Ptr\{UInt8\}\}} \\
\hline
\texttt{jl\_value\_t*} (any Julia Type) &  &  & \texttt{Any} \\
\hline
\texttt{jl\_value\_t**} (a reference to a Julia Type) &  &  & \texttt{Ref\{Any\}} \\
\hline
\texttt{va\_arg} &  &  & Not supported \\
\hline
\texttt{...} (variadic function specification) &  &  & \texttt{T...} (where \texttt{T} is one of the above types, variadic functions of different argument types are not supported) \\
\hline
\end{tabulary}

\end{table}



The \hyperlink{8632604011862685836}{\texttt{Cstring}} type is essentially a synonym for \texttt{Ptr\{UInt8\}}, except the conversion to \texttt{Cstring} throws an error if the Julia string contains any embedded NUL characters (which would cause the string to be silently truncated if the C routine treats NUL as the terminator).  If you are passing a \texttt{char*} to a C routine that does not assume NUL termination (e.g. because you pass an explicit string length), or if you know for certain that your Julia string does not contain NUL and want to skip the check, you can use \texttt{Ptr\{UInt8\}} as the argument type. \texttt{Cstring} can also be used as the \hyperlink{14245046751182637566}{\texttt{ccall}} return type, but in that case it obviously does not introduce any extra checks and is only meant to improve readability of the call.



\textbf{System Dependent Types}




\begin{table}[h]

\begin{tabulary}{\linewidth}{|L|L|L|}
\hline
C 类型 & 标准 Julia 别名 & Julia 基本类型 \\
\hline
\texttt{char} & \texttt{Cchar} & \texttt{Int8} (x86, x86\_64), \texttt{UInt8} (powerpc, arm) \\
\hline
\texttt{long} & \texttt{Clong} & \texttt{Int} (UNIX), \texttt{Int32} (Windows) \\
\hline
\texttt{unsigned long} & \texttt{Culong} & \texttt{UInt} (UNIX), \texttt{UInt32} (Windows) \\
\hline
\texttt{wchar\_t} & \texttt{Cwchar\_t} & \texttt{Int32} (UNIX), \texttt{UInt16} (Windows) \\
\hline
\end{tabulary}

\end{table}



\begin{quote}
\textbf{Note}

When calling Fortran, all inputs must be passed by pointers to heap- or stack-allocated values, so all type correspondences above should contain an additional \texttt{Ptr\{..\}} or \texttt{Ref\{..\}} wrapper around their type specification.

\end{quote}


\begin{quote}
\textbf{Warning}

For string arguments (\texttt{char*}) the Julia type should be \texttt{Cstring} (if NUL- terminated data is expected), or either \texttt{Ptr\{Cchar\}} or \texttt{Ptr\{UInt8\}} otherwise (these two pointer types have the same effect), as described above, not \texttt{String}. Similarly, for array arguments (\texttt{T[]} or \texttt{T*}), the Julia type should again be \texttt{Ptr\{T\}}, not \texttt{Vector\{T\}}.

\end{quote}


\begin{quote}
\textbf{Warning}

Julia{\textquotesingle}s \texttt{Char} type is 32 bits, which is not the same as the wide character type (\texttt{wchar\_t} or \texttt{wint\_t}) on all platforms.

\end{quote}


\begin{quote}
\textbf{Warning}

A return type of \texttt{Union\{\}} means the function will not return, i.e., C++11 \texttt{[[noreturn]]} or C11 \texttt{\_Noreturn} (e.g. \texttt{jl\_throw} or \texttt{longjmp}). Do not use this for functions that return no value (\texttt{void}) but do return, use \texttt{Cvoid} instead.

\end{quote}


\begin{quote}
\textbf{Note}

For \texttt{wchar\_t*} arguments, the Julia type should be \hyperlink{510630608879002831}{\texttt{Cwstring}} (if the C routine expects a NUL-terminated string), or \texttt{Ptr\{Cwchar\_t\}} otherwise. Note also that UTF-8 string data in Julia is internally NUL-terminated, so it can be passed to C functions expecting NUL-terminated data without making a copy (but using the \texttt{Cwstring} type will cause an error to be thrown if the string itself contains NUL characters).

\end{quote}


\begin{quote}
\textbf{Note}

C functions that take an argument of type \texttt{char**} can be called by using a \texttt{Ptr\{Ptr\{UInt8\}\}} type within Julia. For example, C functions of the form:


\begin{lstlisting}
int main(int argc, char **argv);
\end{lstlisting}

can be called via the following Julia code:


\begin{minted}{julia}
argv = [ "a.out", "arg1", "arg2" ]
ccall(:main, Int32, (Int32, Ptr{Ptr{UInt8}}), length(argv), argv)
\end{minted}

\end{quote}


\begin{quote}
\textbf{Note}

For Fortran functions taking variable length strings of type \texttt{character(len=*)} the string lengths are provided as \emph{hidden arguments}. Type and position of these arguments in the list are compiler specific, where compiler vendors usually default to using \texttt{Csize\_t} as type and append the hidden arguments at the end of the argument list. While this behaviour is fixed for some compilers (GNU), others \emph{optionally} permit placing hidden arguments directly after the character argument (Intel, PGI). For example, Fortran subroutines of the form


\begin{lstlisting}
subroutine test(str1, str2)
character(len=*) :: str1,str2
\end{lstlisting}

can be called via the following Julia code, where the lengths are appended


\begin{minted}{julia}
str1 = "foo"
str2 = "bar"
ccall(:test, Cvoid, (Ptr{UInt8}, Ptr{UInt8}, Csize_t, Csize_t),
                    str1, str2, sizeof(str1), sizeof(str2))
\end{minted}

\end{quote}


\begin{quote}
\textbf{Warning}

Fortran compilers \emph{may} also add other hidden arguments for pointers, assumed-shape (\texttt{:}) and assumed-size (\texttt{*}) arrays. Such behaviour can be avoided by using \texttt{ISO\_C\_BINDING} and including \texttt{bind(c)} in the definition of the subroutine, which is strongly recommended for interoperable code. In this case there will be no hidden arguments, at the cost of some language features (e.g. only \texttt{character(len=1)} will be permitted to pass strings).

\end{quote}


\begin{quote}
\textbf{Note}

A C function declared to return \texttt{Cvoid} will return the value \texttt{nothing} in Julia.

\end{quote}


\hypertarget{8277927636807308593}{}


\subsection{Struct Type Correspondences}



Composite types such as \texttt{struct} in C or \texttt{TYPE} in Fortran90 (or \texttt{STRUCTURE} / \texttt{RECORD} in some variants of F77), can be mirrored in Julia by creating a \texttt{struct} definition with the same field layout.



When used recursively, \texttt{isbits} types are stored inline. All other types are stored as a pointer to the data. When mirroring a struct used by-value inside another struct in C, it is imperative that you do not attempt to manually copy the fields over, as this will not preserve the correct field alignment. Instead, declare an \texttt{isbits} struct type and use that instead. Unnamed structs are not possible in the translation to Julia.



Packed structs and union declarations are not supported by Julia.



You can get an approximation of a \texttt{union} if you know, a priori, the field that will have the greatest size (potentially including padding). When translating your fields to Julia, declare the Julia field to be only of that type.



Arrays of parameters can be expressed with \texttt{NTuple}.  For example, the struct in C notation written as




\begin{lstlisting}
struct B {
    int A[3];
};

b_a_2 = B.A[2];
\end{lstlisting}



can be written in Julia as




\begin{minted}{julia}
struct B
    A::NTuple{3, Cint}
end

b_a_2 = B.A[3]  # note the difference in indexing (1-based in Julia, 0-based in C)
\end{minted}



Arrays of unknown size (C99-compliant variable length structs specified by \texttt{[]} or \texttt{[0]}) are not directly supported. Often the best way to deal with these is to deal with the byte offsets directly. For example, if a C library declared a proper string type and returned a pointer to it:




\begin{lstlisting}
struct String {
    int strlen;
    char data[];
};
\end{lstlisting}



In Julia, we can access the parts independently to make a copy of that string:




\begin{minted}{julia}
str = from_c::Ptr{Cvoid}
len = unsafe_load(Ptr{Cint}(str))
unsafe_string(str + Core.sizeof(Cint), len)
\end{minted}



\hypertarget{7624173302473303801}{}


\subsection{Type Parameters}



The type arguments to \texttt{ccall} and \texttt{@cfunction} are evaluated statically, when the method containing the usage is defined. They therefore must take the form of a literal tuple, not a variable, and cannot reference local variables.



This may sound like a strange restriction, but remember that since C is not a dynamic language like Julia, its functions can only accept argument types with a statically-known, fixed signature.



However, while the type layout must be known statically to compute the intended C ABI, the static parameters of the function are considered to be part of this static environment. The static parameters of the function may be used as type parameters in the call signature, as long as they don{\textquotesingle}t affect the layout of the type. For example, \texttt{f(x::T) where \{T\} = ccall(:valid, Ptr\{T\}, (Ptr\{T\},), x)} is valid, since \texttt{Ptr} is always a word-size primitive type. But, \texttt{g(x::T) where \{T\} = ccall(:notvalid, T, (T,), x)} is not valid, since the type layout of \texttt{T} is not known statically.



\hypertarget{12802490213714574525}{}


\subsection{SIMD 值}



Note: This feature is currently implemented on 64-bit x86 and AArch64 platforms only.



If a C/C++ routine has an argument or return value that is a native SIMD type, the corresponding Julia type is a homogeneous tuple of \texttt{VecElement} that naturally maps to the SIMD type.  Specifically:



\begin{quote}
\begin{itemize}
\item The tuple must be the same size as the SIMD type. For example, a tuple representing an \texttt{\_\_m128} on x86 must have a size of 16 bytes.


\item The element type of the tuple must be an instance of \texttt{VecElement\{T\}} where \texttt{T} is a primitive type that is 1, 2, 4 or 8 bytes.

\end{itemize}
\end{quote}


For instance, consider this C routine that uses AVX intrinsics:




\begin{lstlisting}
#include <immintrin.h>

__m256 dist( __m256 a, __m256 b ) {
    return _mm256_sqrt_ps(_mm256_add_ps(_mm256_mul_ps(a, a),
                                        _mm256_mul_ps(b, b)));
}
\end{lstlisting}



The following Julia code calls \texttt{dist} using \texttt{ccall}:




\begin{minted}{julia}
const m256 = NTuple{8, VecElement{Float32}}

a = m256(ntuple(i -> VecElement(sin(Float32(i))), 8))
b = m256(ntuple(i -> VecElement(cos(Float32(i))), 8))

function call_dist(a::m256, b::m256)
    ccall((:dist, "libdist"), m256, (m256, m256), a, b)
end

println(call_dist(a,b))
\end{minted}



The host machine must have the requisite SIMD registers.  For example, the code above will not work on hosts without AVX support.



\hypertarget{10794774929021837783}{}


\subsection{内存所有权}



\textbf{malloc/free}



Memory allocation and deallocation of such objects must be handled by calls to the appropriate cleanup routines in the libraries being used, just like in any C program. Do not try to free an object received from a C library with \hyperlink{1633533624062187737}{\texttt{Libc.free}} in Julia, as this may result in the \texttt{free} function being called via the wrong library and cause the process to abort. The reverse (passing an object allocated in Julia to be freed by an external library) is equally invalid.



\hypertarget{13734604265364549635}{}


\subsection{何时使用 T、Ptr\{T\} 以及 Ref\{T\}}



In Julia code wrapping calls to external C routines, ordinary (non-pointer) data should be declared to be of type \texttt{T} inside the \hyperlink{14245046751182637566}{\texttt{ccall}}, as they are passed by value.  For C code accepting pointers, \hyperlink{7936024700322877457}{\texttt{Ref\{T\}}} should generally be used for the types of input arguments, allowing the use of pointers to memory managed by either Julia or C through the implicit call to \hyperlink{16487788729383051927}{\texttt{Base.cconvert}}. In contrast, pointers returned by the C function called should be declared to be of output type \hyperlink{10630331440513004826}{\texttt{Ptr\{T\}}}, reflecting that the memory pointed to is managed by C only. Pointers contained in C structs should be represented as fields of type \texttt{Ptr\{T\}} within the corresponding Julia struct types designed to mimic the internal structure of corresponding C structs.



In Julia code wrapping calls to external Fortran routines, all input arguments should be declared as of type \texttt{Ref\{T\}}, as Fortran passes all variables by pointers to memory locations. The return type should either be \texttt{Cvoid} for Fortran subroutines, or a \texttt{T} for Fortran functions returning the type \texttt{T}.



\hypertarget{6114319820079574946}{}


\section{Mapping C Functions to Julia}



\hypertarget{14464013813592582244}{}


\subsection{\texttt{ccall} / \texttt{@cfunction} argument translation guide}



For translating a C argument list to Julia:



\begin{itemize}
\item \texttt{T}, where \texttt{T} is one of the primitive types: \texttt{char}, \texttt{int}, \texttt{long}, \texttt{short}, \texttt{float}, \texttt{double}, \texttt{complex}, \texttt{enum} or any of their \texttt{typedef} equivalents

\begin{itemize}
\item \texttt{T}, where \texttt{T} is an equivalent Julia Bits Type (per the table above)


\item if \texttt{T} is an \texttt{enum}, the argument type should be equivalent to \texttt{Cint} or \texttt{Cuint}


\item argument value will be copied (passed by value)

\end{itemize}

\item \texttt{struct T} (including typedef to a struct)

\begin{itemize}
\item \texttt{T}, where \texttt{T} is a Julia leaf type


\item argument value will be copied (passed by value)

\end{itemize}

\item \texttt{void*}

\begin{itemize}
\item depends on how this parameter is used, first translate this to the intended pointer type, then determine the Julia equivalent using the remaining rules in this list


\item this argument may be declared as \texttt{Ptr\{Cvoid\}}, if it really is just an unknown pointer

\end{itemize}

\item \texttt{jl\_value\_t*}

\begin{itemize}
\item \texttt{Any}


\item argument value must be a valid Julia object

\end{itemize}

\item \texttt{jl\_value\_t**}

\begin{itemize}
\item \texttt{Ref\{Any\}}


\item argument value must be a valid Julia object (or \texttt{C\_NULL})

\end{itemize}

\item \texttt{T*}

\begin{itemize}
\item \texttt{Ref\{T\}}, where \texttt{T} is the Julia type corresponding to \texttt{T}


\item argument value will be copied if it is an \texttt{isbits} type otherwise, the value must be a valid Julia object

\end{itemize}

\item \texttt{T (*)(...)} (e.g. a pointer to a function)

\begin{itemize}
\item \texttt{Ptr\{Cvoid\}} (you may need to use \hyperlink{11617107520401351255}{\texttt{@cfunction}} explicitly to create this pointer)

\end{itemize}

\item \texttt{...} (e.g. a vararg)

\begin{itemize}
\item \texttt{T...}, where \texttt{T} is the Julia type


\item currently unsupported by \texttt{@cfunction}

\end{itemize}

\item \texttt{va\_arg}

\begin{itemize}
\item not supported by \texttt{ccall} or \texttt{@cfunction}

\end{itemize}
\end{itemize}


\hypertarget{13507762415394207754}{}


\subsection{\texttt{ccall} / \texttt{@cfunction} return type translation guide}



For translating a C return type to Julia:



\begin{itemize}
\item \texttt{void}

\begin{itemize}
\item \texttt{Cvoid} (this will return the singleton instance \texttt{nothing::Cvoid})

\end{itemize}

\item \texttt{T}, where \texttt{T} is one of the primitive types: \texttt{char}, \texttt{int}, \texttt{long}, \texttt{short}, \texttt{float}, \texttt{double}, \texttt{complex}, \texttt{enum} or any of their \texttt{typedef} equivalents

\begin{itemize}
\item \texttt{T}, where \texttt{T} is an equivalent Julia Bits Type (per the table above)


\item if \texttt{T} is an \texttt{enum}, the argument type should be equivalent to \texttt{Cint} or \texttt{Cuint}


\item argument value will be copied (returned by-value)

\end{itemize}

\item \texttt{struct T} (including typedef to a struct)

\begin{itemize}
\item \texttt{T}, where \texttt{T} is a Julia Leaf Type


\item argument value will be copied (returned by-value)

\end{itemize}

\item \texttt{void*}

\begin{itemize}
\item depends on how this parameter is used, first translate this to the intended pointer type, then determine the Julia equivalent using the remaining rules in this list


\item this argument may be declared as \texttt{Ptr\{Cvoid\}}, if it really is just an unknown pointer

\end{itemize}

\item \texttt{jl\_value\_t*}

\begin{itemize}
\item \texttt{Any}


\item argument value must be a valid Julia object

\end{itemize}

\item \texttt{jl\_value\_t**}

\begin{itemize}
\item \texttt{Ptr\{Any\}} (\texttt{Ref\{Any\}} is invalid as a return type)


\item argument value must be a valid Julia object (or \texttt{C\_NULL})

\end{itemize}

\item \texttt{T*}

\begin{itemize}
\item If the memory is already owned by Julia, or is an \texttt{isbits} type, and is known to be non-null:

\begin{itemize}
\item \texttt{Ref\{T\}}, where \texttt{T} is the Julia type corresponding to \texttt{T}


\item a return type of \texttt{Ref\{Any\}} is invalid, it should either be \texttt{Any} (corresponding to \texttt{jl\_value\_t*}) or \texttt{Ptr\{Any\}} (corresponding to \texttt{jl\_value\_t**})


\item C \textbf{MUST NOT} modify the memory returned via \texttt{Ref\{T\}} if \texttt{T} is an \texttt{isbits} type

\end{itemize}

\item If the memory is owned by C:

\begin{itemize}
\item \texttt{Ptr\{T\}}, where \texttt{T} is the Julia type corresponding to \texttt{T}

\end{itemize}
\end{itemize}

\item \texttt{T (*)(...)} (e.g. a pointer to a function)

\begin{itemize}
\item \texttt{Ptr\{Cvoid\}} (you may need to use \hyperlink{11617107520401351255}{\texttt{@cfunction}} explicitly to create this pointer)

\end{itemize}
\end{itemize}


\hypertarget{12317000517353378133}{}


\subsection{Passing Pointers for Modifying Inputs}



Because C doesn{\textquotesingle}t support multiple return values, often C functions will take pointers to data that the function will modify. To accomplish this within a \hyperlink{14245046751182637566}{\texttt{ccall}}, you need to first encapsulate the value inside a \hyperlink{7936024700322877457}{\texttt{Ref\{T\}}} of the appropriate type. When you pass this \texttt{Ref} object as an argument, Julia will automatically pass a C pointer to the encapsulated data:




\begin{minted}{julia}
width = Ref{Cint}(0)
range = Ref{Cfloat}(0)
ccall(:foo, Cvoid, (Ref{Cint}, Ref{Cfloat}), width, range)
\end{minted}



Upon return, the contents of \texttt{width} and \texttt{range} can be retrieved (if they were changed by \texttt{foo}) by \texttt{width[]} and \texttt{range[]}; that is, they act like zero-dimensional arrays.



\hypertarget{16047542394965419966}{}


\section{C Wrapper Examples}



Let{\textquotesingle}s start with a simple example of a C wrapper that returns a \texttt{Ptr} type:




\begin{minted}{julia}
mutable struct gsl_permutation
end

# The corresponding C signature is
#     gsl_permutation * gsl_permutation_alloc (size_t n);
function permutation_alloc(n::Integer)
    output_ptr = ccall(
        (:gsl_permutation_alloc, :libgsl), # name of C function and library
        Ptr{gsl_permutation},              # output type
        (Csize_t,),                        # tuple of input types
        n                                  # name of Julia variable to pass in
    )
    if output_ptr == C_NULL # Could not allocate memory
        throw(OutOfMemoryError())
    end
    return output_ptr
end
\end{minted}



The \href{https://www.gnu.org/software/gsl/}{GNU Scientific Library} (here assumed to be accessible through \texttt{:libgsl}) defines an opaque pointer, \texttt{gsl\_permutation *}, as the return type of the C function \texttt{gsl\_permutation\_alloc}. As user code never has to look inside the \texttt{gsl\_permutation} struct, the corresponding Julia wrapper simply needs a new type declaration, \texttt{gsl\_permutation}, that has no internal fields and whose sole purpose is to be placed in the type parameter of a \texttt{Ptr} type.  The return type of the \hyperlink{14245046751182637566}{\texttt{ccall}} is declared as \texttt{Ptr\{gsl\_permutation\}}, since the memory allocated and pointed to by \texttt{output\_ptr} is controlled by C.



The input \texttt{n} is passed by value, and so the function{\textquotesingle}s input signature is simply declared as \texttt{(Csize\_t,)} without any \texttt{Ref} or \texttt{Ptr} necessary. (If the wrapper was calling a Fortran function instead, the corresponding function input signature would instead be \texttt{(Ref\{Csize\_t\},)}, since Fortran variables are passed by pointers.) Furthermore, \texttt{n} can be any type that is convertible to a \texttt{Csize\_t} integer; the \hyperlink{14245046751182637566}{\texttt{ccall}} implicitly calls \hyperlink{16487788729383051927}{\texttt{Base.cconvert(Csize\_t, n)}}.



Here is a second example wrapping the corresponding destructor:




\begin{minted}{julia}
# The corresponding C signature is
#     void gsl_permutation_free (gsl_permutation * p);
function permutation_free(p::Ref{gsl_permutation})
    ccall(
        (:gsl_permutation_free, :libgsl), # name of C function and library
        Cvoid,                             # output type
        (Ref{gsl_permutation},),          # tuple of input types
        p                                 # name of Julia variable to pass in
    )
end
\end{minted}



Here, the input \texttt{p} is declared to be of type \texttt{Ref\{gsl\_permutation\}}, meaning that the memory that \texttt{p} points to may be managed by Julia or by C. A pointer to memory allocated by C should be of type \texttt{Ptr\{gsl\_permutation\}}, but it is convertible using \hyperlink{16487788729383051927}{\texttt{Base.cconvert}} and therefore



Now if you look closely enough at this example, you may notice that it is incorrect, given our explanation above of preferred declaration types. Do you see it? The function we are calling is going to free the memory. This type of operation cannot be given a Julia object (it will crash or cause memory corruption). Therefore, it may be preferable to declare the \texttt{p} type as \texttt{Ptr\{gsl\_permutation \}}, to make it harder for the user to mistakenly pass another sort of object there than one obtained via \texttt{gsl\_permutation\_alloc}.



If the C wrapper never expects the user to pass pointers to memory managed by Julia, then using \texttt{p::Ptr\{gsl\_permutation\}} for the method signature of the wrapper and similarly in the \hyperlink{14245046751182637566}{\texttt{ccall}} is also acceptable.



Here is a third example passing Julia arrays:




\begin{minted}{julia}
# The corresponding C signature is
#    int gsl_sf_bessel_Jn_array (int nmin, int nmax, double x,
#                                double result_array[])
function sf_bessel_Jn_array(nmin::Integer, nmax::Integer, x::Real)
    if nmax < nmin
        throw(DomainError())
    end
    result_array = Vector{Cdouble}(undef, nmax - nmin + 1)
    errorcode = ccall(
        (:gsl_sf_bessel_Jn_array, :libgsl), # name of C function and library
        Cint,                               # output type
        (Cint, Cint, Cdouble, Ref{Cdouble}),# tuple of input types
        nmin, nmax, x, result_array         # names of Julia variables to pass in
    )
    if errorcode != 0
        error("GSL error code $errorcode")
    end
    return result_array
end
\end{minted}



The C function wrapped returns an integer error code; the results of the actual evaluation of the Bessel J function populate the Julia array \texttt{result\_array}. This variable is declared as a \texttt{Ref\{Cdouble\}}, since its memory is allocated and managed by Julia. The implicit call to \hyperlink{16487788729383051927}{\texttt{Base.cconvert(Ref\{Cdouble\}, result\_array)}} unpacks the Julia pointer to a Julia array data structure into a form understandable by C.



\hypertarget{17069829158327603815}{}


\section{Fortran Wrapper Example}



The following example utilizes ccall to call a function in a common Fortran library (libBLAS) to computes a dot product. Notice that the argument mapping is a bit different here than above, as we need to map from Julia to Fortran.  On every argument type, we specify \texttt{Ref} or \texttt{Ptr}. This mangling convention may be specific to your fortran compiler and operating system, and is likely undocumented. However, wrapping each in a \texttt{Ref} (or \texttt{Ptr}, where equivalent) is a frequent requirement of Fortran compiler implementations:




\begin{minted}{julia}
function compute_dot(DX::Vector{Float64}, DY::Vector{Float64})
    @assert length(DX) == length(DY)
    n = length(DX)
    incx = incy = 1
    product = ccall((:ddot_, "libLAPACK"),
                    Float64,
                    (Ref{Int32}, Ptr{Float64}, Ref{Int32}, Ptr{Float64}, Ref{Int32}),
                    n, DX, incx, DY, incy)
    return product
end
\end{minted}



\hypertarget{6047327473342163880}{}


\section{垃圾回收安全}



When passing data to a \hyperlink{14245046751182637566}{\texttt{ccall}}, it is best to avoid using the \hyperlink{8901246211940014300}{\texttt{pointer}} function. Instead define a convert method and pass the variables directly to the \hyperlink{14245046751182637566}{\texttt{ccall}}. \hyperlink{14245046751182637566}{\texttt{ccall}} automatically arranges that all of its arguments will be preserved from garbage collection until the call returns. If a C API will store a reference to memory allocated by Julia, after the \hyperlink{14245046751182637566}{\texttt{ccall}} returns, you must ensure that the object remains visible to the garbage collector. The suggested way to do this is to make a global variable of type \texttt{Array\{Ref,1\}} to hold these values, until the C library notifies you that it is finished with them.



Whenever you have created a pointer to Julia data, you must ensure the original data exists until you have finished using the pointer. Many methods in Julia such as \hyperlink{13744149973765810952}{\texttt{unsafe\_load}} and \hyperlink{2825695355940841177}{\texttt{String}} make copies of data instead of taking ownership of the buffer, so that it is safe to free (or alter) the original data without affecting Julia. A notable exception is \hyperlink{14566494858943689253}{\texttt{unsafe\_wrap}} which, for performance reasons, shares (or can be told to take ownership of) the underlying buffer.



The garbage collector does not guarantee any order of finalization. That is, if \texttt{a} contained a reference to \texttt{b} and both \texttt{a} and \texttt{b} are due for garbage collection, there is no guarantee that \texttt{b} would be finalized after \texttt{a}. If proper finalization of \texttt{a} depends on \texttt{b} being valid, it must be handled in other ways.



\hypertarget{14397309909238125480}{}


\section{Non-constant Function Specifications}



A \texttt{(name, library)} function specification must be a constant expression. However, it is possible to use computed values as function names by staging through \hyperlink{7507639810592563424}{\texttt{eval}} as follows:




\begin{lstlisting}
@eval ccall(($(string("a", "b")), "lib"), ...
\end{lstlisting}



This expression constructs a name using \texttt{string}, then substitutes this name into a new \hyperlink{14245046751182637566}{\texttt{ccall}} expression, which is then evaluated. Keep in mind that \texttt{eval} only operates at the top level, so within this expression local variables will not be available (unless their values are substituted with \texttt{\$}). For this reason, \texttt{eval} is typically only used to form top-level definitions, for example when wrapping libraries that contain many similar functions. A similar example can be constructed for \hyperlink{11617107520401351255}{\texttt{@cfunction}}.



However, doing this will also be very slow and leak memory, so you should usually avoid this and instead keep reading. The next section discusses how to use indirect calls to efficiently achieve a similar effect.



\hypertarget{18148794268039727850}{}


\section{非直接调用}



The first argument to \hyperlink{14245046751182637566}{\texttt{ccall}} can also be an expression evaluated at run time. In this case, the expression must evaluate to a \texttt{Ptr}, which will be used as the address of the native function to call. This behavior occurs when the first \hyperlink{14245046751182637566}{\texttt{ccall}} argument contains references to non-constants, such as local variables, function arguments, or non-constant globals.



For example, you might look up the function via \texttt{dlsym}, then cache it in a shared reference for that session. For example:




\begin{minted}{julia}
macro dlsym(func, lib)
    z = Ref{Ptr{Cvoid}}(C_NULL)
    quote
        let zlocal = $z[]
            if zlocal == C_NULL
                zlocal = dlsym($(esc(lib))::Ptr{Cvoid}, $(esc(func)))::Ptr{Cvoid}
                $z[] = $zlocal
            end
            zlocal
        end
    end
end

mylibvar = Libdl.dlopen("mylib")
ccall(@dlsym("myfunc", mylibvar), Cvoid, ())
\end{minted}



\hypertarget{11059323548475736008}{}


\section{Closure cfunctions}



The first argument to \hyperlink{11617107520401351255}{\texttt{@cfunction}} can be marked with a \texttt{\$}, in which case the return value will instead be a \texttt{struct CFunction} which closes over the argument. You must ensure that this return object is kept alive until all uses of it are done. The contents and code at the cfunction pointer will be erased via a \hyperlink{4805357059330171046}{\texttt{finalizer}} when this reference is dropped and atexit. This is not usually needed, since this functionality is not present in C, but can be useful for dealing with ill-designed APIs which don{\textquotesingle}t provide a separate closure environment parameter.




\begin{minted}{julia}
function qsort(a::Vector{T}, cmp) where T
    isbits(T) || throw(ArgumentError("this method can only qsort isbits arrays"))
    callback = @cfunction $cmp Cint (Ref{T}, Ref{T})
    # Here, `callback` isa Base.CFunction, which will be converted to Ptr{Cvoid}
    # (and protected against finalization) by the ccall
    ccall(:qsort, Cvoid, (Ptr{T}, Csize_t, Csize_t, Ptr{Cvoid}),
        a, length(a), Base.elsize(a), callback)
    # We could instead use:
    #    GC.@preserve callback begin
    #        use(Base.unsafe_convert(Ptr{Cvoid}, callback))
    #    end
    # if we needed to use it outside of a `ccall`
    return a
end
\end{minted}



\begin{quote}
\textbf{Note}

Closure \hyperlink{11617107520401351255}{\texttt{@cfunction}} rely on LLVM trampolines, which are not available on all platforms (for example ARM and PowerPC).

\end{quote}


\hypertarget{10600174375111035081}{}


\section{关闭库}



It is sometimes useful to close (unload) a library so that it can be reloaded. For instance, when developing C code for use with Julia, one may need to compile, call the C code from Julia, then close the library, make an edit, recompile, and load in the new changes. One can either restart Julia or use the \texttt{Libdl} functions to manage the library explicitly, such as:




\begin{minted}{julia}
lib = Libdl.dlopen("./my_lib.so") # Open the library explicitly.
sym = Libdl.dlsym(lib, :my_fcn)   # Get a symbol for the function to call.
ccall(sym, ...) # Use the pointer `sym` instead of the (symbol, library) tuple (remaining arguments are the
same).  Libdl.dlclose(lib) # Close the library explicitly.
\end{minted}



Note that when using \texttt{ccall} with the tuple input (e.g., \texttt{ccall((:my\_fcn, {\textquotedbl}./my\_lib.so{\textquotedbl}), ...)}), the library is opened implicitly and it may not be explicitly closed.



\hypertarget{14242265699396110950}{}


\section{调用规约}



The second argument to \hyperlink{14245046751182637566}{\texttt{ccall}} can optionally be a calling convention specifier (immediately preceding return type). Without any specifier, the platform-default C calling convention is used. Other supported conventions are: \texttt{stdcall}, \texttt{cdecl}, \texttt{fastcall}, and \texttt{thiscall} (no-op on 64-bit Windows). For example (from \texttt{base/libc.jl}) we see the same \texttt{gethostname}\hyperlink{14245046751182637566}{\texttt{ccall}} as above, but with the correct signature for Windows:




\begin{minted}{julia}
hn = Vector{UInt8}(undef, 256)
err = ccall(:gethostname, stdcall, Int32, (Ptr{UInt8}, UInt32), hn, length(hn))
\end{minted}



请参阅 \href{http://llvm.org/docs/LangRef.html\#calling-conventions}{LLVM Language Reference} 来获得更多信息。



There is one additional special calling convention \hyperlink{12406828992589210838}{\texttt{llvmcall}}, which allows inserting calls to LLVM intrinsics directly. This can be especially useful when targeting unusual platforms such as GPGPUs. For example, for \href{http://llvm.org/docs/NVPTXUsage.html}{CUDA}, we need to be able to read the thread index:




\begin{minted}{julia}
ccall("llvm.nvvm.read.ptx.sreg.tid.x", llvmcall, Int32, ())
\end{minted}



As with any \texttt{ccall}, it is essential to get the argument signature exactly correct. Also, note that there is no compatibility layer that ensures the intrinsic makes sense and works on the current target, unlike the equivalent Julia functions exposed by \texttt{Core.Intrinsics}.



\hypertarget{6755654760318927882}{}


\section{访问全局变量}



Global variables exported by native libraries can be accessed by name using the \hyperlink{2746947069730856184}{\texttt{cglobal}} function. The arguments to \hyperlink{2746947069730856184}{\texttt{cglobal}} are a symbol specification identical to that used by \hyperlink{14245046751182637566}{\texttt{ccall}}, and a type describing the value stored in the variable:




\begin{minted}{jlcon}
julia> cglobal((:errno, :libc), Int32)
Ptr{Int32} @0x00007f418d0816b8
\end{minted}



The result is a pointer giving the address of the value. The value can be manipulated through this pointer using \hyperlink{13744149973765810952}{\texttt{unsafe\_load}} and \hyperlink{4579672834750013041}{\texttt{unsafe\_store!}}.



\begin{quote}
\textbf{Note}

This \texttt{errno} symbol may not be found in a library named {\textquotedbl}libc{\textquotedbl}, as this is an implementation detail of your system compiler. Typically standard library symbols should be accessed just by name, allowing the compiler to fill in the correct one. Also, however, the \texttt{errno} symbol shown in this example is special in most compilers, and so the value seen here is probably not what you expect or want. Compiling the equivalent code in C on any multi-threaded-capable system would typically actually call a different function (via macro preprocessor overloading), and may give a different result than the legacy value printed here.

\end{quote}


\hypertarget{14428977823562595292}{}


\section{Accessing Data through a Pointer}



The following methods are described as {\textquotedbl}unsafe{\textquotedbl} because a bad pointer or type declaration can cause Julia to terminate abruptly.



Given a \texttt{Ptr\{T\}}, the contents of type \texttt{T} can generally be copied from the referenced memory into a Julia object using \texttt{unsafe\_load(ptr, [index])}. The index argument is optional (default is 1), and follows the Julia-convention of 1-based indexing. This function is intentionally similar to the behavior of \hyperlink{13720608614876840481}{\texttt{getindex}} and \hyperlink{1309244355901386657}{\texttt{setindex!}} (e.g. \texttt{[]} access syntax).



The return value will be a new object initialized to contain a copy of the contents of the referenced memory. The referenced memory can safely be freed or released.



If \texttt{T} is \texttt{Any}, then the memory is assumed to contain a reference to a Julia object (a \texttt{jl\_value\_t*}), the result will be a reference to this object, and the object will not be copied. You must be careful in this case to ensure that the object was always visible to the garbage collector (pointers do not count, but the new reference does) to ensure the memory is not prematurely freed. Note that if the object was not originally allocated by Julia, the new object will never be finalized by Julia{\textquotesingle}s garbage collector.  If the \texttt{Ptr} itself is actually a \texttt{jl\_value\_t*}, it can be converted back to a Julia object reference by \hyperlink{10812596548944930674}{\texttt{unsafe\_pointer\_to\_objref(ptr)}}. (Julia values \texttt{v} can be converted to \texttt{jl\_value\_t*} pointers, as \texttt{Ptr\{Cvoid\}}, by calling \hyperlink{9366554937543398846}{\texttt{pointer\_from\_objref(v)}}.)



The reverse operation (writing data to a \texttt{Ptr\{T\}}), can be performed using \hyperlink{4579672834750013041}{\texttt{unsafe\_store!(ptr, value, [index])}}. Currently, this is only supported for primitive types or other pointer-free (\texttt{isbits}) immutable struct types.



Any operation that throws an error is probably currently unimplemented and should be posted as a bug so that it can be resolved.



If the pointer of interest is a plain-data array (primitive type or immutable struct), the function \hyperlink{14566494858943689253}{\texttt{unsafe\_wrap(Array, ptr,dims, own = false)}} may be more useful. The final parameter should be true if Julia should {\textquotedbl}take ownership{\textquotedbl} of the underlying buffer and call \texttt{free(ptr)} when the returned \texttt{Array} object is finalized.  If the \texttt{own} parameter is omitted or false, the caller must ensure the buffer remains in existence until all access is complete.



Arithmetic on the \texttt{Ptr} type in Julia (e.g. using \texttt{+}) does not behave the same as C{\textquotesingle}s pointer arithmetic. Adding an integer to a \texttt{Ptr} in Julia always moves the pointer by some number of \emph{bytes}, not elements. This way, the address values obtained from pointer arithmetic do not depend on the element types of pointers.



\hypertarget{11710684987427742051}{}


\section{线程安全}



Some C libraries execute their callbacks from a different thread, and since Julia isn{\textquotesingle}t thread-safe you{\textquotesingle}ll need to take some extra precautions. In particular, you{\textquotesingle}ll need to set up a two-layered system: the C callback should only \emph{schedule} (via Julia{\textquotesingle}s event loop) the execution of your {\textquotedbl}real{\textquotedbl} callback. To do this, create an \hyperlink{6110056827764884232}{\texttt{AsyncCondition}} object and \hyperlink{13761789780433862250}{\texttt{wait}} on it:




\begin{minted}{julia}
cond = Base.AsyncCondition()
wait(cond)
\end{minted}



传递给 C 的回调应该只通过 \hyperlink{14245046751182637566}{\texttt{ccall}} 将 \texttt{cond.handle} 作为参数传递给 \texttt{:uv\_async\_send} 并调用,注意避免任何内存分配操作或与 Julia 运行时的其他交互。



注意,事件可能会合并,因此对 \texttt{uv\_async\_send} 的多个调用可能会导致对该条件的单个唤醒通知。



\hypertarget{9551685449391366384}{}


\section{关于 Callbacks 的更多内容}



关于如何传递 callback 到 C 库的更多细节,请参考此\href{https://julialang.org/blog/2013/05/callback}{博客}。



\hypertarget{4039606750368114245}{}


\section{C++}



如需要直接易用的C++接口,即直接用Julia写封装代码,请参考 \href{https://github.com/Keno/Cxx.jl}{Cxx}。如需封装C++库的工具,即用C++写封装/胶水代码,请参考\href{https://github.com/JuliaInterop/CxxWrap.jl}{CxxWrap}。



\footnotetext[1]{Non-library function calls in both C and Julia can be inlined and thus may have even less overhead than calls to shared library functions. The point above is that the cost of actually doing foreign function call is about the same as doing a call in either native language.

}


\footnotetext[2]{The \href{https://github.com/ihnorton/Clang.jl}{Clang package} can be used to auto-generate Julia code from a C header file.

}


\hypertarget{4348488381006315694}{}


\chapter{处理操作系统差异}



当编写跨平台的应用或库时,通常需要考虑到操作系统之间的差异。变量 \texttt{Sys.KERNEL} 可以用于这些场合。在 \texttt{Sys} 模块中有一些函数将会使这些事情更加简单:\texttt{isunix}、 \texttt{islinux}、\texttt{isapple}、\texttt{isbsd}、\texttt{isfreebsd} 以及 \texttt{iswindows}。这些函数可以按如下方式使用:




\begin{minted}{julia}
if Sys.iswindows()
    windows_specific_thing(a)
end
\end{minted}



注意,\texttt{islinux}、\texttt{isapple} 和 \texttt{isfreebsd} 是 \texttt{isunix} 完全互斥的子集。另外,有一个宏 \texttt{@static} 可以使用这些函数有条件地隐藏无效代码,如以下示例所示。



简单例子:




\begin{lstlisting}
ccall((@static Sys.iswindows() ? :_fopen : :fopen), ...)
\end{lstlisting}



复杂例子:




\begin{minted}{julia}
@static if Sys.islinux()
    linux_specific_thing(a)
else
    generic_thing(a)
end
\end{minted}



在链式嵌套的条件表达式中(包括 \texttt{if}/\texttt{elseif}/\texttt{end}),\texttt{@static} 必须在每一层都调用(括号是可选的,但是为了可读性,建议添加)。




\begin{minted}{julia}
@static Sys.iswindows() ? :a : (@static Sys.isapple() ? :b : :c)
\end{minted}



\hypertarget{4919662572867023119}{}


\chapter{环境变量}



Julia 可以配置许多环境变量,一种常见的方式是直接配置操作系统环境变量,另一种更便携的方式是在 Julia 中配置。假设你要将环境变量 \texttt{JULIA\_EDITOR} 设置为 \texttt{vim},可以直接在 REPL 中输入 \texttt{ENV[{\textquotedbl}JULIA\_EDITOR{\textquotedbl}] = {\textquotedbl}vim{\textquotedbl}}(请根据具体情况对此进行修改),也可以将其添加到用户主目录中的配置文件 \texttt{{\textasciitilde}/.julia/config/startup.jl},这样做会使其永久生效。环境变量的当前值是通过执行 \texttt{ENV[{\textquotedbl}JULIA\_EDITOR{\textquotedbl}]} 来确定的。



The environment variables that Julia uses generally start with \texttt{JULIA}. If \hyperlink{11698106121547091928}{\texttt{InteractiveUtils.versioninfo}} is called with the keyword \texttt{verbose=true}, then the output will list any defined environment variables relevant for Julia, including those which include \texttt{JULIA} in their names.



\begin{quote}
\textbf{Note}

Some variables, such as \texttt{JULIA\_NUM\_THREADS} and \texttt{JULIA\_PROJECT}, need to be set before Julia starts, therefore adding these to \texttt{{\textasciitilde}/.julia/config/startup.jl} is too late in the startup process. In Bash, environment variables can either be set manually by running, e.g., \texttt{export JULIA\_NUM\_THREADS=4} before starting Julia, or by adding the same command to \texttt{{\textasciitilde}/.bashrc} or \texttt{{\textasciitilde}/.bash\_profile} to set the variable each time Bash is started.

\end{quote}


\hypertarget{10129404461588265763}{}


\section{文件位置}



\hypertarget{11878722049876551255}{}


\subsection{\texttt{JULIA\_BINDIR}}



包含 Julia 可执行文件的目录的绝对路径,它会设置全局变量 \hyperlink{11034333937761980027}{\texttt{Sys.BINDIR}}。\texttt{\$JULIA\_BINDIR} 如果没有设置,那么 Julia 会在运行时确定 \texttt{Sys.BINDIR} 的值。



在默认情况下,可执行文件是指:




\begin{lstlisting}
$JULIA_BINDIR/julia
$JULIA_BINDIR/julia-debug
\end{lstlisting}



全局变量 \texttt{Base.DATAROOTDIR} 是一个从 \texttt{Sys.BINDIR} 到 Julia 数据目录的相对路径。




\begin{lstlisting}
$JULIA_BINDIR/$DATAROOTDIR/julia/base
\end{lstlisting}



上述路径是 Julia 最初搜索源文件的路径(通过 \texttt{Base.find\_source\_file()})。



同样,全局变量 \texttt{Base.SYSCONFDIR} 是一个到配置文件目录的相对路径。在默认情况下,Julia 会在下列文件中搜索 \texttt{startup.jl} 文件(通过 \texttt{Base.load\_julia\_startup()})




\begin{lstlisting}
$JULIA_BINDIR/$SYSCONFDIR/julia/startup.jl
$JULIA_BINDIR/../etc/julia/startup.jl
\end{lstlisting}



例如,一个 Linux 安装包的 Julia 可执行文件位于 \texttt{/bin/julia},\texttt{DATAROOTDIR} 为 \texttt{../share},\texttt{SYSCONFDIR} 为 \texttt{../etc},\texttt{JULIA\_BINDIR} 会被设置为 \texttt{/bin},会有一个源文件搜索路径:




\begin{lstlisting}
/share/julia/base
\end{lstlisting}



和一个全局配置文件搜索路径:




\begin{lstlisting}
/etc/julia/startup.jl
\end{lstlisting}



\hypertarget{4954349069727209817}{}


\subsection{\texttt{JULIA\_PROJECT}}



A directory path that indicates which project should be the initial active project. Setting this environment variable has the same effect as specifying the \texttt{--project} start-up option, but \texttt{--project} has higher precedence. If the variable is set to \texttt{@.} then Julia tries to find a project directory that contains \texttt{Project.toml} or \texttt{JuliaProject.toml} file from the current directory and its parents. See also the chapter on \href{@ref}{Code Loading}.



\begin{quote}
\textbf{Note}

\texttt{JULIA\_PROJECT} must be defined before starting julia; defining it in \texttt{startup.jl} is too late in the startup process.

\end{quote}


\hypertarget{1363234541366705734}{}


\subsection{\texttt{JULIA\_LOAD\_PATH}}



The \texttt{JULIA\_LOAD\_PATH} environment variable is used to populate the global Julia \hyperlink{17914149694871263675}{\texttt{LOAD\_PATH}} variable, which determines which packages can be loaded via \texttt{import} and \texttt{using} (see \href{@ref}{Code Loading}).



Unlike the shell \texttt{PATH} variable, empty entries in \texttt{JULIA\_LOAD\_PATH} are expanded to the default value of \texttt{LOAD\_PATH}, \texttt{[{\textquotedbl}@{\textquotedbl}, {\textquotedbl}@v\#.\#{\textquotedbl}, {\textquotedbl}@stdlib{\textquotedbl}]} when populating \texttt{LOAD\_PATH}. This allows easy appending, prepending, etc. of the load path value in shell scripts regardless of whether \texttt{JULIA\_LOAD\_PATH} is already set or not. For example, to prepend the directory \texttt{/foo/bar} to \texttt{LOAD\_PATH} just do




\begin{lstlisting}
export JULIA_LOAD_PATH="/foo/bar:$JULIA_LOAD_PATH"
\end{lstlisting}



If the \texttt{JULIA\_LOAD\_PATH} environment variable is already set, its old value will be prepended with \texttt{/foo/bar}. On the other hand, if \texttt{JULIA\_LOAD\_PATH} is not set, then it will be set to \texttt{/foo/bar:} which will expand to a \texttt{LOAD\_PATH} value of \texttt{[{\textquotedbl}/foo/bar{\textquotedbl}, {\textquotedbl}@{\textquotedbl}, {\textquotedbl}@v\#.\#{\textquotedbl}, {\textquotedbl}@stdlib{\textquotedbl}]}. If \texttt{JULIA\_LOAD\_PATH} is set to the empty string, it expands to an empty \texttt{LOAD\_PATH} array. In other words, the empty string is interpreted as a zero-element array, not a one-element array of the empty string. This behavior was chosen so that it would be possible to set an empty load path via the environment variable. If you want the default load path, either unset the environment variable or if it must have a value, set it to the string \texttt{:}.



\hypertarget{7415777056556214668}{}


\subsection{\texttt{JULIA\_DEPOT\_PATH}}



The \texttt{JULIA\_DEPOT\_PATH} environment variable is used to populate the global Julia \hyperlink{15271486679944781836}{\texttt{DEPOT\_PATH}} variable, which controls where the package manager, as well as Julia{\textquotesingle}s code loading mechanisms, look for package registries, installed packages, named environments, repo clones, cached compiled package images, configuration files, and the default location of the REPL{\textquotesingle}s history file.



Unlike the shell \texttt{PATH} variable but similar to \texttt{JULIA\_LOAD\_PATH}, empty entries in \texttt{JULIA\_DEPOT\_PATH} are expanded to the default value of \texttt{DEPOT\_PATH}. This allows easy appending, prepending, etc. of the depot path value in shell scripts regardless of whether \texttt{JULIA\_DEPOT\_PATH} is already set or not. For example, to prepend the directory \texttt{/foo/bar} to \texttt{DEPOT\_PATH} just do




\begin{lstlisting}
export JULIA_DEPOT_PATH="/foo/bar:$JULIA_DEPOT_PATH"
\end{lstlisting}



If the \texttt{JULIA\_DEPOT\_PATH} environment variable is already set, its old value will be prepended with \texttt{/foo/bar}. On the other hand, if \texttt{JULIA\_DEPOT\_PATH} is not set, then it will be set to \texttt{/foo/bar:} which will have the effect of prepending \texttt{/foo/bar} to the default depot path. If \texttt{JULIA\_DEPOT\_PATH} is set to the empty string, it expands to an empty \texttt{DEPOT\_PATH} array. In other words, the empty string is interpreted as a zero-element array, not a one-element array of the empty string. This behavior was chosen so that it would be possible to set an empty depot path via the environment variable. If you want the default depot path, either unset the environment variable or if it must have a value, set it to the string \texttt{:}.



\hypertarget{7464422147684280847}{}


\subsection{\texttt{JULIA\_HISTORY}}



REPL 历史文件中 \texttt{REPL.find\_hist\_file()} 的绝对路径。如果没有设置 \texttt{\$JULIA\_HISTORY},那么 \texttt{REPL.find\_hist\_file()} 默认为




\begin{lstlisting}
$(DEPOT_PATH[1])/logs/repl_history.jl
\end{lstlisting}



\hypertarget{3466341700032254278}{}


\section{外部应用}



\hypertarget{12177211249965413749}{}


\subsection{\texttt{JULIA\_SHELL}}



Julia 用来执行外部命令的 shell 的绝对路径(通过 \texttt{Base.repl\_cmd()})。默认为环境变量 \texttt{\$SHELL},如果 \texttt{\$SHELL} 未设置,则为 \texttt{/bin/sh}。



\begin{quote}
\textbf{Note}

在 Windows 上,此环境变量将被忽略,并且外部命令会直接被执行。

\end{quote}


\hypertarget{327473439132778011}{}


\subsection{\texttt{JULIA\_EDITOR}}



\texttt{InteractiveUtils.editor()} 的返回值–编辑器,例如,\hyperlink{10268751953828531961}{\texttt{InteractiveUtils.edit}},会启动偏好编辑器,比如 \texttt{vim}。



\texttt{\$JULIA\_EDITOR} 优先于 \texttt{\$VISUAL},而后者优先于 \texttt{\$EDITOR}。如果这些环境变量都没有设置,那么在 Windows 和 OS X 上会设置为 \texttt{open},或者 \texttt{/etc/alternatives/editor}(如果存在的话),否则为 \texttt{emacs}。



\hypertarget{5963067566394654848}{}


\section{并行}



\hypertarget{9501730119086472793}{}


\subsection{\texttt{JULIA\_CPU\_THREADS}}



改写全局变量 \hyperlink{5072616208401232599}{\texttt{Base.Sys.CPU\_THREADS}},逻辑 CPU 核心数。



\hypertarget{17625390316676574932}{}


\subsection{\texttt{JULIA\_WORKER\_TIMEOUT}}



一个 \hyperlink{5027751419500983000}{\texttt{Float64}} 值,用来确定 \texttt{Distributed.worker\_timeout()} 的值(默认:\texttt{60.0})。此函数提供 worker 进程在死亡之前等待 master 进程建立连接的秒数。



\hypertarget{7328254851646027731}{}


\subsection{\texttt{JULIA\_NUM\_THREADS}}



一个无符号 64 位整数(\texttt{uint64\_t}),用来设置 Julia 可用线程的最大数。如果 \texttt{\$JULIA\_NUM\_THREADS} 超过可用的物理 CPU 核心数,那么线程数设置为核心数。如果 \texttt{\$JULIA\_NUM\_THREADS} 不是正数或没有设置,或者无法通过系统调用确定 CPU 核心数,那么线程数就会被设置为 \texttt{1}。



\begin{quote}
\textbf{Note}

\texttt{JULIA\_NUM\_THREADS} 必须在启动 julia 前定义;在启动过程中于 \texttt{startup.jl} 中定义它为时已晚。

\end{quote}


\begin{quote}
\textbf{Julia 1.5}

In Julia 1.5 and above the number of threads can also be specified on startup using the \texttt{-t}/\texttt{--threads} command line argument.

\end{quote}


\hypertarget{10532418503410947704}{}


\subsection{\texttt{JULIA\_THREAD\_SLEEP\_THRESHOLD}}



如果被设置为字符串,并且以大小写敏感的子字符串 \texttt{{\textquotedbl}infinite{\textquotedbl}} 开头,那么z自旋线程从不睡眠。否则,\texttt{\$JULIA\_THREAD\_SLEEP\_THRESHOLD} 被解释为一个无符号 64 位整数(\texttt{uint64\_t}),并且提供以纳秒为单位的自旋线程睡眠的时间量。



\hypertarget{12794875033848178110}{}


\subsection{\texttt{JULIA\_EXCLUSIVE}}



如果设置为 \texttt{0} 以外的任何值,那么 Julia 的线程策略与在专用计算机上一致:主线程在 proc 0 上且线程间是关联的。否则,Julia 让操作系统处理线程策略。



\hypertarget{16378430867132816632}{}


\section{REPL 格式化输出}



决定 REPL 应当如何格式化输出的环境变量。通常,这些变量应当被设置为 \href{http://ascii-table.com/ansi-escape-sequences.php}{ANSI 终端转义序列}。Julia 提供了具有相同功能的高级接口;请参阅 \hyperlink{10670790884919535588}{Julia REPL} 章节。



\hypertarget{13891100922495428417}{}


\subsection{\texttt{JULIA\_ERROR\_COLOR}}



\texttt{Base.error\_color()}(默认值:亮红,\texttt{{\textquotedbl}{\textbackslash}033[91m{\textquotedbl}}),errors 在终端中的格式。



\hypertarget{17711733982596187514}{}


\subsection{\texttt{JULIA\_WARN\_COLOR}}



\texttt{Base.warn\_color()}(默认值:黄,\texttt{{\textquotedbl}{\textbackslash}033[93m{\textquotedbl}}),warnings 在终端中的格式。



\hypertarget{10994657891761481518}{}


\subsection{\texttt{JULIA\_INFO\_COLOR}}



\texttt{Base.info\_color()}(默认值:青,\texttt{{\textquotedbl}{\textbackslash}033[36m{\textquotedbl}}),info 在终端中的格式。



\hypertarget{7277467062996316804}{}


\subsection{\texttt{JULIA\_INPUT\_COLOR}}



\texttt{Base.input\_color()}(默认值:标准,\texttt{{\textquotedbl}{\textbackslash}033[0m{\textquotedbl}}),在终端中,输入应有的格式。



\hypertarget{11974933399373427924}{}


\subsection{\texttt{JULIA\_ANSWER\_COLOR}}



\texttt{Base.answer\_color()}(默认值:标准,\texttt{{\textquotedbl}{\textbackslash}033[0m{\textquotedbl}}),在终端中,输出应有的格式。



\hypertarget{1071250415559095053}{}


\subsection{\texttt{JULIA\_STACKFRAME\_LINEINFO\_COLOR}}



\texttt{Base.stackframe\_lineinfo\_color()}(默认值:粗体,\texttt{{\textquotedbl}{\textbackslash}033[1m{\textquotedbl}}),栈跟踪时行信息在终端中的格式。



\hypertarget{15357004504591912181}{}


\subsection{\texttt{JULIA\_STACKFRAME\_FUNCTION\_COLOR}}



\texttt{Base.stackframe\_function\_color()}(默认值:粗体,\texttt{{\textquotedbl}{\textbackslash}033[1m{\textquotedbl}}),栈跟踪期间函数调用在终端中的形式。



\hypertarget{14262570954781492291}{}


\section{调试和性能分析}



\hypertarget{8362221794188602098}{}


\subsection{\texttt{JULIA\_DEBUG}}



Enable debug logging for a file or module, see \href{@ref Logging}{\texttt{Logging}} for more information.



\hypertarget{17935900017233878037}{}


\subsection{\texttt{JULIA\_GC\_ALLOC\_POOL}, \texttt{JULIA\_GC\_ALLOC\_OTHER}, \texttt{JULIA\_GC\_ALLOC\_PRINT}}



这些环境变量取值为字符串,可以以字符 \texttt{‘r’} 开头,后接一个由三个带符号 64 位整数(\texttt{int64\_t})组成的、以冒号分割的列表的插值字符串。这个整数的三元组 \texttt{a:b:c} 代表算术序列 \texttt{a}, \texttt{a + b}, \texttt{a + 2*b}, ... \texttt{c}。



\begin{itemize}
\item 如果是第 \texttt{n} 次调用 \texttt{jl\_gc\_pool\_alloc()},并且 \texttt{n}   属于 \texttt{\$JULIA\_GC\_ALLOC\_POOL} 代表的算术序列,   那么垃圾回收是强制的。


\item 如果是第 \texttt{n} 次调用 \texttt{maybe\_collect()},并且 \texttt{n} 属于   \texttt{\$JULIA\_GC\_ALLOC\_OTHER} 代表的算术序列,那么垃圾   回收是强制的。


\item 如果是第 \texttt{n} 次调用 \texttt{jl\_gc\_alloc()},并且 \texttt{n} 属于   \texttt{\$JULIA\_GC\_ALLOC\_PRINT} 代表的算术序列,那么   调用 \texttt{jl\_gc\_pool\_alloc()} 和 \texttt{maybe\_collect()} 的次数会   被打印。

\end{itemize}


如果这些环境变量的值以字符 \texttt{‘r{\textquotesingle}} 开头,那么垃圾回收事件间的间隔是随机的。



\begin{quote}
\textbf{Note}

这些环境变量生效要求 Julia 在编译时带有垃圾收集调试支持(也就是,在构建配置中将 \texttt{WITH\_GC\_DEBUG\_ENV} 设置为 \texttt{1})。

\end{quote}


\hypertarget{15291982466110123243}{}


\subsection{\texttt{JULIA\_GC\_NO\_GENERATIONAL}}



如果设置为 \texttt{0} 以外的任何值,那么 Julia 的垃圾收集器将从不执行「快速扫描」内存。



\begin{quote}
\textbf{Note}

此环境变量生效要求 Julia 在编译时带有垃圾收集调试支持(也就是,在构建配置中将 \texttt{WITH\_GC\_DEBUG\_ENV} 设置为 \texttt{1})。

\end{quote}


\hypertarget{4439082668862420182}{}


\subsection{\texttt{JULIA\_GC\_WAIT\_FOR\_DEBUGGER}}



如果设置为 \texttt{0} 以外的任何值,Julia 的垃圾收集器每当出现严重错误时将等待调试器连接而不是中止。



\begin{quote}
\textbf{Note}

此环境变量生效要求 Julia 在编译时带有垃圾收集调试支持(也就是,在构建配置中将 \texttt{WITH\_GC\_DEBUG\_ENV} 设置为 \texttt{1})。

\end{quote}


\hypertarget{1100661411174026998}{}


\subsection{\texttt{ENABLE\_JITPROFILING}}



如果设置为 \texttt{0} 以外的任何值,那么编译器将为即时(JIT)性能分析创建并注册一个事件监听器。



\begin{quote}
\textbf{Note}

此变量生效要求 Julia 编译时带有 JIT 性能分析支持,请使用

\begin{itemize}
\item 英特尔的 \href{https://software.intel.com/en-us/intel-vtune-amplifier-xe}{VTune™ Amplifier}(在构建配置中将 \texttt{USE\_INTEL\_JITEVENTS} 设置为 \texttt{1}),或


\item \href{http://oprofile.sourceforge.net/news/}{OProfile}(在构建配置中将 \texttt{USE\_OPROFILE\_JITEVENTS} 设置为 \texttt{1})。

\end{itemize}
\end{quote}


\hypertarget{12744946110825549407}{}


\subsection{\texttt{JULIA\_LLVM\_ARGS}}



传递给 LLVM 后端的参数。



\hypertarget{12804520762715277004}{}


\chapter{嵌入 Julia}



正如我们在 \hyperlink{4974579121496702029}{调用 C 和 Fortran 代码} 中看到的, Julia 有着简单高效的方法来调用 C 编写的函数。但有时恰恰相反,我们需要在 C 中调用 Julia 的函数。这可以将 Julia 代码集成到一个更大的 C/C++ 项目而无需在 C/C++ 中重写所有内容。Julia 有一个 C API 来实现这一目标。几乎所有编程语言都能以某种方式来调用 C 语言的函数,因此 Julia 的 C API 也就能够进行更多语言的桥接。(例如在 Python 或是 C\# 中调用 Julia ).



\hypertarget{10185907435024062430}{}


\section{高级别嵌入}



\textbf{Note}: 本节包含可运行在类 Unix 系统上的、使用 C 编写的嵌入式 Julia 代码。Windows 平台请参阅下一节。



我们从一个简单的 C 程序开始初始化 Julia 并调用一些 Julia 代码:




\begin{lstlisting}
#include <julia.h>
JULIA_DEFINE_FAST_TLS() // only define this once, in an executable (not in a shared library) if you want fast code.

int main(int argc, char *argv[])
{
    /* required: setup the Julia context */
    jl_init();

    /* run Julia commands */
    jl_eval_string("print(sqrt(2.0))");

    /* strongly recommended: notify Julia that the
         program is about to terminate. this allows
         Julia time to cleanup pending write requests
         and run all finalizers
    */
    jl_atexit_hook(0);
    return 0;
}
\end{lstlisting}



为构建这个程序,你必须将 Julia 头文件的路径放入 include 路径并链接 \texttt{libjulia} 。例如 Julia 被安装到 \texttt{\$JULIA\_DIR},则可以用 \texttt{gcc} 来编译上面的测试程序 \texttt{test.c}:




\begin{lstlisting}
gcc -o test -fPIC -I$JULIA_DIR/include/julia -L$JULIA_DIR/lib -Wl,-rpath,$JULIA_DIR/lib test.c -ljulia
\end{lstlisting}



或者查看 Julia 源代码目录 \texttt{test/embedding/} 文件夹下的 \texttt{embedding.c} 文件。 文件 \texttt{ui/repl.c} 则是另一个简单示例,用于设置链接 \texttt{libjulia} 时 \texttt{jl\_options} 的选项 。



在调用任何其他 Julia C 函数之前第一件必须要做的事是初始化 Julia,通过调用 \texttt{jl\_init} 尝试自动确定 Julia 的安装位置来实现。如果需要自定义位置或指定要加载的系统映像,请改用 \texttt{jl\_init\_with\_image}。



测试程序中的第二个语句通过调用 \texttt{jl\_eval\_string} 来执行 Julia 语句。



在程序结束之前,强烈建议调用 \texttt{jl\_atexit\_hook}。上面的示例程序在 \texttt{main} 返回之前进行了调用。



\begin{quote}
\textbf{Note}

现在,动态链接 \texttt{libjulia} 的共享库需要传递选项 \texttt{RTLD\_GLOBAL} 。比如在 Python 中像这样调用:


\begin{lstlisting}
>>> julia=CDLL('./libjulia.dylib',RTLD_GLOBAL)
>>> julia.jl_init.argtypes = []
>>> julia.jl_init()
250593296
\end{lstlisting}

\end{quote}


\begin{quote}
\textbf{Note}

如果 Julia 程序需要访问 主可执行文件 中的符号,那么除了下面描述的由 \texttt{julia-config.jl} 生成的标记之外,可能还需要在 Linux 上的编译时添加 \texttt{-Wl,--export-dynamic} 链接器标志。编译共享库时则不必要。

\end{quote}


\hypertarget{122745226345289239}{}


\subsection{使用 julia-config 自动确定构建参数}



\texttt{julia-config.jl} 创建脚本是为了帮助确定使用嵌入的 Julia 程序所需的构建参数。此脚本使用由其调用的特定 Julia 分发的构建参数和系统配置来导出嵌入程序的必要编译器标志以与该分发交互。此脚本位于 Julia 的 share 目录中。



\hypertarget{18305016561534437319}{}


\subsubsection{例子}




\begin{lstlisting}
#include <julia.h>

int main(int argc, char *argv[])
{
    jl_init();
    (void)jl_eval_string("println(sqrt(2.0))");
    jl_atexit_hook(0);
    return 0;
}
\end{lstlisting}



\hypertarget{15399362667106734470}{}


\subsubsection{在命令行中}



命令行脚本简单用法:假设 \texttt{julia-config.jl} 位于 \texttt{/usr/local/julia/share/julia},它可以直接在命令行上调用,并采用 3 个标志的任意组合:




\begin{lstlisting}
/usr/local/julia/share/julia/julia-config.jl
Usage: julia-config [--cflags|--ldflags|--ldlibs]
\end{lstlisting}



如果上面的示例源代码保存为文件 \texttt{embed\_example.c},则以下命令将其编译为 Linux 和 Windows 上运行的程序(MSYS2 环境),或者如果在 OS/X 上,则用 \texttt{clang} 替换 \texttt{gcc}。:




\begin{lstlisting}
/usr/local/julia/share/julia/julia-config.jl --cflags --ldflags --ldlibs | xargs gcc embed_example.c
\end{lstlisting}



\hypertarget{9935040774481087943}{}


\subsubsection{在 Makefiles 中使用}



但通常来说,嵌入的项目会比上面更复杂,因此一般会提供 makefile 支持。由于使用了 \textbf{shell} 宏扩展,我们就假设用 GNU make 。 另外,尽管很多时候 \texttt{julia-config.jl} 会在目录 \texttt{/usr/local} 中出现多次,不过也未必如此,但 Julia 也定位 \texttt{julia-config.jl},并且可以使用 makefile 来利用它。上面的示例程序使用 Makefile 来扩展。:




\begin{lstlisting}
JL_SHARE = $(shell julia -e 'print(joinpath(Sys.BINDIR, Base.DATAROOTDIR, "julia"))')
CFLAGS += $(shell $(JL_SHARE)/julia-config.jl --cflags)
CXXFLAGS += $(shell $(JL_SHARE)/julia-config.jl --cflags)
LDFLAGS += $(shell $(JL_SHARE)/julia-config.jl --ldflags)
LDLIBS += $(shell $(JL_SHARE)/julia-config.jl --ldlibs)

all: embed_example
\end{lstlisting}



现在构建的命令就只需要简简单单的\texttt{make}了。



\hypertarget{9091699375369199363}{}


\section{在 Windows 使用 Visual Studio 进行高级别嵌入}



If the \texttt{JULIA\_DIR} environment variable hasn{\textquotesingle}t been setup, add it using the System panel before starting Visual Studio. The \texttt{bin} folder under JULIA\_DIR should be on the system PATH.



We start by opening Visual Studio and creating a new Console Application project. To the {\textquotesingle}stdafx.h{\textquotesingle} header file, add the following lines at the end:




\begin{lstlisting}
#include <julia.h>
\end{lstlisting}



Then, replace the main() function in the project with this code:




\begin{lstlisting}
int main(int argc, char *argv[])
{
    /* required: setup the Julia context */
    jl_init();

    /* run Julia commands */
    jl_eval_string("print(sqrt(2.0))");

    /* strongly recommended: notify Julia that the
         program is about to terminate. this allows
         Julia time to cleanup pending write requests
         and run all finalizers
    */
    jl_atexit_hook(0);
    return 0;
}
\end{lstlisting}



The next step is to set up the project to find the Julia include files and the libraries. It{\textquotesingle}s important to know whether the Julia installation is 32- or 64-bits. Remove any platform configuration that doesn{\textquotesingle}t correspond to the Julia installation before proceeding.



Using the project Properties dialog, go to \texttt{C/C++} | \texttt{General} and add \texttt{\$(JULIA\_DIR){\textbackslash}include{\textbackslash}julia{\textbackslash}} to the Additional Include Directories property. Then, go to the \texttt{Linker} | \texttt{General} section and add \texttt{\$(JULIA\_DIR){\textbackslash}lib} to the Additional Library Directories property. Finally, under \texttt{Linker} | \texttt{Input}, add \texttt{libjulia.dll.a;libopenlibm.dll.a;} to the list of libraries.



At this point, the project should build and run.



\hypertarget{4573294506139022977}{}


\section{转换类型}



真正的应用程序不仅仅要执行表达式,还要返回表达式的值给宿主程序。\texttt{jl\_eval\_string} 返回 一个 \texttt{jl\_value\_t*},它是指向堆分配的 Julia 对象的指针。存储像 \hyperlink{5027751419500983000}{\texttt{Float64}} 这些简单数据类型叫做 \texttt{装箱},然后提取存储的基础类型数据叫 \texttt{拆箱}。我们改进的示例程序在 Julia 中计算 2 的平方根,并在 C 中读取回结果,如下所示:




\begin{lstlisting}
jl_value_t *ret = jl_eval_string("sqrt(2.0)");

if (jl_typeis(ret, jl_float64_type)) {
    double ret_unboxed = jl_unbox_float64(ret);
    printf("sqrt(2.0) in C: %e \n", ret_unboxed);
}
else {
    printf("ERROR: unexpected return type from sqrt(::Float64)\n");
}
\end{lstlisting}



为了检查 \texttt{ret} 是否为特定的 Julia 类型,我们可以使用 \texttt{jl\_isa},\texttt{jl\_typeis} 或 \texttt{jl\_is\_...} 函数。通过输入 \texttt{typeof(sqrt(2.0))}到 Julia shell,我们可以看到返回类型是 \hyperlink{5027751419500983000}{\texttt{Float64}}(在C中是 \texttt{double} 类型)。要将装箱的 Julia 值转换为 C 的double,上面的代码片段使用了 \texttt{jl\_unbox\_float64}函数。



相应的, 用 \texttt{jl\_box\_...} 函数是另一种转换的方式。




\begin{lstlisting}
jl_value_t *a = jl_box_float64(3.0);
jl_value_t *b = jl_box_float32(3.0f);
jl_value_t *c = jl_box_int32(3);
\end{lstlisting}



正如我们将在下面看到的那样,装箱需要在调用 Julia 函数时使用特定参数。



\hypertarget{15001488547709567560}{}


\section{调用 Julia 函数}



虽然 \texttt{jl\_eval\_string} 允许 C 获取 Julia 表达式的结果,但它不允许将在 C 中计算的参数传递给 Julia。因此需要使用 \texttt{jl\_call} 来直接调用Julia函数:




\begin{lstlisting}
jl_function_t *func = jl_get_function(jl_base_module, "sqrt");
jl_value_t *argument = jl_box_float64(2.0);
jl_value_t *ret = jl_call1(func, argument);
\end{lstlisting}



在第一步中,通过调用 \texttt{jl\_get\_function} 检索出 Julia 函数 \texttt{sqrt} 的句柄(handle)。 传递给 \texttt{jl\_get\_function} 的第一个参数是 指向 定义\texttt{sqrt}所在的 \texttt{Base} 模块 的指针。 然后,double 值通过 \texttt{jl\_box\_float64} 被装箱。 最后,使用 \texttt{jl\_call1} 调用该函数。也有 \texttt{jl\_call0},\texttt{jl\_call2}和\texttt{jl\_call3} 函数,方便地处理不同数量的参数。 要传递更多参数,使用 \texttt{jl\_call}:




\begin{lstlisting}
jl_value_t *jl_call(jl_function_t *f, jl_value_t **args, int32_t nargs)
\end{lstlisting}



它的第二个参数 \texttt{args} 是 \texttt{jl\_value\_t*} 类型的数组,\texttt{nargs} 是参数的个数 



\hypertarget{3779983102705119396}{}


\section{内存管理}



正如我们所见,Julia 对象在 C 中表示为指针。这就出现了 谁来负责释放这些对象的问题。



通常,Julia 对象由垃圾收集器(GC)释放,但 GC 不会自动就懂我们正C中保留对Julia值的引用。这意味着 GC 会在你的掌控之外释放对象,从而使指针无效。



The GC can only run when Julia objects are allocated. Calls like \texttt{jl\_box\_float64} perform allocation, and allocation might also happen at any point in running Julia code. However, it is generally safe to use pointers in between \texttt{jl\_...} calls. But in order to make sure that values can survive \texttt{jl\_...} calls, we have to tell Julia that we still hold a reference to Julia \href{https://www.cs.purdue.edu/homes/hosking/690M/p611-fenichel.pdf}{root} values, a process called {\textquotedbl}GC rooting{\textquotedbl}. Rooting a value will ensure that the garbage collector does not accidentally identify this value as unused and free the memory backing that value. This can be done using the \texttt{JL\_GC\_PUSH} macros:




\begin{lstlisting}
jl_value_t *ret = jl_eval_string("sqrt(2.0)");
JL_GC_PUSH1(&ret);
// Do something with ret
JL_GC_POP();
\end{lstlisting}



The \texttt{JL\_GC\_POP} call releases the references established by the previous \texttt{JL\_GC\_PUSH}. Note that \texttt{JL\_GC\_PUSH} stores references on the C stack, so it must be exactly paired with a \texttt{JL\_GC\_POP} before the scope is exited. That is, before the function returns, or control flow otherwise leaves the block in which the \texttt{JL\_GC\_PUSH} was invoked.



Several Julia values can be pushed at once using the \texttt{JL\_GC\_PUSH2} , \texttt{JL\_GC\_PUSH3} , \texttt{JL\_GC\_PUSH4} , \texttt{JL\_GC\_PUSH5} , and \texttt{JL\_GC\_PUSH6} macros. To push an array of Julia values one can use the \texttt{JL\_GC\_PUSHARGS} macro, which can be used as follows:




\begin{lstlisting}
jl_value_t **args;
JL_GC_PUSHARGS(args, 2); // args can now hold 2 `jl_value_t*` objects
args[0] = some_value;
args[1] = some_other_value;
// Do something with args (e.g. call jl_... functions)
JL_GC_POP();
\end{lstlisting}



Each scope must have only one call to \texttt{JL\_GC\_PUSH*}. Hence, if all variables cannot be pushed once by a single call to \texttt{JL\_GC\_PUSH*}, or if there are more than 6 variables to be pushed and using an array of arguments is not an option, then one can use inner blocks:




\begin{lstlisting}
jl_value_t *ret1 = jl_eval_string("sqrt(2.0)");
JL_GC_PUSH1(&ret1);
jl_value_t *ret2 = 0;
{
    jl_function_t *func = jl_get_function(jl_base_module, "exp");
    ret2 = jl_call1(func, ret1);
    JL_GC_PUSH1(&ret2);
    // Do something with ret2.
    JL_GC_POP();    // This pops ret2.
}
JL_GC_POP();    // This pops ret1.
\end{lstlisting}



If it is required to hold the pointer to a variable between functions (or block scopes), then it is not possible to use \texttt{JL\_GC\_PUSH*}. In this case, it is necessary to create and keep a reference to the variable in the Julia global scope. One simple way to accomplish this is to use a global \texttt{IdDict} that will hold the references, avoiding deallocation by the GC. However, this method will only work properly with mutable types.




\begin{lstlisting}
// This functions shall be executed only once, during the initialization.
jl_value_t* refs = jl_eval_string("refs = IdDict()");
jl_function_t* setindex = jl_get_function(jl_base_module, "setindex!");

...

// `var` is the variable we want to protect between function calls.
jl_value_t* var = 0;

...

// `var` is a `Vector{Float64}`, which is mutable.
var = jl_eval_string("[sqrt(2.0); sqrt(4.0); sqrt(6.0)]");

// To protect `var`, add its reference to `refs`.
jl_call3(setindex, refs, var, var);
\end{lstlisting}



If the variable is immutable, then it needs to be wrapped in an equivalent mutable container or, preferably, in a \texttt{RefValue\{Any\}} before it is pushed to \texttt{IdDict}. In this approach, the container has to be created or filled in via C code using, for example, the function \texttt{jl\_new\_struct}. If the container is created by \texttt{jl\_call*}, then you will need to reload the pointer to be used in C code.




\begin{lstlisting}
// This functions shall be executed only once, during the initialization.
jl_value_t* refs = jl_eval_string("refs = IdDict()");
jl_function_t* setindex = jl_get_function(jl_base_module, "setindex!");
jl_datatype_t* reft = (jl_datatype_t*)jl_eval_string("Base.RefValue{Any}");

...

// `var` is the variable we want to protect between function calls.
jl_value_t* var = 0;

...

// `var` is a `Float64`, which is immutable.
var = jl_eval_string("sqrt(2.0)");

// Protect `var` until we add its reference to `refs`.
JL_GC_PUSH1(&var);

// Wrap `var` in `RefValue{Any}` and push to `refs` to protect it.
jl_value_t* rvar = jl_new_struct(reft, var);
JL_GC_POP();

jl_call3(setindex, refs, rvar, rvar);
\end{lstlisting}



The GC can be allowed to deallocate a variable by removing the reference to it from \texttt{refs} using the function \texttt{delete!}, provided that no other reference to the variable is kept anywhere:




\begin{lstlisting}
jl_function_t* delete = jl_get_function(jl_base_module, "delete!");
jl_call2(delete, refs, rvar);
\end{lstlisting}



As an alternative for very simple cases, it is possible to just create a global container of type \texttt{Vector\{Any\}} and fetch the elements from that when necessary, or even to create one global variable per pointer using




\begin{lstlisting}
jl_set_global(jl_main_module, jl_symbol("var"), var);
\end{lstlisting}



\hypertarget{16854094057405745675}{}


\subsection{Updating fields of GC-managed objects}



The garbage collector operates under the assumption that it is aware of every old-generation object pointing to a young-generation one. Any time a pointer is updated breaking that assumption, it must be signaled to the collector with the \texttt{jl\_gc\_wb} (write barrier) function like so:




\begin{lstlisting}
jl_value_t *parent = some_old_value, *child = some_young_value;
((some_specific_type*)parent)->field = child;
jl_gc_wb(parent, child);
\end{lstlisting}



通常情况下不可能在运行时预测 值是否是旧的,因此 写屏障 必须被插入在所有显式存储之后。一个需要注意的例外是如果 \texttt{parent} 对象刚分配,垃圾收集之后并不执行。请记住大多数 \texttt{jl\_...} 函数有时候都会执行垃圾收集。



直接更新数据时,对于指针数组来说 写屏障 也是必需的 例如:




\begin{lstlisting}
jl_array_t *some_array = ...; // e.g. a Vector{Any}
void **data = (void**)jl_array_data(some_array);
jl_value_t *some_value = ...;
data[0] = some_value;
jl_gc_wb(some_array, some_value);
\end{lstlisting}



\hypertarget{13611710259554554355}{}


\subsection{控制垃圾收集器}



有一些函数能够控制GC。在正常使用情况下这些不是必要的。




\begin{table}[h]

\begin{tabulary}{\linewidth}{|L|L|}
\hline
函数 & 描述 \\
\hline
\texttt{jl\_gc\_collect()} & 强制执行 GC \\
\hline
\texttt{jl\_gc\_enable(0)} & 禁用 GC, 返回前一个状态作为 int 类型 \\
\hline
\texttt{jl\_gc\_enable(1)} & 启用 GC, 返回前一个状态作为 int 类型 \\
\hline
\texttt{jl\_gc\_is\_enabled()} & 返回当前状态作为 int 类型 \\
\hline
\end{tabulary}

\end{table}



\hypertarget{12793375650632651595}{}


\section{使用数组}



Julia 和 C 可以不通过复制而共享数组数据。下面一个例子将展示它是如何工作的。



Julia数组用数据类型 \texttt{jl\_array\_t *} 表示。基本上,\texttt{jl\_array\_t} 是一个包含以下内容的结构:



\begin{itemize}
\item 关于数据类型的信息


\item 指向数据块的指针


\item 关于数组长度的信息

\end{itemize}


为了让事情比较简单,我们从一维数组开始,创建一个存有 10 个 FLoat64 类型的数组如下所示:




\begin{lstlisting}
jl_value_t* array_type = jl_apply_array_type((jl_value_t*)jl_float64_type, 1);
jl_array_t* x          = jl_alloc_array_1d(array_type, 10);
\end{lstlisting}



或者,如果您已经分配了数组,则可以生成一个简易的包装器来包裹其数据:




\begin{lstlisting}
double *existingArray = (double*)malloc(sizeof(double)*10);
jl_array_t *x = jl_ptr_to_array_1d(array_type, existingArray, 10, 0);
\end{lstlisting}



最后一个参数是一个布尔值,表示 Julia 是否应该获取数据的所有权。 如果这个参数 不为零,当数组不再被引用时,GC 会在数据的指针上调用 \texttt{free} 。



为了访问 x 的数据,我们可以使用 \texttt{jl\_array\_data}:




\begin{lstlisting}
double *xData = (double*)jl_array_data(x);
\end{lstlisting}



现在我们可以填充这个数组:




\begin{lstlisting}
for(size_t i=0; i<jl_array_len(x); i++)
    xData[i] = i;
\end{lstlisting}



现在让我们调用一个对 \texttt{x} 就地操作的 Julia 函数:




\begin{lstlisting}
jl_function_t *func = jl_get_function(jl_base_module, "reverse!");
jl_call1(func, (jl_value_t*)x);
\end{lstlisting}



通过打印数组,可以验证 \texttt{x} 的元素现在是否已被逆置 (reversed)。



\hypertarget{10421624901695308393}{}


\subsection{获取返回的数组}



如果 Julia 函数返回一个数组,\texttt{jl\_eval\_string} 和 \texttt{jl\_call} 的返回值可以被强制转换为\texttt{jl\_array\_t *}:




\begin{lstlisting}
jl_function_t *func  = jl_get_function(jl_base_module, "reverse");
jl_array_t *y = (jl_array_t*)jl_call1(func, (jl_value_t*)x);
\end{lstlisting}



现在使用 \texttt{jl\_array\_data} 可以像前面一样访问 \texttt{y} 的内容。一如既往地,一定要在使用数组的时候确保 持有使用数组的引用。



\hypertarget{14703069974979105074}{}


\subsection{多维数组}



Julia的多维数组以 列序优先 存储在内存中。这是一些 创建一个2D数组并访问其属性 的代码:




\begin{lstlisting}
// Create 2D array of float64 type
jl_value_t *array_type = jl_apply_array_type(jl_float64_type, 2);
jl_array_t *x  = jl_alloc_array_2d(array_type, 10, 5);

// Get array pointer
double *p = (double*)jl_array_data(x);
// Get number of dimensions
int ndims = jl_array_ndims(x);
// Get the size of the i-th dim
size_t size0 = jl_array_dim(x,0);
size_t size1 = jl_array_dim(x,1);

// Fill array with data
for(size_t i=0; i<size1; i++)
    for(size_t j=0; j<size0; j++)
        p[j + size0*i] = i + j;
\end{lstlisting}



请注意,虽然 Julia 的数组使用基于 1 的索引,但C API 中使用基于 0 的索引(例如 在调用\texttt{jl\_array\_dim})以便用C代码的习惯来阅读。



\hypertarget{4029112619480312893}{}


\section{异常}



Julia 代码可以抛出异常。比如:




\begin{lstlisting}
jl_eval_string("this_function_does_not_exist()");
\end{lstlisting}



这个调用似乎什么都没做。但可以检查异常是否抛出:




\begin{lstlisting}
if (jl_exception_occurred())
    printf("%s \n", jl_typeof_str(jl_exception_occurred()));
\end{lstlisting}



如果您使用 支持异常的语言的 Julia C API(例如Python,C#,C ++),使用 检查是否有异常的函数 将每个调用 包装到 \texttt{libjulia} 中是有意义的,然后异常在宿主语言中重新抛出。



\hypertarget{3345506826925468977}{}


\subsection{抛出 Julia 异常}



在编写 Julia 可调用函数时,可能需要验证参数 并抛出异常表示错误。 典型的类型检查像这样:




\begin{lstlisting}
if (!jl_typeis(val, jl_float64_type)) {
    jl_type_error(function_name, (jl_value_t*)jl_float64_type, val);
}
\end{lstlisting}



可以使用以下函数 引发一般异常:




\begin{lstlisting}
void jl_error(const char *str);
void jl_errorf(const char *fmt, ...);
\end{lstlisting}



\texttt{jl\_error}采用 C 字符串,而 \texttt{jl\_errorf} 像 \texttt{printf} 一样调用:




\begin{lstlisting}
jl_errorf("argument x = %d is too large", x);
\end{lstlisting}



在这个例子中假定 \texttt{x} 是一个 int 值。



\hypertarget{15996691828014685993}{}


\chapter{代码加载}



\begin{quote}
\textbf{Note}

这一章包含了加载包的技术细节。如果要安装包,使用 Julia 的内置包管理器\hyperlink{7626139948888930049}{\texttt{Pkg}}将包加入到你的活跃环境中。如果要使用已经在你的活跃环境中的包,使用 \texttt{import X} 或 \texttt{using X},正如在\hyperlink{7031478671373133429}{模块}中所描述的那样。

\end{quote}


\hypertarget{5148923403283115982}{}


\section{定义}



Julia加载代码有两种机制:



\begin{itemize}
\item[1. ] \textbf{代码包含:}例如 \texttt{include({\textquotedbl}source.jl{\textquotedbl})}。包含允许你把一个程序拆分为多个源文件。表达式 \texttt{include({\textquotedbl}source.jl{\textquotedbl})} 使得文件 \texttt{source.jl} 的内容在出现 \texttt{include} 调用的模块的全局作用域中执行。如果多次调用 \texttt{include({\textquotedbl}source.jl{\textquotedbl})},\texttt{source.jl} 就被执行多次。\texttt{source.jl} 的包含路径解释为相对于出现 \texttt{include} 调用的文件路径。重定位源文件子树因此变得简单。在 REPL 中,包含路径为当前工作目录,即 \hyperlink{16313884780490629439}{\texttt{pwd()}}。


\item[2. ] \textbf{加载包:}例如 \texttt{import X} 或 \texttt{using X}。\texttt{import} 通过加载包(一个独立的,可重用的 Julia 代码集合,包含在一个模块中),并导入模块内部的名称 \texttt{X},使得模块 \texttt{X} 可用。 如果在同一个 Julia 会话中,多次导入包 \texttt{X},那么后续导入模块为第一次导入模块的引用。但请注意,\texttt{import X} 可以在不同的上下文中加载不同的包:\texttt{X} 可以引用主工程中名为 \texttt{X} 的一个包,但它在各个依赖中可以引用不同的、名称同为 \texttt{X} 的包。更多机制说明如下。

\end{itemize}


代码包含是非常直接和简单的:其在调用者的上下文中解释运行给定的源文件。包加载是建立在代码包含之上的,它具有不同的\hyperlink{16725527896995457152}{用途}。本章的其余部分将重点介绍程序包加载的行为和机制。



一个 \emph{包(package)} 就是一个源码树,其标准布局中提供了其他 Julia 项目可以复用的功能。包可以使用 \texttt{import X} 或 \texttt{using X} 语句加载,名为 \texttt{X} 的模块在加载包代码时生成,并在包含该 import 语句的模块中可用。\texttt{import X} 中 \texttt{X} 的含义与上下文有关:程序加载哪个 \texttt{X} 包取决于 import 语句出现的位置。因此,处理 \texttt{import X} 分为两步:首先,确定在此上下文中是\textbf{哪个}包被定义为 \texttt{X};其次,确定到\textbf{哪里}找特定的 \texttt{X} 包。



这些问题可通过查询各项目文件(\texttt{Project.toml} 或 \texttt{JuliaProject.toml})、清单文件(\texttt{Manifest.toml} 或 \texttt{JuliaManifest.toml}),或是源文件的文件夹列在\hyperlink{17914149694871263675}{\texttt{LOAD\_PATH}} 中的项目环境解决。



\hypertarget{6907225830267322072}{}


\section{包的联合}



大多数时候,一个包可以通过它的名字唯一确定。但有时在一个项目中,可能需要使用两个有着相同名字的不同的包。尽管你可以通过重命名其中一个包来解决这个问题,但在一个大型的、共享的代码库中被迫做这件事可能是有高度破坏性的。相反,Julia的包加载机制允许相同的包名在一个应用的不同部分指向不同的包。



Julia 支持联合的包管理,这意味着多个独立的部分可以维护公有包、私有包以及包的注册表,并且项目可以依赖于一系列来自不同注册表的公有包和私有包。您也可以使用一组通用工具和工作流(workflow)来安装和管理来自各种注册表的包。Julia 附带的 \texttt{Pkg} 软件包管理器允许安装和管理项目的依赖项,它会帮助创建并操作项目文件(其描述了项目所依赖的其他项目)和清单文件(其为项目完整依赖库的确切版本的快照)。



联合管理的一个可能后果是没有包命名的中央权限。不同组织可以使用相同的名称来引用不相关的包。这并不是没有可能的,因为这些组织可能没有协作,甚至不知道彼此。由于缺乏中央命名权限,单个项目可能最终依赖着具有相同名称的不同包。Julia 的包加载机制不要求包名称是全局唯一的,即使在单个项目的依赖关系图中也是如此。相反,包由\href{https://en.wikipedia.org/wiki/Universally\_unique\_identifier}{通用唯一标识符} (UUID)进行标识,它在每个包创建时进行分配。通常,您不必直接使用这些有点麻烦的 128 位标识符,因为 \texttt{Pkg} 将负责生成和跟踪它们。但是,这些 UUID 为问题\emph{「\texttt{X} 所指的包是什么?」}提供了确定的答案



由于去中心化的命名问题有些抽象,因此可以通过具体情境来理解问题。假设你正在开发一个名为 \texttt{App} 的应用程序,它使用两个包:\texttt{Pub} 和 \texttt{Priv}。\texttt{Priv} 是你创建的私有包,而 \texttt{Pub} 是你使用但不控制的公共包。当你创建 \texttt{Priv} 时,没有名为 \texttt{Priv} 的公共包。然而,随后一个名为 \texttt{Priv} 的不相关软件包发布并变得流行起来,而且 \texttt{Pub} 包已经开始使用它了。因此,当你下次升级 \texttt{Pub} 以获取最新的错误修复和特性时,\texttt{App} 将依赖于两个名为 \texttt{Priv} 的不同包——尽管你除了升级之外什么都没做。\texttt{App} 直接依赖于你的私有 \texttt{Priv} 包,以及通过 \texttt{Pub} 在新的公共 \texttt{Priv} 包上的间接依赖。由于这两个 \texttt{Priv} 包是不同的,但是 \texttt{App} 继续正常工作依赖于他们两者,因此表达式 \texttt{import Priv} 必须引用不同的 \texttt{Priv} 包,具体取决于它是出现在 \texttt{App} 的代码中还是出现在 \texttt{Pub} 的代码中。为了处理这种情况,Julia 的包加载机制通过 UUID 区分两个 \texttt{Priv} 包并根据它(调用 \texttt{import} 的模块)的上下文选择正确的包。这种区分的工作原理取决于环境,如以下各节所述。



\hypertarget{3001628692617530434}{}


\section{环境(Environments)}



\textbf{环境}决定了 \texttt{import X} 和 \texttt{using X} 语句在不同的代码上下文中的含义以及什么文件会被加载。Julia 有两类环境(environment):



\begin{itemize}
\item[1. ] \textbf{A project environment} is a directory with a project file and an optional manifest file, and forms an \emph{explicit environment}. The project file determines what the names and identities of the direct dependencies of a project are. The manifest file, if present, gives a complete dependency graph, including all direct and indirect dependencies, exact versions of each dependency, and sufficient information to locate and load the correct version.


\item[2. ] \textbf{包目录(package directory)}是包含一组包的源码树子目录的目录,并形成一个\emph{隐式环境}。如果 \texttt{X} 是包目录的子目录并且存在 \texttt{X/src/X.jl},那么程序包 \texttt{X} 在包目录环境中可用,而 \texttt{X/src/X.jl} 是加载它使用的源文件。

\end{itemize}


这些环境可以混合并用来创建\textbf{堆栈环境(stacked environment)}:是一组有序的项目环境和包目录,重叠为一个复合环境。然后,结合优先级规则和可见性规则,确定哪些包是可用的以及从哪里加载它们。例如,Julia 的负载路径是一个堆栈环境。



这些环境各有不同的用途:



\begin{itemize}
\item 项目环境提供\textbf{可迁移性}。通过将项目环境以及项目源代码的其余部分存放到版本控制(例如一个 git 存储库),您可以重现项目的确切状态和所有依赖项。特别是,清单文件会记录每个依赖项的确切版本,而依赖项由其源码树的加密哈希值标识;这使得 \texttt{Pkg} 可以检索出正确的版本,并确保你正在运行准确的已记录的所有依赖项的代码。


\item 当不需要完全仔细跟踪的项目环境时,包目录更\textbf{方便}。当你想要把一组包放在某处,并且希望能够直接使用它们而不必为之创建项目环境时,包目录是很实用的。


\item 堆栈环境允许向基本环境\textbf{添加}工具。您可以将包含开发工具在内的环境堆到堆栈环境的末尾,使它们在 REPL 和脚本中可用,但在包内部不可用。

\end{itemize}


从更高层次上,每个环境在概念上定义了三个映射:roots、graph 和 paths。当解析 \texttt{import X} 的含义时,roots 和 graph 映射用于确定 \texttt{X} 的身份,同时 paths 映射用于定位 \texttt{X} 的源代码。这三个映射的具体作用是:



\begin{itemize}
\item \textbf{roots:} \texttt{name::Symbol} ⟶ \texttt{uuid::UUID}

环境的 roots 映射将包名称分配给UUID,以获取环境可用于主项目的所有顶级依赖项(即可以在 \texttt{Main} 中加载的那些依赖项)。当 Julia 在主项目中遇到 \texttt{import X} 时,它会将 \texttt{X} 的标识作为 \texttt{roots[:X]}。


\item \textbf{graph:} \texttt{context::UUID} ⟶ \texttt{name::Symbol} ⟶ \texttt{uuid::UUID}

环境的 graph 是一个多级映射,它为每个 \texttt{context} UUID 分配一个从名称到 UUID 的映射——类似于 roots 映射,但专一于那个 \texttt{context}。当 Julia 在 UUID 为 \texttt{context} 的包代码中运行到 \texttt{import X} 时,它会将 \texttt{X} 的标识看作为 \texttt{graph[context][:X]}。正是因为如此,\texttt{import X} 可以根据 \texttt{context} 引用不同的包。


\item \textbf{paths:} \texttt{uuid::UUID} × \texttt{name::Symbol} ⟶ \texttt{path::String}

paths 映射会为每个包分配 UUID-name 对,即该包的入口点源文件的位置。在 \texttt{import X} 中,\texttt{X} 的标识已经通过 roots 或 graph 解析为 UUID(取决于它是从主项目还是从依赖项加载),Julia 确定要加载哪个文件来获取 \texttt{X} 是通过在环境中查找 \texttt{paths[uuid,:X]}。要包含此文件应该定义一个名为 \texttt{X} 的模块。一旦加载了此包,任何解析为相同的 \texttt{uuid} 的后续导入只会创建一个到同一个已加载的包模块的绑定。

\end{itemize}


每种环境都以不同的方式定义这三种映射,详见以下各节。



\begin{quote}
\textbf{Note}

为了清楚地说明,本章中的示例包括 roots、graph 和 paths 的完整数据结构。但是,为了提高效率,Julia 的包加载代码并没有显式地创建它们。相反,加载一个给定包只会简单地计算所需的结构。

\end{quote}


\hypertarget{2089876833496138047}{}


\subsection{项目环境(Project environments)}



项目环境由包含名为 \texttt{Project.toml} 的项目文件的目录以及名为 \texttt{Manifest.toml} 的清单文件(可选)确定。这些文件也可以命名为 \texttt{JuliaProject.toml} 和 \texttt{JuliaManifest.toml},此时 \texttt{Project.toml} 和 \texttt{Manifest.toml} 被忽略——这允许项目与可能需要名为 \texttt{Project.toml} 和 \texttt{Manifest.toml} 文件的其他重要工具共存。但是对于纯 Julia 项目,名称 \texttt{Project.toml} 和 \texttt{Manifest.toml} 是首选。



项目环境的 roots、graph 和 paths 映射定义如下:



\textbf{roots 映射} 在环境中由其项目文件的内容决定,特别是它的顶级 \texttt{name} 和 \texttt{uuid} 条目及其 \texttt{[deps]} 部分(全部是可选的)。考虑以下一个假想的应用程序 \texttt{App} 的示例项目文件,如先前所述:




\begin{lstlisting}
name = "App"
uuid = "8f986787-14fe-4607-ba5d-fbff2944afa9"

[deps]
Priv = "ba13f791-ae1d-465a-978b-69c3ad90f72b"
Pub  = "c07ecb7d-0dc9-4db7-8803-fadaaeaf08e1"
\end{lstlisting}



如果将它表示为 Julia 字典,那么这个项目文件意味着以下 roots 映射:




\begin{minted}{julia}
roots = Dict(
    :App  => UUID("8f986787-14fe-4607-ba5d-fbff2944afa9"),
    :Priv => UUID("ba13f791-ae1d-465a-978b-69c3ad90f72b"),
    :Pub  => UUID("c07ecb7d-0dc9-4db7-8803-fadaaeaf08e1"),
)
\end{minted}



基于这个 root 映射,在 \texttt{App} 的代码中,语句 \texttt{import Priv} 将使 Julia 查找 \texttt{roots[:Priv]},这将得到 \texttt{ba13f791-ae1d-465a-978b-69c3ad90f72b},也就是要在这一部分加载的 \texttt{Priv} 包的 UUID。当主应用程序解释运行到 \texttt{import Priv} 时,此 UUID 标识了要加载和使用的 \texttt{Priv} 包。



\textbf{依赖图(dependency graph)} 在项目环境中其清单文件的内容决定,如果其存在。如果没有清单文件,则 graph 为空。清单文件包含项目的直接或间接依赖项的节(stanza)。对于每个依赖项,该文件列出该包的 UUID 以及源码树的哈希值或源代码的显式路径。考虑以下 \texttt{App} 的示例清单文件:




\begin{lstlisting}
[[Priv]] # 私有的那个
deps = ["Pub", "Zebra"]
uuid = "ba13f791-ae1d-465a-978b-69c3ad90f72b"
path = "deps/Priv"

[[Priv]] # 公共的那个
uuid = "2d15fe94-a1f7-436c-a4d8-07a9a496e01c"
git-tree-sha1 = "1bf63d3be994fe83456a03b874b409cfd59a6373"
version = "0.1.5"

[[Pub]]
uuid = "c07ecb7d-0dc9-4db7-8803-fadaaeaf08e1"
git-tree-sha1 = "9ebd50e2b0dd1e110e842df3b433cb5869b0dd38"
version = "2.1.4"

  [Pub.deps]
  Priv = "2d15fe94-a1f7-436c-a4d8-07a9a496e01c"
  Zebra = "f7a24cb4-21fc-4002-ac70-f0e3a0dd3f62"

[[Zebra]]
uuid = "f7a24cb4-21fc-4002-ac70-f0e3a0dd3f62"
git-tree-sha1 = "e808e36a5d7173974b90a15a353b564f3494092f"
version = "3.4.2"
\end{lstlisting}



这个清单文件描述了 \texttt{App} 项目可能的完整依赖关系图:



\begin{itemize}
\item 应用程序使用两个名为 \texttt{Priv} 的不同包,一个作为根依赖项的私有包,以及一个通过 \texttt{Pub} 作为间接依赖项的公共包。它们通过不同 UUID 来区分,并且有不同的依赖项:

\begin{itemize}
\item 私有的 \texttt{Priv} 依赖于 \texttt{Pub} 和 \texttt{Zebra} 包。


\item 公有的 \texttt{Priv} 没有依赖关系。

\end{itemize}

\item 该应用程序还依赖于 \texttt{Pub} 包,而后者依赖于公有的 \texttt{Priv} 以及私有的 \texttt{Priv} 包所依赖的那个 \texttt{Zebra} 包。

\end{itemize}


此依赖图以字典表示后如下所示:




\begin{minted}{julia}
graph = Dict(
    # Priv——私有的那个:
    UUID("ba13f791-ae1d-465a-978b-69c3ad90f72b") => Dict(
        :Pub   => UUID("c07ecb7d-0dc9-4db7-8803-fadaaeaf08e1"),
        :Zebra => UUID("f7a24cb4-21fc-4002-ac70-f0e3a0dd3f62"),
    ),
    # Priv——公共的那个:
    UUID("2d15fe94-a1f7-436c-a4d8-07a9a496e01c") => Dict(),
    # Pub:
    UUID("c07ecb7d-0dc9-4db7-8803-fadaaeaf08e1") => Dict(
        :Priv  => UUID("2d15fe94-a1f7-436c-a4d8-07a9a496e01c"),
        :Zebra => UUID("f7a24cb4-21fc-4002-ac70-f0e3a0dd3f62"),
    ),
    # Zebra:
    UUID("f7a24cb4-21fc-4002-ac70-f0e3a0dd3f62") => Dict(),
)
\end{minted}



给定这个依赖图,当 Julia 看到 \texttt{Pub} 包中的 \texttt{import Priv} ——它有 UUID\texttt{c07ecb7d-0dc9-4db7-8803-fadaaeaf08e1} 时,它会查找:




\begin{minted}{julia}
graph[UUID("c07ecb7d-0dc9-4db7-8803-fadaaeaf08e1")][:Priv]
\end{minted}



会得到 \texttt{2d15fe94-a1f7-436c-a4d8-07a9a496e01c},这意味着 \texttt{Pub} 包中的内容,\texttt{import Priv} 指代的是公有的 \texttt{Priv} 内容,而非应用程序直接依赖的私有包。这也是为何 \texttt{Priv} 在主项目中可指代不同的包,而不像其在某个依赖包中另有含义。在包生态中,该特性允许重名的出现。



如果在 \texttt{App} 主代码库中 \texttt{import Zebra} 会如何?因为\texttt{Zebra} 不存在于项目文件,即使它 \emph{确实} 存在于清单文件中,其导入会是失败的。此外,\texttt{import Zebra} 这个行为若发生在公有的 \texttt{Priv} 包——UUID 为 \texttt{2d15fe94-a1f7-436c-a4d8-07a9a496e01c} 的包中,同样会失败。因为公有的 \texttt{Priv} 包未在清单文件中声明依赖,故而无法加载包。仅有在清单文件:\texttt{Pub} 包和一个 \texttt{Priv} 包中作为显式依赖的包可用于加载 \texttt{Zebra}。



项目环境的 \textbf{路径映射} 从 manifest 文件中提取得到。而包的路径 \texttt{uuid} 和名称 \texttt{X} 则 (循序) 依据这些规则确定。



\begin{itemize}
\item[1. ] 如果目录中的项目文件与要求的 \texttt{uuid} 以及名称 \texttt{X} 匹配,那么可能出现以下情况的一种:

\end{itemize}


\begin{itemize}
\item 若该文件具有顶层 \texttt{路径} 入口,则 \texttt{uuid} 会被映射到该路径,文件的执行与包含项目文件的目录相关。


\item 此外,\texttt{uuid} 依照包含项目文件的目录,映射至与\texttt{src/X.jl}。

\end{itemize}


\begin{itemize}
\item[2. ] 若非上述情况,且项目文件具有对应的清单文件,且该清单文件包含匹配 \texttt{uuid} 的节(stanza),那么:

\end{itemize}


\begin{itemize}
\item 若其具有一个 \texttt{路径} 入口,则使用该路径(与包含清单文件的目录相关)。


\item 若其具有一个 \texttt{git-tree-sha1} 入口,计算一个确定的 \texttt{uuid} 与 \texttt{git-tree-sha1} 函数——我们把这个函数称为 \texttt{slug}——并在每个 Julia \texttt{DEPOT\_PATH} 的全局序列中的目录查询名为 \texttt{packages/X/\$slug} 的目录。使用存在的第一个此类目录。

\end{itemize}


若某些结果成功,源码入口点的路径会是这些结果中的某个,结果的相对路径+\texttt{src/X.jl};否则,\texttt{uuid} 不存在路径映射。当加载 \texttt{X} 时,如果没找到源码路径,查找即告失败,用户可能会被提示安装适当的包版本或采取其他纠正措施(例如,将 \texttt{X} 声明为某种依赖性)。



在上述样例清单文件中,为找到首个 \texttt{Priv} 包的路径——该包 UUID 为 \texttt{ba13f791-ae1d-465a-978b-69c3ad90f72b}——Julia 寻找其在清单中的节(stanza)。发现其有 路径\texttt{入口,查看}App\texttt{项目目录中相关的}deps/Priv\texttt{——不妨设}App\texttt{代码在}/home/me/projects/App\texttt{中—则 Julia 发现}/home/me/projects/App/deps/Priv\texttt{存在,并因此从中加载}Priv`。



另一方面,如果Julia加载的是带有\emph{other} \texttt{Priv} 包——即UUID为\texttt{2d15fe94-a1f7-436c-a4d8-07a9a496e01c}——它在清单中找到了它的节,请注意它\emph{没有}\texttt{path}条目,但是它有一个\texttt{git-tree-sha1} 条目。然后计算这个\texttt{slug} 的UUID/SHA-1对,具体是\texttt{HDkrT}(这个计算的确切细节并不重要,但它是始终一致的和确定的)。这意味着这个\texttt{Priv}包的路径\texttt{packages/Priv/HDkrT/src/Priv.jl}将在其中一个包仓库中。假设\texttt{DEPOT\_PATH} 的内容是\texttt{[{\textquotedbl}/home/me/.julia{\textquotedbl}, {\textquotedbl}/usr/local/julia{\textquotedbl}]},Julia将根据下面的路径来查看它们是否存在:



\begin{itemize}
\item[1. ] \texttt{/home/me/.julia/packages/Priv/HDkrT}


\item[2. ] \texttt{/usr/local/julia/packages/Priv/HDkrT}

\end{itemize}


Julia使用以上路径信息在仓库里依次查找 \texttt{packages/Priv/HDKrT/src/Priv.jl}文件,并从第一个查找到的文件中加载公共的 \texttt{Priv}包。



这是我们的示例App项目环境的可能路径映射的表示,  如上面Manifest 中所提供的依赖关系图, 在 搜索本地文件系统后:




\begin{minted}{julia}
paths = Dict(
    # Priv – the private one:
    (UUID("ba13f791-ae1d-465a-978b-69c3ad90f72b"), :Priv) =>
        # relative entry-point inside `App` repo:
        "/home/me/projects/App/deps/Priv/src/Priv.jl",
    # Priv – the public one:
    (UUID("2d15fe94-a1f7-436c-a4d8-07a9a496e01c"), :Priv) =>
        # package installed in the system depot:
        "/usr/local/julia/packages/Priv/HDkr/src/Priv.jl",
    # Pub:
    (UUID("c07ecb7d-0dc9-4db7-8803-fadaaeaf08e1"), :Pub) =>
        # package installed in the user depot:
        "/home/me/.julia/packages/Pub/oKpw/src/Pub.jl",
    # Zebra:
    (UUID("f7a24cb4-21fc-4002-ac70-f0e3a0dd3f62"), :Zebra) =>
        # package installed in the system depot:
        "/usr/local/julia/packages/Zebra/me9k/src/Zebra.jl",
)
\end{minted}



这个例子包含三种不同类型的包位置信息(第一个和第三个是默认加载路径的一部分)



\begin{itemize}
\item[1. ] 私有 \texttt{Priv} 包 {\textquotedbl}\href{https://stackoverflow.com/a/35109534}{vendored}{\textquotedbl}包括在\texttt{App}仓库中。


\item[2. ] 公共 \texttt{Priv} 与 \texttt{Zebra} 包位于系统仓库,系统管理员在此对相关包进行实时安装与管理。这些包允许系统上的所有用户使用。


\item[3. ] \texttt{Pub} 包位于用户仓库,用户实时安装的包都储存在此。 这些包仅限原安装用户使用。

\end{itemize}


\hypertarget{9048837682653155362}{}


\subsection{包目录}



包目录提供了一种更简单的环境,但不能处理名称冲突。在包目录中, 顶层包集合是“类似”包的子目录集合。“X”包存在于包目录中的条件,是目录包含下列“入口点”文件之一:



\begin{itemize}
\item \texttt{X.jl}


\item \texttt{X/src/X.jl}


\item \texttt{X.jl/src/X.jl}

\end{itemize}


包目录中的包可以导入哪些依赖项,取决于该包是否含有项目文件:



\begin{itemize}
\item 如果它有一个项目文件,那么它只能导入那些在项目文件的\texttt{[deps]} 部分中已标识的包。


\item 如果没有项目文件,它可以导入任何顶层包,即与在\texttt{Main} 或者 REPL中可加载的包相同。

\end{itemize}


\textbf{根图}是根据包目录的所有内容而形成的一个列表,包含所有已存在的包。 此外,一个UUID 将被赋予给每一个条目,例如对一个在文件夹\texttt{X}中找到的包



\begin{itemize}
\item[1. ] 如果\texttt{X/Project.toml}文件存在并且有一个\texttt{uuid} 条目,那么这个 \texttt{uuid}就是上述所要赋予的值。


\item[2. ] 如果\texttt{X/Project.toml}文件存在,但\emph{没有}包含一个顶层UUID条目, 该\texttt{uuid}将是一个虚构的UUID,是对\texttt{X/Project.toml}文件所在的规范(真实的)路径信息进行哈希处理而生成。


\item[3. ] 否则(如果\texttt{Project.toml}文件不存在), \texttt{uuid}将是一个全零值 \href{https://en.wikipedia.org/wiki/Universally\_unique\_identifier\#Nil\_UUID}{nil UUID}。

\end{itemize}


项目目录的\textbf{依赖关系图}是根据每个包的子目录中其项目文件的存在与否以及内容而形成。规则是:



\begin{itemize}
\item 如果包子目录没有项目文件,则在该图中忽略它, 其代码中的import语句按顶层处理,与main项目和REPL相同。


\item 如果包子目录有一个项目文件,那么图条目的UUID是项目文件的\texttt{[deps]}映射, 如果该信息项不存在,则视为空。

\end{itemize}


作为一个例子,假设包目录具有以下结构和内容:




\begin{lstlisting}
Aardvark/
    src/Aardvark.jl:
        import Bobcat
        import Cobra

Bobcat/
    Project.toml:
        [deps]
        Cobra = "4725e24d-f727-424b-bca0-c4307a3456fa"
        Dingo = "7a7925be-828c-4418-bbeb-bac8dfc843bc"

    src/Bobcat.jl:
        import Cobra
        import Dingo

Cobra/
    Project.toml:
        uuid = "4725e24d-f727-424b-bca0-c4307a3456fa"
        [deps]
        Dingo = "7a7925be-828c-4418-bbeb-bac8dfc843bc"

    src/Cobra.jl:
        import Dingo

Dingo/
    Project.toml:
        uuid = "7a7925be-828c-4418-bbeb-bac8dfc843bc"

    src/Dingo.jl:
        # no imports
\end{lstlisting}



下面是相应的根结构,表示为字典:




\begin{minted}{julia}
roots = Dict(
    :Aardvark => UUID("00000000-0000-0000-0000-000000000000"), # no project file, nil UUID
    :Bobcat   => UUID("85ad11c7-31f6-5d08-84db-0a4914d4cadf"), # dummy UUID based on path
    :Cobra    => UUID("4725e24d-f727-424b-bca0-c4307a3456fa"), # UUID from project file
    :Dingo    => UUID("7a7925be-828c-4418-bbeb-bac8dfc843bc"), # UUID from project file
)
\end{minted}



下面是对应的图结构,表示为字典:




\begin{minted}{julia}
graph = Dict(
    # Bobcat:
    UUID("85ad11c7-31f6-5d08-84db-0a4914d4cadf") => Dict(
        :Cobra => UUID("4725e24d-f727-424b-bca0-c4307a3456fa"),
        :Dingo => UUID("7a7925be-828c-4418-bbeb-bac8dfc843bc"),
    ),
    # Cobra:
    UUID("4725e24d-f727-424b-bca0-c4307a3456fa") => Dict(
        :Dingo => UUID("7a7925be-828c-4418-bbeb-bac8dfc843bc"),
    ),
    # Dingo:
    UUID("7a7925be-828c-4418-bbeb-bac8dfc843bc") => Dict(),
)
\end{minted}



需要注意的一些概括性规则:



\begin{itemize}
\item[1. ] 缺少项目文件的包能依赖于任何顶层依赖项, 并且由于包目录中的每个包在顶层依赖中可用,因此它可以导入在环境中的所有包。


\item[2. ] 含有项目文件的包不能依赖于缺少项目文件的包。 因为有项目文件的包只能加载那些在\texttt{graph}中的包,而没有项目文件的包不会出现在\texttt{graph}。


\item[3. ] 具有项目文件但没有明确UUID的包只能被由没有项目文件的包所依赖, since dummy UUIDs assigned to these packages are strictly internal.

\end{itemize}


,因为赋予给这些包的虚构UUID全是项目内部的。



Observe the following specific instances of these rules in our example: 请注意以下我们例子中的规则具体实例:



\begin{itemize}
\item \texttt{Aardvark} 包可以导入\texttt{Bobcat}、\texttt{Cobra} 或\texttt{Dingo}中的所有包;它确实导入\texttt{Bobcat} and \texttt{Cobra}包.


\item \texttt{Bobcat} 包能导入\texttt{Cobra}与\texttt{Dingo}包。因为它们都有带有UUID的项目文件,并在\texttt{Bobcat}包的\texttt{[deps]}信息项声明为依赖项。


\item \texttt{Bobcat}包不能依赖于\texttt{Aardvark}包,因为\texttt{Aardvark}包缺少项目文件。


\item \texttt{Cobra}包能导入\texttt{Dingo}包。因为\texttt{Dingo}包有项目文件和UUID,并在\texttt{Cobra}的\texttt{[deps]} 信息项中声明为依赖项。


\item \texttt{Cobra}包不能依赖\texttt{Aardvark}或\texttt{Bobcat}包, 因为两者都没有真实的UUID。


\item \texttt{Dingo}包不能导入任何包,因为它的项目文件中缺少\texttt{[deps]}信息项。

\end{itemize}


包目录中的\textbf{路径映射}很简单: 它将子目录名映射到相应的入口点路径。换句话说,如果指向我们示例项目目录的路径是\texttt{/home/me/animals},那么\texttt{路径}映射可以用此字典表示:




\begin{minted}{julia}
paths = Dict(
    (UUID("00000000-0000-0000-0000-000000000000"), :Aardvark) =>
        "/home/me/AnimalPackages/Aardvark/src/Aardvark.jl",
    (UUID("85ad11c7-31f6-5d08-84db-0a4914d4cadf"), :Bobcat) =>
        "/home/me/AnimalPackages/Bobcat/src/Bobcat.jl",
    (UUID("4725e24d-f727-424b-bca0-c4307a3456fa"), :Cobra) =>
        "/home/me/AnimalPackages/Cobra/src/Cobra.jl",
    (UUID("7a7925be-828c-4418-bbeb-bac8dfc843bc"), :Dingo) =>
        "/home/me/AnimalPackages/Dingo/src/Dingo.jl",
)
\end{minted}



根据定义,包目录环境中的所有包都是具有预期入口点文件的子目录,因此它们的\texttt{路径} 映射条目始终具有此格式。



\hypertarget{14356834175019870606}{}


\subsection{环境堆栈}



第三种也是最后一种环境是通过覆盖其中的几个环境来组合其他环境,使每个环境中的包在单个组合环境中可用。这些复合环境称为\emph{环境堆栈}。Julia的\texttt{LOAD\_PATH}全局定义一个环境堆栈——Julia进程在其中运行的环境。如果希望Julia进程只能访问一个项目或包目录中的包,请将其设置为\texttt{LOAD\_PATH}中的唯一条目。然而,访问一些您喜爱的工具(标准库、探查器、调试器、个人实用程序等)通常是非常有用的,即使它们不是您正在处理的项目的依赖项。通过将包含这些工具的环境添加到加载路径,您可以立即在顶层代码中访问它们,而无需将它们添加到项目中。



组合环境堆栈组件中根、图和路径的数据结构的机制很简单:它们被作为字典进行合并, 在发生键冲突时,优先使用前面的条目而不是后面的条目。换言之,如果我们有\texttt{stack = [env₁, env₂, …]},那么我们有:




\begin{minted}{julia}
roots = reduce(merge, reverse([roots₁, roots₂, …]))
graph = reduce(merge, reverse([graph₁, graph₂, …]))
paths = reduce(merge, reverse([paths₁, paths₂, …]))
\end{minted}



带下标的 \texttt{rootsᵢ}, \texttt{graphᵢ} and \texttt{pathsᵢ}变量对应于在\texttt{stack}中包含的下标环境变量\texttt{envᵢ}。 使用\texttt{reverse} 是因为当参数字典中的键之间发生冲突时,使\texttt{merge} 倾向于使用最后一个参数,而不是第一个参数。这种设计有几个值得注意的特点:



\begin{itemize}
\item[1. ] \emph{主环境}——即堆栈中的第一个环境,被准确地嵌入到堆栈环境中。堆栈中第一个环境的完整依赖关系图是必然被完整包括在含有所有相同版本的依赖项的堆栈环境中。


\item[2. ] 非主环境中的包能最终使用与其依赖项不兼容的版本,即使它们自己的环境是完全兼容。这种情况可能发生,当它们的一个依赖项被堆栈(通过图或路径,或两者)中某个早期环境中的版本所覆盖。

\end{itemize}


由于主环境通常是您正在处理的项目所在的环境,而堆栈中稍后的环境包含其他工具, 因此这是正确的权衡:最好改进您的开发工具,但保持项目能工作。当这种不兼容发生时,你通常要将开发工具升级到与主项目兼容的版本。



\hypertarget{16711373200664757596}{}


\section{总结}



在软件包系统中,联邦软件包管理和精确的软件可复制性是困难但有价值的目标。结合起来,这些目标导致了一个比大多数动态语言更加复杂的包加载机制,但它也产生了通常与静态语言相关的可伸缩性和可复制性。通常,Julia用户应该能够使用内置的包管理器来管理他们的项目,而无需精确理解这些交互细节。通过调用\texttt{Pkg.add({\textquotedbl}X{\textquotedbl})}添加\texttt{X}包到对应的项目,并清晰显示相关文件,选择\texttt{Pkg.activate({\textquotedbl}Y{\textquotedbl})}后, 可调用\texttt{import X} 即可加载\texttt{X}包,而无需作过多考虑。



\hypertarget{9798162676380759856}{}


\chapter{性能分析}



\texttt{Profile} 模块提供了一些工具来帮助开发者提高其代码的性能。在使用时,它运行代码并进行测量,并生成输出,该输出帮助你了解在每行(或几行)上花费了多少时间。最常见的用法是识别性能「瓶颈」并将其作为优化目标。



\texttt{Profile} 实现了所谓的「抽样」或\href{https://en.wikipedia.org/wiki/Profiling\_(computer\_programming)}{统计分析器}。它通过在执行任何任务期间定期进行回溯来工作。每次回溯捕获当前运行的函数和行号,以及导致该行执行的完整函数调用链,因此是当前执行状态的「快照」。



如果大部分运行时间都花在执行特定代码行上,则此行会在所有回溯的集合中频繁出现。换句话说,执行给定行的「成本」——或实际上,调用及包含此行的函数序列的成本——与它在所有回溯的集合中的出现频率成正比。



抽样分析器不提供完整的逐行覆盖功能,因为回溯是间隔发生的(默认情况下,该时间间隔在 Unix 上是 1 ms,而在 Windows 上是 10 ms,但实际调度受操作系统负载的影响)。此外,正如下文中进一步讨论的,因为样本是在所有执行点的稀疏子集处收集的,所以抽样分析器收集的数据会受到统计噪声的影响。



尽管有这些限制,但抽样分析器仍然有很大的优势:



\begin{itemize}
\item You do not have to make any modifications to your code to take timing measurements.


\item 它可以分析 Julia 的核心代码,甚至(可选)可以分析 C 和 Fortran 库。


\item 通过「偶尔」运行,它只有很少的性能开销;代码在性能分析时能以接近本机的速度运行。

\end{itemize}


出于这些原因,建议你在考虑任何替代方案前尝试使用内置的抽样分析器。



\hypertarget{17553054695497272357}{}


\section{基本用法}



让我们使用一个简单的测试用例:




\begin{minted}{jlcon}
julia> function myfunc()
           A = rand(200, 200, 400)
           maximum(A)
       end
\end{minted}



最好先至少运行一次你想要分析的代码(除非你想要分析 Julia 的 JIT 编译器):




\begin{minted}{jlcon}
julia> myfunc() # run once to force compilation
\end{minted}



现在我们准备分析这个函数:




\begin{minted}{jlcon}
julia> using Profile

julia> @profile myfunc()
\end{minted}



To see the profiling results, there are several graphical browsers. One {\textquotedbl}family{\textquotedbl} of visualizers is based on \href{https://github.com/timholy/FlameGraphs.jl}{FlameGraphs.jl}, with each family member providing a different user interface:



\begin{itemize}
\item \href{https://junolab.org/}{Juno} is a full IDE with built-in support for profile visualization


\item \href{https://github.com/timholy/ProfileView.jl}{ProfileView.jl} is a stand-alone visualizer based on GTK


\item \href{https://github.com/davidanthoff/ProfileVega.jl}{ProfileVega.jl} uses VegaLight and integrates well with Jupyter notebooks


\item \href{https://github.com/tkluck/StatProfilerHTML.jl}{StatProfilerHTML} produces HTML and presents some additional summaries, and also integrates well with Jupyter notebooks


\item \href{https://github.com/timholy/ProfileSVG.jl}{ProfileSVG} renders SVG

\end{itemize}


An entirely independent approach to profile visualization is \href{https://github.com/vchuravy/PProf.jl}{PProf.jl}, which uses the external \texttt{pprof} tool.



Here, though, we{\textquotesingle}ll use the text-based display that comes with the standard library:




\begin{minted}{jlcon}
julia> Profile.print()
80 ./event.jl:73; (::Base.REPL.##1#2{Base.REPL.REPLBackend})()
 80 ./REPL.jl:97; macro expansion
  80 ./REPL.jl:66; eval_user_input(::Any, ::Base.REPL.REPLBackend)
   80 ./boot.jl:235; eval(::Module, ::Any)
    80 ./<missing>:?; anonymous
     80 ./profile.jl:23; macro expansion
      52 ./REPL[1]:2; myfunc()
       38 ./random.jl:431; rand!(::MersenneTwister, ::Array{Float64,3}, ::Int64, ::Type{B...
        38 ./dSFMT.jl:84; dsfmt_fill_array_close_open!(::Base.dSFMT.DSFMT_state, ::Ptr{F...
       14 ./random.jl:278; rand
        14 ./random.jl:277; rand
         14 ./random.jl:366; rand
          14 ./random.jl:369; rand
      28 ./REPL[1]:3; myfunc()
       28 ./reduce.jl:270; _mapreduce(::Base.#identity, ::Base.#scalarmax, ::IndexLinear,...
        3  ./reduce.jl:426; mapreduce_impl(::Base.#identity, ::Base.#scalarmax, ::Array{F...
        25 ./reduce.jl:428; mapreduce_impl(::Base.#identity, ::Base.#scalarmax, ::Array{F...
\end{minted}



显示结果中的每行表示代码中的特定点(行数)。缩进用来标明嵌套的函数调用序列,其中缩进更多的行在调用序列中更深。在每一行中,第一个「字段」是在\emph{这一行或由这一行执行的任何函数}中获取的回溯(样本)数量。第二个字段是文件名和行数,第三个字段是函数名。请注意,具体的行号可能会随着 Julia 代码的改变而改变;如果你想跟上,最好自己运行这个示例。



在此例中,我们可以看到顶层的调用函数位于文件 \texttt{event.jl} 中。这是启动 Julia 时运行 REPL 的函数。如果你查看 \texttt{REPL.jl} 的第 97 行,你会看到这是调用函数 \texttt{eval\_user\_input()} 的地方。这是对你在 REPL 上的输入进行求值的函数,因为我们正以交互方式运行,所以当我们输入 \texttt{@profile myfunc()} 时会调用这些函数。下一行反映了 \hyperlink{9691715859147716436}{\texttt{@profile}} 所采取的操作。



第一行显示在 \texttt{event.jl} 的第 73 行获取了 80 次回溯,但这并不是说此行本身「昂贵」:第三行表明所有这些 80 次回溯实际上它调用的 \texttt{eval\_user\_input} 中触发的,以此类推。为了找出实际占用时间的操作,我们需要深入了解调用链。



此输出中第一个「重要」的行是这行:




\begin{lstlisting}
52 ./REPL[1]:2; myfunc()
\end{lstlisting}



\texttt{REPL} 指的是我们在 REPL 中定义了 \texttt{myfunc},而不是把它放在文件中;如果我们使用文件,这将显示文件名。\texttt{[1]} 表示函数 \texttt{myfunc} 是在当前 REPL 会话中第一个进行求值的表达式。\texttt{myfunc()} 的第 2 行包含对 \texttt{rand} 的调用,(80 次中)有 52 次回溯发生在该行。在此之下,你可以看到在 \texttt{dSFMT.jl} 中对 \texttt{dsfmt\_fill\_array\_close\_open!} 的调用。



更进一步,你会看到:




\begin{lstlisting}
28 ./REPL[1]:3; myfunc()
\end{lstlisting}



\texttt{myfunc} 的第 3 行包含对 \texttt{maximum} 的调用,(80 次中)有 28 次回溯发生在这里。在此之下,你可以看到对于这种类型的输入数据,\texttt{maximum} 函数中执行的耗时操作在 \texttt{base/reduce.jl} 中的具体位置。



总的来说,我们可以暂时得出结论,生成随机数的成本大概是找到最大元素的两倍。通过收集更多样本,我们可以增加对此结果的信心:




\begin{minted}{jlcon}
julia> @profile (for i = 1:100; myfunc(); end)

julia> Profile.print()
[....]
 3821 ./REPL[1]:2; myfunc()
  3511 ./random.jl:431; rand!(::MersenneTwister, ::Array{Float64,3}, ::Int64, ::Type...
   3511 ./dSFMT.jl:84; dsfmt_fill_array_close_open!(::Base.dSFMT.DSFMT_state, ::Ptr...
  310  ./random.jl:278; rand
   [....]
 2893 ./REPL[1]:3; myfunc()
  2893 ./reduce.jl:270; _mapreduce(::Base.#identity, ::Base.#scalarmax, ::IndexLinea...
   [....]
\end{minted}



一般来说,如果你在某行上收集到 \texttt{N} 个样本,那你可以预期其有 \texttt{sqrt(N)} 的不确定性(忽略其它噪音源,比如计算机在其它任务上的繁忙程度)。这个规则的主要例外是垃圾收集,它很少运行但往往成本高昂。(因为 Julia 的垃圾收集器是用 C 语言编写的,此类事件可使用下文描述的 \texttt{C=true} 输出模式来检测,或者使用 \href{https://github.com/timholy/ProfileView.jl}{ProfileView.jl} 来检测。)



这展示了默认的「树」形转储;另一种选择是「扁平」形转储,它会累积与其嵌套无关的计数:




\begin{minted}{jlcon}
julia> Profile.print(format=:flat)
 Count File          Line Function
  6714 ./<missing>     -1 anonymous
  6714 ./REPL.jl       66 eval_user_input(::Any, ::Base.REPL.REPLBackend)
  6714 ./REPL.jl       97 macro expansion
  3821 ./REPL[1]        2 myfunc()
  2893 ./REPL[1]        3 myfunc()
  6714 ./REPL[7]        1 macro expansion
  6714 ./boot.jl      235 eval(::Module, ::Any)
  3511 ./dSFMT.jl      84 dsfmt_fill_array_close_open!(::Base.dSFMT.DSFMT_s...
  6714 ./event.jl      73 (::Base.REPL.##1#2{Base.REPL.REPLBackend})()
  6714 ./profile.jl    23 macro expansion
  3511 ./random.jl    431 rand!(::MersenneTwister, ::Array{Float64,3}, ::In...
   310 ./random.jl    277 rand
   310 ./random.jl    278 rand
   310 ./random.jl    366 rand
   310 ./random.jl    369 rand
  2893 ./reduce.jl    270 _mapreduce(::Base.#identity, ::Base.#scalarmax, :...
     5 ./reduce.jl    420 mapreduce_impl(::Base.#identity, ::Base.#scalarma...
   253 ./reduce.jl    426 mapreduce_impl(::Base.#identity, ::Base.#scalarma...
  2592 ./reduce.jl    428 mapreduce_impl(::Base.#identity, ::Base.#scalarma...
    43 ./reduce.jl    429 mapreduce_impl(::Base.#identity, ::Base.#scalarma...
\end{minted}



如果你的代码有递归,那么可能令人困惑的就是「子」函数中的行的累积计数可以多于总回溯次数。考虑以下函数定义:




\begin{minted}{julia}
dumbsum(n::Integer) = n == 1 ? 1 : 1 + dumbsum(n-1)
dumbsum3() = dumbsum(3)
\end{minted}



如果你要分析 \texttt{dumbsum3},并在执行 \texttt{dumbsum(1)} 时执行了回溯,那么该回溯将如下所示:




\begin{minted}{julia}
dumbsum3
    dumbsum(3)
        dumbsum(2)
            dumbsum(1)
\end{minted}



因此,即使父函数只获得 1 个计数,这个子函数也会获得 3 个计数。「树」形表示使这更清晰,因此(以及其它原因)可能是查看结果的最实用方法。



\hypertarget{11920246320852784321}{}


\section{结果累积和清空}



\hyperlink{9691715859147716436}{\texttt{@profile}} 的结果会累积在一个缓冲区中;如果你在 \hyperlink{9691715859147716436}{\texttt{@profile}} 下运行多端代码,那么 \hyperlink{2955792207246042270}{\texttt{Profile.print()}} 会显示合并的结果。这可能非常有用,但有时你会想重新开始,这可通过 \hyperlink{15414823368700214048}{\texttt{Profile.clear()}}。



\hypertarget{3561003529463300631}{}


\section{用于控制性能分析结果显示的选项}



\hyperlink{2955792207246042270}{\texttt{Profile.print}} 还有一些未曾描述的选项。让我们看看完整的声明:




\begin{minted}{julia}
function print(io::IO = stdout, data = fetch(); kwargs...)
\end{minted}



我们先讨论两个位置参数,然后讨论关键字参数:



\begin{itemize}
\item \texttt{io}——允许你将结果保存到缓冲区,例如一个文件,但默认是打印到 \texttt{stdout}(控制台)。


\item \texttt{data}——包含你要分析的数据;默认情况下,它是从 \hyperlink{3578108280181558112}{\texttt{Profile.fetch()}} 中获取的,该函数从预先分配的缓冲区中拉出回溯。例如,如果你要分析性能分析器,可以说:


\begin{minted}{julia}
data = copy(Profile.fetch())
Profile.clear()
@profile Profile.print(stdout, data) # Prints the previous results
Profile.print()                      # Prints results from Profile.print()
\end{minted}

\end{itemize}


关键字参数可以是以下参数的任意组合:



\begin{itemize}
\item \texttt{format}——上文已经介绍,确定是使用(默认值,\texttt{:tree})还是不使用(\texttt{:flat})缩进来表示其树形结构。


\item \texttt{C}——如果为 \texttt{true},则显示 C 和 Fortran 代码中的回溯(通常它们被排除在外)。请尝试用 \texttt{Profile.print(C = true)} 运行介绍性示例。这对于判断是 Julia 代码还是 C 代码导致了性能瓶颈非常有帮助;设置 \texttt{C = true} 也可提高嵌套的可解释性,代价是更长的性能分析转储。


\item \texttt{combine}——某些代码行包含多个操作;例如,\texttt{s += A[i]} 包含一个数组引用(\texttt{A[i]})和一个求和操作。这些操作在所生成的机器代码中对应不同的行,因此回溯期间可能会在此行中捕获两个或以上地址。\texttt{combine = true} 把它们混合在一起,可能你通常想要这样,但使用 \texttt{combine = false},你可为每个唯一的指令指针单独生成输出。


\item \texttt{maxdepth}——限制 \texttt{:tree} 格式中深度大于 \texttt{maxdepth} 的帧。


\item   * \texttt{sortedby}——控制 \texttt{:flat} 格式中的次序。为 \texttt{:filefuncline}(默认值)时按源代码行排序,而为 \texttt{:count} 时按收集的样本数排序。


\item \texttt{noisefloor}——限制低于样本的启发式噪音下限的帧(只适用于格式 \texttt{:tree})。尝试此选项的建议值是 2.0(默认值是 0)。此参数会隐藏 \texttt{n <= noisefloor * √N} 的样本,其中 \texttt{n} 是该行上的样本数,\texttt{N} 是被调用者的样本数。


\item \texttt{mincount}——限制出现次数少于 \texttt{mincount} 的帧。

\end{itemize}


文件/函数名有时会被(用 \texttt{...})截断,缩进也有可能在开头用 \texttt{+n} 截断,其中 \texttt{n} 是在空间充足的情况下应该插入的额外空格数。如果你想要深层嵌套代码的完整性能分析,保存到文件并在 \hyperlink{13454403377667762339}{\texttt{IOContext}} 中使用宽的 \texttt{displaysize} 通常是个好主意:




\begin{minted}{julia}
open("/tmp/prof.txt", "w") do s
    Profile.print(IOContext(s, :displaysize => (24, 500)))
end
\end{minted}



\hypertarget{5894887495287635363}{}


\section{配置}



\hyperlink{9691715859147716436}{\texttt{@profile}} 只是累积回溯,在你调用 \hyperlink{2955792207246042270}{\texttt{Profile.print()}} 时才会进行性能分析。对于长时间运行的计算,完全有可能把用于存储回溯的预分配缓冲区填满。如果发生这种情况,回溯会停止,但你的计算会继续。因此,你也许会丢失一些重要的性能分析数据(当发生这种情况时,你会受到警告)。



你可通过以下方式获取和配置相关参数:




\begin{minted}{julia}
Profile.init() # returns the current settings
Profile.init(n = 10^7, delay = 0.01)
\end{minted}



\texttt{n} 是能够存储的指令指针总数,默认值为 \texttt{10{\textasciicircum}6}。如果通常的回溯是 20 个指令指针,那么可以收集 50000 次回溯,这意味着统计不确定性少于 1\%。这对于大多数应用来说可能已经足够了。



因此,你更可能需要修改 \texttt{delay},它以秒为单位,设置在快照之间 Julia 用于执行所请求计算的时长。长时间运行的工作可能不��要经常回溯。默认设置为 \texttt{delay = 0.001}。当然,你可以减少和增加 delay;但是,一旦 delay 接近执行一次回溯所需的时间(在作者的笔记本上约为 30 微妙),性能分析的开销就会增加。



\hypertarget{15664789060024554869}{}


\chapter{内存分配分析}



减少内存分配是提高性能的最常用技术之一。内存分配总量可以用 \hyperlink{8029752041511656628}{\texttt{@time}} 和 \hyperlink{5377755456008435782}{\texttt{@allocated}},触发内存分配的特定行通常可以通过这些行产生的垃圾分配成本从性能分析中推断出来。但是,直接测量每行代码的内存分配总量有时会更高效。



为了逐行测量内存分配,启动 Julia 时请使用命令行选项 \texttt{--track-allocation=<setting>},该选项的可选值有 \texttt{none}(默认值,不测量内存分配)、\texttt{user}(测量除 Julia core 代码之外的所有代码的内存分配)或 \texttt{all}(测量 Julia 代码中每一行的内存分配)。这会为每行已编译的代码测量内存。在退出 Julia 时,累积的结果将写入到文本文件中,此文本文件名称为该文件名称后加 \texttt{.mem},并与源文件位于同一目录下。该文件的每行列出内存分配的总字节数。\href{https://github.com/JuliaCI/Coverage.jl}{\texttt{Coverage} 包}包括了一些基本分析工具,例如,按照内存分配的字节数对行进行排序的工具。



在解释结果时,有一些需要注意的细节。在 \texttt{user} 设定下,直接从 REPL 调用的任何函数的第一行都将会显示内存分配,这是由发生在 REPL 代码本身的事件造成的。更重要的是,JIT 编译也会添加内存分配计数,因为 Julia 的编译器大部分是用 Julia 编写的(并且编译通常需要内存分配)。建议的分析过程是先通过执行待分析的所有命令来强制编译,然后调用 \hyperlink{12697816172521511938}{\texttt{Profile.clear\_malloc\_data()}} 来重置所有内存计数器。最后,执行所需的命令并退出 Julia 以触发 \texttt{.mem} 文件的生成。



\hypertarget{7694373006850867155}{}


\chapter{外部性能分析}



Julia 目前支持的外部性能分析工具有 \texttt{Intel VTune}、\texttt{OProfile} 和 \texttt{perf}。



根据你所选择的工具,编译时请在 \texttt{Make.user} 中将 \texttt{USE\_INTEL\_JITEVENTS}、\texttt{USE\_OPROFILE\_JITEVENTS} 和 \texttt{USE\_PERF\_JITEVENTS} 设置为 1。多个上述编译标志是支持的。



在运行 Julia 前,请将环境变量 \texttt{ENABLE\_JITPROFILING} 设置为 1。



现在,你可以通过多种方式使用这些工具!例如,可以使用 \texttt{OProfile} 来尝试做个简单的记录:




\begin{lstlisting}
>ENABLE_JITPROFILING=1 sudo operf -Vdebug ./julia test/fastmath.jl
>opreport -l `which ./julia`
\end{lstlisting}



Or similary with \texttt{perf} :




\begin{lstlisting}
$ ENABLE_JITPROFILING=1 perf record -o /tmp/perf.data --call-graph dwarf ./julia /test/fastmath.jl
$ perf report --call-graph -G
\end{lstlisting}



你可以测量关于程序的更多有趣数据,若要获得详尽的列表,请阅读 \href{http://www.brendangregg.com/perf.html}{Linux perf 示例页面}。



请记住,perf 会为每次执行保存一个 \texttt{perf.data} 文件,即使对于小程序,它也可能变得非常大。此外,perf LLVM 模块会将调试对象保存在 \texttt{{\textasciitilde}/.debug/jit} 中,记得经常清理该文件夹。



\hypertarget{5468549229850022418}{}


\chapter{栈跟踪}



\texttt{StackTraces} 模块提供了简单的栈跟踪功能,这些栈跟踪信息既可读又易于编程使用。



\hypertarget{18224694583461024319}{}


\section{查看栈跟踪}



获取栈跟踪信息的主要函数是 \hyperlink{11964270650763140298}{\texttt{stacktrace}}:




\begin{minted}{jlcon}
6-element Array{Base.StackTraces.StackFrame,1}:
 top-level scope
 eval at boot.jl:317 [inlined]
 eval(::Module, ::Expr) at REPL.jl:5
 eval_user_input(::Any, ::REPL.REPLBackend) at REPL.jl:85
 macro expansion at REPL.jl:116 [inlined]
 (::getfield(REPL, Symbol("##28#29")){REPL.REPLBackend})() at event.jl:92
\end{minted}



调用 \hyperlink{11964270650763140298}{\texttt{stacktrace()}} 会返回一个 \hyperlink{16824886840215699957}{\texttt{StackTraces.StackFrame}} 数组。为了使用方便,可以用 \hyperlink{12289172590874430030}{\texttt{StackTraces.StackTrace}} 来代替 \texttt{Vector\{StackFrame\}}。下面例子中 \texttt{[...]} 的意思是这部分输出的内容可能会根据代码的实际执行情况而定。




\begin{minted}{jlcon}
julia> example() = stacktrace()
example (generic function with 1 method)

julia> example()
7-element Array{Base.StackTraces.StackFrame,1}:
 example() at REPL[1]:1
 top-level scope
 eval at boot.jl:317 [inlined]
[...]

julia> @noinline child() = stacktrace()
child (generic function with 1 method)

julia> @noinline parent() = child()
parent (generic function with 1 method)

julia> grandparent() = parent()
grandparent (generic function with 1 method)

julia> grandparent()
9-element Array{Base.StackTraces.StackFrame,1}:
 child() at REPL[3]:1
 parent() at REPL[4]:1
 grandparent() at REPL[5]:1
[...]
\end{minted}



注意,在调用 \hyperlink{11964270650763140298}{\texttt{stacktrace()}} 的时,通常会出现 \texttt{eval at boot.jl} 这帧。 当从 REPL 里调用 \hyperlink{11964270650763140298}{\texttt{stacktrace()}} 的时候,还会显示 \texttt{REPL.jl} 里的一些额外帧,就像下面一样:




\begin{minted}{jlcon}
julia> example() = stacktrace()
example (generic function with 1 method)

julia> example()
7-element Array{Base.StackTraces.StackFrame,1}:
 example() at REPL[1]:1
 top-level scope
 eval at boot.jl:317 [inlined]
 eval(::Module, ::Expr) at REPL.jl:5
 eval_user_input(::Any, ::REPL.REPLBackend) at REPL.jl:85
 macro expansion at REPL.jl:116 [inlined]
 (::getfield(REPL, Symbol("##28#29")){REPL.REPLBackend})() at event.jl:92
\end{minted}



\hypertarget{17184729199221037868}{}


\section{抽取有用信息}



每个 \hyperlink{16824886840215699957}{\texttt{StackTraces.StackFrame}} 都会包含函数名,文件名,代码行数,lambda 信息,一个用于确认此帧是否被内联的标帜,一个用于确认函数是否为 C 函数的标帜(在默认的情况下 C 函数不会出现在栈跟踪信息中)以及一个用整数表示的指针,它是由 \hyperlink{6187626674327343338}{\texttt{backtrace}} 返回的:




\begin{minted}{jlcon}
julia> frame = stacktrace()[3]
eval(::Module, ::Expr) at REPL.jl:5

julia> frame.func
:eval

julia> frame.file
Symbol("~/julia/usr/share/julia/stdlib/v0.7/REPL/src/REPL.jl")

julia> frame.line
5

julia> top_frame.linfo
MethodInstance for eval(::Module, ::Expr)

julia> top_frame.inlined
false

julia> top_frame.from_c
false
\end{minted}




\begin{minted}{jlcon}
julia> top_frame.pointer
0x00007f92d6293171
\end{minted}



这使得我们可以通过编程的方式将栈跟踪信息用于打印日志,处理错误以及其它更多用途。



\hypertarget{6556726272179975372}{}


\section{错误处理}



能够轻松地获取当前调用栈的状态信息在许多场景下都很有用,但最直接的应用是错误处理和调试。




\begin{minted}{jlcon}
julia> @noinline bad_function() = undeclared_variable
bad_function (generic function with 1 method)

julia> @noinline example() = try
           bad_function()
       catch
           stacktrace()
       end
example (generic function with 1 method)

julia> example()
7-element Array{Base.StackTraces.StackFrame,1}:
 example() at REPL[2]:4
 top-level scope
 eval at boot.jl:317 [inlined]
[...]
\end{minted}



你可能已经注意到了,上述例子中第一个栈帧指向了 \hyperlink{11964270650763140298}{\texttt{stacktrace}} 被调用的第 4 行,而不是 \texttt{bad\_function} 被调用的第 2 行,且完全没有出现 \texttt{bad\_function} 的栈帧。这是也是可以理解的,因为 \hyperlink{11964270650763140298}{\texttt{stacktrace}} 是在 \texttt{catch} 的上下文中被调用的。虽然在这个例子中很容易查找到错误的真正源头,但在复杂的情况下查找错误源并不是一件容易的事。



为了补救,我们可以将 \hyperlink{98342946516168163}{\texttt{catch\_backtrace}} 的输出传递给 \hyperlink{11964270650763140298}{\texttt{stacktrace}}。\hyperlink{98342946516168163}{\texttt{catch\_backtrace}} 会返回最近发生异常的上下文中的栈信息,而不是返回当前上下文中的调用栈信息。




\begin{minted}{jlcon}
julia> @noinline bad_function() = undeclared_variable
bad_function (generic function with 1 method)

julia> @noinline example() = try
           bad_function()
       catch
           stacktrace(catch_backtrace())
       end
example (generic function with 1 method)

julia> example()
8-element Array{Base.StackTraces.StackFrame,1}:
 bad_function() at REPL[1]:1
 example() at REPL[2]:2
[...]
\end{minted}



可以看到,现在栈跟踪会显示正确的行号以及之前缺失的栈帧。




\begin{minted}{jlcon}
julia> @noinline child() = error("Whoops!")
child (generic function with 1 method)

julia> @noinline parent() = child()
parent (generic function with 1 method)

julia> @noinline function grandparent()
           try
               parent()
           catch err
               println("ERROR: ", err.msg)
               stacktrace(catch_backtrace())
           end
       end
grandparent (generic function with 1 method)

julia> grandparent()
ERROR: Whoops!
10-element Array{Base.StackTraces.StackFrame,1}:
 error at error.jl:33 [inlined]
 child() at REPL[1]:1
 parent() at REPL[2]:1
 grandparent() at REPL[3]:3
[...]
\end{minted}



\hypertarget{11791934492396691703}{}


\section{异常栈与\texttt{catch\_stack}}



\begin{quote}
\textbf{Julia 1.1}

异常栈需要 Julia 1.1 及以上版本。

\end{quote}


在处理一个异常时,后续的异常同样可能被抛出。观察这些异常对定位问题的源头极有帮助。Julia runtime 支持将每个异常发生后推入一个内部的\emph{异常栈}。当代码正常退出一个\texttt{catch}语句,可认为所有被推入栈中的异常在相应的\texttt{try}语句中被成功处理并已从栈中移除。



存放当前异常的栈可通过测试函数 \hyperlink{5950075931444385711}{\texttt{Base.catch\_stack}} 获取,例如




\begin{minted}{jlcon}
julia> try
           error("(A) The root cause")
       catch
           try
               error("(B) An exception while handling the exception")
           catch
               for (exc, bt) in Base.catch_stack()
                   showerror(stdout, exc, bt)
                   println()
               end
           end
       end
(A) The root cause
Stacktrace:
 [1] error(::String) at error.jl:33
 [2] top-level scope at REPL[7]:2
 [3] eval(::Module, ::Any) at boot.jl:319
 [4] eval_user_input(::Any, ::REPL.REPLBackend) at REPL.jl:85
 [5] macro expansion at REPL.jl:117 [inlined]
 [6] (::getfield(REPL, Symbol("##26#27")){REPL.REPLBackend})() at task.jl:259
(B) An exception while handling the exception
Stacktrace:
 [1] error(::String) at error.jl:33
 [2] top-level scope at REPL[7]:5
 [3] eval(::Module, ::Any) at boot.jl:319
 [4] eval_user_input(::Any, ::REPL.REPLBackend) at REPL.jl:85
 [5] macro expansion at REPL.jl:117 [inlined]
 [6] (::getfield(REPL, Symbol("##26#27")){REPL.REPLBackend})() at task.jl:259
\end{minted}



在本例中,根源异常(A)排在栈头,其后放置着延伸异常(B)。 在正常退出(例如,不抛出新异常)两个 catch 块后,所有异常都被移除出栈,无法访问。



异常栈被存放于发生异常的 \texttt{Task} 处。当某个任务失败,出现意料外的异常时,\texttt{catch\_stack(task)} 可能会被用于观察该任务的异常栈。



\hypertarget{9581632785664784530}{}


\section{\texttt{stacktrace} 与 \texttt{backtrace} 的比较}



调用 \hyperlink{6187626674327343338}{\texttt{backtrace}} 会返回一个 \texttt{Union\{Ptr\{Nothing\}, Base.InterpreterIP\}} 的数组,可以将其传给 \hyperlink{11964270650763140298}{\texttt{stacktrace}} 函数进行转化:




\begin{minted}{jlcon}
julia> trace = backtrace()
18-element Array{Union{Ptr{Nothing}, Base.InterpreterIP},1}:
 Ptr{Nothing} @0x00007fd8734c6209
 Ptr{Nothing} @0x00007fd87362b342
 Ptr{Nothing} @0x00007fd87362c136
 Ptr{Nothing} @0x00007fd87362c986
 Ptr{Nothing} @0x00007fd87362d089
 Base.InterpreterIP(CodeInfo(:(begin
      Core.SSAValue(0) = backtrace()
      trace = Core.SSAValue(0)
      return Core.SSAValue(0)
  end)), 0x0000000000000000)
 Ptr{Nothing} @0x00007fd87362e4cf
[...]

julia> stacktrace(trace)
6-element Array{Base.StackTraces.StackFrame,1}:
 top-level scope
 eval at boot.jl:317 [inlined]
 eval(::Module, ::Expr) at REPL.jl:5
 eval_user_input(::Any, ::REPL.REPLBackend) at REPL.jl:85
 macro expansion at REPL.jl:116 [inlined]
 (::getfield(REPL, Symbol("##28#29")){REPL.REPLBackend})() at event.jl:92
\end{minted}



需要注意的是,\hyperlink{6187626674327343338}{\texttt{backtrace}} 返回的向量有 18 个元素,而 \hyperlink{11964270650763140298}{\texttt{stacktrace}} 返回的向量只包含6 个元素。这是因为 \hyperlink{11964270650763140298}{\texttt{stacktrace}} 在默认情况下会移除所有底层 C 函数的栈信息。如果你想显示 C 函数调用的栈帧,可以这样做:




\begin{minted}{jlcon}
julia> stacktrace(trace, true)
21-element Array{Base.StackTraces.StackFrame,1}:
 jl_apply_generic at gf.c:2167
 do_call at interpreter.c:324
 eval_value at interpreter.c:416
 eval_body at interpreter.c:559
 jl_interpret_toplevel_thunk_callback at interpreter.c:798
 top-level scope
 jl_interpret_toplevel_thunk at interpreter.c:807
 jl_toplevel_eval_flex at toplevel.c:856
 jl_toplevel_eval_in at builtins.c:624
 eval at boot.jl:317 [inlined]
 eval(::Module, ::Expr) at REPL.jl:5
 jl_apply_generic at gf.c:2167
 eval_user_input(::Any, ::REPL.REPLBackend) at REPL.jl:85
 jl_apply_generic at gf.c:2167
 macro expansion at REPL.jl:116 [inlined]
 (::getfield(REPL, Symbol("##28#29")){REPL.REPLBackend})() at event.jl:92
 jl_fptr_trampoline at gf.c:1838
 jl_apply_generic at gf.c:2167
 jl_apply at julia.h:1540 [inlined]
 start_task at task.c:268
 ip:0xffffffffffffffff
\end{minted}



\hyperlink{6187626674327343338}{\texttt{backtrace}} 返回的单个指针可以通过 \hyperlink{1451426077045795515}{\texttt{StackTraces.lookup}} 来转化成一组 \hyperlink{16824886840215699957}{\texttt{StackTraces.StackFrame}}:




\begin{minted}{jlcon}
julia> pointer = backtrace()[1];

julia> frame = StackTraces.lookup(pointer)
1-element Array{Base.StackTraces.StackFrame,1}:
 jl_apply_generic at gf.c:2167

julia> println("The top frame is from $(frame[1].func)!")
The top frame is from jl_apply_generic!
\end{minted}



\hypertarget{3908315974291496321}{}


\chapter{性能建议}



下面几节简要地介绍了一些使 Julia 代码运行得尽可能快的技巧。



\hypertarget{14492220707033250800}{}


\section{避免全局变量}



全局变量的值和类型随时都会发生变化, 这使编译器难以优化使用全局变量的代码。 变量应该是局部的,或者尽可能作为参数传递给函数。



任何注重性能或者需要测试性能的代码都应该被放置在函数之中。



我们发现全局变量经常是常量,将它们声明为常量可大幅提升性能。




\begin{minted}{julia}
const DEFAULT_VAL = 0
\end{minted}



对于非常量的全局变量可以通过在使用的时候标注它们的类型来优化。




\begin{minted}{julia}
global x = rand(1000)

function loop_over_global()
    s = 0.0
    for i in x::Vector{Float64}
        s += i
    end
    return s
end
\end{minted}



一个更好的编程风格是将变量作为参数传给函数。这样可以使得代码更易复用,以及清晰的展示函数的输入和输出。



\begin{quote}
\textbf{Note}

All code in the REPL is evaluated in global scope, so a variable defined and assigned at top level will be a \textbf{global} variable. Variables defined at top level scope inside modules are also global.

\end{quote}


在下面的REPL会话中:




\begin{minted}{jlcon}
julia> x = 1.0
\end{minted}



等价于:




\begin{minted}{jlcon}
julia> global x = 1.0
\end{minted}



因此,所有上文关于性能问题的讨论都适用于它们。



\hypertarget{1547856480373223464}{}


\section{使用 \texttt{@time}评估性能以及注意内存分配}



\hyperlink{8029752041511656628}{\texttt{@time}} 宏是一个有用的性能评估工具。这里我们将重复上面全局变量的例子,但是这次移除类型声明:




\begin{minted}{jlcon}
julia> x = rand(1000);

julia> function sum_global()
           s = 0.0
           for i in x
               s += i
           end
           return s
       end;

julia> @time sum_global()
  0.017705 seconds (15.28 k allocations: 694.484 KiB)
496.84883432553846

julia> @time sum_global()
  0.000140 seconds (3.49 k allocations: 70.313 KiB)
496.84883432553846
\end{minted}



在第一次调用函数(\texttt{@time sum\_global()})的时候,它会被编译。如果你这次会话中还没有使用过\hyperlink{8029752041511656628}{\texttt{@time}},这时也会编译计时需要的相关函数。你不必认真对待这次运行的结果。接下来看第二次运行,除了运行的耗时以外,它还表明了分配了大量的内存。我们这里仅仅是计算了一个64位浮点向量元素和,因此这里应该没有申请内存的必要(至少不用在\texttt{@time}报告的堆上申请内存)。



未被预料的内存分配往往说明你的代码中存在一些问题,这些问题常常是由于类型的稳定性或者创建了太多临时的小数组。因此,除了分配内存本身,这也很可能说明你所写的函数远没有生成性能良好的代码。认真对待这些现象,遵循接下来的建议。



如果你换成将\texttt{x}作为参数传给函数,就可以避免内存的分配(这里报告的内存分配是由于在全局作用域中运行\texttt{@time}导致的),而且在第一次运行之后运行速度也会得到显著的提高。




\begin{minted}{jlcon}
julia> x = rand(1000);

julia> function sum_arg(x)
           s = 0.0
           for i in x
               s += i
           end
           return s
       end;

julia> @time sum_arg(x)
  0.007701 seconds (821 allocations: 43.059 KiB)
496.84883432553846

julia> @time sum_arg(x)
  0.000006 seconds (5 allocations: 176 bytes)
496.84883432553846
\end{minted}



这里出现的5个内存分配是由于在全局作用域中运行\texttt{@time}宏导致的。如果我们在函数内运行时间测试,我们将发现事实上并没有发生任何内存分配。




\begin{minted}{jlcon}
julia> time_sum(x) = @time sum_arg(x);

julia> time_sum(x)
  0.000001 seconds
496.84883432553846
\end{minted}



In some situations, your function may need to allocate memory as part of its operation, and this can complicate the simple picture above. In such cases, consider using one of the \hyperlink{11178925956438684264}{tools} below to diagnose problems, or write a version of your function that separates allocation from its algorithmic aspects (see \href{@ref}{Pre-allocating outputs}).



\begin{quote}
\textbf{Note}

For more serious benchmarking, consider the \href{https://github.com/JuliaCI/BenchmarkTools.jl}{BenchmarkTools.jl} package which among other things evaluates the function multiple times in order to reduce noise.

\end{quote}


\hypertarget{14350444000650775715}{}


\section{Tools}



Julia and its package ecosystem includes tools that may help you diagnose problems and improve the performance of your code:



\begin{itemize}
\item \href{@ref}{Profiling} allows you to measure the performance of your running code and identify lines that serve as bottlenecks. For complex projects, the \href{https://github.com/timholy/ProfileView.jl}{ProfileView} package can help you visualize your profiling results.


\item The \href{https://github.com/JunoLab/Traceur.jl}{Traceur} package can help you find common performance problems in your code.


\item Unexpectedly-large memory allocations–as reported by \hyperlink{8029752041511656628}{\texttt{@time}}, \hyperlink{5377755456008435782}{\texttt{@allocated}}, or the profiler (through calls to the garbage-collection routines)–hint that there might be issues with your code. If you don{\textquotesingle}t see another reason for the allocations, suspect a type problem.  You can also start Julia with the \texttt{--track-allocation=user} option and examine the resulting \texttt{*.mem} files to see information about where those allocations occur. See \href{@ref}{Memory allocation analysis}.


\item \texttt{@code\_warntype} generates a representation of your code that can be helpful in finding expressions that result in type uncertainty. See \hyperlink{8092893264277772840}{\texttt{@code\_warntype}} below.

\end{itemize}


\hypertarget{12159424404022697469}{}


\section{Avoid containers with abstract type parameters}



When working with parameterized types, including arrays, it is best to avoid parameterizing with abstract types where possible.



Consider the following:




\begin{minted}{jlcon}
julia> a = Real[]
Real[]

julia> push!(a, 1); push!(a, 2.0); push!(a, π)
3-element Array{Real,1}:
 1
 2.0
 π = 3.1415926535897...
\end{minted}



因为\texttt{a}是一个抽象类型\hyperlink{6175959395021454412}{\texttt{Real}}的数组,它必须能容纳任何一个\texttt{Real}值。因为\texttt{Real}对象可以有任意的大小和结构,\texttt{a}必须用指针的数组来表示,以便能独立地为\texttt{Real}对象进行内存分配。但是如果我们只允许同样类型的数,比如\hyperlink{5027751419500983000}{\texttt{Float64}},才能存在\texttt{a}中,它们就能被更有效率地存储:




\begin{minted}{jlcon}
julia> a = Float64[]
Float64[]

julia> push!(a, 1); push!(a, 2.0); push!(a,  π)
3-element Array{Float64,1}:
 1.0
 2.0
 3.141592653589793
\end{minted}



把数字赋值给\texttt{a}会即时将数字转换成\texttt{Float64}并且\texttt{a}会按照64位浮点数值的连续的块来储存,这就能高效地处理。



也请参见在\hyperlink{5603543911318150609}{参数类型}下的讨论。



\hypertarget{11271598028486730305}{}


\section{类型声明}



在有可选类型声明的语言中,添加声明是使代码运行更快的原则性方法。在Julia中\emph{并不是}这种情况。在Julia中,编译器都知道所有的函数参数,局部变量和表达式的类型。但是,有一些特殊的情况下声明是有帮助的。



\hypertarget{17004910976199348766}{}


\subsection{避免有抽象类型的字段}



类型能在不指定其字段的类型的情况下被声明:




\begin{minted}{jlcon}
julia> struct MyAmbiguousType
           a
       end
\end{minted}



这就允许\texttt{a}可以是任意类型。这经常很有用,但是有个缺点:对于类型\texttt{MyAmbiguousType}的对象,编译器不能够生成高性能的代码。原因是编译器使用对象的类型,而非值,来确定如何构建代码。不幸的是,几乎没有信息可以从类型\texttt{MyAmbiguousType}的对象中推导出来:




\begin{minted}{jlcon}
julia> b = MyAmbiguousType("Hello")
MyAmbiguousType("Hello")

julia> c = MyAmbiguousType(17)
MyAmbiguousType(17)

julia> typeof(b)
MyAmbiguousType

julia> typeof(c)
MyAmbiguousType
\end{minted}



\texttt{b} 和 \texttt{c} 的值具有相同类型,但它们在内存中的数据的底层表示十分不同。即使你只在字段 \texttt{a} 中存储数值,\hyperlink{6609065134969660118}{\texttt{UInt8}} 的内存表示与 \hyperlink{5027751419500983000}{\texttt{Float64}} 也是不同的,这也意味着 CPU 需要使用两种不同的指令来处理它们。因为该类型中不提供所需的信息,所以必须在运行时进行这些判断。而这会降低性能。



通过声明 \texttt{a} 的类型,你能够做得更好。这里我们关注 \texttt{a} 可能是几种类型中任意一种的情况,在这种情况下,自然的一个解决方法是使用参数。例如:




\begin{minted}{jlcon}
julia> mutable struct MyType{T<:AbstractFloat}
           a::T
       end
\end{minted}



比下面这种更好




\begin{minted}{jlcon}
julia> mutable struct MyStillAmbiguousType
           a::AbstractFloat
       end
\end{minted}



因为第一种通过包装对象的类型指定了 \texttt{a} 的类型。 例如:




\begin{minted}{jlcon}
julia> m = MyType(3.2)
MyType{Float64}(3.2)

julia> t = MyStillAmbiguousType(3.2)
MyStillAmbiguousType(3.2)

julia> typeof(m)
MyType{Float64}

julia> typeof(t)
MyStillAmbiguousType
\end{minted}



字段 \texttt{a} 的类型可以很容易地通过 \texttt{m} 的类型确定,而不是通过 \texttt{t} 的类型确定。事实上,在 \texttt{t} 中是可以改变字段 \texttt{a} 的类型的:




\begin{minted}{jlcon}
julia> typeof(t.a)
Float64

julia> t.a = 4.5f0
4.5f0

julia> typeof(t.a)
Float32
\end{minted}



反之,一旦 \texttt{m} 被构建出来,\texttt{m.a} 的类型就不能够更改了。




\begin{minted}{jlcon}
julia> m.a = 4.5f0
4.5f0

julia> typeof(m.a)
Float64
\end{minted}



\texttt{m.a} 的类型是通过 \texttt{m} 的类型得知这一事实加上它的类型不能改变在函数中改变这一事实,这两者使得对于像 \texttt{m} 这样的对象编译器可以生成高度优化后的代码,但是对 \texttt{t} 这样的对象却不可以。 当然,如果我们将 \texttt{m} 构造成一个具体类型,那么这两者都可以。我们可以通过明确地使用一个抽象类型去构建它来破坏这一点:




\begin{minted}{jlcon}
julia> m = MyType{AbstractFloat}(3.2)
MyType{AbstractFloat}(3.2)

julia> typeof(m.a)
Float64

julia> m.a = 4.5f0
4.5f0

julia> typeof(m.a)
Float32
\end{minted}



对于一个实际的目的来说,这样的对象表现起来和那些 \texttt{MyStillAmbiguousType} 的对象一模一样。



比较为一个简单函数生成的代码的绝对数量是十分有指导意义的,




\begin{minted}{julia}
func(m::MyType) = m.a+1
\end{minted}



使用




\begin{minted}{julia}
code_llvm(func, Tuple{MyType{Float64}})
code_llvm(func, Tuple{MyType{AbstractFloat}})
\end{minted}



由于长度的原因,代码的结果没有在这里显示出来,但是你可能会希望自己去验证这一点。因为在第一种情况中,类型被完全指定了,在运行时,编译器不需要生成任何代码来决定类型。这就带来了更短和更快的代码。



\hypertarget{1665323452340421493}{}


\subsection{避免使用带抽象容器的字段}



上面的做法同样也适用于容器的类型:




\begin{minted}{jlcon}
julia> struct MySimpleContainer{A<:AbstractVector}
           a::A
       end

julia> struct MyAmbiguousContainer{T}
           a::AbstractVector{T}
       end
\end{minted}



例如:




\begin{minted}{jlcon}
julia> c = MySimpleContainer(1:3);

julia> typeof(c)
MySimpleContainer{UnitRange{Int64}}

julia> c = MySimpleContainer([1:3;]);

julia> typeof(c)
MySimpleContainer{Array{Int64,1}}

julia> b = MyAmbiguousContainer(1:3);

julia> typeof(b)
MyAmbiguousContainer{Int64}

julia> b = MyAmbiguousContainer([1:3;]);

julia> typeof(b)
MyAmbiguousContainer{Int64}
\end{minted}



对于 \texttt{MySimpleContainer} 来说,它被它的类型和参数完全确定了,因此编译器能够生成优化过的代码。在大多数实例中,这点能够实现。



尽管编译器现在可以将它的工作做得非常好,但是还是有\textbf{你}可能希望你的代码能够能够根据 \texttt{a} 的\textbf{元素类型}做不同的事情的时候。通常达成这个目的最好的方式是将你的具体操作 (here, \texttt{foo}) 打包到一个独立的函数中。




\begin{minted}{jlcon}
julia> function sumfoo(c::MySimpleContainer)
           s = 0
           for x in c.a
               s += foo(x)
           end
           s
       end
sumfoo (generic function with 1 method)

julia> foo(x::Integer) = x
foo (generic function with 1 method)

julia> foo(x::AbstractFloat) = round(x)
foo (generic function with 2 methods)
\end{minted}



这使事情变得简单,同时也允许编译器在所有情况下生成经过优化的代码。



但是,在某些情况下,你可能需要声明外部函数的不同版本,这可能是为了不同的元素类型,也可能是为了 \texttt{MySimpleContainer} 中的字段 \texttt{a} 所具有的不同 \texttt{AbstractVector} 类型。你可以这样做:




\begin{minted}{jlcon}
julia> function myfunc(c::MySimpleContainer{<:AbstractArray{<:Integer}})
           return c.a[1]+1
       end
myfunc (generic function with 1 method)

julia> function myfunc(c::MySimpleContainer{<:AbstractArray{<:AbstractFloat}})
           return c.a[1]+2
       end
myfunc (generic function with 2 methods)

julia> function myfunc(c::MySimpleContainer{Vector{T}}) where T <: Integer
           return c.a[1]+3
       end
myfunc (generic function with 3 methods)
\end{minted}




\begin{minted}{jlcon}
julia> myfunc(MySimpleContainer(1:3))
2

julia> myfunc(MySimpleContainer(1.0:3))
3.0

julia> myfunc(MySimpleContainer([1:3;]))
4
\end{minted}



\hypertarget{18349216047452588444}{}


\subsection{对从无类型位置获取的值进行类型注释}



使用可能包含任何类型的值的数据结构(如类型为 \texttt{Array\{Any\}} 的数组)经常是很方便的。但是,如果你正在使用这些数据结构之一,并且恰巧知道某个元素的类型,那么让编译器也知道这一点会有所帮助:




\begin{minted}{julia}
function foo(a::Array{Any,1})
    x = a[1]::Int32
    b = x+1
    ...
end
\end{minted}



在这里,我们恰巧知道 \texttt{a} 的第一个元素是个 \hyperlink{10103694114785108551}{\texttt{Int32}}。留下这样的注释还有另外的好处,它将在该值不是预期类型时引发运行时错误,而这可能会更早地捕获某些错误。



在没有确切知道 \texttt{a[1]} 的类型的情况下,\texttt{x} 可以通过 \texttt{x = convert(Int32, a[1])::Int32} 来声明。使用 \hyperlink{1846942650946171605}{\texttt{convert}} 函数则允许 \texttt{a[1]} 是可转换为 \texttt{Int32} 的任何对象(比如 \texttt{UInt8}),从而通过放松类型限制来提高代码的通用性。请注意,\texttt{convert} 本身在此上下文中需要类型注释才能实现类型稳定性。这是因为除非该函数所有参数的类型都已知,否则编译器无法推导出该函数返回值的类型,即使其为 \texttt{convert}。



如果类型注释中的类型是在运行时构造的,那么类型注释不会增强(并且实际上可能会降低)性能。这是因为编译器无法使用该类型注释来专门化代码,而类型检查本身又需要时间。例如,在以下代码中:




\begin{minted}{julia}
function nr(a, prec)
    ctype = prec == 32 ? Float32 : Float64
    b = Complex{ctype}(a)
    c = (b + 1.0f0)::Complex{ctype}
    abs(c)
end
\end{minted}



the annotation of \texttt{c} harms performance. To write performant code involving types constructed at run-time, use the \hyperlink{17509985600836810807}{function-barrier technique} discussed below, and ensure that the constructed type appears among the argument types of the kernel function so that the kernel operations are properly specialized by the compiler. For example, in the above snippet, as soon as \texttt{b} is constructed, it can be passed to another function \texttt{k}, the kernel. If, for example, function \texttt{k} declares \texttt{b} as an argument of type \texttt{Complex\{T\}}, where \texttt{T} is a type parameter, then a type annotation appearing in an assignment statement within \texttt{k} of the form:




\begin{minted}{julia}
c = (b + 1.0f0)::Complex{T}
\end{minted}



不会降低性能(但也不会提高),因为编译器可以在编译 \texttt{k} 时确定 \texttt{c} 的类型。



\hypertarget{13987935381679254535}{}


\subsection{Be aware of when Julia avoids specializing}



As a heuristic, Julia avoids automatically specializing on argument type parameters in three specific cases: \texttt{Type}, \texttt{Function}, and \texttt{Vararg}. Julia will always specialize when the argument is used within the method, but not if the argument is just passed through to another function. This usually has no performance impact at runtime and \hyperlink{14494557220692677892}{improves compiler performance}. If you find it does have a performance impact at runtime in your case, you can trigger specialization by adding a type parameter to the method declaration. Here are some examples:



This will not specialize:




\begin{minted}{julia}
function f_type(t)  # or t::Type
    x = ones(t, 10)
    return sum(map(sin, x))
end
\end{minted}



but this will:




\begin{minted}{julia}
function g_type(t::Type{T}) where T
    x = ones(T, 10)
    return sum(map(sin, x))
end
\end{minted}



These will not specialize:




\begin{minted}{julia}
f_func(f, num) = ntuple(f, div(num, 2))
g_func(g::Function, num) = ntuple(g, div(num, 2))
\end{minted}



but this will:




\begin{minted}{julia}
h_func(h::H, num) where {H} = ntuple(h, div(num, 2))
\end{minted}



This will not specialize:




\begin{minted}{julia}
f_vararg(x::Int...) = tuple(x...)
\end{minted}



but this will:




\begin{minted}{julia}
g_vararg(x::Vararg{Int, N}) where {N} = tuple(x...)
\end{minted}



One only needs to introduce a single type parameter to force specialization, even if the other types are unconstrained. For example, this will also specialize, and is useful when the arguments are not all of the same type:




\begin{minted}{julia}
h_vararg(x::Vararg{Any, N}) where {N} = tuple(x...)
\end{minted}



Note that \hyperlink{6823997547688846780}{\texttt{@code\_typed}} and friends will always show you specialized code, even if Julia would not normally specialize that method call. You need to check the \hyperlink{5484310955311811443}{method internals} if you want to see whether specializations are generated when argument types are changed, i.e., if \texttt{(@which f(...)).specializations} contains specializations for the argument in question.



\hypertarget{5154687489987486672}{}


\section{将函数拆分为多个定义}



将一个函数写成许多小的定义能让编译器直接调用最适合的代码,甚至能够直接将它内联。



这是一个真的该被写成许多小的定义的\textbf{复合函数}的例子:




\begin{minted}{julia}
using LinearAlgebra

function mynorm(A)
    if isa(A, Vector)
        return sqrt(real(dot(A,A)))
    elseif isa(A, Matrix)
        return maximum(svdvals(A))
    else
        error("mynorm: invalid argument")
    end
end
\end{minted}



这可以更简洁有效地写成:




\begin{minted}{julia}
norm(x::Vector) = sqrt(real(dot(x, x)))
norm(A::Matrix) = maximum(svdvals(A))
\end{minted}



然而,应该注意的是,编译器会十分高效地优化掉编写得如同 \texttt{mynorm} 例子的代码中的死分支。



\hypertarget{11831339210471490014}{}


\section{编写「类型稳定的」函数}



如果可能,确保函数总是返回相同类型的值是有好处的。考虑以下定义:




\begin{minted}{julia}
pos(x) = x < 0 ? 0 : x
\end{minted}



虽然这看起来挺合法的,但问题是 \texttt{0} 是一个(\texttt{Int} 类型的)整数而 \texttt{x} 可能是任何类型。于是,根据 \texttt{x} 的值,此函数可能返回两种类型中任何一种的值。这种行为是允许的,并且在某些情况下可能是合乎需要的。但它可以很容易地以如下方式修复:




\begin{minted}{julia}
pos(x) = x < 0 ? zero(x) : x
\end{minted}



还有 \hyperlink{2310843180104103470}{\texttt{oneunit}} 函数,以及更通用的 \hyperlink{374166931194490566}{\texttt{oftype(x, y)}} 函数,它返回被转换为 \texttt{x} 的类型的 \texttt{y}。



\hypertarget{6976604449962643011}{}


\section{避免更改变量类型}



类似的「类型稳定性」问题存在于在函数内重复使用的变量:




\begin{minted}{julia}
function foo()
    x = 1
    for i = 1:10
        x /= rand()
    end
    return x
end
\end{minted}



局部变量 \texttt{x} 一开始是整数,在一次循环迭代后变为浮点数(\hyperlink{4103478871488785445}{\texttt{/}} 运算符的结果)。这使得编译器更难优化循环体。有几种可能的解决方法:



\begin{itemize}
\item 使用 \texttt{x = 1.0} 初始化 \texttt{x}


\item Declare the type of \texttt{x} explicitly as \texttt{x::Float64 = 1}


\item Use an explicit conversion by \texttt{x = oneunit(Float64)}


\item 使用第一个循环迭代初始化,即 \texttt{x = 1 / rand()},接着循环 \texttt{for i = 2:10}

\end{itemize}


\hypertarget{16258602729394793584}{}


\section{Separate kernel functions (aka, function barriers)}



许多函数遵循这一模式:先执行一些设置工作,再通过多次迭代来执行核心计算。如果可行,将这些核心计算放在单独的函数中是个好主意。例如,以下做作的函数返回一个数组,其类型是随机选择的。




\begin{minted}{jlcon}
julia> function strange_twos(n)
           a = Vector{rand(Bool) ? Int64 : Float64}(undef, n)
           for i = 1:n
               a[i] = 2
           end
           return a
       end;

julia> strange_twos(3)
3-element Array{Float64,1}:
 2.0
 2.0
 2.0
\end{minted}



这应该写作:




\begin{minted}{jlcon}
julia> function fill_twos!(a)
           for i = eachindex(a)
               a[i] = 2
           end
       end;

julia> function strange_twos(n)
           a = Vector{rand(Bool) ? Int64 : Float64}(undef, n)
           fill_twos!(a)
           return a
       end;

julia> strange_twos(3)
3-element Array{Float64,1}:
 2.0
 2.0
 2.0
\end{minted}



Julia 的编译器会在函数边界处针对参数类型特化代码,因此在原始的实现中循环期间无法得知 \texttt{a} 的类型(因为它是随即选择的)。于是,第二个版本通常更快,因为对于不同类型的 \texttt{a},内层循环都可被重新编译为 \texttt{fill\_twos!} 的一部分。



第二种形式通常是更好的风格,并且可以带来更多的代码的重复利用。



这个模式在 Julia Base 的几个地方中有使用。相关的例子,请参阅 \href{https://github.com/JuliaLang/julia/blob/40fe264f4ffaa29b749bcf42239a89abdcbba846/base/abstractarray.jl\#L1205-L1206}{\texttt{abstractarray.jl}} 中的 \texttt{vcat} 和 \texttt{hcat},或者 \hyperlink{5162290739791026948}{\texttt{fill!}} 函数,我们可使用该函数而不是编写自己的 \texttt{fill\_twos!}。



诸如 \texttt{strange\_twos} 的函数会在处理具有不确定类型的数据时出现,例如从可能包含整数、浮点数、字符串或其它内容的输入文件中加载的数据。



\hypertarget{13298146121388666827}{}


\section{Types with values-as-parameters}



比方说你想创建一个每个维度大小都是3的 \texttt{N} 维数组。这种数组可以这样创建:




\begin{minted}{jlcon}
julia> A = fill(5.0, (3, 3))
3×3 Array{Float64,2}:
 5.0  5.0  5.0
 5.0  5.0  5.0
 5.0  5.0  5.0
\end{minted}



这个方法工作得很好:编译器可以识别出来 \texttt{A} 是一个 \texttt{Array\{Float64,2\}} 因为它知道填充值 (\texttt{5.0::Float64}) 的类型和维度 (\texttt{(3, 3)::NTuple\{2,Int\}}).



但是现在打比方说你想写一个函数,在任何一个维度下,它都创建一个 3×3×... 的数组;你可能会心动地写下一个函数




\begin{minted}{jlcon}
julia> function array3(fillval, N)
           fill(fillval, ntuple(d->3, N))
       end
array3 (generic function with 1 method)

julia> array3(5.0, 2)
3×3 Array{Float64,2}:
 5.0  5.0  5.0
 5.0  5.0  5.0
 5.0  5.0  5.0
\end{minted}



这确实有用,但是(你可以自己使用 \texttt{@code\_warntype array3(5.0, 2)} 来验证)问题是输出地类型不能被推断出来:参数 \texttt{N} 是一个 \texttt{Int} 类型的\textbf{值},而且类型推断不会(也不能)提前预测它的值。这意味着使用这个函数的结果的代码在每次获取 \texttt{A} 时都不得不保守地检查其类型;这样的代码将会是非常缓慢的。



Now, one very good way to solve such problems is by using the \hyperlink{17509985600836810807}{function-barrier technique}. However, in some cases you might want to eliminate the type-instability altogether. In such cases, one approach is to pass the dimensionality as a parameter, for example through \texttt{Val\{T\}()} (see \href{@ref}{{\textquotedbl}Value types{\textquotedbl}}):




\begin{minted}{jlcon}
julia> function array3(fillval, ::Val{N}) where N
           fill(fillval, ntuple(d->3, Val(N)))
       end
array3 (generic function with 1 method)

julia> array3(5.0, Val(2))
3×3 Array{Float64,2}:
 5.0  5.0  5.0
 5.0  5.0  5.0
 5.0  5.0  5.0
\end{minted}



Julia has a specialized version of \texttt{ntuple} that accepts a \texttt{Val\{::Int\}} instance as the second parameter; by passing \texttt{N} as a type-parameter, you make its {\textquotedbl}value{\textquotedbl} known to the compiler. Consequently, this version of \texttt{array3} allows the compiler to predict the return type.



However, making use of such techniques can be surprisingly subtle. For example, it would be of no help if you called \texttt{array3} from a function like this:




\begin{minted}{julia}
function call_array3(fillval, n)
    A = array3(fillval, Val(n))
end
\end{minted}



Here, you{\textquotesingle}ve created the same problem all over again: the compiler can{\textquotesingle}t guess what \texttt{n} is, so it doesn{\textquotesingle}t know the \emph{type} of \texttt{Val(n)}. Attempting to use \texttt{Val}, but doing so incorrectly, can easily make performance \emph{worse} in many situations. (Only in situations where you{\textquotesingle}re effectively combining \texttt{Val} with the function-barrier trick, to make the kernel function more efficient, should code like the above be used.)



一个正确使用 \texttt{Val} 的例子是这样的:




\begin{minted}{julia}
function filter3(A::AbstractArray{T,N}) where {T,N}
    kernel = array3(1, Val(N))
    filter(A, kernel)
end
\end{minted}



In this example, \texttt{N} is passed as a parameter, so its {\textquotedbl}value{\textquotedbl} is known to the compiler. Essentially, \texttt{Val(T)} works only when \texttt{T} is either hard-coded/literal (\texttt{Val(3)}) or already specified in the type-domain.



\hypertarget{5410118045861360063}{}


\section{The dangers of abusing multiple dispatch (aka, more on types with values-as-parameters)}



Once one learns to appreciate multiple dispatch, there{\textquotesingle}s an understandable tendency to go overboard and try to use it for everything. For example, you might imagine using it to store information, e.g.




\begin{lstlisting}
struct Car{Make, Model}
    year::Int
    ...more fields...
end
\end{lstlisting}



and then dispatch on objects like \texttt{Car\{:Honda,:Accord\}(year, args...)}.



This might be worthwhile when either of the following are true:



\begin{itemize}
\item You require CPU-intensive processing on each \texttt{Car}, and it becomes vastly more efficient if you know the \texttt{Make} and \texttt{Model} at compile time and the total number of different \texttt{Make} or \texttt{Model} that will be used is not too large.


\item You have homogenous lists of the same type of \texttt{Car} to process, so that you can store them all in an \texttt{Array\{Car\{:Honda,:Accord\},N\}}.

\end{itemize}


When the latter holds, a function processing such a homogenous array can be productively specialized: Julia knows the type of each element in advance (all objects in the container have the same concrete type), so Julia can {\textquotedbl}look up{\textquotedbl} the correct method calls when the function is being compiled (obviating the need to check at run-time) and thereby emit efficient code for processing the whole list.



When these do not hold, then it{\textquotesingle}s likely that you{\textquotesingle}ll get no benefit; worse, the resulting {\textquotedbl}combinatorial explosion of types{\textquotedbl} will be counterproductive. If \texttt{items[i+1]} has a different type than \texttt{item[i]}, Julia has to look up the type at run-time, search for the appropriate method in method tables, decide (via type intersection) which one matches, determine whether it has been JIT-compiled yet (and do so if not), and then make the call. In essence, you{\textquotesingle}re asking the full type- system and JIT-compilation machinery to basically execute the equivalent of a switch statement or dictionary lookup in your own code.



Some run-time benchmarks comparing (1) type dispatch, (2) dictionary lookup, and (3) a {\textquotedbl}switch{\textquotedbl} statement can be found \href{https://groups.google.com/forum/\#!msg/julia-users/jUMu9A3QKQQ/qjgVWr7vAwAJ}{on the mailing list}.



Perhaps even worse than the run-time impact is the compile-time impact: Julia will compile specialized functions for each different \texttt{Car\{Make, Model\}}; if you have hundreds or thousands of such types, then every function that accepts such an object as a parameter (from a custom \texttt{get\_year} function you might write yourself, to the generic \texttt{push!} function in Julia Base) will have hundreds or thousands of variants compiled for it. Each of these increases the size of the cache of compiled code, the length of internal lists of methods, etc. Excess enthusiasm for values-as-parameters can easily waste enormous resources.



\hypertarget{16800011477786249644}{}


\section{Access arrays in memory order, along columns}



Julia 中的多维数组以列主序存储。这意味着数组一次堆叠一列。这可使用 \texttt{vec} 函数或语法 \texttt{[:]} 来验证,如下所示(请注意,数组的顺序是 \texttt{[1 3 2 4]},而不是 \texttt{[1 2 3 4]}):




\begin{minted}{jlcon}
julia> x = [1 2; 3 4]
2×2 Array{Int64,2}:
 1  2
 3  4

julia> x[:]
4-element Array{Int64,1}:
 1
 3
 2
 4
\end{minted}



This convention for ordering arrays is common in many languages like Fortran, Matlab, and R (to name a few). The alternative to column-major ordering is row-major ordering, which is the convention adopted by C and Python (\texttt{numpy}) among other languages. Remembering the ordering of arrays can have significant performance effects when looping over arrays. A rule of thumb to keep in mind is that with column-major arrays, the first index changes most rapidly. Essentially this means that looping will be faster if the inner-most loop index is the first to appear in a slice expression. Keep in mind that indexing an array with \texttt{:} is an implicit loop that iteratively accesses all elements within a particular dimension; it can be faster to extract columns than rows, for example.



考虑以下人为示例。假设我们想编写一个接收 \hyperlink{10571362059486397014}{\texttt{Vector}} 并返回方阵 \hyperlink{5448927444601277512}{\texttt{Matrix}} 的函数,所返回方阵的行或列都用输入向量的副本填充。并假设用这些副本填充的是行还是列并不重要(也许可以很容易地相应调整剩余代码)。我们至少可以想到四种方式(除了建议的调用内置函数 \hyperlink{15426606278434194584}{\texttt{repeat}}):




\begin{minted}{julia}
function copy_cols(x::Vector{T}) where T
    inds = axes(x, 1)
    out = similar(Array{T}, inds, inds)
    for i = inds
        out[:, i] = x
    end
    return out
end

function copy_rows(x::Vector{T}) where T
    inds = axes(x, 1)
    out = similar(Array{T}, inds, inds)
    for i = inds
        out[i, :] = x
    end
    return out
end

function copy_col_row(x::Vector{T}) where T
    inds = axes(x, 1)
    out = similar(Array{T}, inds, inds)
    for col = inds, row = inds
        out[row, col] = x[row]
    end
    return out
end

function copy_row_col(x::Vector{T}) where T
    inds = axes(x, 1)
    out = similar(Array{T}, inds, inds)
    for row = inds, col = inds
        out[row, col] = x[col]
    end
    return out
end
\end{minted}



现在,我们使用相同的 \texttt{10000} 乘 \texttt{1} 的随机输入向量来对这些函数计时。




\begin{minted}{jlcon}
julia> x = randn(10000);

julia> fmt(f) = println(rpad(string(f)*": ", 14, ' '), @elapsed f(x))

julia> map(fmt, [copy_cols, copy_rows, copy_col_row, copy_row_col]);
copy_cols:    0.331706323
copy_rows:    1.799009911
copy_col_row: 0.415630047
copy_row_col: 1.721531501
\end{minted}



请注意,\texttt{copy\_cols} 比 \texttt{copy\_rows} 快得多。这与预料的一致,因为 \texttt{copy\_cols} 尊重 \texttt{Matrix} 基于列的内存布局。另外,\texttt{copy\_col\_row} 比 \texttt{copy\_row\_col} 快得多,因为它遵循我们的经验法则,即切片表达式中出现的第一个元素应该与最内层循环耦合。



\hypertarget{2586641269833759347}{}


\section{输出预分配}



如果函数返回 \texttt{Array} 或其它复杂类型,则可能需要分配内存。不幸的是,内存分配及其反面垃圾收集通常是很大的瓶颈。



有时,你可以通过预分配输出结果来避免在每个函数调用上分配内存的需要。作为一个简单的例子,比较




\begin{minted}{jlcon}
julia> function xinc(x)
           return [x, x+1, x+2]
       end;

julia> function loopinc()
           y = 0
           for i = 1:10^7
               ret = xinc(i)
               y += ret[2]
           end
           return y
       end;
\end{minted}



和




\begin{minted}{jlcon}
julia> function xinc!(ret::AbstractVector{T}, x::T) where T
           ret[1] = x
           ret[2] = x+1
           ret[3] = x+2
           nothing
       end;

julia> function loopinc_prealloc()
           ret = Vector{Int}(undef, 3)
           y = 0
           for i = 1:10^7
               xinc!(ret, i)
               y += ret[2]
           end
           return y
       end;
\end{minted}



计时结果:




\begin{minted}{jlcon}
julia> @time loopinc()
  0.529894 seconds (40.00 M allocations: 1.490 GiB, 12.14% gc time)
50000015000000

julia> @time loopinc_prealloc()
  0.030850 seconds (6 allocations: 288 bytes)
50000015000000
\end{minted}



预分配还有其它优点,例如允许调用者在算法中控制「输出」类型。在上述例子中,我们如果需要,可以传递 \texttt{SubArray} 而不是 \hyperlink{15492651498431872487}{\texttt{Array}}。



极端情况下,预分配可能会使你的代码更丑陋,所以可能需要做性能测试和一些判断。但是,对于「向量化」(逐元素)函数,方便的语法 \texttt{x .= f.(y)} 可用于具有融合循环的 in-place 操作且无需临时数组(请参阅\hyperlink{17801130558550430478}{向量化函数的点语法})。



\hypertarget{12965993905966303435}{}


\section{点语法:融合向量化操作}



Julia 有特殊的\hyperlink{17801130558550430478}{点语法},它可以将任何标量函数转换为「向量化」函数调用,将任何运算符转换为「向量化」运算符,其具有的特殊性质是嵌套「点调用」是\emph{融合的}:它们在语法层级被组合为单个循环,无需分配临时数组。如果你使用 \texttt{.=} 和类似的赋值运算符,则结果也可以 in-place 存储在预分配的数组(参见上文)。



在线性代数的上下文中,这意味着即使诸如 \texttt{vector + vector} 和 \texttt{vector * scalar} 之类的运算,使用 \texttt{vector .+ vector} 和 \texttt{vector .* scalar} 来替代也可能是有利的,因为生成的循环可与周围的计算融合。例如,考虑两个函数:




\begin{minted}{jlcon}
julia> f(x) = 3x.^2 + 4x + 7x.^3;

julia> fdot(x) = @. 3x^2 + 4x + 7x^3 # equivalent to 3 .* x.^2 .+ 4 .* x .+ 7 .* x.^3;
\end{minted}



\texttt{f} 和 \texttt{fdot} 都做相同的计算。但是,\texttt{fdot}(在 \hyperlink{16688502228717894452}{\texttt{@.}} 宏的帮助下定义)在作用于数组时明显更快:




\begin{minted}{jlcon}
julia> x = rand(10^6);

julia> @time f(x);
  0.019049 seconds (16 allocations: 45.777 MiB, 18.59% gc time)

julia> @time fdot(x);
  0.002790 seconds (6 allocations: 7.630 MiB)

julia> @time f.(x);
  0.002626 seconds (8 allocations: 7.630 MiB)
\end{minted}



That is, \texttt{fdot(x)} is ten times faster and allocates 1/6 the memory of \texttt{f(x)}, because each \texttt{*} and \texttt{+} operation in \texttt{f(x)} allocates a new temporary array and executes in a separate loop. (Of course, if you just do \texttt{f.(x)} then it is as fast as \texttt{fdot(x)} in this example, but in many contexts it is more convenient to just sprinkle some dots in your expressions rather than defining a separate function for each vectorized operation.)



\hypertarget{16479954806149442392}{}


\section{Consider using views for slices}



In Julia, an array {\textquotedbl}slice{\textquotedbl} expression like \texttt{array[1:5, :]} creates a copy of that data (except on the left-hand side of an assignment, where \texttt{array[1:5, :] = ...} assigns in-place to that portion of \texttt{array}). If you are doing many operations on the slice, this can be good for performance because it is more efficient to work with a smaller contiguous copy than it would be to index into the original array. On the other hand, if you are just doing a few simple operations on the slice, the cost of the allocation and copy operations can be substantial.



An alternative is to create a {\textquotedbl}view{\textquotedbl} of the array, which is an array object (a \texttt{SubArray}) that actually references the data of the original array in-place, without making a copy. (If you write to a view, it modifies the original array{\textquotesingle}s data as well.) This can be done for individual slices by calling \hyperlink{4861450464669906845}{\texttt{view}}, or more simply for a whole expression or block of code by putting \hyperlink{4544474300423667148}{\texttt{@views}} in front of that expression. For example:




\begin{minted}{jlcon}
julia> fcopy(x) = sum(x[2:end-1]);

julia> @views fview(x) = sum(x[2:end-1]);

julia> x = rand(10^6);

julia> @time fcopy(x);
  0.003051 seconds (7 allocations: 7.630 MB)

julia> @time fview(x);
  0.001020 seconds (6 allocations: 224 bytes)
\end{minted}



请注意,该函数的 \texttt{fview} 版本提速了 3 倍且减少了内存分配。



\hypertarget{12328390677917528950}{}


\section{复制数据不总是坏的}



数组被连续地存储在内存中,这使其可被 CPU 向量化,并且会由于缓存减少内存访问。这与建议以列序优先方式访问数组的原因相同(请参见上文)。由于不按顺序访问内存,无规律的访问方式和不连续的视图可能会大大减慢数组上的计算速度。



在对无规律访问的数据进行操作前,将其复制到连续的数组中可能带来巨大的加速,正如下例所示。其中,矩阵和向量在相乘前会访问其 800,000 个已被随机混洗的索引处的值。将视图复制到普通数组会加速乘法,即使考虑了复制操作的成本。




\begin{minted}{jlcon}
julia> using Random

julia> x = randn(1_000_000);

julia> inds = shuffle(1:1_000_000)[1:800000];

julia> A = randn(50, 1_000_000);

julia> xtmp = zeros(800_000);

julia> Atmp = zeros(50, 800_000);

julia> @time sum(view(A, :, inds) * view(x, inds))
  0.412156 seconds (14 allocations: 960 bytes)
-4256.759568345458

julia> @time begin
           copyto!(xtmp, view(x, inds))
           copyto!(Atmp, view(A, :, inds))
           sum(Atmp * xtmp)
       end
  0.285923 seconds (14 allocations: 960 bytes)
-4256.759568345134
\end{minted}



倘若副本本身的内存足够大,那么将视图复制到数组的成本可能远远超过在连续数组上执行矩阵乘法所带来的加速。



\hypertarget{5580481875828136728}{}


\section{避免 I/0 中的字符串插值}



将数据写入到文件(或其他 I/0 设备)中时,生成额外的中间字符串会带来开销。请不要写成这样:




\begin{minted}{julia}
println(file, "$a $b")
\end{minted}



请写成这样:




\begin{minted}{julia}
println(file, a, " ", b)
\end{minted}



第一个版本的代码生成一个字符串,然后将其写入到文件中,而第二个版本直接将值写入到文件中。另请注意,在某些情况下,字符串插值可能更难阅读。请考虑:




\begin{minted}{julia}
println(file, "$(f(a))$(f(b))")
\end{minted}



与:




\begin{minted}{julia}
println(file, f(a), f(b))
\end{minted}



\hypertarget{612858988935113727}{}


\section{并发执行时优化网络 I/O}



当并发地执行一个远程函数时:




\begin{minted}{julia}
using Distributed

responses = Vector{Any}(undef, nworkers())
@sync begin
    for (idx, pid) in enumerate(workers())
        @async responses[idx] = remotecall_fetch(foo, pid, args...)
    end
end
\end{minted}



会快于:




\begin{minted}{julia}
using Distributed

refs = Vector{Any}(undef, nworkers())
for (idx, pid) in enumerate(workers())
    refs[idx] = @spawnat pid foo(args...)
end
responses = [fetch(r) for r in refs]
\end{minted}



第一种方式导致每个worker一次网络往返,而第二种方式是两次网络调用:一次 \hyperlink{11231712027010946923}{\texttt{@spawnat}} 一次\hyperlink{11007884648860062495}{\texttt{fetch}} (甚至是 \hyperlink{13761789780433862250}{\texttt{wait}})。 \hyperlink{11007884648860062495}{\texttt{fetch}} 和\hyperlink{13761789780433862250}{\texttt{wait}} 都是同步执行,会导致较差的性能。



\hypertarget{1574945516556538959}{}


\section{修复过期警告}



过期的函数在内部会执行查找,以便仅打印一次相关警告。 这种额外查找可能会显著影响性能,因此应根据警告建议修复掉过期函数的所有使用。



\hypertarget{3670078412083548982}{}


\section{小技巧}



有一些小的注意事项可能会帮助改善循环性能。



\begin{itemize}
\item 避免使用不必要的数组。比如,使用 \texttt{x+y+z} 而不是 \hyperlink{8666686648688281595}{\texttt{sum([x,y,z])}}。


\item 对于复数 \texttt{z},使用 \hyperlink{15686257922156163743}{\texttt{abs2(z)}} 而不是 \hyperlink{462277561264792021}{\texttt{abs(z){\textasciicircum}2}}。一般的, 对于复数参数,用 \hyperlink{15686257922156163743}{\texttt{abs2}} 代替\hyperlink{9614495866226399167}{\texttt{abs}}。


\item 对于直接截断的整除,使用 \hyperlink{8020976424566491334}{\texttt{div(x,y)}} 而不是 \hyperlink{1728363361565303194}{\texttt{trunc(x/y)}},使用\hyperlink{15067916827074788527}{\texttt{fld(x,y)}} 而不是 \hyperlink{11115257331910840693}{\texttt{floor(x/y)}},使用 \hyperlink{7922388465305816555}{\texttt{cld(x,y)}} 而不是 \hyperlink{10519509038312853061}{\texttt{ceil(x/y)}}。

\end{itemize}


\hypertarget{3688250407374776658}{}


\section{性能标注}



有时,你可以通过承诺某些程序性质来启用更好的优化。



\begin{itemize}
\item 使用 \hyperlink{9619263577270933060}{\texttt{@inbounds}} 来取消表达式中的数组边界检查。使用前请再三确定,如果下标越界,可能会发生崩溃或潜在的故障。


\item 使用 \hyperlink{8577961464469068114}{\texttt{@fastmath}} 来允许对于实数是正确的、但是对于 IEEE 数字会导致差异的浮点数优化。使用时请多多小心,因为这可能改变数值结果。这对应于 clang 的 \texttt{-ffast-math} 选项。


\item 在 \texttt{for} 循环前编写 \hyperlink{8155428559748374852}{\texttt{@simd}} 来承诺迭代是相互独立且可以重新排序的。请注意,在许多情况下,Julia 可以在没有 \texttt{@simd} 宏的情况下自动向量化代码;只有在这种转换原本是非法的情况下才有用,包括允许浮点数重新结合和忽略相互依赖的内存访问(\texttt{@simd ivdep})等情况。此外,在断言 \texttt{@simd} 时要十分小心,因为错误地标注一个具有相互依赖的迭代的循环可能导致意外结果。尤其要注意的是,某些 \texttt{AbstractArray} 子类型的 \texttt{setindex!} 本质上依赖于迭代顺序。\textbf{此功能是实验性的},在 Julia 未来的版本中可能会更改或消失。

\end{itemize}


The common idiom of using 1:n to index into an AbstractArray is not safe if the Array uses unconventional indexing, and may cause a segmentation fault if bounds checking is turned off. Use \texttt{LinearIndices(x)} or \texttt{eachindex(x)} instead (see also \hyperlink{1238988360302116626}{Arrays with custom indices}).



\begin{quote}
\textbf{Note}

虽然 \texttt{@simd} 需要直接放在最内层 \texttt{for} 循环前面,但 \texttt{@inbounds} 和 \texttt{@fastmath} 都可作用于单个表达式或在嵌套代码块中出现的所有表达式,例如,可使用 \texttt{@inbounds begin} 或 \texttt{@inbounds for ...}。

\end{quote}


下面是一个具有 \texttt{@inbounds} 和 \texttt{@simd} 标记的例子(我们这里使用 \texttt{@noinline} 来防止因优化器过于智能而破坏我们的基准测试):




\begin{minted}{julia}
@noinline function inner(x, y)
    s = zero(eltype(x))
    for i=eachindex(x)
        @inbounds s += x[i]*y[i]
    end
    return s
end

@noinline function innersimd(x, y)
    s = zero(eltype(x))
    @simd for i = eachindex(x)
        @inbounds s += x[i] * y[i]
    end
    return s
end

function timeit(n, reps)
    x = rand(Float32, n)
    y = rand(Float32, n)
    s = zero(Float64)
    time = @elapsed for j in 1:reps
        s += inner(x, y)
    end
    println("GFlop/sec        = ", 2n*reps / time*1E-9)
    time = @elapsed for j in 1:reps
        s += innersimd(x, y)
    end
    println("GFlop/sec (SIMD) = ", 2n*reps / time*1E-9)
end

timeit(1000, 1000)
\end{minted}



在配备 2.4GHz Intel Core i5 处理器的计算机上,其结果为:




\begin{lstlisting}
GFlop/sec        = 1.9467069505224963
GFlop/sec (SIMD) = 17.578554163920018
\end{lstlisting}



(\texttt{GFlop/sec} 用来测试性能,数值越大越好。)



下面是一个具有三种标记的例子。此程序首先计算一个一维数组的有限差分,然后计算结果的 L2 范数:




\begin{minted}{julia}
function init!(u::Vector)
    n = length(u)
    dx = 1.0 / (n-1)
    @fastmath @inbounds @simd for i in 1:n # 通过断言 `u` 是一个 `Vector`,我们可以假定它具有 1-based 索引
        u[i] = sin(2pi*dx*i)
    end
end

function deriv!(u::Vector, du)
    n = length(u)
    dx = 1.0 / (n-1)
    @fastmath @inbounds du[1] = (u[2] - u[1]) / dx
    @fastmath @inbounds @simd for i in 2:n-1
        du[i] = (u[i+1] - u[i-1]) / (2*dx)
    end
    @fastmath @inbounds du[n] = (u[n] - u[n-1]) / dx
end

function mynorm(u::Vector)
    n = length(u)
    T = eltype(u)
    s = zero(T)
    @fastmath @inbounds @simd for i in 1:n
        s += u[i]^2
    end
    @fastmath @inbounds return sqrt(s)
end

function main()
    n = 2000
    u = Vector{Float64}(undef, n)
    init!(u)
    du = similar(u)

    deriv!(u, du)
    nu = mynorm(du)

    @time for i in 1:10^6
        deriv!(u, du)
        nu = mynorm(du)
    end

    println(nu)
end

main()
\end{minted}



在配备 2.7 GHz Intel Core i7 处理器的计算机上,其结果为:




\begin{lstlisting}
$ julia wave.jl;
  1.207814709 seconds
4.443986180758249

$ julia --math-mode=ieee wave.jl;
  4.487083643 seconds
4.443986180758249
\end{lstlisting}



在这里,选项 \texttt{--math-mode=ieee} 禁用 \texttt{@fastmath} 宏,好让我们可以比较结果。



在这种情况下,\texttt{@fastmath} 加速了大约 3.7 倍。这非常大——通常来说,加速会更小。(在这个特定的例子中,基准测试的工作集足够小,可以放在该处理器的 L1 缓存中,因此内存访问延迟不起作用,计算时间主要由 CPU 使用率决定。在许多现实世界的程序中,情况并非如此。)此外,在这种情况下,此优化不会改变计算结果——通常来说,结果会略有不同。在某些情况下,尤其是数值不稳定的算法,计算结果可能会差很多。



标注 \texttt{@fastmath} 会重新排列浮点数表达式,例如更改求值顺序,或者假设某些特殊情况(如 inf、nan)不出现。在这种情况中(以及在这个特定的计算机上),主要区别是函数 \texttt{deriv} 中的表达式 \texttt{1 / (2*dx)} 会被提升出循环(即在循环外计算),就像编写了 \texttt{idx = 1 / (2*dx)},然后,在循环中,表达式 \texttt{... / (2*dx)} 变为 \texttt{... * idx},后者计算起来快得多。当然,编译器实际上采用的优化以及由此产生的加速都在很大程度上取决于硬件。你可以使用 Julia 的 \hyperlink{2534314152947301270}{\texttt{code\_native}} 函数来检查所生成代码的更改。



请注意,\texttt{@fastmath} 也假设了在计算中不会出现 \texttt{NaN},这可能导致意想不到的行为:




\begin{minted}{jlcon}
julia> f(x) = isnan(x);

julia> f(NaN)
true

julia> f_fast(x) = @fastmath isnan(x);

julia> f_fast(NaN)
false
\end{minted}



\hypertarget{12370344689233780157}{}


\section{Treat Subnormal Numbers as Zeros}



Subnormal numbers, formerly called \href{https://en.wikipedia.org/wiki/Denormal\_number}{denormal numbers}, are useful in many contexts, but incur a performance penalty on some hardware. A call \hyperlink{2845950135157372113}{\texttt{set\_zero\_subnormals(true)}} grants permission for floating-point operations to treat subnormal inputs or outputs as zeros, which may improve performance on some hardware. A call \hyperlink{2845950135157372113}{\texttt{set\_zero\_subnormals(false)}} enforces strict IEEE behavior for subnormal numbers.



Below is an example where subnormals noticeably impact performance on some hardware:




\begin{minted}{julia}
function timestep(b::Vector{T}, a::Vector{T}, Δt::T) where T
    @assert length(a)==length(b)
    n = length(b)
    b[1] = 1                            # Boundary condition
    for i=2:n-1
        b[i] = a[i] + (a[i-1] - T(2)*a[i] + a[i+1]) * Δt
    end
    b[n] = 0                            # Boundary condition
end

function heatflow(a::Vector{T}, nstep::Integer) where T
    b = similar(a)
    for t=1:div(nstep,2)                # Assume nstep is even
        timestep(b,a,T(0.1))
        timestep(a,b,T(0.1))
    end
end

heatflow(zeros(Float32,10),2)           # Force compilation
for trial=1:6
    a = zeros(Float32,1000)
    set_zero_subnormals(iseven(trial))  # Odd trials use strict IEEE arithmetic
    @time heatflow(a,1000)
end
\end{minted}



This gives an output similar to




\begin{lstlisting}
  0.002202 seconds (1 allocation: 4.063 KiB)
  0.001502 seconds (1 allocation: 4.063 KiB)
  0.002139 seconds (1 allocation: 4.063 KiB)
  0.001454 seconds (1 allocation: 4.063 KiB)
  0.002115 seconds (1 allocation: 4.063 KiB)
  0.001455 seconds (1 allocation: 4.063 KiB)
\end{lstlisting}



Note how each even iteration is significantly faster.



This example generates many subnormal numbers because the values in \texttt{a} become an exponentially decreasing curve, which slowly flattens out over time.



Treating subnormals as zeros should be used with caution, because doing so breaks some identities, such as \texttt{x-y == 0} implies \texttt{x == y}:




\begin{minted}{jlcon}
julia> x = 3f-38; y = 2f-38;

julia> set_zero_subnormals(true); (x - y, x == y)
(0.0f0, false)

julia> set_zero_subnormals(false); (x - y, x == y)
(1.0000001f-38, false)
\end{minted}



In some applications, an alternative to zeroing subnormal numbers is to inject a tiny bit of noise.  For example, instead of initializing \texttt{a} with zeros, initialize it with:




\begin{minted}{julia}
a = rand(Float32,1000) * 1.f-9
\end{minted}



\hypertarget{7082991166860772411}{}


\section{\texttt{@code\_warntype}}



宏 \hyperlink{8092893264277772840}{\texttt{@code\_warntype}}(或其函数变体 \hyperlink{5565852192659724503}{\texttt{code\_warntype}})有时可以帮助诊断类型相关的问题。这是一个例子:




\begin{minted}{jlcon}
julia> @noinline pos(x) = x < 0 ? 0 : x;

julia> function f(x)
           y = pos(x)
           return sin(y*x + 1)
       end;

julia> @code_warntype f(3.2)
Variables
  #self#::Core.Compiler.Const(f, false)
  x::Float64
  y::UNION{FLOAT64, INT64}

Body::Float64
1 ─      (y = Main.pos(x))
│   %2 = (y * x)::Float64
│   %3 = (%2 + 1)::Float64
│   %4 = Main.sin(%3)::Float64
└──      return %4
\end{minted}



Interpreting the output of \hyperlink{8092893264277772840}{\texttt{@code\_warntype}}, like that of its cousins \hyperlink{1376948972689074219}{\texttt{@code\_lowered}}, \hyperlink{6823997547688846780}{\texttt{@code\_typed}}, \hyperlink{18039596607712979441}{\texttt{@code\_llvm}}, and \hyperlink{2629340111434042067}{\texttt{@code\_native}}, takes a little practice. Your code is being presented in form that has been heavily digested on its way to generating compiled machine code. Most of the expressions are annotated by a type, indicated by the \texttt{::T} (where \texttt{T} might be \hyperlink{5027751419500983000}{\texttt{Float64}}, for example). The most important characteristic of \hyperlink{8092893264277772840}{\texttt{@code\_warntype}} is that non-concrete types are displayed in red; since this document is written in Markdown, which has no color, in this document, red text is denoted by uppercase.



在顶部,该函数类型推导后的返回类型显示为 \texttt{Body::Float64}。下一行以 Julia 的 SSA IR 形式表示了 \texttt{f} 的主体。被数字标记的方块表示代码中(通过 \texttt{goto})跳转的目标。查看主体,你会看到首先调用了 \texttt{pos},其返回值经类型推导为 \texttt{Union} 类型 \texttt{UNION\{FLOAT64, INT64\}} 并以大写字母显示,因为它是非具体类型。这意味着我们无法根据输入类型知道 \texttt{pos} 的确切返回类型。但是,无论 \texttt{y} 是 \texttt{Float64} 还是 \texttt{Int64},\texttt{y*x} 的结果都是 \texttt{Float64}。最终的结果是 \texttt{f(x::Float64)} 在其输出中不会是类型不稳定的,即使有些中间计算是类型不稳定的。



如何使用这些信息取决于你。显然,最好将 \texttt{pos} 修改为类型稳定的:如果这样做,\texttt{f} 中的所有变量都是具体的,其性能将是最佳的。但是,在某些情况下,这种\emph{短暂的}类型不稳定性可能无关紧要:例如,如果 \texttt{pos} 从不单独使用,那么 \texttt{f} 的输出(对于 \hyperlink{5027751419500983000}{\texttt{Float64}} 输入)是类型稳定的这一事实将保护之后的代码免受类型不稳定性的传播影响。这与类型不稳定性难以或不可能修复的情况密切相关。在这些情况下,上面的建议(例如,添加类型注释并/或分解函数)是你控制类型不稳定性的「损害」的最佳工具。另请注意,即使是 Julia Base 也有类型不稳定的函数。例如,函数 \hyperlink{13752961745140943082}{\texttt{findfirst}} 如果找到键则返回数组索引,如果没有找到键则返回 \texttt{nothing},这是明显的类型不稳定性。为了更易于找到可能很重要的类型不稳定性,包含 \texttt{missing} 或 \texttt{nothing} 的 \texttt{Union} 会用黄色着重显示,而不是用红色。



以下示例可以帮助你解释被标记为包含非叶类型的表达式:



\begin{itemize}
\item 函数体以 \texttt{Body::UNION\{T1,T2\})} 开头

\begin{itemize}
\item 解释:函数具有不稳定返回类型


\item 建议:使返回值类型稳定,即使你必须对其进行类型注释

\end{itemize}

\item \texttt{invoke Main.g(\%\%x::Int64)::UNION\{FLOAT64, INT64\}}

\begin{itemize}
\item 解释:调用类型不稳定的函数 \texttt{g}。


\item 建议:修改该函数,或在必要时对其返回值进行类型注释

\end{itemize}

\item \texttt{invoke Base.getindex(\%\%x::Array\{Any,1\}, 1::Int64)::ANY}

\begin{itemize}
\item 解释:访问缺乏类型信息的数组的元素


\item 建议:使用具有更佳定义的类型的数组,或在必要时对访问的单个元素进行类型注释

\end{itemize}

\item \texttt{Base.getfield(\%\%x, :(:data))::ARRAY\{FLOAT64,N\} WHERE N}

\begin{itemize}
\item 解释:获取值为非叶类型的字段。在这种情况下,\texttt{ArrayContainer} 具有字段 \texttt{data::Array\{T\}}。但是 \texttt{Array} 也需要维度 \texttt{N} 来成为具体类型。


\item 建议:使用类似于 \texttt{Array\{T,3\}} 或 \texttt{Array\{T,N\}} 的具体类型,其中的 \texttt{N} 现在是 \texttt{ArrayContainer} 的参数

\end{itemize}
\end{itemize}


\hypertarget{16106323250273012356}{}


\section{被捕获变量的性能}



请考虑以下定义内部函数的示例:




\begin{minted}{julia}
function abmult(r::Int)
    if r < 0
        r = -r
    end
    f = x -> x * r
    return f
end
\end{minted}



函数 \texttt{abmult} 返回一个函数 \texttt{f},它将其参数乘以 \texttt{r} 的绝对值。赋值给 \texttt{f} 的函数称为「闭包」。内部函数还被语言用于 \texttt{do} 代码块和生成器表达式。



这种代码风格为语言带来了性能挑战。解析器在将其转换为较低级别的指令时,基本上通过将内部函数提取到单独的代码块来重新组织上述代码。「被捕获的」变量,比如 \texttt{r},被内部函数共享,且包含它们的作用域会被提取到内部函数和外部函数皆可访问的堆分配「box」中,这是因为语言指定内部作用域中的 \texttt{r} 必须与外部作用域中的 \texttt{r} 相同,就算在外部作用域(或另一个内部函数)修改 \texttt{r} 后也需如此。



前一段的讨论中提到了「解析器」,也就是,包含 \texttt{abmult} 的模块被首次加载时发生的编译前期,而不是首次调用它的编译后期。解析器不「知道」\texttt{Int} 是固定类型,也不知道语句 \texttt{r = -r} 将一个 \texttt{Int} 转换为另一个 \texttt{Int}。类型推断的魔力在编译后期生效。



因此,解析器不知道 \texttt{r} 具有固定类型(\texttt{Int})。一旦内部函数被创建,\texttt{r} 的值也不会改变(因此也不需要 box)。因此,解析器向包含具有抽象类型(比如 \texttt{Any})的对象的 box 发出代码,这对于每次出现的 \texttt{r} 都需要运行时类型分派。这可以通过在上述函数中使用 \texttt{@code\_warntype} 来验证。装箱和运行时的类型分派都有可能导致性能损失。



如果捕获的变量用于代码的性能关键部分,那么以下提示有助于确保它们的使用具有高效性。首先,如果已经知道被捕获的变量不会改变类型,则可以使用类型注释来显式声明类型(在变量上,而不是在右侧):




\begin{minted}{julia}
function abmult2(r0::Int)
    r::Int = r0
    if r < 0
        r = -r
    end
    f = x -> x * r
    return f
end
\end{minted}



类型注释部分恢复由于捕获而导致的丢失性能,因为解析器可以将具体类型与 box 中的对象相关联。更进一步,如果被捕获的变量不再需要 box(因为它不会在闭包创建后被重新分配),就可以用 \texttt{let} 代码块表示,如下所示。




\begin{minted}{julia}
function abmult3(r::Int)
    if r < 0
        r = -r
    end
    f = let r = r
            x -> x * r
    end
    return f
end
\end{minted}



\texttt{let} 代码块创建了一个新的变量 \texttt{r},它的作用域只是内部函数。第二种技术在捕获变量存在时完全恢复了语言性能。请注意,这是编译器的一个快速发展的方面,未来的版本可能不需要依靠这种程度的程序员注释来获得性能。与此同时,一些用户提供的包(如 \href{https://github.com/c42f/FastClosures.jl}{FastClosures})会自动插入像在 \texttt{abmult3} 中那样的 \texttt{let} 语句。



\hypertarget{17337125735077390471}{}


\chapter{Checking for equality with a singleton}



When checking if a value is equal to some singleton it can be better for performance to check for identicality (\texttt{===}) instead of equality (\texttt{==}). The same advice applies to using \texttt{!==} over \texttt{!=}. These type of checks frequently occur e.g. when implementing the iteration protocol and checking if \texttt{nothing} is returned from \hyperlink{1722534687975587846}{\texttt{iterate}}.



\hypertarget{14117620934191882930}{}


\chapter{工作流程建议}



这里是高效使用 Julia 的一些建议。



\hypertarget{1742408234615272902}{}


\section{基于 REPL 的工作流程}



正如在 \hyperlink{10670790884919535588}{Julia REPL} 中演示的那样,Julia 的 REPL 为高效的交互式工作流程提供了丰富的功能。这里是一些可能进一步提升你在命令行下的体验的建议。



\hypertarget{11122328157249839079}{}


\subsection{一个基本的编辑器 / REPL 工作流程}



最基本的 Julia 工作流程是将一个文本编辑器配合 \texttt{julia} 的命令行使用。一般会包含下面一些步骤:



\begin{itemize}
\item \textbf{把还在开发中的代码放到一个临时的模块中。}新建一个文件,例如 \texttt{Tmp.jl},并放到模块中。


\begin{minted}{julia}
module Tmp
export say_hello

say_hello() = println("Hello!")

# your other definitions here

end
\end{minted}


\item \textbf{把测试代码放到另一个文件中。}新建另一个文件,例如 \texttt{tst.jl},开头为


\begin{minted}{julia}
include("Tmp.jl")
import .Tmp
# using .Tmp # we can use `using` to bring the exported symbols in `Tmp` into our namespace

Tmp.say_hello()
# say_hello()

# your other test code here
\end{minted}

并把测试作为 \texttt{Tmp} 的内容。或者,你可以把测试文件的内容打包到一个模块中,例如


\begin{minted}{julia}
module Tst
    include("Tmp.jl")
    import .Tmp
    #using .Tmp

    Tmp.say_hello()
    # say_hello()

    # your other test code here
end
\end{minted}

优点是你的测试代码现在包含在一个模块中,并且不会在 \texttt{Main} 的全局作用域中引入新定义,这样更加整洁。


\item 使用 \texttt{include({\textquotedbl}tst.jl{\textquotedbl})} 来在 Julia REPL 中 \texttt{include} \texttt{tst.jl} 文件。


\item \textbf{打肥皂,冲洗,重复。}(译者注:此为英语幽默,被称为\href{https://en.wikipedia.org/wiki/Lather,\_rinse,\_repeat)描述洗头发的过程}{“洗发算法”}在 \texttt{julia} REPL 中摸索不同的想法,把好的想法存入 \texttt{tst.jl}。要在 \texttt{tst.jl} 被更改后执行它,只需再次 \texttt{include} 它。

\end{itemize}


\hypertarget{173246562791795014}{}


\section{基于浏览器的工作流程}



也可以通过 \href{https://github.com/JuliaLang/IJulia.jl}{IJulia} 在浏览器中与 Julia REPL 进行交互,请到该库的主页查看详细用法。



\hypertarget{940613295112476490}{}


\section{Revise-based workflows}



Whether you{\textquotesingle}re at the REPL or in IJulia, you can typically improve your development experience with \href{https://github.com/timholy/Revise.jl}{Revise}. It is common to configure Revise to start whenever julia is started, as per the instructions in the \href{https://timholy.github.io/Revise.jl/stable/}{Revise documentation}. Once configured, Revise will track changes to files in any loaded modules, and to any files loaded in to the REPL with \texttt{includet} (but not with plain \texttt{include}); you can then edit the files and the changes take effect without restarting your julia session. A standard workflow is similar to the REPL-based workflow above, with the following modifications:



\begin{itemize}
\item[1. ] Put your code in a module somewhere on your load path. There are several options for achieving this, of which two recommended choices are:

a. For long-term projects, use    \href{https://github.com/invenia/PkgTemplates.jl}{PkgTemplates}:

\texttt{julia    using PkgTemplates    t = Template()    generate({\textquotedbl}MyPkg{\textquotedbl}, t)}    This will create a blank package, \texttt{{\textquotedbl}MyPkg{\textquotedbl}}, in your \texttt{.julia/dev} directory.    Note that PkgTemplates allows you to control many different options    through its \texttt{Template} constructor.

In step 2 below, edit \texttt{MyPkg/src/MyPkg.jl} to change the source code, and    \texttt{MyPkg/test/runtests.jl} for the tests.

b. For {\textquotedbl}throw-away{\textquotedbl} projects, you can avoid any need for cleanup    by doing your work in your temporary directory (e.g., \texttt{/tmp}).

Navigate to your temporary directory and launch Julia, then do the following:

\texttt{julia    pkg> generate MyPkg              \# type ] to enter pkg mode    julia> push!(LOAD\_PATH, pwd())   \# hit backspace to exit pkg mode}    If you restart your Julia session you{\textquotesingle}ll have to re-issue that command    modifying \texttt{LOAD\_PATH}.

In step 2 below, edit \texttt{MyPkg/src/MyPkg.jl} to change the source code, and create any    test file of your choosing.


\item[2. ] Develop your package

\emph{Before} loading any code, make sure you{\textquotesingle}re running Revise: say \texttt{using Revise} or follow its documentation on configuring it to run automatically.

Then navigate to the directory containing your test file (here assumed to be \texttt{{\textquotedbl}runtests.jl{\textquotedbl}}) and do the following:


\begin{minted}{julia}
julia> using MyPkg

julia> include("runtests.jl")
\end{minted}

You can iteratively modify the code in MyPkg in your editor and re-run the tests with \texttt{include({\textquotedbl}runtests.jl{\textquotedbl})}.  You generally should not need to restart your Julia session to see the changes take effect (subject to a few limitations, see https://timholy.github.io/Revise.jl/stable/limitations/).

\end{itemize}


\hypertarget{8293020885235442145}{}


\chapter{代码风格指南}



接下来的部分将介绍如何写出具有 Julia 风格的代码。当然,这些规则并不是绝对的,它们只是一些建议,以便更好地帮助你熟悉这门语言,以及在不同的代码设计中做出选择。



\hypertarget{14598192216141137141}{}


\section{写函数,而不是仅仅写脚本}



一开始解决问题的时候,直接从最外层一步步写代码的确很便捷,但你应该尽早地将代码组织成函数。函数有更强的复用性和可测试性,并且能更清楚地让人知道哪些步骤做完了,以及每一步骤的输入输出分别是什么。此外,由于 Julia 编译器特殊的工作方式,写在函数中的代码往往要比最外层的代码运行地快得多。



此外值得一提的是,函数应当接受参数,而不是直接使用全局变量进行操作(\hyperlink{4355513108985533054}{\texttt{pi}} 等常数除外)。



\hypertarget{14766885101141675893}{}


\section{类型不要写得过于具体}



代码应该写得尽可能通用。例如,下面这段代码:




\begin{minted}{julia}
Complex{Float64}(x)
\end{minted}



更好的写法是写成下面的通用函数:




\begin{minted}{julia}
complex(float(x))
\end{minted}



第二个版本会把 \texttt{x} 转换成合适的类型,而不是某个写死的类型。



这种代码风格与函数的参数尤其相关。例如,当一个参数可以是任何整型时,不要将它的类型声明为 \texttt{Int} 或 \hyperlink{10103694114785108551}{\texttt{Int32}},而要使用抽象类型(abstract type)\hyperlink{8469131683393450448}{\texttt{Integer}} 来表示。事实上,除非确实需要将其与其它的方法定义区分开,很多情况下你可以干脆完全省略掉参数的类型,因为如果你的操作中有不支持某种参数类型的操作的话,反正都会抛出 \hyperlink{68769522931907606}{\texttt{MethodError}} 的。这也称作 \href{https://zh.wikipedia.org/wiki/\%E9\%B8\%AD\%E5\%AD\%90\%E7\%B1\%BB\%E5\%9E\%8B}{鸭子类型})。



例如,考虑这样的一个叫做 \texttt{addone} 的函数,其返回值为它的参数加 1 :




\begin{minted}{julia}
addone(x::Int) = x + 1                 # works only for Int
addone(x::Integer) = x + oneunit(x)    # any integer type
addone(x::Number) = x + oneunit(x)     # any numeric type
addone(x) = x + oneunit(x)             # any type supporting + and oneunit
\end{minted}



最后一种定义可以处理所有支持 \hyperlink{2310843180104103470}{\texttt{oneunit}} (返回和 \texttt{x} 相同类型的 1,以避免不需要的类型提升(type promotion))以及 \hyperlink{3677358729494553841}{\texttt{+}} 函数的类型。这里的关键点在于,\textbf{只}定义通用的 \texttt{addone(x) = x + oneunit(x)} 并\textbf{不会}带来性能上的损失,因为 Julia 会在需要的时候自动编译特定的版本。比如说,当第一次调用 \texttt{addone(12)} 时,Julia 会自动编译一个特定的 \texttt{addone} 函数,它接受一个 \texttt{x::Int} 的参数,并把调用的 \texttt{oneunit} 替换为内连的值 \texttt{1}。因此,上述的前三种 \texttt{addone} 的定义对于第四种来说是完全多余的。



\hypertarget{14005859971102001375}{}


\section{让调用者处理多余的参数多样性}



如下的代码:




\begin{minted}{julia}
function foo(x, y)
    x = Int(x); y = Int(y)
    ...
end
foo(x, y)
\end{minted}



请写成这样:




\begin{minted}{julia}
function foo(x::Int, y::Int)
    ...
end
foo(Int(x), Int(y))
\end{minted}



这种风格更好,因为 \texttt{foo} 函数其实不需要接受所有类型的数,而只需要接受 \texttt{Int}。



这里的关键在于,如果一个函数需要处理的是整数,强制让调用者来决定非整数如何被转换(比如说向下还是向上取整)会更好。同时,把类型声明得具体一些的话可以为以后的方法定义留有更多的空间。



\hypertarget{1684346418186777370}{}


\section{Append \texttt{!} to names of functions that modify their arguments}



如下的代码:




\begin{minted}{julia}
function double(a::AbstractArray{<:Number})
    for i = firstindex(a):lastindex(a)
        a[i] *= 2
    end
    return a
end
\end{minted}



请写成这样:




\begin{minted}{julia}
function double!(a::AbstractArray{<:Number})
    for i = firstindex(a):lastindex(a)
        a[i] *= 2
    end
    return a
end
\end{minted}



Julia 的 Base 模块中的函数都遵循了这种规范,且包含很多例子:有的函数同时有拷贝和修改的形式(比如 \hyperlink{8473525809131227606}{\texttt{sort}} 和 \hyperlink{12296873681374954808}{\texttt{sort!}}),还有一些只有修改(比如 \hyperlink{18026893834387542681}{\texttt{push!}},\hyperlink{14467641005327674015}{\texttt{pop!}} 和 \hyperlink{13785507599688955371}{\texttt{splice!}})。为了方便起见,这类函数通常也会把修改后的数组作为返回值。



\hypertarget{14194687290331521644}{}


\section{避免使用奇怪的 \texttt{Union} 类型}



使用 \texttt{Union\{Function,AbstractString\}} 这样的类型的时候通常意味着设计还不够清晰。



\hypertarget{6548025962580232135}{}


\section{避免复杂的容器类型}



像下面这样构造数组通常没有什么好处:




\begin{minted}{julia}
a = Vector{Union{Int,AbstractString,Tuple,Array}}(undef, n)
\end{minted}



这种情况下,\texttt{Vector\{Any\}(undef, n)}更合适些。此外,相比将所有可能的类型都打包在一起,直接在使用时标注具体的数据类型(比如:\texttt{a[i]::Int})对编译器来说更有用。



\hypertarget{17294992054689205687}{}


\section{使用和 Julia \texttt{base/} 文件夹中的代码一致的命名习惯}



\begin{itemize}
\item module 和 type 的名字使用大写开头的驼峰命名法:\texttt{module SparseArrays},\texttt{struct UnitRange}。


\item 函数名使用小写字母,且当可读时可以将多个单词拼在一起。必要的时候,可以使用下划线作为单词分隔符。下划线也被用于指明概念的组合(比如 \hyperlink{14909184572421937971}{\texttt{remotecall\_fetch}} 作为 \texttt{fetch(remotecall(...))} 的一个更高效的实现)或者变化。


\item 虽然简洁性很重要,但避免使用缩写(用 \hyperlink{16333903346703258373}{\texttt{indexin}} 而不是 \texttt{indxin}),因为这会让记住单词有没有被缩写或如何被缩写变得十分困难。

\end{itemize}


如果一个函数名需要多个单词,请考虑这个函数是否代表了超过一个概念,是不是分成几个更小的部分更好。



\hypertarget{527161729759687978}{}


\section{使用与 Julia Base 中的函数类似的参数顺序}



一般来说,Base 库使用以下的函数参数顺序(如适用):



\begin{itemize}
\item[1.  ] \textbf{函数参数}. 函数的第一个参数可以接受 \texttt{Function} 类型,以便使用 \hyperlink{16455129305818705265}{\texttt{do}} blocks 来传递多行匿名函数。


\item[2.  ] \textbf{I/O stream}. 函数的第一个参数可以接受 \texttt{IO} 对象,以便将函数传递给 \hyperlink{6652981552509545835}{\texttt{sprint}} 之类的函数,例如 \texttt{sprint(show, x)}。


\item[3.  ] \textbf{在输入参数的内容会被更改的情况下}. 比如,在 \hyperlink{5162290739791026948}{\texttt{fill!(x, v)}} 中,\texttt{x} 是要被修改的对象,所以放在要被插入 \texttt{x} 中的值前面。


\item[4.  ] \textbf{Type}. 把类型作为参数传入函数通常意味着返回值也会是同样的类型。 在 \hyperlink{14207407853646164654}{\texttt{parse(Int, {\textquotedbl}1{\textquotedbl})}} 中,类型在需要解析的字符串之前。 还有很多类似的将类型作为函数第一个参数的例子,但是同时也需要注意到例如 \hyperlink{8104134490906192097}{\texttt{read(io, String)}} 这样的函数中,会把 \texttt{IO} 参数放在类型的更前面,这样还是保持着这里描述的顺序。


\item[5.  ] \textbf{在输入参数的内容不会被更改的情况下}. 比如在 \texttt{fill!(x, v)} 中的\textbf{不}被修改的 \texttt{v},会放在 \texttt{x} 之后传入。


\item[6.  ] \textbf{Key}. 对于关联集合来说,指的是键值对的键。 对于其它有索引的集合来说,指的是索引。


\item[7.  ] \textbf{Value}. 对于关联集合来说,指的是键值对的值。 In cases like \hyperlink{5162290739791026948}{\texttt{fill!(x, v)}}, this is \texttt{v}.


\item[8.  ] \textbf{Everything else}. 任何的其它参数。


\item[9.  ] \textbf{Varargs}. 指的是在函数调用时可以被无限列在后面的参数。 比如在 \texttt{Matrix\{T\}(uninitialized, dims)} 中,维数(dims)可以作为 \hyperlink{17462354060312563026}{\texttt{Tuple}} 被传入(如 \texttt{Matrix\{T\}(uninitialized, (1,2))}),也可以作为可变参数(\hyperlink{5941806424098279588}{\texttt{Vararg}},如 \texttt{Matrix\{T\}(uninitialized, 1, 2)}。


\item[10. ] \textbf{Keyword arguments}. 在 Julia 中,关键字参数本来就不得不定义在函数定义的最后,列在这里仅仅是为了完整性。

\end{itemize}


大多数函数并不会接受上述所有种类的参数,这些数字仅仅是表示当适用时的优先权。



当然,在一些情况下有例外。例如,\hyperlink{1846942650946171605}{\texttt{convert}} 函数总是把类型作为第一个参数。\hyperlink{1309244355901386657}{\texttt{setindex!}} 函数的值参数在索引参数之前,这样可以让索引作为可变参数传入。



设计 API 时,尽可能秉承着这种一般顺序会让函数的使用者有一种更一致的体验。



\hypertarget{16340212137156679332}{}


\section{不要过度使用 try-catch}



比起依赖于捕获错误,更好的是避免错误。



\hypertarget{15308468417216899816}{}


\section{不要给条件语句加括号}



Julia 不要求在 \texttt{if} 和 \texttt{while} 后的条件两边加括号。使用如下写法:




\begin{minted}{julia}
if a == b
\end{minted}



而不是:




\begin{minted}{julia}
if (a == b)
\end{minted}



\hypertarget{13344988799401234550}{}


\section{不要过度使用 \texttt{...}}



拼接函数参数是会上瘾的。请用简单的 \texttt{[a; b]} 来代替 \texttt{[a..., b...]},因为前者已经是被拼接的数组了。\hyperlink{6278865767444641812}{\texttt{collect(a)}} 也比 \texttt{[a...]} 更好,但因为 \texttt{a} 已经是一个可迭代的变量了,通常不把它转换成数组就直接使用甚至更好。



\hypertarget{13140702923511466392}{}


\section{不要使用不必要的静态参数}



如下的函数签名:




\begin{minted}{julia}
foo(x::T) where {T<:Real} = ...
\end{minted}



应当被写作:




\begin{minted}{julia}
foo(x::Real) = ...
\end{minted}



尤其是当 \texttt{T} 没有被用在函数体中时格外有意义。即使 \texttt{T} 被用到了,通常也可以被替换为 \hyperlink{13440452181855594120}{\texttt{typeof(x)}},后者不会导致性能上的差别。注意这并不是针对静态参数的一般警告,而仅仅是针对那些不必要的情况。



同样需要注意的是,容器类型在函数调用中可能明确地需要类型参数。详情参见\hyperlink{11695962945306703148}{避免使用带抽象容器的字段}。



\hypertarget{14321901677948654689}{}


\section{避免判断变量是实例还是类型的混乱}



如下的一组定义容易令人困惑:




\begin{minted}{julia}
foo(::Type{MyType}) = ...
foo(::MyType) = foo(MyType)
\end{minted}



请决定问题里的概念应当是 \texttt{MyType} 还是 \texttt{MyType()},然后坚持使用其一。



默认使用实例是比较受推崇的风格,然后只在为了解决一些问题必要时添加涉及到 \texttt{Type\{MyType\}} 的方法。



如果一个类型实际上是个枚举,它应该被定义成一个单一的类型(理想的情况是不可变结构或原始类型),把枚举值作为它的实例。构造器和转换器可以检查那些值是否有效。这种设计比把枚举做成抽象类型,并把“值”做成子类型来得更受推崇。



\hypertarget{678667303123676019}{}


\section{不要过度使用宏}



请注意有的宏实际上可以被写成一个函数。



在宏内部调用 \hyperlink{7507639810592563424}{\texttt{eval}} 是一个特别危险的警告标志,它意味着这个宏仅在被最外层调用时起作用。如果这样的宏被写成函数,它会自然地访问得到它所需要的运行时值。



\hypertarget{9477312169098625763}{}


\section{不要把不安全的操作暴露在接口层}



如果你有一个使用本地指针的类型:




\begin{minted}{julia}
mutable struct NativeType
    p::Ptr{UInt8}
    ...
end
\end{minted}



不要定义类似如下的函数:




\begin{minted}{julia}
getindex(x::NativeType, i) = unsafe_load(x.p, i)
\end{minted}



这里的问题在于,这个类型的用户可能会在意识不到这个操作不安全的情况下写出 \texttt{x[i]},然后容易遇到内存错误。



在这样的函数中,可以加上对操作的检查来确保安全,或者可以在名字的某处加上 \texttt{unsafe} 来警告调用者。



\hypertarget{11547972187147924005}{}


\section{不要重载基础容器类型的方法}



有时可能会想要写这样的定义:




\begin{minted}{julia}
show(io::IO, v::Vector{MyType}) = ...
\end{minted}



这样可以提供对特定的某种新元素类型的向量的自定义显示。这种做法虽然很诱人,但应当被避免。这里的问题在于用户会想着一个像 \texttt{Vector()} 这样熟知的类型以某种方式表现,但过度自定义的行为会让使用变得更难。



\hypertarget{16979726342714314714}{}


\section{避免类型盗版}



“类型盗版”(type piracy)指的是扩展或是重定义 Base 或其它包中的并不是你所定义的类型的方法。在某些情况下,你可以几乎毫无副作用地逃避类型盗版。但在极端情况下,你甚至会让 Julia 崩溃(比如说你的方法扩展或重定义造成了对 \texttt{ccall} 传入了无效的输入)。类型盗版也让代码推导变得更复杂,且可能会引入难以预料和诊断的不兼容性。



例如,你也许想在一个模块中定义符号上的乘法:




\begin{minted}{julia}
module A
import Base.*
*(x::Symbol, y::Symbol) = Symbol(x,y)
end
\end{minted}



这里的问题时现在其它用到 \texttt{Base.*} 的模块同样会看到这个定义。由于 \texttt{Symbol} 是定义在 Base 里再被其它模块所使用的,这可能不可预料地改变无关代码的行为。这里有几种替代的方式,包括使用一个不同的函数名称,或是把 \texttt{Symbol} 给包在另一个你自己定义的类型中。



有时候,耦合的包可能会使用类型盗版,以此来从定义分隔特性,尤其是当那些包是一些合作的作者设计的时候,且那些定义是可重用的时候。例如,一个包可能提供一些对处理色彩有用的类型,另一个包可能为那些类型定义色彩空间之间转换的方法。再举一个例子,一个包可能是一些 C 代码的简易包装,另一个包可能就“盗版”来实现一些更高级别的、对 Julia 友好的 API。



\hypertarget{11741910651274288820}{}


\section{注意类型相等}



通常会用 \hyperlink{7066325108767373297}{\texttt{isa}} 和 \hyperlink{6254591906563366276}{\texttt{<:}} 来对类型进行测试,而不会用到 \texttt{==}。检测类型的相等通常只对和一个已知的具体类型比较有意义(例如 \texttt{T == Float64}),或者你\textbf{真的真的}知道自己在做什么。



\hypertarget{20919813611265277}{}


\section{不要写 \texttt{x->f(x)}}



因为调用高阶函数时经常会用到匿名函数,很容易认为这是合理甚至必要的。但任何函数都可以被直接传递,并不需要被“包{\textquotedbl}在一个匿名函数中。比如 \texttt{map(x->f(x), a)} 应当被写成 \hyperlink{11483231213869150535}{\texttt{map(f, a)}}。



\hypertarget{12032676153757060243}{}


\section{尽可能避免使用浮点数作为通用代码的字面量}



当写处理数字,且可以处理多种不同数字类型的参数的通用代码时,请使用对参数影响(通过类型提升)尽可能少的类型的字面量。



例如,




\begin{minted}{jlcon}
julia> f(x) = 2.0 * x
f (generic function with 1 method)

julia> f(1//2)
1.0

julia> f(1/2)
1.0

julia> f(1)
2.0
\end{minted}



而应当被写作:




\begin{minted}{jlcon}
julia> g(x) = 2 * x
g (generic function with 1 method)

julia> g(1//2)
1//1

julia> g(1/2)
1.0

julia> g(1)
2
\end{minted}



如你所见,使用了 \texttt{Int} 字面量的第二个版本保留了输入参数的类型,而第一个版本没有。这是因为例如 \texttt{promote\_type(Int, Float64) == Float64},且做乘法时会需要类型提升。类似地,\hyperlink{8304566144531167610}{\texttt{Rational}} 字面量比 \hyperlink{5027751419500983000}{\texttt{Float64}} 字面量对类型有着更小的破坏性,但比 \texttt{Int} 大。




\begin{minted}{jlcon}
julia> h(x) = 2//1 * x
h (generic function with 1 method)

julia> h(1//2)
1//1

julia> h(1/2)
1.0

julia> h(1)
2//1
\end{minted}



所以,可能时尽量使用 \texttt{Int} 字面量,对非整数字面量使用 \texttt{Rational\{Int\}},这样可以让代码变得更容易使用。



\hypertarget{17031291324185486008}{}


\chapter{常见问题}



\hypertarget{13603481244956250670}{}


\section{General}



\hypertarget{4693498942886852861}{}


\subsection{Is Julia named after someone or something?}



No.



\hypertarget{5072634749249852737}{}


\subsection{Why don{\textquotesingle}t you compile Matlab/Python/R/… code to Julia?}



Since many people are familiar with the syntax of other dynamic languages, and lots of code has already been written in those languages, it is natural to wonder why we didn{\textquotesingle}t just plug a Matlab or Python front-end into a Julia back-end (or “transpile” code to Julia) in order to get all the performance benefits of Julia without requiring programmers to learn a new language.  Simple, right?



The basic issue is that there is \emph{nothing special about Julia{\textquotesingle}s compiler}: we use a commonplace compiler (LLVM) with no “secret sauce” that other language developers don{\textquotesingle}t know about.  Indeed, Julia{\textquotesingle}s compiler is in many ways much simpler than those of other dynamic languages (e.g. PyPy or LuaJIT).   Julia{\textquotesingle}s performance advantage derives almost entirely from its front-end: its language semantics allow a \hyperlink{818954303942149020}{well-written Julia program} to \emph{give more opportunities to the compiler} to generate efficient code and memory layouts.  If you tried to compile Matlab or Python code to Julia, our compiler would be limited by the semantics of Matlab or Python to producing code no better than that of existing compilers for those languages (and probably worse).  The key role of semantics is also why several existing Python compilers (like Numba and Pythran) only attempt to optimize a small subset of the language (e.g. operations on Numpy arrays and scalars), and for this subset they are already doing at least as well as we could for the same semantics.  The people working on those projects are incredibly smart and have accomplished amazing things, but retrofitting a compiler onto a language that was designed to be interpreted is a very difficult problem.



Julia{\textquotesingle}s advantage is that good performance is not limited to a small subset of “built-in” types and operations, and one can write high-level type-generic code that works on arbitrary user-defined types while remaining fast and memory-efficient.  Types in languages like Python simply don{\textquotesingle}t provide enough information to the compiler for similar capabilities, so as soon as you used those languages as a Julia front-end you would be stuck.



For similar reasons, automated translation to Julia would also typically generate unreadable, slow, non-idiomatic code that would not be a good starting point for a native Julia port from another language.



On the other hand, language \emph{interoperability} is extremely useful: we want to exploit existing high-quality code in other languages from Julia (and vice versa)!  The best way to enable this is not a transpiler, but rather via easy inter-language calling facilities.  We have worked hard on this, from the built-in \texttt{ccall} intrinsic (to call C and Fortran libraries) to \href{https://github.com/JuliaInterop}{JuliaInterop} packages that connect Julia to Python, Matlab, C++, and more.



\hypertarget{4239210502811614787}{}


\section{会话和 REPL}



\hypertarget{503979811911045199}{}


\subsection{如何从内存中删除某个对象?}



Julia 没有类似于 MATLAB 的 \texttt{clear} 函数,某个名称一旦定义在 Julia 的会话中(准确地说,在 \texttt{Main} 模块中),它就会一直存在下去。



如果关心内存用量,一个对象总能被一个占用更少内存的对象替换掉。例如,如果 \texttt{A} 是一个不再需要的 GB 量级的数组,可以使用 \texttt{A = nothing} 来释放内存。该内存将在下一次垃圾回收器运行时被释放,也可以使用 \hyperlink{16287035550645122381}{\texttt{gc()}} 强制进行垃圾回收。另外,试图使用 \texttt{A} 很可能导致错误,因为大部分方法(method)在 \texttt{Nothing} 类型上没有定义。



\hypertarget{191104846954255908}{}


\subsection{如何在会话中修改某个类型的声明?}



也许你定义了某个类型,后来发现需要向其中增加一个新的域。如果在 REPL 中尝试这样做,会得到一个错误:




\begin{lstlisting}
ERROR: invalid redefinition of constant MyType
\end{lstlisting}



模块 \texttt{Main} 中的类型不能重新定义。



尽管这在开发新代码时会造成不便,但是这个问题仍然有一个不错的解决办法:可以用重新定义的模块替换原有的模块,把所有新代码封装在一个模块里,这样就能重新定义类型和常量了。虽说不能将类型名称导入到 \texttt{Main} 模块中再去重新定义,但是可以用模块名来改变作用范围。换言之,开发时的工作流可能类似这样:




\begin{minted}{julia}
include("mynewcode.jl") # this defines a module MyModule
obj1 = MyModule.ObjConstructor(a, b)
obj2 = MyModule.somefunction(obj1)
# Got an error. Change something in "mynewcode.jl"
include("mynewcode.jl") # reload the module
obj1 = MyModule.ObjConstructor(a, b) # old objects are no longer valid, must reconstruct
obj2 = MyModule.somefunction(obj1) # this time it worked!
obj3 = MyModule.someotherfunction(obj2, c)
...
\end{minted}



\hypertarget{11224408082580741592}{}


\section{脚本}



\hypertarget{14939284717840791757}{}


\subsection{该如何检查当前文件是否正在以主脚本运行?}



当一个文件通过使用 \texttt{julia file.jl} 来当做主脚本运行时,有人也希望激活另外的功能例如命令行参数操作。确定文件是以这个方式运行的一个方法是检查 \texttt{abspath(PROGRAM\_FILE) == @\_\_FILE\_\_} 是不是 \texttt{true}。



\hypertarget{5414221961728470242}{}


\subsection{How do I catch CTRL-C in a script?}



通过 \texttt{julia file.jl} 方式运行的 Julia 脚本,在你尝试按 CTRL-C (SIGINT) 中止它时,并不会抛出 \hyperlink{11255134339055983338}{\texttt{InterruptException}}。如果希望在脚本终止之后运行一些代码,请使用 \hyperlink{17479944696971324992}{\texttt{atexit}},注意:脚本的中止不一定是由 CTRL-C 导致的。 另外你也可以通过 \texttt{julia -e {\textquotesingle}include(popfirst!(ARGS)){\textquotesingle} file.jl} 命令运行脚本,然后可以通过 \hyperlink{16338536928035025961}{\texttt{try}} 捕获 \texttt{InterruptException}。



\hypertarget{15272704088350857853}{}


\subsection{怎样通过 \texttt{\#!/usr/bin/env} 传递参数给 \texttt{julia}?}



通过类似 \texttt{\#!/usr/bin/env julia --startup-file=no} 的方式,使用 shebang 传递选项给 Julia 的方法,可能在像 Linux 这样的平台上无法正常工作。这是因为各平台上 shebang 的参数解析是平台相关的,并且尚未标准化。 在类 Unix 的环境中,可以通过以 \texttt{bash} 脚本作为可执行脚本的开头,并使用 \texttt{exec} 代替给 \texttt{julia} 传递选项的过程,来可靠的为 \texttt{julia} 传递选项。




\begin{minted}{julia}
#!/bin/bash
#=
exec julia --color=yes --startup-file=no "${BASH_SOURCE[0]}" "$@"
=#

@show ARGS  # put any Julia code here
\end{minted}



在以上例子中,位于 \texttt{\#=} 和 \texttt{=\#} 之间的代码可以当作一个 \texttt{bash} 脚本。 因为这些代码放在 Julia 的多行注释中,所以 Julia 会忽略它们。 在 \texttt{=\#} 之后的 Julia 代码会被 \texttt{bash} 忽略,J因为当文件解析到 \texttt{exec} 语句时会停止解析,开始执行命令。



\begin{quote}
\textbf{Note}

In order to \hyperlink{15294909856889946817}{catch CTRL-C} in the script you can use


\begin{minted}{julia}
#!/bin/bash
#=
exec julia --color=yes --startup-file=no -e 'include(popfirst!(ARGS))' \
    "${BASH_SOURCE[0]}" "$@"
=#

@show ARGS  # put any Julia code here
\end{minted}

instead. Note that with this strategy \hyperlink{9054270179006636705}{\texttt{PROGRAM\_FILE}} will not be set.

\end{quote}


\hypertarget{5727276226756196747}{}


\section{函数}



\hypertarget{7382572299104431822}{}


\subsection{向函数传递了参数 \texttt{x},在函数中做了修改,但是在函数外变量 \texttt{x} 的值还是没有变。为什么?}



假设函数被如此调用:




\begin{minted}{jlcon}
julia> x = 10
10

julia> function change_value!(y)
           y = 17
       end
change_value! (generic function with 1 method)

julia> change_value!(x)
17

julia> x # x is unchanged!
10
\end{minted}



在 Julia 中,通过将 \texttt{x} 作为参数传递给函数,不能改变变量 \texttt{x} 的绑定。在上例中,调用 \texttt{change\_value!(x)} 时,\texttt{y} 是一个新建变量,初始时与 \texttt{x} 的值绑定,即 \texttt{10}。然后 \texttt{y} 与常量 \texttt{17} 重新绑定,此时变量外作用域中的 \texttt{x} 并没有变动。



However, if \texttt{x} is bound to an object of type \texttt{Array} (or any other \emph{mutable} type). From within the function, you cannot {\textquotedbl}unbind{\textquotedbl} \texttt{x} from this Array, but you \emph{can} change its content. For example:




\begin{minted}{jlcon}
julia> x = [1,2,3]
3-element Array{Int64,1}:
 1
 2
 3

julia> function change_array!(A)
           A[1] = 5
       end
change_array! (generic function with 1 method)

julia> change_array!(x)
5

julia> x
3-element Array{Int64,1}:
 5
 2
 3
\end{minted}



这里我们新建了一个函数 \texttt{chang\_array!},它把 \texttt{5} 赋值给传入的数组(在调用处与 \texttt{x} 绑定,在函数中与 \texttt{A} 绑定)的第一个元素。注意,在函数调用之后,\texttt{x} 依旧与同一个数组绑定,但是数组的内容变化了:变量 \texttt{A} 和 \texttt{x} 是不同的绑定,引用同一个可变的 \texttt{Array} 对象。



\hypertarget{10749355378766657270}{}


\subsection{函数内部能否使用 \texttt{using} 或 \texttt{import}?}



不可以,不能在函数内部使用 \texttt{using} 或 \texttt{import} 语句。如果你希望导入一个模块,但只在特定的一个或一组函数中使用它的符号,有以下两种方式:



\begin{itemize}
\item[1. ] 使用 \texttt{import}:


\begin{minted}{julia}
import Foo
function bar(...)
    # ... refer to Foo symbols via Foo.baz ...
end
\end{minted}

这会加载 \texttt{Foo} 模块,同时定义一个变量 \texttt{Foo} 引用该模块,但并不会 将其他任何符号从该模块中导入当前的命名空间。 \texttt{Foo} 等符号可以由限定的名称 \texttt{Foo.bar} 等引用。


\item[2. ] 将函数封装到模块中:


\begin{minted}{julia}
module Bar
export bar
using Foo
function bar(...)
    # ... refer to Foo.baz as simply baz ....
end
end
using Bar
\end{minted}

这会从 \texttt{Foo} 中导入所有符号,但仅限于 \texttt{Bar} 模块内。

\end{itemize}


\hypertarget{425348862563535930}{}


\subsection{运算符 \texttt{...} 有何作用?}



\hypertarget{6176809975782961444}{}


\subsection{\texttt{...} 运算符的两个用法:slurping 和 splatting}



很多 Julia 的新手会对运算符 \texttt{...} 的用法感到困惑。让 \texttt{...} 用法如此困惑的部分原因是根据上下文它有两种不同的含义。



\hypertarget{7095517523544633865}{}


\subsection{\texttt{...} 在函数定义中将多个参数组合成一个参数}



在函数定义的上下文中,\texttt{...}运算符用来将多个不同的参数组合成单个参数。\texttt{...}运算符的这种将多个不同参数组合成单个参数的用法称为slurping:




\begin{minted}{jlcon}
julia> function printargs(args...)
           println(typeof(args))
           for (i, arg) in enumerate(args)
               println("Arg #$i = $arg")
           end
       end
printargs (generic function with 1 method)

julia> printargs(1, 2, 3)
Tuple{Int64,Int64,Int64}
Arg #1 = 1
Arg #2 = 2
Arg #3 = 3
\end{minted}



如果Julia是一个使用ASCII字符更加自由的语言的话,slurping运算符可能会写作\texttt{<-...}而非\texttt{...}。



\hypertarget{4752115827323838494}{}


\subsection{\texttt{...}在函数调用中将一个参数分解成多个不同参数}



与在定义函数时表示将多个不同参数组合成一个参数的\texttt{...}运算符用法相对,当用在函数调用的上下文中\texttt{...}运算符也用来将单个的函数参数分成多个不同的参数。\texttt{...}函数的这个用法叫做splatting:




\begin{minted}{jlcon}
julia> function threeargs(a, b, c)
           println("a = $a::$(typeof(a))")
           println("b = $b::$(typeof(b))")
           println("c = $c::$(typeof(c))")
       end
threeargs (generic function with 1 method)

julia> x = [1, 2, 3]
3-element Array{Int64,1}:
 1
 2
 3

julia> threeargs(x...)
a = 1::Int64
b = 2::Int64
c = 3::Int64
\end{minted}



如果Julia是一个使用ASCII字符更加自由的语言的话,splatting运算符可能会写作\texttt{...->}而非\texttt{...}。



\hypertarget{18313029058103158138}{}


\subsection{赋值语句的返回值是什么?}



\texttt{=}运算符始终返回右侧的值,所以:




\begin{minted}{jlcon}
julia> function threeint()
           x::Int = 3.0
           x # returns variable x
       end
threeint (generic function with 1 method)

julia> function threefloat()
           x::Int = 3.0 # returns 3.0
       end
threefloat (generic function with 1 method)

julia> threeint()
3

julia> threefloat()
3.0
\end{minted}



相似地:




\begin{minted}{jlcon}
julia> function threetup()
           x, y = [3, 3]
           x, y # returns a tuple
       end
threetup (generic function with 1 method)

julia> function threearr()
           x, y = [3, 3] # returns an array
       end
threearr (generic function with 1 method)

julia> threetup()
(3, 3)

julia> threearr()
2-element Array{Int64,1}:
 3
 3
\end{minted}



\hypertarget{5608144491570256308}{}


\section{类型,类型声明和构造函数}



\hypertarget{11170875837665758023}{}


\subsection{何谓“类型稳定”?}



这意味着输出的类型可以由输入的类型预测出来。特别地,这意味着输出的类型不会因输入的\emph{值}的不同而变化。以下代码\emph{不是}类型稳定的:




\begin{minted}{jlcon}
julia> function unstable(flag::Bool)
           if flag
               return 1
           else
               return 1.0
           end
       end
unstable (generic function with 1 method)
\end{minted}



It returns either an \texttt{Int} or a \hyperlink{5027751419500983000}{\texttt{Float64}} depending on the value of its argument. Since Julia can{\textquotesingle}t predict the return type of this function at compile-time, any computation that uses it must be able to cope with values of both types, which makes it hard to produce fast machine code.



\hypertarget{6904365807459053438}{}


\subsection{为何 Julia 对某个看似合理的操作返回 \texttt{DomainError}?}



某些运算在数学上有意义,但会产生错误:




\begin{minted}{jlcon}
julia> sqrt(-2.0)
ERROR: DomainError with -2.0:
sqrt will only return a complex result if called with a complex argument. Try sqrt(Complex(x)).
Stacktrace:
[...]
\end{minted}



这一行为是为了保证类型稳定而带来的不便。对于 \hyperlink{4551113327515323898}{\texttt{sqrt}},许多用户会希望 \texttt{sqrt(2.0)} 产生一个实数,如果得到了复数 \texttt{1.4142135623730951 + 0.0im} 则会不高兴。也可以编写 \hyperlink{4551113327515323898}{\texttt{sqrt}} 函数,只有当传递一个负数时才切换到复值输出,但结果将不是\hyperlink{5872221809740029239}{类型稳定}的,而且 \hyperlink{4551113327515323898}{\texttt{sqrt}} 函数的性能会很差。



在这样那样的情况下,若你想得到希望的结果,你可以选择一个\emph{输入类型},它可以使根据你的想法接受一个\emph{输出类型},从而结果可以这样表示:




\begin{minted}{jlcon}
julia> sqrt(-2.0+0im)
0.0 + 1.4142135623730951im
\end{minted}



\hypertarget{11272547309716265284}{}


\subsection{How can I constrain or compute type parameters?}



The parameters of a \href{@ref Parametric-Types}{parametric type} can hold either types or bits values, and the type itself chooses how it makes use of these parameters. For example, \texttt{Array\{Float64, 2\}} is parameterized by the type \texttt{Float64} to express its element type and the integer value \texttt{2} to express its number of dimensions.  When defining your own parametric type, you can use subtype constraints to declare that a certain parameter must be a subtype (\hyperlink{6254591906563366276}{\texttt{<:}}) of some abstract type or a previous type parameter.  There is not, however, a dedicated syntax to declare that a parameter must be a \emph{value} of a given type — that is, you cannot directly declare that a dimensionality-like parameter \hyperlink{7066325108767373297}{\texttt{isa}} \texttt{Int} within the \texttt{struct} definition, for example.  Similarly, you cannot do computations (including simple things like addition or subtraction) on type parameters.  Instead, these sorts of constraints and relationships may be expressed through additional type parameters that are computed and enforced within the type{\textquotesingle}s \hyperlink{1489967485005487723}{constructors}.



As an example, consider




\begin{minted}{julia}
struct ConstrainedType{T,N,N+1} # NOTE: INVALID SYNTAX
    A::Array{T,N}
    B::Array{T,N+1}
end
\end{minted}



where the user would like to enforce that the third type parameter is always the second plus one. This can be implemented with an explicit type parameter that is checked by an \hyperlink{5052047505447273614}{inner constructor method} (where it can be combined with other checks):




\begin{minted}{julia}
struct ConstrainedType{T,N,M}
    A::Array{T,N}
    B::Array{T,M}
    function ConstrainedType(A::Array{T,N}, B::Array{T,M}) where {T,N,M}
        N + 1 == M || throw(ArgumentError("second argument should have one more axis" ))
        new{T,N,M}(A, B)
    end
end
\end{minted}



This check is usually \emph{costless}, as the compiler can elide the check for valid concrete types. If the second argument is also computed, it may be advantageous to provide an \hyperlink{1408947822788665444}{outer constructor method} that performs this calculation:




\begin{minted}{julia}
ConstrainedType(A) = ConstrainedType(A, compute_B(A))
\end{minted}



\hypertarget{3408319939041447292}{}


\subsection{Why does Julia use native machine integer arithmetic?}



Julia使用机器算法进行整数计算。这意味着\texttt{Int}的范围是有界的,值在范围的两端循环,也就是说整数的加法,减法和乘法会出现上溢或者下溢,导致出现某些从开始就令人不安的结果:




\begin{minted}{jlcon}
julia> typemax(Int)
9223372036854775807

julia> ans+1
-9223372036854775808

julia> -ans
-9223372036854775808

julia> 2*ans
0
\end{minted}



无疑,这与数学上的整数的行为很不一样,并且你会想对于高阶编程语言来说把这个暴露给用户难称完美。然而,对于效率优先和透明度优先的数值计算来说,其他的备选方案可谓更糟。



一个备选方案是去检查每个整数运算是否溢出,如果溢出则将结果提升到更大的整数类型比如\hyperlink{8012327724714767060}{\texttt{Int128}}或者\hyperlink{423405808990690832}{\texttt{BigInt}}。 不幸的是,这会给所有的整数操作(比如让循环计数器自增)带来巨大的额外开销 — 这需要生成代码去在算法指令后进行运行溢出检测,并生成分支去处理潜在的溢出。更糟糕的是,这会让涉及整数的所有运算变得类型不稳定。如同上面提到的,对于高效生成高效的代码\hyperlink{5872221809740029239}{类型稳定很重要}。如果不指望整数运算的结果是整数,就无法想C和Fortran编译器一样生成快速简单的代码。



这个方法有个变体可以避免类型不稳定的出现,这个变体是将类型\texttt{Int}和\hyperlink{423405808990690832}{\texttt{BigInt}}合并成单个混合整数类型,当结果不再满足机器整数的大小时会内部自动切换表示。虽然表面上在Julia代码层面解决了类型不稳定,但是这个只是通过将所有的困难硬塞给实现混合整数类型的C代码而掩盖了这个问题。这个方法\emph{可能}有用,甚至在很多情况下速度很快,但是它有很多缺点。一个缺点是整数和整数数组的内存上的表示不再与C、Fortran和其他使用原生机器整数的怨言所使用的自然表示一样。所以,为了与那些语言协作,我们无论如何最终都需要引入原生整数类型。任何整数的无界表示都不会占用固定的比特数,所以无法使用固定大小的槽来内联地存储在数组中 — 大的整数值通常需要单独的堆分配的存储。并且无论使用的混合整数实现多么智能,总会存在性能陷阱 — 无法预期的性能下降的情况。复杂的表示,与C和Fortran协作能力的缺乏,无法在不使用另外的堆存储的情况下表示整数数组,和无法预测的性能特性让即使是最智能化的混合整数实现对于高性能数值计算来说也是个很差的选择。



除了使用混合整数和提升到BigInt,另一个备选方案是使用饱和整数算法,此时最大整数值加一个数时值保持不变,最小整数值减一个数时也是同样的。这就是Matlab™的做法:




\begin{lstlisting}
>> int64(9223372036854775807)

ans =

  9223372036854775807

>> int64(9223372036854775807) + 1

ans =

  9223372036854775807

>> int64(-9223372036854775808)

ans =

 -9223372036854775808

>> int64(-9223372036854775808) - 1

ans =

 -9223372036854775808
\end{lstlisting}



乍一看,这个似乎足够合理,因为9223372036854775807比-9223372036854775808更接近于9223372036854775808并且整数还是以固定大小的自然方式表示的,这与C和Fortran相兼容。但是饱和整数算法是很有问题的。首先最明显的问题是这并不是机器整数算法的工作方式,所以实现饱和整数算法需要生成指令,在每个机器整数运算后检查上溢或者下溢并正确地讲这些结果用\hyperlink{3613894539247233488}{\texttt{typemin(Int)}}或者\hyperlink{17760305803764597758}{\texttt{typemax(Int)}}取代。单单这个就将整数运算从单语句的快速的指令扩展成六个指令,还可能包括分支。哎呦喂{\textasciitilde}{\textasciitilde}但是还有更糟的 — 饱和整数算法并不满足结合律。考虑下列的Matlab计算:




\begin{lstlisting}
>> n = int64(2)^62
4611686018427387904

>> n + (n - 1)
9223372036854775807

>> (n + n) - 1
9223372036854775806
\end{lstlisting}



这就让写很多基础整数算法变得困难因为很多常用技术都是基于有溢出的机器加法\emph{是}满足结合律这一事实的。考虑一下在Julia中求整数值\texttt{lo}和\texttt{hi}之间的中点值,使用表达式\texttt{(lo + hi) >>> 1}:




\begin{minted}{jlcon}
julia> n = 2^62
4611686018427387904

julia> (n + 2n) >>> 1
6917529027641081856
\end{minted}



看到了吗?没有任何问题。那就是2{\textasciicircum}62和2{\textasciicircum}63之间的正确地中点值,虽然\texttt{n + 2n}的值是 -4611686018427387904。现在使用Matlab试一下:




\begin{lstlisting}
>> (n + 2*n)/2

ans =

  4611686018427387904
\end{lstlisting}



哎呦喂。在Matlab中添加\texttt{>>>}运算符没有任何作用,因为在将\texttt{n}与\texttt{2n}相加时已经破坏了能计算出正确地中点值的必要信息,已经出现饱和。



没有结合性不但对于不能依靠像这样的技术的程序员是不幸的,并且让几乎所有的希望优化整数算法的编译器铩羽而归。例如,因为Julia中的整数使用平常的机器整数算法,LLVM就可以自由地激进地优化像\texttt{f(k) = 5k-1}这样的简单地小函数。这个函数的机器码如下所示:




\begin{minted}{jlcon}
julia> code_native(f, Tuple{Int})
  .text
Filename: none
  pushq %rbp
  movq  %rsp, %rbp
Source line: 1
  leaq  -1(%rdi,%rdi,4), %rax
  popq  %rbp
  retq
  nopl  (%rax,%rax)
\end{minted}



这个函数的实际函数体只是一个简单地\texttt{leap}指令,可以立马计算整数乘法与加法。当\texttt{f}内联在其他函数中的时候这个更加有益:




\begin{minted}{jlcon}
julia> function g(k, n)
           for i = 1:n
               k = f(k)
           end
           return k
       end
g (generic function with 1 methods)

julia> code_native(g, Tuple{Int,Int})
  .text
Filename: none
  pushq %rbp
  movq  %rsp, %rbp
Source line: 2
  testq %rsi, %rsi
  jle L26
  nopl  (%rax)
Source line: 3
L16:
  leaq  -1(%rdi,%rdi,4), %rdi
Source line: 2
  decq  %rsi
  jne L16
Source line: 5
L26:
  movq  %rdi, %rax
  popq  %rbp
  retq
  nop
\end{minted}



因为\texttt{f}的调用内联化,循环体就只是简单地\texttt{leap}指令。接着,考虑一下如果循环迭代的次数固定的时候会发生什么:




\begin{minted}{jlcon}
julia> function g(k)
           for i = 1:10
               k = f(k)
           end
           return k
       end
g (generic function with 2 methods)

julia> code_native(g,(Int,))
  .text
Filename: none
  pushq %rbp
  movq  %rsp, %rbp
Source line: 3
  imulq $9765625, %rdi, %rax    # imm = 0x9502F9
  addq  $-2441406, %rax         # imm = 0xFFDABF42
Source line: 5
  popq  %rbp
  retq
  nopw  %cs:(%rax,%rax)
\end{minted}



因为编译器知道整数加法和乘法是满足结合律的并且乘法可以在加法上使用分配律 — 两者在饱和算法中都不成立 — 所以编译器就可以把整个循环优化到只有一个乘法和一个加法。饱和算法完全无法使用这种优化,因为在每个循环迭代中结合律和分配律都会失效导致不同的失效位置会得到不同的结果。编译器可以展开循环,但是不能代数上将多个操作简化到更少的等效操作。



让整数算法静默溢出的最合理的备用方案是所有地方都使用检查算法,当加法、减法和乘法溢出,产生不正确的值时引发错误。在\href{http://danluu.com/integer-overflow/}{blog post}中,Dan Luu分析了这个方案,发现这个方案理论上的性能微不足道,但是最终仍然会消耗大量的性能因为编译器(LLVM和GCC)无法在加法溢出检测处优雅地进行优化。如果未来有所进步我们会考虑在Julia中默认设置为检查整数算法,但是现在,我们需要和溢出可能共同相处。



In the meantime, overflow-safe integer operations can be achieved through the use of external libraries such as \href{https://github.com/JeffreySarnoff/SaferIntegers.jl}{SaferIntegers.jl}. Note that, as stated previously, the use of these libraries significantly increases the execution time of code using the checked integer types. However, for limited usage, this is far less of an issue than if it were used for all integer operations. You can follow the status of the discussion \href{https://github.com/JuliaLang/julia/issues/855}{here}.



\hypertarget{15127813284498705272}{}


\subsection{在远程执行中\texttt{UndefVarError}的可能原因有哪些?}



如同这个错误表述的,远程结点上的\texttt{UndefVarError}的直接原因是变量名的绑定并不存在。让我们探索一下一些可能的原因。




\begin{minted}{jlcon}
julia> module Foo
           foo() = remotecall_fetch(x->x, 2, "Hello")
       end

julia> Foo.foo()
ERROR: On worker 2:
UndefVarError: Foo not defined
Stacktrace:
[...]
\end{minted}



闭包\texttt{x->x}中有\texttt{Foo}的引用,因为\texttt{Foo}在节点2上不存在,所以\texttt{UndefVarError}被扔出。



在模块中而非\texttt{Main}中的全局变量不会在远程节点上按值序列化。只传递了一个引用。新建全局绑定的函数(除了\texttt{Main}中)可能会导致之后扔出\texttt{UndefVarError}。




\begin{minted}{jlcon}
julia> @everywhere module Foo
           function foo()
               global gvar = "Hello"
               remotecall_fetch(()->gvar, 2)
           end
       end

julia> Foo.foo()
ERROR: On worker 2:
UndefVarError: gvar not defined
Stacktrace:
[...]
\end{minted}



在上面的例子中,\texttt{@everywhere module Foo}在所有节点上定义了\texttt{Foo}。但是调用\texttt{Foo.foo()}在本地节点上新建了新的全局绑定\texttt{gvar},但是节点2中并没有找到这个绑定,这会导致\texttt{UndefVarError}错误。



注意着并不适用于在模块\texttt{Main}下新建的全局变量。模块\texttt{Main}下的全局变量会被序列化并且在远程节点的\texttt{Main}下新建新的绑定。




\begin{minted}{jlcon}
julia> gvar_self = "Node1"
"Node1"

julia> remotecall_fetch(()->gvar_self, 2)
"Node1"

julia> remotecall_fetch(varinfo, 2)
name          size summary
––––––––– –––––––– –––––––
Base               Module
Core               Module
Main               Module
gvar_self 13 bytes String
\end{minted}



这并不适用于\texttt{函数}或者\texttt{结构体}声明。但是绑定到全局变量的匿名函数被序列化,如下例所示。




\begin{minted}{jlcon}
julia> bar() = 1
bar (generic function with 1 method)

julia> remotecall_fetch(bar, 2)
ERROR: On worker 2:
UndefVarError: #bar not defined
[...]

julia> anon_bar  = ()->1
(::#21) (generic function with 1 method)

julia> remotecall_fetch(anon_bar, 2)
1
\end{minted}



\hypertarget{6739766638073943445}{}


\subsection{为什么 Julia 使用 \texttt{*} 进行字符串拼接?而不是使用 \texttt{+} 或其他符号?}



使用 \texttt{+}  的\hyperlink{12933998460683957945}{主要依据}是:字符串拼接是不可交换的操作,而 \texttt{+} 通常是一个具有可交换性的操作符。Julia 社区也意识到其他语言使用了不同的操作符,一些用户也可能不熟悉 \texttt{*} 包含的特定代数性值。



注意:你也可以用 \texttt{string(...)} 来拼接字符串和其他能转换成字符串的值; 类似的 \texttt{repeat} 函数可以用于替代用于重复字符串的 \texttt{{\textasciicircum}} 操作符。 \hyperlink{4452850363638134205}{字符串插值语法}在构造字符串时也很常用。



\hypertarget{16669380183019264286}{}


\section{包和模块}



\hypertarget{18070754917834956483}{}


\subsection{{\textquotedbl}using{\textquotedbl}和{\textquotedbl}import{\textquotedbl}的区别是什么?}



只有一个区别,并且在表面上(语法层面)这个区别看来很小。\texttt{using}和\texttt{import}的区别是使用\texttt{using}时你需要写\texttt{function Foo.bar(..}来用一个新方法来扩展模块Foo的函数bar,但是使用\texttt{import Foo.bar}时,你只需要写\texttt{function bar(...},会自动扩展模块Foo的函数bar。



这个区别足够重要以至于提供不同的语法的原因是你不希望意外地扩展一个你根本不知道其存在的函数,因为这很容易造成bug。对于使用像字符串后者整数这样的常用类型的方法最有可能出现这个问题,因为你和其他模块都可能定义了方法来处理这样的常用类型。如果你使用\texttt{import},你会用你自己的新实现覆盖别的函数的\texttt{bar(s::AbstractString)}实现,这会导致做的事情天差地别(并且破坏模块Foo中其他的依赖于调用bar的函数的所有/大部分的将来的使用)。



\hypertarget{16234869582973036611}{}


\section{空值与缺失值}



\hypertarget{7783935872990633567}{}


\subsection{在Julia中{\textquotedbl}null{\textquotedbl},{\textquotedbl}空{\textquotedbl}或者{\textquotedbl}缺失{\textquotedbl}是怎么工作的?}



不像其它很多语言(例如 C 和 Java),Julia 对象默认不能为{\textquotedbl}null{\textquotedbl}。当一个引用(变量,对象域,或者数组元素)没有被初始化,访问它会立即扔出一个错误。这种情况可以使用函数 \hyperlink{11212950246505288748}{\texttt{isdefined}} 或者 \hyperlink{976355747478401147}{\texttt{isassigned}} 检测到。



一些函数只为了其副作用使用,并不需要返回一个值。在这些情况下,约定的是返回 \texttt{nothing} 这个值,这只是 \texttt{Nothing} 类型的一个单例对象。这是一个没有域的一般类型;除了这个约定之外没有任何特殊点,REPL 不会为它打印任何东西。有些语言结构不会有值,也产生 \texttt{nothing},例如 \texttt{if false; end}。



对于类型\texttt{T}的值\texttt{x}只会有时存在的情况,\texttt{Union\{T,Nothing\}}类型可以用作函数参数,对象域和数组元素的类型,与其他语言中的\href{https://en.wikipedia.org/wiki/Nullable\_type}{\texttt{Nullable}, \texttt{Option} or \texttt{Maybe}}相等。如果值本身可以是\texttt{nothing}(显然当\texttt{T}是\texttt{Any}时),\texttt{Union\{Some\{T\}, Nothing\}}类型更加准确因为\texttt{x == nothing}表示值的缺失,\texttt{x == Some(nothing)}表示与\texttt{nothing}相等的值的存在。\hyperlink{12366229165852827603}{\texttt{something}}函数允许使用默认值的展开的\texttt{Some}对象,而非\texttt{nothing}参数。注意在使用\texttt{Union\{T,Nothing\}}参数或者域时编译器能够生成高效的代码。



在统计环境下表示缺失的数据(R 中的 \texttt{NA} 或者 SQL 中的 \texttt{NULL})请使用 \hyperlink{14596725676261444434}{\texttt{missing}} 对象。请参照\hyperlink{5842114294087241506}{\texttt{缺失值}}章节来获取详细信息。



In some languages, the empty tuple (\texttt{()}) is considered the canonical form of nothingness. However, in julia it is best thought of as just a regular tuple that happens to contain zero values.



空(或者{\textquotedbl}底层{\textquotedbl})类型,写作\texttt{Union\{\}}(空的union类型)是没有值和子类型(除了自己)的类型。通常你没有必要用这个类型。



\hypertarget{12500994751877938228}{}


\section{内存}



\hypertarget{2527646600001256276}{}


\subsection{为什么当\texttt{x}和\texttt{y}都是数组时\texttt{x += y}还会申请内存?}



在 Julia 中,\texttt{x += y} 在语法分析中会用 \texttt{x = x + y} 代替。对于数组,结果就是它会申请一个新数组来存储结果,而非把结果存在 \texttt{x} 同一位置的内存上。



这个行为可能会让一些人吃惊,但是这个结果是经过深思熟虑的。主要原因是Julia中的不可变对象,这些对象一旦新建就不能改变他们的值。实际上,数字是不可变对象,语句\texttt{x = 5; x += 1}不会改变\texttt{5}的意义,改变的是与\texttt{x}绑定的值。对于不可变对象,改变其值的唯一方法是重新赋值。



为了稍微详细一点,考虑下列的函数:




\begin{minted}{julia}
function power_by_squaring(x, n::Int)
    ispow2(n) || error("此实现只适用于2的幂")
    while n >= 2
        x *= x
        n >>= 1
    end
    x
end
\end{minted}



在\texttt{x = 5; y = power\_by\_squaring(x, 4)}调用后,你可以得到期望的结果\texttt{x == 5 \&\& y == 625}。然而,现在假设当\texttt{*=}与矩阵一起使用时会改变左边的值,这会有两个问题:



\begin{itemize}
\item 对于普通的方阵,\texttt{A = A*B} 不能在没有临时存储的情况下实现:\texttt{A[1,1]} 会被计算并且在被右边使用完之前存储在左边。


\item 假设你愿意申请一个计算的临时存储(这会消除 \texttt{*=}就地计算的大部分要点);如果你利用了\texttt{x}的可变性, 这个函数会对于可变和不可变的输入有不同的行为。特别地, 对于不可变的\texttt{x},在调用后(通常)你会得到\texttt{y != x},而对可变的\texttt{x},你会有\texttt{y == x}。

\end{itemize}


因为支持范用计算被认为比能使用其他方法完成的潜在的性能优化(比如使用显式循环)更加重要,所以像\texttt{+=}和\texttt{*=}运算符以绑定新值的方式工作。



\hypertarget{11478691918903630142}{}


\section{异步 IO 与并发同步写入}



\hypertarget{6102848140854273508}{}


\subsection{为什么对于同一个流的并发写入会导致相互混合的输出?}



虽然流式 I/O 的 API 是同步的,底层的实现是完全异步的。



思考一下下面的输出:




\begin{minted}{jlcon}
julia> @sync for i in 1:3
           @async write(stdout, string(i), " Foo ", " Bar ")
       end
123 Foo  Foo  Foo  Bar  Bar  Bar
\end{minted}



这是因为,虽然\texttt{write}调用是同步的,每个参数的写入在等待那一部分I/O完成时会生成其他的Tasks。



\texttt{print}和\texttt{println}在调用中会{\textquotedbl}锁定{\textquotedbl}该流。因此把上例中的\texttt{write}改成\texttt{println}会导致:




\begin{minted}{jlcon}
julia> @sync for i in 1:3
           @async println(stdout, string(i), " Foo ", " Bar ")
       end
1 Foo  Bar
2 Foo  Bar
3 Foo  Bar
\end{minted}



你可以使用\texttt{ReentrantLock}来锁定你的写入,就像这样:




\begin{minted}{jlcon}
julia> l = ReentrantLock();

julia> @sync for i in 1:3
           @async begin
               lock(l)
               try
                   write(stdout, string(i), " Foo ", " Bar ")
               finally
                   unlock(l)
               end
           end
       end
1 Foo  Bar 2 Foo  Bar 3 Foo  Bar
\end{minted}



\hypertarget{11524621654961197830}{}


\section{数组}



\hypertarget{8539259979709593447}{}


\subsection{零维数组和标量之间的有什么差别?}



零维数组是\texttt{Array\{T,0\}}形式的数组,它与标量的行为相似,但是有很多重要的不同。这值得一提,因为这是使用数组的范用定义来解释也符合逻辑的特殊情况,虽然最开始看起来有些非直觉。下面一行定义了一个零维数组:




\begin{lstlisting}
julia> A = zeros()
0-dimensional Array{Float64,0}:
0.0
\end{lstlisting}



在这个例子中,\texttt{A}是一个含有一个元素的可变容器,这个元素可以通过\texttt{A[] = 1.0}来设置,通过\texttt{A[]}来读取。所有的零维数组都有同样的大小(\texttt{size(A) == ()})和长度(\texttt{length(A) == 1})。特别地,零维数组不是空数组。如果你觉得这个非直觉,这里有些想法可以帮助理解Julia的这个定义。



\begin{itemize}
\item 类比的话,零维数组是{\textquotedbl}点{\textquotedbl},向量是{\textquotedbl}线{\textquotedbl}而矩阵 是{\textquotedbl}面{\textquotedbl}。就像线没有面积一样(但是也能代表事物的一个集合), 点没有长度和任意一个维度(但是也能表示一个事物)。


\item 我们定义\texttt{prod(())}为1,一个数组中的所有的元素个数是 大小的乘积。零维数组的大小为\texttt{()},所以 它的长度为\texttt{1}。


\item 零维数组原生没有任何你可以索引的维度 – 它们仅仅是\texttt{A[]}。我们可以给它们应用同样的{\textquotedbl}trailing one{\textquotedbl}规则, as for all other array dimensionalities, so you can indeed index them as \texttt{A[1]}, \texttt{A[1,1]}, etc; see \hyperlink{16741454967402507490}{Omitted and extra indices}.

\end{itemize}


理解它与普通的标量之间的区别也很重要。标量不是一个可变的容器(尽管它们是可迭代的,可以定义像\texttt{length},\texttt{getindex}这样的东西,\emph{例如}\texttt{1[] == 1})。特别地,如果\texttt{x = 0.0}是以一个标量来定义,尝试通过\texttt{x[] = 1.0}来改变它的值会报错。标量\texttt{x}能够通过\texttt{fill(x)}转化成包含它的零维数组,并且相对地,一个零维数组\texttt{a}可以通过\texttt{a[]}转化成其包含的标量。另外一个区别是标量可以参与到线性代数运算中,比如\texttt{2 * rand(2,2)},但是零维数组的相似操作\texttt{fill(2) * rand(2,2)}会报错。



\hypertarget{2217516185420134366}{}


\subsection{Why are my Julia benchmarks for linear algebra operations different from other languages?}



You may find that simple benchmarks of linear algebra building blocks like




\begin{minted}{julia}
using BenchmarkTools
A = randn(1000, 1000)
B = randn(1000, 1000)
@btime $A \ $B
@btime $A * $B
\end{minted}



can be different when compared to other languages like Matlab or R.



Since operations like this are very thin wrappers over the relevant BLAS functions, the reason for the discrepancy is very likely to be



\begin{itemize}
\item[1. ] the BLAS library each language is using,


\item[2. ] the number of concurrent threads.

\end{itemize}


Julia compiles and uses its own copy of OpenBLAS, with threads currently capped at \texttt{8} (or the number of your cores).



Modifying OpenBLAS settings or compiling Julia with a different BLAS library, eg \href{https://software.intel.com/en-us/mkl}{Intel MKL}, may provide performance improvements. You can use \href{https://github.com/JuliaComputing/MKL.jl}{MKL.jl}, a package that makes Julia{\textquotesingle}s linear algebra use Intel MKL BLAS and LAPACK instead of OpenBLAS, or search the discussion forum for suggestions on how to set this up manually. Note that Intel MKL cannot be bundled with Julia, as it is not open source.



\hypertarget{13975497382718262394}{}


\section{Julia 版本发布}



\hypertarget{15044504752514653644}{}


\subsection{Do I want to use the Stable, LTS, or nightly version of Julia?}



The Stable version of Julia is the latest released version of Julia, this is the version most people will want to run. It has the latest features, including improved performance. The Stable version of Julia is versioned according to \href{https://semver.org/}{SemVer} as v1.x.y. A new minor release of Julia corresponding to a new Stable version is made approximately every 4-5 months after a few weeks of testing as a release candidate. Unlike the LTS version the a Stable version will not normally recieve bugfixes after another Stable version of Julia has been released. However, upgrading to the next Stable release will always be possible as each release of Julia v1.x will continue to run code written for earlier versions.



You may prefer the LTS (Long Term Support) version of Julia if you are looking for a very stable code base. The current LTS version of Julia is versioned according to SemVer as v1.0.x; this branch will continue to recieve bugfixes until a new LTS branch is chosen, at which point the v1.0.x series will no longer recieved regular bug fixes and all but the most conservative users will be advised to upgrade to the new LTS version series. As a package developer, you may prefer to develop for the LTS version, to maximize the number of users who can use your package. As per SemVer, code written for v1.0 will continue to work for all future LTS and Stable versions. In general, even if targetting the LTS, one can develop and run code in the latest Stable version, to take advantage of the improved performance; so long as one avoids using new features (such as added library functions or new methods).



You may prefer the nightly version of Julia if you want to take advantage of the latest updates to the language, and don{\textquotesingle}t mind if the version available today occasionally doesn{\textquotesingle}t actually work. As the name implies, releases to the nightly version are made roughly every night (depending on build infrastructure stability). In general nightly released are fairly safe to use—your code will not catch on fire. However, they may be occasional regressions and or issues that will not be found until more thorough pre-release testing. You may wish to test against the nightly version to ensure that such regressions that affect your use case are caught before a release is made.



Finally, you may also consider building Julia from source for yourself. This option is mainly for those individuals who are comfortable at the command line, or interested in learning. If this describes you, you may also be interested in reading our \href{https://github.com/JuliaLang/julia/blob/master/CONTRIBUTING.md}{guidelines for contributing}.



可以在\href{https://julialang.org/downloads/}{https://julialang.org/downloads/}的下载页面上找到每种下载类型的链接。 请注意,并非所有版本的Julia都适用于所有平台。



\hypertarget{8132004312407744699}{}


\chapter{与其他语言的显著差异}



\hypertarget{17394269940611152566}{}


\section{与 MATLAB 的显著差异}



虽然 MATLAB 用户可能会发现 Julia 的语法很熟悉,但 Julia 不是 MATLAB 的克隆。 它们之间存在重大的语法和功能差异。 以下是一些可能会使习惯于 MATLAB 的 Julia 用户感到困扰的显著差异:



\begin{itemize}
\item Julia 数组使用方括号 \texttt{A[i,j]} 进行索引。


\item Julia 的数组在赋值给另一个变量时不发生复制。执行 \texttt{A = B} 后,改变 \texttt{B} 中元素也会修改 \texttt{A}。


\item Julia 的值在向函数传递时不发生复制。如果某个函数修改了数组,这一修改对调用者是可见的。


\item Julia 不会在赋值语句中自动增长数组。 而在 MATLAB 中 \texttt{a(4) = 3.2} 可以创建数组 \texttt{a = [0 0 0 3.2]},而 \texttt{a(5) = 7} 可以将它增长为 \texttt{a = [0 0 0 3.2 7]}。如果 \texttt{a} 的长度小于 5 或者这个语句是第一次使用标识符 \texttt{a},则相应的 Julia 语句 \texttt{a[5] = 7} 会抛出错误。Julia 使用 \hyperlink{18026893834387542681}{\texttt{push!}} 和 \hyperlink{2587432243763606566}{\texttt{append!}} 来增长 \texttt{Vector},它们比 MATLAB 的 \texttt{a(end+1) = val} 更高效。


\item 虚数单位 \texttt{sqrt(-1)} 在 Julia 中表示为 \hyperlink{15097910740298861288}{\texttt{im}},而不是在 MATLAB 中的 \texttt{i} 或 \texttt{j}。


\item 在 Julia 中,没有小数点的数字字面量(例如 \texttt{42})会创建整数而不是浮点数。也支持任意大整数字面量。因此,某些操作(如 \texttt{2{\textasciicircum}-1})将抛出 domain error,因为结果不是整数(有关的详细信息,请参阅\hyperlink{1677964623674152967}{常见问题中有关 domain errors 的条目})。 point numbers. As a result, some operations can throw a domain error if they expect a float; for example, \texttt{julia> a = -1; 2{\textasciicircum}a} throws a domain error, as the result is not an integer (see \hyperlink{1677964623674152967}{the FAQ entry on domain errors} for details).


\item 在 Julia 中,能返回多个值并将其赋值为元组,例如 \texttt{(a, b) = (1, 2)} 或 \texttt{a, b = 1, 2}。 在 Julia 中不存在 MATLAB 的 \texttt{nargout},它通常在 MATLAB 中用于根据返回值的数量执行可选工作。取而代之的是,用户可以使用可选参数和关键字参数来实现类似的功能。


\item Julia 拥有真正的一维数组。列向量的大小为 \texttt{N},而不是 \texttt{Nx1}。例如,\hyperlink{7668863842145012694}{\texttt{rand(N)}} 创建一个一维数组。


\item 在 Julia 中,\texttt{[x,y,z]} 将始终构造一个包含\texttt{x}、\texttt{y} 和 \texttt{z} 的 3 元数组。

\begin{itemize}
\item 要在第一个维度(「垂直列」)中连接元素,请使用 \hyperlink{14691815416955507876}{\texttt{vcat(x,y,z)}} 或用分号分隔(\texttt{[x; y; z]})。


\item 要在第二个维度(「水平行」)中连接元素,请使用 \hyperlink{8862791894748483563}{\texttt{hcat(x,y,z)}} 或用空格分隔(\texttt{[x y z]})。


\item 要构造分块矩阵(在前两个维度中连接元素),请使用 \hyperlink{16279083053557795116}{\texttt{hvcat}} 或组合空格和分号(\texttt{[a b; c d]})。

\end{itemize}

\item 在 Julia 中,\texttt{a:b} 和 \texttt{a:b:c} 构造 \texttt{AbstractRange} 对象。使用 \hyperlink{6278865767444641812}{\texttt{collect(a:b)}} 构造一个类似 MATLAB 中完整的向量。通常,不需要调用 \texttt{collect}。在大多数情况下,\texttt{AbstractRange} 对象将像普通数组一样运行,但效率更高,因为它是懒惰求值。这种创建专用对象而不是完整数组的模式经常被使用,并且也可以在诸如 \hyperlink{737600656772861535}{\texttt{range}} 之类的函数中看到,或者在诸如 \texttt{enumerate} 和 \texttt{zip} 之类的迭代器中看到。特殊对象大多可以像正常数组一样使用。


\item Julia 中的函数返回其最后一个表达式或 \texttt{return} 关键字的值而无需在函数定义中列出要返回的变量的名称(有关详细信息,请参阅 \hyperlink{16317991580998959177}{return 关键字})。


\item Julia 脚本可以包含任意数量的函数,并且在加载文件时,所有定义都将在外部可见。可以从当前工作目录之外的文件加载函数定义。


\item 在 Julia 中,例如 \hyperlink{8666686648688281595}{\texttt{sum}}、\hyperlink{13484084847910116333}{\texttt{prod}} 和 \hyperlink{7839419811914289844}{\texttt{max}} 的归约操作会作用到数组的每一个元素上,当调用时只有一个函数,例如 \texttt{sum(A)},即使 \texttt{A} 并不只有一个维度。


\item 在 Julia 中,调用无参数的函数时必须使用小括号,例如 \hyperlink{7668863842145012694}{\texttt{rand()}}。


\item Julia 不鼓励使用分号来结束语句。语句的结果不会自动打印(除了在 REPL 中),并且代码的一行不必使用分号结尾。\hyperlink{783803254548423222}{\texttt{println}} 或者 \hyperlink{13954719910189591998}{\texttt{@printf}} 能用来打印特定输出。


\item 在 Julia 中,如果 \texttt{A} 和 \texttt{B} 是数组,像 \texttt{A == B} 这样的逻辑比较运算符不会返回布尔值数组。相反地,请使用 \texttt{A .== B}。对于其他的像是 \hyperlink{702782232449268329}{\texttt{<}}、\hyperlink{8677991761303191103}{\texttt{>}} 的布尔运算符同理。


\item 在 Julia 中,运算符\hyperlink{1494761116451616317}{\texttt{\&}}、\hyperlink{9633687763646488853}{\texttt{|}} 和 \hyperlink{7071880015536674935}{\texttt{\unicodeveebar{}}}(\hyperlink{7071880015536674935}{\texttt{xor}})进行按位操作,分别与MATLAB中的\texttt{and}、\texttt{or} 和 \texttt{xor} 等价,并且优先级与 Python 的按位运算符相似(不像 C)。他们可以对标量运算或者数组中逐元素运算,可以用来合并逻辑数组,但是注意运算顺序的区别:括号可能是必要的(例如,选择 \texttt{A} 中等于 1 或 2 的元素可使用 \texttt{(A .== 1) .| (A .== 2)})。


\item 在 Julia 中,集合的元素可以使用 splat 运算符 \texttt{...} 来作为参数传递给函数,如 \texttt{xs=[1,2]; f(xs...)}。


\item Julia 的 \hyperlink{6661056220970412040}{\texttt{svd}} 将奇异值作为向量而非密集对角矩阵返回。


\item 在 Julia 中,\texttt{...} 不用于延续代码行。不同的是,Julia 中不完整的表达式会自动延续到下一行。


\item 在 Julia 和 MATLAB 中,变量 \texttt{ans} 被设置为交互式会话中提交的最后一个表达式的值。在 Julia 中与 MATLAB 不同的是,当 Julia 代码以非交互式模式运行时并不会设置 \texttt{ans}。


\item Julia 的 \texttt{struct} 不支持在运行时动态地添加字段,这与 MATLAB 的 \texttt{class} 不同。如需支持,请使用 \hyperlink{3089397136845322041}{\texttt{Dict}}。


\item 在 Julia 中,每个模块有自身的全局作用域/命名空间,而在 MATLAB 中只有一个全局作用域。


\item 在 MATLAB 中,删除不需要的值的惯用方法是使用逻辑索引,如表达式 \texttt{x(x>3)} 或语句 \texttt{x(x>3) = []} 来 in-place 修改 \texttt{x}。相比之下,Julia 提供了更高阶的函数 \hyperlink{11445961893478569145}{\texttt{filter}} 和 \hyperlink{3384092630307389071}{\texttt{filter!}},允许用户编写 \texttt{filter(z->z>3, x)} 和 \texttt{filter!(z->z>3, x)} 来代替相应直译 \texttt{x[x.>3]} 和 \texttt{x = x[x.>3]}。使用 \hyperlink{3384092630307389071}{\texttt{filter!}} 可以减少临时数组的使用。


\item 类似于提取(或「解引用」)元胞数组的所有元素的操作,例如 MATLAB 中的 \texttt{vertcat(A\{:\})},在 Julia 中是使用 splat 运算符编写的,例如 \texttt{vcat(A...)}。


\item In Julia, the \texttt{adjoint} function performs conjugate transposition; in MATLAB, \texttt{adjoint} provides the {\textquotedbl}adjugate{\textquotedbl} or classical adjoint, which is the transpose of the matrix of cofactors.

\end{itemize}


\hypertarget{8033006673941157229}{}


\section{与 R 的显著差异}



Julia 的目标之一是为数据分析和统计编程提供高效的语言。对于从 R 转到 Julia 的用户来说,这是一些显著差异:



\begin{itemize}
\item Julia 的单引号封闭字符,而不是字符串。


\item Julia 可以通过索引字符串来创建子字符串。在 R 中,在创建子字符串之前必须将字符串转换为字符向量。


\item 在 Julia 中,与 Python 相同但与 R 不同的是,字符串可由三重引号 \texttt{{\textquotedbl}{\textquotedbl}{\textquotedbl} ... {\textquotedbl}{\textquotedbl}{\textquotedbl}} 创建。此语法对于构造包含换行符的字符串很方便。


\item 在 Julia 中,可变参数使用 splat 运算符 \texttt{...} 指定,该运算符总是跟在具体变量的名称后面,与 R 的不同,R 的 \texttt{...} 可以单独出现。


\item 在 Julia 中,模数是 \texttt{mod(a, b)},而不是 \texttt{a \%\% b}。Julia 中的 \texttt{\%} 是余数运算符。


\item 在 Julia 中,并非所有数据结构都支持逻辑索引。此外,Julia 中的逻辑索引只支持长度等于被索引对象的向量。例如:

\begin{itemize}
\item 在 R 中,\texttt{c(1, 2, 3, 4)[c(TRUE, FALSE)]} 等价于 \texttt{c(1, 3)}。


\item 在 R 中,\texttt{c(1, 2, 3, 4)[c(TRUE, FALSE, TRUE, FALSE)]} 等价于 \texttt{c(1, 3)}。


\item 在 Julia 中,\texttt{[1, 2, 3, 4][[true, false]]} 抛出 \hyperlink{9731558909100893938}{\texttt{BoundsError}}。


\item 在 Julia 中,\texttt{[1, 2, 3, 4][[true, false, true, false]]} 产生 \texttt{[1, 3]}。

\end{itemize}

\item 与许多语言一样,Julia 并不总是允许对不同长度的向量进行操作,与 R 不同,R 中的向量只需要共享一个公共的索引范围。例如,\texttt{c(1, 2, 3, 4) + c(1, 2)} 是有效的 R,但等价的 \texttt{[1, 2, 3, 4] + [1, 2]} 在 Julia 中会抛出一个错误。


\item 在逗号不改变代码含义时,Julia 允许使用可选的尾随括号。在索引数组时,这可能在 R 用户间造成混淆。例如,R 中的 \texttt{x[1,]} 将返回矩阵的第一行;但是,在 Julia 中,引号被忽略,于是 \texttt{x[1,] == x[1]},并且将返回第一个元素。要提取一行,请务必使用 \texttt{:},如 \texttt{x[1,:]}。


\item Julia 的 \hyperlink{11483231213869150535}{\texttt{map}} 首先接受函数,然后是该函数的参数,这与 R 中的 \texttt{lapply(<structure>, function, ...)} 不同。类似地,R 中的 \texttt{apply(X, MARGIN, FUN, ...)} 等价于 Julia 的 \hyperlink{8678396932318499078}{\texttt{mapslices}},其中函数是第一个参数。


\item R 中的多变量 apply,如 \texttt{mapply(choose, 11:13, 1:3)},在 Julia 中可以编写成 \texttt{broadcast(binomial, 11:13, 1:3)}。等价地,Julia 提供了更短的点语法来向量化函数 \texttt{binomial.(11:13, 1:3)}。


\item Julia 使用 \texttt{end} 来表示条件块(如 \texttt{if})、循环块(如 \texttt{while}/\texttt{for})和函数的结束。为了代替单行 \texttt{if ( cond ) statement},Julia 允许形式为 \texttt{if cond; statement; end}、\texttt{cond \&\& statement} 和 \texttt{!cond || statement} 的语句。后两种语法中的赋值语句必须显式地包含在括号中,例如 \texttt{cond \&\& (x = value)},这是因为运算符的优先级。


\item 在 Julia 中,\texttt{<-}, \texttt{<<-} 和 \texttt{->} 不是赋值运算符。


\item Julia 的 \texttt{->} 创建一个匿名函数。


\item Julia 使用括号构造向量。Julia 的 \texttt{[1, 2, 3]} 等价于 R 的 \texttt{c(1, 2, 3)}。


\item Julia 的 \hyperlink{7592762607639177347}{\texttt{*}} 运算符可以执行矩阵乘法,这与 R 不同。如果 \texttt{A} 和 \texttt{B} 都是矩阵,那么 \texttt{A * B} 在 Julia 中表示矩阵乘法,等价于 R 的 \texttt{A \%*\% B}。在 R 中,相同的符号将执行逐元素(Hadamard)乘积。要在 Julia 中使用逐元素乘法运算,你需要编写 \texttt{A .* B}。


\item Julia 使用 \texttt{transpose} 函数来执行矩阵转置,使用 \texttt{{\textquotesingle}} 运算符或 \texttt{adjoint} 函数来执行共轭转置。因此,Julia 的 \texttt{transpose(A)} 等价于 R 的 \texttt{t(A)}。另外,Julia 中的非递归转置由 \texttt{permutedims} 函数提供。


\item Julia 在编写 \texttt{if} 语句或 \texttt{for}/\texttt{while} 循环时不需要括号:请使用 \texttt{for i in [1, 2, 3]} 代替 \texttt{for (int i=1; i <= 3; i++)},以及 \texttt{if i == 1} 代替 \texttt{if (i == 1)}


\item Julia 不把数字 \texttt{0} 和 \texttt{1} 视为布尔值。在 Julia 中不能编写 \texttt{if (1)},因为 \texttt{if} 语句只接受布尔值。相反,可以编写 \texttt{if true}、\texttt{if Bool(1)} 或 \texttt{if 1==1}。


\item Julia 不提供 \texttt{nrow} 和 \texttt{ncol}。相反,请使用 \texttt{size(M, 1)} 代替 \texttt{nrow(M)} 以及 \texttt{size(M, 2)} 代替 \texttt{ncol(M)}


\item Julia 仔细区分了标量、向量和矩阵。在 R 中,\texttt{1} 和 \texttt{c(1)} 是相同的。在 Julia 中,它们不能互换地使用。


\item Julia 的 \hyperlink{17079356950356685026}{\texttt{diag}} 和 \hyperlink{18133091318829836689}{\texttt{diagm}} 与 R 的不同。


\item Julia 赋值操作的左侧不能为函数调用的结果:你不能编写 \texttt{diag(M) = fill(1, n)}。


\item Julia 不鼓励使用函数填充主命名空间。Julia 的大多数统计功能都可在 \href{https://github.com/JuliaStats}{JuliaStats 组织}的\href{https://pkg.julialang.org/}{包}中找到。例如:

\begin{itemize}
\item 与概率分布相关的函数由 \href{https://github.com/JuliaStats/Distributions.jl}{Distributions 包}提供。


\item \href{https://github.com/JuliaData/DataFrames.jl}{DataFrames 包}提供数据帧。


\item 广义线性模型由 \href{https://github.com/JuliaStats/GLM.jl}{GLM 包}提供。

\end{itemize}

\item Julia 提供了元组和真正的哈希表,但不提供 R 风格的列表。在返回多个项时,通常应使用元组或具名元组:请使用 \texttt{(1, 2)} 或 \texttt{(a=1, b=2)} 代替 \texttt{list(a = 1, b = 2)}。


\item Julia 鼓励用户编写自己的类型,它比 R 中的 S3 或 S4 对象更容易使用。Julia 的多重派发系统意味着 \texttt{table(x::TypeA)} 和 \texttt{table(x::TypeB)} 类似于 R 的 \texttt{table.TypeA(x)} 和 \texttt{table.TypeB(x)}。


\item Julia 的值在向函数传递时不发生复制。如果某个函数修改了数组,这一修改对调用者是可见的。这与 R 非常不同,允许新函数更高效地操作大型数据结构。


\item 在 Julia 中,向量和矩阵使用 \hyperlink{8862791894748483563}{\texttt{hcat}}、\hyperlink{14691815416955507876}{\texttt{vcat}} 和 \hyperlink{16279083053557795116}{\texttt{hvcat}} 拼接,而不是像在 R 中那样使用 \texttt{c}、\texttt{rbind} 和 \texttt{cbind}。


\item 在 Julia 中,像 \texttt{a:b} 这样的 range 不是 R 中的向量简写,而是一个专门的 \texttt{AbstractRange} 对象,该对象用于没有高内存开销地进行迭代。要将 range 转换为 vector,请使用 \hyperlink{6278865767444641812}{\texttt{collect(a:b)}}。


\item Julia 的 \hyperlink{7839419811914289844}{\texttt{max}} 和 \hyperlink{7458766354532817148}{\texttt{min}} 分别等价于 R 中的 \texttt{pmax} 和 \texttt{pmin},但两者的参数都需要具有相同的维度。虽然 \hyperlink{14719513931696680717}{\texttt{maximum}} 和 \hyperlink{13126064576294034099}{\texttt{minimum}} 代替了 R 中的 \texttt{max} 和 \texttt{min},但它们之间有重大区别。


\item Julia 的 \hyperlink{8666686648688281595}{\texttt{sum}}、\hyperlink{13484084847910116333}{\texttt{prod}}、\hyperlink{14719513931696680717}{\texttt{maximum}} 和 \hyperlink{13126064576294034099}{\texttt{minimum}} 与它们在 R 中的对应物不同。它们都接受一个可选的关键字参数 \texttt{dims},它表示执行操作的维度。例如,在 Julia 中令 \texttt{A = [1 2; 3 4]},在 R 中令 \texttt{B <- rbind(c(1,2),c(3,4))} 是与之相同的矩阵。然后 \texttt{sum(A)} 得到与 \texttt{sum(B)} 相同的结果,但 \texttt{sum(A, dims=1)} 是一个包含每一列总和的行向量,\texttt{sum(A, dims=2)} 是一个包含每一行总和的列向量。这与 R 的行为形成了对比,在 R 中,单独的 \texttt{colSums(B)} 和 \texttt{rowSums(B)} 提供了这些功能。如果 \texttt{dims} 关键字参数是向量,则它指定执行求和的所有维度,并同时保持待求和数组的维数,例如 \texttt{sum(A, dims=(1,2)) == hcat(10)}。应该注意的是,没有针对第二个参数的错误检查。


\item Julia 具有一些可以改变其参数的函数。例如,它具有 \hyperlink{8473525809131227606}{\texttt{sort}} 和 \hyperlink{12296873681374954808}{\texttt{sort!}}。


\item 在 R 中,高性能需要向量化。在 Julia 中,这几乎恰恰相反:性能最高的代码通常通过去向量化的循环来实现。


\item Julia 是立即求值的,不支持 R 风格的惰性求值。对于大多数用户来说,这意味着很少有未引用的表达式或列名。


\item Julia 不支持 \texttt{NULL} 类型。最接近的等价物是 \hyperlink{9331422207248206047}{\texttt{nothing}},但它的行为类似于标量值而不是列表。请使用 \texttt{x === nothing} 代替 \texttt{is.null(x)}。


\item 在 Julia 中,缺失值由 \hyperlink{14596725676261444434}{\texttt{missing}} 表示,而不是由 \texttt{NA} 表示。请使用 \hyperlink{3452327148507948899}{\texttt{ismissing(x)}}(或者在向量上使用逐元素操作 \texttt{ismissing.(x)})代替 \texttt{isna(x)}。通常使用 \hyperlink{2012470681884771400}{\texttt{skipmissing}} 代替 \texttt{na.rm=TRUE}(尽管在某些特定情况下函数接受 \texttt{skipmissing} 参数)。


\item Julia 缺少 R 中的 \texttt{assign} 或 \texttt{get} 的等价物。


\item 在 Julia 中,\texttt{return} 不需要括号。


\item 在 R 中,删除不需要的值的惯用方法是使用逻辑索引,如表达式 \texttt{x[x>3]} 或语句 \texttt{x = x[x>3]} 来 in-place 修改 \texttt{x}。相比之下,Julia 提供了更高阶的函数 \hyperlink{11445961893478569145}{\texttt{filter}} 和 \hyperlink{3384092630307389071}{\texttt{filter!}},允许用户编写 \texttt{filter(z->z>3, x)} 和 \texttt{filter!(z->z>3, x)} 来代替相应直译 \texttt{x[x.>3]} 和 \texttt{x = x[x.>3]}。使用 \hyperlink{3384092630307389071}{\texttt{filter!}} 可以减少临时数组的使用。

\end{itemize}


\hypertarget{13471211736636773349}{}


\section{与 Python 的显著差异}



\begin{itemize}
\item Julia 的 \texttt{for}、\texttt{if}、\texttt{while}等代码块由\texttt{end}关键字终止。缩进级别并不像在 Python 中那么重要。 is not significant as it is in Python. Unlike Python, Julia has no \texttt{pass} keyword.


\item Strings are denoted by double quotation marks (\texttt{{\textquotedbl}text{\textquotedbl}}) in Julia (with three double quotation marks for multi-line strings), whereas in Python they can be denoted either by single (\texttt{{\textquotesingle}text{\textquotesingle}}) or double quotation marks (\texttt{{\textquotedbl}text{\textquotedbl}}). Single quotation marks are used for characters in Julia (\texttt{{\textquotesingle}c{\textquotesingle}}).


\item String concatenation is done with \texttt{*} in Julia, not \texttt{+} like in Python. Analogously, string repetition is done with \texttt{{\textasciicircum}}, not \texttt{*}. Implicit string concatenation of string literals like in Python (e.g. \texttt{{\textquotesingle}ab{\textquotesingle} {\textquotesingle}cd{\textquotesingle} == {\textquotesingle}abcd{\textquotesingle}}) is not done in Julia.


\item Python Lists—flexible but slow—correspond to the Julia \texttt{Vector\{Any\}} type or more generally \texttt{Vector\{T\}} where \texttt{T} is some non-concrete element type. {\textquotedbl}Fast{\textquotedbl} arrays like Numpy arrays that store elements in-place (i.e., \texttt{dtype} is \texttt{np.float64}, \texttt{[({\textquotesingle}f1{\textquotesingle}, np.uint64), ({\textquotesingle}f2{\textquotesingle}, np.int32)]}, etc.) can be represented by \texttt{Array\{T\}} where \texttt{T} is a concrete, immutable element type. This includes built-in types like \texttt{Float64}, \texttt{Int32}, \texttt{Int64} but also more complex types like \texttt{Tuple\{UInt64,Float64\}} and many user-defined types as well.


\item 在 Julia 中,数组、字符串等的索引从 1 开始,而不是从 0 开始。


\item Julia 的切片索引包含最后一个元素,这与 Python 不同。Julia 中的 \texttt{a[2:3]} 就是 Python 中的 \texttt{a[1:3]}。


\item Julia 不支持负数索引。特别地,列表或数组的最后一个元素在 Julia 中使用 \texttt{end} 索引,而不像在 Python 中使用 \texttt{-1}。


\item Julia requires \texttt{end} for indexing until the last element. \texttt{x[1:]} in Python is equivalent to \texttt{x[2:end]} in Julia.


\item Julia{\textquotesingle}s range indexing has the format of \texttt{x[start:step:stop]}, whereas Python{\textquotesingle}s format is \texttt{x[start:(stop+1):step]}. Hence, \texttt{x[0:10:2]} in Python is equivalent to \texttt{x[1:2:10]} in Julia. Similarly, \texttt{x[::-1]} in Python, which refers to the reversed array, is equivalent to \texttt{x[end:-1:1]} in Julia.


\item In Julia, indexing a matrix with arrays like \texttt{X[[1,2], [1,3]]} refers to a sub-matrix that contains the intersections of the first and second rows with the first and third columns. In Python, \texttt{X[[1,2], [1,3]]} refers to a vector that contains the values of cell \texttt{[1,1]} and \texttt{[2,3]} in the matrix. \texttt{X[[1,2], [1,3]]} in Julia is equivalent with \texttt{X[np.ix\_([0,1],[0,2])]} in Python. \texttt{X[[0,1], [0,2]]} in Python is equivalent with \texttt{X[[CartesianIndex(1,1), CartesianIndex(2,3)]]} in Julia.


\item Julia 没有用来续行的语法:如果在行的末尾,到目前为止的输入是一个完整的表达式,则认为其已经结束;否则,认为输入继续。强制表达式继续的一种方式是将其包含在括号中。


\item 默认情况下,Julia 数组是列优先的(Fortran 顺序),而 NumPy 数组是行优先(C 顺序)。为了在循环数组时获得最佳性能,循环顺序应该在 Julia 中相对于 NumPy 反转(请参阅 \hyperlink{818954303942149020}{Performance Tips} 中的对应章节)。

be reversed in Julia relative to NumPy (see \hyperlink{11239800376478112527}{relevant section of Performance Tips}).


\item Julia 的更新运算符(例如 \texttt{+=},\texttt{-=},···)是 \emph{not in-place},而 Numpy 的是。这意味着 \texttt{A = [1, 1]; B = A; B += [3, 3]} 不会改变 \texttt{A} 中的值,而将名称 \texttt{B} 重新绑定到右侧表达式 \texttt{B = B + 3} 的结果,这是一个新的数组。对于 in-place 操作,使用 \texttt{B .+= 3}(另请参阅 \hyperlink{15967322336376951940}{dot operators})、显式的循环或者 \texttt{InplaceOps.jl}。


\item 每次调用方法时,Julia 都会计算函数参数的默认值,不像在 Python 中,默认值只会在函数定义时被计算一次。例如,每次无输入参数调用时,函数\texttt{f(x=rand()) = x}都返回一个新的随机数在另一方面,函数 \texttt{g(x=[1,2]) = push!(x,3)} 在每次以 \texttt{g()} 调用时返回 \texttt{[1,2,3]}。


\item 在 Julia 中,\texttt{\%} 是余数运算符,而在 Python 中是模运算符。


\item In Julia, the commonly used \texttt{Int} type corresponds to the machine integer type (\texttt{Int32} or \texttt{Int64}), unlike in Python, where \texttt{int} is an arbitrary length integer. This means in Julia the \texttt{Int} type will overflow, such that \texttt{2{\textasciicircum}64 == 0}. If you need larger values use another appropriate type, such as \texttt{Int128}, \hyperlink{423405808990690832}{\texttt{BigInt}} or a floating point type like \texttt{Float64}.


\item The imaginary unit \texttt{sqrt(-1)} is represented in Julia as \texttt{im}, not \texttt{j} as in Python.


\item In Julia, the exponentiation operator is \texttt{{\textasciicircum}}, not \texttt{**} as in Python.


\item Julia uses \texttt{nothing} of type \texttt{Nothing} to represent a null value, whereas Python uses \texttt{None} of type \texttt{NoneType}.


\item In Julia, the standard operators over a matrix type are matrix operations, whereas, in Python, the standard operators are element-wise operations. When both \texttt{A} and \texttt{B} are matrices, \texttt{A * B} in Julia performs matrix multiplication, not element-wise multiplication as in Python. \texttt{A * B} in Julia is equivalent with \texttt{A @ B} in Python, whereas \texttt{A * B} in Python is equivalent with \texttt{A .* B} in Julia.


\item The adjoint operator \texttt{{\textquotesingle}} in Julia returns an adjoint of a vector (a lazy representation of row vector), whereas the transpose operator \texttt{.T} over a vector in Python returns the original vector (non-op).


\item In Julia, a function may contain multiple concrete implementations (called \emph{Methods}), selected via multiple dispatch, whereas functions in Python have a single implementation (no polymorphism).


\item There are no classes in Julia. Instead they are structures (mutable or immutable), containing data but no methods.


\item Calling a method of a class in Python (\texttt{a = MyClass(x), x.func(y)}) corresponds to a function call in Julia, e.g. \texttt{a = MyStruct(x), func(x::MyStruct, y)}. In general, multiple dispatch is more flexible and powerful than the Python class system.


\item Julia structures may have exactly one abstract supertype, whereas Python classes can inherit from one or more (abstract or concrete) superclasses.


\item The logical Julia program structure (Packages and Modules) is independent of the file strucutre (\texttt{include} for additional files), whereas the Python code structure is defined by directories (Packages) and files (Modules).


\item The ternary operator \texttt{x > 0 ? 1 : -1} in Julia corresponds to conditional expression in Python \texttt{1 if x > 0 else -1}.


\item In Julia the \texttt{@} symbol refers to a macro, whereas in Python it refers to a decorator.


\item Exception handling in Julia is done using \texttt{try} — \texttt{catch} — \texttt{finally}, instead of \texttt{try} — \texttt{except} — \texttt{finally}. In contrast to Python, it is not recommended to use exception handling as part of the normal workflow in Julia due to performance reasons.


\item In Julia loops are fast, there is no need to write {\textquotedbl}vectorized{\textquotedbl} code for performance reasons.


\item Be careful with non-constant global variables in Julia, especially in tight loops. Since you can write close-to-metal code in Julia (unlike Python), the effect of globals can be drastic (see \hyperlink{818954303942149020}{Performance Tips}).


\item In Python, the majority of values can be used in logical contexts (e.g. \texttt{if {\textquotedbl}a{\textquotedbl}:} means the following block is executed, and \texttt{if {\textquotedbl}{\textquotedbl}:} means it is not). In Julia, you need explicit conversion to \texttt{Bool} (e.g. \texttt{if {\textquotedbl}a{\textquotedbl}} throws an exception). If you want to test for a non-empty string in Julia, you would explicitly write \texttt{if !isempty({\textquotedbl}{\textquotedbl})}.


\item In Julia, a new local scope is introduced by most code blocks, including loops and \texttt{try} — \texttt{catch} — \texttt{finally}. Note that comprehensions (list, generator, etc.) introduce a new local scope both in Python and Julia, whereas \texttt{if} blocks do not introduce a new local scope in both languages.

\end{itemize}


\hypertarget{17929505884596702354}{}


\section{与 C/C++ 的显著差异}



\begin{itemize}
\item Julia 的数组由方括号索引,方括号中可以包含不止一个维度 \texttt{A[i,j]}。这样的语法不仅仅是像 C/C++ 中那样对指针或者地址引用的语法糖,参见关于数组构造的语法的 Julia 文档(依版本不同有所变动)。


\item 在 Julia 中,数组、字符串等的索引从 1 开始,而不是从 0 开始。


\item Julia 的数组在赋值给另一个变量时不发生复制。执行 \texttt{A = B} 后,改变 \texttt{B} 中元素也会修改 \texttt{A}。像 \texttt{+=} 这样的更新运算符不会以 in-place 的方式执行,而是相当于 \texttt{A = A + B},将左侧绑定到右侧表达式的计算结果上。


\item Julia 的数组是列优先的(Fortran 顺序),而 C/C++ 的数组默认是行优先的。要使数组上的循环性能最优,在 Julia 中循环的顺序应该与 C/C++ 相反(参见 \hyperlink{818954303942149020}{性能建议})。

reversed in Julia relative to C/C++ (see \hyperlink{11239800376478112527}{relevant section of Performance Tips}).


\item Julia 的值在赋值或向函数传递时不发生复制。如果某个函数修改了数组,这一修改对调用者是可见的。


\item 在 Julia 中,空格是有意义的,这与 C/C++ 不同,所以向 Julia 程序中添加或删除空格时必须谨慎。


\item 在 Julia 中,没有小数点的数值字面量(如 \texttt{42})生成有符号整数,类型为 \texttt{Int},但如果字面量太长,超过了机器字长,则会被自动提升为容量更大的类型,例如 \texttt{Int64}(如果 \texttt{Int} 是 \texttt{Int32})、\texttt{Int128},或者任意精度的 \texttt{BigInt} 类型。不存在诸如 \texttt{L}, \texttt{LL}, \texttt{U}, \texttt{UL}, \texttt{ULL} 这样的数值字面量后缀指示无符号和/或有符号与无符号。十进制字面量始终是有符号的,十六进制字面量(像 C/C++ 一样由 \texttt{0x} 开头)是无符号的。另外,十六进制字面量与 C/C++/Java 不同,也与 Julia 中的十进制字面量不同,它们的类型取决于字面量的\textbf{长度},包括开头的 0。例如,\texttt{0x0} 和 \texttt{0x00} 的类型是 \hyperlink{6609065134969660118}{\texttt{UInt8}},\texttt{0x000} 和 \texttt{0x0000} 的类型是 \hyperlink{7018610346698168012}{\texttt{UInt16}}。同理,字面量的长度在 5-8 之间,类型为 \texttt{UInt32};在 9-16 之间,类型为 \texttt{UInt64};在 17-32 之间,类型为 \texttt{UInt128}。当定义十六进制掩码时,就需要将这一问题考虑在内,比如 \texttt{{\textasciitilde}0xf == 0xf0} 与 \texttt{{\textasciitilde}0x000f == 0xfff0} 完全不同。64 位 \texttt{Float64} 和 32 位 \hyperlink{8101639384272933082}{\texttt{Float32}} 的字面量分别表示为 \texttt{1.0} 和 \texttt{1.0f0}。浮点字面量在无法被精确表示时舍入(且不会提升为 \texttt{BigFloat} 类型)。浮点字面量在行为上与 C/C++ 更接近。八进制(前缀为 \texttt{0o})和二进制(前缀为 \texttt{0b})也被视为无符号的。


\item 字符串字面量可用 \texttt{{\textquotedbl}} 或 \texttt{{\textquotedbl}{\textquotedbl}{\textquotedbl}} 分隔,用 \texttt{{\textquotedbl}{\textquotedbl}{\textquotedbl}} 分隔的字面量可以包含 \texttt{{\textquotedbl}} 字符而无需像 \texttt{{\textquotedbl}{\textbackslash}{\textquotedbl}{\textquotedbl}} 这样来引用它。字符串字面量可以包含插入其中的其他变量或表达式,由 \texttt{\$variablename} 或 \texttt{\$(expression)} 表示,它在该函数所处的上下文中计算变量名或表达式。


\item \texttt{//} 表示 \hyperlink{8304566144531167610}{\texttt{Rational}} 数,而非单行注释(其在 Julia 中是 \texttt{\#})


\item \texttt{\#=} 表示多行注释的开头,\texttt{=\#} 结束之。


\item Julia 中的函数返回其最后一个表达式或 \texttt{return} 关键字的值。可以从函数中返回多个值并将其作为元组赋值,如 \texttt{(a, b) = myfunction()} 或 \texttt{a, b = myfunction()},而不必像在 C/C++ 中那样必须传递指向值的指针(即 \texttt{a = myfunction(\&b)})。


\item Julia 不要求使用分号来结束语句。表达式的结果不会自动打印(除了在交互式提示符中,即 REPL),且代码行不需要以分号结尾。\hyperlink{783803254548423222}{\texttt{println}} 或 \hyperlink{13954719910189591998}{\texttt{@printf}} 可用于打印特定输出。在 REPL 中,\texttt{;} 可用于抑制输出。\texttt{;} 在 \texttt{[ ]} 中也有不同的含义,需要注意。\texttt{;} 可用于在单行中分隔表达式,但在许多情况下不是绝对必要的,更经常是为了可读性。


\item 在 Julia 中,运算符 \hyperlink{7071880015536674935}{\texttt{\unicodeveebar{}}}(\hyperlink{7071880015536674935}{\texttt{xor}})执行按位 XOR 操作,即 C/C++ 中的 \hyperlink{462277561264792021}{\texttt{{\textasciicircum}}}。此外,按位运算符不具有与 C/C++ 相同的优先级,所以可能需要括号。


\item Julia 的 \hyperlink{462277561264792021}{\texttt{{\textasciicircum}}} 是取幂(pow),而非 C/C++ 中的按位 XOR(在 Julia 中请使用 \hyperlink{7071880015536674935}{\texttt{\unicodeveebar{}}} 或 \texttt{xor}) \href{@ref}{ }, in Julia)


\item Julia 中有两个右移运算符,\texttt{>>} 和 \texttt{>>>}。\texttt{>>>} 执行逻辑移位,\texttt{>>} 总是执行算术移位(译注:此处原文为「\texttt{>>>} performs an arithmetic shift, \texttt{>>} always performs a logical shift」,疑误),与 C/C++ 不同,C/C++ 中的 \texttt{>>} 的含义依赖于被移位的值的类型。


\item Julia 的 \texttt{->} 创建一个匿名函数,它并不通过指针访问成员。


\item Julia 在编写 \texttt{if} 语句或 \texttt{for}/\texttt{while} 循环时不需要括号:请使用 \texttt{for i in [1, 2, 3]} 代替 \texttt{for (int i=1; i <= 3; i++)},以及 \texttt{if i == 1} 代替 \texttt{if (i == 1)}


\item Julia 不把数字 \texttt{0} 和 \texttt{1} 视为布尔值。在 Julia 中不能编写 \texttt{if (1)},因为 \texttt{if} 语句只接受布尔值。相反,可以编写 \texttt{if true}、\texttt{if Bool(1)} 或 \texttt{if 1==1}。


\item Julia 使用 \texttt{end} 来表示条件块(如 \texttt{if})、循环块(如 \texttt{while}/\texttt{for})和函数的结束。为了代替单行 \texttt{if ( cond ) statement},Julia 允许形式为 \texttt{if cond; statement; end}、\texttt{cond \&\& statement} 和 \texttt{!cond || statement} 的语句。后两种语法中的赋值语句必须显式地包含在括号中,例如 \texttt{cond \&\& (x = value)},这是因为运算符的优先级。


\item Julia 没有用来续行的语法:如果在行的末尾,到目前为止的输入是一个完整的表达式,则认为其已经结束;否则,认为输入继续。强制表达式继续的一种方式是将其包含在括号中。


\item Julia 宏对已解析的表达式进行操作,而非程序的文本,这允许它们执行复杂的 Julia 代码转换。宏名称以 \texttt{@} 字符开头,具有类似函数的语法 \texttt{@mymacro(arg1, arg2, arg3)} 和类似语句的语法 \texttt{@mymacro arg1 arg2 arg3}。两种形式的语法可以相互转换;如果宏出现在另一个表达式中,则类似函数的形式尤其有用,并且它通常是最清晰的。类似语句的形式通常用于标注块,如在分布式 \texttt{for} 结构中:\texttt{@distributed for i in 1:n; \#= body =\#; end}。如果宏结构的结尾不那么清晰,请使用类似函数的形式。


\item Julia 有一个枚举类型,使用宏 \texttt{@enum(name, value1, value2, ...)} 来表示,例如:\texttt{@enum(Fruit, banana=1, apple, pear)}。


\item 按照惯例,修改其参数的函数在名称的末尾有个 \texttt{!},例如 \texttt{push!}。


\item 在 C++ 中,默认情况下,你具有静态分派,即为了支持动态派发,你需要将函数标注为 virtual 函数。另一方面,Julia 中的每个方法都是「virtual」(尽管它更通用,因为方法是在每个参数类型上派发的,而不仅仅是 \texttt{this},并且使用的是最具体的声明规则)。

\end{itemize}


\hypertarget{14324149948761873740}{}


\section{Noteworthy differences from Common Lisp}



\begin{itemize}
\item Julia uses 1-based indexing for arrays by default, and it can also handle arbitrary \hyperlink{1238988360302116626}{index offsets}.


\item Functions and variables share the same namespace (“Lisp-1”).


\item There is a \hyperlink{14946515604348703614}{\texttt{Pair}} type, but it is not meant to be used as a \texttt{COMMON-LISP:CONS}. Various iterable collections can be used interchangeably in most parts of the language (eg splatting, tuples, etc). \texttt{Tuple}s are the closest to Common Lisp lists for \emph{short} collections of heterogeneous elements. Use \texttt{NamedTuple}s in place of alists. For larger collections of homogeneous types, \texttt{Array}s and \texttt{Dict}s should be used.


\item The typical Julia workflow for prototyping also uses continuous manipulation of the image, implemented with the \href{https://github.com/timholy/Revise.jl}{Revise.jl} package.


\item Bignums are supported, but conversion is not automatic; ordinary integers \hyperlink{17610230595270045080}{overflow}.


\item Modules (namespaces) can be hierarchical. \hyperlink{16252475688663093021}{\texttt{import}} and \hyperlink{169458112978175560}{\texttt{using}} have a dual role: they load the code and make it available in the namespace. \texttt{import} for only the module name is possible (roughly equivalent to \texttt{ASDF:LOAD-OP}). Slot names don{\textquotesingle}t need to be exported separately. Global variables can{\textquotesingle}t be assigned to from outside the module (except with \texttt{eval(mod, :(var = val))} as an escape hatch).


\item Macros start with \texttt{@}, and are not as seamlessly integrated into the language as Common Lisp; consequently, macro usage is not as widespread as in the latter. A form of hygiene for \href{@ref Metaprogramming}{macros} is supported by the language. Because of the different surface syntax, there is no equivalent to \texttt{COMMON-LISP:\&BODY}.


\item \emph{All} functions are generic and use multiple dispatch. Argument lists don{\textquotesingle}t have to follow the same template, which leads to a powerful idiom (see \hyperlink{16455129305818705265}{\texttt{do}}). Optional and keyword arguments are handled differently. Method ambiguities are not resolved like in the Common Lisp Object System, necessitating the definition of a more specific method for the intersection.


\item Symbols do not belong to any package, and do not contain any values \emph{per se}. \texttt{M.var} evaluates the symbol \texttt{var} in the module \texttt{M}.


\item A functional programming style is fully supported by the language, including closures, but isn{\textquotesingle}t always the idiomatic solution for Julia. Some \hyperlink{627547588659365489}{workarounds} may be necessary for performance when modifying captured variables.

\end{itemize}


\hypertarget{9161019699686356187}{}


\chapter{Unicode 输入表}



在 Julia REPL 或其它编辑器中,可以像输入 LaTeX 符号一样,用 tab补全下表列出的 Unicode 字符。在 REPL 中,可以先按 \texttt{?} 进入帮助模式,然后将 Unicode 字符复制粘贴进去,一般在文档开头就会写输入方式。



\begin{quote}
\textbf{Warning}

此表第二列可能会缺失一些字符,对某些字符的显示效果也可能会与在 Julia REPL 中不一致。如果发生了这种状况,强烈建议用户检查一下浏览器或 REPL 的字体设置,目前已知很多字体都有显示问题。

\end{quote}



\begin{table}[h]

\begin{tabulary}{\linewidth}{|L|L|L|L|}
\hline
Code point(s) & Character(s) & Tab completion sequence(s) & Unicode name(s) \\
\hline
U+000A1 & ¡ & {\textbackslash}exclamdown & Inverted Exclamation Mark \\
\hline
U+000A3 & £ & {\textbackslash}sterling & Pound Sign \\
\hline
U+000A5 & ¥ & {\textbackslash}yen & Yen Sign \\
\hline
U+000A6 & ¦ & {\textbackslash}brokenbar & Broken Bar / Broken Vertical Bar \\
\hline
U+000A7 & § & {\textbackslash}S & Section Sign \\
\hline
U+000A9 & © & {\textbackslash}copyright, {\textbackslash}:copyright: & Copyright Sign \\
\hline
U+000AA & ª & {\textbackslash}ordfeminine & Feminine Ordinal Indicator \\
\hline
U+000AC & ¬ & {\textbackslash}neg & Not Sign \\
\hline
U+000AE & ® & {\textbackslash}circledR, {\textbackslash}:registered: & Registered Sign / Registered Trade Mark Sign \\
\hline
U+000AF & ¯ & {\textbackslash}highminus & Macron / Spacing Macron \\
\hline
U+000B0 & ° & {\textbackslash}degree & Degree Sign \\
\hline
U+000B1 & ± & {\textbackslash}pm & Plus-Minus Sign / Plus-Or-Minus Sign \\
\hline
U+000B2 & ² & {\textbackslash}{\textasciicircum}2 & Superscript Two / Superscript Digit Two \\
\hline
U+000B3 & ³ & {\textbackslash}{\textasciicircum}3 & Superscript Three / Superscript Digit Three \\
\hline
U+000B6 & ¶ & {\textbackslash}P & Pilcrow Sign / Paragraph Sign \\
\hline
U+000B7 & · & {\textbackslash}cdotp & Middle Dot \\
\hline
U+000B9 & ¹ & {\textbackslash}{\textasciicircum}1 & Superscript One / Superscript Digit One \\
\hline
U+000BA & º & {\textbackslash}ordmasculine & Masculine Ordinal Indicator \\
\hline
U+000BC & ¼ & {\textbackslash}1/4 & Vulgar Fraction One Quarter / Fraction One Quarter \\
\hline
U+000BD & ½ & {\textbackslash}1/2 & Vulgar Fraction One Half / Fraction One Half \\
\hline
U+000BE & ¾ & {\textbackslash}3/4 & Vulgar Fraction Three Quarters / Fraction Three Quarters \\
\hline
U+000BF & ¿ & {\textbackslash}questiondown & Inverted Question Mark \\
\hline
U+000C5 & Å & {\textbackslash}AA & Latin Capital Letter A With Ring Above / Latin Capital Letter A Ring \\
\hline
U+000C6 & Æ & {\textbackslash}AE & Latin Capital Letter Ae / Latin Capital Letter A E \\
\hline
U+000D0 & Ð & {\textbackslash}DH & Latin Capital Letter Eth \\
\hline
U+000D7 & × & {\textbackslash}times & Multiplication Sign \\
\hline
U+000D8 & Ø & {\textbackslash}O & Latin Capital Letter O With Stroke / Latin Capital Letter O Slash \\
\hline
U+000DE & Þ & {\textbackslash}TH & Latin Capital Letter Thorn \\
\hline
U+000DF & ß & {\textbackslash}ss & Latin Small Letter Sharp S \\
\hline
U+000E5 & å & {\textbackslash}aa & Latin Small Letter A With Ring Above / Latin Small Letter A Ring \\
\hline
U+000E6 & æ & {\textbackslash}ae & Latin Small Letter Ae / Latin Small Letter A E \\
\hline
U+000F0 & ð & {\textbackslash}eth, {\textbackslash}dh & Latin Small Letter Eth \\
\hline
U+000F7 & ÷ & {\textbackslash}div & Division Sign \\
\hline
U+000F8 & ø & {\textbackslash}o & Latin Small Letter O With Stroke / Latin Small Letter O Slash \\
\hline
U+000FE & þ & {\textbackslash}th & Latin Small Letter Thorn \\
\hline
U+00110 & Đ & {\textbackslash}DJ & Latin Capital Letter D With Stroke / Latin Capital Letter D Bar \\
\hline
U+00111 & đ & {\textbackslash}dj & Latin Small Letter D With Stroke / Latin Small Letter D Bar \\
\hline
U+00127 & ħ & {\textbackslash}hbar & Latin Small Letter H With Stroke / Latin Small Letter H Bar \\
\hline
U+00131 & ı & {\textbackslash}imath & Latin Small Letter Dotless I \\
\hline
U+00141 & Ł & {\textbackslash}L & Latin Capital Letter L With Stroke / Latin Capital Letter L Slash \\
\hline
U+00142 & ł & {\textbackslash}l & Latin Small Letter L With Stroke / Latin Small Letter L Slash \\
\hline
U+0014A & Ŋ & {\textbackslash}NG & Latin Capital Letter Eng \\
\hline
U+0014B & ŋ & {\textbackslash}ng & Latin Small Letter Eng \\
\hline
U+00152 & Π& {\textbackslash}OE & Latin Capital Ligature Oe / Latin Capital Letter O E \\
\hline
U+00153 & œ & {\textbackslash}oe & Latin Small Ligature Oe / Latin Small Letter O E \\
\hline
U+00195 & ƕ & {\textbackslash}hvlig & Latin Small Letter Hv / Latin Small Letter H V \\
\hline
U+0019E & ƞ & {\textbackslash}nrleg & Latin Small Letter N With Long Right Leg \\
\hline
U+001B5 & Ƶ & {\textbackslash}Zbar & Latin Capital Letter Z With Stroke / Latin Capital Letter Z Bar \\
\hline
U+001C2 & ǂ & {\textbackslash}doublepipe & Latin Letter Alveolar Click / Latin Letter Pipe Double Bar \\
\hline
U+00237 & ȷ & {\textbackslash}jmath & Latin Small Letter Dotless J \\
\hline
U+00250 & ɐ & {\textbackslash}trna & Latin Small Letter Turned A \\
\hline
U+00252 & ɒ & {\textbackslash}trnsa & Latin Small Letter Turned Alpha / Latin Small Letter Turned Script A \\
\hline
U+00254 & ɔ & {\textbackslash}openo & Latin Small Letter Open O \\
\hline
U+00256 & ɖ & {\textbackslash}rtld & Latin Small Letter D With Tail / Latin Small Letter D Retroflex Hook \\
\hline
U+00259 & ə & {\textbackslash}schwa & Latin Small Letter Schwa \\
\hline
U+00263 & ɣ & {\textbackslash}pgamma & Latin Small Letter Gamma \\
\hline
U+00264 & ɤ & {\textbackslash}pbgam & Latin Small Letter Rams Horn / Latin Small Letter Baby Gamma \\
\hline
U+00265 & ɥ & {\textbackslash}trnh & Latin Small Letter Turned H \\
\hline
U+0026C & ɬ & {\textbackslash}btdl & Latin Small Letter L With Belt / Latin Small Letter L Belt \\
\hline
U+0026D & ɭ & {\textbackslash}rtll & Latin Small Letter L With Retroflex Hook / Latin Small Letter L Retroflex Hook \\
\hline
U+0026F & ɯ & {\textbackslash}trnm & Latin Small Letter Turned M \\
\hline
U+00270 & ɰ & {\textbackslash}trnmlr & Latin Small Letter Turned M With Long Leg \\
\hline
U+00271 & ɱ & {\textbackslash}ltlmr & Latin Small Letter M With Hook / Latin Small Letter M Hook \\
\hline
U+00272 & ɲ & {\textbackslash}ltln & Latin Small Letter N With Left Hook / Latin Small Letter N Hook \\
\hline
U+00273 & ɳ & {\textbackslash}rtln & Latin Small Letter N With Retroflex Hook / Latin Small Letter N Retroflex Hook \\
\hline
U+00277 & ɷ & {\textbackslash}clomeg & Latin Small Letter Closed Omega \\
\hline
U+00278 & ɸ & {\textbackslash}ltphi & Latin Small Letter Phi \\
\hline
U+00279 & ɹ & {\textbackslash}trnr & Latin Small Letter Turned R \\
\hline
U+0027A & ɺ & {\textbackslash}trnrl & Latin Small Letter Turned R With Long Leg \\
\hline
U+0027B & ɻ & {\textbackslash}rttrnr & Latin Small Letter Turned R With Hook / Latin Small Letter Turned R Hook \\
\hline
U+0027C & ɼ & {\textbackslash}rl & Latin Small Letter R With Long Leg \\
\hline
U+0027D & ɽ & {\textbackslash}rtlr & Latin Small Letter R With Tail / Latin Small Letter R Hook \\
\hline
U+0027E & ɾ & {\textbackslash}fhr & Latin Small Letter R With Fishhook / Latin Small Letter Fishhook R \\
\hline
U+00282 & ʂ & {\textbackslash}rtls & Latin Small Letter S With Hook / Latin Small Letter S Hook \\
\hline
U+00283 & ʃ & {\textbackslash}esh & Latin Small Letter Esh \\
\hline
U+00287 & ʇ & {\textbackslash}trnt & Latin Small Letter Turned T \\
\hline
U+00288 & ʈ & {\textbackslash}rtlt & Latin Small Letter T With Retroflex Hook / Latin Small Letter T Retroflex Hook \\
\hline
U+0028A & ʊ & {\textbackslash}pupsil & Latin Small Letter Upsilon \\
\hline
U+0028B & ʋ & {\textbackslash}pscrv & Latin Small Letter V With Hook / Latin Small Letter Script V \\
\hline
U+0028C & ʌ & {\textbackslash}invv & Latin Small Letter Turned V \\
\hline
U+0028D & ʍ & {\textbackslash}invw & Latin Small Letter Turned W \\
\hline
U+0028E & ʎ & {\textbackslash}trny & Latin Small Letter Turned Y \\
\hline
U+00290 & ʐ & {\textbackslash}rtlz & Latin Small Letter Z With Retroflex Hook / Latin Small Letter Z Retroflex Hook \\
\hline
U+00292 & ʒ & {\textbackslash}yogh & Latin Small Letter Ezh / Latin Small Letter Yogh \\
\hline
U+00294 & ʔ & {\textbackslash}glst & Latin Letter Glottal Stop \\
\hline
U+00295 & ʕ & {\textbackslash}reglst & Latin Letter Pharyngeal Voiced Fricative / Latin Letter Reversed Glottal Stop \\
\hline
U+00296 & ʖ & {\textbackslash}inglst & Latin Letter Inverted Glottal Stop \\
\hline
U+0029E & ʞ & {\textbackslash}turnk & Latin Small Letter Turned K \\
\hline
U+002A4 & ʤ & {\textbackslash}dyogh & Latin Small Letter Dezh Digraph / Latin Small Letter D Yogh \\
\hline
U+002A7 & ʧ & {\textbackslash}tesh & Latin Small Letter Tesh Digraph / Latin Small Letter T Esh \\
\hline
U+002B0 & ʰ & {\textbackslash}{\textasciicircum}h & Modifier Letter Small H \\
\hline
U+002B2 & ʲ & {\textbackslash}{\textasciicircum}j & Modifier Letter Small J \\
\hline
U+002B3 & ʳ & {\textbackslash}{\textasciicircum}r & Modifier Letter Small R \\
\hline
U+002B7 & ʷ & {\textbackslash}{\textasciicircum}w & Modifier Letter Small W \\
\hline
U+002B8 & ʸ & {\textbackslash}{\textasciicircum}y & Modifier Letter Small Y \\
\hline
U+002BC & ʼ & {\textbackslash}rasp & Modifier Letter Apostrophe \\
\hline
U+002C8 & ˈ & {\textbackslash}verts & Modifier Letter Vertical Line \\
\hline
U+002CC & ˌ & {\textbackslash}verti & Modifier Letter Low Vertical Line \\
\hline
U+002D0 & ː & {\textbackslash}lmrk & Modifier Letter Triangular Colon \\
\hline
U+002D1 & ˑ & {\textbackslash}hlmrk & Modifier Letter Half Triangular Colon \\
\hline
U+002D2 & ˒ & {\textbackslash}sbrhr & Modifier Letter Centred Right Half Ring / Modifier Letter Centered Right Half Ring \\
\hline
U+002D3 & ˓ & {\textbackslash}sblhr & Modifier Letter Centred Left Half Ring / Modifier Letter Centered Left Half Ring \\
\hline
U+002D4 & ˔ & {\textbackslash}rais & Modifier Letter Up Tack \\
\hline
U+002D5 & ˕ & {\textbackslash}low & Modifier Letter Down Tack \\
\hline
U+002D8 & ˘ & {\textbackslash}u & Breve / Spacing Breve \\
\hline
U+002DC & ˜ & {\textbackslash}tildelow & Small Tilde / Spacing Tilde \\
\hline
U+002E1 & ˡ & {\textbackslash}{\textasciicircum}l & Modifier Letter Small L \\
\hline
U+002E2 & ˢ & {\textbackslash}{\textasciicircum}s & Modifier Letter Small S \\
\hline
U+002E3 & ˣ & {\textbackslash}{\textasciicircum}x & Modifier Letter Small X \\
\hline
U+00300 &  ̀  & {\textbackslash}grave & Combining Grave Accent / Non-Spacing Grave \\
\hline
U+00301 &  ́  & {\textbackslash}acute & Combining Acute Accent / Non-Spacing Acute \\
\hline
U+00302 &  ̂  & {\textbackslash}hat & Combining Circumflex Accent / Non-Spacing Circumflex \\
\hline
U+00303 &  ̃  & {\textbackslash}tilde & Combining Tilde / Non-Spacing Tilde \\
\hline
U+00304 &  ̄  & {\textbackslash}bar & Combining Macron / Non-Spacing Macron \\
\hline
U+00305 &  ̅  & {\textbackslash}overbar & Combining Overline / Non-Spacing Overscore \\
\hline
U+00306 &  ̆  & {\textbackslash}breve & Combining Breve / Non-Spacing Breve \\
\hline
U+00307 &  ̇  & {\textbackslash}dot & Combining Dot Above / Non-Spacing Dot Above \\
\hline
U+00308 &  ̈  & {\textbackslash}ddot & Combining Diaeresis / Non-Spacing Diaeresis \\
\hline
U+00309 &  ̉  & {\textbackslash}ovhook & Combining Hook Above / Non-Spacing Hook Above \\
\hline
U+0030A &  ̊  & {\textbackslash}ocirc & Combining Ring Above / Non-Spacing Ring Above \\
\hline
U+0030B &  ̋  & {\textbackslash}H & Combining Double Acute Accent / Non-Spacing Double Acute \\
\hline
U+0030C &  ̌  & {\textbackslash}check & Combining Caron / Non-Spacing Hacek \\
\hline
U+00310 &  ̐  & {\textbackslash}candra & Combining Candrabindu / Non-Spacing Candrabindu \\
\hline
U+00312 &  ̒  & {\textbackslash}oturnedcomma & Combining Turned Comma Above / Non-Spacing Turned Comma Above \\
\hline
U+00315 &  ̕  & {\textbackslash}ocommatopright & Combining Comma Above Right / Non-Spacing Comma Above Right \\
\hline
U+0031A &  ̚  & {\textbackslash}droang & Combining Left Angle Above / Non-Spacing Left Angle Above \\
\hline
U+00321 &  ̡  & {\textbackslash}palh & Combining Palatalized Hook Below / Non-Spacing Palatalized Hook Below \\
\hline
U+00322 &  ̢  & {\textbackslash}rh & Combining Retroflex Hook Below / Non-Spacing Retroflex Hook Below \\
\hline
U+00327 &  ̧  & {\textbackslash}c & Combining Cedilla / Non-Spacing Cedilla \\
\hline
U+00328 &  ̨  & {\textbackslash}k & Combining Ogonek / Non-Spacing Ogonek \\
\hline
U+0032A &  ̪  & {\textbackslash}sbbrg & Combining Bridge Below / Non-Spacing Bridge Below \\
\hline
U+00330 &  ̰  & {\textbackslash}wideutilde & Combining Tilde Below / Non-Spacing Tilde Below \\
\hline
U+00332 &  ̲  & {\textbackslash}underbar & Combining Low Line / Non-Spacing Underscore \\
\hline
U+00336 &  ̶  & {\textbackslash}strike, {\textbackslash}sout & Combining Long Stroke Overlay / Non-Spacing Long Bar Overlay \\
\hline
U+00338 &  ̸  & {\textbackslash}not & Combining Long Solidus Overlay / Non-Spacing Long Slash Overlay \\
\hline
U+0034D &  ͍  & {\textbackslash}underleftrightarrow & Combining Left Right Arrow Below \\
\hline
U+00391 & Α & {\textbackslash}Alpha & Greek Capital Letter Alpha \\
\hline
U+00392 & Β & {\textbackslash}Beta & Greek Capital Letter Beta \\
\hline
U+00393 & Γ & {\textbackslash}Gamma & Greek Capital Letter Gamma \\
\hline
U+00394 & Δ & {\textbackslash}Delta & Greek Capital Letter Delta \\
\hline
U+00395 & Ε & {\textbackslash}Epsilon & Greek Capital Letter Epsilon \\
\hline
U+00396 & Ζ & {\textbackslash}Zeta & Greek Capital Letter Zeta \\
\hline
U+00397 & Η & {\textbackslash}Eta & Greek Capital Letter Eta \\
\hline
U+00398 & Θ & {\textbackslash}Theta & Greek Capital Letter Theta \\
\hline
U+00399 & Ι & {\textbackslash}Iota & Greek Capital Letter Iota \\
\hline
U+0039A & Κ & {\textbackslash}Kappa & Greek Capital Letter Kappa \\
\hline
U+0039B & Λ & {\textbackslash}Lambda & Greek Capital Letter Lamda / Greek Capital Letter Lambda \\
\hline
U+0039C & Μ & {\textbackslash}upMu & Greek Capital Letter Mu \\
\hline
U+0039D & Ν & {\textbackslash}upNu & Greek Capital Letter Nu \\
\hline
U+0039E & Ξ & {\textbackslash}Xi & Greek Capital Letter Xi \\
\hline
U+0039F & Ο & {\textbackslash}upOmicron & Greek Capital Letter Omicron \\
\hline
U+003A0 & Π & {\textbackslash}Pi & Greek Capital Letter Pi \\
\hline
U+003A1 & Ρ & {\textbackslash}Rho & Greek Capital Letter Rho \\
\hline
U+003A3 & Σ & {\textbackslash}Sigma & Greek Capital Letter Sigma \\
\hline
U+003A4 & Τ & {\textbackslash}Tau & Greek Capital Letter Tau \\
\hline
U+003A5 & Υ & {\textbackslash}Upsilon & Greek Capital Letter Upsilon \\
\hline
U+003A6 & Φ & {\textbackslash}Phi & Greek Capital Letter Phi \\
\hline
U+003A7 & Χ & {\textbackslash}Chi & Greek Capital Letter Chi \\
\hline
U+003A8 & Ψ & {\textbackslash}Psi & Greek Capital Letter Psi \\
\hline
U+003A9 & Ω & {\textbackslash}Omega & Greek Capital Letter Omega \\
\hline
U+003B1 & α & {\textbackslash}alpha & Greek Small Letter Alpha \\
\hline
U+003B2 & β & {\textbackslash}beta & Greek Small Letter Beta \\
\hline
U+003B3 & γ & {\textbackslash}gamma & Greek Small Letter Gamma \\
\hline
U+003B4 & δ & {\textbackslash}delta & Greek Small Letter Delta \\
\hline
U+003B5 & ε & {\textbackslash}upepsilon, {\textbackslash}varepsilon & Greek Small Letter Epsilon \\
\hline
U+003B6 & ζ & {\textbackslash}zeta & Greek Small Letter Zeta \\
\hline
U+003B7 & η & {\textbackslash}eta & Greek Small Letter Eta \\
\hline
U+003B8 & θ & {\textbackslash}theta & Greek Small Letter Theta \\
\hline
U+003B9 & ι & {\textbackslash}iota & Greek Small Letter Iota \\
\hline
U+003BA & κ & {\textbackslash}kappa & Greek Small Letter Kappa \\
\hline
U+003BB & λ & {\textbackslash}lambda & Greek Small Letter Lamda / Greek Small Letter Lambda \\
\hline
U+003BC & μ & {\textbackslash}mu & Greek Small Letter Mu \\
\hline
U+003BD & ν & {\textbackslash}nu & Greek Small Letter Nu \\
\hline
U+003BE & ξ & {\textbackslash}xi & Greek Small Letter Xi \\
\hline
U+003BF & ο & {\textbackslash}upomicron & Greek Small Letter Omicron \\
\hline
U+003C0 & π & {\textbackslash}pi & Greek Small Letter Pi \\
\hline
U+003C1 & ρ & {\textbackslash}rho & Greek Small Letter Rho \\
\hline
U+003C2 & ς & {\textbackslash}varsigma & Greek Small Letter Final Sigma \\
\hline
U+003C3 & σ & {\textbackslash}sigma & Greek Small Letter Sigma \\
\hline
U+003C4 & τ & {\textbackslash}tau & Greek Small Letter Tau \\
\hline
U+003C5 & υ & {\textbackslash}upsilon & Greek Small Letter Upsilon \\
\hline
U+003C6 & φ & {\textbackslash}varphi & Greek Small Letter Phi \\
\hline
U+003C7 & χ & {\textbackslash}chi & Greek Small Letter Chi \\
\hline
U+003C8 & ψ & {\textbackslash}psi & Greek Small Letter Psi \\
\hline
U+003C9 & ω & {\textbackslash}omega & Greek Small Letter Omega \\
\hline
U+003D0 & ϐ & {\textbackslash}upvarbeta & Greek Beta Symbol / Greek Small Letter Curled Beta \\
\hline
U+003D1 & ϑ & {\textbackslash}vartheta & Greek Theta Symbol / Greek Small Letter Script Theta \\
\hline
U+003D5 & ϕ & {\textbackslash}phi & Greek Phi Symbol / Greek Small Letter Script Phi \\
\hline
U+003D6 & ϖ & {\textbackslash}varpi & Greek Pi Symbol / Greek Small Letter Omega Pi \\
\hline
U+003D8 & Ϙ & {\textbackslash}upoldKoppa & Greek Letter Archaic Koppa \\
\hline
U+003D9 & ϙ & {\textbackslash}upoldkoppa & Greek Small Letter Archaic Koppa \\
\hline
U+003DA & Ϛ & {\textbackslash}Stigma & Greek Letter Stigma / Greek Capital Letter Stigma \\
\hline
U+003DB & ϛ & {\textbackslash}upstigma & Greek Small Letter Stigma \\
\hline
U+003DC & Ϝ & {\textbackslash}Digamma & Greek Letter Digamma / Greek Capital Letter Digamma \\
\hline
U+003DD & ϝ & {\textbackslash}digamma & Greek Small Letter Digamma \\
\hline
U+003DE & Ϟ & {\textbackslash}Koppa & Greek Letter Koppa / Greek Capital Letter Koppa \\
\hline
U+003DF & ϟ & {\textbackslash}upkoppa & Greek Small Letter Koppa \\
\hline
U+003E0 & Ϡ & {\textbackslash}Sampi & Greek Letter Sampi / Greek Capital Letter Sampi \\
\hline
U+003E1 & ϡ & {\textbackslash}upsampi & Greek Small Letter Sampi \\
\hline
U+003F0 & ϰ & {\textbackslash}varkappa & Greek Kappa Symbol / Greek Small Letter Script Kappa \\
\hline
U+003F1 & ϱ & {\textbackslash}varrho & Greek Rho Symbol / Greek Small Letter Tailed Rho \\
\hline
U+003F4 & ϴ & {\textbackslash}varTheta & Greek Capital Theta Symbol \\
\hline
U+003F5 & ϵ & {\textbackslash}epsilon & Greek Lunate Epsilon Symbol \\
\hline
U+003F6 & ϶ & {\textbackslash}backepsilon & Greek Reversed Lunate Epsilon Symbol \\
\hline
U+01D2C & ᴬ & {\textbackslash}{\textasciicircum}A & Modifier Letter Capital A \\
\hline
U+01D2E & ᴮ & {\textbackslash}{\textasciicircum}B & Modifier Letter Capital B \\
\hline
U+01D30 & ᴰ & {\textbackslash}{\textasciicircum}D & Modifier Letter Capital D \\
\hline
U+01D31 & ᴱ & {\textbackslash}{\textasciicircum}E & Modifier Letter Capital E \\
\hline
U+01D33 & ᴳ & {\textbackslash}{\textasciicircum}G & Modifier Letter Capital G \\
\hline
U+01D34 & ᴴ & {\textbackslash}{\textasciicircum}H & Modifier Letter Capital H \\
\hline
U+01D35 & ᴵ & {\textbackslash}{\textasciicircum}I & Modifier Letter Capital I \\
\hline
U+01D36 & ᴶ & {\textbackslash}{\textasciicircum}J & Modifier Letter Capital J \\
\hline
U+01D37 & ᴷ & {\textbackslash}{\textasciicircum}K & Modifier Letter Capital K \\
\hline
U+01D38 & ᴸ & {\textbackslash}{\textasciicircum}L & Modifier Letter Capital L \\
\hline
U+01D39 & ᴹ & {\textbackslash}{\textasciicircum}M & Modifier Letter Capital M \\
\hline
U+01D3A & ᴺ & {\textbackslash}{\textasciicircum}N & Modifier Letter Capital N \\
\hline
U+01D3C & ᴼ & {\textbackslash}{\textasciicircum}O & Modifier Letter Capital O \\
\hline
U+01D3E & ᴾ & {\textbackslash}{\textasciicircum}P & Modifier Letter Capital P \\
\hline
U+01D3F & ᴿ & {\textbackslash}{\textasciicircum}R & Modifier Letter Capital R \\
\hline
U+01D40 & ᵀ & {\textbackslash}{\textasciicircum}T & Modifier Letter Capital T \\
\hline
U+01D41 & ᵁ & {\textbackslash}{\textasciicircum}U & Modifier Letter Capital U \\
\hline
U+01D42 & ᵂ & {\textbackslash}{\textasciicircum}W & Modifier Letter Capital W \\
\hline
U+01D43 & ᵃ & {\textbackslash}{\textasciicircum}a & Modifier Letter Small A \\
\hline
U+01D45 & ᵅ & {\textbackslash}{\textasciicircum}alpha & Modifier Letter Small Alpha \\
\hline
U+01D47 & ᵇ & {\textbackslash}{\textasciicircum}b & Modifier Letter Small B \\
\hline
U+01D48 & ᵈ & {\textbackslash}{\textasciicircum}d & Modifier Letter Small D \\
\hline
U+01D49 & ᵉ & {\textbackslash}{\textasciicircum}e & Modifier Letter Small E \\
\hline
U+01D4B & ᵋ & {\textbackslash}{\textasciicircum}epsilon & Modifier Letter Small Open E \\
\hline
U+01D4D & ᵍ & {\textbackslash}{\textasciicircum}g & Modifier Letter Small G \\
\hline
U+01D4F & ᵏ & {\textbackslash}{\textasciicircum}k & Modifier Letter Small K \\
\hline
U+01D50 & ᵐ & {\textbackslash}{\textasciicircum}m & Modifier Letter Small M \\
\hline
U+01D52 & ᵒ & {\textbackslash}{\textasciicircum}o & Modifier Letter Small O \\
\hline
U+01D56 & ᵖ & {\textbackslash}{\textasciicircum}p & Modifier Letter Small P \\
\hline
U+01D57 & ᵗ & {\textbackslash}{\textasciicircum}t & Modifier Letter Small T \\
\hline
U+01D58 & ᵘ & {\textbackslash}{\textasciicircum}u & Modifier Letter Small U \\
\hline
U+01D5B & ᵛ & {\textbackslash}{\textasciicircum}v & Modifier Letter Small V \\
\hline
U+01D5D & ᵝ & {\textbackslash}{\textasciicircum}beta & Modifier Letter Small Beta \\
\hline
U+01D5E & ᵞ & {\textbackslash}{\textasciicircum}gamma & Modifier Letter Small Greek Gamma \\
\hline
U+01D5F & ᵟ & {\textbackslash}{\textasciicircum}delta & Modifier Letter Small Delta \\
\hline
U+01D60 & ᵠ & {\textbackslash}{\textasciicircum}phi & Modifier Letter Small Greek Phi \\
\hline
U+01D61 & ᵡ & {\textbackslash}{\textasciicircum}chi & Modifier Letter Small Chi \\
\hline
U+01D62 & ᵢ & {\textbackslash}\_i & Latin Subscript Small Letter I \\
\hline
U+01D63 & ᵣ & {\textbackslash}\_r & Latin Subscript Small Letter R \\
\hline
U+01D64 & ᵤ & {\textbackslash}\_u & Latin Subscript Small Letter U \\
\hline
U+01D65 & ᵥ & {\textbackslash}\_v & Latin Subscript Small Letter V \\
\hline
U+01D66 & ᵦ & {\textbackslash}\_beta & Greek Subscript Small Letter Beta \\
\hline
U+01D67 & ᵧ & {\textbackslash}\_gamma & Greek Subscript Small Letter Gamma \\
\hline
U+01D68 & ᵨ & {\textbackslash}\_rho & Greek Subscript Small Letter Rho \\
\hline
U+01D69 & ᵩ & {\textbackslash}\_phi & Greek Subscript Small Letter Phi \\
\hline
U+01D6A & ᵪ & {\textbackslash}\_chi & Greek Subscript Small Letter Chi \\
\hline
U+01D9C & ᶜ & {\textbackslash}{\textasciicircum}c & Modifier Letter Small C \\
\hline
U+01DA0 & ᶠ & {\textbackslash}{\textasciicircum}f & Modifier Letter Small F \\
\hline
U+01DA5 & ᶥ & {\textbackslash}{\textasciicircum}iota & Modifier Letter Small Iota \\
\hline
U+01DB2 & ᶲ & {\textbackslash}{\textasciicircum}Phi & Modifier Letter Small Phi \\
\hline
U+01DBB & ᶻ & {\textbackslash}{\textasciicircum}z & Modifier Letter Small Z \\
\hline
U+01DBF & ᶿ & {\textbackslash}{\textasciicircum}theta & Modifier Letter Small Theta \\
\hline
U+02002 &   & {\textbackslash}enspace & En Space \\
\hline
U+02003 &   & {\textbackslash}quad & Em Space \\
\hline
U+02005 &   & {\textbackslash}thickspace & Four-Per-Em Space \\
\hline
U+02009 &   & {\textbackslash}thinspace & Thin Space \\
\hline
U+0200A &   & {\textbackslash}hspace & Hair Space \\
\hline
U+02013 & – & {\textbackslash}endash & En Dash \\
\hline
U+02014 & — & {\textbackslash}emdash & Em Dash \\
\hline
U+02016 & ‖ & {\textbackslash}Vert & Double Vertical Line / Double Vertical Bar \\
\hline
U+02018 & ‘ & {\textbackslash}lq & Left Single Quotation Mark / Single Turned Comma Quotation Mark \\
\hline
U+02019 & ’ & {\textbackslash}rq & Right Single Quotation Mark / Single Comma Quotation Mark \\
\hline
U+0201B & ‛ & {\textbackslash}reapos & Single High-Reversed-9 Quotation Mark / Single Reversed Comma Quotation Mark \\
\hline
U+0201C & “ & {\textbackslash}quotedblleft & Left Double Quotation Mark / Double Turned Comma Quotation Mark \\
\hline
U+0201D & ” & {\textbackslash}quotedblright & Right Double Quotation Mark / Double Comma Quotation Mark \\
\hline
U+02020 & † & {\textbackslash}dagger & Dagger \\
\hline
U+02021 & ‡ & {\textbackslash}ddagger & Double Dagger \\
\hline
U+02022 & • & {\textbackslash}bullet & Bullet \\
\hline
U+02026 & … & {\textbackslash}dots, {\textbackslash}ldots & Horizontal Ellipsis \\
\hline
U+02030 & ‰ & {\textbackslash}perthousand & Per Mille Sign \\
\hline
U+02031 & ‱ & {\textbackslash}pertenthousand & Per Ten Thousand Sign \\
\hline
U+02032 & ′ & {\textbackslash}prime & Prime \\
\hline
U+02033 & ″ & {\textbackslash}pprime & Double Prime \\
\hline
U+02034 & ‴ & {\textbackslash}ppprime & Triple Prime \\
\hline
U+02035 & ‵ & {\textbackslash}backprime & Reversed Prime \\
\hline
U+02036 & ‶ & {\textbackslash}backpprime & Reversed Double Prime \\
\hline
U+02037 & ‷ & {\textbackslash}backppprime & Reversed Triple Prime \\
\hline
U+02039 & ‹ & {\textbackslash}guilsinglleft & Single Left-Pointing Angle Quotation Mark / Left Pointing Single Guillemet \\
\hline
U+0203A & › & {\textbackslash}guilsinglright & Single Right-Pointing Angle Quotation Mark / Right Pointing Single Guillemet \\
\hline
U+0203C & ‼ & {\textbackslash}:bangbang: & Double Exclamation Mark \\
\hline
U+02040 & ⁀ & {\textbackslash}tieconcat & Character Tie \\
\hline
U+02049 & ⁉ & {\textbackslash}:interrobang: & Exclamation Question Mark \\
\hline
U+02057 & ⁗ & {\textbackslash}pppprime & Quadruple Prime \\
\hline
U+0205D & ⁝ & {\textbackslash}tricolon & Tricolon \\
\hline
U+02060 & ⁠ & {\textbackslash}nolinebreak & Word Joiner \\
\hline
U+02070 & ⁰ & {\textbackslash}{\textasciicircum}0 & Superscript Zero / Superscript Digit Zero \\
\hline
U+02071 & ⁱ & {\textbackslash}{\textasciicircum}i & Superscript Latin Small Letter I \\
\hline
U+02074 & ⁴ & {\textbackslash}{\textasciicircum}4 & Superscript Four / Superscript Digit Four \\
\hline
U+02075 & ⁵ & {\textbackslash}{\textasciicircum}5 & Superscript Five / Superscript Digit Five \\
\hline
U+02076 & ⁶ & {\textbackslash}{\textasciicircum}6 & Superscript Six / Superscript Digit Six \\
\hline
U+02077 & ⁷ & {\textbackslash}{\textasciicircum}7 & Superscript Seven / Superscript Digit Seven \\
\hline
U+02078 & ⁸ & {\textbackslash}{\textasciicircum}8 & Superscript Eight / Superscript Digit Eight \\
\hline
U+02079 & ⁹ & {\textbackslash}{\textasciicircum}9 & Superscript Nine / Superscript Digit Nine \\
\hline
U+0207A & ⁺ & {\textbackslash}{\textasciicircum}+ & Superscript Plus Sign \\
\hline
U+0207B & ⁻ & {\textbackslash}{\textasciicircum}- & Superscript Minus / Superscript Hyphen-Minus \\
\hline
U+0207C & ⁼ & {\textbackslash}{\textasciicircum}= & Superscript Equals Sign \\
\hline
U+0207D & ⁽ & {\textbackslash}{\textasciicircum}( & Superscript Left Parenthesis / Superscript Opening Parenthesis \\
\hline
U+0207E & ⁾ & {\textbackslash}{\textasciicircum}) & Superscript Right Parenthesis / Superscript Closing Parenthesis \\
\hline
U+0207F & ⁿ & {\textbackslash}{\textasciicircum}n & Superscript Latin Small Letter N \\
\hline
U+02080 & ₀ & {\textbackslash}\_0 & Subscript Zero / Subscript Digit Zero \\
\hline
U+02081 & ₁ & {\textbackslash}\_1 & Subscript One / Subscript Digit One \\
\hline
U+02082 & ₂ & {\textbackslash}\_2 & Subscript Two / Subscript Digit Two \\
\hline
U+02083 & ₃ & {\textbackslash}\_3 & Subscript Three / Subscript Digit Three \\
\hline
U+02084 & ₄ & {\textbackslash}\_4 & Subscript Four / Subscript Digit Four \\
\hline
U+02085 & ₅ & {\textbackslash}\_5 & Subscript Five / Subscript Digit Five \\
\hline
U+02086 & ₆ & {\textbackslash}\_6 & Subscript Six / Subscript Digit Six \\
\hline
U+02087 & ₇ & {\textbackslash}\_7 & Subscript Seven / Subscript Digit Seven \\
\hline
U+02088 & ₈ & {\textbackslash}\_8 & Subscript Eight / Subscript Digit Eight \\
\hline
U+02089 & ₉ & {\textbackslash}\_9 & Subscript Nine / Subscript Digit Nine \\
\hline
U+0208A & ₊ & {\textbackslash}\_+ & Subscript Plus Sign \\
\hline
U+0208B & ₋ & {\textbackslash}\_- & Subscript Minus / Subscript Hyphen-Minus \\
\hline
U+0208C & ₌ & {\textbackslash}\_= & Subscript Equals Sign \\
\hline
U+0208D & ₍ & {\textbackslash}\_( & Subscript Left Parenthesis / Subscript Opening Parenthesis \\
\hline
U+0208E & ₎ & {\textbackslash}\_) & Subscript Right Parenthesis / Subscript Closing Parenthesis \\
\hline
U+02090 & ₐ & {\textbackslash}\_a & Latin Subscript Small Letter A \\
\hline
U+02091 & ₑ & {\textbackslash}\_e & Latin Subscript Small Letter E \\
\hline
U+02092 & ₒ & {\textbackslash}\_o & Latin Subscript Small Letter O \\
\hline
U+02093 & ₓ & {\textbackslash}\_x & Latin Subscript Small Letter X \\
\hline
U+02094 & ₔ & {\textbackslash}\_schwa & Latin Subscript Small Letter Schwa \\
\hline
U+02095 & ₕ & {\textbackslash}\_h & Latin Subscript Small Letter H \\
\hline
U+02096 & ₖ & {\textbackslash}\_k & Latin Subscript Small Letter K \\
\hline
U+02097 & ₗ & {\textbackslash}\_l & Latin Subscript Small Letter L \\
\hline
U+02098 & ₘ & {\textbackslash}\_m & Latin Subscript Small Letter M \\
\hline
U+02099 & ₙ & {\textbackslash}\_n & Latin Subscript Small Letter N \\
\hline
U+0209A & ₚ & {\textbackslash}\_p & Latin Subscript Small Letter P \\
\hline
U+0209B & ₛ & {\textbackslash}\_s & Latin Subscript Small Letter S \\
\hline
U+0209C & ₜ & {\textbackslash}\_t & Latin Subscript Small Letter T \\
\hline
U+020A7 & ₧ & {\textbackslash}pes & Peseta Sign \\
\hline
U+020AC & € & {\textbackslash}euro & Euro Sign \\
\hline
U+020D0 &  ⃐  & {\textbackslash}leftharpoonaccent & Combining Left Harpoon Above / Non-Spacing Left Harpoon Above \\
\hline
U+020D1 &  ⃑  & {\textbackslash}rightharpoonaccent & Combining Right Harpoon Above / Non-Spacing Right Harpoon Above \\
\hline
U+020D2 &  ⃒  & {\textbackslash}vertoverlay & Combining Long Vertical Line Overlay / Non-Spacing Long Vertical Bar Overlay \\
\hline
U+020D6 &  ⃖  & {\textbackslash}overleftarrow & Combining Left Arrow Above / Non-Spacing Left Arrow Above \\
\hline
U+020D7 &  ⃗  & {\textbackslash}vec & Combining Right Arrow Above / Non-Spacing Right Arrow Above \\
\hline
U+020DB &  ⃛  & {\textbackslash}dddot & Combining Three Dots Above / Non-Spacing Three Dots Above \\
\hline
U+020DC &  ⃜  & {\textbackslash}ddddot & Combining Four Dots Above / Non-Spacing Four Dots Above \\
\hline
U+020DD &  ⃝  & {\textbackslash}enclosecircle & Combining Enclosing Circle / Enclosing Circle \\
\hline
U+020DE &  ⃞  & {\textbackslash}enclosesquare & Combining Enclosing Square / Enclosing Square \\
\hline
U+020DF &  ⃟  & {\textbackslash}enclosediamond & Combining Enclosing Diamond / Enclosing Diamond \\
\hline
U+020E1 &  ⃡  & {\textbackslash}overleftrightarrow & Combining Left Right Arrow Above / Non-Spacing Left Right Arrow Above \\
\hline
U+020E4 &  ⃤  & {\textbackslash}enclosetriangle & Combining Enclosing Upward Pointing Triangle \\
\hline
U+020E7 &  ⃧  & {\textbackslash}annuity & Combining Annuity Symbol \\
\hline
U+020E8 &  ⃨  & {\textbackslash}threeunderdot & Combining Triple Underdot \\
\hline
U+020E9 &  ⃩  & {\textbackslash}widebridgeabove & Combining Wide Bridge Above \\
\hline
U+020EC &  ⃬  & {\textbackslash}underrightharpoondown & Combining Rightwards Harpoon With Barb Downwards \\
\hline
U+020ED &  ⃭  & {\textbackslash}underleftharpoondown & Combining Leftwards Harpoon With Barb Downwards \\
\hline
U+020EE &  ⃮  & {\textbackslash}underleftarrow & Combining Left Arrow Below \\
\hline
U+020EF &  ⃯  & {\textbackslash}underrightarrow & Combining Right Arrow Below \\
\hline
U+020F0 &  ⃰  & {\textbackslash}asteraccent & Combining Asterisk Above \\
\hline
U+02102 & ℂ & {\textbackslash}bbC & Double-Struck Capital C / Double-Struck C \\
\hline
U+02107 & ℇ & {\textbackslash}eulermascheroni & Euler Constant / Eulers \\
\hline
U+0210A & ℊ & {\textbackslash}scrg & Script Small G \\
\hline
U+0210B & ℋ & {\textbackslash}scrH & Script Capital H / Script H \\
\hline
U+0210C & ℌ & {\textbackslash}frakH & Black-Letter Capital H / Black-Letter H \\
\hline
U+0210D & ℍ & {\textbackslash}bbH & Double-Struck Capital H / Double-Struck H \\
\hline
U+0210E & ℎ & {\textbackslash}ith, {\textbackslash}planck & Planck Constant \\
\hline
U+0210F & ℏ & {\textbackslash}hslash & Planck Constant Over Two Pi / Planck Constant Over 2 Pi \\
\hline
U+02110 & ℐ & {\textbackslash}scrI & Script Capital I / Script I \\
\hline
U+02111 & ℑ & {\textbackslash}Im & Black-Letter Capital I / Black-Letter I \\
\hline
U+02112 & ℒ & {\textbackslash}scrL & Script Capital L / Script L \\
\hline
U+02113 & ℓ & {\textbackslash}ell & Script Small L \\
\hline
U+02115 & ℕ & {\textbackslash}bbN & Double-Struck Capital N / Double-Struck N \\
\hline
U+02116 & № & {\textbackslash}numero & Numero Sign / Numero \\
\hline
U+02118 & ℘ & {\textbackslash}wp & Script Capital P / Script P \\
\hline
U+02119 & ℙ & {\textbackslash}bbP & Double-Struck Capital P / Double-Struck P \\
\hline
U+0211A & ℚ & {\textbackslash}bbQ & Double-Struck Capital Q / Double-Struck Q \\
\hline
U+0211B & ℛ & {\textbackslash}scrR & Script Capital R / Script R \\
\hline
U+0211C & ℜ & {\textbackslash}Re & Black-Letter Capital R / Black-Letter R \\
\hline
U+0211D & ℝ & {\textbackslash}bbR & Double-Struck Capital R / Double-Struck R \\
\hline
U+0211E & ℞ & {\textbackslash}xrat & Prescription Take \\
\hline
U+02122 & ™ & {\textbackslash}trademark, {\textbackslash}:tm: & Trade Mark Sign / Trademark \\
\hline
U+02124 & ℤ & {\textbackslash}bbZ & Double-Struck Capital Z / Double-Struck Z \\
\hline
U+02126 & Ω & {\textbackslash}ohm & Ohm Sign / Ohm \\
\hline
U+02127 & ℧ & {\textbackslash}mho & Inverted Ohm Sign / Mho \\
\hline
U+02128 & ℨ & {\textbackslash}frakZ & Black-Letter Capital Z / Black-Letter Z \\
\hline
U+02129 & ℩ & {\textbackslash}turnediota & Turned Greek Small Letter Iota \\
\hline
U+0212B & Å & {\textbackslash}Angstrom & Angstrom Sign / Angstrom Unit \\
\hline
U+0212C & ℬ & {\textbackslash}scrB & Script Capital B / Script B \\
\hline
U+0212D & ℭ & {\textbackslash}frakC & Black-Letter Capital C / Black-Letter C \\
\hline
U+0212F & ℯ & {\textbackslash}scre, {\textbackslash}euler & Script Small E \\
\hline
U+02130 & ℰ & {\textbackslash}scrE & Script Capital E / Script E \\
\hline
U+02131 & ℱ & {\textbackslash}scrF & Script Capital F / Script F \\
\hline
U+02132 & Ⅎ & {\textbackslash}Finv & Turned Capital F / Turned F \\
\hline
U+02133 & ℳ & {\textbackslash}scrM & Script Capital M / Script M \\
\hline
U+02134 & ℴ & {\textbackslash}scro & Script Small O \\
\hline
U+02135 & ℵ & {\textbackslash}aleph & Alef Symbol / First Transfinite Cardinal \\
\hline
U+02136 & ℶ & {\textbackslash}beth & Bet Symbol / Second Transfinite Cardinal \\
\hline
U+02137 & ℷ & {\textbackslash}gimel & Gimel Symbol / Third Transfinite Cardinal \\
\hline
U+02138 & ℸ & {\textbackslash}daleth & Dalet Symbol / Fourth Transfinite Cardinal \\
\hline
U+02139 & ℹ & {\textbackslash}:information\_source: & Information Source \\
\hline
U+0213C & ℼ & {\textbackslash}bbpi & Double-Struck Small Pi \\
\hline
U+0213D & ℽ & {\textbackslash}bbgamma & Double-Struck Small Gamma \\
\hline
U+0213E & ℾ & {\textbackslash}bbGamma & Double-Struck Capital Gamma \\
\hline
U+0213F & ℿ & {\textbackslash}bbPi & Double-Struck Capital Pi \\
\hline
U+02140 & ⅀ & {\textbackslash}bbsum & Double-Struck N-Ary Summation \\
\hline
U+02141 & ⅁ & {\textbackslash}Game & Turned Sans-Serif Capital G \\
\hline
U+02142 & ⅂ & {\textbackslash}sansLturned & Turned Sans-Serif Capital L \\
\hline
U+02143 & ⅃ & {\textbackslash}sansLmirrored & Reversed Sans-Serif Capital L \\
\hline
U+02144 & ⅄ & {\textbackslash}Yup & Turned Sans-Serif Capital Y \\
\hline
U+02145 & ⅅ & {\textbackslash}bbiD & Double-Struck Italic Capital D \\
\hline
U+02146 & ⅆ & {\textbackslash}bbid & Double-Struck Italic Small D \\
\hline
U+02147 & ⅇ & {\textbackslash}bbie & Double-Struck Italic Small E \\
\hline
U+02148 & ⅈ & {\textbackslash}bbii & Double-Struck Italic Small I \\
\hline
U+02149 & ⅉ & {\textbackslash}bbij & Double-Struck Italic Small J \\
\hline
U+0214A & ⅊ & {\textbackslash}PropertyLine & Property Line \\
\hline
U+0214B & ⅋ & {\textbackslash}upand & Turned Ampersand \\
\hline
U+02150 & ⅐ & {\textbackslash}1/7 & Vulgar Fraction One Seventh \\
\hline
U+02151 & ⅑ & {\textbackslash}1/9 & Vulgar Fraction One Ninth \\
\hline
U+02152 & ⅒ & {\textbackslash}1/10 & Vulgar Fraction One Tenth \\
\hline
U+02153 & ⅓ & {\textbackslash}1/3 & Vulgar Fraction One Third / Fraction One Third \\
\hline
U+02154 & ⅔ & {\textbackslash}2/3 & Vulgar Fraction Two Thirds / Fraction Two Thirds \\
\hline
U+02155 & ⅕ & {\textbackslash}1/5 & Vulgar Fraction One Fifth / Fraction One Fifth \\
\hline
U+02156 & ⅖ & {\textbackslash}2/5 & Vulgar Fraction Two Fifths / Fraction Two Fifths \\
\hline
U+02157 & ⅗ & {\textbackslash}3/5 & Vulgar Fraction Three Fifths / Fraction Three Fifths \\
\hline
U+02158 & ⅘ & {\textbackslash}4/5 & Vulgar Fraction Four Fifths / Fraction Four Fifths \\
\hline
U+02159 & ⅙ & {\textbackslash}1/6 & Vulgar Fraction One Sixth / Fraction One Sixth \\
\hline
U+0215A & ⅚ & {\textbackslash}5/6 & Vulgar Fraction Five Sixths / Fraction Five Sixths \\
\hline
U+0215B & ⅛ & {\textbackslash}1/8 & Vulgar Fraction One Eighth / Fraction One Eighth \\
\hline
U+0215C & ⅜ & {\textbackslash}3/8 & Vulgar Fraction Three Eighths / Fraction Three Eighths \\
\hline
U+0215D & ⅝ & {\textbackslash}5/8 & Vulgar Fraction Five Eighths / Fraction Five Eighths \\
\hline
U+0215E & ⅞ & {\textbackslash}7/8 & Vulgar Fraction Seven Eighths / Fraction Seven Eighths \\
\hline
U+0215F & ⅟ & {\textbackslash}1/ & Fraction Numerator One \\
\hline
U+02189 & ↉ & {\textbackslash}0/3 & Vulgar Fraction Zero Thirds \\
\hline
U+02190 & ← & {\textbackslash}leftarrow & Leftwards Arrow / Left Arrow \\
\hline
U+02191 & ↑ & {\textbackslash}uparrow & Upwards Arrow / Up Arrow \\
\hline
U+02192 & → & {\textbackslash}to, {\textbackslash}rightarrow & Rightwards Arrow / Right Arrow \\
\hline
U+02193 & ↓ & {\textbackslash}downarrow & Downwards Arrow / Down Arrow \\
\hline
U+02194 & ↔ & {\textbackslash}leftrightarrow, {\textbackslash}:left\_right\_arrow: & Left Right Arrow \\
\hline
U+02195 & ↕ & {\textbackslash}updownarrow, {\textbackslash}:arrow\_up\_down: & Up Down Arrow \\
\hline
U+02196 & ↖ & {\textbackslash}nwarrow, {\textbackslash}:arrow\_upper\_left: & North West Arrow / Upper Left Arrow \\
\hline
U+02197 & ↗ & {\textbackslash}nearrow, {\textbackslash}:arrow\_upper\_right: & North East Arrow / Upper Right Arrow \\
\hline
U+02198 & ↘ & {\textbackslash}searrow, {\textbackslash}:arrow\_lower\_right: & South East Arrow / Lower Right Arrow \\
\hline
U+02199 & ↙ & {\textbackslash}swarrow, {\textbackslash}:arrow\_lower\_left: & South West Arrow / Lower Left Arrow \\
\hline
U+0219A & ↚ & {\textbackslash}nleftarrow & Leftwards Arrow With Stroke / Left Arrow With Stroke \\
\hline
U+0219B & ↛ & {\textbackslash}nrightarrow & Rightwards Arrow With Stroke / Right Arrow With Stroke \\
\hline
U+0219C & ↜ & {\textbackslash}leftwavearrow & Leftwards Wave Arrow / Left Wave Arrow \\
\hline
U+0219D & ↝ & {\textbackslash}rightwavearrow & Rightwards Wave Arrow / Right Wave Arrow \\
\hline
U+0219E & ↞ & {\textbackslash}twoheadleftarrow & Leftwards Two Headed Arrow / Left Two Headed Arrow \\
\hline
U+0219F & ↟ & {\textbackslash}twoheaduparrow & Upwards Two Headed Arrow / Up Two Headed Arrow \\
\hline
U+021A0 & ↠ & {\textbackslash}twoheadrightarrow & Rightwards Two Headed Arrow / Right Two Headed Arrow \\
\hline
U+021A1 & ↡ & {\textbackslash}twoheaddownarrow & Downwards Two Headed Arrow / Down Two Headed Arrow \\
\hline
U+021A2 & ↢ & {\textbackslash}leftarrowtail & Leftwards Arrow With Tail / Left Arrow With Tail \\
\hline
U+021A3 & ↣ & {\textbackslash}rightarrowtail & Rightwards Arrow With Tail / Right Arrow With Tail \\
\hline
U+021A4 & ↤ & {\textbackslash}mapsfrom & Leftwards Arrow From Bar / Left Arrow From Bar \\
\hline
U+021A5 & ↥ & {\textbackslash}mapsup & Upwards Arrow From Bar / Up Arrow From Bar \\
\hline
U+021A6 & ↦ & {\textbackslash}mapsto & Rightwards Arrow From Bar / Right Arrow From Bar \\
\hline
U+021A7 & ↧ & {\textbackslash}mapsdown & Downwards Arrow From Bar / Down Arrow From Bar \\
\hline
U+021A8 & ↨ & {\textbackslash}updownarrowbar & Up Down Arrow With Base \\
\hline
U+021A9 & ↩ & {\textbackslash}hookleftarrow, {\textbackslash}:leftwards\_arrow\_with\_hook: & Leftwards Arrow With Hook / Left Arrow With Hook \\
\hline
U+021AA & ↪ & {\textbackslash}hookrightarrow, {\textbackslash}:arrow\_right\_hook: & Rightwards Arrow With Hook / Right Arrow With Hook \\
\hline
U+021AB & ↫ & {\textbackslash}looparrowleft & Leftwards Arrow With Loop / Left Arrow With Loop \\
\hline
U+021AC & ↬ & {\textbackslash}looparrowright & Rightwards Arrow With Loop / Right Arrow With Loop \\
\hline
U+021AD & ↭ & {\textbackslash}leftrightsquigarrow & Left Right Wave Arrow \\
\hline
U+021AE & ↮ & {\textbackslash}nleftrightarrow & Left Right Arrow With Stroke \\
\hline
U+021AF & ↯ & {\textbackslash}downzigzagarrow & Downwards Zigzag Arrow / Down Zigzag Arrow \\
\hline
U+021B0 & ↰ & {\textbackslash}Lsh & Upwards Arrow With Tip Leftwards / Up Arrow With Tip Left \\
\hline
U+021B1 & ↱ & {\textbackslash}Rsh & Upwards Arrow With Tip Rightwards / Up Arrow With Tip Right \\
\hline
U+021B2 & ↲ & {\textbackslash}Ldsh & Downwards Arrow With Tip Leftwards / Down Arrow With Tip Left \\
\hline
U+021B3 & ↳ & {\textbackslash}Rdsh & Downwards Arrow With Tip Rightwards / Down Arrow With Tip Right \\
\hline
U+021B4 & ↴ & {\textbackslash}linefeed & Rightwards Arrow With Corner Downwards / Right Arrow With Corner Down \\
\hline
U+021B5 & ↵ & {\textbackslash}carriagereturn & Downwards Arrow With Corner Leftwards / Down Arrow With Corner Left \\
\hline
U+021B6 & ↶ & {\textbackslash}curvearrowleft & Anticlockwise Top Semicircle Arrow \\
\hline
U+021B7 & ↷ & {\textbackslash}curvearrowright & Clockwise Top Semicircle Arrow \\
\hline
U+021B8 & ↸ & {\textbackslash}barovernorthwestarrow & North West Arrow To Long Bar / Upper Left Arrow To Long Bar \\
\hline
U+021B9 & ↹ & {\textbackslash}barleftarrowrightarrowbar & Leftwards Arrow To Bar Over Rightwards Arrow To Bar / Left Arrow To Bar Over Right Arrow To Bar \\
\hline
U+021BA & ↺ & {\textbackslash}circlearrowleft & Anticlockwise Open Circle Arrow \\
\hline
U+021BB & ↻ & {\textbackslash}circlearrowright & Clockwise Open Circle Arrow \\
\hline
U+021BC & ↼ & {\textbackslash}leftharpoonup & Leftwards Harpoon With Barb Upwards / Left Harpoon With Barb Up \\
\hline
U+021BD & ↽ & {\textbackslash}leftharpoondown & Leftwards Harpoon With Barb Downwards / Left Harpoon With Barb Down \\
\hline
U+021BE & ↾ & {\textbackslash}upharpoonright & Upwards Harpoon With Barb Rightwards / Up Harpoon With Barb Right \\
\hline
U+021BF & ↿ & {\textbackslash}upharpoonleft & Upwards Harpoon With Barb Leftwards / Up Harpoon With Barb Left \\
\hline
U+021C0 & ⇀ & {\textbackslash}rightharpoonup & Rightwards Harpoon With Barb Upwards / Right Harpoon With Barb Up \\
\hline
U+021C1 & ⇁ & {\textbackslash}rightharpoondown & Rightwards Harpoon With Barb Downwards / Right Harpoon With Barb Down \\
\hline
U+021C2 & ⇂ & {\textbackslash}downharpoonright & Downwards Harpoon With Barb Rightwards / Down Harpoon With Barb Right \\
\hline
U+021C3 & ⇃ & {\textbackslash}downharpoonleft & Downwards Harpoon With Barb Leftwards / Down Harpoon With Barb Left \\
\hline
U+021C4 & ⇄ & {\textbackslash}rightleftarrows & Rightwards Arrow Over Leftwards Arrow / Right Arrow Over Left Arrow \\
\hline
U+021C5 & ⇅ & {\textbackslash}dblarrowupdown & Upwards Arrow Leftwards Of Downwards Arrow / Up Arrow Left Of Down Arrow \\
\hline
U+021C6 & ⇆ & {\textbackslash}leftrightarrows & Leftwards Arrow Over Rightwards Arrow / Left Arrow Over Right Arrow \\
\hline
U+021C7 & ⇇ & {\textbackslash}leftleftarrows & Leftwards Paired Arrows / Left Paired Arrows \\
\hline
U+021C8 & ⇈ & {\textbackslash}upuparrows & Upwards Paired Arrows / Up Paired Arrows \\
\hline
U+021C9 & ⇉ & {\textbackslash}rightrightarrows & Rightwards Paired Arrows / Right Paired Arrows \\
\hline
U+021CA & ⇊ & {\textbackslash}downdownarrows & Downwards Paired Arrows / Down Paired Arrows \\
\hline
U+021CB & ⇋ & {\textbackslash}leftrightharpoons & Leftwards Harpoon Over Rightwards Harpoon / Left Harpoon Over Right Harpoon \\
\hline
U+021CC & ⇌ & {\textbackslash}rightleftharpoons & Rightwards Harpoon Over Leftwards Harpoon / Right Harpoon Over Left Harpoon \\
\hline
U+021CD & ⇍ & {\textbackslash}nLeftarrow & Leftwards Double Arrow With Stroke / Left Double Arrow With Stroke \\
\hline
U+021CE & ⇎ & {\textbackslash}nLeftrightarrow & Left Right Double Arrow With Stroke \\
\hline
U+021CF & ⇏ & {\textbackslash}nRightarrow & Rightwards Double Arrow With Stroke / Right Double Arrow With Stroke \\
\hline
U+021D0 & ⇐ & {\textbackslash}Leftarrow & Leftwards Double Arrow / Left Double Arrow \\
\hline
U+021D1 & ⇑ & {\textbackslash}Uparrow & Upwards Double Arrow / Up Double Arrow \\
\hline
U+021D2 & ⇒ & {\textbackslash}Rightarrow & Rightwards Double Arrow / Right Double Arrow \\
\hline
U+021D3 & ⇓ & {\textbackslash}Downarrow & Downwards Double Arrow / Down Double Arrow \\
\hline
U+021D4 & ⇔ & {\textbackslash}Leftrightarrow & Left Right Double Arrow \\
\hline
U+021D5 & ⇕ & {\textbackslash}Updownarrow & Up Down Double Arrow \\
\hline
U+021D6 & ⇖ & {\textbackslash}Nwarrow & North West Double Arrow / Upper Left Double Arrow \\
\hline
U+021D7 & ⇗ & {\textbackslash}Nearrow & North East Double Arrow / Upper Right Double Arrow \\
\hline
U+021D8 & ⇘ & {\textbackslash}Searrow & South East Double Arrow / Lower Right Double Arrow \\
\hline
U+021D9 & ⇙ & {\textbackslash}Swarrow & South West Double Arrow / Lower Left Double Arrow \\
\hline
U+021DA & ⇚ & {\textbackslash}Lleftarrow & Leftwards Triple Arrow / Left Triple Arrow \\
\hline
U+021DB & ⇛ & {\textbackslash}Rrightarrow & Rightwards Triple Arrow / Right Triple Arrow \\
\hline
U+021DC & ⇜ & {\textbackslash}leftsquigarrow & Leftwards Squiggle Arrow / Left Squiggle Arrow \\
\hline
U+021DD & ⇝ & {\textbackslash}rightsquigarrow & Rightwards Squiggle Arrow / Right Squiggle Arrow \\
\hline
U+021DE & ⇞ & {\textbackslash}nHuparrow & Upwards Arrow With Double Stroke / Up Arrow With Double Stroke \\
\hline
U+021DF & ⇟ & {\textbackslash}nHdownarrow & Downwards Arrow With Double Stroke / Down Arrow With Double Stroke \\
\hline
U+021E0 & ⇠ & {\textbackslash}leftdasharrow & Leftwards Dashed Arrow / Left Dashed Arrow \\
\hline
U+021E1 & ⇡ & {\textbackslash}updasharrow & Upwards Dashed Arrow / Up Dashed Arrow \\
\hline
U+021E2 & ⇢ & {\textbackslash}rightdasharrow & Rightwards Dashed Arrow / Right Dashed Arrow \\
\hline
U+021E3 & ⇣ & {\textbackslash}downdasharrow & Downwards Dashed Arrow / Down Dashed Arrow \\
\hline
U+021E4 & ⇤ & {\textbackslash}barleftarrow & Leftwards Arrow To Bar / Left Arrow To Bar \\
\hline
U+021E5 & ⇥ & {\textbackslash}rightarrowbar & Rightwards Arrow To Bar / Right Arrow To Bar \\
\hline
U+021E6 & ⇦ & {\textbackslash}leftwhitearrow & Leftwards White Arrow / White Left Arrow \\
\hline
U+021E7 & ⇧ & {\textbackslash}upwhitearrow & Upwards White Arrow / White Up Arrow \\
\hline
U+021E8 & ⇨ & {\textbackslash}rightwhitearrow & Rightwards White Arrow / White Right Arrow \\
\hline
U+021E9 & ⇩ & {\textbackslash}downwhitearrow & Downwards White Arrow / White Down Arrow \\
\hline
U+021EA & ⇪ & {\textbackslash}whitearrowupfrombar & Upwards White Arrow From Bar / White Up Arrow From Bar \\
\hline
U+021F4 & ⇴ & {\textbackslash}circleonrightarrow & Right Arrow With Small Circle \\
\hline
U+021F5 & ⇵ & {\textbackslash}DownArrowUpArrow & Downwards Arrow Leftwards Of Upwards Arrow \\
\hline
U+021F6 & ⇶ & {\textbackslash}rightthreearrows & Three Rightwards Arrows \\
\hline
U+021F7 & ⇷ & {\textbackslash}nvleftarrow & Leftwards Arrow With Vertical Stroke \\
\hline
U+021F8 & ⇸ & {\textbackslash}nvrightarrow & Rightwards Arrow With Vertical Stroke \\
\hline
U+021F9 & ⇹ & {\textbackslash}nvleftrightarrow & Left Right Arrow With Vertical Stroke \\
\hline
U+021FA & ⇺ & {\textbackslash}nVleftarrow & Leftwards Arrow With Double Vertical Stroke \\
\hline
U+021FB & ⇻ & {\textbackslash}nVrightarrow & Rightwards Arrow With Double Vertical Stroke \\
\hline
U+021FC & ⇼ & {\textbackslash}nVleftrightarrow & Left Right Arrow With Double Vertical Stroke \\
\hline
U+021FD & ⇽ & {\textbackslash}leftarrowtriangle & Leftwards Open-Headed Arrow \\
\hline
U+021FE & ⇾ & {\textbackslash}rightarrowtriangle & Rightwards Open-Headed Arrow \\
\hline
U+021FF & ⇿ & {\textbackslash}leftrightarrowtriangle & Left Right Open-Headed Arrow \\
\hline
U+02200 & ∀ & {\textbackslash}forall & For All \\
\hline
U+02201 & ∁ & {\textbackslash}complement & Complement \\
\hline
U+02202 & ∂ & {\textbackslash}partial & Partial Differential \\
\hline
U+02203 & ∃ & {\textbackslash}exists & There Exists \\
\hline
U+02204 & ∄ & {\textbackslash}nexists & There Does Not Exist \\
\hline
U+02205 & ∅ & {\textbackslash}varnothing, {\textbackslash}emptyset & Empty Set \\
\hline
U+02206 & ∆ & {\textbackslash}increment & Increment \\
\hline
U+02207 & ∇ & {\textbackslash}del, {\textbackslash}nabla & Nabla \\
\hline
U+02208 & ∈ & {\textbackslash}in & Element Of \\
\hline
U+02209 & ∉ & {\textbackslash}notin & Not An Element Of \\
\hline
U+0220A & ∊ & {\textbackslash}smallin & Small Element Of \\
\hline
U+0220B & ∋ & {\textbackslash}ni & Contains As Member \\
\hline
U+0220C & ∌ & {\textbackslash}nni & Does Not Contain As Member \\
\hline
U+0220D & ∍ & {\textbackslash}smallni & Small Contains As Member \\
\hline
U+0220E & ∎ & {\textbackslash}QED & End Of Proof \\
\hline
U+0220F & ∏ & {\textbackslash}prod & N-Ary Product \\
\hline
U+02210 & ∐ & {\textbackslash}coprod & N-Ary Coproduct \\
\hline
U+02211 & ∑ & {\textbackslash}sum & N-Ary Summation \\
\hline
U+02212 & − & {\textbackslash}minus & Minus Sign \\
\hline
U+02213 & ∓ & {\textbackslash}mp & Minus-Or-Plus Sign \\
\hline
U+02214 & ∔ & {\textbackslash}dotplus & Dot Plus \\
\hline
U+02216 & ∖ & {\textbackslash}setminus & Set Minus \\
\hline
U+02217 & ∗ & {\textbackslash}ast & Asterisk Operator \\
\hline
U+02218 & ∘ & {\textbackslash}circ & Ring Operator \\
\hline
U+02219 & ∙ & {\textbackslash}vysmblkcircle & Bullet Operator \\
\hline
U+0221A & √ & {\textbackslash}surd, {\textbackslash}sqrt & Square Root \\
\hline
U+0221B & ∛ & {\textbackslash}cbrt & Cube Root \\
\hline
U+0221C & ∜ & {\textbackslash}fourthroot & Fourth Root \\
\hline
U+0221D & ∝ & {\textbackslash}propto & Proportional To \\
\hline
U+0221E & ∞ & {\textbackslash}infty & Infinity \\
\hline
U+0221F & ∟ & {\textbackslash}rightangle & Right Angle \\
\hline
U+02220 & ∠ & {\textbackslash}angle & Angle \\
\hline
U+02221 & ∡ & {\textbackslash}measuredangle & Measured Angle \\
\hline
U+02222 & ∢ & {\textbackslash}sphericalangle & Spherical Angle \\
\hline
U+02223 & ∣ & {\textbackslash}mid & Divides \\
\hline
U+02224 & ∤ & {\textbackslash}nmid & Does Not Divide \\
\hline
U+02225 & ∥ & {\textbackslash}parallel & Parallel To \\
\hline
U+02226 & ∦ & {\textbackslash}nparallel & Not Parallel To \\
\hline
U+02227 & ∧ & {\textbackslash}wedge & Logical And \\
\hline
U+02228 & ∨ & {\textbackslash}vee & Logical Or \\
\hline
U+02229 & ∩ & {\textbackslash}cap & Intersection \\
\hline
U+0222A & ∪ & {\textbackslash}cup & Union \\
\hline
U+0222B & ∫ & {\textbackslash}int & Integral \\
\hline
U+0222C & ∬ & {\textbackslash}iint & Double Integral \\
\hline
U+0222D & ∭ & {\textbackslash}iiint & Triple Integral \\
\hline
U+0222E & ∮ & {\textbackslash}oint & Contour Integral \\
\hline
U+0222F & ∯ & {\textbackslash}oiint & Surface Integral \\
\hline
U+02230 & ∰ & {\textbackslash}oiiint & Volume Integral \\
\hline
U+02231 & ∱ & {\textbackslash}clwintegral & Clockwise Integral \\
\hline
U+02232 & ∲ & {\textbackslash}varointclockwise & Clockwise Contour Integral \\
\hline
U+02233 & ∳ & {\textbackslash}ointctrclockwise & Anticlockwise Contour Integral \\
\hline
U+02234 & ∴ & {\textbackslash}therefore & Therefore \\
\hline
U+02235 & ∵ & {\textbackslash}because & Because \\
\hline
U+02237 & ∷ & {\textbackslash}Colon & Proportion \\
\hline
U+02238 & ∸ & {\textbackslash}dotminus & Dot Minus \\
\hline
U+0223A & ∺ & {\textbackslash}dotsminusdots & Geometric Proportion \\
\hline
U+0223B & ∻ & {\textbackslash}kernelcontraction & Homothetic \\
\hline
U+0223C & ∼ & {\textbackslash}sim & Tilde Operator \\
\hline
U+0223D & ∽ & {\textbackslash}backsim & Reversed Tilde \\
\hline
U+0223E & ∾ & {\textbackslash}lazysinv & Inverted Lazy S \\
\hline
U+0223F & ∿ & {\textbackslash}sinewave & Sine Wave \\
\hline
U+02240 & ≀ & {\textbackslash}wr & Wreath Product \\
\hline
U+02241 & ≁ & {\textbackslash}nsim & Not Tilde \\
\hline
U+02242 & ≂ & {\textbackslash}eqsim & Minus Tilde \\
\hline
U+02242 + U+00338 & ≂̸ & {\textbackslash}neqsim & Minus Tilde + Combining Long Solidus Overlay / Non-Spacing Long Slash Overlay \\
\hline
U+02243 & ≃ & {\textbackslash}simeq & Asymptotically Equal To \\
\hline
U+02244 & ≄ & {\textbackslash}nsime & Not Asymptotically Equal To \\
\hline
U+02245 & ≅ & {\textbackslash}cong & Approximately Equal To \\
\hline
U+02246 & ≆ & {\textbackslash}approxnotequal & Approximately But Not Actually Equal To \\
\hline
U+02247 & ≇ & {\textbackslash}ncong & Neither Approximately Nor Actually Equal To \\
\hline
U+02248 & ≈ & {\textbackslash}approx & Almost Equal To \\
\hline
U+02249 & ≉ & {\textbackslash}napprox & Not Almost Equal To \\
\hline
U+0224A & ≊ & {\textbackslash}approxeq & Almost Equal Or Equal To \\
\hline
U+0224B & ≋ & {\textbackslash}tildetrpl & Triple Tilde \\
\hline
U+0224C & ≌ & {\textbackslash}allequal & All Equal To \\
\hline
U+0224D & ≍ & {\textbackslash}asymp & Equivalent To \\
\hline
U+0224E & ≎ & {\textbackslash}Bumpeq & Geometrically Equivalent To \\
\hline
U+0224E + U+00338 & ≎̸ & {\textbackslash}nBumpeq & Geometrically Equivalent To + Combining Long Solidus Overlay / Non-Spacing Long Slash Overlay \\
\hline
U+0224F & ≏ & {\textbackslash}bumpeq & Difference Between \\
\hline
U+0224F + U+00338 & ≏̸ & {\textbackslash}nbumpeq & Difference Between + Combining Long Solidus Overlay / Non-Spacing Long Slash Overlay \\
\hline
U+02250 & ≐ & {\textbackslash}doteq & Approaches The Limit \\
\hline
U+02251 & ≑ & {\textbackslash}Doteq & Geometrically Equal To \\
\hline
U+02252 & ≒ & {\textbackslash}fallingdotseq & Approximately Equal To Or The Image Of \\
\hline
U+02253 & ≓ & {\textbackslash}risingdotseq & Image Of Or Approximately Equal To \\
\hline
U+02254 & ≔ & {\textbackslash}coloneq & Colon Equals / Colon Equal \\
\hline
U+02255 & ≕ & {\textbackslash}eqcolon & Equals Colon / Equal Colon \\
\hline
U+02256 & ≖ & {\textbackslash}eqcirc & Ring In Equal To \\
\hline
U+02257 & ≗ & {\textbackslash}circeq & Ring Equal To \\
\hline
U+02258 & ≘ & {\textbackslash}arceq & Corresponds To \\
\hline
U+02259 & ≙ & {\textbackslash}wedgeq & Estimates \\
\hline
U+0225A & ≚ & {\textbackslash}veeeq & Equiangular To \\
\hline
U+0225B & ≛ & {\textbackslash}starequal & Star Equals \\
\hline
U+0225C & ≜ & {\textbackslash}triangleq & Delta Equal To \\
\hline
U+0225D & ≝ & {\textbackslash}eqdef & Equal To By Definition \\
\hline
U+0225E & ≞ & {\textbackslash}measeq & Measured By \\
\hline
U+0225F & ≟ & {\textbackslash}questeq & Questioned Equal To \\
\hline
U+02260 & ≠ & {\textbackslash}ne & Not Equal To \\
\hline
U+02261 & ≡ & {\textbackslash}equiv & Identical To \\
\hline
U+02262 & ≢ & {\textbackslash}nequiv & Not Identical To \\
\hline
U+02263 & ≣ & {\textbackslash}Equiv & Strictly Equivalent To \\
\hline
U+02264 & ≤ & {\textbackslash}le, {\textbackslash}leq & Less-Than Or Equal To / Less Than Or Equal To \\
\hline
U+02265 & ≥ & {\textbackslash}ge, {\textbackslash}geq & Greater-Than Or Equal To / Greater Than Or Equal To \\
\hline
U+02266 & ≦ & {\textbackslash}leqq & Less-Than Over Equal To / Less Than Over Equal To \\
\hline
U+02267 & ≧ & {\textbackslash}geqq & Greater-Than Over Equal To / Greater Than Over Equal To \\
\hline
U+02268 & ≨ & {\textbackslash}lneqq & Less-Than But Not Equal To / Less Than But Not Equal To \\
\hline
U+02268 + U+0FE00 & ≨︀ & {\textbackslash}lvertneqq & Less-Than But Not Equal To / Less Than But Not Equal To + Variation Selector-1 \\
\hline
U+02269 & ≩ & {\textbackslash}gneqq & Greater-Than But Not Equal To / Greater Than But Not Equal To \\
\hline
U+02269 + U+0FE00 & ≩︀ & {\textbackslash}gvertneqq & Greater-Than But Not Equal To / Greater Than But Not Equal To + Variation Selector-1 \\
\hline
U+0226A & ≪ & {\textbackslash}ll & Much Less-Than / Much Less Than \\
\hline
U+0226A + U+00338 & ≪̸ & {\textbackslash}NotLessLess & Much Less-Than / Much Less Than + Combining Long Solidus Overlay / Non-Spacing Long Slash Overlay \\
\hline
U+0226B & ≫ & {\textbackslash}gg & Much Greater-Than / Much Greater Than \\
\hline
U+0226B + U+00338 & ≫̸ & {\textbackslash}NotGreaterGreater & Much Greater-Than / Much Greater Than + Combining Long Solidus Overlay / Non-Spacing Long Slash Overlay \\
\hline
U+0226C & ≬ & {\textbackslash}between & Between \\
\hline
U+0226D & ≭ & {\textbackslash}nasymp & Not Equivalent To \\
\hline
U+0226E & ≮ & {\textbackslash}nless & Not Less-Than / Not Less Than \\
\hline
U+0226F & ≯ & {\textbackslash}ngtr & Not Greater-Than / Not Greater Than \\
\hline
U+02270 & ≰ & {\textbackslash}nleq & Neither Less-Than Nor Equal To / Neither Less Than Nor Equal To \\
\hline
U+02271 & ≱ & {\textbackslash}ngeq & Neither Greater-Than Nor Equal To / Neither Greater Than Nor Equal To \\
\hline
U+02272 & ≲ & {\textbackslash}lesssim & Less-Than Or Equivalent To / Less Than Or Equivalent To \\
\hline
U+02273 & ≳ & {\textbackslash}gtrsim & Greater-Than Or Equivalent To / Greater Than Or Equivalent To \\
\hline
U+02274 & ≴ & {\textbackslash}nlesssim & Neither Less-Than Nor Equivalent To / Neither Less Than Nor Equivalent To \\
\hline
U+02275 & ≵ & {\textbackslash}ngtrsim & Neither Greater-Than Nor Equivalent To / Neither Greater Than Nor Equivalent To \\
\hline
U+02276 & ≶ & {\textbackslash}lessgtr & Less-Than Or Greater-Than / Less Than Or Greater Than \\
\hline
U+02277 & ≷ & {\textbackslash}gtrless & Greater-Than Or Less-Than / Greater Than Or Less Than \\
\hline
U+02278 & ≸ & {\textbackslash}notlessgreater & Neither Less-Than Nor Greater-Than / Neither Less Than Nor Greater Than \\
\hline
U+02279 & ≹ & {\textbackslash}notgreaterless & Neither Greater-Than Nor Less-Than / Neither Greater Than Nor Less Than \\
\hline
U+0227A & ≺ & {\textbackslash}prec & Precedes \\
\hline
U+0227B & ≻ & {\textbackslash}succ & Succeeds \\
\hline
U+0227C & ≼ & {\textbackslash}preccurlyeq & Precedes Or Equal To \\
\hline
U+0227D & ≽ & {\textbackslash}succcurlyeq & Succeeds Or Equal To \\
\hline
U+0227E & ≾ & {\textbackslash}precsim & Precedes Or Equivalent To \\
\hline
U+0227E + U+00338 & ≾̸ & {\textbackslash}nprecsim & Precedes Or Equivalent To + Combining Long Solidus Overlay / Non-Spacing Long Slash Overlay \\
\hline
U+0227F & ≿ & {\textbackslash}succsim & Succeeds Or Equivalent To \\
\hline
U+0227F + U+00338 & ≿̸ & {\textbackslash}nsuccsim & Succeeds Or Equivalent To + Combining Long Solidus Overlay / Non-Spacing Long Slash Overlay \\
\hline
U+02280 & ⊀ & {\textbackslash}nprec & Does Not Precede \\
\hline
U+02281 & ⊁ & {\textbackslash}nsucc & Does Not Succeed \\
\hline
U+02282 & ⊂ & {\textbackslash}subset & Subset Of \\
\hline
U+02283 & ⊃ & {\textbackslash}supset & Superset Of \\
\hline
U+02284 & ⊄ & {\textbackslash}nsubset & Not A Subset Of \\
\hline
U+02285 & ⊅ & {\textbackslash}nsupset & Not A Superset Of \\
\hline
U+02286 & ⊆ & {\textbackslash}subseteq & Subset Of Or Equal To \\
\hline
U+02287 & ⊇ & {\textbackslash}supseteq & Superset Of Or Equal To \\
\hline
U+02288 & ⊈ & {\textbackslash}nsubseteq & Neither A Subset Of Nor Equal To \\
\hline
U+02289 & ⊉ & {\textbackslash}nsupseteq & Neither A Superset Of Nor Equal To \\
\hline
U+0228A & ⊊ & {\textbackslash}subsetneq & Subset Of With Not Equal To / Subset Of Or Not Equal To \\
\hline
U+0228A + U+0FE00 & ⊊︀ & {\textbackslash}varsubsetneqq & Subset Of With Not Equal To / Subset Of Or Not Equal To + Variation Selector-1 \\
\hline
U+0228B & ⊋ & {\textbackslash}supsetneq & Superset Of With Not Equal To / Superset Of Or Not Equal To \\
\hline
U+0228B + U+0FE00 & ⊋︀ & {\textbackslash}varsupsetneq & Superset Of With Not Equal To / Superset Of Or Not Equal To + Variation Selector-1 \\
\hline
U+0228D & ⊍ & {\textbackslash}cupdot & Multiset Multiplication \\
\hline
U+0228E & ⊎ & {\textbackslash}uplus & Multiset Union \\
\hline
U+0228F & ⊏ & {\textbackslash}sqsubset & Square Image Of \\
\hline
U+0228F + U+00338 & ⊏̸ & {\textbackslash}NotSquareSubset & Square Image Of + Combining Long Solidus Overlay / Non-Spacing Long Slash Overlay \\
\hline
U+02290 & ⊐ & {\textbackslash}sqsupset & Square Original Of \\
\hline
U+02290 + U+00338 & ⊐̸ & {\textbackslash}NotSquareSuperset & Square Original Of + Combining Long Solidus Overlay / Non-Spacing Long Slash Overlay \\
\hline
U+02291 & ⊑ & {\textbackslash}sqsubseteq & Square Image Of Or Equal To \\
\hline
U+02292 & ⊒ & {\textbackslash}sqsupseteq & Square Original Of Or Equal To \\
\hline
U+02293 & ⊓ & {\textbackslash}sqcap & Square Cap \\
\hline
U+02294 & ⊔ & {\textbackslash}sqcup & Square Cup \\
\hline
U+02295 & ⊕ & {\textbackslash}oplus & Circled Plus \\
\hline
U+02296 & ⊖ & {\textbackslash}ominus & Circled Minus \\
\hline
U+02297 & ⊗ & {\textbackslash}otimes & Circled Times \\
\hline
U+02298 & ⊘ & {\textbackslash}oslash & Circled Division Slash \\
\hline
U+02299 & ⊙ & {\textbackslash}odot & Circled Dot Operator \\
\hline
U+0229A & ⊚ & {\textbackslash}circledcirc & Circled Ring Operator \\
\hline
U+0229B & ⊛ & {\textbackslash}circledast & Circled Asterisk Operator \\
\hline
U+0229C & ⊜ & {\textbackslash}circledequal & Circled Equals \\
\hline
U+0229D & ⊝ & {\textbackslash}circleddash & Circled Dash \\
\hline
U+0229E & ⊞ & {\textbackslash}boxplus & Squared Plus \\
\hline
U+0229F & ⊟ & {\textbackslash}boxminus & Squared Minus \\
\hline
U+022A0 & ⊠ & {\textbackslash}boxtimes & Squared Times \\
\hline
U+022A1 & ⊡ & {\textbackslash}boxdot & Squared Dot Operator \\
\hline
U+022A2 & ⊢ & {\textbackslash}vdash & Right Tack \\
\hline
U+022A3 & ⊣ & {\textbackslash}dashv & Left Tack \\
\hline
U+022A4 & ⊤ & {\textbackslash}top & Down Tack \\
\hline
U+022A5 & ⊥ & {\textbackslash}bot & Up Tack \\
\hline
U+022A7 & ⊧ & {\textbackslash}models & Models \\
\hline
U+022A8 & ⊨ & {\textbackslash}vDash & True \\
\hline
U+022A9 & ⊩ & {\textbackslash}Vdash & Forces \\
\hline
U+022AA & ⊪ & {\textbackslash}Vvdash & Triple Vertical Bar Right Turnstile \\
\hline
U+022AB & ⊫ & {\textbackslash}VDash & Double Vertical Bar Double Right Turnstile \\
\hline
U+022AC & ⊬ & {\textbackslash}nvdash & Does Not Prove \\
\hline
U+022AD & ⊭ & {\textbackslash}nvDash & Not True \\
\hline
U+022AE & ⊮ & {\textbackslash}nVdash & Does Not Force \\
\hline
U+022AF & ⊯ & {\textbackslash}nVDash & Negated Double Vertical Bar Double Right Turnstile \\
\hline
U+022B0 & ⊰ & {\textbackslash}prurel & Precedes Under Relation \\
\hline
U+022B1 & ⊱ & {\textbackslash}scurel & Succeeds Under Relation \\
\hline
U+022B2 & ⊲ & {\textbackslash}vartriangleleft & Normal Subgroup Of \\
\hline
U+022B3 & ⊳ & {\textbackslash}vartriangleright & Contains As Normal Subgroup \\
\hline
U+022B4 & ⊴ & {\textbackslash}trianglelefteq & Normal Subgroup Of Or Equal To \\
\hline
U+022B5 & ⊵ & {\textbackslash}trianglerighteq & Contains As Normal Subgroup Or Equal To \\
\hline
U+022B6 & ⊶ & {\textbackslash}original & Original Of \\
\hline
U+022B7 & ⊷ & {\textbackslash}image & Image Of \\
\hline
U+022B8 & ⊸ & {\textbackslash}multimap & Multimap \\
\hline
U+022B9 & ⊹ & {\textbackslash}hermitconjmatrix & Hermitian Conjugate Matrix \\
\hline
U+022BA & ⊺ & {\textbackslash}intercal & Intercalate \\
\hline
U+022BB & ⊻ & {\textbackslash}veebar, {\textbackslash}xor & Xor \\
\hline
U+022BC & ⊼ & {\textbackslash}barwedge & Nand \\
\hline
U+022BD & ⊽ & {\textbackslash}barvee & Nor \\
\hline
U+022BE & ⊾ & {\textbackslash}rightanglearc & Right Angle With Arc \\
\hline
U+022BF & ⊿ & {\textbackslash}varlrtriangle & Right Triangle \\
\hline
U+022C0 & ⋀ & {\textbackslash}bigwedge & N-Ary Logical And \\
\hline
U+022C1 & ⋁ & {\textbackslash}bigvee & N-Ary Logical Or \\
\hline
U+022C2 & ⋂ & {\textbackslash}bigcap & N-Ary Intersection \\
\hline
U+022C3 & ⋃ & {\textbackslash}bigcup & N-Ary Union \\
\hline
U+022C4 & ⋄ & {\textbackslash}diamond & Diamond Operator \\
\hline
U+022C5 & ⋅ & {\textbackslash}cdot & Dot Operator \\
\hline
U+022C6 & ⋆ & {\textbackslash}star & Star Operator \\
\hline
U+022C7 & ⋇ & {\textbackslash}divideontimes & Division Times \\
\hline
U+022C8 & ⋈ & {\textbackslash}bowtie & Bowtie \\
\hline
U+022C9 & ⋉ & {\textbackslash}ltimes & Left Normal Factor Semidirect Product \\
\hline
U+022CA & ⋊ & {\textbackslash}rtimes & Right Normal Factor Semidirect Product \\
\hline
U+022CB & ⋋ & {\textbackslash}leftthreetimes & Left Semidirect Product \\
\hline
U+022CC & ⋌ & {\textbackslash}rightthreetimes & Right Semidirect Product \\
\hline
U+022CD & ⋍ & {\textbackslash}backsimeq & Reversed Tilde Equals \\
\hline
U+022CE & ⋎ & {\textbackslash}curlyvee & Curly Logical Or \\
\hline
U+022CF & ⋏ & {\textbackslash}curlywedge & Curly Logical And \\
\hline
U+022D0 & ⋐ & {\textbackslash}Subset & Double Subset \\
\hline
U+022D1 & ⋑ & {\textbackslash}Supset & Double Superset \\
\hline
U+022D2 & ⋒ & {\textbackslash}Cap & Double Intersection \\
\hline
U+022D3 & ⋓ & {\textbackslash}Cup & Double Union \\
\hline
U+022D4 & ⋔ & {\textbackslash}pitchfork & Pitchfork \\
\hline
U+022D5 & ⋕ & {\textbackslash}equalparallel & Equal And Parallel To \\
\hline
U+022D6 & ⋖ & {\textbackslash}lessdot & Less-Than With Dot / Less Than With Dot \\
\hline
U+022D7 & ⋗ & {\textbackslash}gtrdot & Greater-Than With Dot / Greater Than With Dot \\
\hline
U+022D8 & ⋘ & {\textbackslash}verymuchless & Very Much Less-Than / Very Much Less Than \\
\hline
U+022D9 & ⋙ & {\textbackslash}ggg & Very Much Greater-Than / Very Much Greater Than \\
\hline
U+022DA & ⋚ & {\textbackslash}lesseqgtr & Less-Than Equal To Or Greater-Than / Less Than Equal To Or Greater Than \\
\hline
U+022DB & ⋛ & {\textbackslash}gtreqless & Greater-Than Equal To Or Less-Than / Greater Than Equal To Or Less Than \\
\hline
U+022DC & ⋜ & {\textbackslash}eqless & Equal To Or Less-Than / Equal To Or Less Than \\
\hline
U+022DD & ⋝ & {\textbackslash}eqgtr & Equal To Or Greater-Than / Equal To Or Greater Than \\
\hline
U+022DE & ⋞ & {\textbackslash}curlyeqprec & Equal To Or Precedes \\
\hline
U+022DF & ⋟ & {\textbackslash}curlyeqsucc & Equal To Or Succeeds \\
\hline
U+022E0 & ⋠ & {\textbackslash}npreccurlyeq & Does Not Precede Or Equal \\
\hline
U+022E1 & ⋡ & {\textbackslash}nsucccurlyeq & Does Not Succeed Or Equal \\
\hline
U+022E2 & ⋢ & {\textbackslash}nsqsubseteq & Not Square Image Of Or Equal To \\
\hline
U+022E3 & ⋣ & {\textbackslash}nsqsupseteq & Not Square Original Of Or Equal To \\
\hline
U+022E4 & ⋤ & {\textbackslash}sqsubsetneq & Square Image Of Or Not Equal To \\
\hline
U+022E5 & ⋥ & {\textbackslash}sqspne & Square Original Of Or Not Equal To \\
\hline
U+022E6 & ⋦ & {\textbackslash}lnsim & Less-Than But Not Equivalent To / Less Than But Not Equivalent To \\
\hline
U+022E7 & ⋧ & {\textbackslash}gnsim & Greater-Than But Not Equivalent To / Greater Than But Not Equivalent To \\
\hline
U+022E8 & ⋨ & {\textbackslash}precnsim & Precedes But Not Equivalent To \\
\hline
U+022E9 & ⋩ & {\textbackslash}succnsim & Succeeds But Not Equivalent To \\
\hline
U+022EA & ⋪ & {\textbackslash}ntriangleleft & Not Normal Subgroup Of \\
\hline
U+022EB & ⋫ & {\textbackslash}ntriangleright & Does Not Contain As Normal Subgroup \\
\hline
U+022EC & ⋬ & {\textbackslash}ntrianglelefteq & Not Normal Subgroup Of Or Equal To \\
\hline
U+022ED & ⋭ & {\textbackslash}ntrianglerighteq & Does Not Contain As Normal Subgroup Or Equal \\
\hline
U+022EE & ⋮ & {\textbackslash}vdots & Vertical Ellipsis \\
\hline
U+022EF & ⋯ & {\textbackslash}cdots & Midline Horizontal Ellipsis \\
\hline
U+022F0 & ⋰ & {\textbackslash}adots & Up Right Diagonal Ellipsis \\
\hline
U+022F1 & ⋱ & {\textbackslash}ddots & Down Right Diagonal Ellipsis \\
\hline
U+022F2 & ⋲ & {\textbackslash}disin & Element Of With Long Horizontal Stroke \\
\hline
U+022F3 & ⋳ & {\textbackslash}varisins & Element Of With Vertical Bar At End Of Horizontal Stroke \\
\hline
U+022F4 & ⋴ & {\textbackslash}isins & Small Element Of With Vertical Bar At End Of Horizontal Stroke \\
\hline
U+022F5 & ⋵ & {\textbackslash}isindot & Element Of With Dot Above \\
\hline
U+022F6 & ⋶ & {\textbackslash}varisinobar & Element Of With Overbar \\
\hline
U+022F7 & ⋷ & {\textbackslash}isinobar & Small Element Of With Overbar \\
\hline
U+022F8 & ⋸ & {\textbackslash}isinvb & Element Of With Underbar \\
\hline
U+022F9 & ⋹ & {\textbackslash}isinE & Element Of With Two Horizontal Strokes \\
\hline
U+022FA & ⋺ & {\textbackslash}nisd & Contains With Long Horizontal Stroke \\
\hline
U+022FB & ⋻ & {\textbackslash}varnis & Contains With Vertical Bar At End Of Horizontal Stroke \\
\hline
U+022FC & ⋼ & {\textbackslash}nis & Small Contains With Vertical Bar At End Of Horizontal Stroke \\
\hline
U+022FD & ⋽ & {\textbackslash}varniobar & Contains With Overbar \\
\hline
U+022FE & ⋾ & {\textbackslash}niobar & Small Contains With Overbar \\
\hline
U+022FF & ⋿ & {\textbackslash}bagmember & Z Notation Bag Membership \\
\hline
U+02300 & ⌀ & {\textbackslash}diameter & Diameter Sign \\
\hline
U+02302 & ⌂ & {\textbackslash}house & House \\
\hline
U+02305 & ⌅ & {\textbackslash}varbarwedge & Projective \\
\hline
U+02306 & ⌆ & {\textbackslash}vardoublebarwedge & Perspective \\
\hline
U+02308 & ⌈ & {\textbackslash}lceil & Left Ceiling \\
\hline
U+02309 & ⌉ & {\textbackslash}rceil & Right Ceiling \\
\hline
U+0230A & ⌊ & {\textbackslash}lfloor & Left Floor \\
\hline
U+0230B & ⌋ & {\textbackslash}rfloor & Right Floor \\
\hline
U+02310 & ⌐ & {\textbackslash}invnot & Reversed Not Sign \\
\hline
U+02311 & ⌑ & {\textbackslash}sqlozenge & Square Lozenge \\
\hline
U+02312 & ⌒ & {\textbackslash}profline & Arc \\
\hline
U+02313 & ⌓ & {\textbackslash}profsurf & Segment \\
\hline
U+02315 & ⌕ & {\textbackslash}recorder & Telephone Recorder \\
\hline
U+02317 & ⌗ & {\textbackslash}viewdata & Viewdata Square \\
\hline
U+02319 & ⌙ & {\textbackslash}turnednot & Turned Not Sign \\
\hline
U+0231A & ⌚ & {\textbackslash}:watch: & Watch \\
\hline
U+0231B & ⌛ & {\textbackslash}:hourglass: & Hourglass \\
\hline
U+0231C & ⌜ & {\textbackslash}ulcorner & Top Left Corner \\
\hline
U+0231D & ⌝ & {\textbackslash}urcorner & Top Right Corner \\
\hline
U+0231E & ⌞ & {\textbackslash}llcorner & Bottom Left Corner \\
\hline
U+0231F & ⌟ & {\textbackslash}lrcorner & Bottom Right Corner \\
\hline
U+02322 & ⌢ & {\textbackslash}frown & Frown \\
\hline
U+02323 & ⌣ & {\textbackslash}smile & Smile \\
\hline
U+0232C & ⌬ & {\textbackslash}varhexagonlrbonds & Benzene Ring \\
\hline
U+02332 & ⌲ & {\textbackslash}conictaper & Conical Taper \\
\hline
U+02336 & ⌶ & {\textbackslash}topbot & Apl Functional Symbol I-Beam \\
\hline
U+0233D & ⌽ & {\textbackslash}obar & Apl Functional Symbol Circle Stile \\
\hline
U+0233F & ⌿ & {\textbackslash}notslash & Apl Functional Symbol Slash Bar \\
\hline
U+02340 & ⍀ & {\textbackslash}notbackslash & Apl Functional Symbol Backslash Bar \\
\hline
U+02353 & ⍓ & {\textbackslash}boxupcaret & Apl Functional Symbol Quad Up Caret \\
\hline
U+02370 & ⍰ & {\textbackslash}boxquestion & Apl Functional Symbol Quad Question \\
\hline
U+02394 & ⎔ & {\textbackslash}hexagon & Software-Function Symbol \\
\hline
U+023A3 & ⎣ & {\textbackslash}dlcorn & Left Square Bracket Lower Corner \\
\hline
U+023B0 & ⎰ & {\textbackslash}lmoustache & Upper Left Or Lower Right Curly Bracket Section \\
\hline
U+023B1 & ⎱ & {\textbackslash}rmoustache & Upper Right Or Lower Left Curly Bracket Section \\
\hline
U+023B4 & ⎴ & {\textbackslash}overbracket & Top Square Bracket \\
\hline
U+023B5 & ⎵ & {\textbackslash}underbracket & Bottom Square Bracket \\
\hline
U+023B6 & ⎶ & {\textbackslash}bbrktbrk & Bottom Square Bracket Over Top Square Bracket \\
\hline
U+023B7 & ⎷ & {\textbackslash}sqrtbottom & Radical Symbol Bottom \\
\hline
U+023B8 & ⎸ & {\textbackslash}lvboxline & Left Vertical Box Line \\
\hline
U+023B9 & ⎹ & {\textbackslash}rvboxline & Right Vertical Box Line \\
\hline
U+023CE & ⏎ & {\textbackslash}varcarriagereturn & Return Symbol \\
\hline
U+023DE & ⏞ & {\textbackslash}overbrace & Top Curly Bracket \\
\hline
U+023DF & ⏟ & {\textbackslash}underbrace & Bottom Curly Bracket \\
\hline
U+023E2 & ⏢ & {\textbackslash}trapezium & White Trapezium \\
\hline
U+023E3 & ⏣ & {\textbackslash}benzenr & Benzene Ring With Circle \\
\hline
U+023E4 & ⏤ & {\textbackslash}strns & Straightness \\
\hline
U+023E5 & ⏥ & {\textbackslash}fltns & Flatness \\
\hline
U+023E6 & ⏦ & {\textbackslash}accurrent & Ac Current \\
\hline
U+023E7 & ⏧ & {\textbackslash}elinters & Electrical Intersection \\
\hline
U+023E9 & ⏩ & {\textbackslash}:fast\_forward: & Black Right-Pointing Double Triangle \\
\hline
U+023EA & ⏪ & {\textbackslash}:rewind: & Black Left-Pointing Double Triangle \\
\hline
U+023EB & ⏫ & {\textbackslash}:arrow\_double\_up: & Black Up-Pointing Double Triangle \\
\hline
U+023EC & ⏬ & {\textbackslash}:arrow\_double\_down: & Black Down-Pointing Double Triangle \\
\hline
U+023F0 & ⏰ & {\textbackslash}:alarm\_clock: & Alarm Clock \\
\hline
U+023F3 & ⏳ & {\textbackslash}:hourglass\_flowing\_sand: & Hourglass With Flowing Sand \\
\hline
U+02422 & ␢ & {\textbackslash}blanksymbol & Blank Symbol / Blank \\
\hline
U+02423 & ␣ & {\textbackslash}visiblespace & Open Box \\
\hline
U+024C2 & Ⓜ & {\textbackslash}:m: & Circled Latin Capital Letter M \\
\hline
U+024C8 & Ⓢ & {\textbackslash}circledS & Circled Latin Capital Letter S \\
\hline
U+02506 & ┆ & {\textbackslash}dshfnc & Box Drawings Light Triple Dash Vertical / Forms Light Triple Dash Vertical \\
\hline
U+02519 & ┙ & {\textbackslash}sqfnw & Box Drawings Up Light And Left Heavy / Forms Up Light And Left Heavy \\
\hline
U+02571 & ╱ & {\textbackslash}diagup & Box Drawings Light Diagonal Upper Right To Lower Left / Forms Light Diagonal Upper Right To Lower Left \\
\hline
U+02572 & ╲ & {\textbackslash}diagdown & Box Drawings Light Diagonal Upper Left To Lower Right / Forms Light Diagonal Upper Left To Lower Right \\
\hline
U+02580 & ▀ & {\textbackslash}blockuphalf & Upper Half Block \\
\hline
U+02584 & ▄ & {\textbackslash}blocklowhalf & Lower Half Block \\
\hline
U+02588 & █ & {\textbackslash}blockfull & Full Block \\
\hline
U+0258C & ▌ & {\textbackslash}blocklefthalf & Left Half Block \\
\hline
U+02590 & ▐ & {\textbackslash}blockrighthalf & Right Half Block \\
\hline
U+02591 & ░ & {\textbackslash}blockqtrshaded & Light Shade \\
\hline
U+02592 & ▒ & {\textbackslash}blockhalfshaded & Medium Shade \\
\hline
U+02593 & ▓ & {\textbackslash}blockthreeqtrshaded & Dark Shade \\
\hline
U+025A0 & ■ & {\textbackslash}blacksquare & Black Square \\
\hline
U+025A1 & □ & {\textbackslash}square & White Square \\
\hline
U+025A2 & ▢ & {\textbackslash}squoval & White Square With Rounded Corners \\
\hline
U+025A3 & ▣ & {\textbackslash}blackinwhitesquare & White Square Containing Black Small Square \\
\hline
U+025A4 & ▤ & {\textbackslash}squarehfill & Square With Horizontal Fill \\
\hline
U+025A5 & ▥ & {\textbackslash}squarevfill & Square With Vertical Fill \\
\hline
U+025A6 & ▦ & {\textbackslash}squarehvfill & Square With Orthogonal Crosshatch Fill \\
\hline
U+025A7 & ▧ & {\textbackslash}squarenwsefill & Square With Upper Left To Lower Right Fill \\
\hline
U+025A8 & ▨ & {\textbackslash}squareneswfill & Square With Upper Right To Lower Left Fill \\
\hline
U+025A9 & ▩ & {\textbackslash}squarecrossfill & Square With Diagonal Crosshatch Fill \\
\hline
U+025AA & ▪ & {\textbackslash}smblksquare, {\textbackslash}:black\_small\_square: & Black Small Square \\
\hline
U+025AB & ▫ & {\textbackslash}smwhtsquare, {\textbackslash}:white\_small\_square: & White Small Square \\
\hline
U+025AC & ▬ & {\textbackslash}hrectangleblack & Black Rectangle \\
\hline
U+025AD & ▭ & {\textbackslash}hrectangle & White Rectangle \\
\hline
U+025AE & ▮ & {\textbackslash}vrectangleblack & Black Vertical Rectangle \\
\hline
U+025AF & ▯ & {\textbackslash}vrecto & White Vertical Rectangle \\
\hline
U+025B0 & ▰ & {\textbackslash}parallelogramblack & Black Parallelogram \\
\hline
U+025B1 & ▱ & {\textbackslash}parallelogram & White Parallelogram \\
\hline
U+025B2 & ▲ & {\textbackslash}bigblacktriangleup & Black Up-Pointing Triangle / Black Up Pointing Triangle \\
\hline
U+025B3 & △ & {\textbackslash}bigtriangleup & White Up-Pointing Triangle / White Up Pointing Triangle \\
\hline
U+025B4 & ▴ & {\textbackslash}blacktriangle & Black Up-Pointing Small Triangle / Black Up Pointing Small Triangle \\
\hline
U+025B5 & ▵ & {\textbackslash}vartriangle & White Up-Pointing Small Triangle / White Up Pointing Small Triangle \\
\hline
U+025B6 & ▶ & {\textbackslash}blacktriangleright, {\textbackslash}:arrow\_forward: & Black Right-Pointing Triangle / Black Right Pointing Triangle \\
\hline
U+025B7 & ▷ & {\textbackslash}triangleright & White Right-Pointing Triangle / White Right Pointing Triangle \\
\hline
U+025B8 & ▸ & {\textbackslash}smallblacktriangleright & Black Right-Pointing Small Triangle / Black Right Pointing Small Triangle \\
\hline
U+025B9 & ▹ & {\textbackslash}smalltriangleright & White Right-Pointing Small Triangle / White Right Pointing Small Triangle \\
\hline
U+025BA & ► & {\textbackslash}blackpointerright & Black Right-Pointing Pointer / Black Right Pointing Pointer \\
\hline
U+025BB & ▻ & {\textbackslash}whitepointerright & White Right-Pointing Pointer / White Right Pointing Pointer \\
\hline
U+025BC & ▼ & {\textbackslash}bigblacktriangledown & Black Down-Pointing Triangle / Black Down Pointing Triangle \\
\hline
U+025BD & ▽ & {\textbackslash}bigtriangledown & White Down-Pointing Triangle / White Down Pointing Triangle \\
\hline
U+025BE & ▾ & {\textbackslash}blacktriangledown & Black Down-Pointing Small Triangle / Black Down Pointing Small Triangle \\
\hline
U+025BF & ▿ & {\textbackslash}triangledown & White Down-Pointing Small Triangle / White Down Pointing Small Triangle \\
\hline
U+025C0 & ◀ & {\textbackslash}blacktriangleleft, {\textbackslash}:arrow\_backward: & Black Left-Pointing Triangle / Black Left Pointing Triangle \\
\hline
U+025C1 & ◁ & {\textbackslash}triangleleft & White Left-Pointing Triangle / White Left Pointing Triangle \\
\hline
U+025C2 & ◂ & {\textbackslash}smallblacktriangleleft & Black Left-Pointing Small Triangle / Black Left Pointing Small Triangle \\
\hline
U+025C3 & ◃ & {\textbackslash}smalltriangleleft & White Left-Pointing Small Triangle / White Left Pointing Small Triangle \\
\hline
U+025C4 & ◄ & {\textbackslash}blackpointerleft & Black Left-Pointing Pointer / Black Left Pointing Pointer \\
\hline
U+025C5 & ◅ & {\textbackslash}whitepointerleft & White Left-Pointing Pointer / White Left Pointing Pointer \\
\hline
U+025C6 & ◆ & {\textbackslash}mdlgblkdiamond & Black Diamond \\
\hline
U+025C7 & ◇ & {\textbackslash}mdlgwhtdiamond & White Diamond \\
\hline
U+025C8 & ◈ & {\textbackslash}blackinwhitediamond & White Diamond Containing Black Small Diamond \\
\hline
U+025C9 & ◉ & {\textbackslash}fisheye & Fisheye \\
\hline
U+025CA & ◊ & {\textbackslash}lozenge & Lozenge \\
\hline
U+025CB & ○ & {\textbackslash}bigcirc & White Circle \\
\hline
U+025CC & ◌ & {\textbackslash}dottedcircle & Dotted Circle \\
\hline
U+025CD & ◍ & {\textbackslash}circlevertfill & Circle With Vertical Fill \\
\hline
U+025CE & ◎ & {\textbackslash}bullseye & Bullseye \\
\hline
U+025CF & ● & {\textbackslash}mdlgblkcircle & Black Circle \\
\hline
U+025D0 & ◐ & {\textbackslash}cirfl & Circle With Left Half Black \\
\hline
U+025D1 & ◑ & {\textbackslash}cirfr & Circle With Right Half Black \\
\hline
U+025D2 & ◒ & {\textbackslash}cirfb & Circle With Lower Half Black \\
\hline
U+025D3 & ◓ & {\textbackslash}circletophalfblack & Circle With Upper Half Black \\
\hline
U+025D4 & ◔ & {\textbackslash}circleurquadblack & Circle With Upper Right Quadrant Black \\
\hline
U+025D5 & ◕ & {\textbackslash}blackcircleulquadwhite & Circle With All But Upper Left Quadrant Black \\
\hline
U+025D6 & ◖ & {\textbackslash}blacklefthalfcircle & Left Half Black Circle \\
\hline
U+025D7 & ◗ & {\textbackslash}blackrighthalfcircle & Right Half Black Circle \\
\hline
U+025D8 & ◘ & {\textbackslash}rvbull & Inverse Bullet \\
\hline
U+025D9 & ◙ & {\textbackslash}inversewhitecircle & Inverse White Circle \\
\hline
U+025DA & ◚ & {\textbackslash}invwhiteupperhalfcircle & Upper Half Inverse White Circle \\
\hline
U+025DB & ◛ & {\textbackslash}invwhitelowerhalfcircle & Lower Half Inverse White Circle \\
\hline
U+025DC & ◜ & {\textbackslash}ularc & Upper Left Quadrant Circular Arc \\
\hline
U+025DD & ◝ & {\textbackslash}urarc & Upper Right Quadrant Circular Arc \\
\hline
U+025DE & ◞ & {\textbackslash}lrarc & Lower Right Quadrant Circular Arc \\
\hline
U+025DF & ◟ & {\textbackslash}llarc & Lower Left Quadrant Circular Arc \\
\hline
U+025E0 & ◠ & {\textbackslash}topsemicircle & Upper Half Circle \\
\hline
U+025E1 & ◡ & {\textbackslash}botsemicircle & Lower Half Circle \\
\hline
U+025E2 & ◢ & {\textbackslash}lrblacktriangle & Black Lower Right Triangle \\
\hline
U+025E3 & ◣ & {\textbackslash}llblacktriangle & Black Lower Left Triangle \\
\hline
U+025E4 & ◤ & {\textbackslash}ulblacktriangle & Black Upper Left Triangle \\
\hline
U+025E5 & ◥ & {\textbackslash}urblacktriangle & Black Upper Right Triangle \\
\hline
U+025E6 & ◦ & {\textbackslash}smwhtcircle & White Bullet \\
\hline
U+025E7 & ◧ & {\textbackslash}sqfl & Square With Left Half Black \\
\hline
U+025E8 & ◨ & {\textbackslash}sqfr & Square With Right Half Black \\
\hline
U+025E9 & ◩ & {\textbackslash}squareulblack & Square With Upper Left Diagonal Half Black \\
\hline
U+025EA & ◪ & {\textbackslash}sqfse & Square With Lower Right Diagonal Half Black \\
\hline
U+025EB & ◫ & {\textbackslash}boxbar & White Square With Vertical Bisecting Line \\
\hline
U+025EC & ◬ & {\textbackslash}trianglecdot & White Up-Pointing Triangle With Dot / White Up Pointing Triangle With Dot \\
\hline
U+025ED & ◭ & {\textbackslash}triangleleftblack & Up-Pointing Triangle With Left Half Black / Up Pointing Triangle With Left Half Black \\
\hline
U+025EE & ◮ & {\textbackslash}trianglerightblack & Up-Pointing Triangle With Right Half Black / Up Pointing Triangle With Right Half Black \\
\hline
U+025EF & ◯ & {\textbackslash}lgwhtcircle & Large Circle \\
\hline
U+025F0 & ◰ & {\textbackslash}squareulquad & White Square With Upper Left Quadrant \\
\hline
U+025F1 & ◱ & {\textbackslash}squarellquad & White Square With Lower Left Quadrant \\
\hline
U+025F2 & ◲ & {\textbackslash}squarelrquad & White Square With Lower Right Quadrant \\
\hline
U+025F3 & ◳ & {\textbackslash}squareurquad & White Square With Upper Right Quadrant \\
\hline
U+025F4 & ◴ & {\textbackslash}circleulquad & White Circle With Upper Left Quadrant \\
\hline
U+025F5 & ◵ & {\textbackslash}circlellquad & White Circle With Lower Left Quadrant \\
\hline
U+025F6 & ◶ & {\textbackslash}circlelrquad & White Circle With Lower Right Quadrant \\
\hline
U+025F7 & ◷ & {\textbackslash}circleurquad & White Circle With Upper Right Quadrant \\
\hline
U+025F8 & ◸ & {\textbackslash}ultriangle & Upper Left Triangle \\
\hline
U+025F9 & ◹ & {\textbackslash}urtriangle & Upper Right Triangle \\
\hline
U+025FA & ◺ & {\textbackslash}lltriangle & Lower Left Triangle \\
\hline
U+025FB & ◻ & {\textbackslash}mdwhtsquare, {\textbackslash}:white\_medium\_square: & White Medium Square \\
\hline
U+025FC & ◼ & {\textbackslash}mdblksquare, {\textbackslash}:black\_medium\_square: & Black Medium Square \\
\hline
U+025FD & ◽ & {\textbackslash}mdsmwhtsquare, {\textbackslash}:white\_medium\_small\_square: & White Medium Small Square \\
\hline
U+025FE & ◾ & {\textbackslash}mdsmblksquare, {\textbackslash}:black\_medium\_small\_square: & Black Medium Small Square \\
\hline
U+025FF & ◿ & {\textbackslash}lrtriangle & Lower Right Triangle \\
\hline
U+02600 & ☀ & {\textbackslash}:sunny: & Black Sun With Rays \\
\hline
U+02601 & ☁ & {\textbackslash}:cloud: & Cloud \\
\hline
U+02605 & ★ & {\textbackslash}bigstar & Black Star \\
\hline
U+02606 & ☆ & {\textbackslash}bigwhitestar & White Star \\
\hline
U+02609 & ☉ & {\textbackslash}astrosun & Sun \\
\hline
U+0260E & ☎ & {\textbackslash}:phone: & Black Telephone \\
\hline
U+02611 & ☑ & {\textbackslash}:ballot\_box\_with\_check: & Ballot Box With Check \\
\hline
U+02614 & ☔ & {\textbackslash}:umbrella: & Umbrella With Rain Drops \\
\hline
U+02615 & ☕ & {\textbackslash}:coffee: & Hot Beverage \\
\hline
U+0261D & ☝ & {\textbackslash}:point\_up: & White Up Pointing Index \\
\hline
U+02621 & ☡ & {\textbackslash}danger & Caution Sign \\
\hline
U+0263A & ☺ & {\textbackslash}:relaxed: & White Smiling Face \\
\hline
U+0263B & ☻ & {\textbackslash}blacksmiley & Black Smiling Face \\
\hline
U+0263C & ☼ & {\textbackslash}sun & White Sun With Rays \\
\hline
U+0263D & ☽ & {\textbackslash}rightmoon & First Quarter Moon \\
\hline
U+0263E & ☾ & {\textbackslash}leftmoon & Last Quarter Moon \\
\hline
U+0263F & ☿ & {\textbackslash}mercury & Mercury \\
\hline
U+02640 & ♀ & {\textbackslash}venus, {\textbackslash}female & Female Sign \\
\hline
U+02642 & ♂ & {\textbackslash}male, {\textbackslash}mars & Male Sign \\
\hline
U+02643 & ♃ & {\textbackslash}jupiter & Jupiter \\
\hline
U+02644 & ♄ & {\textbackslash}saturn & Saturn \\
\hline
U+02645 & ♅ & {\textbackslash}uranus & Uranus \\
\hline
U+02646 & ♆ & {\textbackslash}neptune & Neptune \\
\hline
U+02647 & ♇ & {\textbackslash}pluto & Pluto \\
\hline
U+02648 & ♈ & {\textbackslash}aries, {\textbackslash}:aries: & Aries \\
\hline
U+02649 & ♉ & {\textbackslash}taurus, {\textbackslash}:taurus: & Taurus \\
\hline
U+0264A & ♊ & {\textbackslash}gemini, {\textbackslash}:gemini: & Gemini \\
\hline
U+0264B & ♋ & {\textbackslash}cancer, {\textbackslash}:cancer: & Cancer \\
\hline
U+0264C & ♌ & {\textbackslash}leo, {\textbackslash}:leo: & Leo \\
\hline
U+0264D & ♍ & {\textbackslash}virgo, {\textbackslash}:virgo: & Virgo \\
\hline
U+0264E & ♎ & {\textbackslash}libra, {\textbackslash}:libra: & Libra \\
\hline
U+0264F & ♏ & {\textbackslash}scorpio, {\textbackslash}:scorpius: & Scorpius \\
\hline
U+02650 & ♐ & {\textbackslash}sagittarius, {\textbackslash}:sagittarius: & Sagittarius \\
\hline
U+02651 & ♑ & {\textbackslash}capricornus, {\textbackslash}:capricorn: & Capricorn \\
\hline
U+02652 & ♒ & {\textbackslash}aquarius, {\textbackslash}:aquarius: & Aquarius \\
\hline
U+02653 & ♓ & {\textbackslash}pisces, {\textbackslash}:pisces: & Pisces \\
\hline
U+02660 & ♠ & {\textbackslash}spadesuit, {\textbackslash}:spades: & Black Spade Suit \\
\hline
U+02661 & ♡ & {\textbackslash}heartsuit & White Heart Suit \\
\hline
U+02662 & ♢ & {\textbackslash}diamondsuit & White Diamond Suit \\
\hline
U+02663 & ♣ & {\textbackslash}clubsuit, {\textbackslash}:clubs: & Black Club Suit \\
\hline
U+02664 & ♤ & {\textbackslash}varspadesuit & White Spade Suit \\
\hline
U+02665 & ♥ & {\textbackslash}varheartsuit, {\textbackslash}:hearts: & Black Heart Suit \\
\hline
U+02666 & ♦ & {\textbackslash}vardiamondsuit, {\textbackslash}:diamonds: & Black Diamond Suit \\
\hline
U+02667 & ♧ & {\textbackslash}varclubsuit & White Club Suit \\
\hline
U+02668 & ♨ & {\textbackslash}:hotsprings: & Hot Springs \\
\hline
U+02669 & ♩ & {\textbackslash}quarternote & Quarter Note \\
\hline
U+0266A & ♪ & {\textbackslash}eighthnote & Eighth Note \\
\hline
U+0266B & ♫ & {\textbackslash}twonotes & Beamed Eighth Notes / Barred Eighth Notes \\
\hline
U+0266D & ♭ & {\textbackslash}flat & Music Flat Sign / Flat \\
\hline
U+0266E & ♮ & {\textbackslash}natural & Music Natural Sign / Natural \\
\hline
U+0266F & ♯ & {\textbackslash}sharp & Music Sharp Sign / Sharp \\
\hline
U+0267B & ♻ & {\textbackslash}:recycle: & Black Universal Recycling Symbol \\
\hline
U+0267E & ♾ & {\textbackslash}acidfree & Permanent Paper Sign \\
\hline
U+0267F & ♿ & {\textbackslash}:wheelchair: & Wheelchair Symbol \\
\hline
U+02680 & ⚀ & {\textbackslash}dicei & Die Face-1 \\
\hline
U+02681 & ⚁ & {\textbackslash}diceii & Die Face-2 \\
\hline
U+02682 & ⚂ & {\textbackslash}diceiii & Die Face-3 \\
\hline
U+02683 & ⚃ & {\textbackslash}diceiv & Die Face-4 \\
\hline
U+02684 & ⚄ & {\textbackslash}dicev & Die Face-5 \\
\hline
U+02685 & ⚅ & {\textbackslash}dicevi & Die Face-6 \\
\hline
U+02686 & ⚆ & {\textbackslash}circledrightdot & White Circle With Dot Right \\
\hline
U+02687 & ⚇ & {\textbackslash}circledtwodots & White Circle With Two Dots \\
\hline
U+02688 & ⚈ & {\textbackslash}blackcircledrightdot & Black Circle With White Dot Right \\
\hline
U+02689 & ⚉ & {\textbackslash}blackcircledtwodots & Black Circle With Two White Dots \\
\hline
U+02693 & ⚓ & {\textbackslash}:anchor: & Anchor \\
\hline
U+026A0 & ⚠ & {\textbackslash}:warning: & Warning Sign \\
\hline
U+026A1 & ⚡ & {\textbackslash}:zap: & High Voltage Sign \\
\hline
U+026A5 & ⚥ & {\textbackslash}hermaphrodite & Male And Female Sign \\
\hline
U+026AA & ⚪ & {\textbackslash}mdwhtcircle, {\textbackslash}:white\_circle: & Medium White Circle \\
\hline
U+026AB & ⚫ & {\textbackslash}mdblkcircle, {\textbackslash}:black\_circle: & Medium Black Circle \\
\hline
U+026AC & ⚬ & {\textbackslash}mdsmwhtcircle & Medium Small White Circle \\
\hline
U+026B2 & ⚲ & {\textbackslash}neuter & Neuter \\
\hline
U+026BD & ⚽ & {\textbackslash}:soccer: & Soccer Ball \\
\hline
U+026BE & ⚾ & {\textbackslash}:baseball: & Baseball \\
\hline
U+026C4 & ⛄ & {\textbackslash}:snowman: & Snowman Without Snow \\
\hline
U+026C5 & ⛅ & {\textbackslash}:partly\_sunny: & Sun Behind Cloud \\
\hline
U+026CE & ⛎ & {\textbackslash}:ophiuchus: & Ophiuchus \\
\hline
U+026D4 & ⛔ & {\textbackslash}:no\_entry: & No Entry \\
\hline
U+026EA & ⛪ & {\textbackslash}:church: & Church \\
\hline
U+026F2 & ⛲ & {\textbackslash}:fountain: & Fountain \\
\hline
U+026F3 & ⛳ & {\textbackslash}:golf: & Flag In Hole \\
\hline
U+026F5 & ⛵ & {\textbackslash}:boat: & Sailboat \\
\hline
U+026FA & ⛺ & {\textbackslash}:tent: & Tent \\
\hline
U+026FD & ⛽ & {\textbackslash}:fuelpump: & Fuel Pump \\
\hline
U+02702 & ✂ & {\textbackslash}:scissors: & Black Scissors \\
\hline
U+02705 & ✅ & {\textbackslash}:white\_check\_mark: & White Heavy Check Mark \\
\hline
U+02708 & ✈ & {\textbackslash}:airplane: & Airplane \\
\hline
U+02709 & ✉ & {\textbackslash}:email: & Envelope \\
\hline
U+0270A & ✊ & {\textbackslash}:fist: & Raised Fist \\
\hline
U+0270B & ✋ & {\textbackslash}:hand: & Raised Hand \\
\hline
U+0270C & ✌ & {\textbackslash}:v: & Victory Hand \\
\hline
U+0270F & ✏ & {\textbackslash}:pencil2: & Pencil \\
\hline
U+02712 & ✒ & {\textbackslash}:black\_nib: & Black Nib \\
\hline
U+02713 & ✓ & {\textbackslash}checkmark & Check Mark \\
\hline
U+02714 & ✔ & {\textbackslash}:heavy\_check\_mark: & Heavy Check Mark \\
\hline
U+02716 & ✖ & {\textbackslash}:heavy\_multiplication\_x: & Heavy Multiplication X \\
\hline
U+02720 & ✠ & {\textbackslash}maltese & Maltese Cross \\
\hline
U+02728 & ✨ & {\textbackslash}:sparkles: & Sparkles \\
\hline
U+0272A & ✪ & {\textbackslash}circledstar & Circled White Star \\
\hline
U+02733 & ✳ & {\textbackslash}:eight\_spoked\_asterisk: & Eight Spoked Asterisk \\
\hline
U+02734 & ✴ & {\textbackslash}:eight\_pointed\_black\_star: & Eight Pointed Black Star \\
\hline
U+02736 & ✶ & {\textbackslash}varstar & Six Pointed Black Star \\
\hline
U+0273D & ✽ & {\textbackslash}dingasterisk & Heavy Teardrop-Spoked Asterisk \\
\hline
U+02744 & ❄ & {\textbackslash}:snowflake: & Snowflake \\
\hline
U+02747 & ❇ & {\textbackslash}:sparkle: & Sparkle \\
\hline
U+0274C & ❌ & {\textbackslash}:x: & Cross Mark \\
\hline
U+0274E & ❎ & {\textbackslash}:negative\_squared\_cross\_mark: & Negative Squared Cross Mark \\
\hline
U+02753 & ❓ & {\textbackslash}:question: & Black Question Mark Ornament \\
\hline
U+02754 & ❔ & {\textbackslash}:grey\_question: & White Question Mark Ornament \\
\hline
U+02755 & ❕ & {\textbackslash}:grey\_exclamation: & White Exclamation Mark Ornament \\
\hline
U+02757 & ❗ & {\textbackslash}:exclamation: & Heavy Exclamation Mark Symbol \\
\hline
U+02764 & ❤ & {\textbackslash}:heart: & Heavy Black Heart \\
\hline
U+02795 & ➕ & {\textbackslash}:heavy\_plus\_sign: & Heavy Plus Sign \\
\hline
U+02796 & ➖ & {\textbackslash}:heavy\_minus\_sign: & Heavy Minus Sign \\
\hline
U+02797 & ➗ & {\textbackslash}:heavy\_division\_sign: & Heavy Division Sign \\
\hline
U+0279B & ➛ & {\textbackslash}draftingarrow & Drafting Point Rightwards Arrow / Drafting Point Right Arrow \\
\hline
U+027A1 & ➡ & {\textbackslash}:arrow\_right: & Black Rightwards Arrow / Black Right Arrow \\
\hline
U+027B0 & ➰ & {\textbackslash}:curly\_loop: & Curly Loop \\
\hline
U+027BF & ➿ & {\textbackslash}:loop: & Double Curly Loop \\
\hline
U+027C0 & ⟀ & {\textbackslash}threedangle & Three Dimensional Angle \\
\hline
U+027C1 & ⟁ & {\textbackslash}whiteinwhitetriangle & White Triangle Containing Small White Triangle \\
\hline
U+027C2 & ⟂ & {\textbackslash}perp & Perpendicular \\
\hline
U+027C8 & ⟈ & {\textbackslash}bsolhsub & Reverse Solidus Preceding Subset \\
\hline
U+027C9 & ⟉ & {\textbackslash}suphsol & Superset Preceding Solidus \\
\hline
U+027D1 & ⟑ & {\textbackslash}wedgedot & And With Dot \\
\hline
U+027D2 & ⟒ & {\textbackslash}upin & Element Of Opening Upwards \\
\hline
U+027D5 & ⟕ & {\textbackslash}leftouterjoin & Left Outer Join \\
\hline
U+027D6 & ⟖ & {\textbackslash}rightouterjoin & Right Outer Join \\
\hline
U+027D7 & ⟗ & {\textbackslash}fullouterjoin & Full Outer Join \\
\hline
U+027D8 & ⟘ & {\textbackslash}bigbot & Large Up Tack \\
\hline
U+027D9 & ⟙ & {\textbackslash}bigtop & Large Down Tack \\
\hline
U+027E6 & ⟦ & {\textbackslash}llbracket, {\textbackslash}openbracketleft & Mathematical Left White Square Bracket \\
\hline
U+027E7 & ⟧ & {\textbackslash}openbracketright, {\textbackslash}rrbracket & Mathematical Right White Square Bracket \\
\hline
U+027E8 & ⟨ & {\textbackslash}langle & Mathematical Left Angle Bracket \\
\hline
U+027E9 & ⟩ & {\textbackslash}rangle & Mathematical Right Angle Bracket \\
\hline
U+027F0 & ⟰ & {\textbackslash}UUparrow & Upwards Quadruple Arrow \\
\hline
U+027F1 & ⟱ & {\textbackslash}DDownarrow & Downwards Quadruple Arrow \\
\hline
U+027F5 & ⟵ & {\textbackslash}longleftarrow & Long Leftwards Arrow \\
\hline
U+027F6 & ⟶ & {\textbackslash}longrightarrow & Long Rightwards Arrow \\
\hline
U+027F7 & ⟷ & {\textbackslash}longleftrightarrow & Long Left Right Arrow \\
\hline
U+027F8 & ⟸ & {\textbackslash}impliedby, {\textbackslash}Longleftarrow & Long Leftwards Double Arrow \\
\hline
U+027F9 & ⟹ & {\textbackslash}implies, {\textbackslash}Longrightarrow & Long Rightwards Double Arrow \\
\hline
U+027FA & ⟺ & {\textbackslash}Longleftrightarrow, {\textbackslash}iff & Long Left Right Double Arrow \\
\hline
U+027FB & ⟻ & {\textbackslash}longmapsfrom & Long Leftwards Arrow From Bar \\
\hline
U+027FC & ⟼ & {\textbackslash}longmapsto & Long Rightwards Arrow From Bar \\
\hline
U+027FD & ⟽ & {\textbackslash}Longmapsfrom & Long Leftwards Double Arrow From Bar \\
\hline
U+027FE & ⟾ & {\textbackslash}Longmapsto & Long Rightwards Double Arrow From Bar \\
\hline
U+027FF & ⟿ & {\textbackslash}longrightsquigarrow & Long Rightwards Squiggle Arrow \\
\hline
U+02900 & ⤀ & {\textbackslash}nvtwoheadrightarrow & Rightwards Two-Headed Arrow With Vertical Stroke \\
\hline
U+02901 & ⤁ & {\textbackslash}nVtwoheadrightarrow & Rightwards Two-Headed Arrow With Double Vertical Stroke \\
\hline
U+02902 & ⤂ & {\textbackslash}nvLeftarrow & Leftwards Double Arrow With Vertical Stroke \\
\hline
U+02903 & ⤃ & {\textbackslash}nvRightarrow & Rightwards Double Arrow With Vertical Stroke \\
\hline
U+02904 & ⤄ & {\textbackslash}nvLeftrightarrow & Left Right Double Arrow With Vertical Stroke \\
\hline
U+02905 & ⤅ & {\textbackslash}twoheadmapsto & Rightwards Two-Headed Arrow From Bar \\
\hline
U+02906 & ⤆ & {\textbackslash}Mapsfrom & Leftwards Double Arrow From Bar \\
\hline
U+02907 & ⤇ & {\textbackslash}Mapsto & Rightwards Double Arrow From Bar \\
\hline
U+02908 & ⤈ & {\textbackslash}downarrowbarred & Downwards Arrow With Horizontal Stroke \\
\hline
U+02909 & ⤉ & {\textbackslash}uparrowbarred & Upwards Arrow With Horizontal Stroke \\
\hline
U+0290A & ⤊ & {\textbackslash}Uuparrow & Upwards Triple Arrow \\
\hline
U+0290B & ⤋ & {\textbackslash}Ddownarrow & Downwards Triple Arrow \\
\hline
U+0290C & ⤌ & {\textbackslash}leftbkarrow & Leftwards Double Dash Arrow \\
\hline
U+0290D & ⤍ & {\textbackslash}bkarow & Rightwards Double Dash Arrow \\
\hline
U+0290E & ⤎ & {\textbackslash}leftdbkarrow & Leftwards Triple Dash Arrow \\
\hline
U+0290F & ⤏ & {\textbackslash}dbkarow & Rightwards Triple Dash Arrow \\
\hline
U+02910 & ⤐ & {\textbackslash}drbkarrow & Rightwards Two-Headed Triple Dash Arrow \\
\hline
U+02911 & ⤑ & {\textbackslash}rightdotarrow & Rightwards Arrow With Dotted Stem \\
\hline
U+02912 & ⤒ & {\textbackslash}UpArrowBar & Upwards Arrow To Bar \\
\hline
U+02913 & ⤓ & {\textbackslash}DownArrowBar & Downwards Arrow To Bar \\
\hline
U+02914 & ⤔ & {\textbackslash}nvrightarrowtail & Rightwards Arrow With Tail With Vertical Stroke \\
\hline
U+02915 & ⤕ & {\textbackslash}nVrightarrowtail & Rightwards Arrow With Tail With Double Vertical Stroke \\
\hline
U+02916 & ⤖ & {\textbackslash}twoheadrightarrowtail & Rightwards Two-Headed Arrow With Tail \\
\hline
U+02917 & ⤗ & {\textbackslash}nvtwoheadrightarrowtail & Rightwards Two-Headed Arrow With Tail With Vertical Stroke \\
\hline
U+02918 & ⤘ & {\textbackslash}nVtwoheadrightarrowtail & Rightwards Two-Headed Arrow With Tail With Double Vertical Stroke \\
\hline
U+0291D & ⤝ & {\textbackslash}diamondleftarrow & Leftwards Arrow To Black Diamond \\
\hline
U+0291E & ⤞ & {\textbackslash}rightarrowdiamond & Rightwards Arrow To Black Diamond \\
\hline
U+0291F & ⤟ & {\textbackslash}diamondleftarrowbar & Leftwards Arrow From Bar To Black Diamond \\
\hline
U+02920 & ⤠ & {\textbackslash}barrightarrowdiamond & Rightwards Arrow From Bar To Black Diamond \\
\hline
U+02925 & ⤥ & {\textbackslash}hksearow & South East Arrow With Hook \\
\hline
U+02926 & ⤦ & {\textbackslash}hkswarow & South West Arrow With Hook \\
\hline
U+02927 & ⤧ & {\textbackslash}tona & North West Arrow And North East Arrow \\
\hline
U+02928 & ⤨ & {\textbackslash}toea & North East Arrow And South East Arrow \\
\hline
U+02929 & ⤩ & {\textbackslash}tosa & South East Arrow And South West Arrow \\
\hline
U+0292A & ⤪ & {\textbackslash}towa & South West Arrow And North West Arrow \\
\hline
U+0292B & ⤫ & {\textbackslash}rdiagovfdiag & Rising Diagonal Crossing Falling Diagonal \\
\hline
U+0292C & ⤬ & {\textbackslash}fdiagovrdiag & Falling Diagonal Crossing Rising Diagonal \\
\hline
U+0292D & ⤭ & {\textbackslash}seovnearrow & South East Arrow Crossing North East Arrow \\
\hline
U+0292E & ⤮ & {\textbackslash}neovsearrow & North East Arrow Crossing South East Arrow \\
\hline
U+0292F & ⤯ & {\textbackslash}fdiagovnearrow & Falling Diagonal Crossing North East Arrow \\
\hline
U+02930 & ⤰ & {\textbackslash}rdiagovsearrow & Rising Diagonal Crossing South East Arrow \\
\hline
U+02931 & ⤱ & {\textbackslash}neovnwarrow & North East Arrow Crossing North West Arrow \\
\hline
U+02932 & ⤲ & {\textbackslash}nwovnearrow & North West Arrow Crossing North East Arrow \\
\hline
U+02934 & ⤴ & {\textbackslash}:arrow\_heading\_up: & Arrow Pointing Rightwards Then Curving Upwards \\
\hline
U+02935 & ⤵ & {\textbackslash}:arrow\_heading\_down: & Arrow Pointing Rightwards Then Curving Downwards \\
\hline
U+02942 & ⥂ & {\textbackslash}Rlarr & Rightwards Arrow Above Short Leftwards Arrow \\
\hline
U+02944 & ⥄ & {\textbackslash}rLarr & Short Rightwards Arrow Above Leftwards Arrow \\
\hline
U+02945 & ⥅ & {\textbackslash}rightarrowplus & Rightwards Arrow With Plus Below \\
\hline
U+02946 & ⥆ & {\textbackslash}leftarrowplus & Leftwards Arrow With Plus Below \\
\hline
U+02947 & ⥇ & {\textbackslash}rarrx & Rightwards Arrow Through X \\
\hline
U+02948 & ⥈ & {\textbackslash}leftrightarrowcircle & Left Right Arrow Through Small Circle \\
\hline
U+02949 & ⥉ & {\textbackslash}twoheaduparrowcircle & Upwards Two-Headed Arrow From Small Circle \\
\hline
U+0294A & ⥊ & {\textbackslash}leftrightharpoonupdown & Left Barb Up Right Barb Down Harpoon \\
\hline
U+0294B & ⥋ & {\textbackslash}leftrightharpoondownup & Left Barb Down Right Barb Up Harpoon \\
\hline
U+0294C & ⥌ & {\textbackslash}updownharpoonrightleft & Up Barb Right Down Barb Left Harpoon \\
\hline
U+0294D & ⥍ & {\textbackslash}updownharpoonleftright & Up Barb Left Down Barb Right Harpoon \\
\hline
U+0294E & ⥎ & {\textbackslash}LeftRightVector & Left Barb Up Right Barb Up Harpoon \\
\hline
U+0294F & ⥏ & {\textbackslash}RightUpDownVector & Up Barb Right Down Barb Right Harpoon \\
\hline
U+02950 & ⥐ & {\textbackslash}DownLeftRightVector & Left Barb Down Right Barb Down Harpoon \\
\hline
U+02951 & ⥑ & {\textbackslash}LeftUpDownVector & Up Barb Left Down Barb Left Harpoon \\
\hline
U+02952 & ⥒ & {\textbackslash}LeftVectorBar & Leftwards Harpoon With Barb Up To Bar \\
\hline
U+02953 & ⥓ & {\textbackslash}RightVectorBar & Rightwards Harpoon With Barb Up To Bar \\
\hline
U+02954 & ⥔ & {\textbackslash}RightUpVectorBar & Upwards Harpoon With Barb Right To Bar \\
\hline
U+02955 & ⥕ & {\textbackslash}RightDownVectorBar & Downwards Harpoon With Barb Right To Bar \\
\hline
U+02956 & ⥖ & {\textbackslash}DownLeftVectorBar & Leftwards Harpoon With Barb Down To Bar \\
\hline
U+02957 & ⥗ & {\textbackslash}DownRightVectorBar & Rightwards Harpoon With Barb Down To Bar \\
\hline
U+02958 & ⥘ & {\textbackslash}LeftUpVectorBar & Upwards Harpoon With Barb Left To Bar \\
\hline
U+02959 & ⥙ & {\textbackslash}LeftDownVectorBar & Downwards Harpoon With Barb Left To Bar \\
\hline
U+0295A & ⥚ & {\textbackslash}LeftTeeVector & Leftwards Harpoon With Barb Up From Bar \\
\hline
U+0295B & ⥛ & {\textbackslash}RightTeeVector & Rightwards Harpoon With Barb Up From Bar \\
\hline
U+0295C & ⥜ & {\textbackslash}RightUpTeeVector & Upwards Harpoon With Barb Right From Bar \\
\hline
U+0295D & ⥝ & {\textbackslash}RightDownTeeVector & Downwards Harpoon With Barb Right From Bar \\
\hline
U+0295E & ⥞ & {\textbackslash}DownLeftTeeVector & Leftwards Harpoon With Barb Down From Bar \\
\hline
U+0295F & ⥟ & {\textbackslash}DownRightTeeVector & Rightwards Harpoon With Barb Down From Bar \\
\hline
U+02960 & ⥠ & {\textbackslash}LeftUpTeeVector & Upwards Harpoon With Barb Left From Bar \\
\hline
U+02961 & ⥡ & {\textbackslash}LeftDownTeeVector & Downwards Harpoon With Barb Left From Bar \\
\hline
U+02962 & ⥢ & {\textbackslash}leftharpoonsupdown & Leftwards Harpoon With Barb Up Above Leftwards Harpoon With Barb Down \\
\hline
U+02963 & ⥣ & {\textbackslash}upharpoonsleftright & Upwards Harpoon With Barb Left Beside Upwards Harpoon With Barb Right \\
\hline
U+02964 & ⥤ & {\textbackslash}rightharpoonsupdown & Rightwards Harpoon With Barb Up Above Rightwards Harpoon With Barb Down \\
\hline
U+02965 & ⥥ & {\textbackslash}downharpoonsleftright & Downwards Harpoon With Barb Left Beside Downwards Harpoon With Barb Right \\
\hline
U+02966 & ⥦ & {\textbackslash}leftrightharpoonsup & Leftwards Harpoon With Barb Up Above Rightwards Harpoon With Barb Up \\
\hline
U+02967 & ⥧ & {\textbackslash}leftrightharpoonsdown & Leftwards Harpoon With Barb Down Above Rightwards Harpoon With Barb Down \\
\hline
U+02968 & ⥨ & {\textbackslash}rightleftharpoonsup & Rightwards Harpoon With Barb Up Above Leftwards Harpoon With Barb Up \\
\hline
U+02969 & ⥩ & {\textbackslash}rightleftharpoonsdown & Rightwards Harpoon With Barb Down Above Leftwards Harpoon With Barb Down \\
\hline
U+0296A & ⥪ & {\textbackslash}leftharpoonupdash & Leftwards Harpoon With Barb Up Above Long Dash \\
\hline
U+0296B & ⥫ & {\textbackslash}dashleftharpoondown & Leftwards Harpoon With Barb Down Below Long Dash \\
\hline
U+0296C & ⥬ & {\textbackslash}rightharpoonupdash & Rightwards Harpoon With Barb Up Above Long Dash \\
\hline
U+0296D & ⥭ & {\textbackslash}dashrightharpoondown & Rightwards Harpoon With Barb Down Below Long Dash \\
\hline
U+0296E & ⥮ & {\textbackslash}UpEquilibrium & Upwards Harpoon With Barb Left Beside Downwards Harpoon With Barb Right \\
\hline
U+0296F & ⥯ & {\textbackslash}ReverseUpEquilibrium & Downwards Harpoon With Barb Left Beside Upwards Harpoon With Barb Right \\
\hline
U+02970 & ⥰ & {\textbackslash}RoundImplies & Right Double Arrow With Rounded Head \\
\hline
U+02980 & ⦀ & {\textbackslash}Vvert & Triple Vertical Bar Delimiter \\
\hline
U+02986 & ⦆ & {\textbackslash}Elroang & Right White Parenthesis \\
\hline
U+02999 & ⦙ & {\textbackslash}ddfnc & Dotted Fence \\
\hline
U+0299B & ⦛ & {\textbackslash}measuredangleleft & Measured Angle Opening Left \\
\hline
U+0299C & ⦜ & {\textbackslash}Angle & Right Angle Variant With Square \\
\hline
U+0299D & ⦝ & {\textbackslash}rightanglemdot & Measured Right Angle With Dot \\
\hline
U+0299E & ⦞ & {\textbackslash}angles & Angle With S Inside \\
\hline
U+0299F & ⦟ & {\textbackslash}angdnr & Acute Angle \\
\hline
U+029A0 & ⦠ & {\textbackslash}lpargt & Spherical Angle Opening Left \\
\hline
U+029A1 & ⦡ & {\textbackslash}sphericalangleup & Spherical Angle Opening Up \\
\hline
U+029A2 & ⦢ & {\textbackslash}turnangle & Turned Angle \\
\hline
U+029A3 & ⦣ & {\textbackslash}revangle & Reversed Angle \\
\hline
U+029A4 & ⦤ & {\textbackslash}angleubar & Angle With Underbar \\
\hline
U+029A5 & ⦥ & {\textbackslash}revangleubar & Reversed Angle With Underbar \\
\hline
U+029A6 & ⦦ & {\textbackslash}wideangledown & Oblique Angle Opening Up \\
\hline
U+029A7 & ⦧ & {\textbackslash}wideangleup & Oblique Angle Opening Down \\
\hline
U+029A8 & ⦨ & {\textbackslash}measanglerutone & Measured Angle With Open Arm Ending In Arrow Pointing Up And Right \\
\hline
U+029A9 & ⦩ & {\textbackslash}measanglelutonw & Measured Angle With Open Arm Ending In Arrow Pointing Up And Left \\
\hline
U+029AA & ⦪ & {\textbackslash}measanglerdtose & Measured Angle With Open Arm Ending In Arrow Pointing Down And Right \\
\hline
U+029AB & ⦫ & {\textbackslash}measangleldtosw & Measured Angle With Open Arm Ending In Arrow Pointing Down And Left \\
\hline
U+029AC & ⦬ & {\textbackslash}measangleurtone & Measured Angle With Open Arm Ending In Arrow Pointing Right And Up \\
\hline
U+029AD & ⦭ & {\textbackslash}measangleultonw & Measured Angle With Open Arm Ending In Arrow Pointing Left And Up \\
\hline
U+029AE & ⦮ & {\textbackslash}measangledrtose & Measured Angle With Open Arm Ending In Arrow Pointing Right And Down \\
\hline
U+029AF & ⦯ & {\textbackslash}measangledltosw & Measured Angle With Open Arm Ending In Arrow Pointing Left And Down \\
\hline
U+029B0 & ⦰ & {\textbackslash}revemptyset & Reversed Empty Set \\
\hline
U+029B1 & ⦱ & {\textbackslash}emptysetobar & Empty Set With Overbar \\
\hline
U+029B2 & ⦲ & {\textbackslash}emptysetocirc & Empty Set With Small Circle Above \\
\hline
U+029B3 & ⦳ & {\textbackslash}emptysetoarr & Empty Set With Right Arrow Above \\
\hline
U+029B4 & ⦴ & {\textbackslash}emptysetoarrl & Empty Set With Left Arrow Above \\
\hline
U+029B7 & ⦷ & {\textbackslash}circledparallel & Circled Parallel \\
\hline
U+029B8 & ⦸ & {\textbackslash}obslash & Circled Reverse Solidus \\
\hline
U+029BC & ⦼ & {\textbackslash}odotslashdot & Circled Anticlockwise-Rotated Division Sign \\
\hline
U+029BE & ⦾ & {\textbackslash}circledwhitebullet & Circled White Bullet \\
\hline
U+029BF & ⦿ & {\textbackslash}circledbullet & Circled Bullet \\
\hline
U+029C0 & ⧀ & {\textbackslash}olessthan & Circled Less-Than \\
\hline
U+029C1 & ⧁ & {\textbackslash}ogreaterthan & Circled Greater-Than \\
\hline
U+029C4 & ⧄ & {\textbackslash}boxdiag & Squared Rising Diagonal Slash \\
\hline
U+029C5 & ⧅ & {\textbackslash}boxbslash & Squared Falling Diagonal Slash \\
\hline
U+029C6 & ⧆ & {\textbackslash}boxast & Squared Asterisk \\
\hline
U+029C7 & ⧇ & {\textbackslash}boxcircle & Squared Small Circle \\
\hline
U+029CA & ⧊ & {\textbackslash}Lap & Triangle With Dot Above \\
\hline
U+029CB & ⧋ & {\textbackslash}defas & Triangle With Underbar \\
\hline
U+029CF & ⧏ & {\textbackslash}LeftTriangleBar & Left Triangle Beside Vertical Bar \\
\hline
U+029CF + U+00338 & ⧏̸ & {\textbackslash}NotLeftTriangleBar & Left Triangle Beside Vertical Bar + Combining Long Solidus Overlay / Non-Spacing Long Slash Overlay \\
\hline
U+029D0 & ⧐ & {\textbackslash}RightTriangleBar & Vertical Bar Beside Right Triangle \\
\hline
U+029D0 + U+00338 & ⧐̸ & {\textbackslash}NotRightTriangleBar & Vertical Bar Beside Right Triangle + Combining Long Solidus Overlay / Non-Spacing Long Slash Overlay \\
\hline
U+029DF & ⧟ & {\textbackslash}dualmap & Double-Ended Multimap \\
\hline
U+029E1 & ⧡ & {\textbackslash}lrtriangleeq & Increases As \\
\hline
U+029E2 & ⧢ & {\textbackslash}shuffle & Shuffle Product \\
\hline
U+029E3 & ⧣ & {\textbackslash}eparsl & Equals Sign And Slanted Parallel \\
\hline
U+029E4 & ⧤ & {\textbackslash}smeparsl & Equals Sign And Slanted Parallel With Tilde Above \\
\hline
U+029E5 & ⧥ & {\textbackslash}eqvparsl & Identical To And Slanted Parallel \\
\hline
U+029EB & ⧫ & {\textbackslash}blacklozenge & Black Lozenge \\
\hline
U+029F4 & ⧴ & {\textbackslash}RuleDelayed & Rule-Delayed \\
\hline
U+029F6 & ⧶ & {\textbackslash}dsol & Solidus With Overbar \\
\hline
U+029F7 & ⧷ & {\textbackslash}rsolbar & Reverse Solidus With Horizontal Stroke \\
\hline
U+029FA & ⧺ & {\textbackslash}doubleplus & Double Plus \\
\hline
U+029FB & ⧻ & {\textbackslash}tripleplus & Triple Plus \\
\hline
U+02A00 & ⨀ & {\textbackslash}bigodot & N-Ary Circled Dot Operator \\
\hline
U+02A01 & ⨁ & {\textbackslash}bigoplus & N-Ary Circled Plus Operator \\
\hline
U+02A02 & ⨂ & {\textbackslash}bigotimes & N-Ary Circled Times Operator \\
\hline
U+02A03 & ⨃ & {\textbackslash}bigcupdot & N-Ary Union Operator With Dot \\
\hline
U+02A04 & ⨄ & {\textbackslash}biguplus & N-Ary Union Operator With Plus \\
\hline
U+02A05 & ⨅ & {\textbackslash}bigsqcap & N-Ary Square Intersection Operator \\
\hline
U+02A06 & ⨆ & {\textbackslash}bigsqcup & N-Ary Square Union Operator \\
\hline
U+02A07 & ⨇ & {\textbackslash}conjquant & Two Logical And Operator \\
\hline
U+02A08 & ⨈ & {\textbackslash}disjquant & Two Logical Or Operator \\
\hline
U+02A09 & ⨉ & {\textbackslash}bigtimes & N-Ary Times Operator \\
\hline
U+02A0A & ⨊ & {\textbackslash}modtwosum & Modulo Two Sum \\
\hline
U+02A0B & ⨋ & {\textbackslash}sumint & Summation With Integral \\
\hline
U+02A0C & ⨌ & {\textbackslash}iiiint & Quadruple Integral Operator \\
\hline
U+02A0D & ⨍ & {\textbackslash}intbar & Finite Part Integral \\
\hline
U+02A0E & ⨎ & {\textbackslash}intBar & Integral With Double Stroke \\
\hline
U+02A0F & ⨏ & {\textbackslash}clockoint & Integral Average With Slash \\
\hline
U+02A10 & ⨐ & {\textbackslash}cirfnint & Circulation Function \\
\hline
U+02A11 & ⨑ & {\textbackslash}awint & Anticlockwise Integration \\
\hline
U+02A12 & ⨒ & {\textbackslash}rppolint & Line Integration With Rectangular Path Around Pole \\
\hline
U+02A13 & ⨓ & {\textbackslash}scpolint & Line Integration With Semicircular Path Around Pole \\
\hline
U+02A14 & ⨔ & {\textbackslash}npolint & Line Integration Not Including The Pole \\
\hline
U+02A15 & ⨕ & {\textbackslash}pointint & Integral Around A Point Operator \\
\hline
U+02A16 & ⨖ & {\textbackslash}sqrint & Quaternion Integral Operator \\
\hline
U+02A18 & ⨘ & {\textbackslash}intx & Integral With Times Sign \\
\hline
U+02A19 & ⨙ & {\textbackslash}intcap & Integral With Intersection \\
\hline
U+02A1A & ⨚ & {\textbackslash}intcup & Integral With Union \\
\hline
U+02A1B & ⨛ & {\textbackslash}upint & Integral With Overbar \\
\hline
U+02A1C & ⨜ & {\textbackslash}lowint & Integral With Underbar \\
\hline
U+02A1D & ⨝ & {\textbackslash}Join, {\textbackslash}join & Join \\
\hline
U+02A1F & ⨟ & {\textbackslash}bbsemi & Z Notation Schema Composition \\
\hline
U+02A22 & ⨢ & {\textbackslash}ringplus & Plus Sign With Small Circle Above \\
\hline
U+02A23 & ⨣ & {\textbackslash}plushat & Plus Sign With Circumflex Accent Above \\
\hline
U+02A24 & ⨤ & {\textbackslash}simplus & Plus Sign With Tilde Above \\
\hline
U+02A25 & ⨥ & {\textbackslash}plusdot & Plus Sign With Dot Below \\
\hline
U+02A26 & ⨦ & {\textbackslash}plussim & Plus Sign With Tilde Below \\
\hline
U+02A27 & ⨧ & {\textbackslash}plussubtwo & Plus Sign With Subscript Two \\
\hline
U+02A28 & ⨨ & {\textbackslash}plustrif & Plus Sign With Black Triangle \\
\hline
U+02A29 & ⨩ & {\textbackslash}commaminus & Minus Sign With Comma Above \\
\hline
U+02A2A & ⨪ & {\textbackslash}minusdot & Minus Sign With Dot Below \\
\hline
U+02A2B & ⨫ & {\textbackslash}minusfdots & Minus Sign With Falling Dots \\
\hline
U+02A2C & ⨬ & {\textbackslash}minusrdots & Minus Sign With Rising Dots \\
\hline
U+02A2D & ⨭ & {\textbackslash}opluslhrim & Plus Sign In Left Half Circle \\
\hline
U+02A2E & ⨮ & {\textbackslash}oplusrhrim & Plus Sign In Right Half Circle \\
\hline
U+02A2F & ⨯ & {\textbackslash}Times & Vector Or Cross Product \\
\hline
U+02A30 & ⨰ & {\textbackslash}dottimes & Multiplication Sign With Dot Above \\
\hline
U+02A31 & ⨱ & {\textbackslash}timesbar & Multiplication Sign With Underbar \\
\hline
U+02A32 & ⨲ & {\textbackslash}btimes & Semidirect Product With Bottom Closed \\
\hline
U+02A33 & ⨳ & {\textbackslash}smashtimes & Smash Product \\
\hline
U+02A34 & ⨴ & {\textbackslash}otimeslhrim & Multiplication Sign In Left Half Circle \\
\hline
U+02A35 & ⨵ & {\textbackslash}otimesrhrim & Multiplication Sign In Right Half Circle \\
\hline
U+02A36 & ⨶ & {\textbackslash}otimeshat & Circled Multiplication Sign With Circumflex Accent \\
\hline
U+02A37 & ⨷ & {\textbackslash}Otimes & Multiplication Sign In Double Circle \\
\hline
U+02A38 & ⨸ & {\textbackslash}odiv & Circled Division Sign \\
\hline
U+02A39 & ⨹ & {\textbackslash}triangleplus & Plus Sign In Triangle \\
\hline
U+02A3A & ⨺ & {\textbackslash}triangleminus & Minus Sign In Triangle \\
\hline
U+02A3B & ⨻ & {\textbackslash}triangletimes & Multiplication Sign In Triangle \\
\hline
U+02A3C & ⨼ & {\textbackslash}intprod & Interior Product \\
\hline
U+02A3D & ⨽ & {\textbackslash}intprodr & Righthand Interior Product \\
\hline
U+02A3F & ⨿ & {\textbackslash}amalg & Amalgamation Or Coproduct \\
\hline
U+02A40 & ⩀ & {\textbackslash}capdot & Intersection With Dot \\
\hline
U+02A41 & ⩁ & {\textbackslash}uminus & Union With Minus Sign \\
\hline
U+02A42 & ⩂ & {\textbackslash}barcup & Union With Overbar \\
\hline
U+02A43 & ⩃ & {\textbackslash}barcap & Intersection With Overbar \\
\hline
U+02A44 & ⩄ & {\textbackslash}capwedge & Intersection With Logical And \\
\hline
U+02A45 & ⩅ & {\textbackslash}cupvee & Union With Logical Or \\
\hline
U+02A4A & ⩊ & {\textbackslash}twocups & Union Beside And Joined With Union \\
\hline
U+02A4B & ⩋ & {\textbackslash}twocaps & Intersection Beside And Joined With Intersection \\
\hline
U+02A4C & ⩌ & {\textbackslash}closedvarcup & Closed Union With Serifs \\
\hline
U+02A4D & ⩍ & {\textbackslash}closedvarcap & Closed Intersection With Serifs \\
\hline
U+02A4E & ⩎ & {\textbackslash}Sqcap & Double Square Intersection \\
\hline
U+02A4F & ⩏ & {\textbackslash}Sqcup & Double Square Union \\
\hline
U+02A50 & ⩐ & {\textbackslash}closedvarcupsmashprod & Closed Union With Serifs And Smash Product \\
\hline
U+02A51 & ⩑ & {\textbackslash}wedgeodot & Logical And With Dot Above \\
\hline
U+02A52 & ⩒ & {\textbackslash}veeodot & Logical Or With Dot Above \\
\hline
U+02A53 & ⩓ & {\textbackslash}And & Double Logical And \\
\hline
U+02A54 & ⩔ & {\textbackslash}Or & Double Logical Or \\
\hline
U+02A55 & ⩕ & {\textbackslash}wedgeonwedge & Two Intersecting Logical And \\
\hline
U+02A56 & ⩖ & {\textbackslash}ElOr & Two Intersecting Logical Or \\
\hline
U+02A57 & ⩗ & {\textbackslash}bigslopedvee & Sloping Large Or \\
\hline
U+02A58 & ⩘ & {\textbackslash}bigslopedwedge & Sloping Large And \\
\hline
U+02A5A & ⩚ & {\textbackslash}wedgemidvert & Logical And With Middle Stem \\
\hline
U+02A5B & ⩛ & {\textbackslash}veemidvert & Logical Or With Middle Stem \\
\hline
U+02A5C & ⩜ & {\textbackslash}midbarwedge & Logical And With Horizontal Dash \\
\hline
U+02A5D & ⩝ & {\textbackslash}midbarvee & Logical Or With Horizontal Dash \\
\hline
U+02A5E & ⩞ & {\textbackslash}perspcorrespond & Logical And With Double Overbar \\
\hline
U+02A5F & ⩟ & {\textbackslash}minhat & Logical And With Underbar \\
\hline
U+02A60 & ⩠ & {\textbackslash}wedgedoublebar & Logical And With Double Underbar \\
\hline
U+02A61 & ⩡ & {\textbackslash}varveebar & Small Vee With Underbar \\
\hline
U+02A62 & ⩢ & {\textbackslash}doublebarvee & Logical Or With Double Overbar \\
\hline
U+02A63 & ⩣ & {\textbackslash}veedoublebar & Logical Or With Double Underbar \\
\hline
U+02A66 & ⩦ & {\textbackslash}eqdot & Equals Sign With Dot Below \\
\hline
U+02A67 & ⩧ & {\textbackslash}dotequiv & Identical With Dot Above \\
\hline
U+02A6A & ⩪ & {\textbackslash}dotsim & Tilde Operator With Dot Above \\
\hline
U+02A6B & ⩫ & {\textbackslash}simrdots & Tilde Operator With Rising Dots \\
\hline
U+02A6C & ⩬ & {\textbackslash}simminussim & Similar Minus Similar \\
\hline
U+02A6D & ⩭ & {\textbackslash}congdot & Congruent With Dot Above \\
\hline
U+02A6E & ⩮ & {\textbackslash}asteq & Equals With Asterisk \\
\hline
U+02A6F & ⩯ & {\textbackslash}hatapprox & Almost Equal To With Circumflex Accent \\
\hline
U+02A70 & ⩰ & {\textbackslash}approxeqq & Approximately Equal Or Equal To \\
\hline
U+02A71 & ⩱ & {\textbackslash}eqqplus & Equals Sign Above Plus Sign \\
\hline
U+02A72 & ⩲ & {\textbackslash}pluseqq & Plus Sign Above Equals Sign \\
\hline
U+02A73 & ⩳ & {\textbackslash}eqqsim & Equals Sign Above Tilde Operator \\
\hline
U+02A74 & ⩴ & {\textbackslash}Coloneq & Double Colon Equal \\
\hline
U+02A75 & ⩵ & {\textbackslash}Equal & Two Consecutive Equals Signs \\
\hline
U+02A76 & ⩶ & {\textbackslash}eqeqeq & Three Consecutive Equals Signs \\
\hline
U+02A77 & ⩷ & {\textbackslash}ddotseq & Equals Sign With Two Dots Above And Two Dots Below \\
\hline
U+02A78 & ⩸ & {\textbackslash}equivDD & Equivalent With Four Dots Above \\
\hline
U+02A79 & ⩹ & {\textbackslash}ltcir & Less-Than With Circle Inside \\
\hline
U+02A7A & ⩺ & {\textbackslash}gtcir & Greater-Than With Circle Inside \\
\hline
U+02A7B & ⩻ & {\textbackslash}ltquest & Less-Than With Question Mark Above \\
\hline
U+02A7C & ⩼ & {\textbackslash}gtquest & Greater-Than With Question Mark Above \\
\hline
U+02A7D & ⩽ & {\textbackslash}leqslant & Less-Than Or Slanted Equal To \\
\hline
U+02A7D + U+00338 & ⩽̸ & {\textbackslash}nleqslant & Less-Than Or Slanted Equal To + Combining Long Solidus Overlay / Non-Spacing Long Slash Overlay \\
\hline
U+02A7E & ⩾ & {\textbackslash}geqslant & Greater-Than Or Slanted Equal To \\
\hline
U+02A7E + U+00338 & ⩾̸ & {\textbackslash}ngeqslant & Greater-Than Or Slanted Equal To + Combining Long Solidus Overlay / Non-Spacing Long Slash Overlay \\
\hline
U+02A7F & ⩿ & {\textbackslash}lesdot & Less-Than Or Slanted Equal To With Dot Inside \\
\hline
U+02A80 & ⪀ & {\textbackslash}gesdot & Greater-Than Or Slanted Equal To With Dot Inside \\
\hline
U+02A81 & ⪁ & {\textbackslash}lesdoto & Less-Than Or Slanted Equal To With Dot Above \\
\hline
U+02A82 & ⪂ & {\textbackslash}gesdoto & Greater-Than Or Slanted Equal To With Dot Above \\
\hline
U+02A83 & ⪃ & {\textbackslash}lesdotor & Less-Than Or Slanted Equal To With Dot Above Right \\
\hline
U+02A84 & ⪄ & {\textbackslash}gesdotol & Greater-Than Or Slanted Equal To With Dot Above Left \\
\hline
U+02A85 & ⪅ & {\textbackslash}lessapprox & Less-Than Or Approximate \\
\hline
U+02A86 & ⪆ & {\textbackslash}gtrapprox & Greater-Than Or Approximate \\
\hline
U+02A87 & ⪇ & {\textbackslash}lneq & Less-Than And Single-Line Not Equal To \\
\hline
U+02A88 & ⪈ & {\textbackslash}gneq & Greater-Than And Single-Line Not Equal To \\
\hline
U+02A89 & ⪉ & {\textbackslash}lnapprox & Less-Than And Not Approximate \\
\hline
U+02A8A & ⪊ & {\textbackslash}gnapprox & Greater-Than And Not Approximate \\
\hline
U+02A8B & ⪋ & {\textbackslash}lesseqqgtr & Less-Than Above Double-Line Equal Above Greater-Than \\
\hline
U+02A8C & ⪌ & {\textbackslash}gtreqqless & Greater-Than Above Double-Line Equal Above Less-Than \\
\hline
U+02A8D & ⪍ & {\textbackslash}lsime & Less-Than Above Similar Or Equal \\
\hline
U+02A8E & ⪎ & {\textbackslash}gsime & Greater-Than Above Similar Or Equal \\
\hline
U+02A8F & ⪏ & {\textbackslash}lsimg & Less-Than Above Similar Above Greater-Than \\
\hline
U+02A90 & ⪐ & {\textbackslash}gsiml & Greater-Than Above Similar Above Less-Than \\
\hline
U+02A91 & ⪑ & {\textbackslash}lgE & Less-Than Above Greater-Than Above Double-Line Equal \\
\hline
U+02A92 & ⪒ & {\textbackslash}glE & Greater-Than Above Less-Than Above Double-Line Equal \\
\hline
U+02A93 & ⪓ & {\textbackslash}lesges & Less-Than Above Slanted Equal Above Greater-Than Above Slanted Equal \\
\hline
U+02A94 & ⪔ & {\textbackslash}gesles & Greater-Than Above Slanted Equal Above Less-Than Above Slanted Equal \\
\hline
U+02A95 & ⪕ & {\textbackslash}eqslantless & Slanted Equal To Or Less-Than \\
\hline
U+02A96 & ⪖ & {\textbackslash}eqslantgtr & Slanted Equal To Or Greater-Than \\
\hline
U+02A97 & ⪗ & {\textbackslash}elsdot & Slanted Equal To Or Less-Than With Dot Inside \\
\hline
U+02A98 & ⪘ & {\textbackslash}egsdot & Slanted Equal To Or Greater-Than With Dot Inside \\
\hline
U+02A99 & ⪙ & {\textbackslash}eqqless & Double-Line Equal To Or Less-Than \\
\hline
U+02A9A & ⪚ & {\textbackslash}eqqgtr & Double-Line Equal To Or Greater-Than \\
\hline
U+02A9B & ⪛ & {\textbackslash}eqqslantless & Double-Line Slanted Equal To Or Less-Than \\
\hline
U+02A9C & ⪜ & {\textbackslash}eqqslantgtr & Double-Line Slanted Equal To Or Greater-Than \\
\hline
U+02A9D & ⪝ & {\textbackslash}simless & Similar Or Less-Than \\
\hline
U+02A9E & ⪞ & {\textbackslash}simgtr & Similar Or Greater-Than \\
\hline
U+02A9F & ⪟ & {\textbackslash}simlE & Similar Above Less-Than Above Equals Sign \\
\hline
U+02AA0 & ⪠ & {\textbackslash}simgE & Similar Above Greater-Than Above Equals Sign \\
\hline
U+02AA1 & ⪡ & {\textbackslash}NestedLessLess & Double Nested Less-Than \\
\hline
U+02AA1 + U+00338 & ⪡̸ & {\textbackslash}NotNestedLessLess & Double Nested Less-Than + Combining Long Solidus Overlay / Non-Spacing Long Slash Overlay \\
\hline
U+02AA2 & ⪢ & {\textbackslash}NestedGreaterGreater & Double Nested Greater-Than \\
\hline
U+02AA2 + U+00338 & ⪢̸ & {\textbackslash}NotNestedGreaterGreater & Double Nested Greater-Than + Combining Long Solidus Overlay / Non-Spacing Long Slash Overlay \\
\hline
U+02AA3 & ⪣ & {\textbackslash}partialmeetcontraction & Double Nested Less-Than With Underbar \\
\hline
U+02AA4 & ⪤ & {\textbackslash}glj & Greater-Than Overlapping Less-Than \\
\hline
U+02AA5 & ⪥ & {\textbackslash}gla & Greater-Than Beside Less-Than \\
\hline
U+02AA6 & ⪦ & {\textbackslash}ltcc & Less-Than Closed By Curve \\
\hline
U+02AA7 & ⪧ & {\textbackslash}gtcc & Greater-Than Closed By Curve \\
\hline
U+02AA8 & ⪨ & {\textbackslash}lescc & Less-Than Closed By Curve Above Slanted Equal \\
\hline
U+02AA9 & ⪩ & {\textbackslash}gescc & Greater-Than Closed By Curve Above Slanted Equal \\
\hline
U+02AAA & ⪪ & {\textbackslash}smt & Smaller Than \\
\hline
U+02AAB & ⪫ & {\textbackslash}lat & Larger Than \\
\hline
U+02AAC & ⪬ & {\textbackslash}smte & Smaller Than Or Equal To \\
\hline
U+02AAD & ⪭ & {\textbackslash}late & Larger Than Or Equal To \\
\hline
U+02AAE & ⪮ & {\textbackslash}bumpeqq & Equals Sign With Bumpy Above \\
\hline
U+02AAF & ⪯ & {\textbackslash}preceq & Precedes Above Single-Line Equals Sign \\
\hline
U+02AAF + U+00338 & ⪯̸ & {\textbackslash}npreceq & Precedes Above Single-Line Equals Sign + Combining Long Solidus Overlay / Non-Spacing Long Slash Overlay \\
\hline
U+02AB0 & ⪰ & {\textbackslash}succeq & Succeeds Above Single-Line Equals Sign \\
\hline
U+02AB0 + U+00338 & ⪰̸ & {\textbackslash}nsucceq & Succeeds Above Single-Line Equals Sign + Combining Long Solidus Overlay / Non-Spacing Long Slash Overlay \\
\hline
U+02AB1 & ⪱ & {\textbackslash}precneq & Precedes Above Single-Line Not Equal To \\
\hline
U+02AB2 & ⪲ & {\textbackslash}succneq & Succeeds Above Single-Line Not Equal To \\
\hline
U+02AB3 & ⪳ & {\textbackslash}preceqq & Precedes Above Equals Sign \\
\hline
U+02AB4 & ⪴ & {\textbackslash}succeqq & Succeeds Above Equals Sign \\
\hline
U+02AB5 & ⪵ & {\textbackslash}precneqq & Precedes Above Not Equal To \\
\hline
U+02AB6 & ⪶ & {\textbackslash}succneqq & Succeeds Above Not Equal To \\
\hline
U+02AB7 & ⪷ & {\textbackslash}precapprox & Precedes Above Almost Equal To \\
\hline
U+02AB8 & ⪸ & {\textbackslash}succapprox & Succeeds Above Almost Equal To \\
\hline
U+02AB9 & ⪹ & {\textbackslash}precnapprox & Precedes Above Not Almost Equal To \\
\hline
U+02ABA & ⪺ & {\textbackslash}succnapprox & Succeeds Above Not Almost Equal To \\
\hline
U+02ABB & ⪻ & {\textbackslash}Prec & Double Precedes \\
\hline
U+02ABC & ⪼ & {\textbackslash}Succ & Double Succeeds \\
\hline
U+02ABD & ⪽ & {\textbackslash}subsetdot & Subset With Dot \\
\hline
U+02ABE & ⪾ & {\textbackslash}supsetdot & Superset With Dot \\
\hline
U+02ABF & ⪿ & {\textbackslash}subsetplus & Subset With Plus Sign Below \\
\hline
U+02AC0 & ⫀ & {\textbackslash}supsetplus & Superset With Plus Sign Below \\
\hline
U+02AC1 & ⫁ & {\textbackslash}submult & Subset With Multiplication Sign Below \\
\hline
U+02AC2 & ⫂ & {\textbackslash}supmult & Superset With Multiplication Sign Below \\
\hline
U+02AC3 & ⫃ & {\textbackslash}subedot & Subset Of Or Equal To With Dot Above \\
\hline
U+02AC4 & ⫄ & {\textbackslash}supedot & Superset Of Or Equal To With Dot Above \\
\hline
U+02AC5 & ⫅ & {\textbackslash}subseteqq & Subset Of Above Equals Sign \\
\hline
U+02AC5 + U+00338 & ⫅̸ & {\textbackslash}nsubseteqq & Subset Of Above Equals Sign + Combining Long Solidus Overlay / Non-Spacing Long Slash Overlay \\
\hline
U+02AC6 & ⫆ & {\textbackslash}supseteqq & Superset Of Above Equals Sign \\
\hline
U+02AC6 + U+00338 & ⫆̸ & {\textbackslash}nsupseteqq & Superset Of Above Equals Sign + Combining Long Solidus Overlay / Non-Spacing Long Slash Overlay \\
\hline
U+02AC7 & ⫇ & {\textbackslash}subsim & Subset Of Above Tilde Operator \\
\hline
U+02AC8 & ⫈ & {\textbackslash}supsim & Superset Of Above Tilde Operator \\
\hline
U+02AC9 & ⫉ & {\textbackslash}subsetapprox & Subset Of Above Almost Equal To \\
\hline
U+02ACA & ⫊ & {\textbackslash}supsetapprox & Superset Of Above Almost Equal To \\
\hline
U+02ACB & ⫋ & {\textbackslash}subsetneqq & Subset Of Above Not Equal To \\
\hline
U+02ACC & ⫌ & {\textbackslash}supsetneqq & Superset Of Above Not Equal To \\
\hline
U+02ACD & ⫍ & {\textbackslash}lsqhook & Square Left Open Box Operator \\
\hline
U+02ACE & ⫎ & {\textbackslash}rsqhook & Square Right Open Box Operator \\
\hline
U+02ACF & ⫏ & {\textbackslash}csub & Closed Subset \\
\hline
U+02AD0 & ⫐ & {\textbackslash}csup & Closed Superset \\
\hline
U+02AD1 & ⫑ & {\textbackslash}csube & Closed Subset Or Equal To \\
\hline
U+02AD2 & ⫒ & {\textbackslash}csupe & Closed Superset Or Equal To \\
\hline
U+02AD3 & ⫓ & {\textbackslash}subsup & Subset Above Superset \\
\hline
U+02AD4 & ⫔ & {\textbackslash}supsub & Superset Above Subset \\
\hline
U+02AD5 & ⫕ & {\textbackslash}subsub & Subset Above Subset \\
\hline
U+02AD6 & ⫖ & {\textbackslash}supsup & Superset Above Superset \\
\hline
U+02AD7 & ⫗ & {\textbackslash}suphsub & Superset Beside Subset \\
\hline
U+02AD8 & ⫘ & {\textbackslash}supdsub & Superset Beside And Joined By Dash With Subset \\
\hline
U+02AD9 & ⫙ & {\textbackslash}forkv & Element Of Opening Downwards \\
\hline
U+02ADB & ⫛ & {\textbackslash}mlcp & Transversal Intersection \\
\hline
U+02ADC & ⫝̸ & {\textbackslash}forks & Forking \\
\hline
U+02ADD & ⫝ & {\textbackslash}forksnot & Nonforking \\
\hline
U+02AE3 & ⫣ & {\textbackslash}dashV & Double Vertical Bar Left Turnstile \\
\hline
U+02AE4 & ⫤ & {\textbackslash}Dashv & Vertical Bar Double Left Turnstile \\
\hline
U+02AF4 & ⫴ & {\textbackslash}interleave & Triple Vertical Bar Binary Relation \\
\hline
U+02AF6 & ⫶ & {\textbackslash}tdcol & Triple Colon Operator \\
\hline
U+02AF7 & ⫷ & {\textbackslash}lllnest & Triple Nested Less-Than \\
\hline
U+02AF8 & ⫸ & {\textbackslash}gggnest & Triple Nested Greater-Than \\
\hline
U+02AF9 & ⫹ & {\textbackslash}leqqslant & Double-Line Slanted Less-Than Or Equal To \\
\hline
U+02AFA & ⫺ & {\textbackslash}geqqslant & Double-Line Slanted Greater-Than Or Equal To \\
\hline
U+02B05 & ⬅ & {\textbackslash}:arrow\_left: & Leftwards Black Arrow \\
\hline
U+02B06 & ⬆ & {\textbackslash}:arrow\_up: & Upwards Black Arrow \\
\hline
U+02B07 & ⬇ & {\textbackslash}:arrow\_down: & Downwards Black Arrow \\
\hline
U+02B12 & ⬒ & {\textbackslash}squaretopblack & Square With Top Half Black \\
\hline
U+02B13 & ⬓ & {\textbackslash}squarebotblack & Square With Bottom Half Black \\
\hline
U+02B14 & ⬔ & {\textbackslash}squareurblack & Square With Upper Right Diagonal Half Black \\
\hline
U+02B15 & ⬕ & {\textbackslash}squarellblack & Square With Lower Left Diagonal Half Black \\
\hline
U+02B16 & ⬖ & {\textbackslash}diamondleftblack & Diamond With Left Half Black \\
\hline
U+02B17 & ⬗ & {\textbackslash}diamondrightblack & Diamond With Right Half Black \\
\hline
U+02B18 & ⬘ & {\textbackslash}diamondtopblack & Diamond With Top Half Black \\
\hline
U+02B19 & ⬙ & {\textbackslash}diamondbotblack & Diamond With Bottom Half Black \\
\hline
U+02B1A & ⬚ & {\textbackslash}dottedsquare & Dotted Square \\
\hline
U+02B1B & ⬛ & {\textbackslash}lgblksquare, {\textbackslash}:black\_large\_square: & Black Large Square \\
\hline
U+02B1C & ⬜ & {\textbackslash}lgwhtsquare, {\textbackslash}:white\_large\_square: & White Large Square \\
\hline
U+02B1D & ⬝ & {\textbackslash}vysmblksquare & Black Very Small Square \\
\hline
U+02B1E & ⬞ & {\textbackslash}vysmwhtsquare & White Very Small Square \\
\hline
U+02B1F & ⬟ & {\textbackslash}pentagonblack & Black Pentagon \\
\hline
U+02B20 & ⬠ & {\textbackslash}pentagon & White Pentagon \\
\hline
U+02B21 & ⬡ & {\textbackslash}varhexagon & White Hexagon \\
\hline
U+02B22 & ⬢ & {\textbackslash}varhexagonblack & Black Hexagon \\
\hline
U+02B23 & ⬣ & {\textbackslash}hexagonblack & Horizontal Black Hexagon \\
\hline
U+02B24 & ⬤ & {\textbackslash}lgblkcircle & Black Large Circle \\
\hline
U+02B25 & ⬥ & {\textbackslash}mdblkdiamond & Black Medium Diamond \\
\hline
U+02B26 & ⬦ & {\textbackslash}mdwhtdiamond & White Medium Diamond \\
\hline
U+02B27 & ⬧ & {\textbackslash}mdblklozenge & Black Medium Lozenge \\
\hline
U+02B28 & ⬨ & {\textbackslash}mdwhtlozenge & White Medium Lozenge \\
\hline
U+02B29 & ⬩ & {\textbackslash}smblkdiamond & Black Small Diamond \\
\hline
U+02B2A & ⬪ & {\textbackslash}smblklozenge & Black Small Lozenge \\
\hline
U+02B2B & ⬫ & {\textbackslash}smwhtlozenge & White Small Lozenge \\
\hline
U+02B2C & ⬬ & {\textbackslash}blkhorzoval & Black Horizontal Ellipse \\
\hline
U+02B2D & ⬭ & {\textbackslash}whthorzoval & White Horizontal Ellipse \\
\hline
U+02B2E & ⬮ & {\textbackslash}blkvertoval & Black Vertical Ellipse \\
\hline
U+02B2F & ⬯ & {\textbackslash}whtvertoval & White Vertical Ellipse \\
\hline
U+02B30 & ⬰ & {\textbackslash}circleonleftarrow & Left Arrow With Small Circle \\
\hline
U+02B31 & ⬱ & {\textbackslash}leftthreearrows & Three Leftwards Arrows \\
\hline
U+02B32 & ⬲ & {\textbackslash}leftarrowonoplus & Left Arrow With Circled Plus \\
\hline
U+02B33 & ⬳ & {\textbackslash}longleftsquigarrow & Long Leftwards Squiggle Arrow \\
\hline
U+02B34 & ⬴ & {\textbackslash}nvtwoheadleftarrow & Leftwards Two-Headed Arrow With Vertical Stroke \\
\hline
U+02B35 & ⬵ & {\textbackslash}nVtwoheadleftarrow & Leftwards Two-Headed Arrow With Double Vertical Stroke \\
\hline
U+02B36 & ⬶ & {\textbackslash}twoheadmapsfrom & Leftwards Two-Headed Arrow From Bar \\
\hline
U+02B37 & ⬷ & {\textbackslash}twoheadleftdbkarrow & Leftwards Two-Headed Triple Dash Arrow \\
\hline
U+02B38 & ⬸ & {\textbackslash}leftdotarrow & Leftwards Arrow With Dotted Stem \\
\hline
U+02B39 & ⬹ & {\textbackslash}nvleftarrowtail & Leftwards Arrow With Tail With Vertical Stroke \\
\hline
U+02B3A & ⬺ & {\textbackslash}nVleftarrowtail & Leftwards Arrow With Tail With Double Vertical Stroke \\
\hline
U+02B3B & ⬻ & {\textbackslash}twoheadleftarrowtail & Leftwards Two-Headed Arrow With Tail \\
\hline
U+02B3C & ⬼ & {\textbackslash}nvtwoheadleftarrowtail & Leftwards Two-Headed Arrow With Tail With Vertical Stroke \\
\hline
U+02B3D & ⬽ & {\textbackslash}nVtwoheadleftarrowtail & Leftwards Two-Headed Arrow With Tail With Double Vertical Stroke \\
\hline
U+02B3E & ⬾ & {\textbackslash}leftarrowx & Leftwards Arrow Through X \\
\hline
U+02B3F & ⬿ & {\textbackslash}leftcurvedarrow & Wave Arrow Pointing Directly Left \\
\hline
U+02B40 & ⭀ & {\textbackslash}equalleftarrow & Equals Sign Above Leftwards Arrow \\
\hline
U+02B41 & ⭁ & {\textbackslash}bsimilarleftarrow & Reverse Tilde Operator Above Leftwards Arrow \\
\hline
U+02B42 & ⭂ & {\textbackslash}leftarrowbackapprox & Leftwards Arrow Above Reverse Almost Equal To \\
\hline
U+02B43 & ⭃ & {\textbackslash}rightarrowgtr & Rightwards Arrow Through Greater-Than \\
\hline
U+02B44 & ⭄ & {\textbackslash}rightarrowsupset & Rightwards Arrow Through Superset \\
\hline
U+02B45 & ⭅ & {\textbackslash}LLeftarrow & Leftwards Quadruple Arrow \\
\hline
U+02B46 & ⭆ & {\textbackslash}RRightarrow & Rightwards Quadruple Arrow \\
\hline
U+02B47 & ⭇ & {\textbackslash}bsimilarrightarrow & Reverse Tilde Operator Above Rightwards Arrow \\
\hline
U+02B48 & ⭈ & {\textbackslash}rightarrowbackapprox & Rightwards Arrow Above Reverse Almost Equal To \\
\hline
U+02B49 & ⭉ & {\textbackslash}similarleftarrow & Tilde Operator Above Leftwards Arrow \\
\hline
U+02B4A & ⭊ & {\textbackslash}leftarrowapprox & Leftwards Arrow Above Almost Equal To \\
\hline
U+02B4B & ⭋ & {\textbackslash}leftarrowbsimilar & Leftwards Arrow Above Reverse Tilde Operator \\
\hline
U+02B4C & ⭌ & {\textbackslash}rightarrowbsimilar & Rightwards Arrow Above Reverse Tilde Operator \\
\hline
U+02B50 & ⭐ & {\textbackslash}medwhitestar, {\textbackslash}:star: & White Medium Star \\
\hline
U+02B51 & ⭑ & {\textbackslash}medblackstar & Black Small Star \\
\hline
U+02B52 & ⭒ & {\textbackslash}smwhitestar & White Small Star \\
\hline
U+02B53 & ⭓ & {\textbackslash}rightpentagonblack & Black Right-Pointing Pentagon \\
\hline
U+02B54 & ⭔ & {\textbackslash}rightpentagon & White Right-Pointing Pentagon \\
\hline
U+02B55 & ⭕ & {\textbackslash}:o: & Heavy Large Circle \\
\hline
U+02C7C & ⱼ & {\textbackslash}\_j & Latin Subscript Small Letter J \\
\hline
U+02C7D & ⱽ & {\textbackslash}{\textasciicircum}V & Modifier Letter Capital V \\
\hline
U+03012 & 〒 & {\textbackslash}postalmark & Postal Mark \\
\hline
U+03030 & 〰 & {\textbackslash}:wavy\_dash: & Wavy Dash \\
\hline
U+0303D & 〽 & {\textbackslash}:part\_alternation\_mark: & Part Alternation Mark \\
\hline
U+03297 & ㊗ & {\textbackslash}:congratulations: & Circled Ideograph Congratulation \\
\hline
U+03299 & ㊙ & {\textbackslash}:secret: & Circled Ideograph Secret \\
\hline
U+0A71B & ꜛ & {\textbackslash}{\textasciicircum}uparrow & Modifier Letter Raised Up Arrow \\
\hline
U+0A71C & ꜜ & {\textbackslash}{\textasciicircum}downarrow & Modifier Letter Raised Down Arrow \\
\hline
U+0A71D & ꜝ & {\textbackslash}{\textasciicircum}! & Modifier Letter Raised Exclamation Mark \\
\hline
U+1D400 & 𝐀 & {\textbackslash}bfA & Mathematical Bold Capital A \\
\hline
U+1D401 & 𝐁 & {\textbackslash}bfB & Mathematical Bold Capital B \\
\hline
U+1D402 & 𝐂 & {\textbackslash}bfC & Mathematical Bold Capital C \\
\hline
U+1D403 & 𝐃 & {\textbackslash}bfD & Mathematical Bold Capital D \\
\hline
U+1D404 & 𝐄 & {\textbackslash}bfE & Mathematical Bold Capital E \\
\hline
U+1D405 & 𝐅 & {\textbackslash}bfF & Mathematical Bold Capital F \\
\hline
U+1D406 & 𝐆 & {\textbackslash}bfG & Mathematical Bold Capital G \\
\hline
U+1D407 & 𝐇 & {\textbackslash}bfH & Mathematical Bold Capital H \\
\hline
U+1D408 & 𝐈 & {\textbackslash}bfI & Mathematical Bold Capital I \\
\hline
U+1D409 & 𝐉 & {\textbackslash}bfJ & Mathematical Bold Capital J \\
\hline
U+1D40A & 𝐊 & {\textbackslash}bfK & Mathematical Bold Capital K \\
\hline
U+1D40B & 𝐋 & {\textbackslash}bfL & Mathematical Bold Capital L \\
\hline
U+1D40C & 𝐌 & {\textbackslash}bfM & Mathematical Bold Capital M \\
\hline
U+1D40D & 𝐍 & {\textbackslash}bfN & Mathematical Bold Capital N \\
\hline
U+1D40E & 𝐎 & {\textbackslash}bfO & Mathematical Bold Capital O \\
\hline
U+1D40F & 𝐏 & {\textbackslash}bfP & Mathematical Bold Capital P \\
\hline
U+1D410 & 𝐐 & {\textbackslash}bfQ & Mathematical Bold Capital Q \\
\hline
U+1D411 & 𝐑 & {\textbackslash}bfR & Mathematical Bold Capital R \\
\hline
U+1D412 & 𝐒 & {\textbackslash}bfS & Mathematical Bold Capital S \\
\hline
U+1D413 & 𝐓 & {\textbackslash}bfT & Mathematical Bold Capital T \\
\hline
U+1D414 & 𝐔 & {\textbackslash}bfU & Mathematical Bold Capital U \\
\hline
U+1D415 & 𝐕 & {\textbackslash}bfV & Mathematical Bold Capital V \\
\hline
U+1D416 & 𝐖 & {\textbackslash}bfW & Mathematical Bold Capital W \\
\hline
U+1D417 & 𝐗 & {\textbackslash}bfX & Mathematical Bold Capital X \\
\hline
U+1D418 & 𝐘 & {\textbackslash}bfY & Mathematical Bold Capital Y \\
\hline
U+1D419 & 𝐙 & {\textbackslash}bfZ & Mathematical Bold Capital Z \\
\hline
U+1D41A & 𝐚 & {\textbackslash}bfa & Mathematical Bold Small A \\
\hline
U+1D41B & 𝐛 & {\textbackslash}bfb & Mathematical Bold Small B \\
\hline
U+1D41C & 𝐜 & {\textbackslash}bfc & Mathematical Bold Small C \\
\hline
U+1D41D & 𝐝 & {\textbackslash}bfd & Mathematical Bold Small D \\
\hline
U+1D41E & 𝐞 & {\textbackslash}bfe & Mathematical Bold Small E \\
\hline
U+1D41F & 𝐟 & {\textbackslash}bff & Mathematical Bold Small F \\
\hline
U+1D420 & 𝐠 & {\textbackslash}bfg & Mathematical Bold Small G \\
\hline
U+1D421 & 𝐡 & {\textbackslash}bfh & Mathematical Bold Small H \\
\hline
U+1D422 & 𝐢 & {\textbackslash}bfi & Mathematical Bold Small I \\
\hline
U+1D423 & 𝐣 & {\textbackslash}bfj & Mathematical Bold Small J \\
\hline
U+1D424 & 𝐤 & {\textbackslash}bfk & Mathematical Bold Small K \\
\hline
U+1D425 & 𝐥 & {\textbackslash}bfl & Mathematical Bold Small L \\
\hline
U+1D426 & 𝐦 & {\textbackslash}bfm & Mathematical Bold Small M \\
\hline
U+1D427 & 𝐧 & {\textbackslash}bfn & Mathematical Bold Small N \\
\hline
U+1D428 & 𝐨 & {\textbackslash}bfo & Mathematical Bold Small O \\
\hline
U+1D429 & 𝐩 & {\textbackslash}bfp & Mathematical Bold Small P \\
\hline
U+1D42A & 𝐪 & {\textbackslash}bfq & Mathematical Bold Small Q \\
\hline
U+1D42B & 𝐫 & {\textbackslash}bfr & Mathematical Bold Small R \\
\hline
U+1D42C & 𝐬 & {\textbackslash}bfs & Mathematical Bold Small S \\
\hline
U+1D42D & 𝐭 & {\textbackslash}bft & Mathematical Bold Small T \\
\hline
U+1D42E & 𝐮 & {\textbackslash}bfu & Mathematical Bold Small U \\
\hline
U+1D42F & 𝐯 & {\textbackslash}bfv & Mathematical Bold Small V \\
\hline
U+1D430 & 𝐰 & {\textbackslash}bfw & Mathematical Bold Small W \\
\hline
U+1D431 & 𝐱 & {\textbackslash}bfx & Mathematical Bold Small X \\
\hline
U+1D432 & 𝐲 & {\textbackslash}bfy & Mathematical Bold Small Y \\
\hline
U+1D433 & 𝐳 & {\textbackslash}bfz & Mathematical Bold Small Z \\
\hline
U+1D434 & 𝐴 & {\textbackslash}itA & Mathematical Italic Capital A \\
\hline
U+1D435 & 𝐵 & {\textbackslash}itB & Mathematical Italic Capital B \\
\hline
U+1D436 & 𝐶 & {\textbackslash}itC & Mathematical Italic Capital C \\
\hline
U+1D437 & 𝐷 & {\textbackslash}itD & Mathematical Italic Capital D \\
\hline
U+1D438 & 𝐸 & {\textbackslash}itE & Mathematical Italic Capital E \\
\hline
U+1D439 & 𝐹 & {\textbackslash}itF & Mathematical Italic Capital F \\
\hline
U+1D43A & 𝐺 & {\textbackslash}itG & Mathematical Italic Capital G \\
\hline
U+1D43B & 𝐻 & {\textbackslash}itH & Mathematical Italic Capital H \\
\hline
U+1D43C & 𝐼 & {\textbackslash}itI & Mathematical Italic Capital I \\
\hline
U+1D43D & 𝐽 & {\textbackslash}itJ & Mathematical Italic Capital J \\
\hline
U+1D43E & 𝐾 & {\textbackslash}itK & Mathematical Italic Capital K \\
\hline
U+1D43F & 𝐿 & {\textbackslash}itL & Mathematical Italic Capital L \\
\hline
U+1D440 & 𝑀 & {\textbackslash}itM & Mathematical Italic Capital M \\
\hline
U+1D441 & 𝑁 & {\textbackslash}itN & Mathematical Italic Capital N \\
\hline
U+1D442 & 𝑂 & {\textbackslash}itO & Mathematical Italic Capital O \\
\hline
U+1D443 & 𝑃 & {\textbackslash}itP & Mathematical Italic Capital P \\
\hline
U+1D444 & 𝑄 & {\textbackslash}itQ & Mathematical Italic Capital Q \\
\hline
U+1D445 & 𝑅 & {\textbackslash}itR & Mathematical Italic Capital R \\
\hline
U+1D446 & 𝑆 & {\textbackslash}itS & Mathematical Italic Capital S \\
\hline
U+1D447 & 𝑇 & {\textbackslash}itT & Mathematical Italic Capital T \\
\hline
U+1D448 & 𝑈 & {\textbackslash}itU & Mathematical Italic Capital U \\
\hline
U+1D449 & 𝑉 & {\textbackslash}itV & Mathematical Italic Capital V \\
\hline
U+1D44A & 𝑊 & {\textbackslash}itW & Mathematical Italic Capital W \\
\hline
U+1D44B & 𝑋 & {\textbackslash}itX & Mathematical Italic Capital X \\
\hline
U+1D44C & 𝑌 & {\textbackslash}itY & Mathematical Italic Capital Y \\
\hline
U+1D44D & 𝑍 & {\textbackslash}itZ & Mathematical Italic Capital Z \\
\hline
U+1D44E & 𝑎 & {\textbackslash}ita & Mathematical Italic Small A \\
\hline
U+1D44F & 𝑏 & {\textbackslash}itb & Mathematical Italic Small B \\
\hline
U+1D450 & 𝑐 & {\textbackslash}itc & Mathematical Italic Small C \\
\hline
U+1D451 & 𝑑 & {\textbackslash}itd & Mathematical Italic Small D \\
\hline
U+1D452 & 𝑒 & {\textbackslash}ite & Mathematical Italic Small E \\
\hline
U+1D453 & 𝑓 & {\textbackslash}itf & Mathematical Italic Small F \\
\hline
U+1D454 & 𝑔 & {\textbackslash}itg & Mathematical Italic Small G \\
\hline
U+1D456 & 𝑖 & {\textbackslash}iti & Mathematical Italic Small I \\
\hline
U+1D457 & 𝑗 & {\textbackslash}itj & Mathematical Italic Small J \\
\hline
U+1D458 & 𝑘 & {\textbackslash}itk & Mathematical Italic Small K \\
\hline
U+1D459 & 𝑙 & {\textbackslash}itl & Mathematical Italic Small L \\
\hline
U+1D45A & 𝑚 & {\textbackslash}itm & Mathematical Italic Small M \\
\hline
U+1D45B & 𝑛 & {\textbackslash}itn & Mathematical Italic Small N \\
\hline
U+1D45C & 𝑜 & {\textbackslash}ito & Mathematical Italic Small O \\
\hline
U+1D45D & 𝑝 & {\textbackslash}itp & Mathematical Italic Small P \\
\hline
U+1D45E & 𝑞 & {\textbackslash}itq & Mathematical Italic Small Q \\
\hline
U+1D45F & 𝑟 & {\textbackslash}itr & Mathematical Italic Small R \\
\hline
U+1D460 & 𝑠 & {\textbackslash}its & Mathematical Italic Small S \\
\hline
U+1D461 & 𝑡 & {\textbackslash}itt & Mathematical Italic Small T \\
\hline
U+1D462 & 𝑢 & {\textbackslash}itu & Mathematical Italic Small U \\
\hline
U+1D463 & 𝑣 & {\textbackslash}itv & Mathematical Italic Small V \\
\hline
U+1D464 & 𝑤 & {\textbackslash}itw & Mathematical Italic Small W \\
\hline
U+1D465 & 𝑥 & {\textbackslash}itx & Mathematical Italic Small X \\
\hline
U+1D466 & 𝑦 & {\textbackslash}ity & Mathematical Italic Small Y \\
\hline
U+1D467 & 𝑧 & {\textbackslash}itz & Mathematical Italic Small Z \\
\hline
U+1D468 & 𝑨 & {\textbackslash}biA & Mathematical Bold Italic Capital A \\
\hline
U+1D469 & 𝑩 & {\textbackslash}biB & Mathematical Bold Italic Capital B \\
\hline
U+1D46A & 𝑪 & {\textbackslash}biC & Mathematical Bold Italic Capital C \\
\hline
U+1D46B & 𝑫 & {\textbackslash}biD & Mathematical Bold Italic Capital D \\
\hline
U+1D46C & 𝑬 & {\textbackslash}biE & Mathematical Bold Italic Capital E \\
\hline
U+1D46D & 𝑭 & {\textbackslash}biF & Mathematical Bold Italic Capital F \\
\hline
U+1D46E & 𝑮 & {\textbackslash}biG & Mathematical Bold Italic Capital G \\
\hline
U+1D46F & 𝑯 & {\textbackslash}biH & Mathematical Bold Italic Capital H \\
\hline
U+1D470 & 𝑰 & {\textbackslash}biI & Mathematical Bold Italic Capital I \\
\hline
U+1D471 & 𝑱 & {\textbackslash}biJ & Mathematical Bold Italic Capital J \\
\hline
U+1D472 & 𝑲 & {\textbackslash}biK & Mathematical Bold Italic Capital K \\
\hline
U+1D473 & 𝑳 & {\textbackslash}biL & Mathematical Bold Italic Capital L \\
\hline
U+1D474 & 𝑴 & {\textbackslash}biM & Mathematical Bold Italic Capital M \\
\hline
U+1D475 & 𝑵 & {\textbackslash}biN & Mathematical Bold Italic Capital N \\
\hline
U+1D476 & 𝑶 & {\textbackslash}biO & Mathematical Bold Italic Capital O \\
\hline
U+1D477 & 𝑷 & {\textbackslash}biP & Mathematical Bold Italic Capital P \\
\hline
U+1D478 & 𝑸 & {\textbackslash}biQ & Mathematical Bold Italic Capital Q \\
\hline
U+1D479 & 𝑹 & {\textbackslash}biR & Mathematical Bold Italic Capital R \\
\hline
U+1D47A & 𝑺 & {\textbackslash}biS & Mathematical Bold Italic Capital S \\
\hline
U+1D47B & 𝑻 & {\textbackslash}biT & Mathematical Bold Italic Capital T \\
\hline
U+1D47C & 𝑼 & {\textbackslash}biU & Mathematical Bold Italic Capital U \\
\hline
U+1D47D & 𝑽 & {\textbackslash}biV & Mathematical Bold Italic Capital V \\
\hline
U+1D47E & 𝑾 & {\textbackslash}biW & Mathematical Bold Italic Capital W \\
\hline
U+1D47F & 𝑿 & {\textbackslash}biX & Mathematical Bold Italic Capital X \\
\hline
U+1D480 & 𝒀 & {\textbackslash}biY & Mathematical Bold Italic Capital Y \\
\hline
U+1D481 & 𝒁 & {\textbackslash}biZ & Mathematical Bold Italic Capital Z \\
\hline
U+1D482 & 𝒂 & {\textbackslash}bia & Mathematical Bold Italic Small A \\
\hline
U+1D483 & 𝒃 & {\textbackslash}bib & Mathematical Bold Italic Small B \\
\hline
U+1D484 & 𝒄 & {\textbackslash}bic & Mathematical Bold Italic Small C \\
\hline
U+1D485 & 𝒅 & {\textbackslash}bid & Mathematical Bold Italic Small D \\
\hline
U+1D486 & 𝒆 & {\textbackslash}bie & Mathematical Bold Italic Small E \\
\hline
U+1D487 & 𝒇 & {\textbackslash}bif & Mathematical Bold Italic Small F \\
\hline
U+1D488 & 𝒈 & {\textbackslash}big & Mathematical Bold Italic Small G \\
\hline
U+1D489 & 𝒉 & {\textbackslash}bih & Mathematical Bold Italic Small H \\
\hline
U+1D48A & 𝒊 & {\textbackslash}bii & Mathematical Bold Italic Small I \\
\hline
U+1D48B & 𝒋 & {\textbackslash}bij & Mathematical Bold Italic Small J \\
\hline
U+1D48C & 𝒌 & {\textbackslash}bik & Mathematical Bold Italic Small K \\
\hline
U+1D48D & 𝒍 & {\textbackslash}bil & Mathematical Bold Italic Small L \\
\hline
U+1D48E & 𝒎 & {\textbackslash}bim & Mathematical Bold Italic Small M \\
\hline
U+1D48F & 𝒏 & {\textbackslash}bin & Mathematical Bold Italic Small N \\
\hline
U+1D490 & 𝒐 & {\textbackslash}bio & Mathematical Bold Italic Small O \\
\hline
U+1D491 & 𝒑 & {\textbackslash}bip & Mathematical Bold Italic Small P \\
\hline
U+1D492 & 𝒒 & {\textbackslash}biq & Mathematical Bold Italic Small Q \\
\hline
U+1D493 & 𝒓 & {\textbackslash}bir & Mathematical Bold Italic Small R \\
\hline
U+1D494 & 𝒔 & {\textbackslash}bis & Mathematical Bold Italic Small S \\
\hline
U+1D495 & 𝒕 & {\textbackslash}bit & Mathematical Bold Italic Small T \\
\hline
U+1D496 & 𝒖 & {\textbackslash}biu & Mathematical Bold Italic Small U \\
\hline
U+1D497 & 𝒗 & {\textbackslash}biv & Mathematical Bold Italic Small V \\
\hline
U+1D498 & 𝒘 & {\textbackslash}biw & Mathematical Bold Italic Small W \\
\hline
U+1D499 & 𝒙 & {\textbackslash}bix & Mathematical Bold Italic Small X \\
\hline
U+1D49A & 𝒚 & {\textbackslash}biy & Mathematical Bold Italic Small Y \\
\hline
U+1D49B & 𝒛 & {\textbackslash}biz & Mathematical Bold Italic Small Z \\
\hline
U+1D49C & 𝒜 & {\textbackslash}scrA & Mathematical Script Capital A \\
\hline
U+1D49E & 𝒞 & {\textbackslash}scrC & Mathematical Script Capital C \\
\hline
U+1D49F & 𝒟 & {\textbackslash}scrD & Mathematical Script Capital D \\
\hline
U+1D4A2 & 𝒢 & {\textbackslash}scrG & Mathematical Script Capital G \\
\hline
U+1D4A5 & 𝒥 & {\textbackslash}scrJ & Mathematical Script Capital J \\
\hline
U+1D4A6 & 𝒦 & {\textbackslash}scrK & Mathematical Script Capital K \\
\hline
U+1D4A9 & 𝒩 & {\textbackslash}scrN & Mathematical Script Capital N \\
\hline
U+1D4AA & 𝒪 & {\textbackslash}scrO & Mathematical Script Capital O \\
\hline
U+1D4AB & 𝒫 & {\textbackslash}scrP & Mathematical Script Capital P \\
\hline
U+1D4AC & 𝒬 & {\textbackslash}scrQ & Mathematical Script Capital Q \\
\hline
U+1D4AE & 𝒮 & {\textbackslash}scrS & Mathematical Script Capital S \\
\hline
U+1D4AF & 𝒯 & {\textbackslash}scrT & Mathematical Script Capital T \\
\hline
U+1D4B0 & 𝒰 & {\textbackslash}scrU & Mathematical Script Capital U \\
\hline
U+1D4B1 & 𝒱 & {\textbackslash}scrV & Mathematical Script Capital V \\
\hline
U+1D4B2 & 𝒲 & {\textbackslash}scrW & Mathematical Script Capital W \\
\hline
U+1D4B3 & 𝒳 & {\textbackslash}scrX & Mathematical Script Capital X \\
\hline
U+1D4B4 & 𝒴 & {\textbackslash}scrY & Mathematical Script Capital Y \\
\hline
U+1D4B5 & 𝒵 & {\textbackslash}scrZ & Mathematical Script Capital Z \\
\hline
U+1D4B6 & 𝒶 & {\textbackslash}scra & Mathematical Script Small A \\
\hline
U+1D4B7 & 𝒷 & {\textbackslash}scrb & Mathematical Script Small B \\
\hline
U+1D4B8 & 𝒸 & {\textbackslash}scrc & Mathematical Script Small C \\
\hline
U+1D4B9 & 𝒹 & {\textbackslash}scrd & Mathematical Script Small D \\
\hline
U+1D4BB & 𝒻 & {\textbackslash}scrf & Mathematical Script Small F \\
\hline
U+1D4BD & 𝒽 & {\textbackslash}scrh & Mathematical Script Small H \\
\hline
U+1D4BE & 𝒾 & {\textbackslash}scri & Mathematical Script Small I \\
\hline
U+1D4BF & 𝒿 & {\textbackslash}scrj & Mathematical Script Small J \\
\hline
U+1D4C0 & 𝓀 & {\textbackslash}scrk & Mathematical Script Small K \\
\hline
U+1D4C1 & 𝓁 & {\textbackslash}scrl & Mathematical Script Small L \\
\hline
U+1D4C2 & 𝓂 & {\textbackslash}scrm & Mathematical Script Small M \\
\hline
U+1D4C3 & 𝓃 & {\textbackslash}scrn & Mathematical Script Small N \\
\hline
U+1D4C5 & 𝓅 & {\textbackslash}scrp & Mathematical Script Small P \\
\hline
U+1D4C6 & 𝓆 & {\textbackslash}scrq & Mathematical Script Small Q \\
\hline
U+1D4C7 & 𝓇 & {\textbackslash}scrr & Mathematical Script Small R \\
\hline
U+1D4C8 & 𝓈 & {\textbackslash}scrs & Mathematical Script Small S \\
\hline
U+1D4C9 & 𝓉 & {\textbackslash}scrt & Mathematical Script Small T \\
\hline
U+1D4CA & 𝓊 & {\textbackslash}scru & Mathematical Script Small U \\
\hline
U+1D4CB & 𝓋 & {\textbackslash}scrv & Mathematical Script Small V \\
\hline
U+1D4CC & 𝓌 & {\textbackslash}scrw & Mathematical Script Small W \\
\hline
U+1D4CD & 𝓍 & {\textbackslash}scrx & Mathematical Script Small X \\
\hline
U+1D4CE & 𝓎 & {\textbackslash}scry & Mathematical Script Small Y \\
\hline
U+1D4CF & 𝓏 & {\textbackslash}scrz & Mathematical Script Small Z \\
\hline
U+1D4D0 & 𝓐 & {\textbackslash}bscrA & Mathematical Bold Script Capital A \\
\hline
U+1D4D1 & 𝓑 & {\textbackslash}bscrB & Mathematical Bold Script Capital B \\
\hline
U+1D4D2 & 𝓒 & {\textbackslash}bscrC & Mathematical Bold Script Capital C \\
\hline
U+1D4D3 & 𝓓 & {\textbackslash}bscrD & Mathematical Bold Script Capital D \\
\hline
U+1D4D4 & 𝓔 & {\textbackslash}bscrE & Mathematical Bold Script Capital E \\
\hline
U+1D4D5 & 𝓕 & {\textbackslash}bscrF & Mathematical Bold Script Capital F \\
\hline
U+1D4D6 & 𝓖 & {\textbackslash}bscrG & Mathematical Bold Script Capital G \\
\hline
U+1D4D7 & 𝓗 & {\textbackslash}bscrH & Mathematical Bold Script Capital H \\
\hline
U+1D4D8 & 𝓘 & {\textbackslash}bscrI & Mathematical Bold Script Capital I \\
\hline
U+1D4D9 & 𝓙 & {\textbackslash}bscrJ & Mathematical Bold Script Capital J \\
\hline
U+1D4DA & 𝓚 & {\textbackslash}bscrK & Mathematical Bold Script Capital K \\
\hline
U+1D4DB & 𝓛 & {\textbackslash}bscrL & Mathematical Bold Script Capital L \\
\hline
U+1D4DC & 𝓜 & {\textbackslash}bscrM & Mathematical Bold Script Capital M \\
\hline
U+1D4DD & 𝓝 & {\textbackslash}bscrN & Mathematical Bold Script Capital N \\
\hline
U+1D4DE & 𝓞 & {\textbackslash}bscrO & Mathematical Bold Script Capital O \\
\hline
U+1D4DF & 𝓟 & {\textbackslash}bscrP & Mathematical Bold Script Capital P \\
\hline
U+1D4E0 & 𝓠 & {\textbackslash}bscrQ & Mathematical Bold Script Capital Q \\
\hline
U+1D4E1 & 𝓡 & {\textbackslash}bscrR & Mathematical Bold Script Capital R \\
\hline
U+1D4E2 & 𝓢 & {\textbackslash}bscrS & Mathematical Bold Script Capital S \\
\hline
U+1D4E3 & 𝓣 & {\textbackslash}bscrT & Mathematical Bold Script Capital T \\
\hline
U+1D4E4 & 𝓤 & {\textbackslash}bscrU & Mathematical Bold Script Capital U \\
\hline
U+1D4E5 & 𝓥 & {\textbackslash}bscrV & Mathematical Bold Script Capital V \\
\hline
U+1D4E6 & 𝓦 & {\textbackslash}bscrW & Mathematical Bold Script Capital W \\
\hline
U+1D4E7 & 𝓧 & {\textbackslash}bscrX & Mathematical Bold Script Capital X \\
\hline
U+1D4E8 & 𝓨 & {\textbackslash}bscrY & Mathematical Bold Script Capital Y \\
\hline
U+1D4E9 & 𝓩 & {\textbackslash}bscrZ & Mathematical Bold Script Capital Z \\
\hline
U+1D4EA & 𝓪 & {\textbackslash}bscra & Mathematical Bold Script Small A \\
\hline
U+1D4EB & 𝓫 & {\textbackslash}bscrb & Mathematical Bold Script Small B \\
\hline
U+1D4EC & 𝓬 & {\textbackslash}bscrc & Mathematical Bold Script Small C \\
\hline
U+1D4ED & 𝓭 & {\textbackslash}bscrd & Mathematical Bold Script Small D \\
\hline
U+1D4EE & 𝓮 & {\textbackslash}bscre & Mathematical Bold Script Small E \\
\hline
U+1D4EF & 𝓯 & {\textbackslash}bscrf & Mathematical Bold Script Small F \\
\hline
U+1D4F0 & 𝓰 & {\textbackslash}bscrg & Mathematical Bold Script Small G \\
\hline
U+1D4F1 & 𝓱 & {\textbackslash}bscrh & Mathematical Bold Script Small H \\
\hline
U+1D4F2 & 𝓲 & {\textbackslash}bscri & Mathematical Bold Script Small I \\
\hline
U+1D4F3 & 𝓳 & {\textbackslash}bscrj & Mathematical Bold Script Small J \\
\hline
U+1D4F4 & 𝓴 & {\textbackslash}bscrk & Mathematical Bold Script Small K \\
\hline
U+1D4F5 & 𝓵 & {\textbackslash}bscrl & Mathematical Bold Script Small L \\
\hline
U+1D4F6 & 𝓶 & {\textbackslash}bscrm & Mathematical Bold Script Small M \\
\hline
U+1D4F7 & 𝓷 & {\textbackslash}bscrn & Mathematical Bold Script Small N \\
\hline
U+1D4F8 & 𝓸 & {\textbackslash}bscro & Mathematical Bold Script Small O \\
\hline
U+1D4F9 & 𝓹 & {\textbackslash}bscrp & Mathematical Bold Script Small P \\
\hline
U+1D4FA & 𝓺 & {\textbackslash}bscrq & Mathematical Bold Script Small Q \\
\hline
U+1D4FB & 𝓻 & {\textbackslash}bscrr & Mathematical Bold Script Small R \\
\hline
U+1D4FC & 𝓼 & {\textbackslash}bscrs & Mathematical Bold Script Small S \\
\hline
U+1D4FD & 𝓽 & {\textbackslash}bscrt & Mathematical Bold Script Small T \\
\hline
U+1D4FE & 𝓾 & {\textbackslash}bscru & Mathematical Bold Script Small U \\
\hline
U+1D4FF & 𝓿 & {\textbackslash}bscrv & Mathematical Bold Script Small V \\
\hline
U+1D500 & 𝔀 & {\textbackslash}bscrw & Mathematical Bold Script Small W \\
\hline
U+1D501 & 𝔁 & {\textbackslash}bscrx & Mathematical Bold Script Small X \\
\hline
U+1D502 & 𝔂 & {\textbackslash}bscry & Mathematical Bold Script Small Y \\
\hline
U+1D503 & 𝔃 & {\textbackslash}bscrz & Mathematical Bold Script Small Z \\
\hline
U+1D504 & 𝔄 & {\textbackslash}frakA & Mathematical Fraktur Capital A \\
\hline
U+1D505 & 𝔅 & {\textbackslash}frakB & Mathematical Fraktur Capital B \\
\hline
U+1D507 & 𝔇 & {\textbackslash}frakD & Mathematical Fraktur Capital D \\
\hline
U+1D508 & 𝔈 & {\textbackslash}frakE & Mathematical Fraktur Capital E \\
\hline
U+1D509 & 𝔉 & {\textbackslash}frakF & Mathematical Fraktur Capital F \\
\hline
U+1D50A & 𝔊 & {\textbackslash}frakG & Mathematical Fraktur Capital G \\
\hline
U+1D50D & 𝔍 & {\textbackslash}frakJ & Mathematical Fraktur Capital J \\
\hline
U+1D50E & 𝔎 & {\textbackslash}frakK & Mathematical Fraktur Capital K \\
\hline
U+1D50F & 𝔏 & {\textbackslash}frakL & Mathematical Fraktur Capital L \\
\hline
U+1D510 & 𝔐 & {\textbackslash}frakM & Mathematical Fraktur Capital M \\
\hline
U+1D511 & 𝔑 & {\textbackslash}frakN & Mathematical Fraktur Capital N \\
\hline
U+1D512 & 𝔒 & {\textbackslash}frakO & Mathematical Fraktur Capital O \\
\hline
U+1D513 & 𝔓 & {\textbackslash}frakP & Mathematical Fraktur Capital P \\
\hline
U+1D514 & 𝔔 & {\textbackslash}frakQ & Mathematical Fraktur Capital Q \\
\hline
U+1D516 & 𝔖 & {\textbackslash}frakS & Mathematical Fraktur Capital S \\
\hline
U+1D517 & 𝔗 & {\textbackslash}frakT & Mathematical Fraktur Capital T \\
\hline
U+1D518 & 𝔘 & {\textbackslash}frakU & Mathematical Fraktur Capital U \\
\hline
U+1D519 & 𝔙 & {\textbackslash}frakV & Mathematical Fraktur Capital V \\
\hline
U+1D51A & 𝔚 & {\textbackslash}frakW & Mathematical Fraktur Capital W \\
\hline
U+1D51B & 𝔛 & {\textbackslash}frakX & Mathematical Fraktur Capital X \\
\hline
U+1D51C & 𝔜 & {\textbackslash}frakY & Mathematical Fraktur Capital Y \\
\hline
U+1D51E & 𝔞 & {\textbackslash}fraka & Mathematical Fraktur Small A \\
\hline
U+1D51F & 𝔟 & {\textbackslash}frakb & Mathematical Fraktur Small B \\
\hline
U+1D520 & 𝔠 & {\textbackslash}frakc & Mathematical Fraktur Small C \\
\hline
U+1D521 & 𝔡 & {\textbackslash}frakd & Mathematical Fraktur Small D \\
\hline
U+1D522 & 𝔢 & {\textbackslash}frake & Mathematical Fraktur Small E \\
\hline
U+1D523 & 𝔣 & {\textbackslash}frakf & Mathematical Fraktur Small F \\
\hline
U+1D524 & 𝔤 & {\textbackslash}frakg & Mathematical Fraktur Small G \\
\hline
U+1D525 & 𝔥 & {\textbackslash}frakh & Mathematical Fraktur Small H \\
\hline
U+1D526 & 𝔦 & {\textbackslash}fraki & Mathematical Fraktur Small I \\
\hline
U+1D527 & 𝔧 & {\textbackslash}frakj & Mathematical Fraktur Small J \\
\hline
U+1D528 & 𝔨 & {\textbackslash}frakk & Mathematical Fraktur Small K \\
\hline
U+1D529 & 𝔩 & {\textbackslash}frakl & Mathematical Fraktur Small L \\
\hline
U+1D52A & 𝔪 & {\textbackslash}frakm & Mathematical Fraktur Small M \\
\hline
U+1D52B & 𝔫 & {\textbackslash}frakn & Mathematical Fraktur Small N \\
\hline
U+1D52C & 𝔬 & {\textbackslash}frako & Mathematical Fraktur Small O \\
\hline
U+1D52D & 𝔭 & {\textbackslash}frakp & Mathematical Fraktur Small P \\
\hline
U+1D52E & 𝔮 & {\textbackslash}frakq & Mathematical Fraktur Small Q \\
\hline
U+1D52F & 𝔯 & {\textbackslash}frakr & Mathematical Fraktur Small R \\
\hline
U+1D530 & 𝔰 & {\textbackslash}fraks & Mathematical Fraktur Small S \\
\hline
U+1D531 & 𝔱 & {\textbackslash}frakt & Mathematical Fraktur Small T \\
\hline
U+1D532 & 𝔲 & {\textbackslash}fraku & Mathematical Fraktur Small U \\
\hline
U+1D533 & 𝔳 & {\textbackslash}frakv & Mathematical Fraktur Small V \\
\hline
U+1D534 & 𝔴 & {\textbackslash}frakw & Mathematical Fraktur Small W \\
\hline
U+1D535 & 𝔵 & {\textbackslash}frakx & Mathematical Fraktur Small X \\
\hline
U+1D536 & 𝔶 & {\textbackslash}fraky & Mathematical Fraktur Small Y \\
\hline
U+1D537 & 𝔷 & {\textbackslash}frakz & Mathematical Fraktur Small Z \\
\hline
U+1D538 & 𝔸 & {\textbackslash}bbA & Mathematical Double-Struck Capital A \\
\hline
U+1D539 & 𝔹 & {\textbackslash}bbB & Mathematical Double-Struck Capital B \\
\hline
U+1D53B & 𝔻 & {\textbackslash}bbD & Mathematical Double-Struck Capital D \\
\hline
U+1D53C & 𝔼 & {\textbackslash}bbE & Mathematical Double-Struck Capital E \\
\hline
U+1D53D & 𝔽 & {\textbackslash}bbF & Mathematical Double-Struck Capital F \\
\hline
U+1D53E & 𝔾 & {\textbackslash}bbG & Mathematical Double-Struck Capital G \\
\hline
U+1D540 & 𝕀 & {\textbackslash}bbI & Mathematical Double-Struck Capital I \\
\hline
U+1D541 & 𝕁 & {\textbackslash}bbJ & Mathematical Double-Struck Capital J \\
\hline
U+1D542 & 𝕂 & {\textbackslash}bbK & Mathematical Double-Struck Capital K \\
\hline
U+1D543 & 𝕃 & {\textbackslash}bbL & Mathematical Double-Struck Capital L \\
\hline
U+1D544 & 𝕄 & {\textbackslash}bbM & Mathematical Double-Struck Capital M \\
\hline
U+1D546 & 𝕆 & {\textbackslash}bbO & Mathematical Double-Struck Capital O \\
\hline
U+1D54A & 𝕊 & {\textbackslash}bbS & Mathematical Double-Struck Capital S \\
\hline
U+1D54B & 𝕋 & {\textbackslash}bbT & Mathematical Double-Struck Capital T \\
\hline
U+1D54C & 𝕌 & {\textbackslash}bbU & Mathematical Double-Struck Capital U \\
\hline
U+1D54D & 𝕍 & {\textbackslash}bbV & Mathematical Double-Struck Capital V \\
\hline
U+1D54E & 𝕎 & {\textbackslash}bbW & Mathematical Double-Struck Capital W \\
\hline
U+1D54F & 𝕏 & {\textbackslash}bbX & Mathematical Double-Struck Capital X \\
\hline
U+1D550 & 𝕐 & {\textbackslash}bbY & Mathematical Double-Struck Capital Y \\
\hline
U+1D552 & 𝕒 & {\textbackslash}bba & Mathematical Double-Struck Small A \\
\hline
U+1D553 & 𝕓 & {\textbackslash}bbb & Mathematical Double-Struck Small B \\
\hline
U+1D554 & 𝕔 & {\textbackslash}bbc & Mathematical Double-Struck Small C \\
\hline
U+1D555 & 𝕕 & {\textbackslash}bbd & Mathematical Double-Struck Small D \\
\hline
U+1D556 & 𝕖 & {\textbackslash}bbe & Mathematical Double-Struck Small E \\
\hline
U+1D557 & 𝕗 & {\textbackslash}bbf & Mathematical Double-Struck Small F \\
\hline
U+1D558 & 𝕘 & {\textbackslash}bbg & Mathematical Double-Struck Small G \\
\hline
U+1D559 & 𝕙 & {\textbackslash}bbh & Mathematical Double-Struck Small H \\
\hline
U+1D55A & 𝕚 & {\textbackslash}bbi & Mathematical Double-Struck Small I \\
\hline
U+1D55B & 𝕛 & {\textbackslash}bbj & Mathematical Double-Struck Small J \\
\hline
U+1D55C & 𝕜 & {\textbackslash}bbk & Mathematical Double-Struck Small K \\
\hline
U+1D55D & 𝕝 & {\textbackslash}bbl & Mathematical Double-Struck Small L \\
\hline
U+1D55E & 𝕞 & {\textbackslash}bbm & Mathematical Double-Struck Small M \\
\hline
U+1D55F & 𝕟 & {\textbackslash}bbn & Mathematical Double-Struck Small N \\
\hline
U+1D560 & 𝕠 & {\textbackslash}bbo & Mathematical Double-Struck Small O \\
\hline
U+1D561 & 𝕡 & {\textbackslash}bbp & Mathematical Double-Struck Small P \\
\hline
U+1D562 & 𝕢 & {\textbackslash}bbq & Mathematical Double-Struck Small Q \\
\hline
U+1D563 & 𝕣 & {\textbackslash}bbr & Mathematical Double-Struck Small R \\
\hline
U+1D564 & 𝕤 & {\textbackslash}bbs & Mathematical Double-Struck Small S \\
\hline
U+1D565 & 𝕥 & {\textbackslash}bbt & Mathematical Double-Struck Small T \\
\hline
U+1D566 & 𝕦 & {\textbackslash}bbu & Mathematical Double-Struck Small U \\
\hline
U+1D567 & 𝕧 & {\textbackslash}bbv & Mathematical Double-Struck Small V \\
\hline
U+1D568 & 𝕨 & {\textbackslash}bbw & Mathematical Double-Struck Small W \\
\hline
U+1D569 & 𝕩 & {\textbackslash}bbx & Mathematical Double-Struck Small X \\
\hline
U+1D56A & 𝕪 & {\textbackslash}bby & Mathematical Double-Struck Small Y \\
\hline
U+1D56B & 𝕫 & {\textbackslash}bbz & Mathematical Double-Struck Small Z \\
\hline
U+1D56C & 𝕬 & {\textbackslash}bfrakA & Mathematical Bold Fraktur Capital A \\
\hline
U+1D56D & 𝕭 & {\textbackslash}bfrakB & Mathematical Bold Fraktur Capital B \\
\hline
U+1D56E & 𝕮 & {\textbackslash}bfrakC & Mathematical Bold Fraktur Capital C \\
\hline
U+1D56F & 𝕯 & {\textbackslash}bfrakD & Mathematical Bold Fraktur Capital D \\
\hline
U+1D570 & 𝕰 & {\textbackslash}bfrakE & Mathematical Bold Fraktur Capital E \\
\hline
U+1D571 & 𝕱 & {\textbackslash}bfrakF & Mathematical Bold Fraktur Capital F \\
\hline
U+1D572 & 𝕲 & {\textbackslash}bfrakG & Mathematical Bold Fraktur Capital G \\
\hline
U+1D573 & 𝕳 & {\textbackslash}bfrakH & Mathematical Bold Fraktur Capital H \\
\hline
U+1D574 & 𝕴 & {\textbackslash}bfrakI & Mathematical Bold Fraktur Capital I \\
\hline
U+1D575 & 𝕵 & {\textbackslash}bfrakJ & Mathematical Bold Fraktur Capital J \\
\hline
U+1D576 & 𝕶 & {\textbackslash}bfrakK & Mathematical Bold Fraktur Capital K \\
\hline
U+1D577 & 𝕷 & {\textbackslash}bfrakL & Mathematical Bold Fraktur Capital L \\
\hline
U+1D578 & 𝕸 & {\textbackslash}bfrakM & Mathematical Bold Fraktur Capital M \\
\hline
U+1D579 & 𝕹 & {\textbackslash}bfrakN & Mathematical Bold Fraktur Capital N \\
\hline
U+1D57A & 𝕺 & {\textbackslash}bfrakO & Mathematical Bold Fraktur Capital O \\
\hline
U+1D57B & 𝕻 & {\textbackslash}bfrakP & Mathematical Bold Fraktur Capital P \\
\hline
U+1D57C & 𝕼 & {\textbackslash}bfrakQ & Mathematical Bold Fraktur Capital Q \\
\hline
U+1D57D & 𝕽 & {\textbackslash}bfrakR & Mathematical Bold Fraktur Capital R \\
\hline
U+1D57E & 𝕾 & {\textbackslash}bfrakS & Mathematical Bold Fraktur Capital S \\
\hline
U+1D57F & 𝕿 & {\textbackslash}bfrakT & Mathematical Bold Fraktur Capital T \\
\hline
U+1D580 & 𝖀 & {\textbackslash}bfrakU & Mathematical Bold Fraktur Capital U \\
\hline
U+1D581 & 𝖁 & {\textbackslash}bfrakV & Mathematical Bold Fraktur Capital V \\
\hline
U+1D582 & 𝖂 & {\textbackslash}bfrakW & Mathematical Bold Fraktur Capital W \\
\hline
U+1D583 & 𝖃 & {\textbackslash}bfrakX & Mathematical Bold Fraktur Capital X \\
\hline
U+1D584 & 𝖄 & {\textbackslash}bfrakY & Mathematical Bold Fraktur Capital Y \\
\hline
U+1D585 & 𝖅 & {\textbackslash}bfrakZ & Mathematical Bold Fraktur Capital Z \\
\hline
U+1D586 & 𝖆 & {\textbackslash}bfraka & Mathematical Bold Fraktur Small A \\
\hline
U+1D587 & 𝖇 & {\textbackslash}bfrakb & Mathematical Bold Fraktur Small B \\
\hline
U+1D588 & 𝖈 & {\textbackslash}bfrakc & Mathematical Bold Fraktur Small C \\
\hline
U+1D589 & 𝖉 & {\textbackslash}bfrakd & Mathematical Bold Fraktur Small D \\
\hline
U+1D58A & 𝖊 & {\textbackslash}bfrake & Mathematical Bold Fraktur Small E \\
\hline
U+1D58B & 𝖋 & {\textbackslash}bfrakf & Mathematical Bold Fraktur Small F \\
\hline
U+1D58C & 𝖌 & {\textbackslash}bfrakg & Mathematical Bold Fraktur Small G \\
\hline
U+1D58D & 𝖍 & {\textbackslash}bfrakh & Mathematical Bold Fraktur Small H \\
\hline
U+1D58E & 𝖎 & {\textbackslash}bfraki & Mathematical Bold Fraktur Small I \\
\hline
U+1D58F & 𝖏 & {\textbackslash}bfrakj & Mathematical Bold Fraktur Small J \\
\hline
U+1D590 & 𝖐 & {\textbackslash}bfrakk & Mathematical Bold Fraktur Small K \\
\hline
U+1D591 & 𝖑 & {\textbackslash}bfrakl & Mathematical Bold Fraktur Small L \\
\hline
U+1D592 & 𝖒 & {\textbackslash}bfrakm & Mathematical Bold Fraktur Small M \\
\hline
U+1D593 & 𝖓 & {\textbackslash}bfrakn & Mathematical Bold Fraktur Small N \\
\hline
U+1D594 & 𝖔 & {\textbackslash}bfrako & Mathematical Bold Fraktur Small O \\
\hline
U+1D595 & 𝖕 & {\textbackslash}bfrakp & Mathematical Bold Fraktur Small P \\
\hline
U+1D596 & 𝖖 & {\textbackslash}bfrakq & Mathematical Bold Fraktur Small Q \\
\hline
U+1D597 & 𝖗 & {\textbackslash}bfrakr & Mathematical Bold Fraktur Small R \\
\hline
U+1D598 & 𝖘 & {\textbackslash}bfraks & Mathematical Bold Fraktur Small S \\
\hline
U+1D599 & 𝖙 & {\textbackslash}bfrakt & Mathematical Bold Fraktur Small T \\
\hline
U+1D59A & 𝖚 & {\textbackslash}bfraku & Mathematical Bold Fraktur Small U \\
\hline
U+1D59B & 𝖛 & {\textbackslash}bfrakv & Mathematical Bold Fraktur Small V \\
\hline
U+1D59C & 𝖜 & {\textbackslash}bfrakw & Mathematical Bold Fraktur Small W \\
\hline
U+1D59D & 𝖝 & {\textbackslash}bfrakx & Mathematical Bold Fraktur Small X \\
\hline
U+1D59E & 𝖞 & {\textbackslash}bfraky & Mathematical Bold Fraktur Small Y \\
\hline
U+1D59F & 𝖟 & {\textbackslash}bfrakz & Mathematical Bold Fraktur Small Z \\
\hline
U+1D5A0 & 𝖠 & {\textbackslash}sansA & Mathematical Sans-Serif Capital A \\
\hline
U+1D5A1 & 𝖡 & {\textbackslash}sansB & Mathematical Sans-Serif Capital B \\
\hline
U+1D5A2 & 𝖢 & {\textbackslash}sansC & Mathematical Sans-Serif Capital C \\
\hline
U+1D5A3 & 𝖣 & {\textbackslash}sansD & Mathematical Sans-Serif Capital D \\
\hline
U+1D5A4 & 𝖤 & {\textbackslash}sansE & Mathematical Sans-Serif Capital E \\
\hline
U+1D5A5 & 𝖥 & {\textbackslash}sansF & Mathematical Sans-Serif Capital F \\
\hline
U+1D5A6 & 𝖦 & {\textbackslash}sansG & Mathematical Sans-Serif Capital G \\
\hline
U+1D5A7 & 𝖧 & {\textbackslash}sansH & Mathematical Sans-Serif Capital H \\
\hline
U+1D5A8 & 𝖨 & {\textbackslash}sansI & Mathematical Sans-Serif Capital I \\
\hline
U+1D5A9 & 𝖩 & {\textbackslash}sansJ & Mathematical Sans-Serif Capital J \\
\hline
U+1D5AA & 𝖪 & {\textbackslash}sansK & Mathematical Sans-Serif Capital K \\
\hline
U+1D5AB & 𝖫 & {\textbackslash}sansL & Mathematical Sans-Serif Capital L \\
\hline
U+1D5AC & 𝖬 & {\textbackslash}sansM & Mathematical Sans-Serif Capital M \\
\hline
U+1D5AD & 𝖭 & {\textbackslash}sansN & Mathematical Sans-Serif Capital N \\
\hline
U+1D5AE & 𝖮 & {\textbackslash}sansO & Mathematical Sans-Serif Capital O \\
\hline
U+1D5AF & 𝖯 & {\textbackslash}sansP & Mathematical Sans-Serif Capital P \\
\hline
U+1D5B0 & 𝖰 & {\textbackslash}sansQ & Mathematical Sans-Serif Capital Q \\
\hline
U+1D5B1 & 𝖱 & {\textbackslash}sansR & Mathematical Sans-Serif Capital R \\
\hline
U+1D5B2 & 𝖲 & {\textbackslash}sansS & Mathematical Sans-Serif Capital S \\
\hline
U+1D5B3 & 𝖳 & {\textbackslash}sansT & Mathematical Sans-Serif Capital T \\
\hline
U+1D5B4 & 𝖴 & {\textbackslash}sansU & Mathematical Sans-Serif Capital U \\
\hline
U+1D5B5 & 𝖵 & {\textbackslash}sansV & Mathematical Sans-Serif Capital V \\
\hline
U+1D5B6 & 𝖶 & {\textbackslash}sansW & Mathematical Sans-Serif Capital W \\
\hline
U+1D5B7 & 𝖷 & {\textbackslash}sansX & Mathematical Sans-Serif Capital X \\
\hline
U+1D5B8 & 𝖸 & {\textbackslash}sansY & Mathematical Sans-Serif Capital Y \\
\hline
U+1D5B9 & 𝖹 & {\textbackslash}sansZ & Mathematical Sans-Serif Capital Z \\
\hline
U+1D5BA & 𝖺 & {\textbackslash}sansa & Mathematical Sans-Serif Small A \\
\hline
U+1D5BB & 𝖻 & {\textbackslash}sansb & Mathematical Sans-Serif Small B \\
\hline
U+1D5BC & 𝖼 & {\textbackslash}sansc & Mathematical Sans-Serif Small C \\
\hline
U+1D5BD & 𝖽 & {\textbackslash}sansd & Mathematical Sans-Serif Small D \\
\hline
U+1D5BE & 𝖾 & {\textbackslash}sanse & Mathematical Sans-Serif Small E \\
\hline
U+1D5BF & 𝖿 & {\textbackslash}sansf & Mathematical Sans-Serif Small F \\
\hline
U+1D5C0 & 𝗀 & {\textbackslash}sansg & Mathematical Sans-Serif Small G \\
\hline
U+1D5C1 & 𝗁 & {\textbackslash}sansh & Mathematical Sans-Serif Small H \\
\hline
U+1D5C2 & 𝗂 & {\textbackslash}sansi & Mathematical Sans-Serif Small I \\
\hline
U+1D5C3 & 𝗃 & {\textbackslash}sansj & Mathematical Sans-Serif Small J \\
\hline
U+1D5C4 & 𝗄 & {\textbackslash}sansk & Mathematical Sans-Serif Small K \\
\hline
U+1D5C5 & 𝗅 & {\textbackslash}sansl & Mathematical Sans-Serif Small L \\
\hline
U+1D5C6 & 𝗆 & {\textbackslash}sansm & Mathematical Sans-Serif Small M \\
\hline
U+1D5C7 & 𝗇 & {\textbackslash}sansn & Mathematical Sans-Serif Small N \\
\hline
U+1D5C8 & 𝗈 & {\textbackslash}sanso & Mathematical Sans-Serif Small O \\
\hline
U+1D5C9 & 𝗉 & {\textbackslash}sansp & Mathematical Sans-Serif Small P \\
\hline
U+1D5CA & 𝗊 & {\textbackslash}sansq & Mathematical Sans-Serif Small Q \\
\hline
U+1D5CB & 𝗋 & {\textbackslash}sansr & Mathematical Sans-Serif Small R \\
\hline
U+1D5CC & 𝗌 & {\textbackslash}sanss & Mathematical Sans-Serif Small S \\
\hline
U+1D5CD & 𝗍 & {\textbackslash}sanst & Mathematical Sans-Serif Small T \\
\hline
U+1D5CE & 𝗎 & {\textbackslash}sansu & Mathematical Sans-Serif Small U \\
\hline
U+1D5CF & 𝗏 & {\textbackslash}sansv & Mathematical Sans-Serif Small V \\
\hline
U+1D5D0 & 𝗐 & {\textbackslash}sansw & Mathematical Sans-Serif Small W \\
\hline
U+1D5D1 & 𝗑 & {\textbackslash}sansx & Mathematical Sans-Serif Small X \\
\hline
U+1D5D2 & 𝗒 & {\textbackslash}sansy & Mathematical Sans-Serif Small Y \\
\hline
U+1D5D3 & 𝗓 & {\textbackslash}sansz & Mathematical Sans-Serif Small Z \\
\hline
U+1D5D4 & 𝗔 & {\textbackslash}bsansA & Mathematical Sans-Serif Bold Capital A \\
\hline
U+1D5D5 & 𝗕 & {\textbackslash}bsansB & Mathematical Sans-Serif Bold Capital B \\
\hline
U+1D5D6 & 𝗖 & {\textbackslash}bsansC & Mathematical Sans-Serif Bold Capital C \\
\hline
U+1D5D7 & 𝗗 & {\textbackslash}bsansD & Mathematical Sans-Serif Bold Capital D \\
\hline
U+1D5D8 & 𝗘 & {\textbackslash}bsansE & Mathematical Sans-Serif Bold Capital E \\
\hline
U+1D5D9 & 𝗙 & {\textbackslash}bsansF & Mathematical Sans-Serif Bold Capital F \\
\hline
U+1D5DA & 𝗚 & {\textbackslash}bsansG & Mathematical Sans-Serif Bold Capital G \\
\hline
U+1D5DB & 𝗛 & {\textbackslash}bsansH & Mathematical Sans-Serif Bold Capital H \\
\hline
U+1D5DC & 𝗜 & {\textbackslash}bsansI & Mathematical Sans-Serif Bold Capital I \\
\hline
U+1D5DD & 𝗝 & {\textbackslash}bsansJ & Mathematical Sans-Serif Bold Capital J \\
\hline
U+1D5DE & 𝗞 & {\textbackslash}bsansK & Mathematical Sans-Serif Bold Capital K \\
\hline
U+1D5DF & 𝗟 & {\textbackslash}bsansL & Mathematical Sans-Serif Bold Capital L \\
\hline
U+1D5E0 & 𝗠 & {\textbackslash}bsansM & Mathematical Sans-Serif Bold Capital M \\
\hline
U+1D5E1 & 𝗡 & {\textbackslash}bsansN & Mathematical Sans-Serif Bold Capital N \\
\hline
U+1D5E2 & 𝗢 & {\textbackslash}bsansO & Mathematical Sans-Serif Bold Capital O \\
\hline
U+1D5E3 & 𝗣 & {\textbackslash}bsansP & Mathematical Sans-Serif Bold Capital P \\
\hline
U+1D5E4 & 𝗤 & {\textbackslash}bsansQ & Mathematical Sans-Serif Bold Capital Q \\
\hline
U+1D5E5 & 𝗥 & {\textbackslash}bsansR & Mathematical Sans-Serif Bold Capital R \\
\hline
U+1D5E6 & 𝗦 & {\textbackslash}bsansS & Mathematical Sans-Serif Bold Capital S \\
\hline
U+1D5E7 & 𝗧 & {\textbackslash}bsansT & Mathematical Sans-Serif Bold Capital T \\
\hline
U+1D5E8 & 𝗨 & {\textbackslash}bsansU & Mathematical Sans-Serif Bold Capital U \\
\hline
U+1D5E9 & 𝗩 & {\textbackslash}bsansV & Mathematical Sans-Serif Bold Capital V \\
\hline
U+1D5EA & 𝗪 & {\textbackslash}bsansW & Mathematical Sans-Serif Bold Capital W \\
\hline
U+1D5EB & 𝗫 & {\textbackslash}bsansX & Mathematical Sans-Serif Bold Capital X \\
\hline
U+1D5EC & 𝗬 & {\textbackslash}bsansY & Mathematical Sans-Serif Bold Capital Y \\
\hline
U+1D5ED & 𝗭 & {\textbackslash}bsansZ & Mathematical Sans-Serif Bold Capital Z \\
\hline
U+1D5EE & 𝗮 & {\textbackslash}bsansa & Mathematical Sans-Serif Bold Small A \\
\hline
U+1D5EF & 𝗯 & {\textbackslash}bsansb & Mathematical Sans-Serif Bold Small B \\
\hline
U+1D5F0 & 𝗰 & {\textbackslash}bsansc & Mathematical Sans-Serif Bold Small C \\
\hline
U+1D5F1 & 𝗱 & {\textbackslash}bsansd & Mathematical Sans-Serif Bold Small D \\
\hline
U+1D5F2 & 𝗲 & {\textbackslash}bsanse & Mathematical Sans-Serif Bold Small E \\
\hline
U+1D5F3 & 𝗳 & {\textbackslash}bsansf & Mathematical Sans-Serif Bold Small F \\
\hline
U+1D5F4 & 𝗴 & {\textbackslash}bsansg & Mathematical Sans-Serif Bold Small G \\
\hline
U+1D5F5 & 𝗵 & {\textbackslash}bsansh & Mathematical Sans-Serif Bold Small H \\
\hline
U+1D5F6 & 𝗶 & {\textbackslash}bsansi & Mathematical Sans-Serif Bold Small I \\
\hline
U+1D5F7 & 𝗷 & {\textbackslash}bsansj & Mathematical Sans-Serif Bold Small J \\
\hline
U+1D5F8 & 𝗸 & {\textbackslash}bsansk & Mathematical Sans-Serif Bold Small K \\
\hline
U+1D5F9 & 𝗹 & {\textbackslash}bsansl & Mathematical Sans-Serif Bold Small L \\
\hline
U+1D5FA & 𝗺 & {\textbackslash}bsansm & Mathematical Sans-Serif Bold Small M \\
\hline
U+1D5FB & 𝗻 & {\textbackslash}bsansn & Mathematical Sans-Serif Bold Small N \\
\hline
U+1D5FC & 𝗼 & {\textbackslash}bsanso & Mathematical Sans-Serif Bold Small O \\
\hline
U+1D5FD & 𝗽 & {\textbackslash}bsansp & Mathematical Sans-Serif Bold Small P \\
\hline
U+1D5FE & 𝗾 & {\textbackslash}bsansq & Mathematical Sans-Serif Bold Small Q \\
\hline
U+1D5FF & 𝗿 & {\textbackslash}bsansr & Mathematical Sans-Serif Bold Small R \\
\hline
U+1D600 & 𝘀 & {\textbackslash}bsanss & Mathematical Sans-Serif Bold Small S \\
\hline
U+1D601 & 𝘁 & {\textbackslash}bsanst & Mathematical Sans-Serif Bold Small T \\
\hline
U+1D602 & 𝘂 & {\textbackslash}bsansu & Mathematical Sans-Serif Bold Small U \\
\hline
U+1D603 & 𝘃 & {\textbackslash}bsansv & Mathematical Sans-Serif Bold Small V \\
\hline
U+1D604 & 𝘄 & {\textbackslash}bsansw & Mathematical Sans-Serif Bold Small W \\
\hline
U+1D605 & 𝘅 & {\textbackslash}bsansx & Mathematical Sans-Serif Bold Small X \\
\hline
U+1D606 & 𝘆 & {\textbackslash}bsansy & Mathematical Sans-Serif Bold Small Y \\
\hline
U+1D607 & 𝘇 & {\textbackslash}bsansz & Mathematical Sans-Serif Bold Small Z \\
\hline
U+1D608 & 𝘈 & {\textbackslash}isansA & Mathematical Sans-Serif Italic Capital A \\
\hline
U+1D609 & 𝘉 & {\textbackslash}isansB & Mathematical Sans-Serif Italic Capital B \\
\hline
U+1D60A & 𝘊 & {\textbackslash}isansC & Mathematical Sans-Serif Italic Capital C \\
\hline
U+1D60B & 𝘋 & {\textbackslash}isansD & Mathematical Sans-Serif Italic Capital D \\
\hline
U+1D60C & 𝘌 & {\textbackslash}isansE & Mathematical Sans-Serif Italic Capital E \\
\hline
U+1D60D & 𝘍 & {\textbackslash}isansF & Mathematical Sans-Serif Italic Capital F \\
\hline
U+1D60E & 𝘎 & {\textbackslash}isansG & Mathematical Sans-Serif Italic Capital G \\
\hline
U+1D60F & 𝘏 & {\textbackslash}isansH & Mathematical Sans-Serif Italic Capital H \\
\hline
U+1D610 & 𝘐 & {\textbackslash}isansI & Mathematical Sans-Serif Italic Capital I \\
\hline
U+1D611 & 𝘑 & {\textbackslash}isansJ & Mathematical Sans-Serif Italic Capital J \\
\hline
U+1D612 & 𝘒 & {\textbackslash}isansK & Mathematical Sans-Serif Italic Capital K \\
\hline
U+1D613 & 𝘓 & {\textbackslash}isansL & Mathematical Sans-Serif Italic Capital L \\
\hline
U+1D614 & 𝘔 & {\textbackslash}isansM & Mathematical Sans-Serif Italic Capital M \\
\hline
U+1D615 & 𝘕 & {\textbackslash}isansN & Mathematical Sans-Serif Italic Capital N \\
\hline
U+1D616 & 𝘖 & {\textbackslash}isansO & Mathematical Sans-Serif Italic Capital O \\
\hline
U+1D617 & 𝘗 & {\textbackslash}isansP & Mathematical Sans-Serif Italic Capital P \\
\hline
U+1D618 & 𝘘 & {\textbackslash}isansQ & Mathematical Sans-Serif Italic Capital Q \\
\hline
U+1D619 & 𝘙 & {\textbackslash}isansR & Mathematical Sans-Serif Italic Capital R \\
\hline
U+1D61A & 𝘚 & {\textbackslash}isansS & Mathematical Sans-Serif Italic Capital S \\
\hline
U+1D61B & 𝘛 & {\textbackslash}isansT & Mathematical Sans-Serif Italic Capital T \\
\hline
U+1D61C & 𝘜 & {\textbackslash}isansU & Mathematical Sans-Serif Italic Capital U \\
\hline
U+1D61D & 𝘝 & {\textbackslash}isansV & Mathematical Sans-Serif Italic Capital V \\
\hline
U+1D61E & 𝘞 & {\textbackslash}isansW & Mathematical Sans-Serif Italic Capital W \\
\hline
U+1D61F & 𝘟 & {\textbackslash}isansX & Mathematical Sans-Serif Italic Capital X \\
\hline
U+1D620 & 𝘠 & {\textbackslash}isansY & Mathematical Sans-Serif Italic Capital Y \\
\hline
U+1D621 & 𝘡 & {\textbackslash}isansZ & Mathematical Sans-Serif Italic Capital Z \\
\hline
U+1D622 & 𝘢 & {\textbackslash}isansa & Mathematical Sans-Serif Italic Small A \\
\hline
U+1D623 & 𝘣 & {\textbackslash}isansb & Mathematical Sans-Serif Italic Small B \\
\hline
U+1D624 & 𝘤 & {\textbackslash}isansc & Mathematical Sans-Serif Italic Small C \\
\hline
U+1D625 & 𝘥 & {\textbackslash}isansd & Mathematical Sans-Serif Italic Small D \\
\hline
U+1D626 & 𝘦 & {\textbackslash}isanse & Mathematical Sans-Serif Italic Small E \\
\hline
U+1D627 & 𝘧 & {\textbackslash}isansf & Mathematical Sans-Serif Italic Small F \\
\hline
U+1D628 & 𝘨 & {\textbackslash}isansg & Mathematical Sans-Serif Italic Small G \\
\hline
U+1D629 & 𝘩 & {\textbackslash}isansh & Mathematical Sans-Serif Italic Small H \\
\hline
U+1D62A & 𝘪 & {\textbackslash}isansi & Mathematical Sans-Serif Italic Small I \\
\hline
U+1D62B & 𝘫 & {\textbackslash}isansj & Mathematical Sans-Serif Italic Small J \\
\hline
U+1D62C & 𝘬 & {\textbackslash}isansk & Mathematical Sans-Serif Italic Small K \\
\hline
U+1D62D & 𝘭 & {\textbackslash}isansl & Mathematical Sans-Serif Italic Small L \\
\hline
U+1D62E & 𝘮 & {\textbackslash}isansm & Mathematical Sans-Serif Italic Small M \\
\hline
U+1D62F & 𝘯 & {\textbackslash}isansn & Mathematical Sans-Serif Italic Small N \\
\hline
U+1D630 & 𝘰 & {\textbackslash}isanso & Mathematical Sans-Serif Italic Small O \\
\hline
U+1D631 & 𝘱 & {\textbackslash}isansp & Mathematical Sans-Serif Italic Small P \\
\hline
U+1D632 & 𝘲 & {\textbackslash}isansq & Mathematical Sans-Serif Italic Small Q \\
\hline
U+1D633 & 𝘳 & {\textbackslash}isansr & Mathematical Sans-Serif Italic Small R \\
\hline
U+1D634 & 𝘴 & {\textbackslash}isanss & Mathematical Sans-Serif Italic Small S \\
\hline
U+1D635 & 𝘵 & {\textbackslash}isanst & Mathematical Sans-Serif Italic Small T \\
\hline
U+1D636 & 𝘶 & {\textbackslash}isansu & Mathematical Sans-Serif Italic Small U \\
\hline
U+1D637 & 𝘷 & {\textbackslash}isansv & Mathematical Sans-Serif Italic Small V \\
\hline
U+1D638 & 𝘸 & {\textbackslash}isansw & Mathematical Sans-Serif Italic Small W \\
\hline
U+1D639 & 𝘹 & {\textbackslash}isansx & Mathematical Sans-Serif Italic Small X \\
\hline
U+1D63A & 𝘺 & {\textbackslash}isansy & Mathematical Sans-Serif Italic Small Y \\
\hline
U+1D63B & 𝘻 & {\textbackslash}isansz & Mathematical Sans-Serif Italic Small Z \\
\hline
U+1D63C & 𝘼 & {\textbackslash}bisansA & Mathematical Sans-Serif Bold Italic Capital A \\
\hline
U+1D63D & 𝘽 & {\textbackslash}bisansB & Mathematical Sans-Serif Bold Italic Capital B \\
\hline
U+1D63E & 𝘾 & {\textbackslash}bisansC & Mathematical Sans-Serif Bold Italic Capital C \\
\hline
U+1D63F & 𝘿 & {\textbackslash}bisansD & Mathematical Sans-Serif Bold Italic Capital D \\
\hline
U+1D640 & 𝙀 & {\textbackslash}bisansE & Mathematical Sans-Serif Bold Italic Capital E \\
\hline
U+1D641 & 𝙁 & {\textbackslash}bisansF & Mathematical Sans-Serif Bold Italic Capital F \\
\hline
U+1D642 & 𝙂 & {\textbackslash}bisansG & Mathematical Sans-Serif Bold Italic Capital G \\
\hline
U+1D643 & 𝙃 & {\textbackslash}bisansH & Mathematical Sans-Serif Bold Italic Capital H \\
\hline
U+1D644 & 𝙄 & {\textbackslash}bisansI & Mathematical Sans-Serif Bold Italic Capital I \\
\hline
U+1D645 & 𝙅 & {\textbackslash}bisansJ & Mathematical Sans-Serif Bold Italic Capital J \\
\hline
U+1D646 & 𝙆 & {\textbackslash}bisansK & Mathematical Sans-Serif Bold Italic Capital K \\
\hline
U+1D647 & 𝙇 & {\textbackslash}bisansL & Mathematical Sans-Serif Bold Italic Capital L \\
\hline
U+1D648 & 𝙈 & {\textbackslash}bisansM & Mathematical Sans-Serif Bold Italic Capital M \\
\hline
U+1D649 & 𝙉 & {\textbackslash}bisansN & Mathematical Sans-Serif Bold Italic Capital N \\
\hline
U+1D64A & 𝙊 & {\textbackslash}bisansO & Mathematical Sans-Serif Bold Italic Capital O \\
\hline
U+1D64B & 𝙋 & {\textbackslash}bisansP & Mathematical Sans-Serif Bold Italic Capital P \\
\hline
U+1D64C & 𝙌 & {\textbackslash}bisansQ & Mathematical Sans-Serif Bold Italic Capital Q \\
\hline
U+1D64D & 𝙍 & {\textbackslash}bisansR & Mathematical Sans-Serif Bold Italic Capital R \\
\hline
U+1D64E & 𝙎 & {\textbackslash}bisansS & Mathematical Sans-Serif Bold Italic Capital S \\
\hline
U+1D64F & 𝙏 & {\textbackslash}bisansT & Mathematical Sans-Serif Bold Italic Capital T \\
\hline
U+1D650 & 𝙐 & {\textbackslash}bisansU & Mathematical Sans-Serif Bold Italic Capital U \\
\hline
U+1D651 & 𝙑 & {\textbackslash}bisansV & Mathematical Sans-Serif Bold Italic Capital V \\
\hline
U+1D652 & 𝙒 & {\textbackslash}bisansW & Mathematical Sans-Serif Bold Italic Capital W \\
\hline
U+1D653 & 𝙓 & {\textbackslash}bisansX & Mathematical Sans-Serif Bold Italic Capital X \\
\hline
U+1D654 & 𝙔 & {\textbackslash}bisansY & Mathematical Sans-Serif Bold Italic Capital Y \\
\hline
U+1D655 & 𝙕 & {\textbackslash}bisansZ & Mathematical Sans-Serif Bold Italic Capital Z \\
\hline
U+1D656 & 𝙖 & {\textbackslash}bisansa & Mathematical Sans-Serif Bold Italic Small A \\
\hline
U+1D657 & 𝙗 & {\textbackslash}bisansb & Mathematical Sans-Serif Bold Italic Small B \\
\hline
U+1D658 & 𝙘 & {\textbackslash}bisansc & Mathematical Sans-Serif Bold Italic Small C \\
\hline
U+1D659 & 𝙙 & {\textbackslash}bisansd & Mathematical Sans-Serif Bold Italic Small D \\
\hline
U+1D65A & 𝙚 & {\textbackslash}bisanse & Mathematical Sans-Serif Bold Italic Small E \\
\hline
U+1D65B & 𝙛 & {\textbackslash}bisansf & Mathematical Sans-Serif Bold Italic Small F \\
\hline
U+1D65C & 𝙜 & {\textbackslash}bisansg & Mathematical Sans-Serif Bold Italic Small G \\
\hline
U+1D65D & 𝙝 & {\textbackslash}bisansh & Mathematical Sans-Serif Bold Italic Small H \\
\hline
U+1D65E & 𝙞 & {\textbackslash}bisansi & Mathematical Sans-Serif Bold Italic Small I \\
\hline
U+1D65F & 𝙟 & {\textbackslash}bisansj & Mathematical Sans-Serif Bold Italic Small J \\
\hline
U+1D660 & 𝙠 & {\textbackslash}bisansk & Mathematical Sans-Serif Bold Italic Small K \\
\hline
U+1D661 & 𝙡 & {\textbackslash}bisansl & Mathematical Sans-Serif Bold Italic Small L \\
\hline
U+1D662 & 𝙢 & {\textbackslash}bisansm & Mathematical Sans-Serif Bold Italic Small M \\
\hline
U+1D663 & 𝙣 & {\textbackslash}bisansn & Mathematical Sans-Serif Bold Italic Small N \\
\hline
U+1D664 & 𝙤 & {\textbackslash}bisanso & Mathematical Sans-Serif Bold Italic Small O \\
\hline
U+1D665 & 𝙥 & {\textbackslash}bisansp & Mathematical Sans-Serif Bold Italic Small P \\
\hline
U+1D666 & 𝙦 & {\textbackslash}bisansq & Mathematical Sans-Serif Bold Italic Small Q \\
\hline
U+1D667 & 𝙧 & {\textbackslash}bisansr & Mathematical Sans-Serif Bold Italic Small R \\
\hline
U+1D668 & 𝙨 & {\textbackslash}bisanss & Mathematical Sans-Serif Bold Italic Small S \\
\hline
U+1D669 & 𝙩 & {\textbackslash}bisanst & Mathematical Sans-Serif Bold Italic Small T \\
\hline
U+1D66A & 𝙪 & {\textbackslash}bisansu & Mathematical Sans-Serif Bold Italic Small U \\
\hline
U+1D66B & 𝙫 & {\textbackslash}bisansv & Mathematical Sans-Serif Bold Italic Small V \\
\hline
U+1D66C & 𝙬 & {\textbackslash}bisansw & Mathematical Sans-Serif Bold Italic Small W \\
\hline
U+1D66D & 𝙭 & {\textbackslash}bisansx & Mathematical Sans-Serif Bold Italic Small X \\
\hline
U+1D66E & 𝙮 & {\textbackslash}bisansy & Mathematical Sans-Serif Bold Italic Small Y \\
\hline
U+1D66F & 𝙯 & {\textbackslash}bisansz & Mathematical Sans-Serif Bold Italic Small Z \\
\hline
U+1D670 & 𝙰 & {\textbackslash}ttA & Mathematical Monospace Capital A \\
\hline
U+1D671 & 𝙱 & {\textbackslash}ttB & Mathematical Monospace Capital B \\
\hline
U+1D672 & 𝙲 & {\textbackslash}ttC & Mathematical Monospace Capital C \\
\hline
U+1D673 & 𝙳 & {\textbackslash}ttD & Mathematical Monospace Capital D \\
\hline
U+1D674 & 𝙴 & {\textbackslash}ttE & Mathematical Monospace Capital E \\
\hline
U+1D675 & 𝙵 & {\textbackslash}ttF & Mathematical Monospace Capital F \\
\hline
U+1D676 & 𝙶 & {\textbackslash}ttG & Mathematical Monospace Capital G \\
\hline
U+1D677 & 𝙷 & {\textbackslash}ttH & Mathematical Monospace Capital H \\
\hline
U+1D678 & 𝙸 & {\textbackslash}ttI & Mathematical Monospace Capital I \\
\hline
U+1D679 & 𝙹 & {\textbackslash}ttJ & Mathematical Monospace Capital J \\
\hline
U+1D67A & 𝙺 & {\textbackslash}ttK & Mathematical Monospace Capital K \\
\hline
U+1D67B & 𝙻 & {\textbackslash}ttL & Mathematical Monospace Capital L \\
\hline
U+1D67C & 𝙼 & {\textbackslash}ttM & Mathematical Monospace Capital M \\
\hline
U+1D67D & 𝙽 & {\textbackslash}ttN & Mathematical Monospace Capital N \\
\hline
U+1D67E & 𝙾 & {\textbackslash}ttO & Mathematical Monospace Capital O \\
\hline
U+1D67F & 𝙿 & {\textbackslash}ttP & Mathematical Monospace Capital P \\
\hline
U+1D680 & 𝚀 & {\textbackslash}ttQ & Mathematical Monospace Capital Q \\
\hline
U+1D681 & 𝚁 & {\textbackslash}ttR & Mathematical Monospace Capital R \\
\hline
U+1D682 & 𝚂 & {\textbackslash}ttS & Mathematical Monospace Capital S \\
\hline
U+1D683 & 𝚃 & {\textbackslash}ttT & Mathematical Monospace Capital T \\
\hline
U+1D684 & 𝚄 & {\textbackslash}ttU & Mathematical Monospace Capital U \\
\hline
U+1D685 & 𝚅 & {\textbackslash}ttV & Mathematical Monospace Capital V \\
\hline
U+1D686 & 𝚆 & {\textbackslash}ttW & Mathematical Monospace Capital W \\
\hline
U+1D687 & 𝚇 & {\textbackslash}ttX & Mathematical Monospace Capital X \\
\hline
U+1D688 & 𝚈 & {\textbackslash}ttY & Mathematical Monospace Capital Y \\
\hline
U+1D689 & 𝚉 & {\textbackslash}ttZ & Mathematical Monospace Capital Z \\
\hline
U+1D68A & 𝚊 & {\textbackslash}tta & Mathematical Monospace Small A \\
\hline
U+1D68B & 𝚋 & {\textbackslash}ttb & Mathematical Monospace Small B \\
\hline
U+1D68C & 𝚌 & {\textbackslash}ttc & Mathematical Monospace Small C \\
\hline
U+1D68D & 𝚍 & {\textbackslash}ttd & Mathematical Monospace Small D \\
\hline
U+1D68E & 𝚎 & {\textbackslash}tte & Mathematical Monospace Small E \\
\hline
U+1D68F & 𝚏 & {\textbackslash}ttf & Mathematical Monospace Small F \\
\hline
U+1D690 & 𝚐 & {\textbackslash}ttg & Mathematical Monospace Small G \\
\hline
U+1D691 & 𝚑 & {\textbackslash}tth & Mathematical Monospace Small H \\
\hline
U+1D692 & 𝚒 & {\textbackslash}tti & Mathematical Monospace Small I \\
\hline
U+1D693 & 𝚓 & {\textbackslash}ttj & Mathematical Monospace Small J \\
\hline
U+1D694 & 𝚔 & {\textbackslash}ttk & Mathematical Monospace Small K \\
\hline
U+1D695 & 𝚕 & {\textbackslash}ttl & Mathematical Monospace Small L \\
\hline
U+1D696 & 𝚖 & {\textbackslash}ttm & Mathematical Monospace Small M \\
\hline
U+1D697 & 𝚗 & {\textbackslash}ttn & Mathematical Monospace Small N \\
\hline
U+1D698 & 𝚘 & {\textbackslash}tto & Mathematical Monospace Small O \\
\hline
U+1D699 & 𝚙 & {\textbackslash}ttp & Mathematical Monospace Small P \\
\hline
U+1D69A & 𝚚 & {\textbackslash}ttq & Mathematical Monospace Small Q \\
\hline
U+1D69B & 𝚛 & {\textbackslash}ttr & Mathematical Monospace Small R \\
\hline
U+1D69C & 𝚜 & {\textbackslash}tts & Mathematical Monospace Small S \\
\hline
U+1D69D & 𝚝 & {\textbackslash}ttt & Mathematical Monospace Small T \\
\hline
U+1D69E & 𝚞 & {\textbackslash}ttu & Mathematical Monospace Small U \\
\hline
U+1D69F & 𝚟 & {\textbackslash}ttv & Mathematical Monospace Small V \\
\hline
U+1D6A0 & 𝚠 & {\textbackslash}ttw & Mathematical Monospace Small W \\
\hline
U+1D6A1 & 𝚡 & {\textbackslash}ttx & Mathematical Monospace Small X \\
\hline
U+1D6A2 & 𝚢 & {\textbackslash}tty & Mathematical Monospace Small Y \\
\hline
U+1D6A3 & 𝚣 & {\textbackslash}ttz & Mathematical Monospace Small Z \\
\hline
U+1D6A4 & 𝚤 & {\textbackslash}itimath & Mathematical Italic Small Dotless I \\
\hline
U+1D6A5 & 𝚥 & {\textbackslash}itjmath & Mathematical Italic Small Dotless J \\
\hline
U+1D6A8 & 𝚨 & {\textbackslash}bfAlpha & Mathematical Bold Capital Alpha \\
\hline
U+1D6A9 & 𝚩 & {\textbackslash}bfBeta & Mathematical Bold Capital Beta \\
\hline
U+1D6AA & 𝚪 & {\textbackslash}bfGamma & Mathematical Bold Capital Gamma \\
\hline
U+1D6AB & 𝚫 & {\textbackslash}bfDelta & Mathematical Bold Capital Delta \\
\hline
U+1D6AC & 𝚬 & {\textbackslash}bfEpsilon & Mathematical Bold Capital Epsilon \\
\hline
U+1D6AD & 𝚭 & {\textbackslash}bfZeta & Mathematical Bold Capital Zeta \\
\hline
U+1D6AE & 𝚮 & {\textbackslash}bfEta & Mathematical Bold Capital Eta \\
\hline
U+1D6AF & 𝚯 & {\textbackslash}bfTheta & Mathematical Bold Capital Theta \\
\hline
U+1D6B0 & 𝚰 & {\textbackslash}bfIota & Mathematical Bold Capital Iota \\
\hline
U+1D6B1 & 𝚱 & {\textbackslash}bfKappa & Mathematical Bold Capital Kappa \\
\hline
U+1D6B2 & 𝚲 & {\textbackslash}bfLambda & Mathematical Bold Capital Lamda \\
\hline
U+1D6B3 & 𝚳 & {\textbackslash}bfMu & Mathematical Bold Capital Mu \\
\hline
U+1D6B4 & 𝚴 & {\textbackslash}bfNu & Mathematical Bold Capital Nu \\
\hline
U+1D6B5 & 𝚵 & {\textbackslash}bfXi & Mathematical Bold Capital Xi \\
\hline
U+1D6B6 & 𝚶 & {\textbackslash}bfOmicron & Mathematical Bold Capital Omicron \\
\hline
U+1D6B7 & 𝚷 & {\textbackslash}bfPi & Mathematical Bold Capital Pi \\
\hline
U+1D6B8 & 𝚸 & {\textbackslash}bfRho & Mathematical Bold Capital Rho \\
\hline
U+1D6B9 & 𝚹 & {\textbackslash}bfvarTheta & Mathematical Bold Capital Theta Symbol \\
\hline
U+1D6BA & 𝚺 & {\textbackslash}bfSigma & Mathematical Bold Capital Sigma \\
\hline
U+1D6BB & 𝚻 & {\textbackslash}bfTau & Mathematical Bold Capital Tau \\
\hline
U+1D6BC & 𝚼 & {\textbackslash}bfUpsilon & Mathematical Bold Capital Upsilon \\
\hline
U+1D6BD & 𝚽 & {\textbackslash}bfPhi & Mathematical Bold Capital Phi \\
\hline
U+1D6BE & 𝚾 & {\textbackslash}bfChi & Mathematical Bold Capital Chi \\
\hline
U+1D6BF & 𝚿 & {\textbackslash}bfPsi & Mathematical Bold Capital Psi \\
\hline
U+1D6C0 & 𝛀 & {\textbackslash}bfOmega & Mathematical Bold Capital Omega \\
\hline
U+1D6C1 & 𝛁 & {\textbackslash}bfnabla & Mathematical Bold Nabla \\
\hline
U+1D6C2 & 𝛂 & {\textbackslash}bfalpha & Mathematical Bold Small Alpha \\
\hline
U+1D6C3 & 𝛃 & {\textbackslash}bfbeta & Mathematical Bold Small Beta \\
\hline
U+1D6C4 & 𝛄 & {\textbackslash}bfgamma & Mathematical Bold Small Gamma \\
\hline
U+1D6C5 & 𝛅 & {\textbackslash}bfdelta & Mathematical Bold Small Delta \\
\hline
U+1D6C6 & 𝛆 & {\textbackslash}bfepsilon & Mathematical Bold Small Epsilon \\
\hline
U+1D6C7 & 𝛇 & {\textbackslash}bfzeta & Mathematical Bold Small Zeta \\
\hline
U+1D6C8 & 𝛈 & {\textbackslash}bfeta & Mathematical Bold Small Eta \\
\hline
U+1D6C9 & 𝛉 & {\textbackslash}bftheta & Mathematical Bold Small Theta \\
\hline
U+1D6CA & 𝛊 & {\textbackslash}bfiota & Mathematical Bold Small Iota \\
\hline
U+1D6CB & 𝛋 & {\textbackslash}bfkappa & Mathematical Bold Small Kappa \\
\hline
U+1D6CC & 𝛌 & {\textbackslash}bflambda & Mathematical Bold Small Lamda \\
\hline
U+1D6CD & 𝛍 & {\textbackslash}bfmu & Mathematical Bold Small Mu \\
\hline
U+1D6CE & 𝛎 & {\textbackslash}bfnu & Mathematical Bold Small Nu \\
\hline
U+1D6CF & 𝛏 & {\textbackslash}bfxi & Mathematical Bold Small Xi \\
\hline
U+1D6D0 & 𝛐 & {\textbackslash}bfomicron & Mathematical Bold Small Omicron \\
\hline
U+1D6D1 & 𝛑 & {\textbackslash}bfpi & Mathematical Bold Small Pi \\
\hline
U+1D6D2 & 𝛒 & {\textbackslash}bfrho & Mathematical Bold Small Rho \\
\hline
U+1D6D3 & 𝛓 & {\textbackslash}bfvarsigma & Mathematical Bold Small Final Sigma \\
\hline
U+1D6D4 & 𝛔 & {\textbackslash}bfsigma & Mathematical Bold Small Sigma \\
\hline
U+1D6D5 & 𝛕 & {\textbackslash}bftau & Mathematical Bold Small Tau \\
\hline
U+1D6D6 & 𝛖 & {\textbackslash}bfupsilon & Mathematical Bold Small Upsilon \\
\hline
U+1D6D7 & 𝛗 & {\textbackslash}bfvarphi & Mathematical Bold Small Phi \\
\hline
U+1D6D8 & 𝛘 & {\textbackslash}bfchi & Mathematical Bold Small Chi \\
\hline
U+1D6D9 & 𝛙 & {\textbackslash}bfpsi & Mathematical Bold Small Psi \\
\hline
U+1D6DA & 𝛚 & {\textbackslash}bfomega & Mathematical Bold Small Omega \\
\hline
U+1D6DB & 𝛛 & {\textbackslash}bfpartial & Mathematical Bold Partial Differential \\
\hline
U+1D6DC & 𝛜 & {\textbackslash}bfvarepsilon & Mathematical Bold Epsilon Symbol \\
\hline
U+1D6DD & 𝛝 & {\textbackslash}bfvartheta & Mathematical Bold Theta Symbol \\
\hline
U+1D6DE & 𝛞 & {\textbackslash}bfvarkappa & Mathematical Bold Kappa Symbol \\
\hline
U+1D6DF & 𝛟 & {\textbackslash}bfphi & Mathematical Bold Phi Symbol \\
\hline
U+1D6E0 & 𝛠 & {\textbackslash}bfvarrho & Mathematical Bold Rho Symbol \\
\hline
U+1D6E1 & 𝛡 & {\textbackslash}bfvarpi & Mathematical Bold Pi Symbol \\
\hline
U+1D6E2 & 𝛢 & {\textbackslash}itAlpha & Mathematical Italic Capital Alpha \\
\hline
U+1D6E3 & 𝛣 & {\textbackslash}itBeta & Mathematical Italic Capital Beta \\
\hline
U+1D6E4 & 𝛤 & {\textbackslash}itGamma & Mathematical Italic Capital Gamma \\
\hline
U+1D6E5 & 𝛥 & {\textbackslash}itDelta & Mathematical Italic Capital Delta \\
\hline
U+1D6E6 & 𝛦 & {\textbackslash}itEpsilon & Mathematical Italic Capital Epsilon \\
\hline
U+1D6E7 & 𝛧 & {\textbackslash}itZeta & Mathematical Italic Capital Zeta \\
\hline
U+1D6E8 & 𝛨 & {\textbackslash}itEta & Mathematical Italic Capital Eta \\
\hline
U+1D6E9 & 𝛩 & {\textbackslash}itTheta & Mathematical Italic Capital Theta \\
\hline
U+1D6EA & 𝛪 & {\textbackslash}itIota & Mathematical Italic Capital Iota \\
\hline
U+1D6EB & 𝛫 & {\textbackslash}itKappa & Mathematical Italic Capital Kappa \\
\hline
U+1D6EC & 𝛬 & {\textbackslash}itLambda & Mathematical Italic Capital Lamda \\
\hline
U+1D6ED & 𝛭 & {\textbackslash}itMu & Mathematical Italic Capital Mu \\
\hline
U+1D6EE & 𝛮 & {\textbackslash}itNu & Mathematical Italic Capital Nu \\
\hline
U+1D6EF & 𝛯 & {\textbackslash}itXi & Mathematical Italic Capital Xi \\
\hline
U+1D6F0 & 𝛰 & {\textbackslash}itOmicron & Mathematical Italic Capital Omicron \\
\hline
U+1D6F1 & 𝛱 & {\textbackslash}itPi & Mathematical Italic Capital Pi \\
\hline
U+1D6F2 & 𝛲 & {\textbackslash}itRho & Mathematical Italic Capital Rho \\
\hline
U+1D6F3 & 𝛳 & {\textbackslash}itvarTheta & Mathematical Italic Capital Theta Symbol \\
\hline
U+1D6F4 & 𝛴 & {\textbackslash}itSigma & Mathematical Italic Capital Sigma \\
\hline
U+1D6F5 & 𝛵 & {\textbackslash}itTau & Mathematical Italic Capital Tau \\
\hline
U+1D6F6 & 𝛶 & {\textbackslash}itUpsilon & Mathematical Italic Capital Upsilon \\
\hline
U+1D6F7 & 𝛷 & {\textbackslash}itPhi & Mathematical Italic Capital Phi \\
\hline
U+1D6F8 & 𝛸 & {\textbackslash}itChi & Mathematical Italic Capital Chi \\
\hline
U+1D6F9 & 𝛹 & {\textbackslash}itPsi & Mathematical Italic Capital Psi \\
\hline
U+1D6FA & 𝛺 & {\textbackslash}itOmega & Mathematical Italic Capital Omega \\
\hline
U+1D6FB & 𝛻 & {\textbackslash}itnabla & Mathematical Italic Nabla \\
\hline
U+1D6FC & 𝛼 & {\textbackslash}italpha & Mathematical Italic Small Alpha \\
\hline
U+1D6FD & 𝛽 & {\textbackslash}itbeta & Mathematical Italic Small Beta \\
\hline
U+1D6FE & 𝛾 & {\textbackslash}itgamma & Mathematical Italic Small Gamma \\
\hline
U+1D6FF & 𝛿 & {\textbackslash}itdelta & Mathematical Italic Small Delta \\
\hline
U+1D700 & 𝜀 & {\textbackslash}itepsilon & Mathematical Italic Small Epsilon \\
\hline
U+1D701 & 𝜁 & {\textbackslash}itzeta & Mathematical Italic Small Zeta \\
\hline
U+1D702 & 𝜂 & {\textbackslash}iteta & Mathematical Italic Small Eta \\
\hline
U+1D703 & 𝜃 & {\textbackslash}ittheta & Mathematical Italic Small Theta \\
\hline
U+1D704 & 𝜄 & {\textbackslash}itiota & Mathematical Italic Small Iota \\
\hline
U+1D705 & 𝜅 & {\textbackslash}itkappa & Mathematical Italic Small Kappa \\
\hline
U+1D706 & 𝜆 & {\textbackslash}itlambda & Mathematical Italic Small Lamda \\
\hline
U+1D707 & 𝜇 & {\textbackslash}itmu & Mathematical Italic Small Mu \\
\hline
U+1D708 & 𝜈 & {\textbackslash}itnu & Mathematical Italic Small Nu \\
\hline
U+1D709 & 𝜉 & {\textbackslash}itxi & Mathematical Italic Small Xi \\
\hline
U+1D70A & 𝜊 & {\textbackslash}itomicron & Mathematical Italic Small Omicron \\
\hline
U+1D70B & 𝜋 & {\textbackslash}itpi & Mathematical Italic Small Pi \\
\hline
U+1D70C & 𝜌 & {\textbackslash}itrho & Mathematical Italic Small Rho \\
\hline
U+1D70D & 𝜍 & {\textbackslash}itvarsigma & Mathematical Italic Small Final Sigma \\
\hline
U+1D70E & 𝜎 & {\textbackslash}itsigma & Mathematical Italic Small Sigma \\
\hline
U+1D70F & 𝜏 & {\textbackslash}ittau & Mathematical Italic Small Tau \\
\hline
U+1D710 & 𝜐 & {\textbackslash}itupsilon & Mathematical Italic Small Upsilon \\
\hline
U+1D711 & 𝜑 & {\textbackslash}itphi & Mathematical Italic Small Phi \\
\hline
U+1D712 & 𝜒 & {\textbackslash}itchi & Mathematical Italic Small Chi \\
\hline
U+1D713 & 𝜓 & {\textbackslash}itpsi & Mathematical Italic Small Psi \\
\hline
U+1D714 & 𝜔 & {\textbackslash}itomega & Mathematical Italic Small Omega \\
\hline
U+1D715 & 𝜕 & {\textbackslash}itpartial & Mathematical Italic Partial Differential \\
\hline
U+1D716 & 𝜖 & {\textbackslash}itvarepsilon & Mathematical Italic Epsilon Symbol \\
\hline
U+1D717 & 𝜗 & {\textbackslash}itvartheta & Mathematical Italic Theta Symbol \\
\hline
U+1D718 & 𝜘 & {\textbackslash}itvarkappa & Mathematical Italic Kappa Symbol \\
\hline
U+1D719 & 𝜙 & {\textbackslash}itvarphi & Mathematical Italic Phi Symbol \\
\hline
U+1D71A & 𝜚 & {\textbackslash}itvarrho & Mathematical Italic Rho Symbol \\
\hline
U+1D71B & 𝜛 & {\textbackslash}itvarpi & Mathematical Italic Pi Symbol \\
\hline
U+1D71C & 𝜜 & {\textbackslash}biAlpha & Mathematical Bold Italic Capital Alpha \\
\hline
U+1D71D & 𝜝 & {\textbackslash}biBeta & Mathematical Bold Italic Capital Beta \\
\hline
U+1D71E & 𝜞 & {\textbackslash}biGamma & Mathematical Bold Italic Capital Gamma \\
\hline
U+1D71F & 𝜟 & {\textbackslash}biDelta & Mathematical Bold Italic Capital Delta \\
\hline
U+1D720 & 𝜠 & {\textbackslash}biEpsilon & Mathematical Bold Italic Capital Epsilon \\
\hline
U+1D721 & 𝜡 & {\textbackslash}biZeta & Mathematical Bold Italic Capital Zeta \\
\hline
U+1D722 & 𝜢 & {\textbackslash}biEta & Mathematical Bold Italic Capital Eta \\
\hline
U+1D723 & 𝜣 & {\textbackslash}biTheta & Mathematical Bold Italic Capital Theta \\
\hline
U+1D724 & 𝜤 & {\textbackslash}biIota & Mathematical Bold Italic Capital Iota \\
\hline
U+1D725 & 𝜥 & {\textbackslash}biKappa & Mathematical Bold Italic Capital Kappa \\
\hline
U+1D726 & 𝜦 & {\textbackslash}biLambda & Mathematical Bold Italic Capital Lamda \\
\hline
U+1D727 & 𝜧 & {\textbackslash}biMu & Mathematical Bold Italic Capital Mu \\
\hline
U+1D728 & 𝜨 & {\textbackslash}biNu & Mathematical Bold Italic Capital Nu \\
\hline
U+1D729 & 𝜩 & {\textbackslash}biXi & Mathematical Bold Italic Capital Xi \\
\hline
U+1D72A & 𝜪 & {\textbackslash}biOmicron & Mathematical Bold Italic Capital Omicron \\
\hline
U+1D72B & 𝜫 & {\textbackslash}biPi & Mathematical Bold Italic Capital Pi \\
\hline
U+1D72C & 𝜬 & {\textbackslash}biRho & Mathematical Bold Italic Capital Rho \\
\hline
U+1D72D & 𝜭 & {\textbackslash}bivarTheta & Mathematical Bold Italic Capital Theta Symbol \\
\hline
U+1D72E & 𝜮 & {\textbackslash}biSigma & Mathematical Bold Italic Capital Sigma \\
\hline
U+1D72F & 𝜯 & {\textbackslash}biTau & Mathematical Bold Italic Capital Tau \\
\hline
U+1D730 & 𝜰 & {\textbackslash}biUpsilon & Mathematical Bold Italic Capital Upsilon \\
\hline
U+1D731 & 𝜱 & {\textbackslash}biPhi & Mathematical Bold Italic Capital Phi \\
\hline
U+1D732 & 𝜲 & {\textbackslash}biChi & Mathematical Bold Italic Capital Chi \\
\hline
U+1D733 & 𝜳 & {\textbackslash}biPsi & Mathematical Bold Italic Capital Psi \\
\hline
U+1D734 & 𝜴 & {\textbackslash}biOmega & Mathematical Bold Italic Capital Omega \\
\hline
U+1D735 & 𝜵 & {\textbackslash}binabla & Mathematical Bold Italic Nabla \\
\hline
U+1D736 & 𝜶 & {\textbackslash}bialpha & Mathematical Bold Italic Small Alpha \\
\hline
U+1D737 & 𝜷 & {\textbackslash}bibeta & Mathematical Bold Italic Small Beta \\
\hline
U+1D738 & 𝜸 & {\textbackslash}bigamma & Mathematical Bold Italic Small Gamma \\
\hline
U+1D739 & 𝜹 & {\textbackslash}bidelta & Mathematical Bold Italic Small Delta \\
\hline
U+1D73A & 𝜺 & {\textbackslash}biepsilon & Mathematical Bold Italic Small Epsilon \\
\hline
U+1D73B & 𝜻 & {\textbackslash}bizeta & Mathematical Bold Italic Small Zeta \\
\hline
U+1D73C & 𝜼 & {\textbackslash}bieta & Mathematical Bold Italic Small Eta \\
\hline
U+1D73D & 𝜽 & {\textbackslash}bitheta & Mathematical Bold Italic Small Theta \\
\hline
U+1D73E & 𝜾 & {\textbackslash}biiota & Mathematical Bold Italic Small Iota \\
\hline
U+1D73F & 𝜿 & {\textbackslash}bikappa & Mathematical Bold Italic Small Kappa \\
\hline
U+1D740 & 𝝀 & {\textbackslash}bilambda & Mathematical Bold Italic Small Lamda \\
\hline
U+1D741 & 𝝁 & {\textbackslash}bimu & Mathematical Bold Italic Small Mu \\
\hline
U+1D742 & 𝝂 & {\textbackslash}binu & Mathematical Bold Italic Small Nu \\
\hline
U+1D743 & 𝝃 & {\textbackslash}bixi & Mathematical Bold Italic Small Xi \\
\hline
U+1D744 & 𝝄 & {\textbackslash}biomicron & Mathematical Bold Italic Small Omicron \\
\hline
U+1D745 & 𝝅 & {\textbackslash}bipi & Mathematical Bold Italic Small Pi \\
\hline
U+1D746 & 𝝆 & {\textbackslash}birho & Mathematical Bold Italic Small Rho \\
\hline
U+1D747 & 𝝇 & {\textbackslash}bivarsigma & Mathematical Bold Italic Small Final Sigma \\
\hline
U+1D748 & 𝝈 & {\textbackslash}bisigma & Mathematical Bold Italic Small Sigma \\
\hline
U+1D749 & 𝝉 & {\textbackslash}bitau & Mathematical Bold Italic Small Tau \\
\hline
U+1D74A & 𝝊 & {\textbackslash}biupsilon & Mathematical Bold Italic Small Upsilon \\
\hline
U+1D74B & 𝝋 & {\textbackslash}biphi & Mathematical Bold Italic Small Phi \\
\hline
U+1D74C & 𝝌 & {\textbackslash}bichi & Mathematical Bold Italic Small Chi \\
\hline
U+1D74D & 𝝍 & {\textbackslash}bipsi & Mathematical Bold Italic Small Psi \\
\hline
U+1D74E & 𝝎 & {\textbackslash}biomega & Mathematical Bold Italic Small Omega \\
\hline
U+1D74F & 𝝏 & {\textbackslash}bipartial & Mathematical Bold Italic Partial Differential \\
\hline
U+1D750 & 𝝐 & {\textbackslash}bivarepsilon & Mathematical Bold Italic Epsilon Symbol \\
\hline
U+1D751 & 𝝑 & {\textbackslash}bivartheta & Mathematical Bold Italic Theta Symbol \\
\hline
U+1D752 & 𝝒 & {\textbackslash}bivarkappa & Mathematical Bold Italic Kappa Symbol \\
\hline
U+1D753 & 𝝓 & {\textbackslash}bivarphi & Mathematical Bold Italic Phi Symbol \\
\hline
U+1D754 & 𝝔 & {\textbackslash}bivarrho & Mathematical Bold Italic Rho Symbol \\
\hline
U+1D755 & 𝝕 & {\textbackslash}bivarpi & Mathematical Bold Italic Pi Symbol \\
\hline
U+1D756 & 𝝖 & {\textbackslash}bsansAlpha & Mathematical Sans-Serif Bold Capital Alpha \\
\hline
U+1D757 & 𝝗 & {\textbackslash}bsansBeta & Mathematical Sans-Serif Bold Capital Beta \\
\hline
U+1D758 & 𝝘 & {\textbackslash}bsansGamma & Mathematical Sans-Serif Bold Capital Gamma \\
\hline
U+1D759 & 𝝙 & {\textbackslash}bsansDelta & Mathematical Sans-Serif Bold Capital Delta \\
\hline
U+1D75A & 𝝚 & {\textbackslash}bsansEpsilon & Mathematical Sans-Serif Bold Capital Epsilon \\
\hline
U+1D75B & 𝝛 & {\textbackslash}bsansZeta & Mathematical Sans-Serif Bold Capital Zeta \\
\hline
U+1D75C & 𝝜 & {\textbackslash}bsansEta & Mathematical Sans-Serif Bold Capital Eta \\
\hline
U+1D75D & 𝝝 & {\textbackslash}bsansTheta & Mathematical Sans-Serif Bold Capital Theta \\
\hline
U+1D75E & 𝝞 & {\textbackslash}bsansIota & Mathematical Sans-Serif Bold Capital Iota \\
\hline
U+1D75F & 𝝟 & {\textbackslash}bsansKappa & Mathematical Sans-Serif Bold Capital Kappa \\
\hline
U+1D760 & 𝝠 & {\textbackslash}bsansLambda & Mathematical Sans-Serif Bold Capital Lamda \\
\hline
U+1D761 & 𝝡 & {\textbackslash}bsansMu & Mathematical Sans-Serif Bold Capital Mu \\
\hline
U+1D762 & 𝝢 & {\textbackslash}bsansNu & Mathematical Sans-Serif Bold Capital Nu \\
\hline
U+1D763 & 𝝣 & {\textbackslash}bsansXi & Mathematical Sans-Serif Bold Capital Xi \\
\hline
U+1D764 & 𝝤 & {\textbackslash}bsansOmicron & Mathematical Sans-Serif Bold Capital Omicron \\
\hline
U+1D765 & 𝝥 & {\textbackslash}bsansPi & Mathematical Sans-Serif Bold Capital Pi \\
\hline
U+1D766 & 𝝦 & {\textbackslash}bsansRho & Mathematical Sans-Serif Bold Capital Rho \\
\hline
U+1D767 & 𝝧 & {\textbackslash}bsansvarTheta & Mathematical Sans-Serif Bold Capital Theta Symbol \\
\hline
U+1D768 & 𝝨 & {\textbackslash}bsansSigma & Mathematical Sans-Serif Bold Capital Sigma \\
\hline
U+1D769 & 𝝩 & {\textbackslash}bsansTau & Mathematical Sans-Serif Bold Capital Tau \\
\hline
U+1D76A & 𝝪 & {\textbackslash}bsansUpsilon & Mathematical Sans-Serif Bold Capital Upsilon \\
\hline
U+1D76B & 𝝫 & {\textbackslash}bsansPhi & Mathematical Sans-Serif Bold Capital Phi \\
\hline
U+1D76C & 𝝬 & {\textbackslash}bsansChi & Mathematical Sans-Serif Bold Capital Chi \\
\hline
U+1D76D & 𝝭 & {\textbackslash}bsansPsi & Mathematical Sans-Serif Bold Capital Psi \\
\hline
U+1D76E & 𝝮 & {\textbackslash}bsansOmega & Mathematical Sans-Serif Bold Capital Omega \\
\hline
U+1D76F & 𝝯 & {\textbackslash}bsansnabla & Mathematical Sans-Serif Bold Nabla \\
\hline
U+1D770 & 𝝰 & {\textbackslash}bsansalpha & Mathematical Sans-Serif Bold Small Alpha \\
\hline
U+1D771 & 𝝱 & {\textbackslash}bsansbeta & Mathematical Sans-Serif Bold Small Beta \\
\hline
U+1D772 & 𝝲 & {\textbackslash}bsansgamma & Mathematical Sans-Serif Bold Small Gamma \\
\hline
U+1D773 & 𝝳 & {\textbackslash}bsansdelta & Mathematical Sans-Serif Bold Small Delta \\
\hline
U+1D774 & 𝝴 & {\textbackslash}bsansepsilon & Mathematical Sans-Serif Bold Small Epsilon \\
\hline
U+1D775 & 𝝵 & {\textbackslash}bsanszeta & Mathematical Sans-Serif Bold Small Zeta \\
\hline
U+1D776 & 𝝶 & {\textbackslash}bsanseta & Mathematical Sans-Serif Bold Small Eta \\
\hline
U+1D777 & 𝝷 & {\textbackslash}bsanstheta & Mathematical Sans-Serif Bold Small Theta \\
\hline
U+1D778 & 𝝸 & {\textbackslash}bsansiota & Mathematical Sans-Serif Bold Small Iota \\
\hline
U+1D779 & 𝝹 & {\textbackslash}bsanskappa & Mathematical Sans-Serif Bold Small Kappa \\
\hline
U+1D77A & 𝝺 & {\textbackslash}bsanslambda & Mathematical Sans-Serif Bold Small Lamda \\
\hline
U+1D77B & 𝝻 & {\textbackslash}bsansmu & Mathematical Sans-Serif Bold Small Mu \\
\hline
U+1D77C & 𝝼 & {\textbackslash}bsansnu & Mathematical Sans-Serif Bold Small Nu \\
\hline
U+1D77D & 𝝽 & {\textbackslash}bsansxi & Mathematical Sans-Serif Bold Small Xi \\
\hline
U+1D77E & 𝝾 & {\textbackslash}bsansomicron & Mathematical Sans-Serif Bold Small Omicron \\
\hline
U+1D77F & 𝝿 & {\textbackslash}bsanspi & Mathematical Sans-Serif Bold Small Pi \\
\hline
U+1D780 & 𝞀 & {\textbackslash}bsansrho & Mathematical Sans-Serif Bold Small Rho \\
\hline
U+1D781 & 𝞁 & {\textbackslash}bsansvarsigma & Mathematical Sans-Serif Bold Small Final Sigma \\
\hline
U+1D782 & 𝞂 & {\textbackslash}bsanssigma & Mathematical Sans-Serif Bold Small Sigma \\
\hline
U+1D783 & 𝞃 & {\textbackslash}bsanstau & Mathematical Sans-Serif Bold Small Tau \\
\hline
U+1D784 & 𝞄 & {\textbackslash}bsansupsilon & Mathematical Sans-Serif Bold Small Upsilon \\
\hline
U+1D785 & 𝞅 & {\textbackslash}bsansphi & Mathematical Sans-Serif Bold Small Phi \\
\hline
U+1D786 & 𝞆 & {\textbackslash}bsanschi & Mathematical Sans-Serif Bold Small Chi \\
\hline
U+1D787 & 𝞇 & {\textbackslash}bsanspsi & Mathematical Sans-Serif Bold Small Psi \\
\hline
U+1D788 & 𝞈 & {\textbackslash}bsansomega & Mathematical Sans-Serif Bold Small Omega \\
\hline
U+1D789 & 𝞉 & {\textbackslash}bsanspartial & Mathematical Sans-Serif Bold Partial Differential \\
\hline
U+1D78A & 𝞊 & {\textbackslash}bsansvarepsilon & Mathematical Sans-Serif Bold Epsilon Symbol \\
\hline
U+1D78B & 𝞋 & {\textbackslash}bsansvartheta & Mathematical Sans-Serif Bold Theta Symbol \\
\hline
U+1D78C & 𝞌 & {\textbackslash}bsansvarkappa & Mathematical Sans-Serif Bold Kappa Symbol \\
\hline
U+1D78D & 𝞍 & {\textbackslash}bsansvarphi & Mathematical Sans-Serif Bold Phi Symbol \\
\hline
U+1D78E & 𝞎 & {\textbackslash}bsansvarrho & Mathematical Sans-Serif Bold Rho Symbol \\
\hline
U+1D78F & 𝞏 & {\textbackslash}bsansvarpi & Mathematical Sans-Serif Bold Pi Symbol \\
\hline
U+1D790 & 𝞐 & {\textbackslash}bisansAlpha & Mathematical Sans-Serif Bold Italic Capital Alpha \\
\hline
U+1D791 & 𝞑 & {\textbackslash}bisansBeta & Mathematical Sans-Serif Bold Italic Capital Beta \\
\hline
U+1D792 & 𝞒 & {\textbackslash}bisansGamma & Mathematical Sans-Serif Bold Italic Capital Gamma \\
\hline
U+1D793 & 𝞓 & {\textbackslash}bisansDelta & Mathematical Sans-Serif Bold Italic Capital Delta \\
\hline
U+1D794 & 𝞔 & {\textbackslash}bisansEpsilon & Mathematical Sans-Serif Bold Italic Capital Epsilon \\
\hline
U+1D795 & 𝞕 & {\textbackslash}bisansZeta & Mathematical Sans-Serif Bold Italic Capital Zeta \\
\hline
U+1D796 & 𝞖 & {\textbackslash}bisansEta & Mathematical Sans-Serif Bold Italic Capital Eta \\
\hline
U+1D797 & 𝞗 & {\textbackslash}bisansTheta & Mathematical Sans-Serif Bold Italic Capital Theta \\
\hline
U+1D798 & 𝞘 & {\textbackslash}bisansIota & Mathematical Sans-Serif Bold Italic Capital Iota \\
\hline
U+1D799 & 𝞙 & {\textbackslash}bisansKappa & Mathematical Sans-Serif Bold Italic Capital Kappa \\
\hline
U+1D79A & 𝞚 & {\textbackslash}bisansLambda & Mathematical Sans-Serif Bold Italic Capital Lamda \\
\hline
U+1D79B & 𝞛 & {\textbackslash}bisansMu & Mathematical Sans-Serif Bold Italic Capital Mu \\
\hline
U+1D79C & 𝞜 & {\textbackslash}bisansNu & Mathematical Sans-Serif Bold Italic Capital Nu \\
\hline
U+1D79D & 𝞝 & {\textbackslash}bisansXi & Mathematical Sans-Serif Bold Italic Capital Xi \\
\hline
U+1D79E & 𝞞 & {\textbackslash}bisansOmicron & Mathematical Sans-Serif Bold Italic Capital Omicron \\
\hline
U+1D79F & 𝞟 & {\textbackslash}bisansPi & Mathematical Sans-Serif Bold Italic Capital Pi \\
\hline
U+1D7A0 & 𝞠 & {\textbackslash}bisansRho & Mathematical Sans-Serif Bold Italic Capital Rho \\
\hline
U+1D7A1 & 𝞡 & {\textbackslash}bisansvarTheta & Mathematical Sans-Serif Bold Italic Capital Theta Symbol \\
\hline
U+1D7A2 & 𝞢 & {\textbackslash}bisansSigma & Mathematical Sans-Serif Bold Italic Capital Sigma \\
\hline
U+1D7A3 & 𝞣 & {\textbackslash}bisansTau & Mathematical Sans-Serif Bold Italic Capital Tau \\
\hline
U+1D7A4 & 𝞤 & {\textbackslash}bisansUpsilon & Mathematical Sans-Serif Bold Italic Capital Upsilon \\
\hline
U+1D7A5 & 𝞥 & {\textbackslash}bisansPhi & Mathematical Sans-Serif Bold Italic Capital Phi \\
\hline
U+1D7A6 & 𝞦 & {\textbackslash}bisansChi & Mathematical Sans-Serif Bold Italic Capital Chi \\
\hline
U+1D7A7 & 𝞧 & {\textbackslash}bisansPsi & Mathematical Sans-Serif Bold Italic Capital Psi \\
\hline
U+1D7A8 & 𝞨 & {\textbackslash}bisansOmega & Mathematical Sans-Serif Bold Italic Capital Omega \\
\hline
U+1D7A9 & 𝞩 & {\textbackslash}bisansnabla & Mathematical Sans-Serif Bold Italic Nabla \\
\hline
U+1D7AA & 𝞪 & {\textbackslash}bisansalpha & Mathematical Sans-Serif Bold Italic Small Alpha \\
\hline
U+1D7AB & 𝞫 & {\textbackslash}bisansbeta & Mathematical Sans-Serif Bold Italic Small Beta \\
\hline
U+1D7AC & 𝞬 & {\textbackslash}bisansgamma & Mathematical Sans-Serif Bold Italic Small Gamma \\
\hline
U+1D7AD & 𝞭 & {\textbackslash}bisansdelta & Mathematical Sans-Serif Bold Italic Small Delta \\
\hline
U+1D7AE & 𝞮 & {\textbackslash}bisansepsilon & Mathematical Sans-Serif Bold Italic Small Epsilon \\
\hline
U+1D7AF & 𝞯 & {\textbackslash}bisanszeta & Mathematical Sans-Serif Bold Italic Small Zeta \\
\hline
U+1D7B0 & 𝞰 & {\textbackslash}bisanseta & Mathematical Sans-Serif Bold Italic Small Eta \\
\hline
U+1D7B1 & 𝞱 & {\textbackslash}bisanstheta & Mathematical Sans-Serif Bold Italic Small Theta \\
\hline
U+1D7B2 & 𝞲 & {\textbackslash}bisansiota & Mathematical Sans-Serif Bold Italic Small Iota \\
\hline
U+1D7B3 & 𝞳 & {\textbackslash}bisanskappa & Mathematical Sans-Serif Bold Italic Small Kappa \\
\hline
U+1D7B4 & 𝞴 & {\textbackslash}bisanslambda & Mathematical Sans-Serif Bold Italic Small Lamda \\
\hline
U+1D7B5 & 𝞵 & {\textbackslash}bisansmu & Mathematical Sans-Serif Bold Italic Small Mu \\
\hline
U+1D7B6 & 𝞶 & {\textbackslash}bisansnu & Mathematical Sans-Serif Bold Italic Small Nu \\
\hline
U+1D7B7 & 𝞷 & {\textbackslash}bisansxi & Mathematical Sans-Serif Bold Italic Small Xi \\
\hline
U+1D7B8 & 𝞸 & {\textbackslash}bisansomicron & Mathematical Sans-Serif Bold Italic Small Omicron \\
\hline
U+1D7B9 & 𝞹 & {\textbackslash}bisanspi & Mathematical Sans-Serif Bold Italic Small Pi \\
\hline
U+1D7BA & 𝞺 & {\textbackslash}bisansrho & Mathematical Sans-Serif Bold Italic Small Rho \\
\hline
U+1D7BB & 𝞻 & {\textbackslash}bisansvarsigma & Mathematical Sans-Serif Bold Italic Small Final Sigma \\
\hline
U+1D7BC & 𝞼 & {\textbackslash}bisanssigma & Mathematical Sans-Serif Bold Italic Small Sigma \\
\hline
U+1D7BD & 𝞽 & {\textbackslash}bisanstau & Mathematical Sans-Serif Bold Italic Small Tau \\
\hline
U+1D7BE & 𝞾 & {\textbackslash}bisansupsilon & Mathematical Sans-Serif Bold Italic Small Upsilon \\
\hline
U+1D7BF & 𝞿 & {\textbackslash}bisansphi & Mathematical Sans-Serif Bold Italic Small Phi \\
\hline
U+1D7C0 & 𝟀 & {\textbackslash}bisanschi & Mathematical Sans-Serif Bold Italic Small Chi \\
\hline
U+1D7C1 & 𝟁 & {\textbackslash}bisanspsi & Mathematical Sans-Serif Bold Italic Small Psi \\
\hline
U+1D7C2 & 𝟂 & {\textbackslash}bisansomega & Mathematical Sans-Serif Bold Italic Small Omega \\
\hline
U+1D7C3 & 𝟃 & {\textbackslash}bisanspartial & Mathematical Sans-Serif Bold Italic Partial Differential \\
\hline
U+1D7C4 & 𝟄 & {\textbackslash}bisansvarepsilon & Mathematical Sans-Serif Bold Italic Epsilon Symbol \\
\hline
U+1D7C5 & 𝟅 & {\textbackslash}bisansvartheta & Mathematical Sans-Serif Bold Italic Theta Symbol \\
\hline
U+1D7C6 & 𝟆 & {\textbackslash}bisansvarkappa & Mathematical Sans-Serif Bold Italic Kappa Symbol \\
\hline
U+1D7C7 & 𝟇 & {\textbackslash}bisansvarphi & Mathematical Sans-Serif Bold Italic Phi Symbol \\
\hline
U+1D7C8 & 𝟈 & {\textbackslash}bisansvarrho & Mathematical Sans-Serif Bold Italic Rho Symbol \\
\hline
U+1D7C9 & 𝟉 & {\textbackslash}bisansvarpi & Mathematical Sans-Serif Bold Italic Pi Symbol \\
\hline
U+1D7CA & 𝟊 & {\textbackslash}bfDigamma & Mathematical Bold Capital Digamma \\
\hline
U+1D7CB & 𝟋 & {\textbackslash}bfdigamma & Mathematical Bold Small Digamma \\
\hline
U+1D7CE & 𝟎 & {\textbackslash}bfzero & Mathematical Bold Digit Zero \\
\hline
U+1D7CF & 𝟏 & {\textbackslash}bfone & Mathematical Bold Digit One \\
\hline
U+1D7D0 & 𝟐 & {\textbackslash}bftwo & Mathematical Bold Digit Two \\
\hline
U+1D7D1 & 𝟑 & {\textbackslash}bfthree & Mathematical Bold Digit Three \\
\hline
U+1D7D2 & 𝟒 & {\textbackslash}bffour & Mathematical Bold Digit Four \\
\hline
U+1D7D3 & 𝟓 & {\textbackslash}bffive & Mathematical Bold Digit Five \\
\hline
U+1D7D4 & 𝟔 & {\textbackslash}bfsix & Mathematical Bold Digit Six \\
\hline
U+1D7D5 & 𝟕 & {\textbackslash}bfseven & Mathematical Bold Digit Seven \\
\hline
U+1D7D6 & 𝟖 & {\textbackslash}bfeight & Mathematical Bold Digit Eight \\
\hline
U+1D7D7 & 𝟗 & {\textbackslash}bfnine & Mathematical Bold Digit Nine \\
\hline
U+1D7D8 & 𝟘 & {\textbackslash}bbzero & Mathematical Double-Struck Digit Zero \\
\hline
U+1D7D9 & 𝟙 & {\textbackslash}bbone & Mathematical Double-Struck Digit One \\
\hline
U+1D7DA & 𝟚 & {\textbackslash}bbtwo & Mathematical Double-Struck Digit Two \\
\hline
U+1D7DB & 𝟛 & {\textbackslash}bbthree & Mathematical Double-Struck Digit Three \\
\hline
U+1D7DC & 𝟜 & {\textbackslash}bbfour & Mathematical Double-Struck Digit Four \\
\hline
U+1D7DD & 𝟝 & {\textbackslash}bbfive & Mathematical Double-Struck Digit Five \\
\hline
U+1D7DE & 𝟞 & {\textbackslash}bbsix & Mathematical Double-Struck Digit Six \\
\hline
U+1D7DF & 𝟟 & {\textbackslash}bbseven & Mathematical Double-Struck Digit Seven \\
\hline
U+1D7E0 & 𝟠 & {\textbackslash}bbeight & Mathematical Double-Struck Digit Eight \\
\hline
U+1D7E1 & 𝟡 & {\textbackslash}bbnine & Mathematical Double-Struck Digit Nine \\
\hline
U+1D7E2 & 𝟢 & {\textbackslash}sanszero & Mathematical Sans-Serif Digit Zero \\
\hline
U+1D7E3 & 𝟣 & {\textbackslash}sansone & Mathematical Sans-Serif Digit One \\
\hline
U+1D7E4 & 𝟤 & {\textbackslash}sanstwo & Mathematical Sans-Serif Digit Two \\
\hline
U+1D7E5 & 𝟥 & {\textbackslash}sansthree & Mathematical Sans-Serif Digit Three \\
\hline
U+1D7E6 & 𝟦 & {\textbackslash}sansfour & Mathematical Sans-Serif Digit Four \\
\hline
U+1D7E7 & 𝟧 & {\textbackslash}sansfive & Mathematical Sans-Serif Digit Five \\
\hline
U+1D7E8 & 𝟨 & {\textbackslash}sanssix & Mathematical Sans-Serif Digit Six \\
\hline
U+1D7E9 & 𝟩 & {\textbackslash}sansseven & Mathematical Sans-Serif Digit Seven \\
\hline
U+1D7EA & 𝟪 & {\textbackslash}sanseight & Mathematical Sans-Serif Digit Eight \\
\hline
U+1D7EB & 𝟫 & {\textbackslash}sansnine & Mathematical Sans-Serif Digit Nine \\
\hline
U+1D7EC & 𝟬 & {\textbackslash}bsanszero & Mathematical Sans-Serif Bold Digit Zero \\
\hline
U+1D7ED & 𝟭 & {\textbackslash}bsansone & Mathematical Sans-Serif Bold Digit One \\
\hline
U+1D7EE & 𝟮 & {\textbackslash}bsanstwo & Mathematical Sans-Serif Bold Digit Two \\
\hline
U+1D7EF & 𝟯 & {\textbackslash}bsansthree & Mathematical Sans-Serif Bold Digit Three \\
\hline
U+1D7F0 & 𝟰 & {\textbackslash}bsansfour & Mathematical Sans-Serif Bold Digit Four \\
\hline
U+1D7F1 & 𝟱 & {\textbackslash}bsansfive & Mathematical Sans-Serif Bold Digit Five \\
\hline
U+1D7F2 & 𝟲 & {\textbackslash}bsanssix & Mathematical Sans-Serif Bold Digit Six \\
\hline
U+1D7F3 & 𝟳 & {\textbackslash}bsansseven & Mathematical Sans-Serif Bold Digit Seven \\
\hline
U+1D7F4 & 𝟴 & {\textbackslash}bsanseight & Mathematical Sans-Serif Bold Digit Eight \\
\hline
U+1D7F5 & 𝟵 & {\textbackslash}bsansnine & Mathematical Sans-Serif Bold Digit Nine \\
\hline
U+1D7F6 & 𝟶 & {\textbackslash}ttzero & Mathematical Monospace Digit Zero \\
\hline
U+1D7F7 & 𝟷 & {\textbackslash}ttone & Mathematical Monospace Digit One \\
\hline
U+1D7F8 & 𝟸 & {\textbackslash}tttwo & Mathematical Monospace Digit Two \\
\hline
U+1D7F9 & 𝟹 & {\textbackslash}ttthree & Mathematical Monospace Digit Three \\
\hline
U+1D7FA & 𝟺 & {\textbackslash}ttfour & Mathematical Monospace Digit Four \\
\hline
U+1D7FB & 𝟻 & {\textbackslash}ttfive & Mathematical Monospace Digit Five \\
\hline
U+1D7FC & 𝟼 & {\textbackslash}ttsix & Mathematical Monospace Digit Six \\
\hline
U+1D7FD & 𝟽 & {\textbackslash}ttseven & Mathematical Monospace Digit Seven \\
\hline
U+1D7FE & 𝟾 & {\textbackslash}tteight & Mathematical Monospace Digit Eight \\
\hline
U+1D7FF & 𝟿 & {\textbackslash}ttnine & Mathematical Monospace Digit Nine \\
\hline
U+1F004 & 🀄 & {\textbackslash}:mahjong: & Mahjong Tile Red Dragon \\
\hline
U+1F0CF & 🃏 & {\textbackslash}:black\_joker: & Playing Card Black Joker \\
\hline
U+1F170 & 🅰 & {\textbackslash}:a: & Negative Squared Latin Capital Letter A \\
\hline
U+1F171 & 🅱 & {\textbackslash}:b: & Negative Squared Latin Capital Letter B \\
\hline
U+1F17E & 🅾 & {\textbackslash}:o2: & Negative Squared Latin Capital Letter O \\
\hline
U+1F17F & 🅿 & {\textbackslash}:parking: & Negative Squared Latin Capital Letter P \\
\hline
U+1F18E & 🆎 & {\textbackslash}:ab: & Negative Squared Ab \\
\hline
U+1F191 & 🆑 & {\textbackslash}:cl: & Squared Cl \\
\hline
U+1F192 & 🆒 & {\textbackslash}:cool: & Squared Cool \\
\hline
U+1F193 & 🆓 & {\textbackslash}:free: & Squared Free \\
\hline
U+1F194 & 🆔 & {\textbackslash}:id: & Squared Id \\
\hline
U+1F195 & 🆕 & {\textbackslash}:new: & Squared New \\
\hline
U+1F196 & 🆖 & {\textbackslash}:ng: & Squared Ng \\
\hline
U+1F197 & 🆗 & {\textbackslash}:ok: & Squared Ok \\
\hline
U+1F198 & 🆘 & {\textbackslash}:sos: & Squared Sos \\
\hline
U+1F199 & 🆙 & {\textbackslash}:up: & Squared Up With Exclamation Mark \\
\hline
U+1F19A & 🆚 & {\textbackslash}:vs: & Squared Vs \\
\hline
U+1F201 & 🈁 & {\textbackslash}:koko: & Squared Katakana Koko \\
\hline
U+1F202 & 🈂 & {\textbackslash}:sa: & Squared Katakana Sa \\
\hline
U+1F21A & 🈚 & {\textbackslash}:u7121: & Squared Cjk Unified Ideograph-7121 \\
\hline
U+1F22F & 🈯 & {\textbackslash}:u6307: & Squared Cjk Unified Ideograph-6307 \\
\hline
U+1F232 & 🈲 & {\textbackslash}:u7981: & Squared Cjk Unified Ideograph-7981 \\
\hline
U+1F233 & 🈳 & {\textbackslash}:u7a7a: & Squared Cjk Unified Ideograph-7A7A \\
\hline
U+1F234 & 🈴 & {\textbackslash}:u5408: & Squared Cjk Unified Ideograph-5408 \\
\hline
U+1F235 & 🈵 & {\textbackslash}:u6e80: & Squared Cjk Unified Ideograph-6E80 \\
\hline
U+1F236 & 🈶 & {\textbackslash}:u6709: & Squared Cjk Unified Ideograph-6709 \\
\hline
U+1F237 & 🈷 & {\textbackslash}:u6708: & Squared Cjk Unified Ideograph-6708 \\
\hline
U+1F238 & 🈸 & {\textbackslash}:u7533: & Squared Cjk Unified Ideograph-7533 \\
\hline
U+1F239 & 🈹 & {\textbackslash}:u5272: & Squared Cjk Unified Ideograph-5272 \\
\hline
U+1F23A & 🈺 & {\textbackslash}:u55b6: & Squared Cjk Unified Ideograph-55B6 \\
\hline
U+1F250 & 🉐 & {\textbackslash}:ideograph\_advantage: & Circled Ideograph Advantage \\
\hline
U+1F251 & 🉑 & {\textbackslash}:accept: & Circled Ideograph Accept \\
\hline
U+1F300 & 🌀 & {\textbackslash}:cyclone: & Cyclone \\
\hline
U+1F301 & 🌁 & {\textbackslash}:foggy: & Foggy \\
\hline
U+1F302 & 🌂 & {\textbackslash}:closed\_umbrella: & Closed Umbrella \\
\hline
U+1F303 & 🌃 & {\textbackslash}:night\_with\_stars: & Night With Stars \\
\hline
U+1F304 & 🌄 & {\textbackslash}:sunrise\_over\_mountains: & Sunrise Over Mountains \\
\hline
U+1F305 & 🌅 & {\textbackslash}:sunrise: & Sunrise \\
\hline
U+1F306 & 🌆 & {\textbackslash}:city\_sunset: & Cityscape At Dusk \\
\hline
U+1F307 & 🌇 & {\textbackslash}:city\_sunrise: & Sunset Over Buildings \\
\hline
U+1F308 & 🌈 & {\textbackslash}:rainbow: & Rainbow \\
\hline
U+1F309 & 🌉 & {\textbackslash}:bridge\_at\_night: & Bridge At Night \\
\hline
U+1F30A & 🌊 & {\textbackslash}:ocean: & Water Wave \\
\hline
U+1F30B & 🌋 & {\textbackslash}:volcano: & Volcano \\
\hline
U+1F30C & 🌌 & {\textbackslash}:milky\_way: & Milky Way \\
\hline
U+1F30D & 🌍 & {\textbackslash}:earth\_africa: & Earth Globe Europe-Africa \\
\hline
U+1F30E & 🌎 & {\textbackslash}:earth\_americas: & Earth Globe Americas \\
\hline
U+1F30F & 🌏 & {\textbackslash}:earth\_asia: & Earth Globe Asia-Australia \\
\hline
U+1F310 & 🌐 & {\textbackslash}:globe\_with\_meridians: & Globe With Meridians \\
\hline
U+1F311 & 🌑 & {\textbackslash}:new\_moon: & New Moon Symbol \\
\hline
U+1F312 & 🌒 & {\textbackslash}:waxing\_crescent\_moon: & Waxing Crescent Moon Symbol \\
\hline
U+1F313 & 🌓 & {\textbackslash}:first\_quarter\_moon: & First Quarter Moon Symbol \\
\hline
U+1F314 & 🌔 & {\textbackslash}:moon: & Waxing Gibbous Moon Symbol \\
\hline
U+1F315 & 🌕 & {\textbackslash}:full\_moon: & Full Moon Symbol \\
\hline
U+1F316 & 🌖 & {\textbackslash}:waning\_gibbous\_moon: & Waning Gibbous Moon Symbol \\
\hline
U+1F317 & 🌗 & {\textbackslash}:last\_quarter\_moon: & Last Quarter Moon Symbol \\
\hline
U+1F318 & 🌘 & {\textbackslash}:waning\_crescent\_moon: & Waning Crescent Moon Symbol \\
\hline
U+1F319 & 🌙 & {\textbackslash}:crescent\_moon: & Crescent Moon \\
\hline
U+1F31A & 🌚 & {\textbackslash}:new\_moon\_with\_face: & New Moon With Face \\
\hline
U+1F31B & 🌛 & {\textbackslash}:first\_quarter\_moon\_with\_face: & First Quarter Moon With Face \\
\hline
U+1F31C & 🌜 & {\textbackslash}:last\_quarter\_moon\_with\_face: & Last Quarter Moon With Face \\
\hline
U+1F31D & 🌝 & {\textbackslash}:full\_moon\_with\_face: & Full Moon With Face \\
\hline
U+1F31E & 🌞 & {\textbackslash}:sun\_with\_face: & Sun With Face \\
\hline
U+1F31F & 🌟 & {\textbackslash}:star2: & Glowing Star \\
\hline
U+1F320 & 🌠 & {\textbackslash}:stars: & Shooting Star \\
\hline
U+1F330 & 🌰 & {\textbackslash}:chestnut: & Chestnut \\
\hline
U+1F331 & 🌱 & {\textbackslash}:seedling: & Seedling \\
\hline
U+1F332 & 🌲 & {\textbackslash}:evergreen\_tree: & Evergreen Tree \\
\hline
U+1F333 & 🌳 & {\textbackslash}:deciduous\_tree: & Deciduous Tree \\
\hline
U+1F334 & 🌴 & {\textbackslash}:palm\_tree: & Palm Tree \\
\hline
U+1F335 & 🌵 & {\textbackslash}:cactus: & Cactus \\
\hline
U+1F337 & 🌷 & {\textbackslash}:tulip: & Tulip \\
\hline
U+1F338 & 🌸 & {\textbackslash}:cherry\_blossom: & Cherry Blossom \\
\hline
U+1F339 & 🌹 & {\textbackslash}:rose: & Rose \\
\hline
U+1F33A & 🌺 & {\textbackslash}:hibiscus: & Hibiscus \\
\hline
U+1F33B & 🌻 & {\textbackslash}:sunflower: & Sunflower \\
\hline
U+1F33C & 🌼 & {\textbackslash}:blossom: & Blossom \\
\hline
U+1F33D & 🌽 & {\textbackslash}:corn: & Ear Of Maize \\
\hline
U+1F33E & 🌾 & {\textbackslash}:ear\_of\_rice: & Ear Of Rice \\
\hline
U+1F33F & 🌿 & {\textbackslash}:herb: & Herb \\
\hline
U+1F340 & 🍀 & {\textbackslash}:four\_leaf\_clover: & Four Leaf Clover \\
\hline
U+1F341 & 🍁 & {\textbackslash}:maple\_leaf: & Maple Leaf \\
\hline
U+1F342 & 🍂 & {\textbackslash}:fallen\_leaf: & Fallen Leaf \\
\hline
U+1F343 & 🍃 & {\textbackslash}:leaves: & Leaf Fluttering In Wind \\
\hline
U+1F344 & 🍄 & {\textbackslash}:mushroom: & Mushroom \\
\hline
U+1F345 & 🍅 & {\textbackslash}:tomato: & Tomato \\
\hline
U+1F346 & 🍆 & {\textbackslash}:eggplant: & Aubergine \\
\hline
U+1F347 & 🍇 & {\textbackslash}:grapes: & Grapes \\
\hline
U+1F348 & 🍈 & {\textbackslash}:melon: & Melon \\
\hline
U+1F349 & 🍉 & {\textbackslash}:watermelon: & Watermelon \\
\hline
U+1F34A & 🍊 & {\textbackslash}:tangerine: & Tangerine \\
\hline
U+1F34B & 🍋 & {\textbackslash}:lemon: & Lemon \\
\hline
U+1F34C & 🍌 & {\textbackslash}:banana: & Banana \\
\hline
U+1F34D & 🍍 & {\textbackslash}:pineapple: & Pineapple \\
\hline
U+1F34E & 🍎 & {\textbackslash}:apple: & Red Apple \\
\hline
U+1F34F & 🍏 & {\textbackslash}:green\_apple: & Green Apple \\
\hline
U+1F350 & 🍐 & {\textbackslash}:pear: & Pear \\
\hline
U+1F351 & 🍑 & {\textbackslash}:peach: & Peach \\
\hline
U+1F352 & 🍒 & {\textbackslash}:cherries: & Cherries \\
\hline
U+1F353 & 🍓 & {\textbackslash}:strawberry: & Strawberry \\
\hline
U+1F354 & 🍔 & {\textbackslash}:hamburger: & Hamburger \\
\hline
U+1F355 & 🍕 & {\textbackslash}:pizza: & Slice Of Pizza \\
\hline
U+1F356 & 🍖 & {\textbackslash}:meat\_on\_bone: & Meat On Bone \\
\hline
U+1F357 & 🍗 & {\textbackslash}:poultry\_leg: & Poultry Leg \\
\hline
U+1F358 & 🍘 & {\textbackslash}:rice\_cracker: & Rice Cracker \\
\hline
U+1F359 & 🍙 & {\textbackslash}:rice\_ball: & Rice Ball \\
\hline
U+1F35A & 🍚 & {\textbackslash}:rice: & Cooked Rice \\
\hline
U+1F35B & 🍛 & {\textbackslash}:curry: & Curry And Rice \\
\hline
U+1F35C & 🍜 & {\textbackslash}:ramen: & Steaming Bowl \\
\hline
U+1F35D & 🍝 & {\textbackslash}:spaghetti: & Spaghetti \\
\hline
U+1F35E & 🍞 & {\textbackslash}:bread: & Bread \\
\hline
U+1F35F & 🍟 & {\textbackslash}:fries: & French Fries \\
\hline
U+1F360 & 🍠 & {\textbackslash}:sweet\_potato: & Roasted Sweet Potato \\
\hline
U+1F361 & 🍡 & {\textbackslash}:dango: & Dango \\
\hline
U+1F362 & 🍢 & {\textbackslash}:oden: & Oden \\
\hline
U+1F363 & 🍣 & {\textbackslash}:sushi: & Sushi \\
\hline
U+1F364 & 🍤 & {\textbackslash}:fried\_shrimp: & Fried Shrimp \\
\hline
U+1F365 & 🍥 & {\textbackslash}:fish\_cake: & Fish Cake With Swirl Design \\
\hline
U+1F366 & 🍦 & {\textbackslash}:icecream: & Soft Ice Cream \\
\hline
U+1F367 & 🍧 & {\textbackslash}:shaved\_ice: & Shaved Ice \\
\hline
U+1F368 & 🍨 & {\textbackslash}:ice\_cream: & Ice Cream \\
\hline
U+1F369 & 🍩 & {\textbackslash}:doughnut: & Doughnut \\
\hline
U+1F36A & 🍪 & {\textbackslash}:cookie: & Cookie \\
\hline
U+1F36B & 🍫 & {\textbackslash}:chocolate\_bar: & Chocolate Bar \\
\hline
U+1F36C & 🍬 & {\textbackslash}:candy: & Candy \\
\hline
U+1F36D & 🍭 & {\textbackslash}:lollipop: & Lollipop \\
\hline
U+1F36E & 🍮 & {\textbackslash}:custard: & Custard \\
\hline
U+1F36F & 🍯 & {\textbackslash}:honey\_pot: & Honey Pot \\
\hline
U+1F370 & 🍰 & {\textbackslash}:cake: & Shortcake \\
\hline
U+1F371 & 🍱 & {\textbackslash}:bento: & Bento Box \\
\hline
U+1F372 & 🍲 & {\textbackslash}:stew: & Pot Of Food \\
\hline
U+1F373 & 🍳 & {\textbackslash}:egg: & Cooking \\
\hline
U+1F374 & 🍴 & {\textbackslash}:fork\_and\_knife: & Fork And Knife \\
\hline
U+1F375 & 🍵 & {\textbackslash}:tea: & Teacup Without Handle \\
\hline
U+1F376 & 🍶 & {\textbackslash}:sake: & Sake Bottle And Cup \\
\hline
U+1F377 & 🍷 & {\textbackslash}:wine\_glass: & Wine Glass \\
\hline
U+1F378 & 🍸 & {\textbackslash}:cocktail: & Cocktail Glass \\
\hline
U+1F379 & 🍹 & {\textbackslash}:tropical\_drink: & Tropical Drink \\
\hline
U+1F37A & 🍺 & {\textbackslash}:beer: & Beer Mug \\
\hline
U+1F37B & 🍻 & {\textbackslash}:beers: & Clinking Beer Mugs \\
\hline
U+1F37C & 🍼 & {\textbackslash}:baby\_bottle: & Baby Bottle \\
\hline
U+1F380 & 🎀 & {\textbackslash}:ribbon: & Ribbon \\
\hline
U+1F381 & 🎁 & {\textbackslash}:gift: & Wrapped Present \\
\hline
U+1F382 & 🎂 & {\textbackslash}:birthday: & Birthday Cake \\
\hline
U+1F383 & 🎃 & {\textbackslash}:jack\_o\_lantern: & Jack-O-Lantern \\
\hline
U+1F384 & 🎄 & {\textbackslash}:christmas\_tree: & Christmas Tree \\
\hline
U+1F385 & 🎅 & {\textbackslash}:santa: & Father Christmas \\
\hline
U+1F386 & 🎆 & {\textbackslash}:fireworks: & Fireworks \\
\hline
U+1F387 & 🎇 & {\textbackslash}:sparkler: & Firework Sparkler \\
\hline
U+1F388 & 🎈 & {\textbackslash}:balloon: & Balloon \\
\hline
U+1F389 & 🎉 & {\textbackslash}:tada: & Party Popper \\
\hline
U+1F38A & 🎊 & {\textbackslash}:confetti\_ball: & Confetti Ball \\
\hline
U+1F38B & 🎋 & {\textbackslash}:tanabata\_tree: & Tanabata Tree \\
\hline
U+1F38C & 🎌 & {\textbackslash}:crossed\_flags: & Crossed Flags \\
\hline
U+1F38D & 🎍 & {\textbackslash}:bamboo: & Pine Decoration \\
\hline
U+1F38E & 🎎 & {\textbackslash}:dolls: & Japanese Dolls \\
\hline
U+1F38F & 🎏 & {\textbackslash}:flags: & Carp Streamer \\
\hline
U+1F390 & 🎐 & {\textbackslash}:wind\_chime: & Wind Chime \\
\hline
U+1F391 & 🎑 & {\textbackslash}:rice\_scene: & Moon Viewing Ceremony \\
\hline
U+1F392 & 🎒 & {\textbackslash}:school\_satchel: & School Satchel \\
\hline
U+1F393 & 🎓 & {\textbackslash}:mortar\_board: & Graduation Cap \\
\hline
U+1F3A0 & 🎠 & {\textbackslash}:carousel\_horse: & Carousel Horse \\
\hline
U+1F3A1 & 🎡 & {\textbackslash}:ferris\_wheel: & Ferris Wheel \\
\hline
U+1F3A2 & 🎢 & {\textbackslash}:roller\_coaster: & Roller Coaster \\
\hline
U+1F3A3 & 🎣 & {\textbackslash}:fishing\_pole\_and\_fish: & Fishing Pole And Fish \\
\hline
U+1F3A4 & 🎤 & {\textbackslash}:microphone: & Microphone \\
\hline
U+1F3A5 & 🎥 & {\textbackslash}:movie\_camera: & Movie Camera \\
\hline
U+1F3A6 & 🎦 & {\textbackslash}:cinema: & Cinema \\
\hline
U+1F3A7 & 🎧 & {\textbackslash}:headphones: & Headphone \\
\hline
U+1F3A8 & 🎨 & {\textbackslash}:art: & Artist Palette \\
\hline
U+1F3A9 & 🎩 & {\textbackslash}:tophat: & Top Hat \\
\hline
U+1F3AA & 🎪 & {\textbackslash}:circus\_tent: & Circus Tent \\
\hline
U+1F3AB & 🎫 & {\textbackslash}:ticket: & Ticket \\
\hline
U+1F3AC & 🎬 & {\textbackslash}:clapper: & Clapper Board \\
\hline
U+1F3AD & 🎭 & {\textbackslash}:performing\_arts: & Performing Arts \\
\hline
U+1F3AE & 🎮 & {\textbackslash}:video\_game: & Video Game \\
\hline
U+1F3AF & 🎯 & {\textbackslash}:dart: & Direct Hit \\
\hline
U+1F3B0 & 🎰 & {\textbackslash}:slot\_machine: & Slot Machine \\
\hline
U+1F3B1 & 🎱 & {\textbackslash}:8ball: & Billiards \\
\hline
U+1F3B2 & 🎲 & {\textbackslash}:game\_die: & Game Die \\
\hline
U+1F3B3 & 🎳 & {\textbackslash}:bowling: & Bowling \\
\hline
U+1F3B4 & 🎴 & {\textbackslash}:flower\_playing\_cards: & Flower Playing Cards \\
\hline
U+1F3B5 & 🎵 & {\textbackslash}:musical\_note: & Musical Note \\
\hline
U+1F3B6 & 🎶 & {\textbackslash}:notes: & Multiple Musical Notes \\
\hline
U+1F3B7 & 🎷 & {\textbackslash}:saxophone: & Saxophone \\
\hline
U+1F3B8 & 🎸 & {\textbackslash}:guitar: & Guitar \\
\hline
U+1F3B9 & 🎹 & {\textbackslash}:musical\_keyboard: & Musical Keyboard \\
\hline
U+1F3BA & 🎺 & {\textbackslash}:trumpet: & Trumpet \\
\hline
U+1F3BB & 🎻 & {\textbackslash}:violin: & Violin \\
\hline
U+1F3BC & 🎼 & {\textbackslash}:musical\_score: & Musical Score \\
\hline
U+1F3BD & 🎽 & {\textbackslash}:running\_shirt\_with\_sash: & Running Shirt With Sash \\
\hline
U+1F3BE & 🎾 & {\textbackslash}:tennis: & Tennis Racquet And Ball \\
\hline
U+1F3BF & 🎿 & {\textbackslash}:ski: & Ski And Ski Boot \\
\hline
U+1F3C0 & 🏀 & {\textbackslash}:basketball: & Basketball And Hoop \\
\hline
U+1F3C1 & 🏁 & {\textbackslash}:checkered\_flag: & Chequered Flag \\
\hline
U+1F3C2 & 🏂 & {\textbackslash}:snowboarder: & Snowboarder \\
\hline
U+1F3C3 & 🏃 & {\textbackslash}:runner: & Runner \\
\hline
U+1F3C4 & 🏄 & {\textbackslash}:surfer: & Surfer \\
\hline
U+1F3C6 & 🏆 & {\textbackslash}:trophy: & Trophy \\
\hline
U+1F3C7 & 🏇 & {\textbackslash}:horse\_racing: & Horse Racing \\
\hline
U+1F3C8 & 🏈 & {\textbackslash}:football: & American Football \\
\hline
U+1F3C9 & 🏉 & {\textbackslash}:rugby\_football: & Rugby Football \\
\hline
U+1F3CA & 🏊 & {\textbackslash}:swimmer: & Swimmer \\
\hline
U+1F3E0 & 🏠 & {\textbackslash}:house: & House Building \\
\hline
U+1F3E1 & 🏡 & {\textbackslash}:house\_with\_garden: & House With Garden \\
\hline
U+1F3E2 & 🏢 & {\textbackslash}:office: & Office Building \\
\hline
U+1F3E3 & 🏣 & {\textbackslash}:post\_office: & Japanese Post Office \\
\hline
U+1F3E4 & 🏤 & {\textbackslash}:european\_post\_office: & European Post Office \\
\hline
U+1F3E5 & 🏥 & {\textbackslash}:hospital: & Hospital \\
\hline
U+1F3E6 & 🏦 & {\textbackslash}:bank: & Bank \\
\hline
U+1F3E7 & 🏧 & {\textbackslash}:atm: & Automated Teller Machine \\
\hline
U+1F3E8 & 🏨 & {\textbackslash}:hotel: & Hotel \\
\hline
U+1F3E9 & 🏩 & {\textbackslash}:love\_hotel: & Love Hotel \\
\hline
U+1F3EA & 🏪 & {\textbackslash}:convenience\_store: & Convenience Store \\
\hline
U+1F3EB & 🏫 & {\textbackslash}:school: & School \\
\hline
U+1F3EC & 🏬 & {\textbackslash}:department\_store: & Department Store \\
\hline
U+1F3ED & 🏭 & {\textbackslash}:factory: & Factory \\
\hline
U+1F3EE & 🏮 & {\textbackslash}:izakaya\_lantern: & Izakaya Lantern \\
\hline
U+1F3EF & 🏯 & {\textbackslash}:japanese\_castle: & Japanese Castle \\
\hline
U+1F3F0 & 🏰 & {\textbackslash}:european\_castle: & European Castle \\
\hline
U+1F3FB & 🏻 & {\textbackslash}:skin-tone-2: & Emoji Modifier Fitzpatrick Type-1-2 \\
\hline
U+1F3FC & 🏼 & {\textbackslash}:skin-tone-3: & Emoji Modifier Fitzpatrick Type-3 \\
\hline
U+1F3FD & 🏽 & {\textbackslash}:skin-tone-4: & Emoji Modifier Fitzpatrick Type-4 \\
\hline
U+1F3FE & 🏾 & {\textbackslash}:skin-tone-5: & Emoji Modifier Fitzpatrick Type-5 \\
\hline
U+1F3FF & 🏿 & {\textbackslash}:skin-tone-6: & Emoji Modifier Fitzpatrick Type-6 \\
\hline
U+1F400 & 🐀 & {\textbackslash}:rat: & Rat \\
\hline
U+1F401 & 🐁 & {\textbackslash}:mouse2: & Mouse \\
\hline
U+1F402 & 🐂 & {\textbackslash}:ox: & Ox \\
\hline
U+1F403 & 🐃 & {\textbackslash}:water\_buffalo: & Water Buffalo \\
\hline
U+1F404 & 🐄 & {\textbackslash}:cow2: & Cow \\
\hline
U+1F405 & 🐅 & {\textbackslash}:tiger2: & Tiger \\
\hline
U+1F406 & 🐆 & {\textbackslash}:leopard: & Leopard \\
\hline
U+1F407 & 🐇 & {\textbackslash}:rabbit2: & Rabbit \\
\hline
U+1F408 & 🐈 & {\textbackslash}:cat2: & Cat \\
\hline
U+1F409 & 🐉 & {\textbackslash}:dragon: & Dragon \\
\hline
U+1F40A & 🐊 & {\textbackslash}:crocodile: & Crocodile \\
\hline
U+1F40B & 🐋 & {\textbackslash}:whale2: & Whale \\
\hline
U+1F40C & 🐌 & {\textbackslash}:snail: & Snail \\
\hline
U+1F40D & 🐍 & {\textbackslash}:snake: & Snake \\
\hline
U+1F40E & 🐎 & {\textbackslash}:racehorse: & Horse \\
\hline
U+1F40F & 🐏 & {\textbackslash}:ram: & Ram \\
\hline
U+1F410 & 🐐 & {\textbackslash}:goat: & Goat \\
\hline
U+1F411 & 🐑 & {\textbackslash}:sheep: & Sheep \\
\hline
U+1F412 & 🐒 & {\textbackslash}:monkey: & Monkey \\
\hline
U+1F413 & 🐓 & {\textbackslash}:rooster: & Rooster \\
\hline
U+1F414 & 🐔 & {\textbackslash}:chicken: & Chicken \\
\hline
U+1F415 & 🐕 & {\textbackslash}:dog2: & Dog \\
\hline
U+1F416 & 🐖 & {\textbackslash}:pig2: & Pig \\
\hline
U+1F417 & 🐗 & {\textbackslash}:boar: & Boar \\
\hline
U+1F418 & 🐘 & {\textbackslash}:elephant: & Elephant \\
\hline
U+1F419 & 🐙 & {\textbackslash}:octopus: & Octopus \\
\hline
U+1F41A & 🐚 & {\textbackslash}:shell: & Spiral Shell \\
\hline
U+1F41B & 🐛 & {\textbackslash}:bug: & Bug \\
\hline
U+1F41C & 🐜 & {\textbackslash}:ant: & Ant \\
\hline
U+1F41D & 🐝 & {\textbackslash}:bee: & Honeybee \\
\hline
U+1F41E & 🐞 & {\textbackslash}:beetle: & Lady Beetle \\
\hline
U+1F41F & 🐟 & {\textbackslash}:fish: & Fish \\
\hline
U+1F420 & 🐠 & {\textbackslash}:tropical\_fish: & Tropical Fish \\
\hline
U+1F421 & 🐡 & {\textbackslash}:blowfish: & Blowfish \\
\hline
U+1F422 & 🐢 & {\textbackslash}:turtle: & Turtle \\
\hline
U+1F423 & 🐣 & {\textbackslash}:hatching\_chick: & Hatching Chick \\
\hline
U+1F424 & 🐤 & {\textbackslash}:baby\_chick: & Baby Chick \\
\hline
U+1F425 & 🐥 & {\textbackslash}:hatched\_chick: & Front-Facing Baby Chick \\
\hline
U+1F426 & 🐦 & {\textbackslash}:bird: & Bird \\
\hline
U+1F427 & 🐧 & {\textbackslash}:penguin: & Penguin \\
\hline
U+1F428 & 🐨 & {\textbackslash}:koala: & Koala \\
\hline
U+1F429 & 🐩 & {\textbackslash}:poodle: & Poodle \\
\hline
U+1F42A & 🐪 & {\textbackslash}:dromedary\_camel: & Dromedary Camel \\
\hline
U+1F42B & 🐫 & {\textbackslash}:camel: & Bactrian Camel \\
\hline
U+1F42C & 🐬 & {\textbackslash}:dolphin: & Dolphin \\
\hline
U+1F42D & 🐭 & {\textbackslash}:mouse: & Mouse Face \\
\hline
U+1F42E & 🐮 & {\textbackslash}:cow: & Cow Face \\
\hline
U+1F42F & 🐯 & {\textbackslash}:tiger: & Tiger Face \\
\hline
U+1F430 & 🐰 & {\textbackslash}:rabbit: & Rabbit Face \\
\hline
U+1F431 & 🐱 & {\textbackslash}:cat: & Cat Face \\
\hline
U+1F432 & 🐲 & {\textbackslash}:dragon\_face: & Dragon Face \\
\hline
U+1F433 & 🐳 & {\textbackslash}:whale: & Spouting Whale \\
\hline
U+1F434 & 🐴 & {\textbackslash}:horse: & Horse Face \\
\hline
U+1F435 & 🐵 & {\textbackslash}:monkey\_face: & Monkey Face \\
\hline
U+1F436 & 🐶 & {\textbackslash}:dog: & Dog Face \\
\hline
U+1F437 & 🐷 & {\textbackslash}:pig: & Pig Face \\
\hline
U+1F438 & 🐸 & {\textbackslash}:frog: & Frog Face \\
\hline
U+1F439 & 🐹 & {\textbackslash}:hamster: & Hamster Face \\
\hline
U+1F43A & 🐺 & {\textbackslash}:wolf: & Wolf Face \\
\hline
U+1F43B & 🐻 & {\textbackslash}:bear: & Bear Face \\
\hline
U+1F43C & 🐼 & {\textbackslash}:panda\_face: & Panda Face \\
\hline
U+1F43D & 🐽 & {\textbackslash}:pig\_nose: & Pig Nose \\
\hline
U+1F43E & 🐾 & {\textbackslash}:feet: & Paw Prints \\
\hline
U+1F440 & 👀 & {\textbackslash}:eyes: & Eyes \\
\hline
U+1F442 & 👂 & {\textbackslash}:ear: & Ear \\
\hline
U+1F443 & 👃 & {\textbackslash}:nose: & Nose \\
\hline
U+1F444 & 👄 & {\textbackslash}:lips: & Mouth \\
\hline
U+1F445 & 👅 & {\textbackslash}:tongue: & Tongue \\
\hline
U+1F446 & 👆 & {\textbackslash}:point\_up\_2: & White Up Pointing Backhand Index \\
\hline
U+1F447 & 👇 & {\textbackslash}:point\_down: & White Down Pointing Backhand Index \\
\hline
U+1F448 & 👈 & {\textbackslash}:point\_left: & White Left Pointing Backhand Index \\
\hline
U+1F449 & 👉 & {\textbackslash}:point\_right: & White Right Pointing Backhand Index \\
\hline
U+1F44A & 👊 & {\textbackslash}:facepunch: & Fisted Hand Sign \\
\hline
U+1F44B & 👋 & {\textbackslash}:wave: & Waving Hand Sign \\
\hline
U+1F44C & 👌 & {\textbackslash}:ok\_hand: & Ok Hand Sign \\
\hline
U+1F44D & 👍 & {\textbackslash}:+1: & Thumbs Up Sign \\
\hline
U+1F44E & 👎 & {\textbackslash}:-1: & Thumbs Down Sign \\
\hline
U+1F44F & 👏 & {\textbackslash}:clap: & Clapping Hands Sign \\
\hline
U+1F450 & 👐 & {\textbackslash}:open\_hands: & Open Hands Sign \\
\hline
U+1F451 & 👑 & {\textbackslash}:crown: & Crown \\
\hline
U+1F452 & 👒 & {\textbackslash}:womans\_hat: & Womans Hat \\
\hline
U+1F453 & 👓 & {\textbackslash}:eyeglasses: & Eyeglasses \\
\hline
U+1F454 & 👔 & {\textbackslash}:necktie: & Necktie \\
\hline
U+1F455 & 👕 & {\textbackslash}:shirt: & T-Shirt \\
\hline
U+1F456 & 👖 & {\textbackslash}:jeans: & Jeans \\
\hline
U+1F457 & 👗 & {\textbackslash}:dress: & Dress \\
\hline
U+1F458 & 👘 & {\textbackslash}:kimono: & Kimono \\
\hline
U+1F459 & 👙 & {\textbackslash}:bikini: & Bikini \\
\hline
U+1F45A & 👚 & {\textbackslash}:womans\_clothes: & Womans Clothes \\
\hline
U+1F45B & 👛 & {\textbackslash}:purse: & Purse \\
\hline
U+1F45C & 👜 & {\textbackslash}:handbag: & Handbag \\
\hline
U+1F45D & 👝 & {\textbackslash}:pouch: & Pouch \\
\hline
U+1F45E & 👞 & {\textbackslash}:mans\_shoe: & Mans Shoe \\
\hline
U+1F45F & 👟 & {\textbackslash}:athletic\_shoe: & Athletic Shoe \\
\hline
U+1F460 & 👠 & {\textbackslash}:high\_heel: & High-Heeled Shoe \\
\hline
U+1F461 & 👡 & {\textbackslash}:sandal: & Womans Sandal \\
\hline
U+1F462 & 👢 & {\textbackslash}:boot: & Womans Boots \\
\hline
U+1F463 & 👣 & {\textbackslash}:footprints: & Footprints \\
\hline
U+1F464 & 👤 & {\textbackslash}:bust\_in\_silhouette: & Bust In Silhouette \\
\hline
U+1F465 & 👥 & {\textbackslash}:busts\_in\_silhouette: & Busts In Silhouette \\
\hline
U+1F466 & 👦 & {\textbackslash}:boy: & Boy \\
\hline
U+1F467 & 👧 & {\textbackslash}:girl: & Girl \\
\hline
U+1F468 & 👨 & {\textbackslash}:man: & Man \\
\hline
U+1F469 & 👩 & {\textbackslash}:woman: & Woman \\
\hline
U+1F46A & 👪 & {\textbackslash}:family: & Family \\
\hline
U+1F46B & 👫 & {\textbackslash}:couple: & Man And Woman Holding Hands \\
\hline
U+1F46C & 👬 & {\textbackslash}:two\_men\_holding\_hands: & Two Men Holding Hands \\
\hline
U+1F46D & 👭 & {\textbackslash}:two\_women\_holding\_hands: & Two Women Holding Hands \\
\hline
U+1F46E & 👮 & {\textbackslash}:cop: & Police Officer \\
\hline
U+1F46F & 👯 & {\textbackslash}:dancers: & Woman With Bunny Ears \\
\hline
U+1F470 & 👰 & {\textbackslash}:bride\_with\_veil: & Bride With Veil \\
\hline
U+1F471 & 👱 & {\textbackslash}:person\_with\_blond\_hair: & Person With Blond Hair \\
\hline
U+1F472 & 👲 & {\textbackslash}:man\_with\_gua\_pi\_mao: & Man With Gua Pi Mao \\
\hline
U+1F473 & 👳 & {\textbackslash}:man\_with\_turban: & Man With Turban \\
\hline
U+1F474 & 👴 & {\textbackslash}:older\_man: & Older Man \\
\hline
U+1F475 & 👵 & {\textbackslash}:older\_woman: & Older Woman \\
\hline
U+1F476 & 👶 & {\textbackslash}:baby: & Baby \\
\hline
U+1F477 & 👷 & {\textbackslash}:construction\_worker: & Construction Worker \\
\hline
U+1F478 & 👸 & {\textbackslash}:princess: & Princess \\
\hline
U+1F479 & 👹 & {\textbackslash}:japanese\_ogre: & Japanese Ogre \\
\hline
U+1F47A & 👺 & {\textbackslash}:japanese\_goblin: & Japanese Goblin \\
\hline
U+1F47B & 👻 & {\textbackslash}:ghost: & Ghost \\
\hline
U+1F47C & 👼 & {\textbackslash}:angel: & Baby Angel \\
\hline
U+1F47D & 👽 & {\textbackslash}:alien: & Extraterrestrial Alien \\
\hline
U+1F47E & 👾 & {\textbackslash}:space\_invader: & Alien Monster \\
\hline
U+1F47F & 👿 & {\textbackslash}:imp: & Imp \\
\hline
U+1F480 & 💀 & {\textbackslash}:skull: & Skull \\
\hline
U+1F481 & 💁 & {\textbackslash}:information\_desk\_person: & Information Desk Person \\
\hline
U+1F482 & 💂 & {\textbackslash}:guardsman: & Guardsman \\
\hline
U+1F483 & 💃 & {\textbackslash}:dancer: & Dancer \\
\hline
U+1F484 & 💄 & {\textbackslash}:lipstick: & Lipstick \\
\hline
U+1F485 & 💅 & {\textbackslash}:nail\_care: & Nail Polish \\
\hline
U+1F486 & 💆 & {\textbackslash}:massage: & Face Massage \\
\hline
U+1F487 & 💇 & {\textbackslash}:haircut: & Haircut \\
\hline
U+1F488 & 💈 & {\textbackslash}:barber: & Barber Pole \\
\hline
U+1F489 & 💉 & {\textbackslash}:syringe: & Syringe \\
\hline
U+1F48A & 💊 & {\textbackslash}:pill: & Pill \\
\hline
U+1F48B & 💋 & {\textbackslash}:kiss: & Kiss Mark \\
\hline
U+1F48C & 💌 & {\textbackslash}:love\_letter: & Love Letter \\
\hline
U+1F48D & 💍 & {\textbackslash}:ring: & Ring \\
\hline
U+1F48E & 💎 & {\textbackslash}:gem: & Gem Stone \\
\hline
U+1F48F & 💏 & {\textbackslash}:couplekiss: & Kiss \\
\hline
U+1F490 & 💐 & {\textbackslash}:bouquet: & Bouquet \\
\hline
U+1F491 & 💑 & {\textbackslash}:couple\_with\_heart: & Couple With Heart \\
\hline
U+1F492 & 💒 & {\textbackslash}:wedding: & Wedding \\
\hline
U+1F493 & 💓 & {\textbackslash}:heartbeat: & Beating Heart \\
\hline
U+1F494 & 💔 & {\textbackslash}:broken\_heart: & Broken Heart \\
\hline
U+1F495 & 💕 & {\textbackslash}:two\_hearts: & Two Hearts \\
\hline
U+1F496 & 💖 & {\textbackslash}:sparkling\_heart: & Sparkling Heart \\
\hline
U+1F497 & 💗 & {\textbackslash}:heartpulse: & Growing Heart \\
\hline
U+1F498 & 💘 & {\textbackslash}:cupid: & Heart With Arrow \\
\hline
U+1F499 & 💙 & {\textbackslash}:blue\_heart: & Blue Heart \\
\hline
U+1F49A & 💚 & {\textbackslash}:green\_heart: & Green Heart \\
\hline
U+1F49B & 💛 & {\textbackslash}:yellow\_heart: & Yellow Heart \\
\hline
U+1F49C & 💜 & {\textbackslash}:purple\_heart: & Purple Heart \\
\hline
U+1F49D & 💝 & {\textbackslash}:gift\_heart: & Heart With Ribbon \\
\hline
U+1F49E & 💞 & {\textbackslash}:revolving\_hearts: & Revolving Hearts \\
\hline
U+1F49F & 💟 & {\textbackslash}:heart\_decoration: & Heart Decoration \\
\hline
U+1F4A0 & 💠 & {\textbackslash}:diamond\_shape\_with\_a\_dot\_inside: & Diamond Shape With A Dot Inside \\
\hline
U+1F4A1 & 💡 & {\textbackslash}:bulb: & Electric Light Bulb \\
\hline
U+1F4A2 & 💢 & {\textbackslash}:anger: & Anger Symbol \\
\hline
U+1F4A3 & 💣 & {\textbackslash}:bomb: & Bomb \\
\hline
U+1F4A4 & 💤 & {\textbackslash}:zzz: & Sleeping Symbol \\
\hline
U+1F4A5 & 💥 & {\textbackslash}:boom: & Collision Symbol \\
\hline
U+1F4A6 & 💦 & {\textbackslash}:sweat\_drops: & Splashing Sweat Symbol \\
\hline
U+1F4A7 & 💧 & {\textbackslash}:droplet: & Droplet \\
\hline
U+1F4A8 & 💨 & {\textbackslash}:dash: & Dash Symbol \\
\hline
U+1F4A9 & 💩 & {\textbackslash}:hankey: & Pile Of Poo \\
\hline
U+1F4AA & 💪 & {\textbackslash}:muscle: & Flexed Biceps \\
\hline
U+1F4AB & 💫 & {\textbackslash}:dizzy: & Dizzy Symbol \\
\hline
U+1F4AC & 💬 & {\textbackslash}:speech\_balloon: & Speech Balloon \\
\hline
U+1F4AD & 💭 & {\textbackslash}:thought\_balloon: & Thought Balloon \\
\hline
U+1F4AE & 💮 & {\textbackslash}:white\_flower: & White Flower \\
\hline
U+1F4AF & 💯 & {\textbackslash}:100: & Hundred Points Symbol \\
\hline
U+1F4B0 & 💰 & {\textbackslash}:moneybag: & Money Bag \\
\hline
U+1F4B1 & 💱 & {\textbackslash}:currency\_exchange: & Currency Exchange \\
\hline
U+1F4B2 & 💲 & {\textbackslash}:heavy\_dollar\_sign: & Heavy Dollar Sign \\
\hline
U+1F4B3 & 💳 & {\textbackslash}:credit\_card: & Credit Card \\
\hline
U+1F4B4 & 💴 & {\textbackslash}:yen: & Banknote With Yen Sign \\
\hline
U+1F4B5 & 💵 & {\textbackslash}:dollar: & Banknote With Dollar Sign \\
\hline
U+1F4B6 & 💶 & {\textbackslash}:euro: & Banknote With Euro Sign \\
\hline
U+1F4B7 & 💷 & {\textbackslash}:pound: & Banknote With Pound Sign \\
\hline
U+1F4B8 & 💸 & {\textbackslash}:money\_with\_wings: & Money With Wings \\
\hline
U+1F4B9 & 💹 & {\textbackslash}:chart: & Chart With Upwards Trend And Yen Sign \\
\hline
U+1F4BA & 💺 & {\textbackslash}:seat: & Seat \\
\hline
U+1F4BB & 💻 & {\textbackslash}:computer: & Personal Computer \\
\hline
U+1F4BC & 💼 & {\textbackslash}:briefcase: & Briefcase \\
\hline
U+1F4BD & 💽 & {\textbackslash}:minidisc: & Minidisc \\
\hline
U+1F4BE & 💾 & {\textbackslash}:floppy\_disk: & Floppy Disk \\
\hline
U+1F4BF & 💿 & {\textbackslash}:cd: & Optical Disc \\
\hline
U+1F4C0 & 📀 & {\textbackslash}:dvd: & Dvd \\
\hline
U+1F4C1 & 📁 & {\textbackslash}:file\_folder: & File Folder \\
\hline
U+1F4C2 & 📂 & {\textbackslash}:open\_file\_folder: & Open File Folder \\
\hline
U+1F4C3 & 📃 & {\textbackslash}:page\_with\_curl: & Page With Curl \\
\hline
U+1F4C4 & 📄 & {\textbackslash}:page\_facing\_up: & Page Facing Up \\
\hline
U+1F4C5 & 📅 & {\textbackslash}:date: & Calendar \\
\hline
U+1F4C6 & 📆 & {\textbackslash}:calendar: & Tear-Off Calendar \\
\hline
U+1F4C7 & 📇 & {\textbackslash}:card\_index: & Card Index \\
\hline
U+1F4C8 & 📈 & {\textbackslash}:chart\_with\_upwards\_trend: & Chart With Upwards Trend \\
\hline
U+1F4C9 & 📉 & {\textbackslash}:chart\_with\_downwards\_trend: & Chart With Downwards Trend \\
\hline
U+1F4CA & 📊 & {\textbackslash}:bar\_chart: & Bar Chart \\
\hline
U+1F4CB & 📋 & {\textbackslash}:clipboard: & Clipboard \\
\hline
U+1F4CC & 📌 & {\textbackslash}:pushpin: & Pushpin \\
\hline
U+1F4CD & 📍 & {\textbackslash}:round\_pushpin: & Round Pushpin \\
\hline
U+1F4CE & 📎 & {\textbackslash}:paperclip: & Paperclip \\
\hline
U+1F4CF & 📏 & {\textbackslash}:straight\_ruler: & Straight Ruler \\
\hline
U+1F4D0 & 📐 & {\textbackslash}:triangular\_ruler: & Triangular Ruler \\
\hline
U+1F4D1 & 📑 & {\textbackslash}:bookmark\_tabs: & Bookmark Tabs \\
\hline
U+1F4D2 & 📒 & {\textbackslash}:ledger: & Ledger \\
\hline
U+1F4D3 & 📓 & {\textbackslash}:notebook: & Notebook \\
\hline
U+1F4D4 & 📔 & {\textbackslash}:notebook\_with\_decorative\_cover: & Notebook With Decorative Cover \\
\hline
U+1F4D5 & 📕 & {\textbackslash}:closed\_book: & Closed Book \\
\hline
U+1F4D6 & 📖 & {\textbackslash}:book: & Open Book \\
\hline
U+1F4D7 & 📗 & {\textbackslash}:green\_book: & Green Book \\
\hline
U+1F4D8 & 📘 & {\textbackslash}:blue\_book: & Blue Book \\
\hline
U+1F4D9 & 📙 & {\textbackslash}:orange\_book: & Orange Book \\
\hline
U+1F4DA & 📚 & {\textbackslash}:books: & Books \\
\hline
U+1F4DB & 📛 & {\textbackslash}:name\_badge: & Name Badge \\
\hline
U+1F4DC & 📜 & {\textbackslash}:scroll: & Scroll \\
\hline
U+1F4DD & 📝 & {\textbackslash}:memo: & Memo \\
\hline
U+1F4DE & 📞 & {\textbackslash}:telephone\_receiver: & Telephone Receiver \\
\hline
U+1F4DF & 📟 & {\textbackslash}:pager: & Pager \\
\hline
U+1F4E0 & 📠 & {\textbackslash}:fax: & Fax Machine \\
\hline
U+1F4E1 & 📡 & {\textbackslash}:satellite: & Satellite Antenna \\
\hline
U+1F4E2 & 📢 & {\textbackslash}:loudspeaker: & Public Address Loudspeaker \\
\hline
U+1F4E3 & 📣 & {\textbackslash}:mega: & Cheering Megaphone \\
\hline
U+1F4E4 & 📤 & {\textbackslash}:outbox\_tray: & Outbox Tray \\
\hline
U+1F4E5 & 📥 & {\textbackslash}:inbox\_tray: & Inbox Tray \\
\hline
U+1F4E6 & 📦 & {\textbackslash}:package: & Package \\
\hline
U+1F4E7 & 📧 & {\textbackslash}:e-mail: & E-Mail Symbol \\
\hline
U+1F4E8 & 📨 & {\textbackslash}:incoming\_envelope: & Incoming Envelope \\
\hline
U+1F4E9 & 📩 & {\textbackslash}:envelope\_with\_arrow: & Envelope With Downwards Arrow Above \\
\hline
U+1F4EA & 📪 & {\textbackslash}:mailbox\_closed: & Closed Mailbox With Lowered Flag \\
\hline
U+1F4EB & 📫 & {\textbackslash}:mailbox: & Closed Mailbox With Raised Flag \\
\hline
U+1F4EC & 📬 & {\textbackslash}:mailbox\_with\_mail: & Open Mailbox With Raised Flag \\
\hline
U+1F4ED & 📭 & {\textbackslash}:mailbox\_with\_no\_mail: & Open Mailbox With Lowered Flag \\
\hline
U+1F4EE & 📮 & {\textbackslash}:postbox: & Postbox \\
\hline
U+1F4EF & 📯 & {\textbackslash}:postal\_horn: & Postal Horn \\
\hline
U+1F4F0 & 📰 & {\textbackslash}:newspaper: & Newspaper \\
\hline
U+1F4F1 & 📱 & {\textbackslash}:iphone: & Mobile Phone \\
\hline
U+1F4F2 & 📲 & {\textbackslash}:calling: & Mobile Phone With Rightwards Arrow At Left \\
\hline
U+1F4F3 & 📳 & {\textbackslash}:vibration\_mode: & Vibration Mode \\
\hline
U+1F4F4 & 📴 & {\textbackslash}:mobile\_phone\_off: & Mobile Phone Off \\
\hline
U+1F4F5 & 📵 & {\textbackslash}:no\_mobile\_phones: & No Mobile Phones \\
\hline
U+1F4F6 & 📶 & {\textbackslash}:signal\_strength: & Antenna With Bars \\
\hline
U+1F4F7 & 📷 & {\textbackslash}:camera: & Camera \\
\hline
U+1F4F9 & 📹 & {\textbackslash}:video\_camera: & Video Camera \\
\hline
U+1F4FA & 📺 & {\textbackslash}:tv: & Television \\
\hline
U+1F4FB & 📻 & {\textbackslash}:radio: & Radio \\
\hline
U+1F4FC & 📼 & {\textbackslash}:vhs: & Videocassette \\
\hline
U+1F500 & 🔀 & {\textbackslash}:twisted\_rightwards\_arrows: & Twisted Rightwards Arrows \\
\hline
U+1F501 & 🔁 & {\textbackslash}:repeat: & Clockwise Rightwards And Leftwards Open Circle Arrows \\
\hline
U+1F502 & 🔂 & {\textbackslash}:repeat\_one: & Clockwise Rightwards And Leftwards Open Circle Arrows With Circled One Overlay \\
\hline
U+1F503 & 🔃 & {\textbackslash}:arrows\_clockwise: & Clockwise Downwards And Upwards Open Circle Arrows \\
\hline
U+1F504 & 🔄 & {\textbackslash}:arrows\_counterclockwise: & Anticlockwise Downwards And Upwards Open Circle Arrows \\
\hline
U+1F505 & 🔅 & {\textbackslash}:low\_brightness: & Low Brightness Symbol \\
\hline
U+1F506 & 🔆 & {\textbackslash}:high\_brightness: & High Brightness Symbol \\
\hline
U+1F507 & 🔇 & {\textbackslash}:mute: & Speaker With Cancellation Stroke \\
\hline
U+1F508 & 🔈 & {\textbackslash}:speaker: & Speaker \\
\hline
U+1F509 & 🔉 & {\textbackslash}:sound: & Speaker With One Sound Wave \\
\hline
U+1F50A & 🔊 & {\textbackslash}:loud\_sound: & Speaker With Three Sound Waves \\
\hline
U+1F50B & 🔋 & {\textbackslash}:battery: & Battery \\
\hline
U+1F50C & 🔌 & {\textbackslash}:electric\_plug: & Electric Plug \\
\hline
U+1F50D & 🔍 & {\textbackslash}:mag: & Left-Pointing Magnifying Glass \\
\hline
U+1F50E & 🔎 & {\textbackslash}:mag\_right: & Right-Pointing Magnifying Glass \\
\hline
U+1F50F & 🔏 & {\textbackslash}:lock\_with\_ink\_pen: & Lock With Ink Pen \\
\hline
U+1F510 & 🔐 & {\textbackslash}:closed\_lock\_with\_key: & Closed Lock With Key \\
\hline
U+1F511 & 🔑 & {\textbackslash}:key: & Key \\
\hline
U+1F512 & 🔒 & {\textbackslash}:lock: & Lock \\
\hline
U+1F513 & 🔓 & {\textbackslash}:unlock: & Open Lock \\
\hline
U+1F514 & 🔔 & {\textbackslash}:bell: & Bell \\
\hline
U+1F515 & 🔕 & {\textbackslash}:no\_bell: & Bell With Cancellation Stroke \\
\hline
U+1F516 & 🔖 & {\textbackslash}:bookmark: & Bookmark \\
\hline
U+1F517 & 🔗 & {\textbackslash}:link: & Link Symbol \\
\hline
U+1F518 & 🔘 & {\textbackslash}:radio\_button: & Radio Button \\
\hline
U+1F519 & 🔙 & {\textbackslash}:back: & Back With Leftwards Arrow Above \\
\hline
U+1F51A & 🔚 & {\textbackslash}:end: & End With Leftwards Arrow Above \\
\hline
U+1F51B & 🔛 & {\textbackslash}:on: & On With Exclamation Mark With Left Right Arrow Above \\
\hline
U+1F51C & 🔜 & {\textbackslash}:soon: & Soon With Rightwards Arrow Above \\
\hline
U+1F51D & 🔝 & {\textbackslash}:top: & Top With Upwards Arrow Above \\
\hline
U+1F51E & 🔞 & {\textbackslash}:underage: & No One Under Eighteen Symbol \\
\hline
U+1F51F & 🔟 & {\textbackslash}:keycap\_ten: & Keycap Ten \\
\hline
U+1F520 & 🔠 & {\textbackslash}:capital\_abcd: & Input Symbol For Latin Capital Letters \\
\hline
U+1F521 & 🔡 & {\textbackslash}:abcd: & Input Symbol For Latin Small Letters \\
\hline
U+1F522 & 🔢 & {\textbackslash}:1234: & Input Symbol For Numbers \\
\hline
U+1F523 & 🔣 & {\textbackslash}:symbols: & Input Symbol For Symbols \\
\hline
U+1F524 & 🔤 & {\textbackslash}:abc: & Input Symbol For Latin Letters \\
\hline
U+1F525 & 🔥 & {\textbackslash}:fire: & Fire \\
\hline
U+1F526 & 🔦 & {\textbackslash}:flashlight: & Electric Torch \\
\hline
U+1F527 & 🔧 & {\textbackslash}:wrench: & Wrench \\
\hline
U+1F528 & 🔨 & {\textbackslash}:hammer: & Hammer \\
\hline
U+1F529 & 🔩 & {\textbackslash}:nut\_and\_bolt: & Nut And Bolt \\
\hline
U+1F52A & 🔪 & {\textbackslash}:hocho: & Hocho \\
\hline
U+1F52B & 🔫 & {\textbackslash}:gun: & Pistol \\
\hline
U+1F52C & 🔬 & {\textbackslash}:microscope: & Microscope \\
\hline
U+1F52D & 🔭 & {\textbackslash}:telescope: & Telescope \\
\hline
U+1F52E & 🔮 & {\textbackslash}:crystal\_ball: & Crystal Ball \\
\hline
U+1F52F & 🔯 & {\textbackslash}:six\_pointed\_star: & Six Pointed Star With Middle Dot \\
\hline
U+1F530 & 🔰 & {\textbackslash}:beginner: & Japanese Symbol For Beginner \\
\hline
U+1F531 & 🔱 & {\textbackslash}:trident: & Trident Emblem \\
\hline
U+1F532 & 🔲 & {\textbackslash}:black\_square\_button: & Black Square Button \\
\hline
U+1F533 & 🔳 & {\textbackslash}:white\_square\_button: & White Square Button \\
\hline
U+1F534 & 🔴 & {\textbackslash}:red\_circle: & Large Red Circle \\
\hline
U+1F535 & 🔵 & {\textbackslash}:large\_blue\_circle: & Large Blue Circle \\
\hline
U+1F536 & 🔶 & {\textbackslash}:large\_orange\_diamond: & Large Orange Diamond \\
\hline
U+1F537 & 🔷 & {\textbackslash}:large\_blue\_diamond: & Large Blue Diamond \\
\hline
U+1F538 & 🔸 & {\textbackslash}:small\_orange\_diamond: & Small Orange Diamond \\
\hline
U+1F539 & 🔹 & {\textbackslash}:small\_blue\_diamond: & Small Blue Diamond \\
\hline
U+1F53A & 🔺 & {\textbackslash}:small\_red\_triangle: & Up-Pointing Red Triangle \\
\hline
U+1F53B & 🔻 & {\textbackslash}:small\_red\_triangle\_down: & Down-Pointing Red Triangle \\
\hline
U+1F53C & 🔼 & {\textbackslash}:arrow\_up\_small: & Up-Pointing Small Red Triangle \\
\hline
U+1F53D & 🔽 & {\textbackslash}:arrow\_down\_small: & Down-Pointing Small Red Triangle \\
\hline
U+1F550 & 🕐 & {\textbackslash}:clock1: & Clock Face One Oclock \\
\hline
U+1F551 & 🕑 & {\textbackslash}:clock2: & Clock Face Two Oclock \\
\hline
U+1F552 & 🕒 & {\textbackslash}:clock3: & Clock Face Three Oclock \\
\hline
U+1F553 & 🕓 & {\textbackslash}:clock4: & Clock Face Four Oclock \\
\hline
U+1F554 & 🕔 & {\textbackslash}:clock5: & Clock Face Five Oclock \\
\hline
U+1F555 & 🕕 & {\textbackslash}:clock6: & Clock Face Six Oclock \\
\hline
U+1F556 & 🕖 & {\textbackslash}:clock7: & Clock Face Seven Oclock \\
\hline
U+1F557 & 🕗 & {\textbackslash}:clock8: & Clock Face Eight Oclock \\
\hline
U+1F558 & 🕘 & {\textbackslash}:clock9: & Clock Face Nine Oclock \\
\hline
U+1F559 & 🕙 & {\textbackslash}:clock10: & Clock Face Ten Oclock \\
\hline
U+1F55A & 🕚 & {\textbackslash}:clock11: & Clock Face Eleven Oclock \\
\hline
U+1F55B & 🕛 & {\textbackslash}:clock12: & Clock Face Twelve Oclock \\
\hline
U+1F55C & 🕜 & {\textbackslash}:clock130: & Clock Face One-Thirty \\
\hline
U+1F55D & 🕝 & {\textbackslash}:clock230: & Clock Face Two-Thirty \\
\hline
U+1F55E & 🕞 & {\textbackslash}:clock330: & Clock Face Three-Thirty \\
\hline
U+1F55F & 🕟 & {\textbackslash}:clock430: & Clock Face Four-Thirty \\
\hline
U+1F560 & 🕠 & {\textbackslash}:clock530: & Clock Face Five-Thirty \\
\hline
U+1F561 & 🕡 & {\textbackslash}:clock630: & Clock Face Six-Thirty \\
\hline
U+1F562 & 🕢 & {\textbackslash}:clock730: & Clock Face Seven-Thirty \\
\hline
U+1F563 & 🕣 & {\textbackslash}:clock830: & Clock Face Eight-Thirty \\
\hline
U+1F564 & 🕤 & {\textbackslash}:clock930: & Clock Face Nine-Thirty \\
\hline
U+1F565 & 🕥 & {\textbackslash}:clock1030: & Clock Face Ten-Thirty \\
\hline
U+1F566 & 🕦 & {\textbackslash}:clock1130: & Clock Face Eleven-Thirty \\
\hline
U+1F567 & 🕧 & {\textbackslash}:clock1230: & Clock Face Twelve-Thirty \\
\hline
U+1F5FB & 🗻 & {\textbackslash}:mount\_fuji: & Mount Fuji \\
\hline
U+1F5FC & 🗼 & {\textbackslash}:tokyo\_tower: & Tokyo Tower \\
\hline
U+1F5FD & 🗽 & {\textbackslash}:statue\_of\_liberty: & Statue Of Liberty \\
\hline
U+1F5FE & 🗾 & {\textbackslash}:japan: & Silhouette Of Japan \\
\hline
U+1F5FF & 🗿 & {\textbackslash}:moyai: & Moyai \\
\hline
U+1F600 & 😀 & {\textbackslash}:grinning: & Grinning Face \\
\hline
U+1F601 & 😁 & {\textbackslash}:grin: & Grinning Face With Smiling Eyes \\
\hline
U+1F602 & 😂 & {\textbackslash}:joy: & Face With Tears Of Joy \\
\hline
U+1F603 & 😃 & {\textbackslash}:smiley: & Smiling Face With Open Mouth \\
\hline
U+1F604 & 😄 & {\textbackslash}:smile: & Smiling Face With Open Mouth And Smiling Eyes \\
\hline
U+1F605 & 😅 & {\textbackslash}:sweat\_smile: & Smiling Face With Open Mouth And Cold Sweat \\
\hline
U+1F606 & 😆 & {\textbackslash}:laughing: & Smiling Face With Open Mouth And Tightly-Closed Eyes \\
\hline
U+1F607 & 😇 & {\textbackslash}:innocent: & Smiling Face With Halo \\
\hline
U+1F608 & 😈 & {\textbackslash}:smiling\_imp: & Smiling Face With Horns \\
\hline
U+1F609 & 😉 & {\textbackslash}:wink: & Winking Face \\
\hline
U+1F60A & 😊 & {\textbackslash}:blush: & Smiling Face With Smiling Eyes \\
\hline
U+1F60B & 😋 & {\textbackslash}:yum: & Face Savouring Delicious Food \\
\hline
U+1F60C & 😌 & {\textbackslash}:relieved: & Relieved Face \\
\hline
U+1F60D & 😍 & {\textbackslash}:heart\_eyes: & Smiling Face With Heart-Shaped Eyes \\
\hline
U+1F60E & 😎 & {\textbackslash}:sunglasses: & Smiling Face With Sunglasses \\
\hline
U+1F60F & 😏 & {\textbackslash}:smirk: & Smirking Face \\
\hline
U+1F610 & 😐 & {\textbackslash}:neutral\_face: & Neutral Face \\
\hline
U+1F611 & 😑 & {\textbackslash}:expressionless: & Expressionless Face \\
\hline
U+1F612 & 😒 & {\textbackslash}:unamused: & Unamused Face \\
\hline
U+1F613 & 😓 & {\textbackslash}:sweat: & Face With Cold Sweat \\
\hline
U+1F614 & 😔 & {\textbackslash}:pensive: & Pensive Face \\
\hline
U+1F615 & 😕 & {\textbackslash}:confused: & Confused Face \\
\hline
U+1F616 & 😖 & {\textbackslash}:confounded: & Confounded Face \\
\hline
U+1F617 & 😗 & {\textbackslash}:kissing: & Kissing Face \\
\hline
U+1F618 & 😘 & {\textbackslash}:kissing\_heart: & Face Throwing A Kiss \\
\hline
U+1F619 & 😙 & {\textbackslash}:kissing\_smiling\_eyes: & Kissing Face With Smiling Eyes \\
\hline
U+1F61A & 😚 & {\textbackslash}:kissing\_closed\_eyes: & Kissing Face With Closed Eyes \\
\hline
U+1F61B & 😛 & {\textbackslash}:stuck\_out\_tongue: & Face With Stuck-Out Tongue \\
\hline
U+1F61C & 😜 & {\textbackslash}:stuck\_out\_tongue\_winking\_eye: & Face With Stuck-Out Tongue And Winking Eye \\
\hline
U+1F61D & 😝 & {\textbackslash}:stuck\_out\_tongue\_closed\_eyes: & Face With Stuck-Out Tongue And Tightly-Closed Eyes \\
\hline
U+1F61E & 😞 & {\textbackslash}:disappointed: & Disappointed Face \\
\hline
U+1F61F & 😟 & {\textbackslash}:worried: & Worried Face \\
\hline
U+1F620 & 😠 & {\textbackslash}:angry: & Angry Face \\
\hline
U+1F621 & 😡 & {\textbackslash}:rage: & Pouting Face \\
\hline
U+1F622 & 😢 & {\textbackslash}:cry: & Crying Face \\
\hline
U+1F623 & 😣 & {\textbackslash}:persevere: & Persevering Face \\
\hline
U+1F624 & 😤 & {\textbackslash}:triumph: & Face With Look Of Triumph \\
\hline
U+1F625 & 😥 & {\textbackslash}:disappointed\_relieved: & Disappointed But Relieved Face \\
\hline
U+1F626 & 😦 & {\textbackslash}:frowning: & Frowning Face With Open Mouth \\
\hline
U+1F627 & 😧 & {\textbackslash}:anguished: & Anguished Face \\
\hline
U+1F628 & 😨 & {\textbackslash}:fearful: & Fearful Face \\
\hline
U+1F629 & 😩 & {\textbackslash}:weary: & Weary Face \\
\hline
U+1F62A & 😪 & {\textbackslash}:sleepy: & Sleepy Face \\
\hline
U+1F62B & 😫 & {\textbackslash}:tired\_face: & Tired Face \\
\hline
U+1F62C & 😬 & {\textbackslash}:grimacing: & Grimacing Face \\
\hline
U+1F62D & 😭 & {\textbackslash}:sob: & Loudly Crying Face \\
\hline
U+1F62E & 😮 & {\textbackslash}:open\_mouth: & Face With Open Mouth \\
\hline
U+1F62F & 😯 & {\textbackslash}:hushed: & Hushed Face \\
\hline
U+1F630 & 😰 & {\textbackslash}:cold\_sweat: & Face With Open Mouth And Cold Sweat \\
\hline
U+1F631 & 😱 & {\textbackslash}:scream: & Face Screaming In Fear \\
\hline
U+1F632 & 😲 & {\textbackslash}:astonished: & Astonished Face \\
\hline
U+1F633 & 😳 & {\textbackslash}:flushed: & Flushed Face \\
\hline
U+1F634 & 😴 & {\textbackslash}:sleeping: & Sleeping Face \\
\hline
U+1F635 & 😵 & {\textbackslash}:dizzy\_face: & Dizzy Face \\
\hline
U+1F636 & 😶 & {\textbackslash}:no\_mouth: & Face Without Mouth \\
\hline
U+1F637 & 😷 & {\textbackslash}:mask: & Face With Medical Mask \\
\hline
U+1F638 & 😸 & {\textbackslash}:smile\_cat: & Grinning Cat Face With Smiling Eyes \\
\hline
U+1F639 & 😹 & {\textbackslash}:joy\_cat: & Cat Face With Tears Of Joy \\
\hline
U+1F63A & 😺 & {\textbackslash}:smiley\_cat: & Smiling Cat Face With Open Mouth \\
\hline
U+1F63B & 😻 & {\textbackslash}:heart\_eyes\_cat: & Smiling Cat Face With Heart-Shaped Eyes \\
\hline
U+1F63C & 😼 & {\textbackslash}:smirk\_cat: & Cat Face With Wry Smile \\
\hline
U+1F63D & 😽 & {\textbackslash}:kissing\_cat: & Kissing Cat Face With Closed Eyes \\
\hline
U+1F63E & 😾 & {\textbackslash}:pouting\_cat: & Pouting Cat Face \\
\hline
U+1F63F & 😿 & {\textbackslash}:crying\_cat\_face: & Crying Cat Face \\
\hline
U+1F640 & 🙀 & {\textbackslash}:scream\_cat: & Weary Cat Face \\
\hline
U+1F645 & 🙅 & {\textbackslash}:no\_good: & Face With No Good Gesture \\
\hline
U+1F646 & 🙆 & {\textbackslash}:ok\_woman: & Face With Ok Gesture \\
\hline
U+1F647 & 🙇 & {\textbackslash}:bow: & Person Bowing Deeply \\
\hline
U+1F648 & 🙈 & {\textbackslash}:see\_no\_evil: & See-No-Evil Monkey \\
\hline
U+1F649 & 🙉 & {\textbackslash}:hear\_no\_evil: & Hear-No-Evil Monkey \\
\hline
U+1F64A & 🙊 & {\textbackslash}:speak\_no\_evil: & Speak-No-Evil Monkey \\
\hline
U+1F64B & 🙋 & {\textbackslash}:raising\_hand: & Happy Person Raising One Hand \\
\hline
U+1F64C & 🙌 & {\textbackslash}:raised\_hands: & Person Raising Both Hands In Celebration \\
\hline
U+1F64D & 🙍 & {\textbackslash}:person\_frowning: & Person Frowning \\
\hline
U+1F64E & 🙎 & {\textbackslash}:person\_with\_pouting\_face: & Person With Pouting Face \\
\hline
U+1F64F & 🙏 & {\textbackslash}:pray: & Person With Folded Hands \\
\hline
U+1F680 & 🚀 & {\textbackslash}:rocket: & Rocket \\
\hline
U+1F681 & 🚁 & {\textbackslash}:helicopter: & Helicopter \\
\hline
U+1F682 & 🚂 & {\textbackslash}:steam\_locomotive: & Steam Locomotive \\
\hline
U+1F683 & 🚃 & {\textbackslash}:railway\_car: & Railway Car \\
\hline
U+1F684 & 🚄 & {\textbackslash}:bullettrain\_side: & High-Speed Train \\
\hline
U+1F685 & 🚅 & {\textbackslash}:bullettrain\_front: & High-Speed Train With Bullet Nose \\
\hline
U+1F686 & 🚆 & {\textbackslash}:train2: & Train \\
\hline
U+1F687 & 🚇 & {\textbackslash}:metro: & Metro \\
\hline
U+1F688 & 🚈 & {\textbackslash}:light\_rail: & Light Rail \\
\hline
U+1F689 & 🚉 & {\textbackslash}:station: & Station \\
\hline
U+1F68A & 🚊 & {\textbackslash}:tram: & Tram \\
\hline
U+1F68B & 🚋 & {\textbackslash}:train: & Tram Car \\
\hline
U+1F68C & 🚌 & {\textbackslash}:bus: & Bus \\
\hline
U+1F68D & 🚍 & {\textbackslash}:oncoming\_bus: & Oncoming Bus \\
\hline
U+1F68E & 🚎 & {\textbackslash}:trolleybus: & Trolleybus \\
\hline
U+1F68F & 🚏 & {\textbackslash}:busstop: & Bus Stop \\
\hline
U+1F690 & 🚐 & {\textbackslash}:minibus: & Minibus \\
\hline
U+1F691 & 🚑 & {\textbackslash}:ambulance: & Ambulance \\
\hline
U+1F692 & 🚒 & {\textbackslash}:fire\_engine: & Fire Engine \\
\hline
U+1F693 & 🚓 & {\textbackslash}:police\_car: & Police Car \\
\hline
U+1F694 & 🚔 & {\textbackslash}:oncoming\_police\_car: & Oncoming Police Car \\
\hline
U+1F695 & 🚕 & {\textbackslash}:taxi: & Taxi \\
\hline
U+1F696 & 🚖 & {\textbackslash}:oncoming\_taxi: & Oncoming Taxi \\
\hline
U+1F697 & 🚗 & {\textbackslash}:car: & Automobile \\
\hline
U+1F698 & 🚘 & {\textbackslash}:oncoming\_automobile: & Oncoming Automobile \\
\hline
U+1F699 & 🚙 & {\textbackslash}:blue\_car: & Recreational Vehicle \\
\hline
U+1F69A & 🚚 & {\textbackslash}:truck: & Delivery Truck \\
\hline
U+1F69B & 🚛 & {\textbackslash}:articulated\_lorry: & Articulated Lorry \\
\hline
U+1F69C & 🚜 & {\textbackslash}:tractor: & Tractor \\
\hline
U+1F69D & 🚝 & {\textbackslash}:monorail: & Monorail \\
\hline
U+1F69E & 🚞 & {\textbackslash}:mountain\_railway: & Mountain Railway \\
\hline
U+1F69F & 🚟 & {\textbackslash}:suspension\_railway: & Suspension Railway \\
\hline
U+1F6A0 & 🚠 & {\textbackslash}:mountain\_cableway: & Mountain Cableway \\
\hline
U+1F6A1 & 🚡 & {\textbackslash}:aerial\_tramway: & Aerial Tramway \\
\hline
U+1F6A2 & 🚢 & {\textbackslash}:ship: & Ship \\
\hline
U+1F6A3 & 🚣 & {\textbackslash}:rowboat: & Rowboat \\
\hline
U+1F6A4 & 🚤 & {\textbackslash}:speedboat: & Speedboat \\
\hline
U+1F6A5 & 🚥 & {\textbackslash}:traffic\_light: & Horizontal Traffic Light \\
\hline
U+1F6A6 & 🚦 & {\textbackslash}:vertical\_traffic\_light: & Vertical Traffic Light \\
\hline
U+1F6A7 & 🚧 & {\textbackslash}:construction: & Construction Sign \\
\hline
U+1F6A8 & 🚨 & {\textbackslash}:rotating\_light: & Police Cars Revolving Light \\
\hline
U+1F6A9 & 🚩 & {\textbackslash}:triangular\_flag\_on\_post: & Triangular Flag On Post \\
\hline
U+1F6AA & 🚪 & {\textbackslash}:door: & Door \\
\hline
U+1F6AB & 🚫 & {\textbackslash}:no\_entry\_sign: & No Entry Sign \\
\hline
U+1F6AC & 🚬 & {\textbackslash}:smoking: & Smoking Symbol \\
\hline
U+1F6AD & 🚭 & {\textbackslash}:no\_smoking: & No Smoking Symbol \\
\hline
U+1F6AE & 🚮 & {\textbackslash}:put\_litter\_in\_its\_place: & Put Litter In Its Place Symbol \\
\hline
U+1F6AF & 🚯 & {\textbackslash}:do\_not\_litter: & Do Not Litter Symbol \\
\hline
U+1F6B0 & 🚰 & {\textbackslash}:potable\_water: & Potable Water Symbol \\
\hline
U+1F6B1 & 🚱 & {\textbackslash}:non-potable\_water: & Non-Potable Water Symbol \\
\hline
U+1F6B2 & 🚲 & {\textbackslash}:bike: & Bicycle \\
\hline
U+1F6B3 & 🚳 & {\textbackslash}:no\_bicycles: & No Bicycles \\
\hline
U+1F6B4 & 🚴 & {\textbackslash}:bicyclist: & Bicyclist \\
\hline
U+1F6B5 & 🚵 & {\textbackslash}:mountain\_bicyclist: & Mountain Bicyclist \\
\hline
U+1F6B6 & 🚶 & {\textbackslash}:walking: & Pedestrian \\
\hline
U+1F6B7 & 🚷 & {\textbackslash}:no\_pedestrians: & No Pedestrians \\
\hline
U+1F6B8 & 🚸 & {\textbackslash}:children\_crossing: & Children Crossing \\
\hline
U+1F6B9 & 🚹 & {\textbackslash}:mens: & Mens Symbol \\
\hline
U+1F6BA & 🚺 & {\textbackslash}:womens: & Womens Symbol \\
\hline
U+1F6BB & 🚻 & {\textbackslash}:restroom: & Restroom \\
\hline
U+1F6BC & 🚼 & {\textbackslash}:baby\_symbol: & Baby Symbol \\
\hline
U+1F6BD & 🚽 & {\textbackslash}:toilet: & Toilet \\
\hline
U+1F6BE & 🚾 & {\textbackslash}:wc: & Water Closet \\
\hline
U+1F6BF & 🚿 & {\textbackslash}:shower: & Shower \\
\hline
U+1F6C0 & 🛀 & {\textbackslash}:bath: & Bath \\
\hline
U+1F6C1 & 🛁 & {\textbackslash}:bathtub: & Bathtub \\
\hline
U+1F6C2 & 🛂 & {\textbackslash}:passport\_control: & Passport Control \\
\hline
U+1F6C3 & 🛃 & {\textbackslash}:customs: & Customs \\
\hline
U+1F6C4 & 🛄 & {\textbackslash}:baggage\_claim: & Baggage Claim \\
\hline
U+1F6C5 & 🛅 & {\textbackslash}:left\_luggage: & Left Luggage \\
\hline
\end{tabulary}

\end{table}


